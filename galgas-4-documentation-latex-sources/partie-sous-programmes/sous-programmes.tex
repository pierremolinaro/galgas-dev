%!TEX encoding = UTF-8 Unicode
%!TEX root = ../galgas-book.tex

%--------------------------------------------------------------
\chapterLabel{Sous-programmes}{sousprgm}\index{Sous-programmes}
%-------------------------------------------------------------

\tableDesMatieresLocaleDeProfondeurRelative{1}


GALGAS définit les sous-programmes suivants :
\begin{itemize}
  \item les \emph{fonctions} (dans ce chapitre, \refSectionPage{declarationFonction}) ;
  \item les \emph{procédures} (dans ce chapitre, \refSectionPage{declarationProcedure}) ;
  \item les \emph{méthodes} (\refSectionPage{categoryMethod}) ;
  \item les \emph{getters} (\refSectionPage{categoryGetter}) ;
  \item les \emph{setters} (\refSectionPage{categorySetter}).
\end{itemize}

En GALGAS, \emph{méthodes}, \emph{getters} et \emph{setters} s'appliquent sur un objet d'un type quelconque (qui n'est donc pas forcément un type \emph{classe}). Pour les types définis par l'utilisateur, \emph{méthodes}, \emph{getters} et \emph{setters} sont toujours déclarés en dehors de la déclaration du type auquel ils s'appliquent.


À chaque nature de sous-programme correspond une construction particulière pour l'appeler (\refTableau{appelSousProgramme}).

\begin{table}[t]
  \centering
  \small
    \begin{tabular}{lll}
      \textbf{Sous-programme} & \textbf{Construction}  & \textbf{Référence} \\
      \emph{routine} & Instruction d'appel de routine & \refSectionPage{appelRoutine} \\
      \emph{fonction} & Appel de fonction (dans une expression) & \refSubsectionPage{appelFonction} \\
      \emph{méthode} & Instruction d'appel de méthode & \refSectionPage{methodCallInstruction} \\
      \emph{getter} & Appel de getter (dans une expression) & \refSubsectionPage{appelGetter} \\
      \emph{setter} & Instruction d'appel de setter & \refSectionPage{setterCallInstruction} \\
    \end{tabular}
  \caption{Constructions d'appel de sous programme}
  \labelTableau{appelSousProgramme}
\end{table}








\sectionLabel{Arguments formels et paramètres effectifs}{correspondanceArgFormelsParametresEffectifs}

GALGAS distingue trois sortes d'arguments formels :
\begin{itemize}
  \item en entrée (\refSubsectionPage{argumentFormelEntree}) ;
  \item en entrée/sortie (\refSubsectionPage{argumentFormelEntreeSortie}) ;
  \item en sortie (\refSubsectionPage{argumentFormelSortie}).
\end{itemize}

\subsectionLabel{Argument formel en entrée}{argumentFormelEntree}

Le \refTableau{ArgumentFormelEntree} liste les différentes formes d'un argument formel en entrée. Le paramètre effectif correspondant est une expression précédée par \ggs+!+.

\begin{table}[t]
  \centering
  \small
  \begin{tabular}{lll}
    \textbf{Argument formel  en entrée} & \textbf{Remarque} & \textbf{Paramètre effectif en sortie} \\
    \ggs+?selector:@T variable+ & Variable (modifiable) & \ggs+!selector:expression+ \\
    \ggs+?selector:@T unused variable+ & Variable inutilisée &  \\
    \ggs+?selector:let @T constante+ & Constante & \\
    \ggs+?selector:let @T unused constante+ & Constante inutilisée & \\
  \end{tabular}
  \caption{Argument formel en entrée, paramètre effectif en sortie}
  \labelTableau{ArgumentFormelEntree}
\end{table}


\subsectionLabel{Argument formel en entrée/sortie}{argumentFormelEntreeSortie}

Le \refTableau{ArgumentFormelEntreeSortie} liste les différentes formes d'un argument formel en entrée. Le paramètre effectif correspondant est une \emph{cible} précédée par \ggs+!?+. Une \emph{cible} est soit une variable, soit l'accès à un champ d'une variable de type \ggs+struct+.

\begin{table}[t]
  \centering
  \small
  \begin{tabular}{lll}
    \textbf{Argument formel en entrée/sortie} & \textbf{Paramètre effectif} & \textbf{Remarque}\\
                                              & \textbf{en sortie/entrée} & \\
    \ggs+?!selector:@T variable+ & \ggs+!?selector:cible+ & \\
    \ggs+?!selector:@T unused variable+ & \ggs+!?selector:*+ & Variable anonyme\\
    & \ggs+!?n*+ & $n$ variables anonymes\\
  \end{tabular}
  \caption{Argument formel en entrée/sortie, paramètre effectif en sortie/entrée}
  \labelTableau{ArgumentFormelEntreeSortie}
\end{table}

\subsectionLabel{Argument formel en sortie}{argumentFormelSortie}

\begin{table}[t]
  \centering
  \small
    \begin{tabular}{lll}
      \textbf{Argument formel en sortie} & \textbf{Paramètre effectif en entrée} & \textbf{Remarque} \\
        \ggs+!selector:@T variable+ & \ggs+?selector:variable+ & Affectation \\
      & \ggs+?selector:@T variable+ & Déclaration et affectation \\
      & \ggs+?selector:let @T constante+ & Déclaration et affectation \\
      & \ggs+?selector:let @T unused constante+ & Déclaration et affectation \\
      & \ggs+?selector:let constante+ & Déclaration et affectation \\
      & \ggs+?selector:let unused constante+ & Déclaration et affectation \\
      & \ggs+?selector:*+ & Une variable anonyme \\
      & \ggs+?n*+ & $n$ variables anonymes \\
    \end{tabular}
  \caption{Argument formel en sortie, paramètre effectif en entrée}
  \labelTableau{ArgumentFormelSortie}
\end{table}

Le \refTableau{ArgumentFormelSortie} liste l'unique forme d'un argument formel en sortie. Le compilateur vérifie que les instructions du sous-programme fixent une valeur à chaque argument formel en sortie.Le paramètre effectif correspondant est une \emph{cible} précédée par \ggs+?+. Une \emph{cible} est soit une variable, soit l'accès à un champ d'une variable de type \ggs+struct+.





\sectionLabel{Liste d'arguments formels en entrée, en sortie, ou en entrée/sortie}{listeArgumentsFormels}






\sectionLabel{Liste de paramètres effectifs en entrée}{listeParametresEffectifsEntree}



\section{Sélecteur}

Il est possible d'associer un nom avec chaque symbole \ggs+?+, \ggs+?!+, \ggs+!+ et \ggs+!?+. Ce nom est appelé \emph{sélecteur}.

Un sélecteur commence par une lettre et est suivi par zéro, un ou plusieurs lettres ou chiffres, et termine obligatoirement par deux points \ggs+:+. Aucun espace n'est autorisé. Par exemple :

\ggs+?valeur:+

\ggs+!par2:+

Les sélecteurs peuvent être utilisés à chaque fois que le symbole \ggs+?+, \ggs+?!+, \ggs+!+ et \ggs+!?+ apparaît. Quand un argument formel est déclaré avec un sélecteur, alors le paramètre effectif doit nommer le même sélecteur.

Par exemple, si on considère une routine déclarée par :

\begin{galgas}
proc aireRectangle
  ?longueur: @uint inA
  ?largeur: @uint inA
  !aire: @uint outAire 
{
  ...
}
\end{galgas}

Alors son appel s'exprimera par :
\begin{galgas}
  aireRectangle (!longueur:2 !largeur:3 ?aire:let aire) 
\end{galgas}


