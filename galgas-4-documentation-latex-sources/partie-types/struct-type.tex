%!TEX encoding = UTF-8 Unicode
%!TEX root = ../galgas-book.tex

%--------------------------------------------------------------
\chapterLabel{Le type structure}{typeStructure}
%-------------------------------------------------------------

\tableDesMatieresLocaleDeProfondeurRelative{1}



Le mot-clé \ggs!struct! permet de définir des types de structure. Un objet de type structure a une sémantique de valeur.

La syntaxe de définition d'un type structure est de la forme~:

\begin{galgas}
struct @MaStructure {
  # Liste de déclaration de propriétés, par exemple :
  @uint mProp1
  @bool mProp2
}
\end{galgas}

Il n'est pas possible de définir du code dans une déclaration de structure~: la seule possibilité est de le définir dans des \ggs=extension= (\refChapterPage{chapitreExtensions}).












\section{Constructeurs}

\subsection{Constructeur \texttt{new}}

Tout type structure définit implicitement le constructeur \ggs!new!. Son appel comprend une valeur par propriété déclarée par le type structure.

Par exemple, pour la déclaration~:
\begin{galgas}
struct @maStructure {
  @uint mProp1
  @bool mProp2
}
\end{galgas}

L'appel du constructeur \ggs!new! est~:
\begin{galgas}
var aVariable = @maStructure.new {!123 !true}
\end{galgas}

Si le contexte le permet, l'annotation de type peut être omis lors de l'appel du constructeur~:
\begin{galgas}
@maStructure aVariable = .new {!123 !true}
\end{galgas}

À partir de la version 3.3.8, il est possible d'ajouter l'attribut \ggs=%selector= à la déclaration d'une propriété de structure. Le faire impose d'utiliser le sélecteur portant le nom de la propriété dans l'appel du constructeur \ggs=new=. Par exemple, si on déclare~:
\begin{galgas}
struct @maStructure {
  @uint mProp1 %selector
  @bool mProp2
}
\end{galgas}

Alors l'appel du constructeur \ggs!new! devient~:
\begin{galgas}
var aVariable = @maStructure.new {!mProp1:123 !true}
\end{galgas}


\subsection{Constructeur \texttt{default}}

Si chacune des propriétés accepte le constructeur par défaut, alors le type structure accepte le constructeur pas défaut. C'est le cas de la structure \ggs!@maStructure! définie au dessus~: \ggs!@uint! accepte le constructeur par défaut (initialisation à \ggs!0!), ainsi que \ggs!@bool! (initialisation à \ggs!false!). Donc~:
\begin{galgas}
var aVariable = @maStructure.default
\end{galgas}
Initialise les propriétés de \ggs!aVariable! respectivement à \ggs!0! et \ggs!false!. On peut aussi écrire~:
\begin{galgas}
@maStructure aVariable = .default
\end{galgas}


\section{Accès aux propriétés}

La notation pointée \ggs!variable.propriété! permet d'accéder à une propriété d'une structure, aussi bien en lecture, en écriture et en lecture/écriture.

Exemple d'accès en lecture~:
\begin{galgas}
@uint v = aVariable.mProp1
\end{galgas}

Exemple d'accès en écriture~:
\begin{galgas}
aVariable.mProp1 = 10
\end{galgas}


Exemple d'accès en lecture/écriture~:
\begin{galgas}
aVariable.mProp1 ++
\end{galgas}





\section{Getters}

Un type structure définit un \emph{getter} sans argument par propriété, qui permet d'accéder en lecture à cette propriété. Son nom est celui de la propriété. Par exemple, à la place de~:
\begin{galgas}
@uint v = aVariable.mProp1
\end{galgas}

On peut écrire~:
\begin{galgas}
@uint v = [aVariable mProp1]
\end{galgas}



\section{Types structure prédéfinis}

Plusieurs types préféfinis GALGAS sont des structures.
 
\subsectionTypePredefiniLabelIndex{lbigint}

\begin{galgas}
struct @lbigint {
  @bigint bigint
  @location location
}
\end{galgas}


 
\subsectionTypePredefiniLabelIndex{lbool}

\begin{galgas}
struct @lbool {
  @bool bool
  @location location
}
\end{galgas}



\subsectionTypePredefiniLabelIndex{lchar}

\begin{galgas}
struct @lchar {
  @char char
  @location location
}
\end{galgas}


\subsectionTypePredefiniLabelIndex{ldouble}

\begin{galgas}
struct @ldouble {
  @double double
  @location location
}
\end{galgas}







\subsectionTypePredefiniLabelIndex{lsint}

\begin{galgas}
struct @lsint {
  @sint sint
  @location location
}
\end{galgas}








\subsectionTypePredefiniLabelIndex{lsint64}

\begin{galgas}
struct @lsint64 {
  @sint64 sint64
  @location location
}
\end{galgas}







\subsectionTypePredefiniLabelIndex{lstring}

\begin{galgas}
struct @lstring {
  @string string
  @location location
}
\end{galgas}







\subsectionTypePredefiniLabelIndex{luint}

\begin{galgas}
struct @luint {
  @uint uint
  @location location
}
\end{galgas}





\subsectionTypePredefiniLabelIndex{luint64}


\begin{galgas}
struct @luint64 {
  @uint64 uint64
  @location location
}
\end{galgas}


\subsectionTypePredefiniLabelIndex{range}

Le type \ggs!@range! définit les intervalles d'entiers non signés 32 bits (\ggs!@uint!).

\begin{galgas}
struct @range {
  @uint start
  @uint length
}
\end{galgas}

La plupart des propriétés du type \ggs!@range! découle de cette définition (\refChapterPage{typeStructure}).

\ggs+@range.new {!a !b}+, où \ggs!a! et \ggs!b! sont des expressions de type \ggs!@uint!, représente~:
\begin{itemize}
  \item un intervalle vide si \ggs!b! est égal à zéro ;
  \item l'intervalle $[a, a+b-1]$ si \ggs!b! est strictement positif.
\end{itemize}



\subsubsectionLabel{Opérateurs \texttt{...} et \texttt{..<}}{operateurIntervalleRange}

Deux opérateurs permettent de construire plus facilement des objets de type \ggs!@range!.

L'opérateur \ggs!...! permet de définir un intervalle fermé à partir de sa borne inférieure et de sa borne supérieure~: si \ggs!a! et \ggs!b! sont des expressions de type \ggs!@uint!, l'expression \ggs!a ... b! est équivalente à la construction \ggs*@range.new {!a !b - a + 1}*. Une exception est levée si $b < a$. 

L'opérateur \ggs!..<! permet de définir un intervalle ouvert à gauche à partir de sa borne inférieure et de sa borne supérieure~: si \ggs!a! et \ggs!b! sont des expressions de type \ggs!@uint!, l'expression \ggs!a ..< b! est équivalente à \ggs*@range.new {!a !b - a}*. Une exception est levée si $b < a$.

\subsubsection{Type \texttt{@range} et instruction \texttt{for}}

On peut utiliser une expression de type \ggs!@range! avec l'instruction \ggs!for!~:

\begin{galgas}
for i in @range.new {!12 !5} do
  # i prend successivement les valeurs 12, 13, 14, 15, 16
end
\end{galgas}

Et, avec l'opérateur \ggs!...!~:
\begin{galgas}
for i in 12 ... 16 do
  # i prend successivement les valeurs 12, 13, 14, 15, 16
end
\end{galgas}

Et l'opérateur \ggs!..<!~:
\begin{galgas}
for i in 12 ..< 17 do
  # i prend successivement les valeurs 12, 13, 14, 15, 16
end
\end{galgas}

Si l'on veut parcourir l'énumération à partir de la dernière valeur, on utilise le modificateur \ggs!>! après le mot-clé \ggs!for!~:
\begin{galgas}
for > i in @range.new {!12 !5} do
  # i prend successivement les valeurs 16, 15, 14, 13, 12
end
\end{galgas}
 
\begin{galgas}
for > i in 12 ... 16 do
  # i prend successivement les valeurs 16, 15, 14, 13, 12
end
\end{galgas}
 
\begin{galgas}
for > i in 12 ..< 17 do
  # i prend successivement les valeurs 16, 15, 14, 13, 12
end
\end{galgas}
 

