%!TEX encoding = UTF-8 Unicode
%!TEX root = ../galgas-book.tex

\chapitreTypePredefiniLabelIndex{timer}

\tableDesMatieresLocaleDeProfondeurRelative{1}


Le type \ggs!@timer! permet de mesurer des durées d'exécution de portions de code ; une utilisation typique est :

\begin{galgas}
var @timer t = .start
  # instructions
message "Durée : " + [t string] + "\n" # Affiche la durée d'exécution des instructions
\end{galgas}


\section{Constructeurs}

Le type \ggs!@timer! accepte deux constructeurs :
\begin{itemize}
  \item le contructeur \ggs!start! ;
  \item le constructeur \ggs!default! (\refSubsectionPage{constructeurParDefaut}), qui a le même effet que le constructeur \ggs!start!.
\end{itemize}


\subsectionConstructor{start}{timer}

\begin{galgasbox}
constructor @timer start -> @timer
\end{galgasbox}

Appeler le constructeur \ggs!start! est la seule façon d'instancier un objet \ggs!@timer!. Le chronomètre est enclenché, c'est-à-dire qu'il compte la durée à partir de laquelle le constructeur \ggs!start! a été appelé.







\section{Setters}

Le type \ggs!@timer! accepte deux \emph{setters} :
\begin{itemize}
  \item le \emph{setter} \ggs!resume! ;
  \item le \emph{setter} \ggs!stop!.
\end{itemize}

\subsectionSetter{resume}{timer}

\begin{galgasbox}
setter @timer resume
\end{galgasbox}

Le \emph{setter} \ggs!resume! redémarre le chronomètre si il est arrêté, et le réinitialise si il est en marche.
\subsectionSetter{stop}{timer}

\begin{galgasbox}
setter @timer stop
\end{galgasbox}

Le \emph{setter} \ggs!stop! arrête le chronomètre. Si il est déjà arrêté, appeler ce \emph{setter} n'a aucun effet.







\section{Getters}

Le type \ggs!@timer! accepte trois \emph{getters} :
\begin{itemize}
  \item le \emph{getter} \ggs!isRunning! ;
  \item le \emph{getter} \ggs!msFromStart! ;
  \item le \emph{getter} \ggs!string!.
\end{itemize}

\subsectionGetter{isRunning}{timer}

\begin{galgasbox}
getter @timer isRunning -> @bool
\end{galgasbox}

Ce \emph{getter} renvoie  \ggs!true! si le récepteur décompte le temps, ou \ggs!false! si il a été arrêté par un appel au \emph{setter} \ggs!stop!.


\subsectionGetter{msFromStart}{timer}

\begin{galgasbox}
getter @timer msFromStart -> @uint
\end{galgasbox}

La valeur obtenue par le \emph{getter} \ggs!msFromStart! est la durée écoulée depuis son instanciation (par le constructeur \ggs!start!) ou depuis le dernier appel au \emph{setter} \ggs!resume!. La durée est exprimée en millisecondes.


\subsectionGetter{string}{timer}

\begin{galgasbox}
getter @timer string -> @string
\end{galgasbox}

La valeur obtenue par le \emph{getter} \ggs!string! est la durée écoulée depuis son instanciation (par le constructeur \ggs!start!) ou depuis le dernier appel au \emph{setter} \ggs!resume!. La durée est exprimée sous la forme d'une chaîne de caractères.

