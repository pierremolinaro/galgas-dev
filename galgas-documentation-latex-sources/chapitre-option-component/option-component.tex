%!TEX encoding = UTF-8 Unicode
%!TEX root = ../galgas-book.tex

%--------------------------------------------------------------
\chapterLabel{Le composant \texttt{option}}{composantOption}
%-------------------------------------------------------------

\tableDesMatieresLocaleDeProfondeurRelative{1}



Le composant \ggst+option+ permet de définir des options qui sont appelables à partir de la ligne de commande. Dans le code, la valeur d'une option est obtenue à partir de l'opérande \emph{appel d'une option}, décrit dans la \refSubsectionPage{appelOption}.

Voici l'exemple d'un composant \ggst+option+ qui déclare une option (évidement, un composant \ggst+option+ peut déclarer un nombre quelconque d'options) :
\begin{galgas3}
option nom_composant {
  @bool nom_option : 'S', "asm" -> "Extract assembly code"
}
\end{galgas3}


\section{Déclaration d'une option}

La déclaration d'une option présente la syntaxe suivante :
\begin{galgas3}
  @T nom_option : caractere, chaine -> description
\end{galgas3}

Les cinq champs qui définissent une option sont :
\begin{itemize}
  \item \ggst+@T+ : le type de l'option ; trois types sont autorisés : \ggst+@bool+, \ggst+@uint+ et \ggst+@string+ ;
  \item \ggst+nom_option+ : c'est le nom, interne à GALGAS, qui permettra de désigner l'option dans l'\emph{appel d'une option} (\refSubsectionPage{appelOption}) ;
  \item \ggst+caractere+ : le caractère qui activera l'option dans la ligne de commande ; par exemple, en écrivant \ggst+'A'+, l'option sera activée par \texttt{-A} dans la ligne de commande ; si vous ne voulez pas d'activation par un caractère, écrivez \ggst+'\0'+ ;
  \item \ggst+chaine+ : la chaîne de caractères qui activera l'option dans la ligne de commande ; par exemple, en écrivant \ggst+"ABEDEF"+, l'option sera activée par \texttt{-{}-ABCDEF} dans la ligne de commande ; si vous ne voulez pas d'activation par une chaîne, écrivez \ggst+""+ ;
  \item \ggst+description+ : une chaîne de caractères qui contient une description de l'option, qui sera affichée par l'option \tpp{-{}-help} de votre compilateur.
\end{itemize}








\section{Option booléenne}

Le champ qui définit le type de l'option est \ggst+@bool+ ; par exemple :
\begin{galgas3}
  @bool nom_option : 'S', "asm" -> "Extract assembly code"
\end{galgas3}

Dans la ligne de commande, l'option est activée par \tpp{-S} ou \tpp{-{}-asm}.

Par défaut, l'option n'est pas activée, et sa valeur associée est \ggst+false+. Quand l'option est activée dans la ligne de commande, sa valeur associée est \ggst+true+.








\section{Option entière}

Le champ qui définit le type de l'option est \ggst+@uint+ ; par exemple :
\begin{galgas3}
  @uint nom_option : 'M', "max-iterations-count" -> "Max of iteration count"
\end{galgas3}

Dans la ligne de commande, l'option est activée par \tpp{-M=xxx} ou \tpp{-{}-max-iterations-count=xxx}, où \tpp{xxx} est un nombre entier positif ou nul (et inférieur ou égal à $2^{32}-1$).

Par défaut, l'option n'est pas activée, et sa valeur associée est $0$. Quand l'option est activée dans la ligne de commande, sa valeur associée est la valeur \tpp{xxx}. Ainsi, l'option \tpp{-M=0}, comme l'option \tpp{-{}-max-iterations-count=0} n'a aucun effet.










\section{Option chaîne de caractères}

Le champ qui définit le type de l'option est \ggst+@string+ ; par exemple :
\begin{galgas3}
  @string nom_option : 'F', "file-name" -> "File name"
\end{galgas3}

Dans la ligne de commande, l'option est activée par \tpp{-F=abc} ou \tpp{-{}-file-name=abc}, où \texttt{abc} est une chaîne de caractères sans espaces. Si vous voulez entrer une chaîne de caractères qui comprend des espaces, par exemple \tpp{abc def}, écrivez : \tpp{"-F=abc def"} ou \tpp{"-{}-file-name=abc def"}.

Par défaut, l'option n'est pas activée, et sa valeur associée est la chaîne vide. Quand l'option est activée dans la ligne de commande, sa valeur associée est la chaîne \tpp{abc}. Ainsi, l'option \tpp{-F=}, comme l'option \tpp{-{}-file-name=} n'a aucun effet.


