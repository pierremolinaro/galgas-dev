%!TEX encoding = UTF-8 Unicode
%!TEX root = ../galgas-book.tex

%--------------------------------------------------------------
\chapter{Élements lexicaux (GALGAS 4)}
%-------------------------------------------------------------

\tableDesMatieresLocaleDeProfondeurRelative{1}


Les éléments lexicaux du langage GALGAS 4 sont~:
\begin{itemize}
  \item les identificateurs (\refSectionPage{identificateursGALGAS4})~;
  \item les mots réservés (\refSectionPage{motReservesGALGAS4})~;
  \item les délimiteurs (\refSectionPage{delimiteursGALGAS4})~;
  \item les sélecteurs  (\refSectionPage{selecteursGALGAS4})~;
  \item les séparateurs  (\refSectionPage{separateursGALGAS4})~;
  \item les commentaires  (\refSectionPage{commentairesGALGAS4})~;
  \item les non terminaux  (\refSectionPage{nonTerminauxGALGAS4})~;
  \item les terminaux (\refSectionPage{terminauxGALGAS4})~;
  \item les constantes littérales entières (\refSectionPage{constantesLitteralesEntiersGALGAS4})~;
  \item les constantes littérales flottantes (\refSectionPage{constantesLitteralesFlottantesGALGAS4})~;
  \item les caractères littéraux (\refSectionPage{constantesLitteralesCaracteresGALGAS4})~;
  \item les constantes chaînes de caractères (\refSectionPage{constantesLitteralesChainesGALGAS4})~;
  \item les noms de types (\refSectionPage{nomTypeGALGAS4})~;
  \item les attributs (\refSectionPage{attributsGALGAS4}).
\end{itemize}


\sectionLabel{Les identificateurs}{identificateursGALGAS4}

Un identificateur commence par une lettre minuscule ou majuscule, suivie de zéro, un ou plusieurs chiffres décimaux, lettres minuscules ou majuscules ou caractères \ggsq='_'=. Par exemple~:

\ggsq=element=, \ggsq=element0=, \ggsq=element_0=, \ggsq=instructionList=, \ggsq=instruction_list=.

Toutes les lettres Unicode sont acceptées~: il est possible d'utiliser des lettres accentuées, des lettres grecques, ... Par exemple~:

\begin{galgas4}
let constanteAccentuée = 12
let π = 3.14
let α = 1
var переменная = 7
\end{galgas4}


\sectionLabel{Les mots réservés}{motReservesGALGAS4}

Les mots réservés de GALGAS sont les identificateurs listés dans le \refTableauPage{mots-reserves}.

\begin{table}[t]
  \centering
  \begin{tabular}{llllllll}
      \ggsq!abstract!  &  \ggsq!after!  &  \ggsq!as!  &  \ggsq!before!  &  \ggsq!between!   \\
  \ggsq!block!  &  \ggsq!boolset!  &  \ggsq!case!  &  \ggsq!class!  &  \ggsq!default!   \\
  \ggsq!dict!  &  \ggsq!do!  &  \ggsq!drop!  &  \ggsq!else!  &  \ggsq!elsif!   \\
  \ggsq!end!  &  \ggsq!enum!  &  \ggsq!error!  &  \ggsq!extension!  &  \ggsq!extern!   \\
  \ggsq!false!  &  \ggsq!fileprivate!  &  \ggsq!filewrapper!  &  \ggsq!final!  &  \ggsq!fixit!   \\
  \ggsq!for!  &  \ggsq!func!  &  \ggsq!grammar!  &  \ggsq!graph!  &  \ggsq!gui!   \\
  \ggsq!if!  &  \ggsq!in!  &  \ggsq!indexing!  &  \ggsq!init!  &  \ggsq!is!   \\
  \ggsq!label!  &  \ggsq!let!  &  \ggsq!lexique!  &  \ggsq!list!  &  \ggsq!log!   \\
  \ggsq!loop!  &  \ggsq!map!  &  \ggsq!mod!  &  \ggsq!mutating!  &  \ggsq!nil!   \\
  \ggsq!not!  &  \ggsq!on!  &  \ggsq!operator!  &  \ggsq!option!  &  \ggsq!or!   \\
  \ggsq!override!  &  \ggsq!package!  &  \ggsq!parse!  &  \ggsq!private!  &  \ggsq!proc!   \\
  \ggsq!project!  &  \ggsq!protected!  &  \ggsq!public!  &  \ggsq!repeat!  &  \ggsq!rewind!   \\
  \ggsq!rule!  &  \ggsq!select!  &  \ggsq!self!  &  \ggsq!send!  &  \ggsq!sortedlist!   \\
  \ggsq!spoil!  &  \ggsq!struct!  &  \ggsq!style!  &  \ggsq!super!  &  \ggsq!switch!   \\
  \ggsq!syntax!  &  \ggsq!tag!  &  \ggsq!template!  &  \ggsq!then!  &  \ggsq!true!   \\
  \ggsq!typealias!  &  \ggsq!unused!  &  \ggsq!var!  &  \ggsq!warning!  &  \ggsq!while!   \\
  \ggsq!with!  &  &    &    &    \\

  \end{tabular}
  \caption{Mots réservés du langage GALGAS4}
  \labelTableau{mots-reserves}
\end{table}


\sectionLabel{Les délimiteurs}{delimiteursGALGAS4}

Les délimiteurs du langage GALGAS sont listés dans le \refTableauPage{delimiteurs}.

\begin{table}[t]
  \centering
  \begin{tabular}{lllllllllllllllll}
      \ggsq0!=0  &  \ggsq0!==0  &  \ggsq0!^0  &  \ggsq0&0  &  \ggsq0&&0  &  \ggsq0&*0  &  \ggsq0&+0  &  \ggsq0&++0  &  \ggsq0&-0  &  \ggsq0&--0  &  \ggsq0&/0  &  \ggsq0(0  &  \ggsq0)0  &  \ggsq0*0  &  \ggsq0*=0   \\
  \ggsq0+0  &  \ggsq0++0  &  \ggsq0+=0  &  \ggsq0,0  &  \ggsq0-0  &  \ggsq0--0  &  \ggsq0-=0  &  \ggsq0->0  &  \ggsq0/0  &  \ggsq0/=0  &  \ggsq0:0  &  \ggsq0:>0  &  \ggsq0;0  &  \ggsq0=0  &  \ggsq0==0   \\
  \ggsq0===0  &  \ggsq0>0  &  \ggsq0>=0  &  \ggsq0>>0  &  \ggsq0?^0  &  \ggsq0[0  &  \ggsq0]0  &  \ggsq0^0  &  \ggsq0`0  &  \ggsq0{0  &  \ggsq0|0  &  \ggsq0||0  &  \ggsq0}0  &  \ggsq0~0  &  \\

  \end{tabular}
  \caption{Délimiteurs du langage GALGAS4}
  \labelTableau{delimiteurs}
\end{table}



\sectionLabel{Les sélecteurs}{selecteursGALGAS4}

\begin{table}[t]
  \centering
  \begin{tabular}{llllllllllllll}
    \ggsq=!=  & \ggsq=!selecteur:=  & \ggsq=!?=  & \ggsq=!?selecteur:= & \ggsq=?= & \ggsq=?selecteur:= & \ggsq=?!= & \ggsq=?!selecteur:= \\
   \end{tabular}
  \caption{Sélecteurs du langage GALGAS4}
  \labelTableau{selecteurs}
\end{table}

Les sélecteurs du langage GALGAS sont listés dans le \refTableauPage{selecteurs}.



\sectionLabel{Les séparateurs}{separateursGALGAS4}

Les séparateurs du langage GALGAS sont~:
\begin{itemize}
  \item le caractère \emph{espace}~;
  \item tout caractère dont le point de code est compris entre \texttt{U+0000} et \texttt{U+001F}.
\end{itemize}



\sectionLabel{Les commentaires}{commentairesGALGAS4}

Un commentaire commence par le caractère « \texttt{\#} » s'étend jusqu'à la fin de la ligne courante.








\sectionLabel{Les non terminaux}{nonTerminauxGALGAS4}

Un \emph{non terminal} d'une grammaire est un identificateur placé entre les caractères \texttt{<} et \texttt{>}. Exemple~:

\begin{galgas4}
 <expression>, <instruction>
\end{galgas4}

Les lettres Unicode y sont acceptées.





\sectionLabel{Les terminaux}{terminauxGALGAS4}

Un \emph{terminal} d'une grammaire est une chaîne de caractères placée entre deux caractères «~\texttt{\$}~». Exemple~:

\begin{galgas4}
 $identifier$, $constant$
\end{galgas4}

Tout caractère Unicode dont le point de code est compris entre \texttt{0x21} (« \texttt{!} ») et \texttt{0xFFFD} peut apparaître dans un terminal~:
\begin{galgas4}
 $=$, $($, $--$, $≠$
\end{galgas4}

Deux échappements sont définis~:
\begin{itemize}
\item «~\texttt{\textbackslash\textbackslash}~» qui permet de définir un unique «~\texttt{\textbackslash}~»~;
\item «~\texttt{\textbackslash\$}~» qui permet de définir un «~\texttt{\$}~».
\end{itemize}

Ceci permet par exemple de définir les terminaux suivants~:
\begin{galgas4}
 $\\$, $\$terminal\$$
\end{galgas4}



\sectionLabel{Les constantes littérales entières}{constantesLitteralesEntiersGALGAS4}

Une constante littérale entière peut être écrite~:
\begin{itemize}
  \item en \emph{décimal}~: elle est constituée de un ou plusieurs chiffres décimaux~; exemple~: \ggsq=123=, \ggsq=9=, \ggsq=05=~;
  \item en \emph{hexadécimal}~: elle commence par \texttt{0x}, suivi d'un ou plusieurs chiffres hexadécimaux~; exemple~: \ggsq=0x12A=, \ggsq=0xabcd=.
\end{itemize}

Le caractère « \texttt{\_} » peut être utilisé pour séparer les chiffres décimaux ou hexadécimaux~: \ggsq=1_234=, \ggsq=0x123_4567=.

Une constante littérale entière est typée~; son type est fixé par son suffixe (\refTableau{suffixeConstantesLiteralesEntieres}).

\begin{table}[t]
  \centering
  \begin{tabular}{llllllllllllll}
    \textbf{Suffixe} & \textbf{Type} & \textbf{Exemples}\\
    \emph{Pas de suffixe}  & \ggsq=@uint=  & \ggsq=1_234=, \ggsq=0x1234_5678= \\
    \texttt{L}  & \ggsq=@uint64=  & \ggsq=1_234L=, \ggsq=0x1234_5678L= \\
    \texttt{S}  & \ggsq=@sint=  & \ggsq=1_234S=, \ggsq=0x1234_5678S= \\
    \texttt{LS}  & \ggsq=@sint64=  & \ggsq=1_234LS=, \ggsq=0x1234_5678LS= \\
    \texttt{G}  & \ggsq=@bigint=  & \ggsq=1_234G=, \ggsq=0x1234_5678G= \\
   \end{tabular}
  \caption{Suffixes et types des constantes littérales entières}
  \labelTableau{suffixeConstantesLiteralesEntieres}
\end{table}







\sectionLabel{Les constantes littérales flottantes}{constantesLitteralesFlottantesGALGAS4}

Une constante littérale flottante comprend toujours un point. Elle est constituée~:
\begin{itemize}
  \item d'un ou plusieurs chiffres décimaux~;
  \item suivis d'un point~;
  \item suivi de zéro, un ou plusieurs chiffres décimaux.
\end{itemize}

Par exemple~: \ggsq=0.=, \ggsq=12.34=.

Le caractère « \texttt{\_} » peut être utilisé pour séparer les chiffres~: \ggsq=1_234.567_890=.

Une constante littérale flottante est du type \ggsq=@double=.




\sectionLabel{Les caractères littéraux}{constantesLitteralesCaracteresGALGAS4}

Un \emph{caractère littéral} est un caractère Unicode placé entre deux apostrophes « \texttt{\textquotesingle} ». Exemple~:

\begin{galgas4}
 'a', 'æ', 'Œ'
\end{galgas4}

Plusieurs séquences d'échappements sont définies et listées dans le \refTableau{echappementConstantesLiteralesCaracteres}.

\begin{table}[t]
  \centering
  \begin{tabular}{llllllllllllll}
    \textbf{Échappement} & \textbf{Caractère} & \textbf{Point de code}\\
    \ggsq='\f'=  & Nouvelle page & \texttt{U+0C} \\
    \ggsq='\n'=  & Passage à la ligne (\emph{Line Feed}) & \texttt{U+0A} \\
    \ggsq='\r'=  & Retour chariot & \texttt{U+0D} \\
    \ggsq='\t'=  & Tabulation horizontale & \texttt{U+09} \\
    \ggsq='\v'=  & Tabulation verticale & \texttt{U+0B} \\
    \ggsq='\\'=  & Barre oblique inversée & \texttt{U+5C} \\
    \ggsq='\0'=  & Caractère nul & \texttt{U+0} \\
    \ggsq='\''=  & Apostrophe & \texttt{U+27} \\
    \ggsq='\uabcd'=  & Caractère du plan de base (4 chiffres hexadécimaux) & \texttt{U+ABCD} \\
    \texttt{\textquotesingle\textbackslash Uabcdefgh\textquotesingle}  & Caractère Unicode (8 chiffres hexadécimaux) & \texttt{U+abcdefgh} \\
   \end{tabular}
  \caption{Séquence d'échappement des constantes littérales caractère}
  \labelTableau{echappementConstantesLiteralesCaracteres}
\end{table}




\sectionLabel{Les constantes chaînes de caractères}{constantesLitteralesChainesGALGAS4}

Un \emph{chaîne de caractères littérale} est une séquence de caractères Unicode placé entre deux guillemets « \texttt{"} ». Exemple~:

\begin{galgas4}
 "une chaîne", "Œnologie"
\end{galgas4}

Plusieurs séquences d'échappements sont définies et listées dans le \refTableau{echappementConstantesLiteralesChaine}.

\begin{table}[t]
  \centering
  \begin{tabular}{llllllllllllll}
    \textbf{Échappement} & \textbf{Caractère} & \textbf{Point de code}\\
    \ggsq="\f"=  & Nouvelle page & \texttt{U+0C} \\
    \ggsq="\n"=  & Passage à la ligne (\emph{Line Feed}) & \texttt{U+0A} \\
    \ggsq="\r"=  & Retour chariot & \texttt{U+0D} \\
    \ggsq="\t"=  & Tabulation horizontale & \texttt{U+09} \\
    \ggsq="\v"=  & Tabulation verticale & \texttt{U+0B} \\
    \ggsq="\\"=  & Barre oblique inversée & \texttt{U+5C} \\
    \ggsq="\""=  & Guillemet & \texttt{U+22} \\
    \ggsq="\uabcd"=  & Caractère du plan de base (4 chiffres hexadécimaux) & \texttt{U+ABCD} \\
%    \texttt{\textquotedbl\textbackslash Uabcdefgh\textquotedbl}  & Caractère Unicode (8 chiffres hexadécimaux) & \texttt{U+abcdefgh} \\
    \texttt{"\textbackslash Uabcdefgh"}  & Caractère Unicode (8 chiffres hexadécimaux) & \texttt{U+abcdefgh} \\
   \end{tabular}
  \caption{Séquence d'échappement des constantes littérales chaîne de caractères}
  \labelTableau{echappementConstantesLiteralesChaine}
\end{table}








\sectionLabel{Les noms de types}{nomTypeGALGAS4}

Un nom de type~:
\begin{itemize}
  \item commence par un caractère « \texttt{@}~»~;
  \item est suivi par un ou plusieurs chiffres ou lettres~;
  \item est suivi éventuellement par un tiret « \texttt{-} », lui-même suivi par un ou plusieurs chiffres ou lettres.
\end{itemize}

Par exemple~:
\begin{galgas4}
 @string, @stringlist-element, @2stringlist
\end{galgas4}





\sectionLabel{Les attributs}{attributsGALGAS4}

Un attribut~:
\begin{itemize}
  \item commence par un caractère «~\texttt{\%}~»~;
  \item est suivi par une lettre Unicode~;
  \item est suivi par une ou plusieurs lettres Unicode, chiffres décimaux, «~\texttt{-}~» ou «~\texttt{\_}~».
\end{itemize}

La liste des attributs est donnée dans le \refTableauPage{attributs-reserves}. Un attribut est un mot réservé « secondaire ».

\begin{table}[ht]
  \centering
  \begin{tabular}{llllllll}
      \ggsq!%MacOS!  &  \ggsq!%MacOSDeployment!   \\
  \ggsq!%MacSwiftApp!  &  \ggsq!%app-link!   \\
  \ggsq!%app-source!  &  \ggsq!%applicationBundleBase!   \\
  \ggsq!%clonable!  &  \ggsq!%codeblocks-linux32!   \\
  \ggsq!%codeblocks-linux64!  &  \ggsq!%codeblocks-windows!   \\
  \ggsq!%comparable!  &  \ggsq!%equatable!   \\
  \ggsq!%errorMessage!  &  \ggsq!%from!   \\
  \ggsq!%generatedInSeparateFile!  &  \ggsq!%initArgLabel!   \\
  \ggsq!%insertAfter!  &  \ggsq!%insertBefore!   \\
  \ggsq!%insertOrReplaceSetter!  &  \ggsq!%insertSetter!   \\
  \ggsq!%libpmAtPath!  &  \ggsq!%macCodeSign!   \\
  \ggsq!%makefile-macosx!  &  \ggsq!%makefile-unix!   \\
  \ggsq!%makefile-win32-on-macosx!  &  \ggsq!%makefile-x86linux32-on-macosx!   \\
  \ggsq!%makefile-x86linux64-on-macosx!  &  \ggsq!%nonAtomicSelection!   \\
  \ggsq!%once!  &  \ggsq!%preserved!   \\
  \ggsq!%quietOutputByDefault!  &  \ggsq!%remove!   \\
  \ggsq!%removeSetter!  &  \ggsq!%replaceBy!   \\
  \ggsq!%searchMethod!  &  \ggsq!%searchString!   \\
  \ggsq!%templateEndMark!  &  \ggsq!%templateReplacement!   \\
  \ggsq!%tool-source!  &  \ggsq!%translate!   \\
  \ggsq!%usefull!  &  \\

  \end{tabular}
  \caption{Attributs du langage GALGAS4}
  \labelTableau{attributs-reserves}
\end{table}



