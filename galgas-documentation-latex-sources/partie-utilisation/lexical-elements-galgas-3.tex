%!TEX encoding = UTF-8 Unicode
%!TEX root = ../galgas-book.tex

%--------------------------------------------------------------
\chapter{Élements lexicaux (GALGAS 3)}
%-------------------------------------------------------------

\tableDesMatieresLocaleDeProfondeurRelative{1}


Les éléments lexicaux du langage GALGAS 3 sont~:
\begin{itemize}
  \item les identificateurs (\refSectionPage{identificateursGALGAS3})~;
  \item les mots réservés (\refSectionPage{motReservesGALGAS3})~;
  \item les délimiteurs (\refSectionPage{delimiteursGALGAS3})~;
  \item les sélecteurs  (\refSectionPage{selecteursGALGAS3})~;
  \item les séparateurs  (\refSectionPage{separateursGALGAS3})~;
  \item les commentaires  (\refSectionPage{commentairesGALGAS3})~;
  \item les non terminaux  (\refSectionPage{nonTerminauxGALGAS3})~;
  \item les terminaux (\refSectionPage{terminauxGALGAS3})~;
  \item les constantes littérales entières (\refSectionPage{constantesLitteralesEntiersGALGAS3})~;
  \item les constantes littérales flottantes (\refSectionPage{constantesLitteralesFlottantesGALGAS3})~;
  \item les caractères littéraux (\refSectionPage{constantesLitteralesCaracteresGALGAS3})~;
  \item les constantes chaînes de caractères (\refSectionPage{constantesLitteralesChainesGALGAS3})~;
  \item les noms de types (\refSectionPage{nomTypeGALGAS3})~;
  \item les attributs (\refSectionPage{attributsGALGAS3}).
\end{itemize}


\sectionLabel{Les identificateurs}{identificateursGALGAS3}

Un identificateur commence par une lettre minuscule ou majuscule, suivie de zéro, un ou plusieurs chiffres décimaux, lettres minuscules ou majuscules ou caractères \ggst='_'=. Par exemple~:

\ggst=element=, \ggst=element0=, \ggst=element_0=, \ggst=instructionList=, \ggst=instruction_list=.

Toutes les lettres Unicode sont acceptées~: il est possible d'utiliser des lettres accentuées, des lettres grecques, ... Par exemple~:

\begin{galgas3}
let constanteAccentuée = 12
let π = 3.14
let α = 1
var переменная = 7
\end{galgas3}


\sectionLabel{Les mots réservés}{motReservesGALGAS3}

Les mots réservés de GALGAS sont les identificateurs listés dans le \refTableauPage{mots-reserves-ggs3}.

\begin{table}[t]
  \centering
  \begin{tabular}{llllllll}
      \ggst!abstract!  &  \ggst!after!  &  \ggst!array!  &  \ggst!as!  &  \ggst!bang!   \\
  \ggst!before!  &  \ggst!between!  &  \ggst!block!  &  \ggst!boolset!  &  \ggst!case!   \\
  \ggst!cast!  &  \ggst!class!  &  \ggst!default!  &  \ggst!dict!  &  \ggst!do!   \\
  \ggst!drop!  &  \ggst!else!  &  \ggst!elsif!  &  \ggst!end!  &  \ggst!enum!   \\
  \ggst!error!  &  \ggst!extension!  &  \ggst!extern!  &  \ggst!false!  &  \ggst!fileprivate!   \\
  \ggst!filewrapper!  &  \ggst!fixit!  &  \ggst!for!  &  \ggst!func!  &  \ggst!getter!   \\
  \ggst!grammar!  &  \ggst!graph!  &  \ggst!gui!  &  \ggst!if!  &  \ggst!in!   \\
  \ggst!indexing!  &  \ggst!init!  &  \ggst!insert!  &  \ggst!is!  &  \ggst!label!   \\
  \ggst!let!  &  \ggst!lexique!  &  \ggst!list!  &  \ggst!listmap!  &  \ggst!log!   \\
  \ggst!loop!  &  \ggst!map!  &  \ggst!message!  &  \ggst!method!  &  \ggst!mod!   \\
  \ggst!mutating!  &  \ggst!not!  &  \ggst!on!  &  \ggst!operator!  &  \ggst!option!   \\
  \ggst!or!  &  \ggst!override!  &  \ggst!parse!  &  \ggst!private!  &  \ggst!proc!   \\
  \ggst!project!  &  \ggst!protected!  &  \ggst!public!  &  \ggst!refclass!  &  \ggst!remove!   \\
  \ggst!repeat!  &  \ggst!replace!  &  \ggst!rewind!  &  \ggst!rule!  &  \ggst!search!   \\
  \ggst!select!  &  \ggst!self!  &  \ggst!send!  &  \ggst!setter!  &  \ggst!sortedlist!   \\
  \ggst!spoil!  &  \ggst!struct!  &  \ggst!style!  &  \ggst!super!  &  \ggst!switch!   \\
  \ggst!syntax!  &  \ggst!tag!  &  \ggst!template!  &  \ggst!then!  &  \ggst!true!   \\
  \ggst!typealias!  &  \ggst!unused!  &  \ggst!var!  &  \ggst!warning!  &  \ggst!while!   \\
  \ggst!with!  &  &    &    &    \\

  \end{tabular}
  \caption{Mots réservés du langage GALGAS3}
  \labelTableau{mots-reserves-ggs3}
\end{table}


\sectionLabel{Les délimiteurs}{delimiteursGALGAS3}

Les délimiteurs du langage GALGAS sont listés dans le \refTableauPage{delimiteursGGS3}.

\begin{table}[t]
  \centering
  \begin{tabular}{lllllllllllllllll}
      \ggst0!=0  &  \ggst0!==0  &  \ggst0!^0  &  \ggst0&0  &  \ggst0&&0  &  \ggst0&*0  &  \ggst0&+0  &  \ggst0&++0  &  \ggst0&-0  &  \ggst0&--0   \\
  \ggst0&/0  &  \ggst0(0  &  \ggst0)0  &  \ggst0*0  &  \ggst0*=0  &  \ggst0+0  &  \ggst0++0  &  \ggst0+=0  &  \ggst0,0  &  \ggst0-0   \\
  \ggst0--0  &  \ggst0-=0  &  \ggst0->0  &  \ggst0/0  &  \ggst0/=0  &  \ggst0:0  &  \ggst0:>0  &  \ggst0;0  &  \ggst0=0  &  \ggst0==0   \\
  \ggst0===0  &  \ggst0>0  &  \ggst0>=0  &  \ggst0>>0  &  \ggst0?^0  &  \ggst0[0  &  \ggst0]0  &  \ggst0^0  &  \ggst0`0  &  \ggst0{0   \\
  \ggst0|0  &  \ggst0||0  &  \ggst0}0  &  \ggst0~0  &  &    &    &    &    &    \\

  \end{tabular}
  \caption{Délimiteurs du langage GALGAS3}
  \labelTableau{delimiteursGGS3}
\end{table}



\sectionLabel{Les sélecteurs}{selecteursGALGAS3}

\begin{table}[t]
  \centering
  \begin{tabular}{llllllllllllll}
    \ggst=!=  & \ggst=!selecteur:=  & \ggst=!?=  & \ggst=!?selecteur:= & \ggst=?= & \ggst=?selecteur:= & \ggst=?!= & \ggst=?!selecteur:= \\
   \end{tabular}
  \caption{Sélecteurs du langage GALGAS3}
  \labelTableau{selecteursGGS3}
\end{table}

Les sélecteurs du langage GALGAS sont listés dans le \refTableauPage{selecteursGGS3}.



\sectionLabel{Les séparateurs}{separateursGALGAS3}

Les séparateurs du langage GALGAS sont~:
\begin{itemize}
  \item le caractère \emph{espace}~;
  \item tout caractère dont le point de code est compris entre \texttt{U+0000} et \texttt{U+001F}.
\end{itemize}



\sectionLabel{Les commentaires}{commentairesGALGAS3}

Un commentaire commence par le caractère « \texttt{\#} » s'étend jusqu'à la fin de la ligne courante.








\sectionLabel{Les non terminaux}{nonTerminauxGALGAS3}

Un \emph{non terminal} d'une grammaire est un identificateur placé entre les caractères \texttt{<} et \texttt{>}. Exemple~:

\begin{galgas3}
 <expression>, <instruction>
\end{galgas3}

Les lettres Unicode y sont acceptées.





\sectionLabel{Les terminaux}{terminauxGALGAS3}

Un \emph{terminal} d'une grammaire est une chaîne de caractères placée entre deux caractères «~\texttt{\$}~». Exemple~:

\begin{galgas3}
 $identifier$, $constant$
\end{galgas3}

Tout caractère Unicode dont le point de code est compris entre \texttt{0x21} (« \texttt{!} ») et \texttt{0xFFFD} peut apparaître dans un terminal~:
\begin{galgas3}
 $=$, $($, $--$, $≠$
\end{galgas3}

Deux échappements sont définis~:
\begin{itemize}
\item «~\texttt{\textbackslash\textbackslash}~» qui permet de définir un unique «~\texttt{\textbackslash}~»~;
\item «~\texttt{\textbackslash\$}~» qui permet de définir un «~\texttt{\$}~».
\end{itemize}

Ceci permet par exemple de définir les terminaux suivants~:
\begin{galgas3}
 $\\$, $\$terminal\$$
\end{galgas3}



\sectionLabel{Les constantes littérales entières}{constantesLitteralesEntiersGALGAS3}

Une constante littérale entière peut être écrite~:
\begin{itemize}
  \item en \emph{décimal}~: elle est constituée de un ou plusieurs chiffres décimaux~; exemple~: \ggst=123=, \ggst=9=, \ggst=05=~;
  \item en \emph{hexadécimal}~: elle commence par \texttt{0x}, suivi d'un ou plusieurs chiffres hexadécimaux~; exemple~: \ggst=0x12A=, \ggst=0xabcd=.
\end{itemize}

Le caractère « \texttt{\_} » peut être utilisé pour séparer les chiffres décimaux ou hexadécimaux~: \ggst=1_234=, \ggst=0x123_4567=.

Une constante littérale entière a pour type \ggst=@bigint=.







\sectionLabel{Les constantes littérales flottantes}{constantesLitteralesFlottantesGALGAS3}

Une constante littérale flottante comprend toujours un point. Elle est constituée~:
\begin{itemize}
  \item d'un ou plusieurs chiffres décimaux~;
  \item suivis d'un point~;
  \item suivi de zéro, un ou plusieurs chiffres décimaux.
\end{itemize}

Par exemple~: \ggst=0.=, \ggst=12.34=.

Le caractère « \texttt{\_} » peut être utilisé pour séparer les chiffres~: \ggst=1_234.567_890=.

Une constante littérale flottante est du type \ggst=@double=.




\sectionLabel{Les caractères littéraux}{constantesLitteralesCaracteresGALGAS3}

Un \emph{caractère littéral} est un caractère Unicode placé entre deux apostrophes « \texttt{\textquotesingle} ». Exemple~:

\begin{galgas3}
 'a', 'æ', 'Œ'
\end{galgas3}

Plusieurs séquences d'échappements sont définies et listées dans le \refTableau{echappementConstantesLiteralesCaracteres-ggs3}.

\begin{table}[t]
  \centering
  \begin{tabular}{llllllllllllll}
    \textbf{Échappement} & \textbf{Caractère} & \textbf{Point de code}\\
    \ggst='\f'=  & Nouvelle page & \texttt{U+0C} \\
    \ggst='\n'=  & Passage à la ligne (\emph{Line Feed}) & \texttt{U+0A} \\
    \ggst='\r'=  & Retour chariot & \texttt{U+0D} \\
    \ggst='\t'=  & Tabulation horizontale & \texttt{U+09} \\
    \ggst='\v'=  & Tabulation verticale & \texttt{U+0B} \\
    \ggst='\\'=  & Barre oblique inversée & \texttt{U+5C} \\
    \ggst='\0'=  & Caractère nul & \texttt{U+0} \\
    \ggst='\''=  & Apostrophe & \texttt{U+27} \\
    \ggst='\uabcd'=  & Caractère du plan de base (4 chiffres hexadécimaux) & \texttt{U+ABCD} \\
    \texttt{\textquotesingle\textbackslash Uabcdefgh\textquotesingle}  & Caractère Unicode (8 chiffres hexadécimaux) & \texttt{U+abcdefgh} \\
   \end{tabular}
  \caption{Séquence d'échappement des constantes littérales caractère}
  \labelTableau{echappementConstantesLiteralesCaracteres-ggs3}
\end{table}




\sectionLabel{Les constantes chaînes de caractères}{constantesLitteralesChainesGALGAS3}

Un \emph{chaîne de caractères littérale} est une séquence de caractères Unicode placé entre deux guillemets « \texttt{"} ». Exemple~:

\begin{galgas3}
 "une chaîne", "Œnologie"
\end{galgas3}

Plusieurs séquences d'échappements sont définies et listées dans le \refTableau{echappementConstantesLiteralesChaine-ggs3}.

\begin{table}[t]
  \centering
  \begin{tabular}{llllllllllllll}
    \textbf{Échappement} & \textbf{Caractère} & \textbf{Point de code}\\
    \ggst="\f"=  & Nouvelle page & \texttt{U+0C} \\
    \ggst="\n"=  & Passage à la ligne (\emph{Line Feed}) & \texttt{U+0A} \\
    \ggst="\r"=  & Retour chariot & \texttt{U+0D} \\
    \ggst="\t"=  & Tabulation horizontale & \texttt{U+09} \\
    \ggst="\v"=  & Tabulation verticale & \texttt{U+0B} \\
    \ggst="\\"=  & Barre oblique inversée & \texttt{U+5C} \\
    \ggst="\""=  & Guillemet & \texttt{U+22} \\
    \ggst="\uabcd"=  & Caractère du plan de base (4 chiffres hexadécimaux) & \texttt{U+ABCD} \\
%    \texttt{\textquotedbl\textbackslash Uabcdefgh\textquotedbl}  & Caractère Unicode (8 chiffres hexadécimaux) & \texttt{U+abcdefgh} \\
    \texttt{"\textbackslash Uabcdefgh"}  & Caractère Unicode (8 chiffres hexadécimaux) & \texttt{U+abcdefgh} \\
   \end{tabular}
  \caption{Séquence d'échappement des constantes littérales chaîne de caractères}
  \labelTableau{echappementConstantesLiteralesChaine-ggs3}
\end{table}








\sectionLabel{Les noms de types}{nomTypeGALGAS3}

Un nom de type~:
\begin{itemize}
  \item commence par un caractère « \texttt{@}~»~;
  \item est suivi par un ou plusieurs chiffres ou lettres~;
  \item est suivi éventuellement par un tiret « \texttt{-} », lui-même suivi par un ou plusieurs chiffres ou lettres.
\end{itemize}

Par exemple~:
\begin{galgas3}
 @string, @stringlist-element, @2stringlist
\end{galgas3}





\sectionLabel{Les attributs}{attributsGALGAS3}

Un attribut~:
\begin{itemize}
  \item commence par un caractère «~\texttt{\%}~»~;
  \item est suivi par une lettre Unicode~;
  \item est suivi par une ou plusieurs lettres Unicode, chiffres décimaux, «~\texttt{-}~» ou «~\texttt{\_}~».
\end{itemize}

La liste des attributs est donnée dans le \refTableauPage{attributs-reserves-ggs3}. Un attribut est un mot réservé « secondaire ».

\begin{table}[ht]
  \centering
  \begin{tabular}{llllllll}
      \ggst!%MacOS!  &  \ggst!%MacOSDeployment!  &  \ggst!%app-link!   \\
  \ggst!%app-source!  &  \ggst!%applicationBundleBase!  &  \ggst!%codeblocks-linux32!   \\
  \ggst!%codeblocks-linux64!  &  \ggst!%codeblocks-windows!  &  \ggst!%generatedInSeparateFile!   \\
  \ggst!%libpmAtPath!  &  \ggst!%macCodeSign!  &  \ggst!%makefile-macosx!   \\
  \ggst!%makefile-unix!  &  \ggst!%makefile-win32-on-macosx!  &  \ggst!%makefile-x86linux32-on-macosx!   \\
  \ggst!%makefile-x86linux64-on-macosx!  &  \ggst!%nonAtomicSelection!  &  \ggst!%once!   \\
  \ggst!%preserved!  &  \ggst!%quietOutputByDefault!  &  \ggst!%selector!   \\
  \ggst!%templateEndMark!  &  \ggst!%tool-source!  &  \ggst!%translate!   \\
  \ggst!%useGrammar!  &  \ggst!%usefull!  &  \\

  \end{tabular}
  \caption{Attributs du langage GALGAS3}
  \labelTableau{attributs-reserves-ggs3}
\end{table}



