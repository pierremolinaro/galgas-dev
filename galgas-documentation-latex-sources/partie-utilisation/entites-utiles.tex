%!TEX encoding = UTF-8 Unicode
%!TEX root = ../galgas-book.tex

%--------------------------------------------------------------
\chapterLabel{Calcul des entités utiles}{chapitreCalculEntitesUtiles}
%-------------------------------------------------------------

Le compilateur GALGAS implémente l'option \tpp{-{}-check-usefulness} qui permet de déceler si des constructions sont inutiles. Au fur et à mesure de l'évolution de la conception de votre compilateur, il se peut que des constructions (type, functions, …) deviennent inutilisées. Cette option permet de déceler ces constructions.

Le compilateur GALGAS construit le graphe d'utilité et écrit le fichier \emph{graphviz}\index{graphviz} dans le répertoire \texttt{build/usefulness}. Ce graphe est le plus souvent illisible car il contient un graphe avec de nombreux nœuds et de nombreux arcs.



Les constructions qui sont évaluées sont :
\begin{itemize}
  \item les routines \ggs=extension getter= ;
  \item les routines \ggs=extension setter= ;
  \item les routines \ggs=extension method= ;
  \item les composants \ggs=lexique= ;
  \item les composants \ggs=grammar= ;
  \item les composants \ggs=syntax= ;
  \item les composants \ggs=option= ;
  \item les \ggs=filewrapper= ;
  \item les fonctions ;
  \item les procédures ;
  \item les types ;
  \item les routines \ggs=after= ;
  \item les routines \ggs=before= ;
  \item les routines d'analyse \ggs=.fileExtension=.
\end{itemize}

À chaque construction correspond un nœud du graphe. Les nœuds racines sont :
\begin{itemize}
  \item les routines \ggs=after= ;
  \item les routines \ggs=before= ;
  \item les routines d'analyse \ggs=.fileExtension=.
\end{itemize}

% Les arcs 

