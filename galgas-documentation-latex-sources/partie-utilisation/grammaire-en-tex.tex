%!TEX encoding = UTF-8 Unicode
%!TEX root = ../galgas-book.tex

%--------------------------------------------------------------
\chapter{Diagrammes syntaxiques des grammaires en TeX}
%-------------------------------------------------------------

Le compilateur GALGAS implémente l'option \tpp{-{}-emit-syntax-diagrams} qui permet d'obtenir les diagrammes syntaxiques de chaque grammaire de votre langage. Ceux-ci sont décrits en \LaTeX en utilisant le paquetage \texttt{tikz}\label{tikz}. C'est ainsi que les diagrammes du langage GALGAS présentés au \refChapterPage{grammairesGALGAS} ont été obtenus.

Dans tout ce chapitre, nous appliquons cette démarche au langage LOGO, défini à la \refSectionPage{presentation-logo}.

\section{Mise en œuvre}

La mise en œuvre est très simple : il suffit d'ajouter l'option indiquée ci-dessus lors de la compilation de votre projet :
\begin{description}
  \item[ ] \tpp{galgas --emit-syntax-diagrams -v chezmoi/logo/+logo.galgasProject}
\end{description}

Les fichiers \LaTeX produits sont rangés dans le répertoire \tpp{chezmoi/build/tex}. Deux fichiers sont produits pour chaque grammaire implémentée par votre projet ; pour le projet LOGO, la grammaire définie s'appelle \ggs=logo_grammar= (\refSectionPage{analyseurSyntaxiqueLOGO}), ces fichiers sont :
\begin{itemize}
\item \tpp{chezmoi/build/logo\_grammar.document.tex} ;
\item \tpp{chezmoi/build/logo\_grammar.tex}.
\end{itemize}

Le fichier \tpp{chezmoi/build/logo\_grammar.document.tex} contient un document \LaTeX directement compilable qui vous permet d'obtenir immédiatement un document PDF contenant les diagrammes syntaxiques de votre langage ; il inclut le fichier \tpp{chezmoi/build/logo\_grammar.tex} qui contient les diagrammes syntaxiques.




