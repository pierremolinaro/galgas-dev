%!TEX encoding = UTF-8 Unicode
%!TEX root = ../galgas-book.tex

%--------------------------------------------------------------
\chapterLabel{Diagrammes syntaxiques des grammaires en TeX}{chapitreDiagrammesSyntaxiques}
%-------------------------------------------------------------

Le compilateur GALGAS implémente l'option \tpp{-{}-emit-syntax-diagrams} qui permet d'obtenir les diagrammes syntaxiques de chaque grammaire de votre langage. Ceux-ci sont décrits en \LaTeX en utilisant le paquetage \texttt{tikz}\label{tikz}.C'est ainsi que les diagrammes des langages GALGAS présentés au \refChapterPage{grammaireProjet}, \refChapterPage{grammaireSource} et \refChapterPage{grammaireTemplate} ont été obtenus.

Il n'y a pas de miracle, les diagrammes syntaxiques peuvent être tronqués parce qu'ils débordent dans la marge droite ou dans le bas de la page~: d'ailleurs vouloir exploiter ces diagrammes peut être l'occasion de revisiter la forme des règles de production.

{\bf Note.} La compilation \LaTeX~des diagrammes syntaxiques est très lente !

Dans tout ce chapitre, nous appliquons cette démarche au langage LOGO, défini à la \refSectionPage{presentation-logo}.

\section{Mise en œuvre}

La mise en œuvre est très simple~: il suffit d'ajouter l'option indiquée ci-dessus lors de la compilation de votre projet~:
\begin{description}
  \item[ ] \tpp{galgas --emit-syntax-diagrams chezmoi/logo/+logo.galgasProject}
\end{description}

Les fichiers \LaTeX produits sont rangés dans le répertoire \tpp{chezmoi/build/tex}. Deux fichiers sont produits pour chaque grammaire implémentée par votre projet ; pour le projet LOGO, la grammaire définie s'appelle \ggs=logo_grammar= (\refSectionPage{analyseurSyntaxiqueLOGO}), ces fichiers sont~:
\begin{itemize}
\item \tpp{chezmoi/build/logo\_grammar.document.tex} (\refSectionPage{documentDiagrammesSyntaxiquesTex}) ;
\item \tpp{chezmoi/build/logo\_grammar.tex} (\refSectionPage{fichierDiagrammesSyntaxiquesTex}).
\end{itemize}


\sectionLabel{Le document \texttt{logo\_grammar.document.tex}}{documentDiagrammesSyntaxiquesTex}

Le fichier \tpp{chezmoi/build/logo\_grammar.document.tex} contient un document \LaTeX~ di-rectement compilable qui vous permet d'obtenir immédiatement un document PDF contenant les diagrammes syntaxiques de votre langage ; il inclut le fichier \tpp{chezmoi/build/lo-} \tpp{go\_grammar.tex} qui contient les diagrammes syntaxiques.

Ce fichier sert d'exemple de configuration de \texttt{tikz} et de l'affichage des diagrammes. Pour donner une chance aux règles de production de s'afficher complètement, le format du document est \emph{paysage A3}. Ce fichier contient plusieurs définitions de commandes qui permettent de paramétrer l'affichage des diagrammes, et qui sont donc appelées dans \tpp{chezmoi/build/logo\_grammar.tex}.




\fbox{\ttfamily\footnotesize\begin{minipage}{1.0\textwidth}
\textbackslash newcommand\textbackslash nonTerminalSection[2]\{\textbackslash section\{Nonterminal \textbackslash texttt\{\#1\}\}\textbackslash label\{nt:\#2\}\}
\end{minipage}}

Cette commande est émise avant chaque non terminal. La définition ci-dessus définit une \emph{section}, et une étiquette qui permet d'établir des hyper-liens sur les non terminaux. Le premier argument est le nom du non terminal, le second est son numéro, utilisé pour les hyper-liens.

\fbox{\ttfamily\footnotesize\begin{minipage}{1.0\textwidth}
\textbackslash newcommand\textbackslash ruleSubsection[3]\{\textbackslash subsection\{Component \textbackslash texttt\{\#1\}, in file \textbackslash texttt\{\#2\}, line \#3\}\}
\end{minipage}}



Cette commande est émise avant chaque diagramme syntaxique d'un non terminal. La définition ci-dessus définit une \emph{sous-section}, et une étiquette qui permet d'établir des hyper-liens sur les non terminaux. Les trois arguments sont~: le nom du composant syntaxique qui contient la règle, le fichier dans lequel  la règle apparaît, et le numéro de ligne dans ce fichier.

Définir une sous-section par règle n'est pas forcément souhaitable, on peut vouloir obtenir la liste des règles, sans aucun texte intermédiaire. Pour cela, il suffit d'écrire~:

\mbox{\ttfamily\footnotesize
\textbackslash newcommand\textbackslash ruleSubsection[3]\{\}
}




\fbox{\ttfamily\footnotesize\begin{minipage}{1.0\textwidth}
\textbackslash newcommand\textbackslash ruleMatrixColumnSeparation\{3mm\}
\end{minipage}}

Définit l'espacement horizontal entre deux colonnes dans les diagrammes syntaxiques.



\fbox{\ttfamily\footnotesize\begin{minipage}{1.0\textwidth}
\textbackslash newcommand\textbackslash ruleMatrixRowSeparation\{3mm\}
\end{minipage}}

Définit l'espacement vertical entre deux rangées dans les diagrammes syntaxiques.



\fbox{\ttfamily\footnotesize\begin{minipage}{1.0\textwidth}
\textbackslash newcommand\textbackslash nonTerminalSymbol[2]\{\textbackslash hyperref[nt:\#2]\{\#1\}\}
\end{minipage}}

Cette commande est émise pour chaque occurrence d'un non terminal dans un diagramme syntaxique. Le premier argument est son nom, le second son numéro. Le numéro permet de définir l'hyper-lien vers la définition du non terminal.



\fbox{\ttfamily\footnotesize\begin{minipage}{1.0\textwidth}
\textbackslash newcommand\textbackslash startSymbol[2]\{The start symbol is \textbackslash hyperref[nt:\#2]\{\#1\}.\}
\end{minipage}}

Commande émise une fois, pour définir le texte qui annonce l'axiome de la grammaire. Les deux arguments sont le nom de l'axiome et son numéro, qui sert à établir un hyper-lien vers sa définition.




\fbox{\ttfamily\footnotesize\begin{minipage}{1.0\textwidth}
\textbackslash newcommand\textbackslash nonTerminalSummaryStart\{This is the alphabetical list of non terminal : \}
\end{minipage}}

C'est le texte introductif de la table des non-terminaux.




\fbox{\ttfamily\footnotesize\begin{minipage}{1.0\textwidth}
\textbackslash newcommand\textbackslash nonTerminalSummary[2]\{\textbackslash hyperref[nt:\#2]\{\#1\}\}
\end{minipage}}

Commande émise à chaque occurrence d'un terminal dans la table.  Les deux arguments sont le nom de l'axiome et son numéro, qui sert à établir un hyper-liens vers sa définition.




\fbox{\ttfamily\footnotesize\begin{minipage}{1.0\textwidth}
\textbackslash newcommand\textbackslash nonTerminalSummarySeparator\{, \}
\end{minipage}}

Commande émise pour séparer deux non terminaux consécutifs dans la table.



\fbox{\ttfamily\footnotesize\begin{minipage}{1.0\textwidth}
\textbackslash newcommand\textbackslash nonTerminalSummaryEnd\{.\textbackslash\textbackslash\}
\end{minipage}}

Commande émise une fois, pour terminer la table des non terminaux.


\sectionLabel{Le fichier \texttt{logo\_grammar.tex}}{fichierDiagrammesSyntaxiquesTex}

La présentation adoptée dans ce fichier est~:
\begin{itemize}
  \item l'axiome de la grammaire (émission de la commande \texttt{\small\textbackslash startSymbol}) ;
  \item la table des non terminaux (émission des commandes \texttt{\small\textbackslash nonTerminalSummaryStart}, \texttt{\small\textbackslash nonTerminalSummary}, \texttt{\small\textbackslash nonTerminalSummarySeparator}, \texttt{\small\textbackslash nonTerminalSummaryEnd})~;
  \item pour chaque non terminal~:
  \begin{itemize}
    \item son annonce par la commande \texttt{\small\textbackslash nonTerminalSection} ;
    \item pour chaque règle de production de ce non terminal~:
    \begin{itemize}
      \item son annonce par la commande \texttt{\small\textbackslash ruleSubsection} ;
      \item son diagramme syntaxique par un environnement \texttt{tikzpicture}.
    \end{itemize}
  \end{itemize}
\end{itemize}



