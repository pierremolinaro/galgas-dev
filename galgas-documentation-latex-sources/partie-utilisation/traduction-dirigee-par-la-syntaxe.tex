%!TEX encoding = UTF-8 Unicode
%!TEX root = ../galgas-book.tex

%--------------------------------------------------------------
\chapter{Traduction dirigée par la syntaxe}
%-------------------------------------------------------------

GALGAS permet de construire un \emph{traducteur dirigée par la syntaxe}. Ce type de traduction permet de transformer le texte source d'une grammaire en un autre texte source, tout en conservant les commentaires. C'est donc bien adapté pour mettre à jour des textes sources suite à un changement de syntaxe.

Mettre en place une traduction dirigée par la syntaxe en GALGAS fait appel aux constructions suivantes :
\begin{itemize}
  \item activer la traduction dirigée par la syntaxe pour chaque composant \galgas{syntax} ;
  \item activer la traduction dirigée par la syntaxe pour le composant \galgas{grammar} ;
  \item modifier l'instruction \galgas{grammar}, de façon à récupérer les informations de traduction ;
  \item modifier l'instruction d'appel de terminal, de façon à récupérer les informations relatives à l'occurrence du terminal ;
  \item modifier l'instruction d'appel de non terminal, de façon à récupérer la traduction du non terminal ;
  \item appeler l'instruction \galgas{send} pour insérer du texte dans la chaîne produite.
\end{itemize}








\section{Le programme d'exemple}

Pour illustrer les différentes possibilités, on prend pour exemple une grammaire qui analyse les expressions arithmétiques, dont les opérandes sont des identificateurs, et dont les deux opérateurs sont l'addition et la multiplication (l'exemple s'étend facilement à d'autres opérateurs). Les parenthèses sont utilisées pour forcer le groupement.

L'analyseur lexical -- non décrit -- définit les symboles terminaux \galgas{$idf$ !@lstring}, \galgas{$+$}, \galgas{$*$}, \galgas{$($} et \galgas{$)$}.

L'analyseur syntaxique est le suivant :
\begin{galgascode}
syntax expSyntax :
  rule <expression> :
    <terme> ;
    repeat while $+$ ; <terme> ; end repeat ;
  end rule ;
  rule <terme> :
    <facteur> ;
    repeat while $*$ ; <facteur> ; end repeat ;
  end rule ;
  rule <facteur> :
    $idf$ ?* ;
  end rule ;
  rule <facteur> :
    $($ ;
    <expression> ;
    $)$ ;
  end rule ;
end syntax ;
\end{galgascode}

La grammaire :
\begin{galgascode}
grammar expGrammar "LL1" (expSyntax) :
  root <expression> ;
end grammar ;
\end{galgascode}

La classe de la grammaire (ici \texttt{LL1}) n'a pas d'importance pour la traduction dirigée par la syntaxe : celle-ci fonctionne pour toutes les classes de grammaire. 

Enfin, le lien entre l'extension des fichiers source et l'analyseur est réalisé par le code suivant :
\begin{galgascode}
when . "expression"
message "an '.expression' source file"
??@lstring inSourceFile {
  grammar expGrammar in inSourceFile ;
}
\end{galgascode}








\section{Activer la traduction dirigée par la syntaxe}

Activer la traduction dirigée par la syntaxe indique à GALGAS d'engendrer le code supplémentaire qui prend en charge la traduction. L'activation doit être indiquée à la fois sur le composant \galgas{syntax} et le composant \galgas{grammar} en ajoutant la directive \galgas{feature translate} dans chaque en-tête\footnote{Dans le cas où les règles syntaxiques sont réparties dans plusieurs composants \galgas{syntax}, l'activation doit être indiquée dans tous.}.

Pour le composant \galgas{syntax} :
\begin{galgascode}
syntax expSyntax feature translate :
  ...
\end{galgascode}

Et pour la grammaire :
\begin{galgascode}
grammar expGrammar "LL1" (expSyntax) feature translate :
  ...
\end{galgascode}

Quand la traduction est activée, l'analyse d'un fichier construit une chaîne de caractères, et par défaut celle-ci est identique à la chaîne source. Par défaut, la chaîne construite est perdue, la section suivante va montrer comment l'obtenir.








\section{Obtenir la chaîne traduite}

La chaîne traduite est obtenue en modifiant l'instruction \galgas{grammar}. Comme on l'a vu, celle-ci est : 
\begin{galgascode}
grammar expGrammar in inSourceFile ;
\end{galgascode}

Obtenir la chaine traduite s'exprime en utilisant l'opérateur  \galgas{\:>} :
\begin{galgascode}
grammar expGrammar in inSourceFile :> ?@string s ;
\end{galgascode}

L'instruction déclare une variable \galgas{s} de type \galgas{@string} et lui affecte la chaîne traduite \footnote{Il existe des variantes pour exprimer l'obtention de la chaîne traduite, voir la description de l'instruction \galgas{grammar} à la \refSectionPage{instruction-grammar}.}.

Par défaut, la chaîne traduite est identique à la chaîne source. Obtenir une chaîne différente est contrôlé par trois instructions :
\begin{itemize}
  \item l'instruction d'appel de terminal, de façon à récupérer les informations relatives à l'occurrence du terminal ;
  \item l'instruction d'appel de non terminal, de façon à récupérer la traduction du non terminal ;
  \item l'instruction \galgas{send} pour insérer du texte dans la chaîne produite.
\end{itemize}







\section{Modifier l'instruction d'appel de terminal}

Une instruction d'appel de terminal a l'allure suivante (par exemple pour \galgas{$idf$}) :
\begin{galgascode}
$idf$ ?* ;
\end{galgascode}

Par défaut, cette instruction recopie à l'identique dans la chaîne produite deux informations :
\begin{itemize}
  \item les séparateurs qui précèdent le terminal ;
  \item le terminal lui-même.
\end{itemize}

Prenons un exemple. On suppose que la chaîne source est : \galgas{@1@a+@2@b@3@}, les commentaires étant constitués des séquences \galgas{@...@}. Cet exemple considère des commentaires, mais il en est de même pour les séparateurs (espaces, retours à la ligne). La séquence des terminaux encontrés lors de l'analyse de cette phrase est :

\begin{center}
  \begin{tabular}{lllllll@{}}
  \textbf{Instruction} & \textbf{Séparateurs précédent le terminal}  & \textbf{Terminal} \\
  \hline
  \galgas{$idf$ ?*} & \galgas{@1@} & \galgas{a} \\
  \galgas{$+$} &  & \galgas{+} \\
  \galgas{$idf$ ?*} & \galgas{@2@} & \galgas{b} \\
  \hline
  \end{tabular}
\end{center}

Le dernier commentaire (\galgas{@3@}), placé après le dernier symbole non terminal, est toujours ajouté à la fin de la chaîne produite.

Pour obtenir les deux informations attachés à chaque terminal\footnote{Il existe d'autres variantes de cet opérateur, voir la description de l'\emph{instruction d'appel de terminal} à la \refSectionPage{instruction-appel-terminal}.}, on utilise l'opérateur \galgas{\:>} :
\begin{galgascode}
$idf$ ?* :> ?@string separateur ?@string token ;
\end{galgascode}

Cette écriture a pour effet que le séparateur précédent le terminal et le terminal lui-même ne sont plus transmis dans la chaîne traduite, mais affectés respectivement à \galgas{separateur} et à \galgas{token}.

On va prendre un exemple pour illustrer cette construction : produite une chaîne dont les identificateurs et les séparateurs qui les précèdent auront disparus. On modifie le composant \galgas{syntax} comme suit\footnote{Il existe une expression plus simple de l'instruction \galgas{$idf$ ?* :> ?@string s ?@string t ;}, puisque \galgas{s} et \galgas{t} ne sont pas utilisés : c'est \galgas{$idf$ ?* :> ?* ?* ;}, décrite  à la \refSectionPage{instruction-appel-terminal}.} :
\begin{galgascode}
syntax expSyntax :
  rule <expression> :
    <terme> ;
    repeat while $+$ ; <terme> ; end repeat ;
  end rule ;
  rule <terme> :
    <facteur> ;
    repeat while $*$ ; <facteur> ; end repeat ;
  end rule ;
  rule <facteur> :
    $idf$ ?* :> ?@string s ?@string t ;
  end rule ;
  rule <facteur> :
    $($ ;
    <expression> ;
    $)$ ;
  end rule ;
end syntax ;
\end{galgascode}

Si la chaîne source est \galgas{@1@a+@2@b@3@}, alors la chaîne produite est \galgas{+@3@}.

Cette première instruction permet donc de ne pas transmettre les informations attachées un terminal. L'instruction \galgas{send}, décrite à la section suivante, va montrer comment insérer du texte dans la chaîne produite.










\section{Insérer du texte : instruction \texttt{send}}

L'instruction \galgas{send} a la syntaxe suivante\footnote{L'instruction \galgas{send} est décrite à la \refSectionPage{instruction-send-syntaxique}.} :
\lstset{emph={exp}, emphstyle=\emph}
\begin{galgascode}
send exp ;
\end{galgascode}

\galgas{exp} est une expression de type \galgas{@string}. Son comportement est simple : la valeur de l'expression chaîne de caractères est simplement transmise à la chaîne produite.

Par exemple, supposons que l'on veuille transformer les parenthèses en accolades ; on écrit le composant \galgas{syntax} comme suit\footnote{Là encore, il existe une forme plus concise de l'instruction \galgas{$($ :> ?@string s ?@string t ;}, puisque \galgas{t} est inutilisé : c'est \galgas{$($ :> ?@string s ?* ;}, décrite  à la \refSectionPage{instruction-appel-terminal}.} :
\begin{galgascode}
syntax expSyntax :
  rule <expression> :
    <terme> ;
    repeat while $+$ ; <terme> ; end repeat ;
  end rule ;
  rule <terme> :
    <facteur> ;
    repeat while $*$ ; <facteur> ; end repeat ;
  end rule ;
  rule <facteur> :
    $idf$ ?* ;
  end rule ;
  rule <facteur> :
    $($ :> ?@string s ?@string t ; send s . "{" ;
    <expression> ;
    $)$ :> ?@string s ?@string t ; send s . "}" ;
  end rule ;
end syntax ;
\end{galgascode}

Mentionner \galgas{s} dans l'instruction \galgas{send} permet de transmettre les séparateurs qui précèdent les parenthèses. Ainsi à partir de la chaîne source \galgas{(@1@a+@2@b)@3@}, on obtient \galgas{\{@1@a+@2@b\}@3@}.


L'instruction \galgas{send} permet de reconstituer le comportement par défaut de l'instruction d'appel de terminal : par exemple, \galgas{$($ \:> ?@string s ?@string t ; send s . t ;} a le même effet que \galgas{$($ ;}.


Attention, l'instruction \galgas{send} est une instruction syntaxique. Cela signifie que le code suivant est incorrect :
\lstset{emph={condition, A, B}, emphstyle=\emph}
\begin{galgascode}
if condition then
  send A ; # Erreur
else
  send B ; # Erreur
end if ;
\end{galgascode}

L'analyse des instructions \galgas{send A ;} et  \galgas{send B ;} déclenche une erreur ; en effet, les branches d'une instruction \galgas{if} ne peuvent contenir que des instructions sémantiques. Les instructions \galgas{send} ne peuvent que directement dans des règles de production, soient dans les branches des instructions \galgas{select}, \galgas{repeat} ou \galgas{parse}. Pour contourner cette interdiction, écrire :
\begin{galgascode}
@string s ;
if condition then
  s := A ;
else
  s := B ;
end if ;
send s ;
\end{galgascode}



\section{Modifier l'instruction d'appel de non-terminal}

L'instruction d'appel de non terminal capture la chaîne obtenue par la dérivation de ce non terminal :
\begin{galgascode}
<expression> ;
\end{galgascode}


Par défaut, cette chaîne est ajoutée à la chaîne produite.

Là encore, l'opérateur \galgas{\:>} permet d'effectuer une interception. On écrit :
\begin{galgascode}
<expression> :> ?@string e ;
\end{galgascode}

La chaîne obtenue par la dérivation du non terminal \galgas{<expression>} n'est pas ajoutée à la chaîne produite, mais affectée à la variable \galgas{e}. D'une manière analogue à l'instruction d'appel de terminal, l'instruction \galgas{send} permet de retrouver le comportement par défaut :
\begin{galgascode}
<expression> :> ?@string e ; send e ;
\end{galgascode}

On utilise souvent cette construction pour ne pas transmettre la chaîne obtenue par la dérivation d'un non terminal ; par exemple, si on ne veut pas transmettre les expressions entre parenthèses, on modifie la dernière règle \galgas{facteur} en\footnote{Ou encore : \galgas{<expression> \:> ?* ;}.} :
\begin{galgascode}
syntax expSyntax :
  ...
  rule <facteur> :
    $($ ;
    <expression> :> ?@string e ;
    $)$ ;
  end rule ;
end syntax ;
\end{galgascode}


