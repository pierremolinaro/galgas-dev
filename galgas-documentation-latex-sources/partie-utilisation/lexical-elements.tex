%!TEX encoding = UTF-8 Unicode
%!TEX root = ../galgas-book.tex

%--------------------------------------------------------------
\chapter{Élements lexicaux}
%-------------------------------------------------------------

Les éléments lexicaux du langage GALGAS sont :
\begin{itemize}
  \item les identificateurs (\refSectionPage{identificateursGALGAS}) ;
  \item les mots réservés (\refSectionPage{motReservesGALGAS}) ;
  \item les délimiteurs (\refSectionPage{delimiteursGALGAS}) ;
  \item les sélecteurs  (\refSectionPage{selecteursGALGAS}) ;
  \item les séparateurs  (\refSectionPage{separateursGALGAS}) ;
  \item les commentaires  (\refSectionPage{commentairesGALGAS}) ;
  \item les non terminaux  (\refSectionPage{nonTerminauxGALGAS}) ;
  \item les terminaux (\refSectionPage{terminauxGALGAS}) ;
  \item les constantes littérales entières (\refSectionPage{constantesLitteralesEntiersGALGAS}) ;
  \item les constantes littérales flottantes (\refSectionPage{constantesLitteralesFlottantesGALGAS}) ;
  \item les caractères littéraux (\refSectionPage{constantesLitteralesCaracteresGALGAS}) ;
  \item les constantes chaînes de caractères (\refSectionPage{constantesLitteralesChainesGALGAS}) ;
  \item les noms de types (\refSectionPage{nomTypeGALGAS}) ;
  \item les attributs (\refSectionPage{attributsGALGAS}).
\end{itemize}


\sectionLabel{Les identificateurs}{identificateursGALGAS}

Un identificateur commence par une lettre minuscule ou majuscule, suivie de zéro, un ou plusieurs chiffres décimaux, lettres minuscules ou majuscules ou caractères \ggs='_'=. Par exemple :

\ggs=element=, \ggs=element0=, \ggs=element_0=, \ggs=instructionList=, \ggs=instruction_list=.

Toutes les lettres Unicode sont acceptées : il est possible d'utiliser des lettres accentuées, des lettres grecques, ... Par exemple :

\begin{galgas}
let constanteAccentuée = 12
let π = 3.14
let α = 1
var переменная = 7
\end{galgas}


\sectionLabel{Les mots réservés}{motReservesGALGAS}

Les mots réservés de GALGAS sont les identificateurs listés dans le \refTableauPage{mots-reserves}.

\begin{table}[t]
  \centering
  \begin{tabular}{llllllll}
      \ggs!abstract!  &  \ggs!after!  &  \ggs!array!  &  \ggs!as!  &  \ggs!before!   \\
  \ggs!between!  &  \ggs!block!  &  \ggs!case!  &  \ggs!cast!  &  \ggs!class!   \\
  \ggs!constructor!  &  \ggs!default!  &  \ggs!do!  &  \ggs!drop!  &  \ggs!else!   \\
  \ggs!elsif!  &  \ggs!end!  &  \ggs!enum!  &  \ggs!error!  &  \ggs!extension!   \\
  \ggs!extern!  &  \ggs!false!  &  \ggs!filewrapper!  &  \ggs!for!  &  \ggs!func!   \\
  \ggs!getter!  &  \ggs!grammar!  &  \ggs!graph!  &  \ggs!gui!  &  \ggs!if!   \\
  \ggs!in!  &  \ggs!indexing!  &  \ggs!insert!  &  \ggs!is!  &  \ggs!label!   \\
  \ggs!let!  &  \ggs!lexique!  &  \ggs!list!  &  \ggs!listmap!  &  \ggs!log!   \\
  \ggs!loop!  &  \ggs!map!  &  \ggs!match!  &  \ggs!message!  &  \ggs!method!   \\
  \ggs!mod!  &  \ggs!not!  &  \ggs!on!  &  \ggs!operator!  &  \ggs!option!   \\
  \ggs!or!  &  \ggs!override!  &  \ggs!parse!  &  \ggs!private!  &  \ggs!proc!   \\
  \ggs!project!  &  \ggs!remove!  &  \ggs!repeat!  &  \ggs!replace!  &  \ggs!rewind!   \\
  \ggs!rule!  &  \ggs!search!  &  \ggs!select!  &  \ggs!self!  &  \ggs!send!   \\
  \ggs!setter!  &  \ggs!sharedmap!  &  \ggs!sortedlist!  &  \ggs!state!  &  \ggs!struct!   \\
  \ggs!style!  &  \ggs!switch!  &  \ggs!syntax!  &  \ggs!tag!  &  \ggs!template!   \\
  \ggs!then!  &  \ggs!true!  &  \ggs!unused!  &  \ggs!var!  &  \ggs!warning!   \\
  \ggs!while!  &  \ggs!with!  &  &    &    \\

  \end{tabular}
  \caption{Mots réservés du langage GALGAS}
  \labelTableau{mots-reserves}
  \ligne
\end{table}


\sectionLabel{Les délimiteurs}{delimiteursGALGAS}

Les délimiteurs du langage GALGAS sont listés dans le \refTableauPage{delimiteurs}.

\begin{table}[t]
  \centering
  \begin{tabular}{lllllllllllllllll}
      \ggs0!=0  &  \ggs0&0  &  \ggs0&&0  &  \ggs0&*0  &  \ggs0&+0  &  \ggs0&++0  &  \ggs0&-0  &  \ggs0&--0  &  \ggs0&/0  &  \ggs0(0  &  \ggs0)0  &  \ggs0*0  &  \ggs0*=0  &  \ggs0+0  &  \ggs0++0   \\
  \ggs0+=0  &  \ggs0,0  &  \ggs0-0  &  \ggs0--0  &  \ggs0-=0  &  \ggs0->0  &  \ggs0/0  &  \ggs0/=0  &  \ggs0:0  &  \ggs0:>0  &  \ggs0;0  &  \ggs0=0  &  \ggs0==0  &  \ggs0>0  &  \ggs0>=0   \\
  \ggs0>>0  &  \ggs0[0  &  \ggs0]0  &  \ggs0^0  &  \ggs0`0  &  \ggs0{0  &  \ggs0|0  &  \ggs0||0  &  \ggs0}0  &  \ggs0~0  &  &    &    &    &    \\

  \end{tabular}
  \caption{Délimiteurs du langage GALGAS}
  \labelTableau{delimiteurs}
  \ligne
\end{table}



\sectionLabel{Les sélecteurs}{selecteursGALGAS}

\begin{table}[t]
  \centering
  \begin{tabular}{llllllllllllll}
    \ggs=!=  & \ggs=!selecteur:=  & \ggs=!?=  & \ggs=!?selecteur:= & \ggs=?= & \ggs=?selecteur:= & \ggs=?!= & \ggs=?!selecteur:= \\
   \end{tabular}
  \caption{Sélecteurs du langage GALGAS}
  \labelTableau{selecteurs}
  \ligne
\end{table}

Les sélecteurs du langage GALGAS sont listés dans le \refTableauPage{selecteurs}.



\sectionLabel{Les séparateurs}{separateursGALGAS}

Les séparateurs du langage GALGAS sont :
\begin{itemize}
  \item le caractère \emph{espace} ;
  \item tout caractère dont le point de code est compris entre \texttt{U+0000} et \texttt{U+001F}. 
\end{itemize}



\sectionLabel{Les commentaires}{commentairesGALGAS}

Un commentaire commence par le caractère « \texttt{\#} » s'étend jusqu'à la fin de la ligne courante.








\sectionLabel{Les non terminaux}{nonTerminauxGALGAS}

Un \emph{non terminal} d'une grammaire est un identificateur placé entre les caractères \texttt{<} et \texttt{>}. Exemple :

\begin{galgas}
 <expression>, <instruction>
\end{galgas}

Les lettres Unicode y sont acceptées.





\sectionLabel{Les terminaux}{terminauxGALGAS}

Un \emph{terminal} d'une grammaire est une chaîne de caractères placée entre deux caractères «~\texttt{\$}~». Exemple :

\begin{galgas}
 $identifier$, $constant$
\end{galgas}

Tout caractère Unicode dont le point de code est compris entre \texttt{0x21} (« \texttt{!} ») et \texttt{0xFFFD} peut apparaître dans un terminal :
\begin{galgas}
 $=$, $($, $--$, $≠$
\end{galgas}

Deux échappements sont définis :
\begin{itemize}
\item «~\texttt{\textbackslash\textbackslash}~» qui permet de définir un unique «~\texttt{\textbackslash}~» ;
\item «~\texttt{\textbackslash\$}~» qui permet de définir un «~\texttt{\$}~».
\end{itemize}

Ceci permet par exemple de définir les terminaux suivants :
\begin{galgas}
 $\\$, $\$terminal\$$
\end{galgas}



\sectionLabel{Les constantes littérales entières}{constantesLitteralesEntiersGALGAS}

Une constante littérale entière peut être écrite :
\begin{itemize}
  \item en \emph{décimal} : elle est constituée de un ou plusieurs chiffres décimaux ; exemple : \ggs=123=, \ggs=9=, \ggs=05= ;
  \item en \emph{hexadécimal} : elle commence par \texttt{0x}, suivi d'un ou plusieurs chiffres hexadécimaux ; exemple : \ggs=0x12A=, \ggs=0xabcd=.
\end{itemize}

Le caractère « \texttt{\_} » peut être utilisé pour séparer les chiffres décimaux ou hexadécimaux : \ggs=1_234=, \ggs=0x123_4567=.

Une constante littérale entière est typée ; son type est fixé par son suffixe (\refTableau{suffixeConstantesLiteralesEntieres}).

\begin{table}[t]
  \centering
  \begin{tabular}{llllllllllllll}
    \textbf{Suffixe} & \textbf{Type} & \textbf{Exemples}\\
    \emph{Pas de suffixe}  & \ggs=@uint=  & \ggs=1_234=, \ggs=0x1234_5678= \\
    \texttt{L}  & \ggs=@uint64=  & \ggs=1_234L=, \ggs=0x1234_5678L= \\
    \texttt{S}  & \ggs=@sint=  & \ggs=1_234S=, \ggs=0x1234_5678S= \\
    \texttt{LS}  & \ggs=@sint64=  & \ggs=1_234LS=, \ggs=0x1234_5678LS= \\
    \texttt{G}  & \ggs=@bigint=  & \ggs=1_234G=, \ggs=0x1234_5678G= \\
   \end{tabular}
  \caption{Suffixes et types des constantes littérales entières}
  \labelTableau{suffixeConstantesLiteralesEntieres}
  \ligne
\end{table}







\sectionLabel{Les constantes littérales flottantes}{constantesLitteralesFlottantesGALGAS}

Une constante littérale flottante comprend toujours un point. Elle est constituée :
\begin{itemize}
  \item d'un ou plusieurs chiffres décimaux ;
  \item suivis d'un point ;
  \item suivi de zéro, un ou plusieurs chiffres décimaux.
\end{itemize}

Par exemple : \ggs=0.=, \ggs=12.34=.

Le caractère « \texttt{\_} » peut être utilisé pour séparer les chiffres : \ggs=1_234.567_890=.

Une constante littérale flottante est du type \ggs=@double=.




\sectionLabel{Les caractères littéraux}{constantesLitteralesCaracteresGALGAS}

Un \emph{caractère littéral} est un caractère Unicode placé entre deux apostrophes « \texttt{\textquotesingle} ». Exemple :

\begin{galgas}
 'a', 'æ', 'Œ'
\end{galgas}

Plusieurs séquences d'échappements sont définies et listées dans le \refTableau{echappementConstantesLiteralesCaracteres}.

\begin{table}[t]
  \centering
  \begin{tabular}{llllllllllllll}
    \textbf{Échappement} & \textbf{Caractère} & \textbf{Point de code}\\
    \ggs='\f'=  & Nouvelle page & \texttt{U+0C} \\
    \ggs='\n'=  & Passage à la ligne (\emph{Line Feed}) & \texttt{U+0A} \\
    \ggs='\r'=  & Retour chariot & \texttt{U+0D} \\
    \ggs='\t'=  & Tabulation horizontale & \texttt{U+09} \\
    \ggs='\v'=  & Tabulation verticale & \texttt{U+0B} \\
    \ggs='\\'=  & Barre oblique inversée & \texttt{U+5C} \\
    \ggs='\0'=  & Caractère nul & \texttt{U+0} \\
    \ggs='\''=  & Apostrophe & \texttt{U+27} \\
    \ggs='\uabcd'=  & Caractère du plan de base (4 chiffres hexadécimaux) & \texttt{U+ABCD} \\
    \texttt{\textquotesingle\textbackslash Uabcdefgh\textquotesingle}  & Caractère Unicode (8 chiffres hexadécimaux) & \texttt{U+abcdefgh} \\
   \end{tabular}
  \caption{Séquence d'échappement des constantes littérales caractère}
  \labelTableau{echappementConstantesLiteralesCaracteres}
  \ligne
\end{table}




\sectionLabel{Les constantes chaînes de caractères}{constantesLitteralesChainesGALGAS}

Un \emph{chaîne de caractères littérale} est une séquence de caractères Unicode placé entre deux guillemets « \texttt{\textquotedbl} ». Exemple :

\begin{galgas}
 "une chaîne", "Œnologie"
\end{galgas}

Plusieurs séquences d'échappements sont définies et listées dans le \refTableau{echappementConstantesLiteralesChaine}.

\begin{table}[t]
  \centering
  \begin{tabular}{llllllllllllll}
    \textbf{Échappement} & \textbf{Caractère} & \textbf{Point de code}\\
    \ggs="\f"=  & Nouvelle page & \texttt{U+0C} \\
    \ggs="\n"=  & Passage à la ligne (\emph{Line Feed}) & \texttt{U+0A} \\
    \ggs="\r"=  & Retour chariot & \texttt{U+0D} \\
    \ggs="\t"=  & Tabulation horizontale & \texttt{U+09} \\
    \ggs="\v"=  & Tabulation verticale & \texttt{U+0B} \\
    \ggs="\\"=  & Barre oblique inversée & \texttt{U+5C} \\
    \ggs="\""=  & Guillemet & \texttt{U+22} \\
    \ggs="\uabcd"=  & Caractère du plan de base (4 chiffres hexadécimaux) & \texttt{U+ABCD} \\
    \texttt{\textquotedbl\textbackslash Uabcdefgh\textquotedbl}  & Caractère Unicode (8 chiffres hexadécimaux) & \texttt{U+abcdefgh} \\
   \end{tabular}
  \caption{Séquence d'échappement des constantes littérales chaîne de caractères}
  \labelTableau{echappementConstantesLiteralesChaine}
  \ligne
\end{table}








\sectionLabel{Les noms de types}{nomTypeGALGAS}

Un nom de type :
\begin{itemize}
  \item commence par un caractère \texttt{@} ;
  \item est suivi par un ou plusieurs chiffres ou lettres ;
  \item est suivi éventuellement par un tiret « \texttt{-} », lui-même suivi par un ou plusieurs chiffres ou lettres.
\end{itemize}

Par exemple :
\begin{galgas}
 @string, @stringlist-element, @2stringlist
\end{galgas}




\sectionLabel{Les attributs}{attributsGALGAS}

Un attribut :
\begin{itemize}
  \item commence par un caractère «~\texttt{\%}~» ;
  \item est suivi par une lettre Unicode ;
  \item est suivi par une ou plusieurs lettres Unicode, chiffres décimaux, «~\texttt{-}~» ou «~\texttt{\_}~».
\end{itemize}

Par exemple :
\begin{galgas}
 %once, %translate, %héhé-π
\end{galgas}


