%!TEX encoding = UTF-8 Unicode
%!TEX root = ../galgas-book.tex

\chapitreTypePredefiniLabelIndex{double}

\tableDesMatieresLocaleDeProfondeurRelative{1}


The \ggst+@double+ object values correspond to the C type \ggst+@double+ values. You can initialize an \ggst+@double+ object from a float constant:

\begin{galgas3}
@double myDouble = 123.456
\end{galgas3}

Note that a \ggst+@double+ constant is characterized by the occurrence of the decimal point (.)

\section{Constructor}

\subsectionConstructor{doubleWithBinaryImage}{double}

\begin{galgas3}
constructor doubleWithBinaryImage ?@uint inValue -> @double
\end{galgas3}


Returns a double object from the binary image of the argument.



\subsectionConstructor{pi}{double}

\begin{galgas3}
constructor pi -> @double
\end{galgas3}



Returns an approximation of the $\pi$ constant value (\ggst+3.14159265358979323846264338327950288+).

\section{Getters}

\subsectionGetter{binaryImage}{double}

\begin{galgas3}
getter binaryImage -> @uint64
\end{galgas3}

Returns the binary image of the value of receiver's value.




\subsectionGetter{cos}{double}

\begin{galgas3}
getter cos -> @double
\end{galgas3}

Returns the \emph{cosine} value of receiver's value, expressed in radian.




\subsectionGetter{sin}{double}

\begin{galgas3}
getter sint -> @double
\end{galgas3}

Returns the \emph{sine} value of receiver's value, expressed in radian.




\subsectionGetter{sint}{double}

\begin{galgas3}
getter sint -> @sint
\end{galgas3}

Returns the receiver's value in an \refTypePredefini{sint} (32-bit signed integer) object: if receiver's value is outside \ggst+@sint+ bounds, a runtime error is raised.



\subsectionGetter{sint64}{double}

\begin{galgas3}
getter sint64 -> @sint64
\end{galgas3}

Returns the receiver's value in an \refTypePredefini{sint64} (64-bit signed integer) object: if receiver's value is outside \ggst+@sint64+ bounds, a runtime error is raised.




\subsectionGetter{string}{double}

\begin{galgas3}
getter string -> @string
\end{galgas3}

Returns a decimal string representation of the receiver's value (this getter never fails).




\subsectionGetter{tan}{double}

\begin{galgas3}
getter tan -> @double
\end{galgas3}

Returns the \emph{tangent} value of receiver's value, expressed in radian.







\subsectionGetter{uint}{double}

\begin{galgas3}
getter uint -> @uint
\end{galgas3}

Returns the receiver's value in an \refTypePredefini{uint} (32-bit unsigned integer) object: if receiver's value is outside \ggst+@uint+ bounds, a runtime error is raised.





\subsectionGetter{uint64}{double}

\begin{galgas3}
getter uint64 -> @uint64
\end{galgas3}

Returns the receiver's value in an \refTypePredefini{uint64} (64-bit unsigned integer) object: if receiver's value is outside \ggst+@uint64+ bounds, a runtime error is raised.



\section{Arithmétique}

\subsection{Opérateurs infixés}

Le type \ggst+@double+ accepte les opérateurs arithmétiques infixés suivants :
\begin{itemize}
  \item \ggst!+!, addition ;
  \item \ggst!-!, soustraction ;
  \item \ggst!*!, multiplication ;
  \item \ggst!/!, division, une erreur d'exécution est déclenchée si le diviseur est nul ;
  \item \ggst!mod!, calcul du reste, une erreur d'exécution est déclenchée si le diviseur est nul ;
  \item \ggst!&/!, division, qui retourne zéro si le diviseur est nul.
\end{itemize}

Ces opérateurs exigent que les deux opérandes soient des objets du même type \ggst+@double+.

\subsection{Opérateurs préfixés}
Le type \ggst+@double+ accepte les opérateurs arithmétiques préfixés suivants :
\begin{itemize}
  \item \ggst!+!, qui retourne simplement la valeur de l'opérande ;
  \item \ggst!-!, négation arithmétique.
\end{itemize}

La valeur renvoyée est du même type  \ggst+@double+.


\subsectionLabel{Instructions}{affectationsCombineesDouble}

Le type \ggst+@double+ accepte les instructions arithmétiques suivantes :
\begin{itemize}
  \item \ggst!+=!, addition ;
  \item \ggst!-=!, soustraction ;
  \item \ggst!*=!, multiplication ;
  \item \ggst!/=!, division.
\end{itemize}

\ggst!x+=y! est équivalent à \ggst!x=x+y! ; \ggst!x-=y! est équivalent à \ggst!x=x-y!.
La variable cible \ggst!x!, comme l'expression source \ggst!y! doivent être du même type \ggst+@double+.







\section{Comparison Operators}

The \ggst+@double+ type supports the six comparison operators:\newline

\begin{tabular}{|c|c|}
\hline
$=$ & Equality \\
\hline
$!=$ & Non Equality \\
\hline
$<$  & Strict Lower Than \\
\hline
$<=$  & Lower or Equal \\
\hline
$>$  & Strict Greater Than \\
\hline
$>=$  & Greater or Equal \\
\hline
\end{tabular}

Theses operators require both arguments to be \ggst+@double+ objects, and return a \ggst+@bool+ object.


