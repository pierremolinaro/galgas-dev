%!TEX encoding = UTF-8 Unicode
%!TEX root = ../galgas-book.tex

\chapitreTypePredefiniLabelIndex{stringset}

\tableDesMatieresLocaleDeProfondeurRelative{1}


An \ggst+@stringset+ object value is a set of \ggst+@string+ values.\\

\section{Constructors}

\subsectionConstructor{emptySet}{stringset}

\begin{galgas3}
constructor emptySet -> @stringset
\end{galgas3}


Creates and returns an empty \ggst+@stringset+ object.

\subsectionConstructor{setWithString}{stringset}

\begin{galgas3}
constructor setWithString ?@string inString -> @stringset
\end{galgas3}


Creates and returns an \ggst+@stringset+ object that contains the value of the \emph{inString} argument object.

\section{Getters}



\subsectionGetter{anyString}{stringset}

\begin{galgas3}
getter anyString -> @string
\end{galgas3}

Retourne une des chaînes de caractères contenue dans le récepteur. Si le récepteur est vide, une erreur d'exécution est déclenchée.
\subsectionGetter{count}{stringset}

\begin{galgas3}
getter count -> @uint
\end{galgas3}

Returns the number of strings in the set.



\subsectionGetter{hasKey}{stringset}

\begin{galgas3}
getter hasKey ?@string inString -> @bool
\end{galgas3}

Returns a boolean value that indicates whether the value of \emph{inString} argument is present in the set: \ggst+true+ if the value of \emph{inString} argument is present in the set, \ggst+false+ otherwise.




\subsectionGetter{stringList}{stringset}

\begin{galgas3}
getter stringList -> @stringlist
\end{galgas3}

Retourne la valeur du récepteur sous la forme d'une liste. L'ordre de la liste est l'ordre alphabétique.



\section{Setter}

\subsectionSetter{removeKey}{stringset}

\begin{galgas3}
setter removeKey ?@string inString
\end{galgas3}


Removes the value of \emph{inString} argument from the receiver's value. If the receiver's value does not contain the value of \emph{inString} argument, this setter leaves the receiver's value unchanged.






\section{the \texttt{+=} Operator}

The \emph{+=} operator adds a string value to the receiver. If the receiver's value already contains the added value, this operator has no effect.

\textbf{exemple :}
\begin{galgas3}
@string aString = ... ;
@stringset aStringSet = ... ;
aStringSet += !aString ;
\end{galgas3}




\section{the \texttt{\&} Operator}

The \emph{$\&$} operator returns the intersection of its operand values.

\textbf{exemple :}
\begin{galgas3}
@stringset s1 = ... ;
@stringset s2 = ... ;
@stringset s = s1 & s2 ; # s is the intersection of s1 and s2
\end{galgas3}






\section{the \texttt{\textbar} Operator}

The \emph{$\textbar$} operator returns the union of its operand values.

\textbf{exemple :}
\begin{galgas3}
@stringset s1 = ...
@stringset s2 = ...
@stringset s = s1 | s2 # s is the union of s1 and s2
\end{galgas3}






\section{the \texttt{-} Operator}

The \emph{$-$} operator returns the difference of its operand values.

\textbf{exemple :}
\begin{galgas3}
@stringset s1 = ...
@stringset s2 = ...
@stringset s = s1 - s2 # s is the difference of s1 and s2
\end{galgas3}








\section{Enumerating \texttt{@stringset} objects}


The \ggst+for+ instruction can be used for enumerating \ggst+@stringset+ values; enumeration is performed in the ascending order, or in the reverse alphabetical order using the '>' qualifier.

\texttt{@stringset s = ... ;}\newline
\textbf{foreach} \texttt {s} \textbf {do}\newline
\texttt{\# the \emph{key} constant has the value of current entry of \emph{s} stringset}\newline
\textbf{end foreach} \texttt{;}







\section{Comparison Operators}

The \ggst+@stringset+ type supports the six comparison operators:\newline

\begin{tabular}{|c|c|}
\hline
$=$ & Equality \\
\hline
$!=$ & Non Equality \\
\hline
$<$  & Strict Inclusion \\
\hline
$<=$  & Inclusion or Equality \\
\hline
$>$  & Strict Greater \\
\hline
$>=$  & Greater or Equality \\
\hline
\end{tabular}

Theses operators require both arguments to be \ggst+@stringset+ objects, and return a \ggst+@stringset+ object.


