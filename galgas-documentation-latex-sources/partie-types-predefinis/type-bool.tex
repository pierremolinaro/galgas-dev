%!TEX encoding = UTF-8 Unicode
%!TEX root = ../galgas-book.tex

\chapitreTypePredefiniLabelIndex{bool}

\tableDesMatieresLocaleDeProfondeurRelative{1}


Le type \ggst+@bool+ est le type booléen. Les deux mots réservés \ggst+true+ et \ggst+false+ sont du type \ggst+@bool+ type, et dénote les valeurs \emph{vari} et \emph{faux}. Le seul constructeur du \ggst+@bool+ type est le constructeur \ggst!default!, qui initialise un booléen à \ggst+false+.


\section{Conversion en chaîne de caractères}

\subsectionGetter{cString}{bool}

\begin{galgas3box}
getter cString -> @string
\end{galgas3box}

Retourne la chaîne \ggst!"true"! si le booléen est vrai, et la chaîne \ggst!"false"! dans le cas contraire.







\subsectionGetter{ocString}{bool}

\begin{galgas3box}
getter ocString -> @string
\end{galgas3box}

Retourne la chaîne \ggst!"YES"! si le booléen est vrai, et la chaîne \ggst!"NO"! dans le cas contraire.




\section{Conversion en entier}


\subsectionGetter{bigint}{bool}

\begin{galgas3box}
getter bigint -> @bigint
\end{galgas3box}

Retourne l'entier \ggst!1G! si le booléen est vrai, et l'entier \ggst!0G! dans le cas contraire.

\begin{galgas3}
  message [[false bigint] string] + "\n" # 0
  message [[true bigint] string] + "\n" # 1
\end{galgas3}


\subsectionGetter{sint}{bool}

\begin{galgas3box}
getter sint -> @sint
\end{galgas3box}

Retourne l'entier \ggst!1S! si le booléen est vrai, et l'entier \ggst!0S! dans le cas contraire.




\subsectionGetter{sint64}{bool}

\begin{galgas3box}
getter sint64 -> @sint64
\end{galgas3box}

Retourne l'entier \ggst!1LS! si le booléen est vrai, et l'entier \ggst!0LS! dans le cas contraire.




\subsectionGetter{uint}{bool}

\begin{galgas3box}
getter uint -> @uint
\end{galgas3box}

Retourne l'entier \ggst!1! si le booléen est vrai, et l'entier \ggst!0! dans le cas contraire.




\subsectionGetter{uint64}{bool}

\begin{galgas3box}
getter uint64 -> @uint64
\end{galgas3box}

Retourne l'entier \ggst!1L! si le booléen est vrai, et l'entier \ggst!0L! dans le cas contraire.




\section{Opérateurs logiques}

\begin{galgas3box}
operator @bool & @bool -> @bool
operator @bool | @bool -> @bool
operator @bool ^ @bool -> @bool
operator not @bool -> @bool
\end{galgas3box}

Le type \ggst+@bool+ accepte les trois opérateurs suivants
\begin{itemize}
\item l'opérateur \ggst!&! infixé qui effectue un \emph{et logique} ;
\item l'opérateur \ggst!|! infixé qui effectue un \emph{ou logique} ;
\item l'opérateur \ggst!^! infixé qui effectue un \emph{ou exclusif logique} ;
\item l'opérateur \ggst!not! infixe qui effectue la\emph{négation logique}.
\end{itemize}








\section{Comparaison}

Le type \ggst!@bool! implémente les six opérateurs de comparaison \ggst!==!, \ggst+!=+, \ggst!<!, \ggst!<=!, \ggst!>! et \ggst!>=!, avec \ggst!false < true!.
