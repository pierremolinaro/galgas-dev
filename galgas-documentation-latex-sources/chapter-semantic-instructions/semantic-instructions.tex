%!TEX encoding = UTF-8 Unicode
%!TEX root = ../galgas-book.tex

%--------------------------------------------------------------
\chapter{Instructions sémantiques}
%-------------------------------------------------------------



\sectionLabel{Append Instruction}{appendInstruction}


\sectionLabel{Assignment Instruction}{assignmentInstruction}


\sectionLabel{L'instruction \texttt{cast}}{instructionCast}

L'instruction \galgas{cast} permet simplement d'exprimer de manière élégante une série de tests de conversions polymorphiques inverses. Sa syntaxe est :

\lstset{emph={expression, variable, conversion}, emphstyle=\emph}
\begin{galgascode}
cast expression
when conversion @T1 variable :
  ...
when conversion @T2 variable :
  ...
else
  ...
end cast ;
\end{galgascode}

L'instruction accepte une ou plusieurs branches \galgas{when}, et zéro ou une branche \galgas{else}. \galgas{conversion} est soit \galgas{==}, soit \galgas{>=}. \galgas{variable} est une constante dont le type est le type nommé dans la branche \galgas{when} qui la déclare, et dont la portée est limitée à cette branche \galgas{when}.

Lors de l'exécution, le type dynamique de \galgas{expression} est comparé successivement aux types (\galgas{@T1}, \galgas{@T2}) nommés dans les branches \galgas{when} ; dès que ce type dynamique est :
\begin{itemize}
  \item exactement la classe \galgas{@T} (\galgas{conversion} est \galgas{==}), 
  \item la classe \galgas{@T} ou de l'une de ses classes héritières (\galgas{conversion} est \galgas{>=}),
\end{itemize}
\galgas{variable} prend la valeur de \galgas{expression} et les instructions de la branche correspondante sont exécutées.

Si toutes les comparaisons échouent, la branche \galgas{else} est exécutée (si elle existe). La forme typique de cette instruction est donc :


\begin{galgascode}
cast expression
when >= @B variable :
  ...
when >= @C variable :
  ...
else
  message "conversion impossible"
end if ;
\end{galgascode}

Note : si la variable \galgas{var} n'est pas utilisée dans la branche correspondante, une alerte est émise. Pour la supprimer, ne pas mentionner la variable en écrivant \galgas{when @T :}.



\sectionLabel{Concat Instruction}{concatInstruction}


\sectionLabel{Decrement Instruction}{decrementInstruction}




\section{L'instruction \texttt{drop}}

La syntaxe de l'instruction \galgas{drop} est la suivante :
{\lstset{emph={variable}, emphstyle=\emph}
\begin{galgascode}
drop variable, ... ;
\end{galgascode}
}

Chaque variable nommée est placée dans l'état \emph{non construit}.

\section{Error Instruction}


\section{Extern Action Call Instruction}




\section{L'instruction \texttt{for}}














%-----------------------------------------------------------------------------------

\sectionLabel{L'instruction \texttt{foreach}}{instructionForeach}

L'instruction \galgas{foreach} permet d'énumérer :
\begin{itemize}
  \item une collection ;
  \item plusieurs collections de manière synchrone.
\end{itemize}

Cette instruction s'est présentée sous plusieurs formes au cours des versions successives de GALGAS, seule la version non obsolète est exposée.

\subsection{Présentation}

Pour énumérer une collection, la syntaxe est la suivante :

\lstset{emph={expression, sens, condition, nom_index, instructions_before, instructions_between, instructions_after, instructions\_do, expression1, sens1, expression2, sens2}, emphstyle=\emph}
\begin{galgascode}
foreach sens expression
index nom_index # Optionnel
while condition # Optionnel
before instructions_before  # Optionnel
do instructions_do
between instructions_between  # Optionnel
after instructions_after  # Optionnel
end foreach ;
\end{galgascode}


Pour énumérer plusieurs collections, la syntaxe est :
\begin{galgascode}
foreach sens1 expression1, sens2 expression2, ...
index nom_index # Optionnel
while condition # Optionnel
before instructions_before  # Optionnel
do instructions_do
between instructions_between  # Optionnel
after instructions_after  # Optionnel
end foreach ;
\end{galgascode}

Les collections à énumérer sont les valeurs de \galgas{expression}, \galgas{expression1}, \galgas{expression2}. Les types pouvant être énumérés sont listés dans le \refTableau{typesEnumerablesForeach}. Pour accéder aux valeurs courantes énumérées, à chaque \galgas{expression} correspond des constantes implicitement déclarées dont les noms sont indiqués dans la dernière colonne du \refTableau{typesEnumerablesForeach}. Cette caractéristique peut provoquer des conflits de noms, que l'on résoud en indiquant explicitement un préfixe (voir \refSubsectionPage{prefixageConstantesForeach}).

Par exemple, pour énumérer une valeur de type \galgas{@stringset}, on écrira :
\begin{galgascode}
@stringset v := ... ;
foreach v do
  log key ; # Affichage des cles dans l'ordre alphabetique
end foreach ;
\end{galgascode}

\begin{table}[ht]
  \centering
%  \rowcolors{2}{\fondTableau}{}
  \begin{tabular}{llp{7cm}}
  \textbf{Type} & \textbf{Ordre d'énumération}  & \textbf{Constantes déclarées}\\
  \hline
  \galgas{@data} & Ordre croissant des indices & \galgas{data}, de type \galgas{@uint}\\
  \galgas{list @T} & Ordre croissant des indices & À chaque champ de la liste, correspond une constante de même nom.\\
  \galgas{map @T} & Ordre alphabétique des clés & \galgas{lkey}, de type \galgas{@lstring}, qui représente la clé, et à chaque champ de la table, correspond une constante de même nom.\\
  \galgas{listmap @T} & Ordre alphabétique des clés & \galgas{key}, de type \galgas{@string}, qui représente la clé, et \galgas{mList}, qui représente la liste associée.\\
  \galgas{sortedlist @T} & Ordre croissant des indices & À chaque champ de la liste, correspond une constante de même nom.\\
  \galgas{@stringset} & Ordre alphabétique & \galgas{key}, de type \galgas{@string} \\
  \hline
  \end{tabular}
  \caption{Types énumérables par l'instruction \texttt{foreach}}
  \labelTableau{typesEnumerablesForeach}
\end{table}

\subsection{Organigramme d'exécution}

L'organigramme illustrant l'exécution de l'instruction \galgas{foreach} est donné à la \refFigure{}{organigrammeForeach}.

\begin{figure}[!ht]
  \centering
  \small
  \begin{tikzpicture}[
      cloud/.style ={draw=red, thick, ellipse,fill=red!20, minimum height=2em},
      block/.style ={rectangle, draw=blue, thick, fill=green!20, align=center},
      decision/.style={chamfered rectangle, draw=blue, thick, fill=green!20},
      node distance=7mm
    ]
    \node [cloud] (start) {\textsc{begin}} ;
    \node [block] (init) [below=of start] {{\tt \emph{nom\_index}} := 0} ;
    \node [decision] (premierTest) [below=of init] {has element \& {\tt \emph{while\_expression}} ?} ;
    \node [block] (before) [below=of premierTest] {\tt \emph{before\_instructions}} ;
    \node [block] (gotoFirst) [below=of before] {goto first element} ;
    \node [block] (doInstructions) [below=of gotoFirst] {\tt \emph{instructions\_do}} ;
    \node [block] (incLoopIndex) [below=of doInstructions] {{\tt \emph{nom\_index}} ++} ;
    \node [decision] (exp) [below=of incLoopIndex] {has next element \& {\tt \emph{while\_expression}} ?} ;
    \node [block] (after) [below=of exp] {\tt \emph{instructions\_after}} ;
    \node [cloud] (end) [below=of after] {\textsc{end}} ;
    \node [block] (between) [right=of doInstructions] {\tt \emph{instructions\_between}} ;
    \node [block] (gotoNext) [below=of between] {goto next element} ;
    
    \draw [-stealth, thick] (start) -- (init) ;
    \draw [-stealth, thick] (init) -- (premierTest) ;
    \draw [-stealth, thick] (premierTest) to node[right] {\bf yes} (before) ;
    \draw [-stealth, thick] (premierTest.west) -- node[above] {\bf no} +(-1, 0) |- (end.west) ;
    \draw [-stealth, thick] (before) -- (gotoFirst) ;
    \draw [-stealth, thick] (gotoFirst) -- (doInstructions) ;
    \draw [-stealth, thick] (doInstructions) -- (incLoopIndex) ;
    \draw [-stealth, thick] (incLoopIndex) -- (exp) ;
    \draw [-stealth, thick] (exp) to node[right] {\bf no} (after) ;
    \draw [-stealth, thick] (exp.east) -| node[right] {\bf yes} (gotoNext.south) ;
    \draw [-stealth, thick] (after) -- (end) ;
    \draw [-stealth, thick] (gotoNext) -- (between) ;
    \draw [-stealth, thick] (between) -- (doInstructions) ;
  \end{tikzpicture}
  \caption{Organigramme d'exécution d'une instruction \texttt{foreach}}
  \labelFigure{organigrammeForeach}
\end{figure}


\subsection{Champs optionnels}

Plusieurs champs de l'instruction \galgas{foreach} sont optionnels.

\galgas{sens}. Ce champ peut prendre trois valeurs, et fixe l'ordre dans lequel les éléments sont énumérés :
\begin{itemize}
  \item si le champ est vide, dans l'ordre indiqué par le tableau \refTableau{typesEnumerablesForeach} ;
  \item \galgas{<}, dans l'ordre indiqué par le tableau \refTableau{typesEnumerablesForeach} ;
  \item \galgas{>}, dans l'ordre inverse à celui indiqué par le tableau \refTableau{typesEnumerablesForeach}.
\end{itemize}


\galgas{index nom\_index}. Vous pouvez mentionner un identificateur après le mot réservé \galgas{index}. Cet identificateur est le nom d'une variable qui a implicitement le type \galgas{@uint} et qui est initialisée à 0 avant toute exécution de la boucle, et incrémentée après chaque exécution des \galgas{instructions\_do}, et avant l'exécution des \galgas{instructions\_between}. Vous ne pouvez pas vous même changer la valeur de cette variable. Sa visibilité inclut l'ensemble des constructions optionnelles.

\galgas{while expression}. L'énumération est exécutée tant que l'\galgas{expression} est vraie. L'absence de cette construction est équivalent à \galgas{while true} et permet d'énumérer toutes les valeurs.


\galgas{before instructions\_before}. Ces instructions sont exécutées une seule fois, au début de l'exécution de l'instruction. Aucun accès aux objets énumérés n'est possible. Si l'énumération est vide, ces instructions ne sont pas exécutées.

\galgas{between instructions\_between}. Ces instructions sont exécutées entre deux exécutions consécutives des \galgas{instructions\_do}. Aucun accès aux objets énumérés n'est possible.

\galgas{after instructions\_after}. Ces instructions sont exécutées une seule fois, à la fin de l'exécution de l'instruction. Aucun accès aux objets énumérés n'est possible. Si l'énumération est vide, ces instructions ne sont pas exécutées.

\subsectionLabel{Préfixage des constantes}{prefixageConstantesForeach}

Considérons l'exemple suivant :

\begin{galgascode}
@stringlist v1 := ... ;
@stringlist v2 := ... ;
foreach v1, v2 do # Erreur !
 ...
end foreach ;
\end{galgascode}

Le compilateur GALGAS déclenche une erreur, car il y a ambiguïté sur la signification de \galgas{mValue} à l'intérieur de la boucle : désigne-t'elle l'élément courant de \galgas{v1} ou l'élément courant de \galgas{v2} ?

Pour lever l'ambiguïté, on complète l'instruction en précisant un \emph{préfixe} pour l'une des deux listes (par exemple la seconde) :
\begin{galgascode}
@stringlist v1 := ... ;
@stringlist v2 := ... ;
foreach v1, v2 : l2_ do
  ...
end foreach ;
\end{galgascode}

La déclaration du préfixe \galgas{l2\_} signifie que les constantes associées à la seconde liste auront leur nom préfixé par \galgas{l2\_}. De cette façon, \galgas{l2\_mValue} désigne la valeur courante de la seconde liste, et \galgas{mValue} désigne sans ambiguïté la valeur courante de la première liste.
\begin{galgascode}
@stringlist v1 := ... ;
@stringlist v2 := ... ;
foreach v1, v2 : l2_ do
  log mValue, l2_mValue ;
end foreach ;
\end{galgascode}

\subsection{Modification de la collection}

Au début de l'exécution de l'instruction \galgas{foreach}, les valeurs des \galgas{expression} enumérées sont capturées et mémorisées. L'énumération s'effectue sur ces valeurs mémorisées. Aussi, il est possible de modifier la collection en cours d'énumération sans que cela affectue les itérations :
\begin{galgascode}
@stringlist v [emptyList] ;
v += !"A" ;
v += !"B" ;
v += !"C" ;
log v ; # "A", "B", "C"
foreach v do
  v += !mValue ;
end foreach ;
log v ; # "A", "B", "C", "A", "B", "C"
\end{galgascode}





\sectionLabel{Increment Instruction}{incrementInstruction}










\section{L'instruction \texttt{if}}


\subsection{Syntax}

The \emph{if} instruction has the following syntax:
{\lstset{emph={expression, instructions, expression2, instructions2, else_instructions}, emphstyle=\emph}
\begin{galgascode}
if expression then
  instructions
elsif expression2 then
  instructions2
...
else
  else_instructions
end if ;  
\end{galgascode}
}

More precisely, it contains :
\begin{itemize}
\item zero, one or more \emph{elsif} branches ;
\item zero or one \emph{else} branch.
\end{itemize}


\subsection{Static semantics}


No \emph{else} branch is equivalent to an \emph{else} branch without any instruction.


The \emph{elsif} branches are just syntactic sugar: it is semantically equivalent to use embedded \emph{if} instructions instead. For example:
{\lstset{emph={expression, instructions, expression2, instructions2, else_instructions}, emphstyle=\emph}
\begin{galgascode}
if expression then
  instructions
elsif expression2 then
  instructions2
else
  else_instructions
end if ;  
\end{galgascode}
}
is equivalent to:
{\lstset{emph={expression, instructions, expression2, instructions2, else_instructions}, emphstyle=\emph}
\begin{galgascode}
if expression then
  instructions
else
  if expression2 then
    instructions2
  else
    else_instructions
  end if ;  
end if ;  
\end{galgascode}
}

So, for describing \emph{if} instruction static and dynamic semantics, we only need to describe an \emph{if} instruction without any \emph{elsif} branch and with an \emph{else} branch:
{\lstset{emph={expression, instructions, else_instructions}, emphstyle=\emph}
\begin{galgascode}
if expression then
  instructions
else
  else_instructions
end if ;
\end{galgascode}
}

The static semantics evaluates the \emph{expression} type, and applies the following rules until success:
\begin{enumerate}
\item the \emph{expression} type is \refTypePredefini{bool};
\item the \emph{expression} type is an \emph{structure} type, it has a attribute named \emph{bool}, whose type is \refTypePredefini{bool};
\item the \emph{expression} type has a reader without any argument named \emph{bool} that returns a \refTypePredefini{bool} value.
\end{enumerate}

Most expressions you write fall in the first case.

Applying the second rule enables to use an \refTypePredefini{lbool} expression as an \emph{if} expression. For example:
{\lstset{emph={expression, instructions, else_instructions}, emphstyle=\emph}
\begin{galgascode}
@lbool var := ... ;
if var then
  instructions
else
  else_instructions
end if ;
\end{galgascode}
}

The \emph{var} object belongs to the \refTypePredefini{lbool} type: so first rule fails. But \refTypePredefini{lbool} is a \emph{structure} type, it has a \emph{bool} attribute with the \refTypePredefini{bool} type, so the second rule succeeds. It is semantically equivalent to write:
{\lstset{emph={expression, instructions, else_instructions}, emphstyle=\emph}
\begin{galgascode}
@lbool var := ... ;
if var->bool then
  instructions
else
  else_instructions
end if ;
\end{galgascode}
}

The third rule applies on a \emph{class} type that defines a category reader with argument named \emph{bool} that returns a \refTypePredefini{bool} type. For example, declaring:
\begin{galgascode}
class @myClass { ... }

reader @myClass bool -> @bool outResult : ... end reader ;
\end{galgascode}

enables to write:
{\lstset{emph={expression, instructions, else_instructions}, emphstyle=\emph}
\begin{galgascode}
@myClass myObject := ... ;
if myObject then
  instructions
else
  else_instructions
end if ;
\end{galgascode}
}

It is semantically equivalent to write:
{\lstset{emph={expression, instructions, else_instructions}, emphstyle=\emph}
\begin{galgascode}
@myClass myObject := ... ;
if [myObject bool] then
  instructions
else
  else_instructions
end if ;
\end{galgascode}
}


\subsection{Dynamic semantics}

According to the preceding section, we only need to describe the dynamic semantic of an \emph{if} instruction without any \emph{elsif} branch and with an \emph{else} branch:
{\lstset{emph={expression, instructions, else_instructions}, emphstyle=\emph}
\begin{galgascode}
if expression then
  instructions
else
  else_instructions
end if ;  
\end{galgascode}
}



The \emph{expression} is first computed :
\begin{itemize}
\item if the evaluation fails, neither the \emph{if} instructions, neither the \emph{else} instructions are executed;
\item if the evaluation result is \emph{true}, the \emph{if} instructions are executed ;
\item if the evaluation result is \emph{false}, the \emph{else} instructions are executed.
\end{itemize}


\section{Grammar Instruction}

\section{Local Variable Declaration Instruction}


{\lstset{emph={variable}, emphstyle=\emph}
\begin{galgascode}
@type variable ;
\end{galgascode}
}

{\lstset{emph={variable, expression}, emphstyle=\emph}
\begin{galgascode}
@type variable := expression ;
\end{galgascode}
}

{\lstset{emph={variable, constructor, arguments}, emphstyle=\emph}
\begin{galgascode}
@type variable [constructor arguments] ;
\end{galgascode}
}


\section{Local Constant Declaration Instruction}




\sectionLabel{L'instruction \texttt{log}}{instructionLog}




\section{L'instruction \texttt{loop}}


\subsection{Syntax}

The \emph{loop} instruction has the following syntax:
{\lstset{emph={expression, instructions_1, instructions_2, variant_expression}, emphstyle=\emph}
\begin{galgascode}
loop variant_expression
: instructions_1
while expression do
  instructions_2
end loop ;  
\end{galgascode}
}

The \emph{instructions\_1} and \emph{instructions\_2} are possibly empty instruction lists. If the \emph{instructions\_1} is empty, the preceeding « : » can be omitted :
{\lstset{emph={expression, instructions_1, instructions_2, variant_expression}, emphstyle=\emph}
\begin{galgascode}
loop variant_expression
while expression do
  instructions_2
end loop ;  
\end{galgascode}
}

\subsection{Semantics}

The \emph{variant\_expression} is an \galgas{@uint} expression that ensures the loop is not endless: it is computed at the beginning of the loop execution, and is decremented by one at the end of every iteration. If it reaches zero, a run-time error is raised.

The \emph{expression} is an \galgas{@bool} expression that expresses repetitive execution.

The \emph{loop} instruction execution is illustrated by the flowchart given in \refFigure{}{loopInstructionFlowchart}.

%\begin{figure}[!ht]
%  \centering
%  \small
%  \begin{tikzpicture}[very thick]
%    \node [rounded corners=5pt, shape=rectangle, draw] (start) {\textsc{begin}} ;
%    \node [shape=rectangle, draw] (variant) [below=of start] {$variant := variant\_expression~value$} ;
%    \node [shape=diamond, draw] (premierTest) [below=of variant] {$variant > 0$} ;
%    \node [shape=rectangle, draw] (error1) [right=of premierTest] {$loop~variant~error$} ;
%    \node [shape=rectangle, draw] (body0) [below=of premierTest] {$instructions\_1$} ;
%    \node [shape=diamond, draw] (exp) [below=of body0] {$expression$} ;
%    \node [shape=diamond, draw] (variantTest) [below=of exp] {$variant > 0$} ;
%    \node [shape=rectangle, draw] (decTest) [left=of variantTest] {$variant {-}{-}$} ;
%    \node [shape=rectangle, draw] (body1) [above=of decTest] {$instructions\_2$} ;
%    \node [shape=rectangle, draw] (error) [right=of variantTest] {$loop~variant~error$} ;
%    \node [rounded corners=5pt, shape=rectangle, draw] (end) [right=of error] {\textsc{end}} ;
%    
%    \draw [->] (start) -- (variant) ;
%    \draw [->] (variant) -- (premierTest) ;
%    \draw [->] (premierTest) to node[right] {$yes$} (body0) ;
%    \draw [->] (premierTest) to node[above] {$no$} (error1) ;
%    \draw [->] (body0) -- (exp) ;
%    \draw [->] (exp) to node[right] {$true$} (variantTest) ;
%    \draw [->] (variantTest) to node[above] {$yes$} (decTest) ;
%    \draw [->] (variantTest) to node[above] {$no$} (error) ;
%    \draw [->] (decTest) -- (body1) ;
%    \draw [->, bend left] (exp.east) to node[above] {$false$} (end.north) ;
%    \draw [->] (body1.north) .. controls +(north:2cm) and +(left:2cm) .. (body0.west) ;
%    \draw [->] (error) -- (end) ;
%    \draw [->] (error1.east) .. controls +(right:2cm) .. (end) ;
%  \end{tikzpicture}
%  \caption{\emph{loop} instruction flowchart}
%  \labelFigure{loopInstructionFlowchart}
%\end{figure}


\begin{figure}[!ht]
  \centering
  \small
  \begin{tikzpicture}[
      cloud/.style ={draw=red, thick, ellipse,fill=red!20, minimum height=2em},
      block/.style ={rectangle, draw=blue, thick, fill=green!20, align=center},
      error/.style ={rectangle, draw=red, thick, fill=green!20, align=center},
      decision/.style={chamfered rectangle, draw=blue, thick, fill=green!20},
      node distance=7mm
    ]
    \node [cloud] (start) {\textsc{begin}} ;
    \node [block] (affectationVariant) [below=of start] {variant := {\tt \emph{variant\_expression}}} ;
    \node [decision] (premierTest) [below=of affectationVariant] {variant > 0 ?} ;
    \node [error] (loopVariantError) [right=of premierTest] {loop variant error} ;
    \node [block] (instructions1) [below=of premierTest] {\tt \emph{instructions\_1}} ;
    \node [decision] (expression) [below=of instructions1] {\tt \emph{expression} ?} ;
    \node [decision] (secondTest) [below=of expression] {variant > 0 ?} ;
    \node [error] (loopVariantError2) [right=of secondTest] {loop variant error} ;
    \node [block] (decVariant) [below=of secondTest] {variant {-}{-}} ;
    \node [block] (instructions2) [below=of decVariant] {\tt \emph{instructions\_2}} ;
    \node [cloud] (end) [below=of instructions2] {\textsc{end}} ;
    
    \draw [-stealth, thick] (start) -- (affectationVariant) ;
    \draw [-stealth, thick] (affectationVariant) -- (premierTest) ;
    \draw [-stealth, thick] (premierTest) -- node[above] {\bf no} (loopVariantError) ;
    \draw [-stealth, thick] (secondTest) -- node[above] {\bf no} (loopVariantError2) ;
    \draw [-stealth, thick] (loopVariantError.east) -- +(1, 0) |- (end.east) ;
    \draw [-stealth, thick] (loopVariantError2.east) -- +(1, 0) ;
    \draw [-stealth, thick] (expression.east) -- node[above] {\bf false} +(4.5, 0) ;
    \draw [-stealth, thick] (instructions1) -- (expression) ;
    \draw [-stealth, thick] (premierTest) to node[right] {\bf yes} (instructions1) ;
    \draw [-stealth, thick] (expression) -- node[right] {\bf true} (secondTest) ;
    \draw [-stealth, thick] (secondTest) -- node[right] {\bf yes} (decVariant) ;
    \draw [-stealth, thick] (decVariant) -- (instructions2) ;
    \draw [-stealth, thick] (instructions2.west) -- +(-1, 0) |- (instructions1.west) ;
  \end{tikzpicture}
  \caption{Organigramme d'exécution d'une instruction \texttt{loop}}
  \labelFigure{loopInstructionFlowchart}
\end{figure}
















\sectionLabel{Method Call Instruction}{methodCallInstruction}




\sectionLabel{Modifier Call Instruction}{modifierCallInstruction}













\sectionLabel{L'instruction \texttt{switch}}{instructionSwitch}

L'instruction \galgas{switch} est dédiée aux types énumérés. Elle présente la syntaxe suivante :

\lstset{emph={constante, expression, liste_instructions}, emphstyle=\emph}
\begin{galgascode}
switch expression
when constante, constante, ... :
  liste_instructions
when constante, constante, ... :
  liste_instructions
...
end switch ;
\end{galgascode}


Où \galgas{expression} est une expression d'un type énuméré. Toutes les constantes de ce type doivent être nommées dans les branches \galgas{when}, une et une seule fois.

Par exemple, avec la déclaration :

\begin{galgascode}
enum @feuTricolore {
  vert, orange, rouge   
}
\end{galgascode}

On peut écrire :

\begin{galgascode}
@feuTricolore feu := ... ;

switch feu
when vert, orange:
  ...
when rouge :
  ...
end switch ;
\end{galgascode}










\section{Send Instruction}




\section{Warning Instruction}




\section{L'instruction \texttt{with}}


