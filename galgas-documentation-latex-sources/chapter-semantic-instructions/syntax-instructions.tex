%!TEX encoding = UTF-8 Unicode
%!TEX root = ../galgas-book.tex

%--------------------------------------------------------------
\chapter{Instructions syntaxiques}
%-------------------------------------------------------------

Les six instructions décrites dans ce chapitre ne peuvent être utilisées qu'à l'intérieur des règles de production, elles-mêmes obligatoirement placées dans un composant \galgas{syntax}.


\sectionLabel{Instruction d'appel de terminal}{instruction-appel-terminal}

\lstset{emph={$terminal$, parametres_entree, nom_index, traduction_dirigee_par_la_syntaxe}, emphstyle=\galgasEmphStyle}

Cette instruction permet de vérifier l'occurrence d'un terminal. Sa syntaxe présente deux options :
\begin{galgascode}
$terminal$ parametres_entree
indexing nom_index # Optionnel
traduction_dirigee_par_la_syntaxe # Optionnel
\end{galgascode}

Où :
\begin{itemize}
  \item \galgas{$terminal$} est le nom du terminal à vérifier ; il doit être l'un des terminaux déclarés par le lexique importé par le composant syntaxique ;
  \item \galgas{parametres_entree} est une liste de zéro, un ou plusieurs paramètres effectifs en entrée, en accord avec la déclaration du \galgas{$terminal$} dans le lexique ; comment écrire une liste de paramètres en entrée est présenté à la \refSectionPage{listeParametresEffectifsEntree} ;
  \item \galgas{nom_index} est le nom d'un index, tel que déclaré dans le lexique ; cet index sert à peupler le menu contextual de cross référence en Cocoa (\refSectionPage{indexingYourSourceFiles}) ;
  \item \galgas{traduction_dirigee_par_la_syntaxe} permet de préciser les options de \emph{traduction dirigée par la syntaxe} ; celle-ci est présentée en détail au \refChapterPage{chapitreTraductionDirigeeParSyntaxe}. 
\end{itemize}






\section{Instruction d'appel de non terminal}





\sectionLabel{Instruction \texttt{select}}{instructionSelectSyntaxique}

La syntaxe de l'instruction \galgas{select} syntaxique\footnote{Ne confondre avec l'instruction \galgas{select} lexicale, qui ne peut être utilisée que dans les analyseurs lexicaux (\refSubsectionPage{instructionSelectLexical}).} est la suivante :
\begin{galgascode}
A
select
  I0
or
  I1
or
  I2
...
end
B
\end{galgascode}

L'instruction est présentée avec trois branches \galgas{or} : l'instruction doit comporter au moins deux branches.

Sa signification est la suivante : l'occurrence de chaque \galgas{select} syntaxique peut être remplacée par un nouvel non-terminal particulier, que l'on va nommer \galgas{T}. La séquence précédente devient donc :
\begin{galgascode}
A
<T>
B
\end{galgascode}

Le non-terminal \galgas{T} se dérive de la façon suivante :
\begin{galgascode}
rule <T> { I0 }

rule <T> { I1 }

rule <T> { I2 }

...
\end{galgascode}







\sectionLabel{Instruction \texttt{repeat}}{instructionRepeatSyntaxique}

La syntaxe de l'instruction \galgas{repeat} syntaxique\footnote{Ne confondre avec l'instruction \galgas{repeat} lexicale, qui ne peut être utilisée que dans les analyseurs lexicaux (\refSubsectionPage{instructionRepeatLexical}).} est la suivante :
\begin{galgascode}
A
repeat
  I0
while
  I1
while
  I2
...
end
B
\end{galgascode}

L'instruction est présentée avec deux branches \galgas{while} : l'instruction doit comporter au moins une branche.

Sa signification est la suivante : l'occurrence de chaque \galgas{repeat} syntaxique peut être remplacée par un nouvel non-terminal particulier, que l'on va nommer \galgas{T}. La séquence précédente devient donc :
\begin{galgascode}
A
I0
<T>
B
\end{galgascode}

Le non-terminal \galgas{T} se dérive de la façon suivante :
\begin{galgascode}
rule <T> { I1 I0 <T> }

rule <T> { I2 I0 <T> }

...

rule <T> {  }
\end{galgascode}







\section{Instruction \texttt{parse}}







\sectionLabel{Instruction \texttt{send}}{instruction-send-syntaxique}




