%!TEX encoding = UTF-8 Unicode
%!TEX root = ../galgas-book.tex

%--------------------------------------------------------------
\chapter{Instructions syntaxiques}
%-------------------------------------------------------------

Les six instructions décrites dans ce chapitre ne peuvent être utilisées qu'à l'intérieur des règles de production, elles-mêmes obligatoirement placées dans un composant \galgas{syntax}.


\sectionLabel{Instruction d'appel de terminal}{instruction-appel-terminal}

\lstset{emph={$terminal$, parametres_entree, nom_index, traduction_dirigee_par_la_syntaxe}, emphstyle=\galgasEmphStyle}

Cette instruction permet de vérifier l'occurrence d'un terminal. Sa syntaxe présente deux options :
\begin{galgascode}
$terminal$ parametres_entree
indexing nom_index # Optionnel
traduction_dirigee_par_la_syntaxe # Optionnel
\end{galgascode}

Où :
\begin{itemize}
  \item \galgas{$terminal$} est le nom du terminal à vérifier ; il doit être l'un des terminaux déclarés par le lexique importé par le composant syntaxique ;
  \item \galgas{parametres_entree} est une liste de zéro, un ou plusieurs paramètres effectifs en entrée, en accord avec la déclaration du \galgas{$terminal$} dans le lexique ; comment écrire une liste de paramètres en entrée est présenté à la \refSectionPage{listeParametresEffectifsEntree} ;
  \item \galgas{nom_index} est le nom d'un index, tel que déclaré dans le lexique ; cet index sert à peupler le menu contextual de cross référence en Cocoa (\refSectionPage{indexingYourSourceFiles}) ;
  \item \galgas{traduction_dirigee_par_la_syntaxe} permet de préciser les options de \emph{traduction dirigée par la syntaxe} ; celle-ci est présentée en détail au \refChapterPage{chapitreTraductionDirigeeParSyntaxe}. 
\end{itemize}






\section{Instruction d'appel de non terminal}





\section{Instruction \texttt{select}}








\section{Instruction \texttt{repeat}}







\section{Instruction \texttt{parse}}







\sectionLabel{Instruction \texttt{send}}{instruction-send-syntaxique}




