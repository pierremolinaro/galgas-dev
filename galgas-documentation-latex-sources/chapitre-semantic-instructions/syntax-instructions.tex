%!TEX encoding = UTF-8 Unicode
%!TEX root = ../galgas-book.tex

%--------------------------------------------------------------
\chapter{Instructions syntaxiques}
%-------------------------------------------------------------

Les six instructions décrites dans ce chapitre ne peuvent être utilisées qu'à l'intérieur des règles de production, elles-mêmes obligatoirement placées dans un composant \ggs+syntax+.


\sectionLabel{Instruction d'appel de terminal}{instruction-appel-terminal}


Cette instruction permet de vérifier l'occurrence d'un terminal. Sa syntaxe présente deux options :
\begin{galgas}
$terminal$
parametres_entree # Optionnels
indexing nom_index # Optionnel
traduction_dirigee_par_la_syntaxe # Optionnel
\end{galgas}

Où :
\begin{itemize}
  \item \ggs+$terminal$+ est le nom du terminal à vérifier ; il doit être l'un des terminaux déclarés par le lexique importé par le composant syntaxique ;
  \item \ggs+parametres_entree+ est une liste de zéro, un ou plusieurs paramètres effectifs en entrée, en accord avec la déclaration du \ggs+$terminal$+ dans le lexique ; comment écrire une liste de paramètres en entrée est présenté à la \refSectionPage{listeParametresEffectifsEntree} ;
  \item \ggs+nom_index+ est le nom d'un index, tel que déclaré dans le lexique ; cet index sert à peupler le menu contextual de cross référence en Cocoa (\refSectionPage{indexingYourSourceFiles}) ;
  \item \ggs+traduction_dirigee_par_la_syntaxe+ permet de préciser les options de \emph{traduction dirigée par la syntaxe} ; celle-ci est présentée en détail au \refChapterPage{chapitreTraductionDirigeeParSyntaxe}. 
\end{itemize}






\section{Instruction d'appel de non terminal}





\sectionLabel{Instruction \texttt{select}}{instructionSelectSyntaxique}

La syntaxe de l'instruction \ggs+select+ syntaxique\footnote{Ne pas confondre avec l'instruction de sélection lexicale, qui ne peut être utilisée que dans les analyseurs lexicaux (\refSubsectionPage{instructionSelectLexical}).} est la suivante :
\begin{galgas}
A
select
  I0
or
  I1
or
  I2
end
B
\end{galgas}

L'instruction est présentée avec trois branches \ggs+or+ : l'instruction doit comporter au moins deux branches.

Sa signification est la suivante : l'occurrence de chaque \ggs+select+ syntaxique peut être remplacée par un nouvel non-terminal particulier, que l'on va nommer \ggs+T+. La séquence précédente devient donc :
\begin{galgas}
A
<T>
B
\end{galgas}

Le non-terminal \ggs+T+ se dérive de la façon suivante :
\begin{galgas}
rule <T> { I0 }

rule <T> { I1 }

rule <T> { I2 }
\end{galgas}







\sectionLabel{Instruction \texttt{repeat}}{instructionRepeatSyntaxique}

La syntaxe de l'instruction \ggs+repeat+ syntaxique\footnote{Ne pas confondre avec l'instruction de répétition lexicale, qui ne peut être utilisée que dans les analyseurs lexicaux (\refSubsectionPage{instructionRepeatLexical}).} est la suivante :
\begin{galgas}
A
repeat
  I0
while
  I1
while
  I2
end
B
\end{galgas}

L'instruction est présentée avec deux branches \ggs+while+ : l'instruction doit comporter au moins une branche.

Sa signification est la suivante : l'occurrence de chaque \ggs+repeat+ syntaxique peut être remplacée par un nouvel non-terminal particulier, que l'on va nommer \ggs+T+. La séquence précédente devient donc :
\begin{galgas}
A
I0
<T>
B
\end{galgas}

Le non-terminal \ggs+T+ se dérive de la façon suivante :
\begin{galgas}
rule <T> { I1 I0 <T> }

rule <T> { I2 I0 <T> }

rule <T> {  }
\end{galgas}







\section{Instruction \texttt{parse}}







\sectionLabel{Instruction \texttt{send}}{instruction-send-syntaxique}




