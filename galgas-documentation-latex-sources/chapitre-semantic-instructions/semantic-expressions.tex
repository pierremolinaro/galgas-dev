%!TEX encoding = UTF-8 Unicode
%!TEX root = ../galgas-book.tex

%--------------------------------------------------------------
\chapter{Expressions}
%-------------------------------------------------------------

\tableDesMatieresLocaleDeProfondeurRelative{1}


D'une manière classique, une expression est constituée d'\emph{opérandes} (\refSectionPage{operandesDansExpression}) et d'\emph{opérateurs} (\refSectionPage{operateursDansExpression}). La priorité des opérateurs est définie dans le \refTableauPage{prioriteOperateurs}.



%-------------------------------------------------------------

\sectionLabel{Opérandes}{operandesDansExpression}

\subsection{Identificateur}

\subsection{\texttt{self}}\index{self}

Dans une expression, \ggst+self+ représente une copie de l'objet courant. On ne peut donc utiliser \ggst+self+ que dans une expression à l'intérieur d'une \emph{méthode}, d'un \emph{getter}, d'un \emph{setter}, ou d'une extension (\refChapterPage{chapitreExtensions}). Sont donc exclues les procédures et les fonctions.

\ggst+self+ effectue un accès en lecture seule de l'objet courant.

Voici un exemple extrait de la section décrivant les \emph{extensions de getter} (\refSectionPage{categoryGetter}) :
\begin{galgas3}
getter @uint64 square -> @uint64 outResult {
  outResult = self * self
}
\end{galgas3}








\subsectionLabel{Expression de conversion polymorphique inverse}{expConversionPolymorphiqueInverse}

\fbox{\begin{minipage}{1.0\textwidth}
Cette construction est obsolète, et est remplacée avantageusement par l'\emph{affectation conditionnelle} (\refSubsectionPage{affectationConditionnelle}).
\end{minipage}}

La syntaxe de l'\emph{expression de conversion polymorphique inverse} est :

\begin{galgas3box}
expression as @T
\end{galgas3box}

Elle permet de renvoyer la valeur de \ggst+expression+ sous la forme d'un objet de type statique \ggst+@T+. À l'exécution, la conversion échoue si le type dynamique de l'\ggst+expression+ n'est pas \ggst+@T+ ou une de ses classes héritières~; une erreur sémantique est alors déclenchée, et l'expression renvoie un objet \emph{non construit}.

Pour tester le type dynamique de l'expression avant d'effectuer la conversion, utiliser la construction décrite à la \refSubsectionPage{testTypeDynamiqueExpression}. On peut aussi utiliser l'instruction \ggst+cast+ (\refSectionPage{instructionCast}).







\subsectionLabel{Test du type dynamique d'une expression}{testTypeDynamiqueExpression}

L'opérande \ggst+expression is conversion @T+ teste le type dynamique de \ggst+expression+ vis à vis du type \ggst+@T+ :
\begin{itemize}
\item si \ggst+conversion+ est \ggst+==+, la valeur renvoyée est \ggst+true+ si le type dynamique de l'\ggst+expression+ est exactement \ggst+@T+, et \ggst+false+ dans le cas contraire~;
\item si \ggst+conversion+ est  \ggst+>=+, la valeur renvoyée est \ggst+true+ si le type dynamique de l'\ggst+expression+ est \ggst+@T+ ou une de ses classes héritières, et \ggst+false+ dans le cas contraire~;
\item si \ggst+conversion+ est  \ggst+>+, la valeur renvoyée est \ggst+true+ si le type dynamique de l'\ggst+expression+ n'est pas \ggst+@T+ mais une de ses classes héritières, et \ggst+false+ dans le cas contraire.
\end{itemize}



Alliée à la construction précédente, elle permet de lancer une conversion uniquement si elle est possible :

\begin{galgas3}
if expression is == @B then
  let @B cst = expression as @B
  ...
elsif expression is >= @C then
  let @C cst = expression as @C
  ...
else
  message "conversion impossible"
end
\end{galgas3}





\subsection{Parenthèses}

Les parenthèses \ggst+(+ et \ggst+)+ permettent de forcer le groupement d'opérandes.






\subsection{\texttt{true} et \texttt{false}}

\ggst+true+ et \ggst+false+ sont les constantes du type \ggst+@bool+.

\subsection{Constante chaîne de caractères}

\subsection{Constante caractère}

\subsection{Constante entière}

Une constante entière est une séquence de chiffres décimaux, éventuellement séparés par le caractère de soulignement \texttt{\_}, et terminé par un suffixe. Ce suffixe détermine le type de la constante :
\begin{itemize}
  \item pas de suffixe : \ggst+@uint+~;
  \item suffixe \texttt{S} : \ggst+@sint+~;
  \item suffixe \texttt{L} : \ggst+@uint64+~;
  \item suffixe \texttt{LS} : \ggst+@sint64+~;
  \item suffixe \texttt{G} : \ggst+@bigint+.
\end{itemize}

\subsection{Constante flottante}

\subsection{Expression \texttt{if}}

\subsectionLabel{Appel de fonction}{appelFonction}

\subsectionLabel{Appel de getter}{appelGetter}








\subsection{Constructeur}

L'appel d'un constructeur instancie un nouvel objet. Sa syntaxe est :

\begin{galgas3box}
@T.constructeur {!exp0 !exp1 ...}
\end{galgas3box}

Par exemple :
\begin{galgas3}
@lstring ls = @lstring.new {!"" !@location.here{}}
@stringlist str = @lstringlist.emptyList {}
\end{galgas3}

Deux simplifications syntaxiques sont proposées :
\begin{itemize}
  \item si la liste des arguments est vide, les accolades peuvent être omises~;
  \item si le type peut être inféré, il peut être omis.
\end{itemize}


\subsubsection{Suppression des accolades}

Si la liste des arguments est vide, les accolades peuvent être omises.

\begin{galgas3}
@lstring ls = @lstring.new {!"" !@location.here}
@stringlist str = @lstringlist.emptyList
\end{galgas3}



\subsubsection{Inférence du type}

Si le type peut être inféré, il peut être omis (remarquer que ceci est valable aussi pour \ggst+@location.here+ qui peut être simplifié en \ggst+.here+.

\begin{galgas3}
@lstring ls = .new {!"" !.here}
@stringlist str = .emptyList
\end{galgas3}

















\subsectionLabel{Valeur d'une option}{appelOption}

Les options de la ligne de commande sont définies dans un composant \ggst+option+ (\refChapterPage{composantOption}). L'opérande \emph{appel d'option} permet d'obtenir des informations sur une option.

Sa syntaxe est \ggst+[option nom_composant_option.nom_option nom_info]+, où :
\begin{itemize}
  \item \ggst+nom_composant_option+ est le nom du composant \ggst+option+ qui déclare l'option~;
  \item \ggst+nom_option+ est le nom donné à l'option lors de sa déclaration~;
  \item \ggst+nom_info+ est le nom de l'information dont la valeur sera retournée par l'opérande.
\end{itemize}


Les informations qui peuvent être ainsi obtenues sont décrites dans le \refTableau{infosDeAppelOption}.
\begin{table}[t]
  \centering
  \begin{tabular}{llll}
  \textbf{nom\_info} & \textbf{Commentaire}  & \textbf{Type de la valeur retournée}\\
  \ggst+value+ & Valeur de l'option & \ggst+@T+ (le type de l'option)\\
  \ggst+char+ & Caractère d'appel de l'option & \ggst+@char+\\
  \ggst+string+ & Chaîne d'appel de l'option & \ggst+@string+\\
  \ggst+comment+ & Description de l'option & \ggst+@string+\\
  \end{tabular}
  \caption{Informations relatives à une option de la ligne de commande}
  \labelTableau{infosDeAppelOption}
\end{table}

Par exemple, si un composant \ggst+option+ est déclaré comme suit :
\begin{galgas3}
option mesOptions {
  @bool extractOption : 'S', "asm" -> "Extract assembly code"
}
\end{galgas3}

Alors :
\begin{itemize}
  \item \ggst+[option mesOptions.extractOption value]+ renvoie un \ggst+@bool+ qui vaut \ggst+true+ si l'option a été activée, \ggst+false+ dans le cas contraire~;
  \item \ggst+[option mesOptions.extractOption char]+ renvoie un \ggst+@char+ qui vaut \ggst+'S'+~;
  \item \ggst+[option mesOptions.extractOption string]+ renvoie un \ggst+@string+ qui vaut \ggst+"asm"+~;
  \item \ggst+[option mesOptions.extractOption comment]+ renvoie un \ggst+@string+ qui vaut \ggst+"Extract assembly code"+.
\end{itemize}

Noter qu'à partir de la version 3.1.4, les options \emph{quiet} et \emph{verbose} (\refSubsectionPage{optionsQuietVerbose}) ne peuvent pas être appelées par cette construction : il faut utiliser le \refConstructorPage{application}{verboseOutput}.



%-------------------------------------------------------------------

\sectionLabel{Opérateurs}{operateursDansExpression}

\subsectionLabel{Priorité des opérateurs}{prioriteOperateurs}

La priorité des opérateurs est définie dans le \refTableau{prioriteOperateurs}. Pour des opérateurs de même priorité, le groupement s'effectue de gauche à droite. Les parenthèses permettent de forcer l'ordre d'évaluation. Par exemple, \ggst*4 + 3 - 2 - 3* est équivalent à \ggst*((4 + 3) - 2) - 3*.

\begin{table}[t]
  \small
  \centering
  \begin{tabular}{llll}
  \textbf{Priorité} & \textbf{Opérateur}  & \textbf{Commentaire} & \textbf{Référence}\\
  0 (plus faible) & \ggst+|+ & « ou » logique & \refSubsectionPage{operateursLogiques}\\
    & \ggst+||+ & « ou », évaluation en court-circuit & \refSubsectionPage{operateursCourtCircuit}\\
    & \ggst+^+ & « ou exclusif » logique & \refSubsectionPage{operateursLogiques}\\
  1 & \ggst+&+ & « et » logique & \refSubsectionPage{operateursLogiques}\\
    & \ggst+&&+ & « et », évaluation en court-circuit & \refSubsectionPage{operateursCourtCircuit}\\
  2 & \ggst+==+, \ggst+!=+ & Test d'identité & \refSubsectionPage{operateursComparaison}\\
    & \ggst+<+, \ggst+<=+ & Comparaison & \refSubsectionPage{operateursComparaison}\\
    & \ggst+>+, \ggst+>=+ & Comparaison & \refSubsectionPage{operateursComparaison}\\
  3 & \ggst+<<+, \ggst+>>+ & Décalage & \refSubsectionPage{operateursDecalage}\\
    & \ggst*+*, \ggst*&+* & Addition, concaténation & \refSubsectionPage{operateursArithmétique}\\
    & \ggst+-+, \ggst+&-+ & Soustraction & \refSubsectionPage{operateursArithmétique}\\
  4 & \ggst+*+,  \ggst+&*+ & Multiplication & \refSubsectionPage{operateursArithmétique}\\
    & \ggst+/+, \ggst+&/+ & Division & \refSubsectionPage{operateursArithmétique}\\
    & \ggst+mod+ & Modulo & \refSubsectionPage{operateursArithmétique}\\
  5 & \ggst+-+, \ggst+&-+ & Négation arithmétique & \refSubsectionPage{operateursArithmétique}\\
  6 & \ggst+not+ & Complémentation booléenne & \refSubsectionPage{operateursLogiques}\\
  7 & \ggst+~+ & Complémentation bit-à-bit & \refSubsectionPage{complementationBitABit}\\
  8 (plus forte) & \ggst+.+ & Accès à un champ d'une structure & \refSubsectionPage{accesChampStructure}\\
  \end{tabular}
  \caption{Priorité des opérateurs}
  \labelTableau{prioriteOperateurs}
\end{table}

\subsectionLabel{Logique}{operateursLogiques}

\ggst+|+, \ggst+^+, \ggst+&+, \ggst+not+

\subsectionLabel{Logique, évaluation en court-circuit}{operateursCourtCircuit}

\ggst+||+ et \ggst+&&+

\subsectionLabel{Complémentation bit-à-bit}{complementationBitABit}

 \ggst+~+


\subsectionLabel{Comparaison}{operateursComparaison}


\subsectionLabel{Décalage}{operateursDecalage}

\ggst+<<+ et \ggst+>>+

\subsectionLabel{Arithmétique}{operateursArithmétique}

GALGAS propose des opérateurs arithmétiques qui détectent les dépassements de capacité : \ggst*+*, \ggst+-+, \ggst+*+. Une erreur est déclenchée à chaque débordement.

L'opérateur de division \ggst+/+ déclenche une erreur lors d'une division par zéro. Pour les types signés, une erreur est aussi déclenchée par la division de \ggst+@sint.min+ par \ggst+-1S+, et la division de \ggst+@sint64.min+ par \ggst+-1LS+.

Ces opérateurs ne détectent pas un dépassement de capacité : \ggst*&+*, \ggst+&-+, \ggst+&*+, et fonctionnent donc comme les opérateurs arithmétiques du C.

L'opérateur de division \ggst+&/+ ne déclenche aucune erreur lors d'une division par zéro, la valeur renvoyée est zéro. La division de \ggst+@sint.min+ par \ggst+-1S+ retourne \ggst+@sint.min+, et celle de \ggst+@sint64.min+ par \ggst+-1LS+ retourne \ggst+@sint64.min+.

L'opérateur \ggst+mod+ déclenche une erreur en cas de division par zéro.

La négation arithmétique (opérateur \ggst+-+ unaire) est uniquement définie sur les types entiers signés, et détecte les débordements provoqués par \ggst+- @sint.min+ et \ggst+- @sint64.min+.

Il existe un second opérateur de négation arithmétique (\ggst+&-+), lui aussi uniquement défini sur les types entiers signés, qui ne détecte aucun débordement: \ggst+&- @sint.min+ renvoie \ggst+@sint.min+ et \ggst+&- @sint64.min+ renvoie  \ggst+@sint64.min+.


On peut tester l'absence de débordement grâce aux \emph{getters} \ggst+canAdd+, \ggst+canSubstract+, \ggst+canMultiply+ et \ggst+canDivide+, définis pour les quatre types \ggst+@uint+, \ggst+@uint64+, \ggst+@sint+ et \ggst+@sint64+. Si ces \emph{getters} renvoient \ggst+true+, alors les opérations correspondantes s'effectueront sans débordement. Par exemple :
\begin{galgas3}
@uint a = ...
@uint b = ...
if [a canAdd !b] then
  @uint r = a + b # Pas de débordement
else
  # L'addition provoquerait un débordement
end
\end{galgas3}


\subsectionLabel{Accès à un champ d'une structure}{accesChampStructure}

\ggst+.+


