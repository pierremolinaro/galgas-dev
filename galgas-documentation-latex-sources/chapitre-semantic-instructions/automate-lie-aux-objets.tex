%!TEX encoding = UTF-8 Unicode
%!TEX root = ../galgas-book.tex

%--------------------------------------------------------------
\chapter{Contrôle de l'accès aux variables et aux constantes}
%-------------------------------------------------------------

\tableDesMatieresLocaleDeProfondeurRelative{1}


%---------------------- PARAMÉTRAGE DE L'AFFICHAGE DES AUTOMATES ------------------------------
% https://en.wikibooks.org/wiki/LaTeX/Colors

\newcommand\FondAutomate{LightGray!15}

\newcommand\VertAutomate{PineGreen}
\newcommand\OrangeAutomate{Orange}
\newcommand\RougeAutomate{Red}

\newcommand\EtatVert[4]{
  \node[draw=\VertAutomate,thick,fill=\FondAutomate,chamfered rectangle] (#1) at (#3, #4) {\small #2};
}

\newcommand\EtatOrange[4]{
  \node[draw=\OrangeAutomate,thick,fill=\FondAutomate,chamfered rectangle] (#1) at (#3, #4) {\small #2};
}

\newcommand\EtatRouge[4]{
  \node[draw=\RougeAutomate,thick,fill=\FondAutomate,chamfered rectangle] (#1) at (#3, #4) {\small #2};
}

\newcommand\FlecheVerte[4]{
  \draw [-stealth, thick, \VertAutomate] (#1) edge[#4] node[draw=\VertAutomate, rounded corners, fill=\FondAutomate] {\it #2} (#3) ;
}

\newcommand\FlecheOrange[4]{
  \draw [-stealth, thick, \OrangeAutomate] (#1) edge[#4] node[draw=\OrangeAutomate, rounded corners, fill=\FondAutomate] {\it#2} (#3) ;
}

\newcommand\FlecheRouge[4]{
  \draw [-stealth, thick, \RougeAutomate] (#1) edge[#4] node[draw=\RougeAutomate, rounded corners, fill=\FondAutomate] {\it#2} (#3) ;
}


\newcommand\BoucleVerte[3]{
  \path (#1) edge [loop #2, -stealth, thick, \VertAutomate] node[draw=\VertAutomate, rounded corners, fill=\FondAutomate] {\it#3} (#1) ;
}

\newcommand\BoucleOrange[3]{
  \path (#1) edge [loop #2, -stealth, thick, \OrangeAutomate] node[draw=\OrangeAutomate, rounded corners, fill=\FondAutomate] {\it#3} (#1) ;
}

\newcommand\BoucleRouge[3]{
  \path (#1) edge [loop #2, -stealth, thick, \RougeAutomate] node[draw=\RougeAutomate, rounded corners, fill=\FondAutomate] {\it#3} (#1) ;
}

%\newcommand\ActionInterdite[3]{
%  \node[#2 of #1,draw=\RougeAutomate,thick,fill=\FondAutomate] (Z) {~};
%  \draw[-stealth,thick,\RougeAutomate] (#1) edge node[draw=\RougeAutomate,rounded corners,fill=\FondAutomate] {\it #3} (Z);
%}

\newcommand\FlecheEtatInitial[2]{
  \node[#2 of #1,draw=black,circle,thick,fill=\FondAutomate] (Z) {};
  \draw[-stealth,thick,black] (Z) -- (#1);
}

%----------------------------------------------------------------------------------------------

Le compilateur GALGAS effectue une surveillance très stricte des accès aux objets -- constantes, variables et paramètres formels. Il signale ainsi par des \emph{alertes} et des \emph{erreurs} tout violation des règles d'accès.

On peut illustrer le résultat de cette surveillance par le fragment de code suivant :
\begin{galgas}
var @uint x
if condition then
  x = 2
end # Une erreur de compilation est déclenchée ici
\end{galgas}

Quelle serait la valeur de la variable \ggs!x! à l'issue de l'exécution de ce code ? Si \ggs!condition! est vrai, \ggs!x! vaut $2$ ; sinon, \ggs!x! n'a pas de valeur.

Le compilateur GALGAS détecte cette situation et émet un message d'erreur à l'issue de l'analyse sémantique de l'instruction \ggs!if!. Pour que l'analyse sémantique ne détecte pas d'erreur, il faut soit que les deux branches affectent une valeur à \ggs!x! :

\begin{galgas}
var @uint x
if condition then
  x = 2
else
  x = 4
end
\end{galgas}















\section{Variable locale}

\begin{figure}[t]
  \centering
  \small
  \begin{tikzpicture}
    \EtatVert{INVALID_STATE}{Invalid}{0cm}{0cm}
    \EtatOrange{DECLARED_STATE}{Declared}{3cm}{0cm}
    \EtatOrange{INITIALIZED_STATE}{Initialized}{6cm}{0cm}
    \EtatVert{READ_STATE}{Read}{9cm}{0cm}
    
    \FlecheEtatInitial{DECLARED_STATE}{above = 1cm}
    \FlecheEtatInitial{INITIALIZED_STATE}{above = 1cm}

    \FlecheVerte{DECLARED_STATE}{write}{INITIALIZED_STATE}{bend left}
    \FlecheRouge{DECLARED_STATE}{read}{INVALID_STATE}{bend right}
%    \ActionInterdite{DECLARED_STATE}{left = 2cm}{read}

    \BoucleOrange{INITIALIZED_STATE}{below}{write}
    \FlecheVerte{INITIALIZED_STATE}{read}{READ_STATE}{bend left=35}

    \FlecheVerte{READ_STATE}{write}{INITIALIZED_STATE}{bend left=35}
    \BoucleVerte{READ_STATE}{above}{read}

    \BoucleVerte{INVALID_STATE}{above}{read}
    \BoucleVerte{INVALID_STATE}{below}{read}
  \end{tikzpicture}
  \caption{Automate des états d'une variable locale}
  \labelFigure{automateEtatsVariableLocale}
\end{figure}










\section{Constante locale}

\begin{figure}[t]
  \centering
  \small
  \begin{tikzpicture}
    \EtatVert{INVALID_STATE}{Invalid}{0cm}{0cm}
    \EtatOrange{DECLARED_STATE}{Declared}{3cm}{0cm}
    \EtatOrange{INITIALIZED_STATE}{Initialized}{6cm}{0cm}
    \EtatVert{READ_STATE}{Read}{9cm}{0cm}
    
    \FlecheEtatInitial{DECLARED_STATE}{above = 1cm}
    \FlecheEtatInitial{INITIALIZED_STATE}{above = 1cm}

    \FlecheVerte{DECLARED_STATE}{write}{INITIALIZED_STATE}{bend left}
    \FlecheRouge{DECLARED_STATE}{read}{INVALID_STATE}{bend right}

%    \ActionInterdite{INITIALIZED_STATE}{below = 1cm}{write}
    \FlecheRouge{INITIALIZED_STATE}{write}{INVALID_STATE}{bend left= 20}
    \FlecheVerte{INITIALIZED_STATE}{read}{READ_STATE}{bend left=35}

    \FlecheRouge{READ_STATE}{write}{INVALID_STATE}{bend left}
%    \ActionInterdite{READ_STATE}{below = 1cm}{write}
    \BoucleVerte{READ_STATE}{above}{read}


    \BoucleVerte{INVALID_STATE}{above}{read}
    \BoucleVerte{INVALID_STATE}{below}{read}
  \end{tikzpicture}
  \caption{Automate des états d'une constante locale}
  \labelFigure{automateEtatsConstanteLocale}
\end{figure}







\section{Paramètre formel constant en entrée}

\begin{figure}[t]
  \centering
  \small
  \begin{tikzpicture}
    \EtatVert{constantInputDeclaredAsUnused}{Constant input declared unused}{0cm}{3cm}
    \EtatOrange{constantInput}{Constant input}{6cm}{3cm}
    \EtatVert{readConstantInput}{Read constant input}{0cm}{0cm}
    \EtatVert{droppedConstantInput}{Dropped constant input}{6cm}{0cm}

    \FlecheOrange{constantInputDeclaredAsUnused}{drop}{droppedConstantInput}{bend left=20}
    \FlecheVerte{constantInput}{read}{readConstantInput}{bend right=20}
    \FlecheVerte{constantInput}{drop}{droppedConstantInput}{bend left}

    \FlecheVerte{readConstantInput}{drop}{droppedConstantInput}{bend right}

    \FlecheOrange{constantInputDeclaredAsUnused}{read}{readConstantInput}{bend right}

    \BoucleRouge{droppedConstantInput}{right}{write, rw, read}
    \BoucleOrange{droppedConstantInput}{below}{drop}
    \BoucleRouge{readConstantInput}{below}{write, rw}
    \BoucleRouge{constantInput}{right}{write, rw}
    \BoucleVerte{readConstantInput}{left}{read}
    \BoucleRouge{constantInputDeclaredAsUnused}{above}{rw, write}
  \end{tikzpicture}
  \caption{Automate des états d'un paramètre formel constant en entrée}
  \labelFigure{automateEtatsParametreEntreeConstant}
\end{figure}









\section{Paramètre formel variable en entrée}

\begin{figure}[t]
  \centering
  \small
  \begin{tikzpicture}
    \EtatVert{inputDeclaredAsUnused}{Mutable input parameter declared unused}{0cm}{3cm}
    \EtatOrange{input}{Mutable unread input parameter}{6cm}{3cm}
    \EtatVert{readInput}{Read mutable input parameter}{0cm}{0cm}
    \EtatRouge{droppedInput}{Dropped mutable input parameter}{6cm}{0cm}

    \FlecheOrange{inputDeclaredAsUnused}{drop}{droppedInput}{bend right=20}
    \FlecheVerte{input}{read}{readInput}{bend right=20}
    \FlecheVerte{readInput}{write, rw}{input}{bend right=10}
    \FlecheVerte{input}{drop}{droppedInput}{bend left}
    \FlecheVerte{droppedInput}{write}{input}{bend left}

    \FlecheVerte{readInput}{drop}{droppedInput}{bend right}

    \FlecheOrange{inputDeclaredAsUnused}{read}{readInput}{bend right}
    \FlecheOrange{inputDeclaredAsUnused}{rw, write}{input}{bend left}

    \BoucleOrange{input}{above}{write}
    \BoucleVerte{input}{right}{rw}
    \BoucleRouge{droppedInput}{right}{rw, read}
    \BoucleOrange{droppedInput}{below}{drop}
    \BoucleVerte{readInput}{left}{read}
  \end{tikzpicture}
  \caption{Automate des états d'un paramètre formel variable en entrée}
  \labelFigure{automateEtatsParametreEntreeVariable}
\end{figure}















\section{Paramètre formel en entrée / sortie}

\begin{figure}[t]
  \centering
  \small
  \begin{tikzpicture}
    \EtatVert{inputOutputDeclaredAsUnused}{Inout parameter declared unused}{0cm}{3cm}
    \EtatOrange{unaccessedInputOutput}{Unaccessed inout parameter}{6cm}{3cm}
    \EtatVert{accessedInputOutput}{Accessed inout parameter}{0cm}{0cm}
    \EtatRouge{droppedInputOutput}{Dropped inout parameter}{6cm}{0cm}

    \FlecheVerte{unaccessedInputOutput}{read, rw, write}{accessedInputOutput}{bend right=20}

    \FlecheVerte{unaccessedInputOutput}{drop}{droppedInputOutput}{bend left}
    \FlecheVerte{droppedInputOutput}{write}{accessedInputOutput}{bend right}

    \FlecheVerte{accessedInputOutput}{drop}{droppedInputOutput}{bend right}

    \FlecheOrange{inputOutputDeclaredAsUnused}{read, rw, write}{accessedInputOutput}{bend right}
    \FlecheOrange{inputOutputDeclaredAsUnused}{drop}{droppedInputOutput}{bend left=20}

    \BoucleRouge{droppedInputOutput}{right}{rw, read}
    \BoucleOrange{droppedInputOutput}{below}{drop}
    \BoucleVerte{accessedInputOutput}{left}{read, rw, write}
  \end{tikzpicture}
  \caption{Automate des états d'un paramètre formel en entrée / sortie}
  \labelFigure{automateEtatsParametreEntreeSortie}
\end{figure}















\section{Paramètre formel en sortie}


\begin{figure}[t]
  \centering
  \small
  \begin{tikzpicture}
    \EtatVert{definedOutputParameter}{Defined output parameter}{0cm}{3cm}
    \EtatRouge{undefinedOutputParameter}{Undefined output parameter}{6cm}{3cm}

    \FlecheVerte{undefinedOutputParameter}{write}{definedOutputParameter}{bend right}

    \FlecheVerte{definedOutputParameter}{drop}{undefinedOutputParameter}{bend right}

    \BoucleOrange{definedOutputParameter}{left}{write}
    \BoucleRouge{undefinedOutputParameter}{right}{read, rw}
    \BoucleVerte{definedOutputParameter}{above}{rw, read}
    \BoucleOrange{undefinedOutputParameter}{below}{drop}
  \end{tikzpicture}
  \caption{Automate des états d'un paramètre formel en sortie}
  \labelFigure{automateEtatsParametreSortie}
\end{figure}

