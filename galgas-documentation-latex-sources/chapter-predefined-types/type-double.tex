\definitionSectionType{@double}

The \nomType{@double} object values correspond to the C type \nomType{double} values. You can initialize an \nomType{@double} object from a float constant:

\texttt{@double myDouble := 123.456 ;}

Note that a \nomType{@double} constant is characterized by the occurrence of the decimal point (.)


\readerSansArgument{sint}
{@double}
{1.9.9}
{@sint}
{Returns returns the receiver's value in an \lienSectionType{@sint} (32-bit signed integer) object.}
{if receiver?s value is outside \nomType{@sint} bounds, a runtime error is raised.}



\readerSansArgument{sint64}
{@double}
{1.9.9}
{@sint64}
{Returns returns the receiver's value in an \lienSectionType{@sint64} (64-bit signed integer) object.}
{if receiver?s value is outside \nomType{@sint64} bounds, a runtime error is raised.}




\readerSansArgument{string}
{@double}
{1.7.7}
{@string}
{Returns returns a decimal string representation of the receiver's value.}
{this reader never fails.}







\readerSansArgument{uint}
{@double}
{1.9.9}
{@uint}
{Returns returns the receiver's value in an \lienSectionType{@uint} (32-bit unsigned integer) object.}
{if receiver?s value is outside \nomType{@uint} bounds, a runtime error is raised.}





\readerSansArgument{uint64}
{@double}
{1.9.9}
{@uint64}
{Returns returns the receiver's value in an \lienSectionType{@uint64} (64-bit unsigned integer) object.}
{if receiver?s value is outside \nomType{@uint64} bounds, a runtime error is raised.}




\subsection{Arithmetic Operators}

The \nomType{@double} type supports the five arithmetic diadic operators:\newline

\begin{tabular}{|c|c|}
\hline
$+$ & Addition \\
\hline
$-$ & Substraction \\
\hline
$*$ & Multiplication \\
\hline
$/$ & Division \\
\hline
$\%$ & Modulo \\
\hline
\end{tabular}\newline

Theses operators require both arguments to be \nomType{@double} objects.\newline

A run-time error is raised if the operation leads to an overflow.

The \nomType{@double} type supports the following arithmetic unary operators:\newline

\begin{tabular}{|c|c|}
\hline
$+$ & No operation \\
$-$ & Negate \\
\hline
\end{tabular}\newline

This operator returns the receiver's value (an \nomType{@double} object).






\subsection{Comparison Operators}

The \nomType{@double} type supports the six comparison operators:\newline

\begin{tabular}{|c|c|}
\hline
$=$ & Equality \\
\hline
$!=$ & Non Equality \\
\hline
$<$  & Strict Lower Than \\
\hline
$<=$  & Lower or Equal \\
\hline
$>$  & Strict Greater Than \\
\hline
$>=$  & Greater or Equal \\
\hline
\end{tabular}\newline

Theses operators require both arguments to be \nomType{@double} objects, and return a \nomType{@bool} object.


