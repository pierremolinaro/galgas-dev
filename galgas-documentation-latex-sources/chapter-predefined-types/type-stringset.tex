%!TEX encoding = UTF-8 Unicode
%!TEX root = ../galgas-book.tex

\sectionTypePredefiniLabelIndex{stringset}

An \nomType{stringset} object value is a set of \nomType{string} values.\\


\constructeurSansArgument{emptySet}
{@stringset}
{1.3.0}
{@stringset}
{Creates and returns an empty \nomType{stringset} object.}
{}

\constructeurUnArgument{setWithString}
{@stringset}
{1.3.0}
{@stringset}
{@string inString}
{Creates and returns an \nomType{stringset} object that contains the value of the \emph{inString} argument object.}
{}


\readerSansArgument{count}
{@stringset}
{1.3.0}
{@uint}
{Returns the number of strings in the set.}
{}



\readerUnArgument{hasKey}
{@stringset}
{1.3.0}
{@bool}
{@string inString}
{Returns a boolean value that indicates whether the value of \emph{inString} argument is present in the set.}
{returns \motCle{true} if the value of \emph{inString} argument is present in the set, \motCle{false} otherwise.}




\modifierUnArgument{removeKey}
{@stringset}
{1.3.0}
{@string inString}
{Removes the value of \emph{inString} argument from the receiver's value.}
{if the receiver's value does not contain the value of \emph{inString} argument, this modifier leaves the receiver's value unchanged.}






\subsection{the \emph{+=} Operator}

The \emph{+=} operator adds a string value to the receiver. If the receiver's value already contains the added value, this operator has no effect.

\exempleTroisLignes
{}
{@string aString := ... ;}
{@stringset aStringSet := ... ;}
{aStringSet += !aString ;}




\subsection{the \emph{$\&$} Operator}

The \emph{$\&$} operator returns the intersection of its operand values.

\exempleTroisLignes
{}
{@stringset s1 := ... ;}
{@stringset s2 := ... ;}
{@stringset s := s1 \& s2 ; \# s is the intersection of s1 and s2}






\subsection{the \emph{$\textbar$} Operator}

The \emph{$\textbar$} operator returns the union of its operand values.

\exempleTroisLignes
{}
{@stringset s1 := ... ;}
{@stringset s2 := ... ;}
{@stringset s := s1 \textbar s2 ; \# s is the union of s1 and s2}






\subsection{the \emph{$-$} Operator}

The \emph{$-$} operator returns the difference of its operand values.

\exempleTroisLignes
{}
{@stringset s1 := ... ;}
{@stringset s2 := ... ;}
{@stringset s := s1 - s2 ; \# s is the difference of s1 and s2}








\subsection{Enumerating \nomType{stringset} objects}


The \motCle{foreach} instruction can be used for enumerating \nomType{stringset} values; enumeration is performed in the ascending order, or in the reverse alphabetical order using the '>' qualifier.

\texttt{@stringset s := ... ;}\newline
\textbf{foreach} \texttt {s} \textbf {do}\newline
\texttt{\# the \emph{key} constant has the value of current entry of \emph{s} stringset}\newline
\textbf{end foreach} \texttt{;}







\subsection{Comparison Operators}

The \nomType{stringset} type supports the six comparison operators:\newline

\begin{tabular}{|c|c|}
\hline
$=$ & Equality \\
\hline
$!=$ & Non Equality \\
\hline
$<$  & Strict Inclusion \\
\hline
$<=$  & Inclusion or Equality \\
\hline
$>$  & Strict Greater \\
\hline
$>=$  & Greater or Equality \\
\hline
\end{tabular}

Theses operators require both arguments to be \nomType{stringset} objects, and return a \nomType{stringset} object.


