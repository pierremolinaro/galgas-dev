%!TEX encoding = UTF-8 Unicode
%!TEX root = ../galgas-book.tex

\sectionTypePredefiniLabelIndex{sint}

An \nomType{sint} object value is a 32-bit signed integer value. You can initialize an \nomType{sint} object from an 32-bit signed integer constant:\\

\texttt{@sint mySignedInteger := 123\_456S ;}

Note that a 32-bit signed integer constant is characterized by the 'S' suffix.




\constructeurSansArgument{min}
{@sint}
{1.3.0}
{@sint}
{Returns an \nomType{sint} object that the minimum value of the 32-bit signed range.}
{the returned value is $-2^{31}$.}





\constructeurSansArgument{max}
{@sint}
{1.3.0}
{@sint}
{Returns an \nomType{sint} object that the maximum value of the 32-bit signed range.}
{the returned value is $2^{31}-1$.}





\readerSansArgument{double}
{@sint}
{1.9.8}
{@double}
{Returns the receiver's value converted in a \nomType{double} object.}
{as a 32-bit integer value can always be converted in a \nomType{double} value, this reader never fails.}





\readerSansArgument{sint64}
{@sint}
{1.6.12}
{@sint64}
{Returns the receiver's value in an \refTypePredefini{sint64} (64-bit signed integer) object.}
{as a 32-bit signed value can always be converted in a 64-bit signed value, this reader never fails.}

This reader is the only way to convert an \refTypePredefini{sint} object into an \refTypePredefini{sint64} object.





\readerSansArgument{string}
{@sint}
{1.6.12}
{@string}
{Returns a decimal string representation of the receiver's value.}
{for an hexadecimal string representation of the receiver's value, see \refReaderPage{uint}{hexString} and \refReaderPage{uint}{xString}.}







\readerSansArgument{uint}
{@sint}
{1.3.0}
{@uint}
{Returns the receiver's value in an \refTypePredefini{uint} (32-bit unsigned integer) object.}
{an error is raised is receiver's value is negative.}

This reader is the only way to convert an \refTypePredefini{sint} object into an \refTypePredefini{uint} object.




\readerSansArgument{uint64}
{@sint}
{1.3.0}
{@uint64}
{Returns the receiver's value in an \refTypePredefini{uint64} (64-bit unsigned integer) object.}
{an error is raised is receiver's value is negative.}

This reader is the only way to convert an \refTypePredefini{sint} object into an \refTypePredefini{uint64} object.





\subsection{Incrementation and decrementation}

The \refTypePredefini{sint} supports incrementation and decrementation instructions.

\texttt{@sint n := ... ; n ++ ; \# Incrementation}

\texttt{@sint p := ... ; p -- ; \# Decrementation}\newline

The incrementation instruction raises an error if receiver's value is equal to $2^{31}-1$.\newline

The decrementation instruction raises an error if receiver's value is equal to $-2^{31}$.\newline

Note that incrementation and decrementation are not available within an expression.




\subsection{Arithmetic Operators}

The \nomType{sint} type supports the five arithmetic diadic operators:\newline

\begin{tabular}{|c|c|}
\hline
$+$ & Addition \\
\hline
$-$ & Substraction \\
\hline
$*$ & Multiplication \\
\hline
$/$ & Division \\
\hline
$\%$ & Modulo \\
\hline
\end{tabular}

Theses operators require both arguments to be \nomType{sint} objects.\newline

A run-time error is raised if the operation leads to a 32-bit signed overflow.

The \nomType{sint} type supports the following arithmetic unary operators:\newline

\begin{tabular}{|c|c|}
\hline
$+$ & No operation \\
\hline
$-$ & Negate \\
\hline
\end{tabular}

This operator returns the receiver's value (an \nomType{sint} object). A run-time error is raised if "-" operator is invoked on an object whose value is $-2^{31}$.






\subsection{Shift Operators}


The \nomType{sint} type supports right and left shift operators:\newline

\begin{tabular}{|c|c|}
\hline
$<<$ & Left shift \\
\hline
$>>$ & Right shift \\
\hline
\end{tabular}

Theses operators require the right argument to be \nomType{sint} object, and the left argument to be \nomType{uint} object.\newline

Note the right shift inserts a zero bit in the most significant bit location if the receiver's value is negative, and a one bit otherwise (it is a arithmetic right shift).\newline

The actual amount of the shift is the value of the right-hand operand masked by 31, i.e. the shift distance is always between 0 and 31.




\subsection{Logical Operators}

The \nomType{sint} type supports the three bit-wise logical operators:\newline

\begin{tabular}{|c|c|}
\hline
$\&$ & Bit-wise and \\
\hline
\textbar & Bit-wise or \\
\hline
\^\  & Bit-wise exclusive or \\
\hline
\end{tabular}

Theses operators require both arguments to be \nomType{sint} objects.\newline


The \nomType{sint} type supports the bit-wise logical unary operator:\newline

\begin{tabular}{|c|c|}
\hline
$\sim$ & Bit-wise complementation \\
\hline
\end{tabular}

This operator returns an \nomType{sint} object.







\subsection{Comparison Operators}

The \nomType{sint} type supports the six comparison operators:\newline

\begin{tabular}{|c|c|}
\hline
$=$ & Equality \\
\hline
$!=$ & Non Equality \\
\hline
$<$  & Strict Lower Than \\
\hline
$<=$  & Lower or Equal \\
\hline
$>$  & Strict Greater Than \\
\hline
$>=$  & Greater or Equal \\
\hline
\end{tabular}

Theses operators require both arguments to be \nomType{sint} objects, and return a \nomType{bool} object.


