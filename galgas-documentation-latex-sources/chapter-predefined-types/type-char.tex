%!TEX encoding = UTF-8 Unicode
%!TEX root = ../galgas-book.tex

\chapitreTypePredefiniLabelIndex{char}

An \nomType{char} object value is an Unicode character. You can initialize an \nomType{char} object from a character constant:\\

\texttt{@char myCharacter := \textquotesingle  A\textquotesingle ;}

You have several ways for writing a literal character constant. In any case, it should define an assigned Unicode character. A compile-time error is raised if it does not.


A literal character constant is a single character or an escape sequence enclosed by single quotes ('\textquotesingle').

For an ASCII printable character:\\

\texttt{@char myCharacter := \textquotesingle a\textquotesingle ;}\\

If you want to get ASCII source text file, any character that does not correspond to an ASCII printable character should be expressed with an escape sequence.

Otherwise, for any printable Unicode character, you can write it directly, without escape sequence, provided your text file encoding supports this character:\\

\texttt{@char myCharacter := \textquotesingle\ae\textquotesingle ;}\\

The following escape sequences are defined (they begin with a '\textquotesingle').

\begin{tabular}{|c|c|}
\hline
Character Constant & Meaning \\
\hline
\texttt{\textquotesingle\textbackslash f\textquotesingle} & A Form Feed Character \\
\hline
\texttt{\textquotesingle\textbackslash n\textquotesingle} & A New Line Character \\
\hline
\texttt{\textquotesingle\textbackslash r\textquotesingle} & A Carriage Return Character \\
\hline
\texttt{\textquotesingle\textbackslash v\textquotesingle} & A Vertical Tabulation Character \\
\hline
\texttt{\textquotesingle\textbackslash\textbackslash\textquotesingle} & A back slash Character \\
\hline
\texttt{\textquotesingle\textbackslash 0\textquotesingle} & A Nul Character \\
\hline
\texttt{\textquotesingle\textbackslash\textquotesingle\textquotesingle} & A Single Quote Character \\
\hline
\end{tabular}


\begin{tabular}{|c|c|}
\hline
Character Constant & Meaning \\
\hline
\texttt{\textquotesingle\textbackslash uABCD\textquotesingle} & An Unicode Character \\
\hline
\end{tabular}

Where \emph{ABCD} is a four digit hexadecimal number that represents an assigned Unicode point code. For example:

\texttt{@char myCharacter := \textquotesingle\textbackslash u03A0\textquotesingle ; \# The '$\Sigma$' character}\\

Note: an unassigned point code raises a compile-time error:\\
\texttt{@char myCharacter := \textquotesingle\textbackslash uFFFF\textquotesingle ; \# The \textbackslash uFFFF point code is not assigned}\\


\begin{tabular}{|c|c|}
\hline
Character Constant & Meaning \\
\hline
\texttt{\textquotesingle\textbackslash Uabcdxyzt\textquotesingle} & An Unicode Character \\
\hline
\end{tabular}

Where \emph{abcdxyzt} is a eight digit hexadecimal number that represents an assigned Unicode point code. For example:

\texttt{@char myCharacter := \textquotesingle\textbackslash U0010170\textquotesingle ; \# The 'GREEK ACROPHONIC NAXIAN FIVE HUNDRED' character}\\

Note: an unassigned point code raises a compile-time error:\\

\texttt{@char myCharacter := \textquotesingle\textbackslash U0000FFFF\textquotesingle ; \# Raises a compile-time error: \textbackslash U0000FFFF is not assigned.}\\

Any point code beyond \textbackslash U0010FFFF is invalid and not assigned.




\section{Constructors}



\constructeurSansArgument{replacementCharacter}
{char}
{1.8.3}
{char}
{Returns an \nomType{char} object corresponding to Unicode replacement character (\texttt{\textquotesingle\textbackslash uFFFD}.}
{}







\constructeurUnArgument{unicodeCharacterWithUnsigned}
{char}
{1.8.3}
{char}
{@uint inValue}
{Returns an \nomType{char} object from an Unicode code point.}
{A run-time error is raised if the \emph{inValue} value does not represent an assigned Unicode value. You can check if an \nomType{uint} value represents an assigned Unicode value with the \refReaderPage{uint}{isUnicodeValueAssigned}.}


\section{Readers}


\readerSansArgument{isalnum}
{char}
{1.7.2}
{@bool}
{Returns an \nomType{bool} value indicating whether the receiver'value represents an ASCII letter or an ASCII digit.}
{returns \motCle{true} if the receiver'value represents an ASCII letter or an ASCII digit (between \texttt{\textquotesingle A\textquotesingle} and \texttt{\textquotesingle Z\textquotesingle}, or between \texttt{\textquotesingle a\textquotesingle} and \texttt{\textquotesingle z\textquotesingle}, or between \texttt{\textquotesingle 0\textquotesingle} and \texttt{\textquotesingle 9\textquotesingle}), and \motCle{false} otherwise.}




\readerSansArgument{isalpha}
{char}
{1.7.2}
{@bool}
{Returns an \nomType{bool} value indicating whether the receiver'value represents an ASCII letter.}
{returns \motCle{true} if the receiver'value represents an ASCII letter (between \texttt{\textquotesingle A\textquotesingle} and \texttt{\textquotesingle Z\textquotesingle}, or between \texttt{\textquotesingle a\textquotesingle} and \texttt{\textquotesingle z\textquotesingle}), and \motCle{false} otherwise.}




\readerSansArgument{iscntrl}
{char}
{1.7.2}
{@bool}
{Returns an \nomType{bool} value indicating whether the receiver'value represents an ASCII control character.}
{returns \motCle{true} if the receiver'value represents an ASCII control character (strictly before the \emph{SPACE} character), and \motCle{false} otherwise.}





\readerSansArgument{isdigit}
{char}
{1.7.2}
{@bool}
{Returns an \nomType{bool} value indicating whether the receiver'value represents an ASCII digit.}
{returns \motCle{true} if the receiver'value represents an ASCII digit (between \texttt{\textquotesingle 0\textquotesingle} and \texttt{\textquotesingle 9\textquotesingle}), and \motCle{false} otherwise.}





\readerSansArgument{islower}
{char}
{1.7.2}
{@bool}
{Returns an \nomType{bool} value indicating whether the receiver'value represents an ASCII lowercase ASCII letter.}
{returns \motCle{true} if the receiver'value represents an ASCII lowercase letter (between \texttt{\textquotesingle a\textquotesingle} and \texttt{\textquotesingle z\textquotesingle}), and \motCle{false} otherwise.}






\readerSansArgument{isUnicodeCommand}
{char}
{1.8.3}
{@bool}
{Returns an \nomType{bool} value indicating whether the receiver'value represents an Unicode command.}
{returns \motCle{true} if the receiver'value represents an Unicode command, and \motCle{false} otherwise.}






\readerSansArgument{isUnicodeLetter}
{char}
{1.8.3}
{@bool}
{Returns an \nomType{bool} value indicating whether the receiver'value represents an Unicode letter.}
{returns \motCle{true} if the receiver'value represents an Unicode letter, and \motCle{false} otherwise.}






\readerSansArgument{isUnicodeMark}
{char}
{1.8.3}
{@bool}
{Returns an \nomType{bool} value indicating whether the receiver'value represents an Unicode mark character.}
{returns \motCle{true} if the receiver'value represents an Unicode mark character, and \motCle{false} otherwise.}






\readerSansArgument{isUnicodePunctuation}
{char}
{1.8.3}
{@bool}
{Returns an \nomType{bool} value indicating whether the receiver'value represents an Unicode punctuation character.}
{returns \motCle{true} if the receiver'value represents an Unicode punctuation character, and \motCle{false} otherwise.}






\readerSansArgument{isUnicodeSeparator}
{char}
{1.8.3}
{@bool}
{Returns an \nomType{bool} value indicating whether the receiver'value represents an Unicode separator character.}
{returns \motCle{true} if the receiver'value represents an Unicode separator character, and \motCle{false} otherwise.}






\readerSansArgument{isUnicodeSymbol}
{char}
{1.8.3}
{@bool}
{Returns an \nomType{bool} value indicating whether the receiver'value represents an Unicode symbol character.}
{returns \motCle{true} if the receiver'value represents an Unicode symbol character, and \motCle{false} otherwise.}









\readerSansArgument{isupper}
{char}
{1.7.2}
{@bool}
{Returns an \nomType{bool} value indicating whether the receiver'value represents an ASCII uppercase ASCII letter.}
{returns \motCle{true} if the receiver'value represents an ASCII uppercase letter (between \texttt{\textquotesingle A\textquotesingle} and \texttt{\textquotesingle Z\textquotesingle}), and \motCle{false} otherwise.}





\readerSansArgument{string}
{char}
{1.5.5}
{@string}
{Returns returns a string representation of the receiver's value.}
{returns a one character \nomType{string} object, containing the receiver's value.}




\readerSansArgument{uint}
{char}
{1.7.7}
{@uint64}
{Returns an \nomType{uint} object representing the Unicode code point of the receiver's value.}
{}




\readerSansArgument{unicodeName}
{char}
{1.8.3}
{@string}
{Returns the unicode name of the receiver's value.}
{for an decimal string representation of the receiver's value, see the \refReaderPage{uint}{hexString}; for a decimal string representation of the receiver's value, see the \refReaderPage{uint}{string}.}

\exempleUneLigne
{}
{[\textquotesingle \AE \textquotesingle unicodeName] returns \textquotedbl LATIN CAPITAL LETTER AE \textquotedbl}




\readerSansArgument{unicodeToLower}
{char}
{1.8.3}
{char}
{Returns the lowercase character corresponding to the receiver's value.}
{if the receiver's value is an Unicode uppercase character, this reader returns the corresponding lowercase character. Otherwise, it returns the receiver's value.}

\exempleDeuxLignes
{}
{[\textquotesingle\AE\textquotesingle unicodeToLower] returns \textquotesingle\ae \textquotesingle}
{[\textquotesingle\ae\textquotesingle unicodeToLower] returns \textquotesingle\ae \textquotesingle}




\readerSansArgument{unicodeToUpper}
{char}
{1.8.3}
{char}
{Returns the uppercase character corresponding to the receiver's value.}
{if the receiver's value is an Unicode lowercase character, this reader returns the corresponding uppercase character. Otherwise, it returns the receiver's value.}

\exempleDeuxLignes
{}
{[\textquotesingle\AE\textquotesingle unicodeToUpper] returns \textquotesingle\AE\textquotesingle}
{[\textquotesingle\ae\textquotesingle unicodeToUpper] returns \textquotesingle\AE\textquotesingle}





\section{Comparison Operators}

The \nomType{char} type supports the six comparison operators:\newline

\begin{tabular}{|c|c|}
\hline
$=$ & Equality \\
\hline
$!=$ & Non Equality \\
\hline
$<$  & Strict Lower Than \\
\hline
$<=$  & Lower or Equal \\
\hline
$>$  & Strict Greater Than \\
\hline
$>=$  & Greater or Equal \\
\hline
\end{tabular}

Theses operators require both arguments to be \nomType{char} objects, and return a \nomType{bool} object. Comparison is done by comparing of the Unicode code point's value.


