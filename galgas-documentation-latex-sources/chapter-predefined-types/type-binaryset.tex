\definitionSectionType{@binaryset}

The \nomType{@binaryset} type encodes sets, binary relations, boolean mathematical expressions. It is implemented with Binary Decision Diagrams (BDD).




\constructeurUnArgument{binarySetWithBit}
{@binaryset}
{1.6.0}
{@binaryset}
{@uint inBitIndex}
{Returns an \nomType{@binaryset} object whose \emph{inBitIndex} bit is constraint to one.}
{}

\exempleDeuxLignes
{}
{@binaryset s [binarySetWithBit !2] ;}
{log s ; \# displays <@binaryset: 1XX>}





\constructeurTroisArguments{binarySetWithEqualComparison}
{@binaryset}
{1.6.0}
{@binaryset}
{@uint inLeftFirstIndex}
{@uint inBitCount}
{@uint inRightFirstIndex}
{Returns an \nomType{@binaryset} object that encodes a equality relation between two variables.}
{the constructor returns an binary set that encodes the \emph{a~==~b} relation, where \emph{a} is encoded from bit index \emph{inLeftFirstIndex} to bit index \emph{inLeftFirstIndex  + inBitCount - 1}, and \emph{b} is encoded from bit index \emph{inRightFirstIndex} to \emph{inRightFirstIndex + inBitCount - 1}.}

\exempleDeuxLignes
{}
{@binaryset s [binarySetWithEqualComparison !0 !2 !3] ;}
{log s; \# displays <@binaryset:~00x00, 01X01, 10X10, 11X11>}





\constructeurTroisArguments{binarySetWithEqualToConstant}
{@binaryset}
{1.6.0}
{@binaryset}
{@uint inLeftFirstIndex}
{@uint inBitCount}
{@uint64 inConstant}
{Returns an \nomType{@binaryset} object that encodes a equality relation between a variable and a constant.}
{the constructor returns an binary set that encodes the \emph{a~==~cst} relation, where \emph {a} is encoded from bit index \emph{inBitIndex} to bit index \emph{inBitIndex  + inBitCount - 1}, and \emph{cst} is defined by the \emph{inConstant} argument.}

\exempleDeuxLignes
{}
{@binaryset s [binarySetWithEqualToConstant !0 !6 !23L] ;}
{log s; \# displays <@binaryset:~10111>}





\constructeurTroisArguments{binarySetWithGreaterOrEqualComparison}
{@binaryset}
{1.6.0}
{@binaryset}
{@uint inLeftFirstIndex}
{@uint inBitCount}
{@uint inRightFirstIndex}
{Returns an \nomType{@binaryset} object that encodes a greater or equal relation between two variables.}
{the constructor returns an binary set that encodes the \emph{a~>=~b} relation, where \emph{a} is encoded from bit index \emph{inLeftFirstIndex} to bit index \emph{inLeftFirstIndex  + inBitCount - 1}, and \emph{b} is encoded from bit index \emph{inRightFirstIndex} to \emph{inRightFirstIndex + inBitCount - 1}.}

\exempleDeuxLignes
{}
{@binaryset s [binarySetWithGreaterOrEqualComparison !0 !2 !3] ;}
{log s; \# displays <@binaryset:~00XXX, 01X01, 01X1X, 10X1X, 11X11>}





\constructeurTroisArguments{binarySetWithGreaterOrEqualToConstant}
{@binaryset}
{1.6.0}
{@binaryset}
{@uint inLeftFirstIndex}
{@uint inBitCount}
{@uint64 inConstant}
{Returns an \nomType{@binaryset} object that encodes a greater or equal relation between a variable and a constant.}
{the constructor returns an binary set that encodes the \emph{a~>=~cst} relation, where \emph {a} is encoded from bit index \emph{inBitIndex} to bit index \emph{inBitIndex  + inBitCount - 1}, and \emph{cst} is defined by the \emph{inConstant} argument.}





\constructeurTroisArguments{binarySetWithLowerOrEqualComparison}
{@binaryset}
{1.6.0}
{@binaryset}
{@uint inLeftFirstIndex}
{@uint inBitCount}
{@uint inRightFirstIndex}
{Returns an \nomType{@binaryset} object that encodes a lower or equal relation between two variables.}
{the constructor returns an binary set that encodes the \emph{a~<=~b} relation, where \emph{a} is encoded from bit index \emph{inLeftFirstIndex} to bit index \emph{inLeftFirstIndex  + inBitCount - 1}, and \emph{b} is encoded from bit index \emph{inRightFirstIndex} to \emph{inRightFirstIndex + inBitCount - 1}.}

\exempleDeuxLignes
{}
{@binaryset s [binarySetWithLowerOrEqualComparison !0 !2 !3] ;}
{log s; \# displays <@binaryset:~00X00, 01X0X, 10X0X, 10X10, 11XXX>}





\constructeurTroisArguments{binarySetWithLowerOrEqualToConstant}
{@binaryset}
{1.6.0}
{@binaryset}
{@uint inLeftFirstIndex}
{@uint inBitCount}
{@uint64 inConstant}
{Returns an \nomType{@binaryset} object that encodes a lower or equal relation between a variable and a constant.}
{the constructor returns an binary set that encodes the \emph{a~<=~cst} relation, where \emph {a} is encoded from bit index \emph{inBitIndex} to bit index \emph{inBitIndex  + inBitCount - 1}, and \emph{cst} is defined by the \emph{inConstant} argument.}





\constructeurTroisArguments{binarySetWithNotEqualComparison}
{@binaryset}
{1.6.0}
{@binaryset}
{@uint inLeftFirstIndex}
{@uint inBitCount}
{@uint inRightFirstIndex}
{Returns an \nomType{@binaryset} object that encodes an inequality relation between two variables.}
{the constructor returns an binary set that encodes the \emph{a~!=~b} relation, where \emph{a} is encoded from bit index \emph{inLeftFirstIndex} to bit index \emph{inLeftFirstIndex  + inBitCount - 1}, and \emph{b} is encoded from bit index \emph{inRightFirstIndex} to \emph{inRightFirstIndex + inBitCount - 1}.}

\exempleDeuxLignes
{}
{@binaryset s [binarySetWithNotEqualComparison !0 !2 !3] ;}
{log s; \# displays <@binaryset:~00X01, 00X1X, 01X00, 01X1X, 10X0X, 10X11, 11X0X, 11X10>}





\constructeurTroisArguments{binarySetWithNotEqualToConstant}
{@binaryset}
{1.6.0}
{@binaryset}
{@uint inLeftFirstIndex}
{@uint inBitCount}
{@uint64 inConstant}
{Returns an \nomType{@binaryset} object that encodes an inequality relation between a variable and a constant.}
{the constructor returns an binary set that encodes the \emph{a~!=~cst} relation, where \emph {a} is encoded from bit index \emph{inBitIndex} to bit index \emph{inBitIndex  + inBitCount - 1}, and \emph{cst} is defined by the \emph{inConstant} argument.}







\constructeurUnArgument{binarySetWithPredicateString}
{@binaryset}
{1.6.0}
{@binaryset}
{@string inPredicateString}
{Returns the \nomType{@binaryset} object described by the \emph{inPredicateString} argument.}
{the \emph{inBitString} argument string encodes a predicate string, such as those returned by \lienReader{@binaryset}{predicateStringValue}.}
\begin{description}
\item The \emph{inBitString} argument string characters should have one of the five following values:
\begin{itemize}
\item \texttt{\textquotesingle 0\textquotesingle}: a bit set to zero;
\item \texttt{\textquotesingle 1\textquotesingle}: a bit set to one;
\item \texttt{\textquotesingle X\textquotesingle}: a don't care bit;
\item \texttt{\textquotesingle~\textquotesingle}: a separator (non significant character);
\item \texttt{\textquotesingle\textbar\textquotesingle}: the boolean \emph{or} operation (in infix notation).
\end{itemize}
\end{description}


\exempleDeuxLignes
{An empty predicate string (or a string containing only spaces) provides an empty binary set:}
{@binaryset s [binarySetWithPredicateString !\textquotedbl~\textquotedbl] ;}
{@bool b := [s isEmptySet]; \# b is true}


\exempleDeuxLignes
{A predicate string containing only \texttt{\textquotesingle X\textquotesingle} characters (at least one) provides an full binary set:}
{@binaryset s [binarySetWithPredicateString !\textquotedbl X X\textquotedbl] ; \# Spaces are non significant}
{@bool b := [s isFullSet]; \# b is true}


\exempleDeuxLignes
{You can use the boolean '\textbar' operator for providing an or'ed values:}
{@binaryset s [binarySetWithPredicateString !\textquotedbl 1100\textquotedbl] ; \# 12 in decimal}
{log s; \# Displays <@binaryset:~1100>}


\exempleDeuxLignes
{A predicate string can encode a binary value (MSB first):}
{@binaryset s [binarySetWithPredicateString !\textquotedbl 1100\textbar 1101\textquotedbl] ;}
{log s; \# Displays <@binaryset:~110X>}



\exempleDeuxLignes
{You can use you can use don't care bits and '\textbar' operator together:}
{@binaryset s [binarySetWithPredicateString !\textquotedbl 11X00X0\textbar 111XXX\textquotedbl] ;}
{log s; \# Displays <@binaryset:~1100X0, 111XXX>}





\constructeurTroisArguments{binarySetWithStrictGreaterComparison}
{@binaryset}
{1.6.0}
{@binaryset}
{@uint inLeftFirstIndex}
{@uint inBitCount}
{@uint inRightFirstIndex}
{Returns an \nomType{@binaryset} object that encodes a strict greater than relation between two variables.}
{the constructor returns an binary set that encodes the \emph{a~>~b} relation, where \emph{a} is encoded from bit index \emph{inLeftFirstIndex} to bit index \emph{inLeftFirstIndex  + inBitCount - 1}, and \emph{b} is encoded from bit index \emph{inRightFirstIndex} to \emph{inRightFirstIndex + inBitCount - 1}.}

\exempleDeuxLignes
{}
{@binaryset s [binarySetWithStrictGreaterComparison !0 !2 !3] ;}
{log s; \# displays <@binaryset:~00X01, 00X1X, 01X1X, 10X11>}





\constructeurTroisArguments{binarySetWithStrictGreaterThanConstant}
{@binaryset}
{1.6.0}
{@binaryset}
{@uint inLeftFirstIndex}
{@uint inBitCount}
{@uint64 inConstant}
{Returns an \nomType{@binaryset} object that encodes a strict greater than relation between a variable and a constant.}
{the constructor returns an binary set that encodes the \emph{a~>~cst} relation, where \emph {a} is encoded from bit index \emph{inBitIndex} to bit index \emph{inBitIndex  + inBitCount - 1}, and \emph{cst} is defined by the \emph{inConstant} argument.}





\constructeurTroisArguments{binarySetWithStrictLowerComparison}
{@binaryset}
{1.6.0}
{@binaryset}
{@uint inLeftFirstIndex}
{@uint inBitCount}
{@uint inRightFirstIndex}
{Returns an \nomType{@binaryset} object that encodes a strict lower than relation between two variables.}
{the constructor returns an binary set that encodes the \emph{a~<~b} relation, where \emph{a} is encoded from bit index \emph{inLeftFirstIndex} to bit index \emph{inLeftFirstIndex  + inBitCount - 1}, and \emph{b} is encoded from bit index \emph{inRightFirstIndex} to \emph{inRightFirstIndex + inBitCount - 1}.}

\exempleDeuxLignes
{}
{@binaryset s [binarySetWithStrictLowerComparison !0 !2 !3] ;}
{log s; \# displays <@binaryset:~01X00, 10X0X, 11X0X, 11X10>}





\constructeurTroisArguments{binarySetWithStrictLowerThanConstant}
{@binaryset}
{1.6.0}
{@binaryset}
{@uint inLeftFirstIndex}
{@uint inBitCount}
{@uint64 inConstant}
{Returns an \nomType{@binaryset} object that encodes a strict lower than relation between a variable and a constant.}
{the constructor returns an binary set that encodes the \emph{a~<~cst} relation, where \emph {a} is encoded from bit index \emph{inBitIndex} to bit index \emph{inBitIndex  + inBitCount - 1}, and \emph{cst} is defined by the \emph{inConstant} argument.}





\constructeurSansArgument{emptyBinarySet}
{@binaryset}
{1.6.0}
{@binaryset}
{Returns an empty \nomType{@binaryset} object.}
{}





\constructeurSansArgument{fullBinarySet}
{@binaryset}
{1.6.0}
{@binaryset}
{Returns a full \nomType{@binaryset} object.}
{}






\readerSansArgument{accessibleStates}
{@binaryset}
{1.6.0}
{@binaryset}
{Returns the set of accessible states from an initial state set.}
{computes the set of accessible states from the \emph{inInitialStateSet} state set using the accessibility relation encoded by the receiver.}

\exempleUneLigne
{}
{@binaryset graph [binarySetWithPredicateString !\textquotedbl 0001 0000\textquotedbl] ; \# Edge 0 -> 1}
\texttt{graph := graph | [@binaryset binarySetWithPredicateString !\textquotedbl 0010 0001\textquotedbl] ; \# Edge 1 -> 2}\\
\texttt{graph := graph | [@binaryset binarySetWithPredicateString !\textquotedbl 0011 0010\textquotedbl] ; \# Edge 2 -> 3}\\
\texttt{graph := graph | [@binaryset binarySetWithPredicateString !\textquotedbl 0100 0011\textquotedbl] ; \# Edge 3 -> 4}\\
\texttt{graph := graph | [@binaryset binarySetWithPredicateString !\textquotedbl 0101 0100\textquotedbl] ; \# Edge 4 -> 5}\\
\texttt{@binaryset initialState [binarySetWithPredicateString !\textquotedbl 0000\textquotedbl] ; \# 0 is the initial state}\\
\texttt{@binaryset accessibleStates := [graph accessibleStates !initialState !4] ;}\\
\texttt{message \textquotedbl Accessible:\textquotedbl ;}\\
\texttt{@uint64list valueList := [accessibleStates uint64ValueList !4] ;}\\
\texttt{\motCle{foreach} valueList do}\\
\texttt{message \textquotedbl~\textquotedbl~. [mValue string] ;}\\
\texttt{end foreach ;}\\
\texttt{message \textquotedbl\textbackslash n\textquotedbl ;}\\


This program displays: \texttt{Accessible: 0 1 2 3 4 5}.




\readerDeuxArguments{binarySetByTranslatingFromIndex}
{@binaryset}
{1.6.3}
{@binaryset}
{@uint inFirstIndex}
{@uint inTranslation}
{Returns a \nomType{@binaryset} value computed by translating the receiver's value by \emph{inTranslation} bits from index \emph{inFirstIndex}.}
{}



\readerSansArgument{compressedValueCount}
{@binaryset}
{1.6.0}
{@uint64}
{Returns in an \nomType{@uint64} value the number of different compressed string values encoded by receiver's value.}
{}




\readerUnArgument{compressedStringValueList}
{@binaryset}
{1.6.0}
{@stringlist}
{@uint inBitCount}
{Returns the list of compressed string values corresponding to receiver's value, considering it uses \emph{inBitCount} bits.}
{}










\readerDeuxArguments{containsValue}
{@binaryset}
{1.6.4}
{@bool}
{@uint inFirstBit}
{@uint inBitCount}
{Returns an \nomType{@bool} value indicating whether the receiver'value contains a given value.}
{returns \motCle{true} if the receiver's contains a value, and \motCle{false} otherwise; this value is computed from the \emph{inBitCount} first bits of \emph{inValue} value, shifted left by \emph{inFirstBit}.}


\exempleUneLigne
{}
{@binaryset s [binarySetWithPredicateString !\textquotedbl 0 00XX X111\textbar 1 1111 1111\textquotedbl] ;}
\texttt{log s ; \# Displays <@binaryset:~000XXX111, 111111111>}\\
\texttt{@bool b := [s containsValue !127L !0 !7] ;}\\
\texttt{log b ; \# Displays <@bool:true>}\\
\texttt{b := [s containsValue !31L !1 !7] ;}\\
\texttt{log b ; \# Displays <@bool:true>}\\
\texttt{b := [s containsValue !63L !1 !8] ;}\\
\texttt{log b ; \# Displays <@bool:false>}\\
\texttt{b := [s containsValue !7L !0 !9] ;}\\
\texttt{log b ; \# Displays <@bool:true>}\\
\texttt{b := [s containsValue !7L !0 !10] ;}\\
\texttt{log b ; \# Displays <@bool:true>}\\
\texttt{b := [s containsValue !32767L !1 !12] ;}\\
\texttt{log b ; \# Displays <@bool:true>}\\








\readerUnArgument{equalTo}
{@binaryset}
{1.6.0}
{@binaryset}
{@binaryset inOperand}
{Returns the complement of the exclusive or between the receiver's value and the operand's value.}
{}

Note that \texttt{[a equalTo !b]} is equivalent to \texttt{$\sim$ (a $\wedge$ b)}.

This operation returns an \nomType{@binaryset} value; do not confuse with \emph{==} operator that returns an \nomType{@bool} value.







\readerUnArgument{existOnBitIndex}
{@binaryset}
{1.6.3}
{@binaryset}
{@uint inBitIndex}
{Returns the binary computed by applying the \emph{exist} operator on the \emph{inBitIndex} bit of the receiver's value.}
{}







\readerDeuxArguments{existsOnBitRange}
{@binaryset}
{1.6.3}
{@binaryset}
{@uint inFirstBitIndex}
{@uint inBitCount}
{Returns the binary computed by applying the \emph{exist} operator on the receiver's value, from \emph{inFirstBitIndex} bit index until the \emph{inFirstBitIndex + inBitCount - 1} bit index.}
{}


\exempleUneLigne
{}
{@binaryset s [binarySetWithPredicateString !\textquotedbl 01110010\textquotedbl] ;}
\texttt{log s ; \# Displays <@binaryset:~01110010>}\\
\texttt{@binaryset ss := [s existsOnBitRange !2 !3] ;}\\
\texttt{log s ; \# Displays <@binaryset:~011XXX10>}\\







\readerUnArgument{existOnBitIndexAndBeyond}
{@binaryset}
{1.6.3}
{@binaryset}
{@uint inBitIndex}
{Returns the binary set computed by applying the \emph{exist} operator on all bits from \emph{inFirstBitIndex} bit index of the receiver's value.}
{}







\readerUnArgument{forAllOnBitIndex}
{@binaryset}
{1.6.0}
{@binaryset}
{@uint inBitIndex}
{Returns the binary set computed by applying the \emph{for all} operator on the \emph{inFirstBitIndex} bit index of the receiver's value.}
{}







\readerUnArgument{forAllOnBitIndexAndBeyond}
{@binaryset}
{1.6.0}
{@binaryset}
{@uint inBitIndex}
{Returns the binary computed by applying the \emph{for all} operator on all bits from \emph{inFirstBitIndex} bit index of the receiver's value.}
{}








\readerUnArgument{greaterOrEqualTo}
{@binaryset}
{1.6.0}
{@binaryset}
{@binaryset inOperand}
{Returns the complement of the exclusive or between the receiver's value and the operand's value.}
{}

Note that \texttt{[a greaterOrEqualTo !b]} is equivalent to \texttt{(a \textbar ~$\sim$b)}.








\readerSansArgument{isEmpty}
{@binaryset}
{1.6.0}
{@bool}
{Returns a \nomType{@bool} value that indicates whether the receiver's value is the empty set.}
{returns \motCle{true} if receiver's value is the empty set, and \motCle{false} otherwise.}







\readerSansArgument{isFull}
{@binaryset}
{1.6.0}
{@bool}
{Returns a \nomType{@bool} value that indicates whether the receiver's value is the full set.}
{returns \motCle{true} if receiver's value is the full set, and \motCle{false} otherwise.}







\readerDeuxArguments{ITE}
{@binaryset}
{1.6.3}
{@binaryset}
{@binaryset inThenOperand}
{@binaryset inElseOperand}
{Returns the binary set computed by applying the \emph{ite} operator on the receiver's value, the \emph{inThenOperand} argument, and the  \emph{inElseOperand} argument.}
{\texttt{ite (x, y, z)} is \texttt{(x \& y) \textbar ($\sim$x \& z)}.}







\readerUnArgument{lowerOrEqualTo}
{@binaryset}
{1.6.0}
{@binaryset}
{@binaryset inOperand}
{Returns the binary set computed by applying the \emph{lower or equal} operator on the receiver's value and the \emph{inOperand} argument.}
{\texttt{[a lowerOrEqualTo !b]} is \texttt{(($\sim$x) \textbar y)}.}







\readerUnArgument{notEqualTo}
{@binaryset}
{1.6.0}
{@binaryset}
{@binaryset inOperand}
{Returns the binary set computed by applying the \emph{not equal} operator on the receiver's value and the \emph{inOperand} argument.}
{\texttt{[a notEqualTo !b]} is \texttt{(x $\wedge$ y)}.}







\readerSansArgument{predicateStringValue}
{@binaryset}
{1.6.0}
{@string}
{Returns a string representation of the receiver's value.}
{the returned string is compatible with the \lienConstructeur{@binaryset}{binarySetWithPredicateString}.}







\readerUnArgument{strictGreaterThan}
{@binaryset}
{1.6.0}
{@binaryset}
{@binaryset inOperand}
{Returns the binary set computed by applying the \emph{strict greater} operator on the receiver's value and the \emph{inOperand} argument.}
{\texttt{[a strictGreaterThan !b]} is \texttt{(x \& $\sim$y)}.}







\readerUnArgument{strictLowerThan}
{@binaryset}
{1.6.0}
{@binaryset}
{@binaryset inOperand}
{Returns the binary set computed by applying the \emph{strict lower} operator on the receiver's value and the \emph{inOperand} argument.}
{\texttt{[a strictLowerThan !b]} is \texttt{($\sim$x \& y)}.}







\readerUnArgument{stringValueList}
{@binaryset}
{1.6.0}
{@stringlist}
{@uint inBitCount}
{Returns the list of string values corresponding to receiver's value, considering it uses \emph{inBitCount} bits.}
{}







\readerDeuxArguments{stringValueListWithNameList}
{@binaryset}
{1.9.3}
{@stringlist}
{@uint inBitCount}
{@stringlist inNameList}
{Returns the list of named values corresponding to receiver's value, considering it uses \emph{inBitCount} bits.}
{first, the receiver is enumerated, considering it uses \emph{inBitCount} bits. Each enumerated value is used as an index of \emph{inNameList}, and the string value at this index is appended at the end of the returned value.}







\readerTroisArguments{swap132}
{@binaryset}
{1.6.0}
{@binaryset}
{@uint inBitCount1}
{@uint inBitCount2}
{@uint inBitCount3}
{Returns the transposed \emph{(x, z, y)} relation.}
{this reader considers that the receiver encodes an \emph{(x, y, z)} relation, where \emph{x} is defined by bits index \emph{0} to \emph{inBitCount1  - 1}, \emph{y} is defined by bits index \emph{inBitCount1} to \emph{inBitCount1 + inBitCount2 - 1} and  \emph{z} is defined by bits index \emph{inBitCount1 + inBitCount2} to \emph{inBitCount1 + inBitCount2 + inBitCount3 - 1}.}







\readerDeuxArguments{swap21}
{@binaryset}
{1.6.0}
{@binaryset}
{@uint inBitCount1}
{@uint inBitCount2}
{Returns the transposed \emph{(y, x)} relation.}
{this reader considers that the receiver encodes an \emph{(x, y)} relation, where \emph{x} is defined by bits index \emph{0} to \emph{inBitCount1  - 1}, \emph{y} is defined by bits index \emph{inBitCount1} to \emph{inBitCount1 + inBitCount2 - 1}.}







\readerTroisArguments{swap213}
{@binaryset}
{1.6.0}
{@binaryset}
{@uint inBitCount1}
{@uint inBitCount2}
{@uint inBitCount3}
{Returns the transposed \emph{(y, x, z)} relation.}
{this reader considers that the receiver encodes an \emph{(x, y, z)} relation, where \emph{x} is defined by bits index \emph{0} to \emph{inBitCount1  - 1}, \emph{y} is defined by bits index \emph{inBitCount1} to \emph{inBitCount1 + inBitCount2 - 1} and  \emph{z} is defined by bits index \emph{inBitCount1 + inBitCount2} to \emph{inBitCount1 + inBitCount2 + inBitCount3 - 1}.}







\readerTroisArguments{swap231}
{@binaryset}
{1.6.0}
{@binaryset}
{@uint inBitCount1}
{@uint inBitCount2}
{@uint inBitCount3}
{Returns the transposed \emph{(y, z, x)} relation.}
{this reader considers that the receiver encodes an \emph{(x, y, z)} relation, where \emph{x} is defined by bits index \emph{0} to \emph{inBitCount1  - 1}, \emph{y} is defined by bits index \emph{inBitCount1} to \emph{inBitCount1 + inBitCount2 - 1} and  \emph{z} is defined by bits index \emph{inBitCount1 + inBitCount2} to \emph{inBitCount1 + inBitCount2 + inBitCount3 - 1}.}







\readerTroisArguments{swap312}
{@binaryset}
{1.6.0}
{@binaryset}
{@uint inBitCount1}
{@uint inBitCount2}
{@uint inBitCount3}
{Returns the transposed \emph{(z, x, y)} relation.}
{this reader considers that the receiver encodes an \emph{(x, y, z)} relation, where \emph{x} is defined by bits index \emph{0} to \emph{inBitCount1  - 1}, \emph{y} is defined by bits index \emph{inBitCount1} to \emph{inBitCount1 + inBitCount2 - 1} and  \emph{z} is defined by bits index \emph{inBitCount1 + inBitCount2} to \emph{inBitCount1 + inBitCount2 + inBitCount3 - 1}.}







\readerTroisArguments{swap321}
{@binaryset}
{1.6.0}
{@binaryset}
{@uint inBitCount1}
{@uint inBitCount2}
{@uint inBitCount3}
{Returns the transposed \emph{(z, y, x)} relation.}
{this reader considers that the receiver encodes an \emph{(x, y, z)} relation, where \emph{x} is defined by bits index \emph{0} to \emph{inBitCount1  - 1}, \emph{y} is defined by bits index \emph{inBitCount1} to \emph{inBitCount1 + inBitCount2 - 1} and  \emph{z} is defined by bits index \emph{inBitCount1 + inBitCount2} to \emph{inBitCount1 + inBitCount2 + inBitCount3 - 1}.}








\readerUnArgument{transitiveClosure}
{@binaryset}
{1.6.0}
{@binaryset}
{@uint inBitCount}
{Returns the transitive closure of the relation encoded by the receiver.}
{this reader considers that the receiver encodes an \emph{(x, y)} relation, where \emph{x} is defined by bits index \emph{0} to \emph{inBitCount  - 1}, \emph{y} is defined by bits index \emph{inBitCount} to \emph{2 * inBitCount - 1}.}








\readerUnArgument{uint64ValueList}
{@binaryset}
{1.6.0}
{@uint64list}
{@uint64 inBitCount}
{Returns the list of \nomType{@uint64} values corresponding to receiver's value, considering it uses \emph{inBitCount} bits.}
{}








\readerUnArgument{valueCount}
{@binaryset}
{1.6.0}
{@uint64}
{@uint inBitCount}
{Returns in an \nomType{@uint64} object the number of different values encoded by receiver, considering it uses \emph{inBitCount} bits.}
{no overflow test in performed.}







%-------------------------------

\subsection{Logical Operators}

The \nomType{@binaryset} type supports the three logical operators:\newline

\begin{tabular}{|c|c|}
\hline
\texttt{$\&$} & Logical And, intersection \\
\hline
\texttt{\textbar} & Logical Or, union \\
\hline
\texttt{$\wedge$}  & Exclusive or \\
\hline
\end{tabular}

Theses operators require both arguments to be \nomType{@binaryset} objects and return an \nomType{@binaryset} object.\newline


The \nomType{@binaryset} type supports the logical unary operator:\newline

\begin{tabular}{|c|c|}
\hline
$\sim$ & Negation, Complementation \\
\hline
\end{tabular}

This operator returns an \nomType{@binaryset} object.







\subsection{Comparison Operators}

The \nomType{@binaryset} type supports the two comparison operators:\newline

\begin{tabular}{|c|c|}
\hline
$=$ & Equality \\
\hline
$!=$ & Non Equality \\
\hline
\end{tabular}

Theses operators require both arguments to be \nomType{@binaryset} objects, and return a \nomType{@bool} object. These operations are very fast and are performed in a constant time (integer equality comparison).

Do not confuse with \lienReader{@binaryset}{equalTo} and \lienReader{@binaryset}{notEqualTo} that return a  \nomType{@binaryset} object.







\subsection{Shift Operators}

The \nomType{@binaryset} type supports the two shift operators:\newline

\begin{tabular}{|c|c|}
\hline
$<<$ & Left Shift \\
\hline
$>>$ & Right Shift \\
\hline
\end{tabular}

\exempleUneLigne
{}
{@binaryset b [binarySetWithPredicateString !\textquotedbl 1010\textquotedbl] ;}
\texttt{log b ; \# displays: <@binaryset:~1010>}\\
\texttt{@binaryset bb := b $<<$ 3 ;}\\
\texttt{log bb ; \# displays: <@binaryset:~1010XXX>}\\

