%!TEX encoding = UTF-8 Unicode
%!TEX root = ../galgas-book.tex

%--------------------------------------------------------------
\chapterLabel{Filewrappers}{filewrapper}
%-------------------------------------------------------------

Un \emph{filewrapper} permet d'embarquer dans le code engendré une arborescence de fichiers. Comme on va le voir dans la section suivante, la déclaration d'un \emph{filewrapper} désigne un répertoire, qui va être exploré au moment de la compilation GALGAS de façon à embarquer dans le code engendré la copie de certains fichiers. Ces  fichiers peuvent être de trois sortes :
\begin{itemize}
  \item des fichiers \emph{texte} ; ils sont sélectionnés par leur extension : la déclaration d'un \emph{filewrapper} liste toutes les extensions des fichiers texte embarqués ;
  \item des fichiers \emph{binaires} ; de même, ils sont sélectionnés par leur extension, et la déclaration d'un \emph{filewrapper} liste toutes les extensions des fichiers binaires embarqués ;
  \item des \emph{templates}, qui sont sélectionnés par leur nom ; ils sont analysés lors de leur lecture.
\end{itemize}


L'exploration des fichiers embarqués peut s'effectuer soit de manière statique, soit dynamique à l'aide d'un objet de \refTypePredefini{filewrapper}.









\section{Déclararation d'un \texttt{filewrapper}}

Un \emph{filewrapper} peut être déclaré dans un composant \emph{syntax}, \emph{semantics} ou \emph{program}. Sa déclaration est la suivante :

\begin{galgascode}
filewrapper nom in "chemin" {
 "extension_texte", ...
}{
 "extension_binaire", ...
}{
 declaration_de_templates
}
\end{galgascode}

Où :
\begin{itemize}
  \item \galgas{nom} est le nom, interne à GALGAS, donné au \emph{filewrapper} ; ce nom doit être unique à chaque \emph{filewrapper} ;
  \item \galgas{"chemin"} est le chemin du répertoire qui va être exploré récursivement au moment de la compilation ; c'est soit un chemin absolu (il commence par un \galgas{/}), soit un chemin relatif, par rapport au répertoire qui contient le fichier source qui déclare le \emph{filewrapper}.
\end{itemize}

La déclaration est divisée en trois parties délimitées par des accolades \galgas{\{ ... \}} :
\begin{itemize}
  \item la première partie (\galgas{"extension_texte", ...}) liste les extensions des fichiers texte qui sont embarqués ; à la compilation GALGAS, le répertoire désigné est exploré récursivement, et les fichiers dont l'extension est l'une des extensions citées sont embarqués, ainsi que leurs chemins relatifs ;
  \item la deuxième partie (\galgas{"extension_binaire", ...}) liste les extensions des fichiers binaires qui sont embarqués ; de même, à la compilation GALGAS, le répertoire désigné est exploré récursivement, et les fichiers dont l'extension est l'une des extensions citées sont embarqués, ainsi que leurs chemins relatifs ;
  \item la troisième et dernière partie (\galgas{declaration_de_templates}) contient les déclarations de \emph{templates}.
\end{itemize}

Chacune de ces parties peut être vide si on ne veut pas embarquer de ficher ou ne définir aucun template.


