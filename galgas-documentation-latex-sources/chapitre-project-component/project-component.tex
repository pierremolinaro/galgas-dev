%!TEX encoding = UTF-8 Unicode
%!TEX root = ../galgas-book.tex

%--------------------------------------------------------------
\chapterLabel{Le composant \texttt{project}}{composantProjet}\index{Component!Project}
%-------------------------------------------------------------


Le composant \ggs!project! permet de paramétrer un projet GALGAS. Il doit être placé dans un fichier source particulier, d'extension « \texttt{.galgasProject} ». Sont déclarés dans un fichier projet :
\begin{itemize}
  \item la version du projet (dans l'en-tête : \refSectionPage{enTeteProjet}) ;
  \item le nom des exécutables engendrés (dans l'en-tête : \refSectionPage{enTeteProjet}) ;
  \item les cibles de compilation : \refSectionPage{ciblesCompilation}  ;
  \item les fichiers sources : \refSectionPage{defFichierSourceProjet}.
\end{itemize}

Voici un exemple de composant projet :

\begin{galgas}
project (1:2:3) -> "logo" {
#--- Targets
  %makefile-macosx
  %makefile-unix
  %makefile-x86linux32-on-macosx
  %makefile-x86linux64-on-macosx
  %makefile-win32-on-macosx
  %makefile-msys32-on-windows
  %Mavericks
  %applicationBundleBase : "fr.what"
  %codeblocks-windows
  %codeblocks-mac

#--- Source files
  "galgas-sources/logo-lexique.galgas"
  "galgas-sources/logo-options.galgas"
  "galgas-sources/logo-semantics.galgas"
  "galgas-sources/logo-syntax.galgas"
  "galgas-sources/logo-grammar.galgas"
  "galgas-sources/logo-cocoa.galgas"
  "galgas-sources/logo-program.galgas"
}
\end{galgas}







\sectionLabel{En-tête du fichier projet}{enTeteProjet}

L'en-tête d'un projet définit deux informations :
\begin{itemize}
  \item la version du projet ;
  \item le nom des exécutables engendrés.
\end{itemize}

\subsectionLabel{Version du projet}{versionProjet}

\begin{galgas}
project (1:2:3) -> "logo" {
  ...
}
\end{galgas}

La version du projet apparaît sous la forme d'un triplet qui suit le mot-clé \ggs!project! : \ggs!1:2:3! dans le code ci-dessus. C'est ce triplet (sous la forme \texttt{1.2.3}) qui apparaît lorsque l'on invoque l'option \texttt{-{}-version} sur l'utilitaire ligne de commande engendré.
Dans le code, cette information peut être obtenu par le \refConstructorPage{application}{projectVersionString}.


\subsection{Nom des exécutables engendrés}

\begin{galgas}
project (1:2:3) -> "logo" {
  ...
}
\end{galgas}

Le nom des exécutables engendrés est fixé par la chaîne de caractères qui apparaît dans l'en-tête : \texttt{logo} dans l'exemple ci-dessus. Les exécutables compilés en mode \emph{release} portent directement ce nom, ceux compilés en mode \emph{debug} portent ce nom augmenté du suffixe « \texttt{-debug} » : \texttt{logo-debug}.


\sectionLabel{Cibles de compilation}{ciblesCompilation}

GALGAS peut engendrer des cibles de compilation pour Mac, Linux et Windows. Les outils engendrés sont des \emph{utilitaires en ligne de commande}, sauf sur Mac où une application Cocoa peut être engendrée.

\subsection{Cibles pour Linux}

Deux choix sont possibles :
\begin{itemize}
\item \texttt{Code::Blocks} ;
\item compilation en ligne de commande.
\end{itemize}

\subsubsection{\texttt{Code::Blocks} pour Linux}

L'option \ggs!%codeblocks-linux32! engendre une cible qui peut être compilée sur Linux 32 bits, et \ggs!%codeblocks-linux64! une cible compilable sur Linux 64 bits, en utilisant \texttt{Code::Blocks}\footnote{\url{http://www.codeblocks.org}}.
\begin{galgas}
project (0:0:1) -> "logo" {
  %codeblocks-linux32
  ...
}
\end{galgas}

\subsubsection{Compilation en ligne de commande pour Linux}

La déclaration \ggs!%makefile-unix! engendre une cible qui peut être compilée indifféremment sur Linux ou sur Mac. L'exécutable engendré est un exécutable 32 bits sur un Linux 32 bits, et un 64 bits sur un Linux 64 bits.
\begin{galgas}
project (0:0:1) -> "logo" {
  %makefile-unix
  ...
}
\end{galgas}



\subsection{Cibles pour Mac}

Comme GALGAS est développé sur Mac, c'est pour cette plateforme que l'on trouve le plus grand nombre de cibles :
\begin{itemize}
  \item application Cocoa ;
  \item compilation via Code::Blocks ;
  \item compilation en ligne de commande ;
  \item cross-compilation pour Win32 ;
  \item cross-compilation pour Linux32 ;
  \item cross-compilation pour Linux64.
\end{itemize}

\subsubsection{Application Cocoa}

Cette cible est l'objet du \refChapterPage{appliCocoa}.

\subsubsection{Compilation en ligne de commande pour Mac}

La déclaration \ggs!%makefile-macosx! engendre une cible pour obtenir un exécutable en ligne de commande sur Mac. Note : on peut aussi utiliser \ggs!%makefile-unix!.
\begin{galgas}
project (0:0:1) -> "logo" {
  %makefile-macosx
  ...
}
\end{galgas}

\subsubsection{\texttt{Code::Blocks} pour Mac}

L'option \ggs!%codeblocks-mac! engendre une cible qui peut être compilée sur Mac en utilisant \texttt{Code::Blocks}\footnote{\url{http://www.codeblocks.org}}.
\begin{galgas}
project (0:0:1) -> "logo" {
  %codeblocks-linux32
  ...
}
\end{galgas}

\subsubsection{Cross-compilation en ligne de commande pour Win32}

La déclaration \ggs!%makefile-win32-on-macosx! engendre une cible pour obtenir sur Mac un exécutable en ligne de commande pour Win32. À la première cross-compilation, le cross-compilateur est téléchargé à partir du site \texttt{rts-software} et placé dans \texttt{$\sim$/galgas-tools-for-cross-compilation}.

\begin{galgas}
project (0:0:1) -> "logo" {
  %makefile-win32-on-macosx
  ...
}
\end{galgas}


\subsubsection{Cross-compilation en ligne de commande pour Linux32}

La déclaration \ggs!%makefile-x86linux32-on-macosx! engendre une cible pour obtenir sur Mac un exécutable en ligne de commande pour Linux 32 bits sur x86. À la première cross-compilation, le cross-compilateur est téléchargé à partir du site \texttt{rts-software} et placé dans \texttt{$\sim$/galgas-tools-for-cross-compilation}.

\begin{galgas}
project (0:0:1) -> "logo" {
  %makefile-x86linux32-on-macosx
  ...
}
\end{galgas}


\subsubsection{Cross-compilation en ligne de commande pour Linux64}

La déclaration \ggs!%makefile-x86linux64-on-macosx! engendre une cible pour obtenir sur Mac un exécutable en ligne de commande pour Linux 64 bits sur x86. À la première cross-compilation, le cross-compilateur est téléchargé à partir du site \texttt{rts-software} et placé dans \texttt{$\sim$/galgas-tools-for-cross-compilation}.

\begin{galgas}
project (0:0:1) -> "logo" {
  %makefile-x86linux64-on-macosx
  ...
}
\end{galgas}



\subsection{Cible pour Windows : \texttt{CodeBlocks}}

Sur Windows, la compilation C++ du projet engendré s'effectue avec \texttt{Code::Blocks}\footnote{\url{http://www.codeblocks.org}}. La cible est engendrée par la déclaration \ggs!%codeblocks-windows!.

\begin{galgas}
project (0:0:1) -> "logo" {
  %codeblocks-windows
  ...
}
\end{galgas}





\sectionLabel{Déclaration \texttt{\%quietOutputByDefault}}{projetDeclarationQuietOutputByDefault}\index{\%quietOutputByDefault}

À partir de la version 3.1.4, GALGAS et les exécutables engendrés par GALGAS sont verbeux par défaut, c'est-à-dire que leur exécution affiche sur le terminal de nombreuses informations sur le déroulement de l'exécution, comme par exemple la mise à jour ou la création de fichiers. L'option de la ligne de commande \emph{quiet} (\refSectionPage{optionsQuietVerbose}) permet d'inhiber l'émission de ces messages.

On peut inverser ce comportement en faisant figurer \ggs!%quietOutputByDefault! parmi les déclarations du fichier projet :
\begin{galgas}
project (0:0:1) -> "logo" {
  ...
  %quietOutputByDefault
  ...
}
\end{galgas}

 Dans ce cas, l'exécutable engendré par GALGAS est silencieux par défaut, et bavard grâce à l'option de la ligne de commande \emph{verbose} (\refSectionPage{optionsQuietVerbose}).

En résumé :
\begin{itemize}
\item par défaut, sans l'option \ggs!%quietOutputByDefault! parmi les déclarations du fichier projet, l'exécutable est bavard par défaut, et l'option de la ligne de commande \emph{quiet} permet de le rendre silencieux ; l'option de la ligne de commande \emph{verbose} n'existe pas ;
\item si l'option \ggs!%quietOutputByDefault! est présente parmi les déclarations du fichier projet, l'exécutable est silencieux par défaut, et l'option de la ligne de commande \emph{verbose} permet de le rendre bavard ; l'option de la ligne de commande \emph{quiet} n'existe pas.
\end{itemize}

Une conséquence est que ni la présence de l'option \emph{quiet} ni la présence de l'option \emph{verbose} ne peuvent être testées par la construction \ggs+[option nom_composant_option.nom_option nom_info]+ (voir \refSubsectionPage{appelOption}). Il faut utiliser le \refConstructorPage{application}{verboseOutput}.







\sectionLabel{Déclaration des fichiers sources du projet}{defFichierSourceProjet}

Deux types de fichiers sources peuvent être déclarés :
\begin{itemize}
  \item des fichiers sources GALGAS ;
  \item des fichiers sources C++.
\end{itemize}

Un fichier source est déclaré sous la forme d'une chaîne de caractères qui définit son chemin :
\begin{itemize}
\item le chemin est absolu si il commence par un « / » ;
\item sinon il est relatif au répertoire qui contient le fichier projet ;
\item l'extension du chemin définit le type : « \texttt{.galgas} » pour un source GALGAS, « \texttt{.cpp} » pour un source C++.
\end{itemize}

Les sources GALGAS déclarés sont inclus dans la compilation GALGAS. L'ordre dans lequel apparaissent ces fichiers n'a pas d'importance sémantique, il définit simplement l'ordre dans lesquels les analyses lexicale et syntaxique sont effectuées.

Les sources C++ déclarés sont ignorés par la compilation GALGAS, et sont simplement ajoutés à la liste des foichiers C++ à compiler.



