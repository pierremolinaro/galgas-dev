%!TEX encoding = UTF-8 Unicode
%!TEX root = ../galgas-book.tex

%--------------------------------------------------------------
\chapterLabel{Fonctions et procédures}{fonctions}
%-------------------------------------------------------------

GALGAS définit les sous-programmes suivants :
\begin{itemize}
  \item les \emph{fonctions} (dans ce chapitre, \refSectionPage{declarationFonction}) ;
  \item les \emph{procédures} (dans ce chapitre, \refSectionPage{declarationProcedure}) ;
  \item les \emph{méthodes} (\refSectionPage{categoryMethod}) ;
  \item les \emph{getters} (\refSectionPage{categoryReader}) ;
  \item les \emph{setters} (\refSectionPage{categoryModifier}).
\end{itemize}

\sectionLabel{Fonction}{declarationFonction}

Une fonction GALGAS n'accepte que des arguments en entrée, et retourne une valeur. Elle est appelée dans une expression (\refSubsectionPage{appelFonction}).

\subsection{déclaration d'une fonction}\index{Fonction!Déclaration}

\lstset{emph={nom_fonction, liste_arguments_entree, var_resultat, liste_instructions}, emphstyle=\galgasEmphStyle}
\begin{galgascode}
internal # Optionnel
func nom_fonction liste_arguments_entree -> @T var_resultat {
  liste_instructions
}
\end{galgascode}

Une fonction est désignée par \galgas{nom_fonction}. Ce nom est unique dans un projet GALGAS. La liste des paramètres d'entrée peut être vide (\refSectionPage{listeParametresEffectifsEntree}). La valeur renvoyée par l'exécution de la fonction est la valeur de \galgas{var_resultat} à l'issue de l'exécution de la \galgas{liste_instructions}. Aussi, l'exécution de la \galgas{liste_instructions} doit valuer \galgas{var_resultat}.

Exemple :

\begin{galgascode}
func produit ?@uint a ?@uint b -> @uint resultat {
  resultat = a * b
}
\end{galgascode}



\subsection{Fonction interne à un fichier}

En préfixant la déclaration d'une fonction par \galgas{internal}, on limite son appel aux expressions situées dans le même fichier que la déclaration.




\subsection{Fonction \texttt{\%once}}

Une fonction sans argument accepte le qualificatif \galgast{\%once} :

\begin{galgascode}
func %once masque-> @uint resultat {
  resultat = 1 << 16
}
\end{galgascode}

Le qualificatif \galgast{\%once} organise le cache du résultat : celui-ci est calculé lors du premier appel, est mémorisé internement, et est retourné directement lors des appels ultérieurs.

Une fonction \galgast{\%once} peut être déclarée interne en la préfixant par \galgas{internal}.

\begin{galgascode}
internal func %once masque-> @uint resultat {
  resultat = 1 << 16
}
\end{galgascode}





\sectionLabel{Procédure}{declarationProcedure}\index{Procédure!Déclaration}

Une procédure GALGAS accepte que des arguments en entrée, en sortie, en entrée/sortie. Elle est appelée dans une instruction (\refSectionPage{instructionAppelProcedure}).

\subsection{déclaration d'une procédure}\index{Procédure!Déclaration}

\lstset{emph={nom_procedure, liste_arguments, liste_instructions}, emphstyle=\galgasEmphStyle}
\begin{galgascode}
internal # Optionnel
proc nom_procedure liste_arguments {
  liste_instructions
}
\end{galgascode}

Une procédure est désignée par \galgas{nom_procedure}. Ce nom est unique dans un projet GALGAS. La liste des paramètres en entrée, en sortie ou en entrée/sortie est décrite à la \refSectionPage{listeArgumentsFormels}).

Exemple :

\begin{galgascode}
proc produit ?@uint a ?@uint b !@uint resultat {
  resultat = a * b
}
\end{galgascode}



\subsection{Procédure interne à un fichier}

En préfixant la déclaration d'une procédure par \galgas{internal}, on limite son appel aux instruction situées dans le même fichier que la déclaration.



