%!TEX encoding = UTF-8 Unicode
%!TEX root = ../galgas-book.tex

%--------------------------------------------------------------
\chapterLabel{Fonctions et procédures}{fonctions}
%-------------------------------------------------------------

GALGAS définit les sous-programmes suivants :
\begin{itemize}
  \item les \emph{fonctions} (dans ce chapitre, \refSectionPage{declarationFonction}) ;
  \item les \emph{procédures} (dans ce chapitre, \refSectionPage{declarationProcedure}) ;
  \item les \emph{méthodes} (\refSectionPage{categoryMethod}) ;
  \item les \emph{getters} (\refSectionPage{categoryGetter}) ;
  \item les \emph{setters} (\refSectionPage{categorySetter}).
\end{itemize}

\sectionLabel{Fonction}{declarationFonction}

Une fonction GALGAS n'accepte que des arguments en entrée, et retourne une valeur. Elle est appelée dans une expression (\refSubsectionPage{appelFonction}).

\subsection{déclaration d'une fonction}\index{Fonction!Déclaration}

\begin{galgas}
private # Optionnel
func nom_fonction liste_arguments_entree -> @T var_resultat {
  liste_instructions
}
\end{galgas}

Une fonction est désignée par \ggs+nom_fonction+. Ce nom est unique dans un projet GALGAS. La liste des paramètres d'entrée peut être vide (\refSectionPage{listeParametresEffectifsEntree}). La valeur renvoyée par l'exécution de la fonction est la valeur de \ggs+var_resultat+ à l'issue de l'exécution de la \ggs+liste_instructions+. Aussi, l'exécution de la \ggs+liste_instructions+ doit valuer \ggs+var_resultat+.

Exemple :

\begin{galgas}
func produit ?@uint a ?@uint b -> @uint resultat {
  resultat = a * b
}
\end{galgas}

Mentionner la variable \ggs=resultat= est optionnel. Par défaut, en son absence, une variable nommée \ggs=result= est implictement déclarée :

\begin{galgas}
func produit ?@uint a ?@uint b -> @uint {
  result = a * b
}
\end{galgas}




\subsection{Fonction interne à un fichier}

En préfixant la déclaration d'une fonction par \ggs+private+, on limite son appel aux expressions situées dans le même fichier que la déclaration.




\subsection{Fonction \texttt{\%once}}

Une fonction sans argument accepte le qualificatif \ggs+%once+ :

\begin{galgas}
func %once masque -> @uint {
  result = 1 << 16
}
\end{galgas}

Le qualificatif \ggs+%once+ organise le cache du résultat : celui-ci est calculé lors du premier appel, est mémorisé internement, et est retourné directement lors des appels ultérieurs.

Une fonction \ggs+%once+ peut être déclarée interne en la préfixant par \ggs+private+.

\begin{galgas}
private func %once masque -> @uint {
  result = 1 << 16
}
\end{galgas}





\sectionLabel{Procédure}{declarationProcedure}\index{Procédure!Déclaration}

Une procédure GALGAS accepte que des arguments en entrée, en sortie, en entrée/sortie. Elle est appelée dans une instruction (\refSectionPage{instructionAppelProcedure}).

\subsection{Déclaration d'une procédure}\index{Procédure!Déclaration}

\begin{galgas}
private # Optionnel
proc nom_procedure liste_arguments {
  liste_instructions
}
\end{galgas}

Une procédure est désignée par \ggs+nom_procedure+. Ce nom est unique dans un projet GALGAS. La liste des paramètres en entrée, en sortie ou en entrée/sortie est décrite à la \refSectionPage{listeArgumentsFormels}).

Exemple :

\begin{galgas}
proc produit ?@uint a ?@uint b !@uint resultat {
  resultat = a * b
}
\end{galgas}



\subsection{Procédure interne à un fichier}

En préfixant la déclaration d'une procédure par \ggs+private+, on limite son appel aux instructions situées dans le même fichier que la déclaration.



