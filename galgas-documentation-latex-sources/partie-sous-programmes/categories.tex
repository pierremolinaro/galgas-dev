%!TEX encoding = UTF-8 Unicode
%!TEX root = ../galgas-book.tex

%--------------------------------------------------------------
\chapterLabel{Catégories}{categories}\index{Catégories}
%-------------------------------------------------------------

\emph{Categories} are the way for adding \emph{readers}, \emph{methods} and \emph{modifiers} to any type. They are defined outside type declarations.

You can declare for any type:
\begin{itemize}
\item \emph{category readers};
\item \emph{category methods};
\item \emph{category modifiers}.
\end{itemize}

Additional features are available for classes and are described in \refSectionPage{categoriesAndClasses}.


A \emph{category reader} is called in an expression. As expressions have no side-effect, a category reader cannot change current object's value.

A \emph{category method} is called by the \emph{method call instruction} (\refSectionPage{methodCallInstruction}). A category method cannot modify current object's value.

A \emph{category modifier} is called by the \emph{modifier call instruction} (\refSectionPage{modifierCallInstruction}). A category modifier can modify current object's value.

Within the category reader, method and modifier instruction list, the \lstinline[language=galgas]!selfcopy! key word is allowed in any expression. It represents a copy of the current object. Of course, the current is lazily copied only when required.

The \lstinline[language=galgas]!self! key word is just a syntactic tag for representing a write or a read/write access to the current object. Using \lstinline[language=galgas]!self! is not allowed in category methods and category readers since they cannot modify the current object. Using \lstinline[language=galgas]!self! in category modifiers is explained in \refSectionPage{categoryModifier}. 


A category reader, method and modifier can be declared in:
\begin{itemize}
\item a \emph{semantics} component;
\item a \emph{syntax} component;
\item a \emph{program} component.
\end{itemize}

A declared category reader, method and modifier has a global scope, meaning it is available in the current component, and in any component that includes it directly or indirectly.

A type does not accept several category readers with the same name. During compilation of the project file, the project global checking mechanism detects such declarations and issues an error. Consequently, it is forbidden to declare a category reader with the same name than a predefined reader: the compiler issues an error on on a such declaration. The same rules apply on category methods and category modifiers.

However, it is safe to declare for a given type a category reader, a category method and a category modifier with the same name. GALGAS compiler uses different naming spaces for them, and call syntax are different, so there is no ambiguity.










\sectionLabel{Category reader}{categoryReader}

A category reader is declared like a function, but its header names a type and a reader name. As a function, it accepts zero, one or more input and constant input formal parameters.

For example, the following code add a reader to the \refTypePredefini{uint64} that computes the square of its value:
\begin{lstlisting}[language=galgas]
reader @uint64 square -> @uint64 outResult :
  outResult := selfcopy * selfcopy ;
end reader ;
\end{lstlisting}

This reader is called like a predefined reader:
\begin{lstlisting}[language=galgas]
@uint64 v := 7L ;
log "Square of 7": [v square] ; # LOGGING Square of 7 : <@uint64:49>
\end{lstlisting}

You can add a category reader to a list :
\begin{lstlisting}[language=galgas]
reader @uintlist sum -> @uint outResult :
  outResult := 0 ;
  foreach selfcopy do
    outResult := outResult + mValue ;
  end foreach ;
end reader ;
\end{lstlisting}

For counting the number of element values greater than the value given in argument:
\begin{lstlisting}[language=galgas]
reader @uintlist countValuesGreaterThan
  ??@uint inTestValue -> @uint outResult
:
  outResult := 0 ;
  foreach selfcopy do
    if mValue > inTestValue then
      outResult ++ ;
    end if ;
  end foreach ;
end reader ;
\end{lstlisting}

When used with a struct or class type, current object attributes values can be read by naming the attribute in an expression. For example, the \refTypePredefini{lstring} has an attribute 
\lstinline[language=galgas]!string! whose type is \galgas{@string}. The following reader returns the value of this attribute, appended with the \lstinline[language=galgas]?" !"? string:
\begin{lstlisting}[language=galgas]
reader @lstring op -> @string outResult :
  outResult := string . " !" ;
end reader ;
\end{lstlisting}







\sectionLabel{Category method}{categoryMethod}

A category method is declared like a routine, but its header names a type and a method name. As a routine, it accepts zero, one or more input, output, input/output constant input formal parameters.

For example, the following code add a method to the \refTypePredefini{uint64} that computes the square of its value:
\begin{lstlisting}[language=galgas]
method @uint64 square !@uint64 outResult :
  outResult := selfcopy * selfcopy ;
end method ;
\end{lstlisting}

This reader is called like a predefined method:
\begin{lstlisting}[language=galgas]
@uint64 v ;
[7L square ?v] ;
log "Square of 7": v ; # LOGGING Square of 7 : <@uint64:49>
\end{lstlisting}

You can add a category method to a list :
\begin{lstlisting}[language=galgas]
method @uintlist sum !@uint outResult :
  outResult := 0 ;
  foreach selfcopy do
    outResult := outResult + mValue ;
  end foreach ;
end method ;
\end{lstlisting}

For counting the number of element values greater than the value given in argument:
\begin{lstlisting}[language=galgas]
method @uintlist countValuesGreaterThan
  ??@uint inTestValue
  !@uint outResult
:
  outResult := 0 ;
  foreach selfcopy do
    if mValue > inTestValue then
      outResult ++ ;
    end if ;
  end foreach ;
end method ;
\end{lstlisting}

When used with a struct or class type, current object attributes values can be read by naming the attribute in an expression. For example, the \refTypePredefini{lstring} has an attribute 
\lstinline[language=galgas]!string! whose type is \galgas{@string}. The following method returns the value of this attribute, appended with the \lstinline[language=galgas]?" !"? string:
\begin{lstlisting}[language=galgas]
method @lstring op !@string outResult :
  outResult := string . " !" ;
end method ;
\end{lstlisting}











\sectionLabel{Category modifier}{categoryModifier}

A category method is declared like a routine, but its header names a type and a modifier name. As a routine, it accepts zero, one or more output, input/output, input and constant input formal parameters. Unlike a category method, a category modifier can change the value of the current object.

For structure and classes types, attributes can be read, written, read / written. For example :
\begin{lstlisting}[language=galgas]
modifier @lstring appendInt ??@uint inValue :
  string .= [inValue string] ;
end modifier ;
\end{lstlisting}


The \lstinline[language=galgas]!self! key word is used as a syntactic tag for denoting a read/write or a write access on the current object. This key word is syntactically accepted in the following constructs:
\begin{enumerate}
\item the \emph{modifier call instruction} (\refSectionPage{modifierCallInstruction});
\item the \emph{append instruction} (\refSectionPage{appendInstruction});
\item the \emph{concat instruction} (\refSectionPage{concatInstruction});
\item the \emph{increment instruction} (\refSectionPage{incrementInstruction});
\item the \emph{decrement instruction} (\refSectionPage{decrementInstruction});
\item the \emph{assignment instruction} (\refSectionPage{assignmentInstruction}).
\end{enumerate}

Example of using \lstinline[language=galgas]!self! in modifier call instruction; this modifier prepends the square of argument value to the \galgas{@uint64list} value:
\begin{lstlisting}[language=galgas]
modifier @uint64list prependSquare ??@uint64 inValue :
  [!?self prependValue !inValue * inValue] ;
end modifier ;
\end{lstlisting}


Example of using \lstinline[language=galgas]!self! in the append instruction; this modifier appends the square of argument value to the \galgas{@uintlist} value:
\begin{lstlisting}[language=galgas]
modifier @uintlist appendSquare ??@uint inValue :
  self += !inValue * inValue ;
end modifier ;
\end{lstlisting}
This construct is valid only for types that handle the \lstinline[language=galgas]!+=! operator.


Example of using \lstinline[language=galgas]!self! in the concat instruction; this modifier appends to the string all items of the \galgas{@stringlist} argument value:
\begin{lstlisting}[language=galgas]
modifier @string concatList ??@stringlist inList :
  foreach inList do
    self .= mValue ;
  end foreach ;
end modifier ;
\end{lstlisting}
This construct is valid only for types that handle the \lstinline[language=galgas]!.=! operator.




Example of using \lstinline[language=galgas]!self! in the increment instruction; this modifier increments the receiver's value:
\begin{lstlisting}[language=galgas]
modifier @uint increment :
  self ++ ;
end modifier ;
\end{lstlisting}
This construct is valid only for types that handle the \lstinline[language=galgas]!++! operator, such as \refTypePredefini{uint}, \refTypePredefini{uint64}, \refTypePredefini{sint}, \refTypePredefini{sint64}.





Example of using \lstinline[language=galgas]!self! in the assignment instruction; this modifier removes all odd values of the receiver:
\begin{lstlisting}[language=galgas]
modifier @uintlist removeOddValues :
  @uintlist listWithEvenValues [emptyList] ;
  foreach selfcopy do
    if (mValue & 1) == 0 then
      listWithEvenValues += !mValue ;
    end if ;
  end foreach ;
  self := listWithEvenValues ;
end modifier ;
\end{lstlisting}
This construct is valid only for types, but class types.










\sectionLabel{Categories and classes}{categoriesAndClasses}


Additional features are available only for classes; in addition to category readers, methods and modifiers described in the above sections, you can declare:
\begin{itemize}
\item \emph{abstract} category readers, methods, modifiers;
\item \emph{overriding} category readers, methods, modifiers;
\item \emph{overriding abstract} category readers, methods, modifiers.
\end{itemize}

\emph{Abstract} category readers, methods, modifiers and \emph{overriding abstract} category readers, methods, modifiers do not contain any instruction list: they act as \emph{prototypes}.

Examples of \emph{abstract} category readers, methods, modifiers declarations :
{\lstset{emph={readerName, methodName, modifierName}, emphstyle=\emph}
\begin{lstlisting}[language=galgas]
abstract reader @aType readerName
  ?anOtherType aParameter
  -> @resultType outResult
;

abstract method @aType methodName
  ?anOtherType aParameter
;

abstract modifier @aType modifierName
  ?anOtherType aParameter
;
\end{lstlisting}
}

Examples of \emph{overriding} category readers, methods, modifiers declarations :
{\lstset{emph={instructions, readerName, methodName, modifierName}, emphstyle=\emph}
\begin{lstlisting}[language=galgas]
override reader @aType readerName
  ?anOtherType aParameter
  -> @resultType outResult
:
  instructions
;

override method @aType methodName
  ?anOtherType aParameter
:
  instructions
;

override modifier @aType modifierName
  ?anOtherType aParameter
:
  instructions
;
\end{lstlisting}
}

Examples of \emph{overriding abstract} category readers, methods, modifiers declarations :
{\lstset{emph={readerName, methodName, modifierName}, emphstyle=\emph}
\begin{lstlisting}[language=galgas]
override abstract reader @aType readerName
  ?anOtherType aParameter
  -> @resultType outResult
;

override abstract method @aType methodName
  ?anOtherType aParameter
;

override abstract modifier @aType modifierName
  ?anOtherType aParameter
;
\end{lstlisting}
}


Neither \emph{abstract} category readers, methods, modifiers, neither \emph{overriding abstract} category readers, methods, modifiers cannot be declared for concrete classes. Any kind of category reader, method, modifier can be declared for abstract classes.

If an \emph{abstract} category reader, method, modifier, or an \emph{overriding abstract} category reader, method, modifier is declared for an abstract class, it should be also declared as \emph{overriding} with the same name for every first concrete successor class.

A category reader, method, modifier that has the same name as a category reader, method, modifier declared for one of its super classes should be declared as \emph{overriding}.

An abstract category reader, method, modifier that has the same name as a category reader, method, modifier declared for one of its super classes should be declared as \emph{overriding abstract}.

The following example illustrates how theses rules should be applied. In the \refFigure{}{figureCategoryExample}, four classes are shown. An arrow links a class to its super class. The \lstinline[language=galgas]!@A! and \lstinline[language=galgas]!@C! classes are abstract. \lstinline[language=galgas]!m1! is a name for any reader, method or modifier.

\begin{figure}[ht]
  \centering
   \begin{tikzpicture}[every state/.style={draw, circle}, node distance=1.5cm, <-, thick]
     \node at (0, 4.5) [state, accepting] (A) {\lstinline[language=galgas]!@A!};
     \node at (0, 3) [state]            (B) {\lstinline[language=galgas]!@B!};
     \node at (0, 1.5) [state, accepting] (C) {\lstinline[language=galgas]!@C!};
     \node at (0, 0) [state]            (D) {\lstinline[language=galgas]!@D!};
     \path (A) edge (B) ;
     \path (B) edge (C) ;
     \path (C) edge (D) ;
     \node at (1.5, 4.5) [right] {\lstinline[language=galgas]!abstract @A m1!} ;
     \node at (1.5, 3) [right] {\lstinline[language=galgas]!override @B m1!} ;
     \node at (1.5, 1.5) [right] {\lstinline[language=galgas]!override abstract @C m1!} ;
     \node at (1.5, 0) [right] {\lstinline[language=galgas]!override @D m1!} ;
   \end{tikzpicture}
  \caption{inheritance graph and categories}
  \labelFigure{figureCategoryExample}
\end{figure}

\lstinline[language=galgas]!m1! is declared as \lstinline[language=galgas]!abstract! for the \lstinline[language=galgas]!@A! class. It is allowed since \lstinline[language=galgas]!@A! is abstract. Consequently, the concrete \lstinline[language=galgas]!@B! class should override \lstinline[language=galgas]!m1!. The \lstinline[language=galgas]!@C! class is also abstract, and \lstinline[language=galgas]!m1! can be declared as \lstinline[language=galgas]!abstract! for this class. But as it has been also declared for one of theses super class, it should also declared as \lstinline[language=galgas]!override!. As \lstinline[language=galgas]!@D! is concrete, \lstinline[language=galgas]!m1! should be declared for this class with \lstinline[language=galgas]!override! tag.










