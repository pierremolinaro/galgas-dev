%!TEX encoding = UTF-8 Unicode

\documentclass [a4paper, 10pt] {book}

%-----------------------------------------------------------------------------------------------------------------------*
%                                                                                                                       *
%   « N O I R    E T    B L A N C »    O U    « C O U L E U R »                                                         *
%                                                                                                                       *
%-----------------------------------------------------------------------------------------------------------------------*

%--- Par défaut, l'impression se fait en couleur
\providecommand{\sortieEnCouleur}{true}

%-----------------------------------------------------------------------------------------------------------------------*
%                                                                                                                       *
%   E N C O D A G E    D E S    S O U R C E S     :     L A T I N 1                                                     *
%                                                                                                                       *
%-----------------------------------------------------------------------------------------------------------------------*

%--- Paquetage pour le codage des sources en UTF-8
\usepackage[utf8]{inputenc}

%--- Latex demande ce paquetage pour mieux afficher le caractère "°" et \textquotesingle "'"
\usepackage{textcomp}

%--- Ce paquetage permet d'effectuer certaines césures, et ainsi d'éviter les messages "Overfull \hbox"
\usepackage[T1]{fontenc}

%-----------------------------------------------------------------------------------------------------------------------*
%                                                                                                                       *
%   M I S E    E N    P A G E S                                                                                         *
%                                                                                                                       *
%-----------------------------------------------------------------------------------------------------------------------*

% Voir "Une courte introduction à Latex2e", § 6.4

%--- Marge gauche : 2,8 cm ; le paramètre \hoffset contient cette valeur, moins 1 pouce
%    \hoffset = 2,8 cm - 2,54 cm = 0,26 cm
\setlength{\hoffset}{0.26 cm}

%--- Marges supplémentaires, différenciées pour les pages gauches et droites ; ici, aucune.
\setlength{\oddsidemargin }{0 cm}
\setlength{\evensidemargin}{0 cm}

%--- Largeur du texte
%    \textwidth = 210 mm - 28 mm - 28 mm = 15,4 cm
\setlength{\textwidth}{15.4 cm}

%--- Marge haute : 2,8 cm ; le paramètre \voffset contient cette valeur, moins 1 pouce
%    \voffset = 2,8 cm - 2,54 cm = 0,26 cm
\setlength{\voffset}{0.26 cm}

%--- Distance entre la marge haute et l'en-tête : 0 cm
\setlength{\topmargin}{0 cm}

%--- Hauteur de l'en-tête de chaque page : 1 cm
\setlength{\headheight}{1 cm}

%--- Distance entre l'en-tête de chaque page et le corps : 0,5 cm
\setlength{\headsep}{0.5 cm}

%--- Hauteur du corps
%    \textheight = 29,7 cm - 2,8 cm - 2,8 cm - 1,5 cm = 22,6 cm
\setlength{\textheight}{22.6 cm}

%-----------------------------------------------------------------------------------------------------------------------*
%                                                                                                                       *
%   C H O I X    D E    L A    P O L I C E                                                                              *
%                                                                                                                       *
%-----------------------------------------------------------------------------------------------------------------------*

% http://www.cuk.ch/articles/4237
% Un seul des choix suivants doit être validé ; si aucun, c'est la police par défaut qui est utilisée

%---------------------------------------------------- Pour utiliser la police "Times"
%\renewcommand{\rmdefault}{ptm}
%\usepackage{mathptmx}

%---------------------------------------------------- Pour utiliser la police "Palatino"
%\renewcommand{\rmdefault}{ppl}
%\usepackage{mathpazo}

%---------------------------------------------------- Pour utiliser la police "Bookman"
%\usepackage{bookman}

%---------------------------------------------------- Pour utiliser la police "Fourier"
\usepackage{fouriernc}
\usepackage[scaled=0.875]{helvet}
%\usepackage{courier}

%---------------------------------------------------- Pour utiliser la police "Beton, euler"
%\usepackage{beton, euler}

%---------------------------------------------------- Pour utiliser la police "Utopia - MathDesign"
%\usepackage[utopia]{mathdesign}

%---------------------------------------------------- Pour utiliser la police "Charter - MathDesign"
%\usepackage[charter]{mathdesign}

%-----------------------------------------------------------------------------------------------------------------------*
%                                                                                                                       *
%   G E S T I O N   D E    L A     C O U L E U R                                                                        *
%                                                                                                                       *
%-----------------------------------------------------------------------------------------------------------------------*

%--- Ce paquetage permet de définir des couleurs de la forme yellow!50 (jaune à 50 %)
\usepackage{xcolor}


%-----------------------------------------------------------------------------------------------------------------------*
%                                                                                                                       *
%   E X T E N S I O N S    P O U R    L ' É C R I T U R E    D E S     F O R M U L E S    M A T H É M A T I Q U E S     *
%                                                                                                                       *
%-----------------------------------------------------------------------------------------------------------------------*

%--- Extensions pour l'écriture des formules mathématiques
\usepackage{amsmath}
\usepackage{amssymb}
\usepackage{amsfonts}

%--- Paquetage "IEEEtrantools"
% Pour créer des tableaux d'équations, bien alignées
% Voir courte-intro-latex.pdf, page §3.5.2 page 83
\usepackage[retainorgcmds]{IEEEtrantools}


%-----------------------------------------------------------------------------------------------------------------------*
%                                                                                                                       *
%   P A Q U E T A G E    « I F T H E N »                                                                                *
%                                                                                                                       *
%-----------------------------------------------------------------------------------------------------------------------*

%--- Ce paquetage permet d'effectuer des tests : \ifthenelse{test}{bloc then}{bloc else}
\usepackage{ifthen}

%-----------------------------------------------------------------------------------------------------------------------*
%                                                                                                                       *
%   P A Q U E T A G E    « L I S T I N G S »                                                                            *
%                                                                                                                       *
%-----------------------------------------------------------------------------------------------------------------------*

\usepackage{listings}

\lstdefinelanguage{galgas}{
  keywordsprefix=@,
  morekeywords={abstract, after, before, between, block, cast, class, const, default, do, drop, else, elsif, end, enum,
    error, extends, extern, extract, false, feature, filewrapper, foreach, function, grammar, graph, gui, here, if,
    import, in, index, indexing, insert, label, lexique, list, listmap, local, log, loop, map, mapproxy, match,
    message, method, mod, modifier, nonterminal, not, on, once, operator, option, or, override, parse, pragma,
    prefixedby, project, program, reader, remove, replace, repeat, rewind, root, routine, rule, search, select,
    self, selfcopy, semantics, send, sortedlist, state, struct, style, super, switch, syntax, tag, template,
    then, true, uniquemap, unused, warning, when, while, with
  },
  comment=[l]\#,
  string=[b]", string=[d]",
  classoffset=1,
  morekeywords={;}, keywordstyle=\color{green}\textbf,
  classoffset=2
}

\ifthenelse{\equal{\sortieEnCouleur}{true}}{
  \lstset{
    language=galgas,
    basicstyle=\normalsize,
    frame=l,
    keywordstyle=\color{blue}\textbf,
    backgroundcolor=\color{yellow!50},
    identifierstyle=\color{brown},
    commentstyle=\color{red},
    stringstyle=\color{gray}
  }
}{
  \lstset{
    language=galgas,
    basicstyle=\normalsize,
    frame=l,
    keywordstyle=\textbf,
  }
}

%-----------------------------------------------------------------------------------------------------------------------*
%                                                                                                                       *
%   P A Q U E T A G E    « L O N G T A B L E »                                                                          *
%                                                                                                                       *
%-----------------------------------------------------------------------------------------------------------------------*

%--- Pour afficher correctement des tables sur plusieurs pages
\usepackage{longtable}

%-----------------------------------------------------------------------------------------------------------------------*
%                                                                                                                       *
%   E N - T Ê T E S    E T    P I E D S    D E    P A G E S                                                             *
%                                                                                                                       *
%-----------------------------------------------------------------------------------------------------------------------*

\usepackage{fancyhdr}
\pagestyle{fancy}


%---------------------------------------------------- Include my new commands
%---------------------------------------------------------------------*
%                                                                     *
% This file is included in galgas-book.tex file                       *
%                                                                     *
%---------------------------------------------------------------------*

%--- Mot clef
\newcommand \motCle[1] {\texttt{\textbf{#1}}}

%---------------------------------------------------------------------*

%--- Nom de type
\newcommand \nomType[1] {\texttt{#1}}

%---------------------------------------------------------------------*
% Section pour la d�finition des types

\newcommand \definitionSectionType[1] {\section{The \nomType{#1} Type}\label{#1}}

\newcommand \lienSectionType[1] {\hyperref[#1]{#1 type (page \pageref{#1})}}

%---------------------------------------------------------------------*

%--- D�finition d'un reader sans argument
% Exemple d'appel :
% \readerSansArgument{line} % Nom du reader
% {@location} % Nom du type
% {1.8.2} % Premi�re version GALGAS qui impl�mente ce reader
% {@uint} % Type renvoy�
% {Returns the line of the receiver's value.} % Description
% {this reader raises a run-time error if ...} % Discussion

\newcommand \readerSansArgument[6] {
  \subsection{\texttt{#1} Reader}\label{reader #2 #1}
  #5
  \newline
  \newline
  \begin{tabular}{|l}
  \texttt{\emph{reader} \nomType{#2} #1 -> #4 ;}
  \end{tabular}
  \newline
  \newline
  \textbf{Availability:} available in GALGAS #3 and later.
  \ifthenelse{\equal{#6}{}}{
  }{
    \newline
    \newline
    \textbf{Discussion:} #6
  }
  \newline
}

\newcommand \lienReader[2] {\hyperref[reader #1 #2]{\texttt{#2} reader (page \pageref{reader #1 #2})}}

%---------------------------------------------------------------------*

%--- D�finition d'un reader � 1 argument
% Exemple d'appel :
% \readerSansArgument{line} % Nom du reader
% {@location} % Nom du type
% {1.8.2} % Premi�re version GALGAS qui impl�mente ce reader
% {@uint} % Type renvoy�
% {Returns the line of the receiver's value.} % Description
% {this reader raises a run-time error if ...} % Discussion

\newcommand \readerUnArgument[7] {
  \subsection{\texttt{#1} Reader}\label{reader #2 #1}
  #6
  \newline
  \newline
  \begin{tabular}{|l}
  \texttt{\emph{reader} \nomType{#2} #1}\\
  \texttt{\ \ ?#5}\\
  \texttt{\ \ -> #4 ;}
  \end{tabular}
  \newline
  \newline
  \textbf{Availability:} available in GALGAS #3 and later.
  \ifthenelse{\equal{#7}{}}{
  }{
    \newline
    \newline
    \textbf{Discussion:} #7
  }
  \newline
}

%---------------------------------------------------------------------*

%--- D�finition d'un reader � 2 arguments
% Exemple d'appel :
% \readerSansArgument{line} % Nom du reader
% {@location} % Nom du type
% {1.8.2} % Premi�re version GALGAS qui impl�mente ce reader
% {@uint} % Type renvoy�
% {Returns the line of the receiver's value.} % Description
% {this reader raises a run-time error if ...} % Discussion

\newcommand \readerDeuxArguments[8] {
  \subsection{\texttt{#1} Reader}\label{reader #2 #1}
  #7
  \newline
  \newline
  \begin{tabular}{|l}
  \texttt{\emph{reader} \nomType{#2} #1}\\
  \texttt{\ \ ?#5}\\
  \texttt{\ \ ?#6}\\
  \texttt{\ \ -> #4 ;}
  \end{tabular}
  \newline
  \newline
  \textbf{Availability:} available in GALGAS #3 and later.
  \ifthenelse{\equal{#8}{}}{
  }{
    \newline
    \newline
    \textbf{Discussion:} #8
  }
  \newline
}

%---------------------------------------------------------------------*

%--- D�finition d'un constructeur sans argument
% Exemple d'appel :
% \constructeurSansArgument{nowhere} % Nom du constructeur
% {@location} % Nom du type
% {1.8.2} % Premi�re version GALGAS qui impl�mente ce constructeur
% {@uint} % Type renvoy�
%{Returns an \nomType{@location} that does not points out any location.} % Description
%{The returned object responds \motCle{true} to the isNowhere reader.} % Discussion

\newcommand \constructeurSansArgument[6] {
  \subsection{\texttt{#1} Constructor}\label{constructor #2 #1}
  #5
  \newline
  \newline
  \begin{tabular}{|l}
  \texttt{\emph{constructor} \nomType{#2} #1 -> #4 ;}
  \end{tabular}
  \newline
  \newline
  \textbf{Availability:} available in GALGAS #3 and later.
  \ifthenelse{\equal{#6}{}}{
  }{
    \newline
    \newline
    \textbf{Discussion:} #6
  }
  \newline
}

%---------------------------------------------------------------------*

%--- D�finition d'un constructeur avec 1 argument

\newcommand \constructeurUnArgument[7] {
  \subsection{\texttt{#1} Constructor}\label{constructor #2 #1}
  #6
  \newline
  \newline
  \begin{tabular}{|l}
  \texttt{\emph{constructor} \nomType{#2} #1}\\
  \texttt{\ \ ?#5}\\
  \texttt{\ \ -> #4 ;}
  \end{tabular}
  \newline
  \newline
  \textbf{Availability:} available in GALGAS #3 and later.
  \ifthenelse{\equal{#7}{}}{
  }{
    \newline
    \newline
    \textbf{Discussion:} #7
  }
  \newline
}

%---------------------------------------------------------------------*

%--- D�finition d'un constructeur avec 2 arguments

\newcommand \constructeurDeuxArguments[8] {
  \subsection{\texttt{#1} Constructor}\label{constructor #2 #1}
  #7
  \newline
  \newline
  \begin{tabular}{|l}
  \texttt{\emph{constructor} \nomType{#2} #1}\\
  \texttt{\ \ ?#5}\\
  \texttt{\ \ ?#6}\\
  \texttt{\ \ -> #4 ;}
  \end{tabular}
  \newline
  \newline
  \textbf{Availability:} available in GALGAS #3 and later.
  \ifthenelse{\equal{#8}{}}{
  }{
    \newline
    \newline
    \textbf{Discussion:} #8
  }
  \newline
}

%---------------------------------------------------------------------*

%--- D�finition d'un constructeur avec 3 arguments

\newcommand \constructeurTroisArguments[9] {
  \subsection{\texttt{#1} Constructor}\label{constructor #2 #1}
  #8
  \newline
  \newline
  \begin{tabular}{|l}
  \texttt{\emph{constructor} \nomType{#2} #1}\\
  \texttt{\ \ ?#5}\\
  \texttt{\ \ ?#6}\\
  \texttt{\ \ ?#7}\\
  \texttt{\ \ -> #4 ;}
  \end{tabular}
  \newline
  \newline
  \textbf{Availability:} available in GALGAS #3 and later.
  \ifthenelse{\equal{#9}{}}{
  }{
    \newline
    \newline
    \textbf{Discussion:} #9
  }
  \newline
}

%---------------------------------------------------------------------*

%--- D�finition d'un modifier � 1 argument

\newcommand \modifierUnArgument[6] {
  \subsection{\texttt{#1} Modifier}\label{modifier #2 #1}
  #5
  \newline
  \newline
  \begin{tabular}{|l}
  \texttt{\emph{modifier} \nomType{#2} #1}\\
  \texttt{\ \ ?#4}\\
  \end{tabular}
  \newline
  \newline
  \textbf{Availability:} available in GALGAS #3 and later.
  \ifthenelse{\equal{#6}{}}{
  }{
    \newline
    \newline
    \textbf{Discussion:} #6
  }
  \newline
}

%---------------------------------------------------------------------*

%--- Exemple un ligne

\newcommand \exempleUneLigne[2] {
  \noindent
  \ifthenelse{\equal{#1}{}}{
    \textbf{Example:}\newline
  }{
    \textbf{Example.} #1\newline
  }
  \texttt{#2}\newline
}

%---------------------------------------------------------------------*

%--- Exemple 2 lignes

\newcommand \exempleDeuxLignes[3] {
  \noindent
  \ifthenelse{\equal{#1}{}}{
    \textbf{Example:}\newline
  }{
    \textbf{Example.} #1\newline
  }
  \texttt{#2}\newline
  \texttt{#3}\newline
}

%---------------------------------------------------------------------*

%--- Exemple 3 lignes

\newcommand \exempleTroisLignes[4] {
  \noindent
  \ifthenelse{\equal{#1}{}}{
    \textbf{Example:}\newline
  }{
    \textbf{Example.} #1\newline
  }
  \texttt{#2}\newline
  \texttt{#3}\newline
  \texttt{#4}\newline
}

%---------------------------------------------------------------------*



%-----------------------------------------------------------------------------------------------------------------------*
%                                                                                                                       *
%   G E S T I O N    D E    L ' I N D E X                                                                               *
%                                                                                                                       *
%-----------------------------------------------------------------------------------------------------------------------*

% http://www.cuk.ch/articles/4097
% http://www.tuteurs.ens.fr/logiciels/latex/makeindex.html
% http://linux.die.net/man/1/makeindex
%
% Attention ! Les deux commandes suivantes, ainsi que le \printindex placé plus bas ne
% sont pas suffisants pour construire l'index : il faut utiliser l'utilitaire "makeIndex"
% Voir le fichier de commande "build.command"
\usepackage{makeidx}
\makeindex

%-----------------------------------------------------------------------------------------------------------------------*
%                                                                                                                       *
%   T O C B I D I N D                                                                                                   *
%                                                                                                                       *
%-----------------------------------------------------------------------------------------------------------------------*

%    Pour faire figurer la liste des tableaux (et la table des matières)
%    dans la table des matières
\usepackage{tocbibind}

\setcounter{tocdepth}{3}

%-----------------------------------------------------------------------------------------------------------------------*
%                                                                                                                       *
%   H Y P E R R E F                                                                                                     *
%                                                                                                                       *
%-----------------------------------------------------------------------------------------------------------------------*

%--- Pour les hyperliens, et le contrôle de la génération PDF 
\usepackage{hyperref}
\hypersetup{colorlinks=true}
\hypersetup{linkcolor=blue}

\hypersetup{breaklinks=true}
%\hypersetup{verbose=true}

%-----------------------------------------------------------------------------------------------------------------------*
%                                                                                                                       *
%   S H O W K E Y S    ( P O U R    D É B O G U E R )                                                                   *
%                                                                                                                       *
%-----------------------------------------------------------------------------------------------------------------------*

%\usepackage{showkeys}

%-----------------------------------------------------------------------------------------------------------------------*
%                                                                                                                       *
%   T I T L E T O C                                                                                                     *
%                                                                                                                       *
%-----------------------------------------------------------------------------------------------------------------------*

%--- Description dans le paquetage titlesec
% http://forum.mathematex.net/latex-f6/formatage-avance-de-la-table-des-matieres-t11559.html

\usepackage{titletoc}


%--- Par défaut dans la tables des matières, le numéro de sous-section est trop long et mange le début du titre
%\titlecontents{subsection}[2.5cm]{}{\hspace*{-5.0em}\hyperref[subsection \thecontentslabel]{\thecontentslabel}\hspace*{0.5em}}{}{\titlerule*[0.66pc]{.}\contentspage}{}

\titlecontents{subsection}[2.5cm]{}{\hspace*{-5.0em}\thecontentslabel\hspace*{0.5em}}{}{\titlerule*[0.66pc]{.}\contentspage}{}

%\titlecontents{subsection}
%  [2cm]% retrait gauche
%  {}% pour les entrées numérotées et non numérotées
%  {\hspace*{-3.0em}\makebox[2.5em]{\hspace*{0pt plus 1 fill minus 1fill}\thecontentslabel.}\hspace*{0.5em}}% pour les entrées numérotées uniquement
%  {}% pour les entrées non numérotées uniquement
%  {\titlerule*[0.66em]{.}\contentspage}% numéro de page

%-----------------------------------------------------------------------------------------------------------------------*
%                                                                                                                       *
%   D P R O G R E S S                                                                                                   *
%                                                                                                                       *
%-----------------------------------------------------------------------------------------------------------------------*

%--- Affiche les sections dans le log
\usepackage{dprogress}

%-----------------------------------------------------------------------------------------------------------------------*
%                                                                                                                       *
%   D É B U T    D U    D O C U M E N T                                                                                 *
%                                                                                                                       *
%-----------------------------------------------------------------------------------------------------------------------*


\begin{document} 

%-----------------------------------------------------------------------------------------------------------------------*
%                                                                                                                       *
%   P A G E    D E    T I T R E                                                                                         *
%                                                                                                                       *
%-----------------------------------------------------------------------------------------------------------------------*

\title{\Huge{\textbf{GALGAS Book}}\\~\\ \normalsize{For release GALGAS-CURRENT-VERSION}}
\author{Pierre Molinaro}
\date \today 

\maketitle

%-----------------------------------------------------------------------------------------------------------------------*
%                                                                                                                       *
%   T A B L E    D E S    M A T I È R E S                                                                               *
%                                                                                                                       *
%-----------------------------------------------------------------------------------------------------------------------*

\tableofcontents
 
%-----------------------------------------------------------------------------------------------------------------------*
%                                                                                                                       *
%   L I S T E    D E S    T A B L E A U X                                                                               *
%                                                                                                                       *
%-----------------------------------------------------------------------------------------------------------------------*

\listoftables
\addtocontents{lot}{\protect\thispagestyle{empty}\protect\pagestyle{empty}}

%-----------------------------------------------------------------------------------------------------------------------*
%                                                                                                                       *
%   L I S T E    D E S    F I G U R E S                                                                                 *
%                                                                                                                       *
%-----------------------------------------------------------------------------------------------------------------------*

\listoffigures
\addtocontents{lof}{\protect\thispagestyle{empty}\protect\pagestyle{empty}}

%-----------------------------------------------------------------------------------------------------------------------*
%                                                                                                                       *
%   L E S    C H A P I T R E S                                                                                          *
%                                                                                                                       *
%-----------------------------------------------------------------------------------------------------------------------*

%!TEX encoding = UTF-8 Unicode
%!TEX root = ../galgas-book.tex

%--------------------------------------------------------------
\chapter{Getting and installing GALGAS}
%-------------------------------------------------------------


%!TEX encoding = UTF-8 Unicode
%!TEX root = ../galgas-book.tex

%--------------------------------------------------------------
\chapter{Options de la ligne de commande}\index{Options de la ligne de commande}
%-------------------------------------------------------------

GALGAS accepte un certain nombre d’options, qui sont détaillées dans les pages suivantes.

L’analyse des arguments de la ligne de commande est simple :
\begin{itemize}
  \item tout argument qui commence par un « - » est une option ;
  \item tout argument qui ne commence pas par un « - » est considéré comme un fichier source GALGAS ;
  \item les extensions acceptables par le compilateur GALGAS sont :
  \begin{itemize}
    \item « \texttt{.galgas} », un fichier source ;
    \item « \texttt{.galgasProject} », un fichier de description de projet ;
    \item « \texttt{.galgasTemplate} », un fichier de description de template.
  \end{itemize}
\end{itemize}

L’ordre des options et des fichiers sources est quelconque. La ligne de commande est complètement analysée avant le traitement des fichiers sources. Si plusieurs fichiers sources apparaissent dans la ligne de commande, ils sont traités dans leur ordre d’apparition.

{\bf Note pour Windows.} L’outil GALGAS pour Windows propose par défaut un dialogue invitant à entrer les références d’un fichier source si la ligne ne contient aucun fichier source (c’est le cas quand on double-clique sur l’icône de l’application). L'option \optionGGS{-{-}no-dialog}, spécifique à cette plate forme, permet d'inhiber l’apparition du dialogue.

\sectionLabel{Options générales}{optionGenerales}

\optionGGS{-{-}help} Affiche la liste des options.

\optionGGS{-{-}version} Affiche le numéro de version.

\optionGGS{-{-}no-color} Les messages émis sur le terminal sont en texte pur, sans coloration.

\optionGGS{-{-}no-dialog} (\emph{uniquement sur Windows}) L’outil GALGAS pour Windows propose par défaut un dialogue invitant à entrer la référence d’un fichier source si la ligne ne contient aucun fichier source (c’est le cas quand on double-clique sur l’icône de l’application). Cette option permet d'inhiber l’apparition du dialogue.


\sectionLabel{Options \emph{quiet} et \emph{verbose}}{optionsQuietVerbose}


\optionGGS{-v}, \optionGGS{-{-}verbose} Affiche des messages complémentaires sur le terminal. Par défaut, quand toutes les étapes se déroulent correctement, aucun message n’est affiché.

\optionGGS{-q}, \optionGGS{-{-}quiet} N'affiche aucun message complémentaire sur le terminal. Par défaut, des messages complémentaires sur le terminal sont affichés.

Ces deux options s'excluent, c'est-à-dire qu'un exécutable définit soit l'option \emph{quiet}, soit l'option \emph{verbose}, mais par les deux :
\begin{itemize}
  \item le compilateur GALGAS implémente l'option \emph{quiet}, mais pas l'option \emph{verbose} ;
  \item par défaut, un compilateur engendré par GALGAS implémente l'option \emph{quiet}, mais pas l'option \emph{verbose} ;
  \item Si la déclaration \ggs!%quietOutputByDefault!\index{\%quietOutputByDefault} parmi les déclarations du fichier projet (voir \refSectionPage{projetDeclarationQuietOutputByDefault}), le compilateur engendré par GALGAS implémente l'option \emph{verbose}, mais pas l'option \emph{quiet}.
\end{itemize}





\section{Option de création d'un projet}


\optionGGS{-{-}create-project=$nom$} Crée un nouveau projet GALGAS nommé $nom$ dans le répertoire courant.




\section{Options contrôlant le compilateur}





\optionGGS{-W}, \optionGGS{-{-}Werror} Tout \emph{warning} est considéré comme une erreur. Cela peut être important dans un script, l’outil de commande renvoyant un code non nul si une ou plusieurs erreurs ont été détectées.

\optionGGS{-{-}max-errors=$n$} Stoppe après $n$ erreurs.

\optionGGS{-{-}max-warnings=$n$} Stoppe après $n$ alertes.

\optionGGS{-{-}check-usefulness} Calcul de l'utilité des constructions. L'utilisation de cette option est décrite dans le \refChapterPage{chapitreCalculEntitesUtiles}.


\sectionLabel{Options contrôlant la génération de fichiers}{optionsGeneration}



\optionGGS{-{-}emit-issue-json-file=$fichier$} Écrit dans un $fichier$ au format JSON la liste des erreurs et des alertes.


\optionGGS{-{-}log-f{}ile-read} Affiche sur la console tout accès en lecture à un fichier.


\optionGGS{-{-}no-f{}ile-generation} Inhibe l'écriture de tout fichier.


\optionGGS{-{-}mode=$nom$} Contrôle l'opération du compilateur : si $nom$ est vide, le compilateur opère normalement. Si $nom$ est \texttt{lexical-only}, le compilateur affiche le résultat de l'analyse lexicale et s'arrête ; aucun fichier n'est engendré. Si $nom$ est \texttt{syntax-only}, le compilateur affiche le résultat de l'analyse syntaxique et s'arrête ; aucun fichier n'est engendré.





\optionGGS{-{-}compile=$nom$} Enchaîne une compilation C++ après une compilation GALGAS sans erreur. Le $nom$ est le nom d'une cible de type \emph{makefile} ; par exemple, \texttt{-{-}compile=makefile-macosx} enchaîne la compilation C++ de la cible \emph{makefile-macosx}.


\optionGGS{-{-}macosx=$n$} Force la génération d'un projet Xcode dont le SDK et le \emph{macos Deployement Target} sont fixés à 10.$n$.  Attention, cette option ne vous dispense pas de préciser \ggs=%applicationBundleBase= (\refSectionPage{parametrageProjetGALGAS}).




\section{Options de débogage du compilateur}

Ces options ne sont pas destinées à être utilisées lors de l'exploitation de GALGAS : elles permettent de déboguer le compilateur lui-même, et non pas le fichier source compilé.



\optionGGS{-{-}generate-many-cpp-files} Engendre le code C++ dans une multitude de fichiers. Ceci permet un débogage plus simple du compilateur GALGAS lui-même, mais ralentit ensuite l'étape de compilation C++.


\optionGGS{-{-}generate-one-cpp-header} Engendre un seul fichier d'en-tête C++ pour tout le projet. Ceci permet un débogage plus simple du compilateur GALGAS lui-même, mais ralentit ensuite l'étape de compilation C++.


\optionGGS{-{-}check-gmp} Exécute au démarrage une série de calculs afin de vérifier si la librairie GMP s'exécute correctement.





\section{Options de documentation}

Ces options produisent des fichiers qui facilitent la documention \LaTeX de votre compilateur.

\optionGGS{-{-}emit-syntax-diagrams} Cette option provoquent l'émission de fichiers \LaTeX qui contiennent les diagrammes syntaxiques des grammaires des projets compilés. Son utilisation est détaillée au \refChapterPage{chapitreDiagrammesSyntaxiques}. C'est ainsi que les diagrammes des langages GALGAS présentés au \refChapterPage{grammaireProjet}, \refChapterPage{grammaireSource} et \refChapterPage{grammaireTemplate} ont été obtenus.





\optionGGS{-{-}print-predefined-lexical-actions} Affiche sur la console la liste des routines lexicales prédéfinies.




\optionGGS{-{-}generate-shared-map-automaton-dot-files} Exporte les automates d'états finis associés à chaque table de symboles de type \ggs!shared map!. Les fichiers de sortie sont placés dans le répertoire \texttt{build/helpers}, et portent le nom du type table postfixé par l'extension \texttt{.dot}.





\optionGGS{-{-}output-concrete-syntax-tree} Exporte dans un fichier l'arbre syntaxique concret du code source analysé sous la forme d'un graphe dont le format est compatible avec \emph{Graphviz}. Le nom du fichier de sortie est le nom du fichier source doté de l'extension complémentaire \texttt{.dot}.


\optionGGS{-{-}output-keyword-list-file=$nomLexique$:$nomListe$:$colonnes$:$prefixe$:$postfixe$:$fichier$} Cette option permet d'engendrer un fichier au format contenant la liste des mots réservés de votre langage. L'argument qui suit le signe « \texttt{=} » est une séquence de six champs :
\begin{itemize}
  \item $nomLexique$ est le nom du lexique ;
  \item $nomListe$ est le nom de la liste ;
  \item $colonnes$ est un nombre entier naturel, qui représente le nombre de colonnes de la sortie ;
  \item $prefixe$ est une chaîne (éventuellement vide) qui est placée avant chaque élément de liste ;
  \item $postfixe$ est une chaîne (éventuellement vide) qui est placée après chaque élément de liste ;
  \item $fichier$ est une chaîne qui désigne le fichier de sortie.
\end{itemize}

Prenons un exemple ; supposons que le composant \ggs!lexique! de votre langage soit~:
\begin{galgas}
lexique lex {
  ...
  list mots ... { "a", "b", "c" }
  ...
}
\end{galgas}

En appelant votre compilateur avec l'option \optionGGS{-{-}output-keyword-list-file=lex:mots:2:::motsreserves.tex}, la liste des mots réservés définies par la liste \ggs!mots! du lexique \ggs!lex! sera écrite dans le fichier \texttt{motsreserves.tex}. Ce fichier aura le contenu suivant :
\begin{lstlisting}[backgroundcolor=\color{yellow!10}, frame=tlbr, basicstyle=\small\tt]
  a & b \\
  c & \\
\end{lstlisting}

C'est un fichier qui peut être inclus dans une définition de tableau à deux colonnes. Si le nombre d'éléments n'est pas un multiple du nombre de colonnes, la dernière ligne est complétée par des champs vides. Par exemple, on écrit en \LaTeX :
\begin{lstlisting}[backgroundcolor=\color{yellow!10}, frame=tlbr, basicstyle=\small\tt]
\begin{table}[!t]
  \centering
  \begin{tabular}{ll}
    \input{motsreserves.tex}
  \end{tabular}
\end{table}
\end{lstlisting}

On peut utiliser les champs $prefixe$ et $postfixe$ pour afficher de manière particulière chaque élément : avec l'option \optionGGS{-{-}output-keyword-list-file=lex:mots:2:\textbackslash texttt\{:\}:motsreserves.tex}, le fichier \texttt{motsreserves.tex} aura le contenu suivant :
\begin{lstlisting}[backgroundcolor=\color{yellow!10}, frame=tlbr, basicstyle=\small\tt]
  \texttt{a} & \texttt{b} \\
  \texttt{c} & \\
\end{lstlisting}




%!TEX encoding = UTF-8 Unicode
%!TEX root = ../galgas-book.tex

%--------------------------------------------------------------
\chapter{Élements lexicaux}
%-------------------------------------------------------------

Les éléments lexicaux du langage GALGAS sont~:
\begin{itemize}
  \item les identificateurs (\refSectionPage{identificateursGALGAS})~;
  \item les mots réservés (\refSectionPage{motReservesGALGAS})~;
  \item les délimiteurs (\refSectionPage{delimiteursGALGAS})~;
  \item les sélecteurs  (\refSectionPage{selecteursGALGAS})~;
  \item les séparateurs  (\refSectionPage{separateursGALGAS})~;
  \item les commentaires  (\refSectionPage{commentairesGALGAS})~;
  \item les non terminaux  (\refSectionPage{nonTerminauxGALGAS})~;
  \item les terminaux (\refSectionPage{terminauxGALGAS})~;
  \item les constantes littérales entières (\refSectionPage{constantesLitteralesEntiersGALGAS})~;
  \item les constantes littérales flottantes (\refSectionPage{constantesLitteralesFlottantesGALGAS})~;
  \item les caractères littéraux (\refSectionPage{constantesLitteralesCaracteresGALGAS})~;
  \item les constantes chaînes de caractères (\refSectionPage{constantesLitteralesChainesGALGAS})~;
  \item les noms de types (\refSectionPage{nomTypeGALGAS})~;
  \item les attributs (\refSectionPage{attributsGALGAS}).
\end{itemize}


\sectionLabel{Les identificateurs}{identificateursGALGAS}

Un identificateur commence par une lettre minuscule ou majuscule, suivie de zéro, un ou plusieurs chiffres décimaux, lettres minuscules ou majuscules ou caractères \ggs='_'=. Par exemple~:

\ggs=element=, \ggs=element0=, \ggs=element_0=, \ggs=instructionList=, \ggs=instruction_list=.

Toutes les lettres Unicode sont acceptées~: il est possible d'utiliser des lettres accentuées, des lettres grecques, ... Par exemple~:

\begin{galgas}
let constanteAccentuée = 12
let π = 3.14
let α = 1
var переменная = 7
\end{galgas}


\sectionLabel{Les mots réservés}{motReservesGALGAS}

Les mots réservés de GALGAS sont les identificateurs listés dans le \refTableauPage{mots-reserves}.

\begin{table}[t]
  \centering
  \begin{tabular}{llllllll}
      \ggs!abstract!  &  \ggs!after!  &  \ggs!array!  &  \ggs!as!  &  \ggs!before!   \\
  \ggs!between!  &  \ggs!block!  &  \ggs!case!  &  \ggs!cast!  &  \ggs!class!   \\
  \ggs!constructor!  &  \ggs!default!  &  \ggs!do!  &  \ggs!drop!  &  \ggs!else!   \\
  \ggs!elsif!  &  \ggs!end!  &  \ggs!enum!  &  \ggs!error!  &  \ggs!extension!   \\
  \ggs!extern!  &  \ggs!false!  &  \ggs!filewrapper!  &  \ggs!for!  &  \ggs!func!   \\
  \ggs!getter!  &  \ggs!grammar!  &  \ggs!graph!  &  \ggs!gui!  &  \ggs!if!   \\
  \ggs!in!  &  \ggs!indexing!  &  \ggs!insert!  &  \ggs!is!  &  \ggs!label!   \\
  \ggs!let!  &  \ggs!lexique!  &  \ggs!list!  &  \ggs!listmap!  &  \ggs!log!   \\
  \ggs!loop!  &  \ggs!map!  &  \ggs!match!  &  \ggs!message!  &  \ggs!method!   \\
  \ggs!mod!  &  \ggs!not!  &  \ggs!on!  &  \ggs!operator!  &  \ggs!option!   \\
  \ggs!or!  &  \ggs!override!  &  \ggs!parse!  &  \ggs!private!  &  \ggs!proc!   \\
  \ggs!project!  &  \ggs!remove!  &  \ggs!repeat!  &  \ggs!replace!  &  \ggs!rewind!   \\
  \ggs!rule!  &  \ggs!search!  &  \ggs!select!  &  \ggs!self!  &  \ggs!send!   \\
  \ggs!setter!  &  \ggs!sharedmap!  &  \ggs!sortedlist!  &  \ggs!state!  &  \ggs!struct!   \\
  \ggs!style!  &  \ggs!switch!  &  \ggs!syntax!  &  \ggs!tag!  &  \ggs!template!   \\
  \ggs!then!  &  \ggs!true!  &  \ggs!unused!  &  \ggs!var!  &  \ggs!warning!   \\
  \ggs!while!  &  \ggs!with!  &  &    &    \\

  \end{tabular}
  \caption{Mots réservés du langage GALGAS}
  \labelTableau{mots-reserves}
  \ligne
\end{table}


\sectionLabel{Les délimiteurs}{delimiteursGALGAS}

Les délimiteurs du langage GALGAS sont listés dans le \refTableauPage{delimiteurs}.

\begin{table}[t]
  \centering
  \begin{tabular}{lllllllllllllllll}
      \ggs0!=0  &  \ggs0&0  &  \ggs0&&0  &  \ggs0&*0  &  \ggs0&+0  &  \ggs0&++0  &  \ggs0&-0  &  \ggs0&--0  &  \ggs0&/0  &  \ggs0(0  &  \ggs0)0  &  \ggs0*0  &  \ggs0*=0  &  \ggs0+0  &  \ggs0++0   \\
  \ggs0+=0  &  \ggs0,0  &  \ggs0-0  &  \ggs0--0  &  \ggs0-=0  &  \ggs0->0  &  \ggs0/0  &  \ggs0/=0  &  \ggs0:0  &  \ggs0:>0  &  \ggs0;0  &  \ggs0=0  &  \ggs0==0  &  \ggs0>0  &  \ggs0>=0   \\
  \ggs0>>0  &  \ggs0[0  &  \ggs0]0  &  \ggs0^0  &  \ggs0`0  &  \ggs0{0  &  \ggs0|0  &  \ggs0||0  &  \ggs0}0  &  \ggs0~0  &  &    &    &    &    \\

  \end{tabular}
  \caption{Délimiteurs du langage GALGAS}
  \labelTableau{delimiteurs}
  \ligne
\end{table}



\sectionLabel{Les sélecteurs}{selecteursGALGAS}

\begin{table}[t]
  \centering
  \begin{tabular}{llllllllllllll}
    \ggs=!=  & \ggs=!selecteur:=  & \ggs=!?=  & \ggs=!?selecteur:= & \ggs=?= & \ggs=?selecteur:= & \ggs=?!= & \ggs=?!selecteur:= \\
   \end{tabular}
  \caption{Sélecteurs du langage GALGAS}
  \labelTableau{selecteurs}
  \ligne
\end{table}

Les sélecteurs du langage GALGAS sont listés dans le \refTableauPage{selecteurs}.



\sectionLabel{Les séparateurs}{separateursGALGAS}

Les séparateurs du langage GALGAS sont~:
\begin{itemize}
  \item le caractère \emph{espace}~;
  \item tout caractère dont le point de code est compris entre \texttt{U+0000} et \texttt{U+001F}. 
\end{itemize}



\sectionLabel{Les commentaires}{commentairesGALGAS}

Un commentaire commence par le caractère « \texttt{\#} » s'étend jusqu'à la fin de la ligne courante.








\sectionLabel{Les non terminaux}{nonTerminauxGALGAS}

Un \emph{non terminal} d'une grammaire est un identificateur placé entre les caractères \texttt{<} et \texttt{>}. Exemple~:

\begin{galgas}
 <expression>, <instruction>
\end{galgas}

Les lettres Unicode y sont acceptées.





\sectionLabel{Les terminaux}{terminauxGALGAS}

Un \emph{terminal} d'une grammaire est une chaîne de caractères placée entre deux caractères «~\texttt{\$}~». Exemple~:

\begin{galgas}
 $identifier$, $constant$
\end{galgas}

Tout caractère Unicode dont le point de code est compris entre \texttt{0x21} (« \texttt{!} ») et \texttt{0xFFFD} peut apparaître dans un terminal~:
\begin{galgas}
 $=$, $($, $--$, $≠$
\end{galgas}

Deux échappements sont définis~:
\begin{itemize}
\item «~\texttt{\textbackslash\textbackslash}~» qui permet de définir un unique «~\texttt{\textbackslash}~»~;
\item «~\texttt{\textbackslash\$}~» qui permet de définir un «~\texttt{\$}~».
\end{itemize}

Ceci permet par exemple de définir les terminaux suivants~:
\begin{galgas}
 $\\$, $\$terminal\$$
\end{galgas}



\sectionLabel{Les constantes littérales entières}{constantesLitteralesEntiersGALGAS}

Une constante littérale entière peut être écrite~:
\begin{itemize}
  \item en \emph{décimal}~: elle est constituée de un ou plusieurs chiffres décimaux~; exemple~: \ggs=123=, \ggs=9=, \ggs=05=~;
  \item en \emph{hexadécimal}~: elle commence par \texttt{0x}, suivi d'un ou plusieurs chiffres hexadécimaux~; exemple~: \ggs=0x12A=, \ggs=0xabcd=.
\end{itemize}

Le caractère « \texttt{\_} » peut être utilisé pour séparer les chiffres décimaux ou hexadécimaux~: \ggs=1_234=, \ggs=0x123_4567=.

Une constante littérale entière est typée~; son type est fixé par son suffixe (\refTableau{suffixeConstantesLiteralesEntieres}).

\begin{table}[t]
  \centering
  \begin{tabular}{llllllllllllll}
    \textbf{Suffixe} & \textbf{Type} & \textbf{Exemples}\\
    \emph{Pas de suffixe}  & \ggs=@uint=  & \ggs=1_234=, \ggs=0x1234_5678= \\
    \texttt{L}  & \ggs=@uint64=  & \ggs=1_234L=, \ggs=0x1234_5678L= \\
    \texttt{S}  & \ggs=@sint=  & \ggs=1_234S=, \ggs=0x1234_5678S= \\
    \texttt{LS}  & \ggs=@sint64=  & \ggs=1_234LS=, \ggs=0x1234_5678LS= \\
    \texttt{G}  & \ggs=@bigint=  & \ggs=1_234G=, \ggs=0x1234_5678G= \\
   \end{tabular}
  \caption{Suffixes et types des constantes littérales entières}
  \labelTableau{suffixeConstantesLiteralesEntieres}
  \ligne
\end{table}







\sectionLabel{Les constantes littérales flottantes}{constantesLitteralesFlottantesGALGAS}

Une constante littérale flottante comprend toujours un point. Elle est constituée~:
\begin{itemize}
  \item d'un ou plusieurs chiffres décimaux~;
  \item suivis d'un point~;
  \item suivi de zéro, un ou plusieurs chiffres décimaux.
\end{itemize}

Par exemple~: \ggs=0.=, \ggs=12.34=.

Le caractère « \texttt{\_} » peut être utilisé pour séparer les chiffres~: \ggs=1_234.567_890=.

Une constante littérale flottante est du type \ggs=@double=.




\sectionLabel{Les caractères littéraux}{constantesLitteralesCaracteresGALGAS}

Un \emph{caractère littéral} est un caractère Unicode placé entre deux apostrophes « \texttt{\textquotesingle} ». Exemple~:

\begin{galgas}
 'a', 'æ', 'Œ'
\end{galgas}

Plusieurs séquences d'échappements sont définies et listées dans le \refTableau{echappementConstantesLiteralesCaracteres}.

\begin{table}[t]
  \centering
  \begin{tabular}{llllllllllllll}
    \textbf{Échappement} & \textbf{Caractère} & \textbf{Point de code}\\
    \ggs='\f'=  & Nouvelle page & \texttt{U+0C} \\
    \ggs='\n'=  & Passage à la ligne (\emph{Line Feed}) & \texttt{U+0A} \\
    \ggs='\r'=  & Retour chariot & \texttt{U+0D} \\
    \ggs='\t'=  & Tabulation horizontale & \texttt{U+09} \\
    \ggs='\v'=  & Tabulation verticale & \texttt{U+0B} \\
    \ggs='\\'=  & Barre oblique inversée & \texttt{U+5C} \\
    \ggs='\0'=  & Caractère nul & \texttt{U+0} \\
    \ggs='\''=  & Apostrophe & \texttt{U+27} \\
    \ggs='\uabcd'=  & Caractère du plan de base (4 chiffres hexadécimaux) & \texttt{U+ABCD} \\
    \texttt{\textquotesingle\textbackslash Uabcdefgh\textquotesingle}  & Caractère Unicode (8 chiffres hexadécimaux) & \texttt{U+abcdefgh} \\
   \end{tabular}
  \caption{Séquence d'échappement des constantes littérales caractère}
  \labelTableau{echappementConstantesLiteralesCaracteres}
  \ligne
\end{table}




\sectionLabel{Les constantes chaînes de caractères}{constantesLitteralesChainesGALGAS}

Un \emph{chaîne de caractères littérale} est une séquence de caractères Unicode placé entre deux guillemets « \texttt{"} ». Exemple~:

\begin{galgas}
 "une chaîne", "Œnologie"
\end{galgas}

Plusieurs séquences d'échappements sont définies et listées dans le \refTableau{echappementConstantesLiteralesChaine}.

\begin{table}[t]
  \centering
  \begin{tabular}{llllllllllllll}
    \textbf{Échappement} & \textbf{Caractère} & \textbf{Point de code}\\
    \ggs="\f"=  & Nouvelle page & \texttt{U+0C} \\
    \ggs="\n"=  & Passage à la ligne (\emph{Line Feed}) & \texttt{U+0A} \\
    \ggs="\r"=  & Retour chariot & \texttt{U+0D} \\
    \ggs="\t"=  & Tabulation horizontale & \texttt{U+09} \\
    \ggs="\v"=  & Tabulation verticale & \texttt{U+0B} \\
    \ggs="\\"=  & Barre oblique inversée & \texttt{U+5C} \\
    \ggs="\""=  & Guillemet & \texttt{U+22} \\
    \ggs="\uabcd"=  & Caractère du plan de base (4 chiffres hexadécimaux) & \texttt{U+ABCD} \\
%    \texttt{\textquotedbl\textbackslash Uabcdefgh\textquotedbl}  & Caractère Unicode (8 chiffres hexadécimaux) & \texttt{U+abcdefgh} \\
    \texttt{"\textbackslash Uabcdefgh"}  & Caractère Unicode (8 chiffres hexadécimaux) & \texttt{U+abcdefgh} \\
   \end{tabular}
  \caption{Séquence d'échappement des constantes littérales chaîne de caractères}
  \labelTableau{echappementConstantesLiteralesChaine}
  \ligne
\end{table}








\sectionLabel{Les noms de types}{nomTypeGALGAS}

Un nom de type~:
\begin{itemize}
  \item commence par un caractère « \texttt{@}~»~;
  \item est suivi par un ou plusieurs chiffres ou lettres~;
  \item est suivi éventuellement par un tiret « \texttt{-} », lui-même suivi par un ou plusieurs chiffres ou lettres.
\end{itemize}

Par exemple~:
\begin{galgas}
 @string, @stringlist-element, @2stringlist
\end{galgas}




\sectionLabel{Les attributs}{attributsGALGAS}

Un attribut~:
\begin{itemize}
  \item commence par un caractère «~\texttt{\%}~»~;
  \item est suivi par une lettre Unicode~;
  \item est suivi par une ou plusieurs lettres Unicode, chiffres décimaux, «~\texttt{-}~» ou «~\texttt{\_}~».
\end{itemize}

Par exemple~:
\begin{galgas}
 %once, %translate, %héhé-π
\end{galgas}





%!TEX encoding = UTF-8 Unicode
%!TEX root = ../galgas-book.tex

%--------------------------------------------------------------
\chapterLabel{Le composant \texttt{project}}{composantProjet}\index{Component!Project}
%-------------------------------------------------------------


Le composant \ggs!project! permet de paramétrer un projet GALGAS. Il doit être placé dans un fichier source particulier, d'extension « \texttt{.galgasProject} ». Sont déclarés dans un fichier projet :
\begin{itemize}
  \item la version du projet (dans l'en-tête : \refSectionPage{enTeteProjet}) ;
  \item le nom des exécutables engendrés (dans l'en-tête : \refSectionPage{enTeteProjet}) ;
  \item les cibles de compilation : \refSectionPage{ciblesCompilation}  ;
  \item les fichiers sources : \refSectionPage{defFichierSourceProjet}.
\end{itemize}

Voici un exemple de composant projet :

\begin{galgas}
project (1:2:3) -> "logo" {
#--- Targets
  %makefile-macosx
  %makefile-unix
  %makefile-x86linux32-on-macosx
  %makefile-x86linux64-on-macosx
  %makefile-win32-on-macosx
  %makefile-msys32-on-windows
  %Mavericks
  %applicationBundleBase : "fr.what"
  %codeblocks-windows
  %codeblocks-mac

#--- Source files
  "galgas-sources/logo-lexique.galgas"
  "galgas-sources/logo-options.galgas"
  "galgas-sources/logo-semantics.galgas"
  "galgas-sources/logo-syntax.galgas"
  "galgas-sources/logo-grammar.galgas"
  "galgas-sources/logo-cocoa.galgas"
  "galgas-sources/logo-program.galgas"
}
\end{galgas}







\sectionLabel{En-tête du fichier projet}{enTeteProjet}

L'en-tête d'un projet définit deux informations :
\begin{itemize}
  \item la version du projet ;
  \item le nom des exécutables engendrés.
\end{itemize}

\subsectionLabel{Version du projet}{versionProjet}

\begin{galgas}
project (1:2:3) -> "logo" {
  ...
}
\end{galgas}

La version du projet apparaît sous la forme d'un triplet qui suit le mot-clé \ggs!project! : \ggs!1:2:3! dans le code ci-dessus. C'est ce triplet (sous la forme \texttt{1.2.3}) qui apparaît lorsque l'on invoque l'option \texttt{-{}-version} sur l'utilitaire ligne de commande engendré.
Dans le code, cette information peut être obtenu par le \refConstructorPage{application}{projectVersionString}.


\subsection{Nom des exécutables engendrés}

\begin{galgas}
project (1:2:3) -> "logo" {
  ...
}
\end{galgas}

Le nom des exécutables engendrés est fixé par la chaîne de caractères qui apparaît dans l'en-tête : \texttt{logo} dans l'exemple ci-dessus. Les exécutables compilés en mode \emph{release} portent directement ce nom, ceux compilés en mode \emph{debug} portent ce nom augmenté du suffixe « \texttt{-debug} » : \texttt{logo-debug}.


\sectionLabel{Cibles de compilation}{ciblesCompilation}

GALGAS peut engendrer des cibles de compilation pour Mac, Linux et Windows. Les outils engendrés sont des \emph{utilitaires en ligne de commande}, sauf sur Mac où une application Cocoa peut être engendrée.

\subsection{Cibles pour Linux}

Deux choix sont possibles :
\begin{itemize}
\item \texttt{Code::Blocks} ;
\item compilation en ligne de commande.
\end{itemize}

\subsubsection{\texttt{Code::Blocks} pour Linux}

L'option \ggs!%codeblocks-linux32! engendre une cible qui peut être compilée sur Linux 32 bits, et \ggs!%codeblocks-linux64! une cible compilable sur Linux 64 bits, en utilisant \texttt{Code::Blocks}\footnote{\url{http://www.codeblocks.org}}.
\begin{galgas}
project (0:0:1) -> "logo" {
  %codeblocks-linux32
  ...
}
\end{galgas}

\subsubsection{Compilation en ligne de commande pour Linux}

La déclaration \ggs!%makefile-unix! engendre une cible qui peut être compilée indifféremment sur Linux ou sur Mac. L'exécutable engendré est un exécutable 32 bits sur un Linux 32 bits, et un 64 bits sur un Linux 64 bits.
\begin{galgas}
project (0:0:1) -> "logo" {
  %makefile-unix
  ...
}
\end{galgas}



\subsection{Cibles pour Mac}

Comme GALGAS est développé sur Mac, c'est pour cette plateforme que l'on trouve le plus grand nombre de cibles :
\begin{itemize}
  \item application Cocoa ;
  \item compilation via Code::Blocks ;
  \item compilation en ligne de commande ;
  \item cross-compilation pour Win32 ;
  \item cross-compilation pour Linux32 ;
  \item cross-compilation pour Linux64.
\end{itemize}

\subsubsection{Application Cocoa}

Cette cible est l'objet du \refChapterPage{appliCocoa}.

\subsubsection{Compilation en ligne de commande pour Mac}

La déclaration \ggs!%makefile-macosx! engendre une cible pour obtenir un exécutable en ligne de commande sur Mac. Note : on peut aussi utiliser \ggs!%makefile-unix!.
\begin{galgas}
project (0:0:1) -> "logo" {
  %makefile-macosx
  ...
}
\end{galgas}

\subsubsection{\texttt{Code::Blocks} pour Mac}

L'option \ggs!%codeblocks-mac! engendre une cible qui peut être compilée sur Mac en utilisant \texttt{Code::Blocks}\footnote{\url{http://www.codeblocks.org}}.
\begin{galgas}
project (0:0:1) -> "logo" {
  %codeblocks-linux32
  ...
}
\end{galgas}

\subsubsection{Cross-compilation en ligne de commande pour Win32}

La déclaration \ggs!%makefile-win32-on-macosx! engendre une cible pour obtenir sur Mac un exécutable en ligne de commande pour Win32. À la première cross-compilation, le cross-compilateur est téléchargé à partir du site \texttt{rts-software} et placé dans \texttt{$\sim$/galgas-tools-for-cross-compilation}.

\begin{galgas}
project (0:0:1) -> "logo" {
  %makefile-win32-on-macosx
  ...
}
\end{galgas}


\subsubsection{Cross-compilation en ligne de commande pour Linux32}

La déclaration \ggs!%makefile-x86linux32-on-macosx! engendre une cible pour obtenir sur Mac un exécutable en ligne de commande pour Linux 32 bits sur x86. À la première cross-compilation, le cross-compilateur est téléchargé à partir du site \texttt{rts-software} et placé dans \texttt{$\sim$/galgas-tools-for-cross-compilation}.

\begin{galgas}
project (0:0:1) -> "logo" {
  %makefile-x86linux32-on-macosx
  ...
}
\end{galgas}


\subsubsection{Cross-compilation en ligne de commande pour Linux64}

La déclaration \ggs!%makefile-x86linux64-on-macosx! engendre une cible pour obtenir sur Mac un exécutable en ligne de commande pour Linux 64 bits sur x86. À la première cross-compilation, le cross-compilateur est téléchargé à partir du site \texttt{rts-software} et placé dans \texttt{$\sim$/galgas-tools-for-cross-compilation}.

\begin{galgas}
project (0:0:1) -> "logo" {
  %makefile-x86linux64-on-macosx
  ...
}
\end{galgas}



\subsection{Cible pour Windows : \texttt{CodeBlocks}}

Sur Windows, la compilation C++ du projet engendré s'effectue avec \texttt{Code::Blocks}\footnote{\url{http://www.codeblocks.org}}. La cible est engendrée par la déclaration \ggs!%codeblocks-windows!.

\begin{galgas}
project (0:0:1) -> "logo" {
  %codeblocks-windows
  ...
}
\end{galgas}





\sectionLabel{Déclaration \texttt{\%quietOutputByDefault}}{projetDeclarationQuietOutputByDefault}\index{\%quietOutputByDefault}

À partir de la version 3.1.4, GALGAS et les exécutables engendrés par GALGAS sont verbeux par défaut, c'est-à-dire que leur exécution affiche sur le terminal de nombreuses informations sur le déroulement de l'exécution, comme par exemple la mise à jour ou la création de fichiers. L'option de la ligne de commande \emph{quiet} (\refSectionPage{optionsQuietVerbose}) permet d'inhiber l'émission de ces messages.

On peut inverser ce comportement en faisant figurer \ggs!%quietOutputByDefault! parmi les déclarations du fichier projet :
\begin{galgas}
project (0:0:1) -> "logo" {
  ...
  %quietOutputByDefault
  ...
}
\end{galgas}

 Dans ce cas, l'exécutable engendré par GALGAS est silencieux par défaut, et bavard grâce à l'option de la ligne de commande \emph{verbose} (\refSectionPage{optionsQuietVerbose}).

En résumé :
\begin{itemize}
\item par défaut, sans l'option \ggs!%quietOutputByDefault! parmi les déclarations du fichier projet, l'exécutable est bavard par défaut, et l'option de la ligne de commande \emph{quiet} permet de le rendre silencieux ; l'option de la ligne de commande \emph{verbose} n'existe pas ;
\item si l'option \ggs!%quietOutputByDefault! est présente parmi les déclarations du fichier projet, l'exécutable est silencieux par défaut, et l'option de la ligne de commande \emph{verbose} permet de le rendre bavard ; l'option de la ligne de commande \emph{quiet} n'existe pas.
\end{itemize}

Une conséquence est que ni la présence de l'option \emph{quiet} ni la présence de l'option \emph{verbose} ne peuvent être testées par la construction \ggs+[option nom_composant_option.nom_option nom_info]+ (voir \refSubsectionPage{appelOption}). Il faut utiliser le \refConstructorPage{application}{verboseOutput}.







\sectionLabel{Déclaration des fichiers sources du projet}{defFichierSourceProjet}

Deux types de fichiers sources peuvent être déclarés :
\begin{itemize}
  \item des fichiers sources GALGAS ;
  \item des fichiers sources C++.
\end{itemize}

Un fichier source est déclaré sous la forme d'une chaîne de caractères qui définit son chemin :
\begin{itemize}
\item le chemin est absolu si il commence par un « / » ;
\item sinon il est relatif au répertoire qui contient le fichier projet ;
\item l'extension du chemin définit le type : « \texttt{.galgas} » pour un source GALGAS, « \texttt{.cpp} » pour un source C++.
\end{itemize}

Les sources GALGAS déclarés sont inclus dans la compilation GALGAS. L'ordre dans lequel apparaissent ces fichiers n'a pas d'importance sémantique, il définit simplement l'ordre dans lesquels les analyses lexicale et syntaxique sont effectuées.

Les sources C++ déclarés sont ignorés par la compilation GALGAS, et sont simplement ajoutés à la liste des foichiers C++ à compiler.





%!TEX encoding = UTF-8 Unicode
%!TEX root = ../galgas-book.tex

%--------------------------------------------------------------
\chapter{Le composant \texttt{lexique}}
%-------------------------------------------------------------

Le rôle d'un analyseur lexical est de grouper les caractères de la chaîne d'entrée en \emph{symboles terminaux}, ou encore \emph{terminaux}, en écartant les séparateurs comment les espaces ou les commentaires. 

En GALGAS, un analyseur lexical est défini par un composant \galgas{lexique}. Les composants \galgas{syntax}, qui définissent un ensemble de règles de production, font référence à un composant \galgas{lexique}.



\section{Définition d'un composant \texttt{lexique}}


En GALGAS, un composant \galgas{lexique} a la structure suivante :

{\lstset{emph={nom, declarations}, emphstyle=\emph}
\begin{galgascode}
lexique nom :
  declarations
end lexique ;
\end{galgascode}

Le \galgas{nom} est le nom donné au composant ; il est utilisé pour référencer le composant \galgas{lexique} dans un composant \galgas{grammar}.


Dans un composant \galgas{lexique}, cinq types de déclarations sont définies :
\begin{itemize}
  \item déclaration d'attribut lexical ;
  \item déclaration d'un symbole terminal ;
  \item déclaration d'une liste de symboles terminaux ;
  \item déclaration d'un message d'erreur lexical ;
  \item déclaration d'un style ;
  \item déclaration de règles d'analyse.
\end{itemize}

A //lexical attribute// carries the value associated with a terminal symbol: for example, the integer value of a literal integer constant, the string value of a character string constant, ...

In GALGAS, all terminal symbols must be declared either by a //single terminal symbol declaration//, either by a //list of terminal symbols declaration//. This defines the set of defined terminal symbols of your grammar.

Lexical error messages need also to be explicitly declared by //lexical error message declaration//. 

A //style declaration// declares a style identifier, for defining automatic coloring in a text editor. Currently, coloring is only available for Mac OS X Cocoa applications.

The order of declarations is not significant, but any entity must be declared before being used.

==== Lexical Rules Overview ====
The //lexical rules// define the executable part of a lexical component. Every lexical rule define //matching strings// that are are tested against substring from current location in input string. A matching string has a one character or more.

===== Generated Files =====

A lexical component description is translated in C++ code; for every lexical component, GALGAS generates a specific C++ class:
  * the name of the class is the name of the ''**lexique**'' component;
  * this class is declared in a header file that is named the name of the ''**lexique**'' component with the ''%%'%%.h%%'%%'' extension;
  * this class is implemented in a file that is named the name of the ''**lexique**'' component with the ''%%'%%.cpp%%'%%'' extension;
  * this class inherits from ''C\_Lexique'' class (declared in ''libpm/galgas/C\_Lexique.h'' and implemented in ''libpm/galgas/C\_Lexique.cpp'').

The two generated files are generated according the [[generated\_files|GALGAS file generation process]].


===== How a lexical analyzer Works =====

You can consider the lexical analyzer as an autonomous thread which analyzes the input string and which sends the sequence of the terminal symbols to the parser. Of course, for efficiency, the lexical analyzer is actually a parser subroutine.

The flowchart of a GALGAS lexical analyzer execution is:

{{ how\_works\_a\_lexical\_analyzer.png }}

When the input string is loaded from source file, a ''NUL'' character is appended as End Of String (eos) mark.

During execution, the lexical analyzer maintains a //current location// that designates the next character of the input string to be analyzed. Initially, current location points out the first character of the input string.

The lexical analyzer loops until the end of input string is reached. At the beginning of every loop, lexical attributes are reset to their default value.

Then, the first lexical rule matching expressions are tested against substring at current location in input string:
  * on match success, the first lexical rule is executed; usually, this execution sends a terminal symbol to the parser; however, in some cases as parsing a delimitor or a comment, no terminal symbol is sent;
  * on match failure, the lexical analyzer tries to find a match with the second lexical rule, and so on.

If no lexical rule matches, the character at current location is tested against eos character. On match success, the lexical analyzer sends once a predefined terminal symbol (denoted by ''\\$\\$'') to the parser, for telling it the end of input string is reached. On match failure, the //unknow character// lexical error is raised. The character at current location is discarded, that is the current location points out the next character of the input string.

===== Lexical Ambiguities =====

**GALGAS does not currently check that the set of lexical rules is unambiguous.** So, if the set is unambiguous, the rule order is not significant; if two or more rules introduce an ambiguity, the first defined one is used. 

===== A very Simple Example =====

This is very simple scanner, from ''galgas/samples/notSLRgrammar.ggs'':

|''**lexique** my\_scanner\_for\_not\_SLR\_grammar:\\ 
\#--- Identifiers\\ 
\\$id\\$ **error** **message** %%"%%an identifier%%"%% ;\\ 
**rule** %%'%%a%%'%% %%->%% %%'%%z%%'%% | %%'%%A%%'%% %%->%% %%'%%Z%%'%% :\\ 
 **send** \\$id\\$ ;\\ **end** **rule** ;\\ 
\#--- Delimitors\\ 
**list** delimitorsList **error** **message** %%"%%the %%'"%% . * . %%"'%% delimitor%%"%%: %%"%%=%%"%% , %%"%%*%%"%% ;\\ 
**rule** **list** delimitorsList ;\\ 
\#--- Separators\\ 
**rule** %%'%%\1%%'%% %%->%% %%' '%%:\\ 
**end** **rule** ;\\ 
**end** **lexique** ;''|

This ''**lexique**'' component defines the following set of terminal symbols: ''\\$id\\$'' (explicitly declared), ''\\$=\\$'' and ''\\$*\\$'' (declared  by ''delimitorsList'' list.

The first rule sends the ''\\$id\\$'' terminal symbol each time a lower case or upper case character is found. The second rule names the ''delimitorsList'' list and sends the ''\\$=\\$'' or ''\\$*\\$'' terminal symbol each time the corresponding character is found. The last rule discards silently the space character and any control character.

Note that this scanner considers identifiers of only one character: ''ab'' is scanned as two consecutive identifiers.

===== Finding Sample Code =====

You can find examples of ''**lexique**'' components in:
  * ''galgas/sample/alt\_sample.ggs'' file; this is a very basic scanner that handles one-letter identifier and four delimitors;
  * ''galgas/sample/arith\_expression.ggs'' file (for scanning literal integers); 
  * ''galgas/sample/test\_LR1\_grammar.ggs'' file gives an example of a small scanner for "toy" parser;
  * ''galgas/galgas/galgas\_sources/galgas\_scanner.ggs'' file: this is the actual scanner of the GALGAS language, and scans identifiers, keywords, delimiters, literal integers, literal characters, literal character strings, galgas type names (the '@' character followed by a sequence of letters), comments, ...   

====== Lexical Declarations ======

===== Single Terminal Symbol Declaration =====

The //single terminal symbol declaration// declares a name used for naming a terminal symbol. This declaration just performs declaration, not scanning. For sending this terminal symbol to the parser, it must be named in a ''**send**'' lexical instruction within a lexical rule.

The declaration associates to the terminal symbol a possibly empty list of lexical attributes and a syntax error message (not a //lexical// error message), defined by a character string.

First example:

|''\$literal\_integer\$ **error** **message** %%"%%a decimal number%%"%%;''|

This declaration names no lexical attribute. Consequently, when the lexical send instruction ''**send** \$literal\_integer\$;'' will be called from a lexical rule, only the terminal symbol will be sent to the parser, but not the literal integer value. The parser has no way to get the actual value: all integer values share the same terminal symbol. It is sufficient for a pure parser, however a real compiler needs the actual value.

Second example:

|''@uint unsignedValueAttribute;\\ 
\$literal\_integer\$ !unsignedValueAttribute **error** **message** %%"%%a decimal number%%"%%;''|

In this declaration, the ''unsignedValueAttribute'' attribute is named in the terminal symbol declaration. So, when the lexical send instruction ''**send** \$literal\_integer\$;'' will be called from a lexical rule, the terminal symbol will be sent to the parser together with the unsigned value of the ''unsignedValueAttribute'' attribute, enabling the semantic instructions to catch it.

===== List of Terminal Symbols Declaration =====

The //list of terminal symbol declaration// associates to a name a list of terminal symbols with a generic syntax error message. It is typically used for declaring the keywords and the delimiters.

An example of key words declaration:

| ''**list** keywordList **error** **message** %%"%%the '%K' key word%%"%%: %%"%%if%%"%%, %%"%%then%%"%%, %%"%%else%%"%% ;'' |

The declared terminal symbols are: ''\$if\$'', ''\$then\$'', ''\$else\$''. The actual syntax error message is built from generic error message by replacing ''%K'' with terminal symbol string (for outputing a single ''%'', write ''%''''%''). So the syntax error message associated to the ''\$if\$'' terminal symbol is: "''the 'if' key word''".

An other example is a delimitor list declaration:

|''**list** delimitorList **error** **message** %%"%%the '%K' delimitor%%"%%: %%"%%.%%"%%, %%"%%;%%"%%, %%"%%(%%"%%, %%"%%)%%"%% ;''|

Actual scanning of a delimitor is done by a ''**rule** **list**'' lexical instruction.

===== Lexical Attribute Declaration =====

Lexical attributes carry values associated with terminal symbol. GALGAS handles string, unsigned, character, float lexical attributes. Every lexical attribute needs to be declared and its declaration names a GALGAS type name.


 The following table summerizes the attributes features and type notation:

%\^ Attribute Type \^ Type Name \^ Default Value \^ Corresponding C++ type \^
| ASCII String | ''@string'' | ''%%""%%'' (the empty string) | ''C\_String'' |
| ASCII Character | ''@char'' | ''%%'\0'%%'' | ''char'' |
| 32-bit Unsigned Integer | ''@uint'' | ''0'' | ''uint32'' |
| 32-bit Signed Integer | ''@sint'' | ''0'' | ''sint32'' |
| Float | ''@double'' | ''0.0'' | ''double'' |

In GALGAS, type names are identifiers prefixed by a ''@'' character.

An ''@string'', ''@char'', ''@uint'', ''@sint'', ''@double'' lexical attribute carry a string, character, unsigned, signed, double value.

In a ''**syntax**'' component, information that defines the location of the scanned terminal symbol in the input string is added to attribute value: so an ''@string'' object in the lexique component corresponds to an ''@lstring'' object in the syntax component. Location information is used by the parser and the semantic instructions for building syntax and semantic error messages that indicates //where// the error is located.

The //default value// is the one used at the beginning of every scanning loop for resetting lexical attribute.

The //corresponding C type// is useful if you want to write your own lexical actions (in C++). Please note that this correspondance is **only** available for lexical actions, and not for semantic action. The ''C\_String'' type is a C++ class that handles mutable character strings, without being worried about memory management. It is declared in the ''libpm/strings/C\_string.h'' file. The ''uint32'' type is the 32-bit unsigned integer type, and the ''sint32'' type is the 32-bit signed integer type. 
 

===== Lexical Error Message Declaration =====

The //lexical error message declaration// associates a name to a string. These error messages are used in lexical actions, and define the message that are displayed when a lexical error occurs.

|  ''**message** decimalNumberTooLarge: %%"%%decimal number too large%%"%%;'' |

 

====== Lexical Rules ======

There are two kinds of //lexical rules//:
  - the //list lexical rule//;
  - the //single lexical rule//.

===== List Lexical Rule =====

This is the simpliest form: it just names a previously defined list of terminal symbols; for example:

|''**rule** **list** delimitorList;''|

//Matching expressions// are the set of strings defined by the list. This rule tries to find a substring from input string at current location that matches a terminal symbol string defined in the list, sorted by decreasing length (so longest strings are tested first). On match success, //executing the rule// consists of sending the corresponding terminal symbol.

This kind of rule is typically used for scanning for a delimitor.

===== Single Lexical Rule =====

A //single lexical rule// has the following form:

|''**rule** //matching\_expression//:\\  //lexical\_instructions//\\ **end** **rule**;''|

The //matching expression// defines a set of matching strings, that are tested against the substring from input string at current location. On match, the //lexical instructions// are executed.

==== Matching Expression ====

A matching expression can be:
  - a one-character string (for example, ''%%'%%a%%'%%'' matches the ''a'' character);
  - an union of one-character strings, defined by a character subrange (for example, ''%%'%%a%%'%% %%->%% %%'%%z%%'%%'' matches a lower case letter);
  - a one or more characters string (for example, ''%%"%%:=%%"%%'' matches the corresponding string);
  - an union of above (for example: ''%%'%%A%%'%% %%->%% %%'%%Z%%'%% | %%'%%a%%'%% %%->%% %%'%%z%%'%%'' matches a lower or upper case letter).

On match success, the current location is moved to designate the character after the matching string.

==== Lexical Select Instruction ====

The //lexical select instruction// is the following:

|''**select**\\ **when** //matching\_expression\_1\_in\_select//: //lexical\_instructions\_1//\\ **when** //matching\_expression\_2\_in\_select//: //lexical\_instructions\_2//\\ ...\\ **default** //default\_lexical\_instructions//\\ **end** **select**;''|

A //lexical select instruction// has one or more ''**when**'' branches.

//matching expression\_1\_in\_select//, //matching expression\_2\_in\_select// conform to the defined above //matching\_expression//.

This instruction tries to match the different //matching expressions// until a matching success is found. In such case, the corresponding //lexical instructions// are executed. If all matching fail, the //default lexical instructions// are executed.

==== Lexical Repeat Instruction ====

The //lexical repeat instruction// is the following:

|''**repeat**\\  //lexical\_instructions\_0//\\ **while** //matching\_expression\_1\_in\_repeat//: //lexical\_instructions\_1//\\ **while** //matching\_expression\_2\_in\_repeat//: //lexical\_instructions\_2//\\ ...\\ **end** **repeat**;''|

A //lexical while instruction// has one or more ''**while**'' branches.

//matching expression\_1\_in\_repeat//, //matching expression\_2\_in\_repeat// can be:
  - an expression conform to the defined above //matching\_expression//;
  - the ''~ //string//'' construct: the match succeeds when the //string// **is not** the current string;
  - the ''~ //string1//, //string2//, ...'' construct: the match succeeds when neither of //string1//, //string2//, ... are the current string.

This instruction first executes the //lexical instructions 0//. Then, it tries to match the different //matching expressions// until a matching success is found. In such case, the corresponding //lexical instructions// are executed, then the instruction is executed again (from //lexical instructions 0//). If all matching fail, execution of this instruction is complete (excution goes on the next instruction).

==== Lexical Action Call Instruction ====

The //lexical action call instruction// calls a C++ defined method for performing computation and checking on lexical attributes. Its syntax is the following:

|''lexical\_action\_name (parameter, ...) ;''|

or

|''lexical\_action\_name (parameter, ...) **error** message\_name, ... ;''|

A lexical action is designated by its name. It accepts one or more parameters, and zero, one or more messages names.

A parameter is:
  - either a lexical attribute,
  - either a lexical function call;
  - either the joker character ''%%'%%*%%'%%'' that represents the character at current location.

A lexical action can be predefined or defined by the user. Predefined lexical actions are actually methods of ''C\_Lexique'' class (the generated scanner is a class that inherits from this class). User defined lexical actions must be implemented as methods of the generated scanner class.

**Note that no parameter type checking, no error message count checking is performed by GALGAS. ** A parameter type error or a message count error is detected at C++ compilation stage.
 
==== Lexical Function Call ====

The //lexical function call// calls a C++ defined method for performing computation on lexical attributes. It can only appear as parameter of a lexical action call or a parameter of an other lexical function call. Its syntax is the following:

|''lexical\_function\_name (parameter, ...) ;''|

A lexical function is designated by its name. It accepts one or more parameters.

A lexical function parameter is:
  - either a lexical attribute,
  - either a lexical function call;
  - either the joker character ''%%'%%*%%'%%'' that represents the character at current location.

A lexical function can be predefined or defined by the user. Predefined lexical actions are actually methods of ''C\_Lexique'' class (the generated scanner is a class that inherits from this class). User defined lexical functions must be implemented as methods of the generated scanner class.

**Note that no parameter type checking is performed by GALGAS. ** A parameter type error is detected at C++ compilation stage.
 
==== Lexical Error Instruction ====

The //lexical error instruction// raises a lexical error. Its syntax is:

|''**error** message\_name ;''|

The //message name// is the name of a previously declared lexical error message.

==== Lexical Send Instruction ====

The //lexical send instruction// sends a terminal symbol to the parser. It has several forms:

=== First Form ===

|''**send** terminal\_symbol ;''|

This instruction sends inconditionnaly the //terminal symbol// to the parser.

=== Second Form ===

|''**send** **search** //attribute\_name// **in** //lexical\_list// **default** terminal\_symbol ;''|

This instruction first search for //attribute name// value in the //lexical list//. If found, the corresponding terminal symbol is sent to the parser. If not found, the default //terminal symbol// is sent.

Several consecutive ''**search**'' are accepted, allowing sequential searching in different lists:

|''**send** **search** //attribute\_name\_1// **in** //lexical\_list\_1// **default** **search** //attribute\_name\_2// **in** //lexical\_list\_2// **default** terminal\_symbol ;''|

==== Lexical drop Instruction ====

|Available in GALGAS 1.5.6 and later.|


The //lexical drop instruction// does not send any terminal symbol to the parser. It is only significant for lexical coloring (see [[\#coloring\_comments|coloring comments]]).

This instruction names a terminal symbol:
|''**drop** //terminal\_symbol// ;''|


==== Lexical tag Instruction ====

|Available in GALGAS 1.5.6 and later.|

This instruction declares a new //tag identifier//.

|''**tag** //tag\_identifier// ;''|

A ''**tag**'' instruction records a location in the scanned file. The only way to use the declared tag identifier is the [[\#lexical\_rewind\_instruction|lexical rewind instruction]].

==== Lexical rewind Instruction ====

|Available in GALGAS 1.5.6 and later.|

|''**rewind** //tag\_identifier// **send** //terminal\_symbol//;''|

This instruction rewinds the scanned location from the tag identifier value, and sends the terminal symbol to the parser.

====== Lexical Routines and Lexical Functions ======

%\^ Available in GALGAS 1.8.4 and later.\^ 

There are two kinds of lexical actions:
  * lexical routines;
  * lexical actions.

Lexical routine calls are instructions. Lexical function calls can appear as actual output parameters of routine calls and function calls. GALGAS predefines several lexical routines and several lexical functions (listed below).

A lexical routine accepts:
  * zero, one or more input/output or input formal arguments;
  * zero, one or more error messages.

A lexical function accepts:
  * zero, one or more input formal arguments.

Running the ''%%--print-predefined-lexical-actions%%'' command line option lists all predefined routines and functions prototype.

===== Predefined Lexical Routines =====


==== codePointToUnicode ====
''codePointToUnicode !@string inCodePointString ?!@string ioString ;''

==== convertDecimalStringIntoSInt ====
''convertDecimalStringIntoSInt !@string inString ?!@sint ioSignedNumber **error** inNumberTooLargeError, inCharacterIsNotDecimalDigitError ;''

==== convertDecimalStringIntoSInt64 ====
''convertDecimalStringIntoSInt64 !@string inString ?!@sint64 ioSignedNumber **error** inNumberTooLargeError, inCharacterIsNotDecimalDigitError ;''

==== convertDecimalStringIntoUInt ====
''convertDecimalStringIntoUInt !@string inString ?!@uint ioUnsignedNumber **error** inNumberTooLargeError, inCharacterIsNotDecimalDigitError ;''

==== convertDecimalStringIntoUInt64 ====
''convertDecimalStringIntoUInt64 !@string inString ?!@uint64 ioUnsignedNumber **error** inNumberTooLargeError, inCharacterIsNotDecimalDigitError ;''

==== convertHTMLSequenceToUnicodeCharacter ====
''convertHTMLSequenceToUnicodeCharacter ?!@string inString ?!@char ioUnicodeCharacter **error** inUnassignedHTMLSequenceError ;''

==== convertHexStringIntoSInt ====
''convertHexStringIntoSInt !@string inString ?!@sint ioSignedNumber **error** inNumberTooLargeError, inCharacterIsNotHexDigitError ;''

==== convertHexStringIntoSInt64 ====
''convertHexStringIntoSInt64 !@string inString ?!@sint64 ioSignedNumber **error** inNumberTooLargeError, inCharacterIsNotHexDigitError ;''

==== convertHexStringIntoUInt ====
''convertHexStringIntoUInt !@string inString ?!@uint ioUnsignedNumber **error** inNumberTooLargeError, inCharacterIsNotHexDigitError ;''

==== convertHexStringIntoUInt64 ====
''convertHexStringIntoUInt64 !@string inString ?!@uint64 ioUnsignedNumber **error** inNumberTooLargeError, inCharacterIsNotHexDigitError ;''

==== convertStringToDouble ====
''convertStringToDouble !@string inString ?!@double ioDouble **error** inConversionError ;''

This action tries to convert the string value of the first argument into a double value. On success, the resulting double is set to the second argument. The conversion error message is displayed on conversion error.

==== convertUInt64ToSInt64 ====
''convertUInt64ToSInt64 !@uint64 inUnsignedNumber ?!@sint64 ioSignedNumber **error** inNumberTooLargeError ;''

If the unsigned value of the ''inUnsignedNumber'' argument is greater than ''2<sup>63</sup>-1'', the error is raised. Otherwise, the value is assigned to the ''ioSignedNumber'' argument.

==== convertUIntToSInt ====
''convertUIntToSInt !@uint inUnsignedNumber ?!@sint ioSignedNumber **error** inNumberTooLargeError ;''

If the unsigned value of the ''inUnsignedNumber'' argument is greater than ''2<sup>31</sup>-1'', the error is raised. Otherwise, the value is assigned to the ''ioSignedNumber'' argument.

==== convertUnsignedNumberToUnicodeChar ====
''convertUnsignedNumberToUnicodeChar ?!@uint inUnsignedNumber ?!@char ioUnicodeCharacter **error** inUnassignedUnicodeValueError ;''

==== enterBinDigitIntoUInt ====
''enterBinDigitIntoUInt !@char inCharacter ?!@uint ioUnsignedNumber **error** inNumberTooLargeError, inCharacterIsNotBinDigitError ;''

==== enterBinDigitIntoUInt64 ====
''enterBinDigitIntoUInt64 !@char inCharacter ?!@uint64 ioUnsignedNumber **error** inNumberTooLargeError, inCharacterIsNotBinDigitError ;''

==== enterCharacterIntoCharacter ====
''enterCharacterIntoCharacter ?!@char ioCharacter !@char inCharacter ;''

This routine performs ''ioCharacter := inCharacter'' assignment.

==== enterCharacterIntoString ====
''enterCharacterIntoString ?!@string ioString !@char inCharacter ;''

Appends the character value of the second argument to the string value of the first argument. The resulting string is set to the first argument.

==== enterDigitIntoASCIIcharacter ====
''enterDigitIntoASCIIcharacter ?!@char ioASCIICharacter !@char inDecimalDigitCharacter **error** inErrorCodeGreaterThan255, inErrorNotDecimalDigitCharacter ;''

Build an ASCII character from its decimal definition.

First, the character value of the ''inDecimalDigitCharacter'' argument is tested to be a valid decimal digit, that is in one range ''[%%'%%0%%'%%, %%'%%9%%'%%]''. On failure, the ''inErrorNotDecimalDigitCharacter'' error message is displayed. On success, the unsigned value of the ''ioASCIICharacter'' argument is multiplied by ten, and is added the decimal value corresponding to second argument. If the result is lower or equal to ''2<sup>8</sup>-1'', it is set to the ''ioASCIICharacter'' argument. Otherwise, the ''inErrorCodeGreaterThan255'' error is raised.

Note: this lexical action treats characters as unsigned values.

==== enterDigitIntoUInt ====
''enterDigitIntoUInt !@char inDecimalDigitCharacter ?!@uint ioUnsignedNumber **error** inNumberTooLargeError, inCharacterIsNotDecimalDigitError ;''

First, the value of ''inDecimalDigitCharacter'' argument is tested to be in the range ''[%%'%%0%%'%%, %%'%%9%%'%%]''. On failure, the ''inCharacterIsNotDecimalDigitError'' error message is displayed. On success, the unsigned value of the first argument is multiplied by ten, and is added the decimal value corresponding to the ''ioUnsignedNumber'' argument. If the result is lower or equal to ''2<sup>32</sup>-1'', it is set to the ''ioUnsignedNumber'' argument. Otherwise, the ''inNumberTooLargeError'' error is raised.

==== enterDigitIntoUInt64 ====
''enterDigitIntoUInt64 !@char inDecimalDigitCharacter ?!@uint64 ioUnsignedNumber **error** inNumberTooLargeError, inCharacterIsNotDecimalDigitError ;''

First, the value of ''inDecimalDigitCharacter'' argument is tested to be in the range ''[%%'%%0%%'%%, %%'%%9%%'%%]''. On failure, the ''inCharacterIsNotDecimalDigitError'' error message is displayed. On success, the unsigned value of the first argument is multiplied by ten, and is added the decimal value corresponding to the ''ioUnsignedNumber'' argument. If the result is lower or equal to ''2<sup>64</sup>-1'', it is set to the ''ioUnsignedNumber'' argument. Otherwise, the ''inNumberTooLargeError'' error is raised.

==== enterHexDigitIntoASCIIcharacter ====
''enterHexDigitIntoASCIIcharacter ?!@char ioASCIICharacter !@char inHexDigitCharacter **error** inErrorCodeGreaterThan255, inErrorNotHexDigitCharacter ;''

Build an ASCII character from its hexadecimal definition.

First, the character value of the ''inHexDigitCharacter'' argument is tested to be a valid hexadecimal digit, that is in one of the ranges ''[%%'%%0%%'%%, %%'%%9%%'%%]'', ''[%%'%%a%%'%%, %%'%%f%%'%%]'', ''[%%'%%A%%'%%, %%'%%F%%'%%]''. On failure, the ''inErrorNotHexDigitCharacter'' error message is displayed. On success, the unsigned value of the first argument is multiplied by sixteen, and is added the hexadecimal value corresponding to ''ioASCIICharacter'' argument. If the result is lower or equal to ''2<sup>8</sup>-1'', it is set to the ''ioASCIICharacter'' argument. Otherwise, the ''inErrorCodeGreaterThan255'' error is raised.

Note: this lexical action treats characters as unsigned values.

==== enterHexDigitIntoUInt ====
''enterHexDigitIntoUInt !@char inHexDigitCharacter ?!@uint ioUnsignedNumber **error** inNumberTooLargeError, inCharacterIsNotHexDigitError ;''

First, the character value of the ''inHexDigitCharacter'' argument is tested to be a valid hexadecimal digit, that in one of the the ranges ''[%%'%%0%%'%%, %%'%%9%%'%%]'', ''[%%'%%a%%'%%, %%'%%f%%'%%]'', ''[%%'%%A%%'%%, %%'%%F%%'%%]''. On failure, the ''inCharacterIsNotHexDigitError'' error message is displayed. On success, the unsigned value of the ''ioUnsignedNumber'' argument is multiplied by sixteen, and is added the hexadecimal value corresponding to second argument. If the result is lower or equal to ''2<sup>32</sup>-1'', it is set to the ''ioUnsignedNumber'' argument. Otherwise, the first error is raised.

==== enterHexDigitIntoUInt64 ====
''enterHexDigitIntoUInt64 !@char inHexDigitCharacter ?!@uint64 ioUnsignedNumber **error** inNumberTooLargeError, inCharacterIsNotHexDigitError ;''

First, the character value of the ''inHexDigitCharacter'' argument is tested to be a valid hexadecimal digit, that in one of the the ranges ''[%%'%%0%%'%%, %%'%%9%%'%%]'', ''[%%'%%a%%'%%, %%'%%f%%'%%]'', ''[%%'%%A%%'%%, %%'%%F%%'%%]''. On failure, the ''inCharacterIsNotHexDigitError'' error message is displayed. On success, the unsigned value of the ''ioUnsignedNumber'' argument is multiplied by sixteen, and is added the hexadecimal value corresponding to second argument. If the result is lower or equal to ''2<sup>64</sup>-1'', it is set to the ''ioUnsignedNumber'' argument. Otherwise, the first error is raised.

==== enterOctDigitIntoUInt ====
''enterOctDigitIntoUInt !@char inString ?!@uint ioUnsignedNumber **error** inNumberTooLargeError, inCharacterIsNotOctDigitError ;''

==== enterOctDigitIntoUInt64 ====
''enterOctDigitIntoUInt64 !@char inString ?!@uint64 ioUnsignedNumber **error** inNumberTooLargeError, inCharacterIsNotOctDigitError ;''

==== multiplyUInt ====
''multiplyUInt !@uint inUnsignedNumber ?!@uint ioUnsignedNumber **error** inResultTooLargeError ;''

Multiply the ''ioUnsignedNumber'' value by ''inUnsignedNumber'' value. Detection of overflow is performed.

==== multiplyUInt64 ====
''multiplyUInt64 !@uint inUnsignedNumber ?!@uint64 ioUnsignedNumber **error** inResultTooLargeError ;''

Multiply the ''ioUnsignedNumber'' value by ''inUnsignedNumber'' value. Detection of overflow is performed.

==== negateSInt ====
''negateSInt ?!@sint ioNumber **error** inNumberTooLargeError ;''

==== negateSInt64 ====
''negateSInt64 ?!@sint64 ioNumber **error** inNumberTooLargeError ;''

===== Predefined Lexical functions =====

==== toLower ====
''toLower ?@char inCharacter %%->%% @char ;''

If the character value of the argument is an upper case letter, this function returns the corresponding lower case letter. Otherwise, it returns the unchanged character value of the argument.

==== toUpper ====
''toUpper ?@char inCharacter %%->%% @char ;''


If the character value of the argument is an lower case letter, this function returns the corresponding upper case letter. Otherwise, it returns the unchanged character value of the argument.



===== Defining your own Lexical Actions and Lexical Functions =====

You can define your own lexical actions and functions in C++ and make them available to called by lexical action call instructions.

==== Where ? ====

You must define your lexical actions and functions as a method of the C++ class generated by compilation of the ''**lexique**'' component. You need to modify the generated code, adding method prototype declaration in class declaration.

**So that the method declaration that you added is not deleted at the time of a future compilation, define it in user zone 2 of the generated header file.** For more details, see [[generated\_files |file generation process page]].

For implementing your method, you can insert it in user zone 2 of the generated implementation file (for more details, see [[generated\_files |file generation process page]]). Alternatively, you can implement it in any other file, provided you include the needed header files.

===== Correspondance between Lexical Action Calls and C++ Called Methods =====

This table gives the correspondance between lexical argument types and C++ types. **Note this correspondance is only available for lexical arguments**.

%\^Lexical Formal Argument Type  \^C++ Type  \^
|''? @string''  |''**const** C\_String \&''|
|''?! @string''  |''C\_String \&''|
|''? @char''  |''**const** **char**''|
|''?! @char''  |''**char** \&''|
|''? @uint''  |''**const** uint32''|
|''?! @uint''  |''uint32 \&''|
|''? @sint''  |''**const** sint32''|
|''?! @sint''  |''sint32 \&''|
|''? @double''  |''**const** **double**''|
|''?! @double''  |''**double** \&''|

''?'' means the formal argument has input passing mode: it cannot be modified by the lexical action. ''?!'' means the formal argument has in/out passing mode: its value is got from the caller, can modified by the lexical action and is returned to the caller.

An error message argument corresponds to the C++ type ''**const** **char** *''.

In C++ generated code, the method call instruction generated by lexical action call names the lexical action name, prefixed by ''scanner\_routine\_''.

For example, consider the ''convertStringToDouble'' lexical action described below. This corresponds to the following method prototype:

''**void** scanner\_routine\_convertStringToDouble (**const** C\_String \&, **double** \&, **const char** *) ;''
==== Defining Action and Function Prototype ====

The prototype must conform to the rules presented in the [[\#Correspondance between Lexical Action Calls and C++ Called Methods|above]] section.

%\^Remember that GALGAS does not perform any checking on lexical action calls. Errors are detected at C++ compilation stage.\^

====== Scanner Examples ======

===== Scanning identifiers =====

|''@string identifierString;\\ 
\$identifier\$ !identifierString **error** **message** %%"%%an identifier%%"%%;\\ 
**rule** %%'a'->'z' | 'A'->'Z'%%:\\ 
 **repeat**\\ 
  enterCharacterIntoString !?identifierString !* ;\\ 
 **while** %%'a'->'z' | 'A'->'Z' | '\_' | '0'->'9'%%:\\ 
 **end** **repeat** ;\\ 
 **send** \$identifier\$ ;\\
**end** **rule** ;''|

|''@string identifierString;\\ 
\$identifier\$ !identifierString **error** **message** %%"%%an identifier%%"%%;\\ 
**rule** %%'a'->'z' | 'A'->'Z'%%:\\ 
 **repeat**\\ 
  enterCharacterIntoString !?identifierString !toLower (!*) ;\\ 
 **while** %%'a'->'z' | 'A'->'Z' | '\_' | '0'->'9'%%:\\ 
 **end** **repeat** ;\\ 
 **send** \$identifier\$ ;\\
**end** **rule** ;''|

===== Scanning identifiers and key words =====

|''@string identifierString;\\ 
\\ 
\$identifier\$ !identifierString **error** **message** %%"%%an identifier%%"%%;\\ 
\\ 
**list** keywordList **error** **message** %%"the '%K' key word": "begin", "else", "end"%%;\\
\\ 
**rule** %%'a'->'z' | 'A'->'Z'%%:\\ 
 **repeat**\\ 
  enterCharacterIntoString !?identifierString !* ;\\ 
 **while** %%'a'->'z' | 'A'->'Z' | '\_' | '0'->'9'%%:\\ 
 **end** **repeat** ;\\ 
 **send** **search** identifierString **in** keywordList  **default** \$identifier\$ ;\\
**end** **rule** ;''|

===== Scanning delimitors =====

|''**list** galgasDelimitorsList **error message** %%"the '%K' delimitor"%%:\\ 
 %%"*",  "|", ",",  ".",  "<>", "::", ">",  "<",  ";",  ":",%%\\ 
 %%"-",  "(", ")",  "->", "?", "==", "??", "!",  ":=", "...",%%\\ 
 %%"[",  "]", "+=", "?!", "!?", "/",  "!=", "<=", ">=", "\&",%%\\ 
 %%"++", "{", "}"%% ;\\ 
\\ 
**rule list** galgasDelimitorsList ;''|

===== Scanning separators =====

|''**rule** %%'\u0001' -> ' '%% :\\ 
**end rule** ;''|

===== Scanning comments =====

|''**rule** '\#' :\\ 
 **repeat**\\ 
 **while** %%'\u0001' -> '\u0009' | '\u000B' -> '\uFFFD'%% :\\ 
 **end repeat** ;\\ 
**end rule** ;''|

===== Scanning decimal unsigned integers =====

|''\$unsigned\_literal\_integer\$ !ulongValue **error message** %%"a decimal number"%% ;\\ 
\$signed\_literal\_integer\$ !longValue **error** **message** %%"a signed decimal number"%% ;\\ 
\\ 
**message** decimalNumberTooLarge : %%"decimal number too large"%% ;\\ 
\\ 
**message** internalError : %%"internal error"%% ;\\ 
\\ 
**rule** %%'0'->'9'%% :\\ 
 enterDigitIntoUlong !?ulongValue !* **error** decimalNumberTooLarge, internalError ;\\ 
 **repeat**\\ 
 **while** %%'0'->'9'%% :\\ 
  enterDigitIntoUlong !?ulongValue !* **error** decimalNumberTooLarge, internalError ;\\ 
 **while** %%'\_'%% :\\ 
 **end repeat** ;\\ 
 **select**\\ 
 **when** %%'S' | 's'%% :\\ 
  convertUlongToLong !?longValue !ulongValue %%error%% decimalNumberTooLarge ;\\ 
  **send** \$signed\_literal\_integer\$ ;\\ 
 **default**\\ 
  **send** \$unsigned\_literal\_integer\$ ;\\ 
 **end select** ;\\ 
**end rule** ;''|

===== Scanning hexadecimal unsigned integers =====

===== Scanning character constant =====

|''\$literal\_char\$ ! charValue **error message** %%"a character constant"%% ;\\ 
\\ 
**message** incorrectCharConstant : %%"incorrect literal character"%% ;\\ 
\\ 
**message** ASCIIcodeTooLargeError : %%"ASCII code > 255"%% ;\\ 
\\ 
**rule** %%'\''%% :\\ 
 **select**\\ 
 **when** %%'\\'%% :\\ 
  **select**\\ 
  **when** %%'f'%% :\\ 
   enterCharacterIntoCharacter !?charValue !%%'\f'%% ;\\ 
  **when** %%'n'%% :\\ 
   enterCharacterIntoCharacter !?charValue !%%'\n'%% ;\\ 
  **when** %%'r'%% :\\ 
   enterCharacterIntoCharacter !?charValue !%%'\r'%% ;\\ 
  **when** %%'t'%% :\\ 
   enterCharacterIntoCharacter !?charValue !%%'\t'%% ;\\ 
  **when** %%'v'%% :\\ 
   enterCharacterIntoCharacter !?charValue !%%'\v'%% ;\\ 
  **when** %%'\\'%% :\\ 
   enterCharacterIntoCharacter !?charValue !%%'\\'%% ;\\ 
  **when** %%'0'%% :\\ 
   enterCharacterIntoCharacter !?charValue !%%'\0'%% ;\\ 
  **when** %%'\''%% :\\ 
   enterCharacterIntoCharacter !?charValue !%%'\''%% ;\\ 
  **when** %%'0' -> '9'%% :\\ 
   **repeat**\\ 
    enterHexDigitIntoASCIIcharacter !?charValue !* **error** ASCIIcodeTooLargeError, internalError ;\\ 
   **while** %%'0' -> '9'%% :\\ 
   **end repeat** ;\\ 
  **default**\\ 
   **error** incorrectCharConstant ;\\ 
  **end select** ;\\ 
 **when** %%' ' -> '\uFFFD'%% :\\ 
  enterCharacterIntoCharacter !?charValue !* ;\\ 
 **default**\\ 
  **error** incorrectCharConstant ;\\ 
 **end select** ;\\ 
 **select**\\ 
 **when** %%'\''%% :\\ 
  **send** \$literal\_char\$ ;\\ 
 **default**\\ 
  **error** incorrectCharConstant ;\\ 
 **end select** ;\\ 
**end rule** ;''|

===== Scanning string constant =====

===== Scanning Floating Point constant =====

|''\$literal\_double\$ !floatValue !tokenString **error message** %%"a float number"%%;\\ 
\\ 
\$.\$ **error message** %%"the '.' delimitor"%%;\\ 
\\ 
**message** floatNumberConversionError : %%"invalid float number"%% ;\\ 
\\ 
**rule** %%'.'%% :\\ 
 **select**\\ 
 **when** %%'0'->'9'%% :\\ 
  enterCharacterIntoString !?tokenString !%%'0'%% ;\\ 
  enterCharacterIntoString !?tokenString !%%'.'%% ;\\ 
  enterCharacterIntoString !?tokenString !* ;\\ 
  **repeat**\\ 
  **while** %%'0'->'9'%% :\\ 
   enterCharacterIntoString !?tokenString !* ;\\ 
  **while** %%'\_'%% :\\ 
  **end repeat** ;\\ 
  convertStringToDouble !tokenString !?floatValue **error** floatNumberConversionError ;\\ 
  **send** \$literal\_double\$ ;\\ 
 **default**\\ 
  **send** \$.\$ ;\\ 
 **end select** ;\\
**end rule** ;''|

===== Back tracking using tag and rewind instructions =====

|Available in GALGAS 1.5.6 and later.|

The ''**tag**'' and ''**rewind**'' instructions can be used for performing back tracking.

The first example is the way the non terminal symbols are scanned in GALGAS 1.5.6 (and later).

A non terminal is composed of a single '<' character, followed by a letter, zero, one or more letters, digits or underscore characters, is ended by a single '>' character. For example ''<abcdef>'' is a valid non terminal. However, ''<abcdef >'' is //not// a valid non terminal (because of the space before the final '>' character): it is considered as a '<' delimitor, followed by the ''abcdef'' identifier and by the '>' delimitor.

In the file ''galgas/galgas\_sources/galgas\_scanner.ggs'', the three delimitors befgging with a '<' character and the non terminal symbols are scanned by the following code:

''\$<\$ **error message** "the '<' delimitor" **style** delimitersStyle ;''\\
''%%\$<=\$%% **error message** "the '<=' delimitor" **style** delimitersStyle ;''\\
''%%\$<<\$%% **error message** "the '<<' delimitor" **style** delimitersStyle ;''\\
''\$non\_terminal\_symbol\$ ! tokenString **error message** "a non terminal symbol <...>" **style** nonTerminalStyle ;''\\

''**rule** '<' :''\\
'' **tag** onlyInfDelimiter ;''\\
'' **select**''\\
'' **when** '=' :''\\
'' **send** %%\$<=\$%% ;''\\
'' **when** '<' :''\\
''  **send** %%\$<<\$%% ;''\\
'' **when** %%'a' -> 'z' | 'A' ->'Z'%% :''\\
''  **repeat**''\\
''   enterCharacterIntoString !?tokenString !* ;''\\
''  **while** %%'a' -> 'z' | 'A' ->'Z' | '0' -> '9' | '\_'%% :''\\
''  **end repeat** ;''\\
''  **select**''\\
''  **when** '>' :''\\
''   **send** \$non\_terminal\_symbol\$ ;''\\
''  **default**''\\
''   **rewind** onlyInfDelimiter **send** \$<\$ ;''\\
''  **end select** ;''\\
'' **default**''\\
''  **send** \$<\$ ;''\\
'' **end select** ;''\\
''**end rule** ;''\\

The ''**tag**'' instruction records a scanning location. When the final '>' character is not found, the scanner is rewinded at the character following the '<' character, and the ''\$<\$'' terminal is sent. On next scanning, an identifier (or a key word) will be found.

The second examples shows how to scan for integer constants, float constants, and array bounds in Pascal :
  * an integer constant is a (non empty) sequence of digits ;
  * a float constant is a (non empty) sequence of digits, following by a dot and a (possibly empty) sequence of digits;
  * an array bound is an integer constant, followied by the '..' delimitor (two dots) and an integer constant.

The problem is that ''1..2'' should not be scanned as a float constant, a single dot delimitor, and an integer constant.

This can be achieved by the following code:

''**rule** %%'0' -> '9'%% :''\\
'' **repeat**''\\
'' **while** %%'0' -> '9'%% :''\\
'' **end repeat** ;''\\
'' **tag** endOfIntegerConstant ;''\\
'' **select**''\\
'' **when** %%'.'%% :''\\
''  **select**''\\
''  **when** %%'.'%% :''\\
''   **rewind** endOfIntegerConstant **send** \$integer\_constant\$ ;''\\
''  **when** %%'0' -> '9'%% :''\\
''   **repeat**''\\
''   **while** %%'0' -> '9'%% :''\\
''   **end repeat** ;''\\
''   **send** \$float\_constant\$ ;''\\
''  **default**''\\
''   **send** \$float\_constant\$ ;''\\
''  **end select** ;''\\
'' **default**''\\
''  **send** \$integer\_constant\$ ;''\\
'' **end select** ;''\\
''**end rule** ;''\\


====== Adding syntax coloring (Mac OS X Only) ======

With GALGAS, you can easily embbed your compiler in a GUI application (currently available only for Mac OS X). This application has a built-in text editor, from which you can modify, save and compile source file. With //style declarations//, you can add automatic coloring in the built-in text editor.

A //style declaration// associates a message to a style identifier. For example:

|''**style** keywordsStyle %%->%% %%"%%Keywords:%%"%% ;''|

The associated message is used in application preferences window as a comment of each color selection item.

A //style declaration// does not link a style identifier to any terminal symbol. You need to add this information to //single terminal symbol declaration// and //list of terminal symbols declaration// by naming the style identifier after the syntax error message:

|''\$literal\_integer\$ **error** **message** %%"%%a decimal number%%"%% **style** integerStyle;''|

|''**list** delimitorList **error** **message** %%"%%the '%%"%% . * . %%"%%' delimitor%%"%% **style** keywordsStyle: %%"%%.%%"%%, %%"%%;%%"%%, %%"%%(%%"%%, %%"%%)%%"%%;''|

===== Example: Styles provided by GALGAS Scanner =====

As an example, you can take a look on GALGAS scanner, in ''galgas/galgas\_sources/galgas\_scanner.ggs'' file. The style declarations are the following:

|''**style** keywordsStyle %%->%% %%"%%Keywords:%%"%% ;\\ **style** delimitersStyle %%->%% %%"%%Delimiters:%%"%% ;\\ **style** terminalStyle %%->%% %%"%%Terminal symbols:%%"%% ;\\ **style** integerStyle %%->%% %%"%%Integer constants:%%"%% ;\\ **style** characterStyle %%->%% %%"%%Character constants:%%"%% ;\\ **style** stringStyle %%->%% %%"%%String constants:%%"%% ;\\ **style** typeNameStyle %%->%% %%"%%Type names (@...):%%"%%'';|

You can search for the occurrence of style identifiers, to see how they are used.

In Cocoa GALGAS application, the Color tab of the Preferences window lists all style comments, each of them being associated to a ''NSColorWell'' for color selection:

{{cocoa\_galgas\_color\_styles.png}}

Note that no default color is defined in style declaration. Until you define yourself a color from Preference window, it defaults to black color.

===== Coloring comments =====

|Available in GALGAS 1.5.6 and later.|

In GALGAS 1.5.6 and later, you can define a color for comments. Proceed as follows:
  - declare a new terminal symbol, for example ''\$comment\$'';
  - declare a style for this new terminal symbol;
  - when a comment is scanned, use the ''**drop**'' instruction for naming the new terminal symbol (instead of the usual ''**send**'' instruction).

The ''**drop**'' instruction is only significant for syntax coloring.

For example, GALGAS comments are defined in ''galgas/galgas\_sources/galgas\_scanner.ggs'' in this way:

''**style** commentStyle %%->%% "Comments:" ;''\\
''...''\\
''\$comment\$ **error** **message** %%"%%a comment%%"%% **style** commentStyle ;''\\
''**rule** %%'\#'%% :''\\
'' **repeat**''\\
'' **while** %%'\u0001' -> '\u0009' | '\u000B' | '\u000C' | '\u000E' -> '\uFFFD'%% :''\\
'' **end repeat** ;''\\
'' **drop** \$comment\$ ;''\\
''**end rule** ;''\\

%!TEX encoding = UTF-8 Unicode
%!TEX root = ../galgas-book.tex

%--------------------------------------------------------------
\chapterLabel{Le composant \texttt{option}}{composantOption}
%-------------------------------------------------------------


Le composant \galgas{option} permet de définir des options qui sont appelables à partir de la ligne de commande. Dans le code, la valeur d'une option est obtenue à partir de l'opérande \emph{appel d'une option}, décrit dans la \refSubsectionPage{appelOption}.

Voici l'exemple d'un composant \galgas{option} qui déclare une option (évidement, un composant \galgas{option} peut déclarer un nombre quelconque d'options) :
\begin{galgascode}
option nom_composant {
  @bool nom_option : 'S', "asm" -> "Extract assembly code"
}
\end{galgascode}


\section{Déclaration d'une option}

La déclaration d'une option présente le syntaxe suivante :
\begin{galgascode}
  @T nom_option : caractere, chaine -> description
\end{galgascode}

Les cinq champs qui définissent une option sont :
\begin{itemize}
  \item \galgas{@T} : le type de l'option ; trois types sont autorisés : \galgas{@bool}, \galgas{@uint} et \galgas{@string} ;
  \item \galgas{nom_option} : c'est le nom, interne à GALGAS, qui permettra de désigner l'option dans l'\emph{appel d'une option} (\refSubsectionPage{appelOption}) ; 
  \item \galgas{caractere} : le caractère qui activera l'option dans la ligne de commande ; par exemple, en écrivant \galgas{'A'}, l'option sera activée par \texttt{-A} dans la ligne de commande ; si vous ne voulez pas d'activation par un caractère, écrivez \galgas{'\\0'} ;
  \item \galgas{chaine} : la chaîne de caractères qui activera l'option dans la ligne de commande ; par exemple, en écrivant \galgas{"ABEDEF"}, l'option sera activée par \texttt{-{}-ABCDEF} dans la ligne de commande ; si vous ne voulez pas d'activation par une chaîne, écrivez \galgas{""} ;
  \item \galgas{description} : une chaîne de caractère qui contient une description de l'option, qui sera affichée par l'option \texttt{-{}-help} de votre compilateur.
\end{itemize}








\section{Option booléenne}

Le champ qui définit le type de l'option est \galgas{@bool} ; par exemple :
\begin{galgascode}
  @bool nom_option : 'S', "asm" -> "Extract assembly code"
\end{galgascode}

Dans la ligne de commande, l'option est activée par \texttt{-A} ou \texttt{-{}-asm}.

Par défaut, l'option n'est pas activée, et sa valeur associée est \galgas{false}. Quand l'option est activée dans la ligne de commande, sa valeur associée est \galgas{true}.








\section{Option entière}

Le champ qui définit le type de l'option est \galgas{@uint} ; par exemple :
\begin{galgascode}
  @uint nom_option : 'M', "max-iterations-count" -> "Max of iteration count"
\end{galgascode}

Dans la ligne de commande, l'option est activée par \texttt{-N=xxx} ou \texttt{-{}-max-iterations-count=xxx}, où \texttt{xxx} est un nombre entier positif ou nul (et inférieur ou égal à $2^{32}-1$).

Par défaut, l'option n'est pas activée, et sa valeur associée est $0$. Quand l'option est activée dans la ligne de commande, sa valeur associée est la valeur \texttt{xxx}. Ainsi, l'option \texttt{-N=0}, comme l'option \texttt{-{}-max-iterations-count=0} n'a aucun effet.










\section{Option chaîne de caractères}

Le champ qui définit le type de l'option est \galgas{@string} ; par exemple :
\begin{galgascode}
  @string nom_option : 'F', "file-name" -> "File name"
\end{galgascode}

Dans la ligne de commande, l'option est activée par \texttt{-F=abc} ou \texttt{-{}-file-name=abc}, où \texttt{abc} est une chaîne de caractères sans espaces. Si vous voulez entrer une chaîne de caractères qui comprend des espaces, écrivez : \texttt{"-F=abc"} ou \texttt{"-{}-file-name=abc"}.

Par défaut, l'option n'est pas activée, et sa valeur associée est la chaîne vide. Quand l'option est activée dans la ligne de commande, sa valeur associée est la chaîne \texttt{abc}. Ainsi, l'option \texttt{-F=}, comme l'option \texttt{-{}-file-name=} n'a aucun effet.





%!TEX encoding = UTF-8 Unicode
%!TEX root = ../galgas-book.tex

%--------------------------------------------------------------
\chapter{Types de base}\label{predefinedTypes}
%-------------------------------------------------------------

GALGAS predefines several types. This chapter presents all their features, including their constructors, getters, setters, methods, ...


Les types prédéfinis sont :
\begin{itemize}
\item \ggst+@binaryset+, binary set objects (implemented with Binary Decision Diagrams);
\item \ggst+@bool+, boolean objects;
\item \ggst+@char+, Unicode characters;
\item \ggst+@double+, floating point numbers;
\item \ggst+@filewrapper+, dont les objets permettent d'explorer les \emph{filewrappers} ;
\item \ggst+@location+, whose value points out a location in a source file;
\item \ggst+@object+, dont une instance peut encapsuler toute valeur ;
\item \ggst+@range+, intervalle d'entiers 32 non signés ;
\item \ggst+@sint+, the 32-bit signed integers;
\item \ggst+@sint64+, the 64-bit signed integers;
\item \ggst+@string+, the Unicode string objects;
\item \ggst+@stringset+, set of \ggst+@string+ objects;
\item \ggst+@type+, dont une instance représente un type ;
\item \ggst+@uint+, the 32-bit unsigned integers;
\item \ggst+@uint64+, the 64-bit unsigned integers.
\end{itemize}



%!TEX encoding = UTF-8 Unicode
%!TEX root = ../galgas-book.tex

%--------------------------------------------------------------
\chapter{User Types} \label{userTypes}
%-------------------------------------------------------------


\section{List type}

\subsection{List Type Declaration}

A \lstinline[language=galgas]!list! type declaration names all attributes of the list elements:

\begin{lstlisting}[language=galgas]
list @MyList {
  @string mFirstAttribute ;
  @bool mSecondAttribute ;
}
\end{lstlisting}

\subsection{Constructors}

\subsubsection{The \lstinline[language=galgas]!emptyList! constructor}

For every list, an \lstinline[language=galgas]!emptyList! constructor is implicitly declared. It returns an empty list:

\begin{lstlisting}[language=galgas]
@MyList aList := [@MyList emptyList] ;
\end{lstlisting}


\subsubsection{The \lstinline[language=galgas]!listWithValue! constructor}

A list can be constructed directly with one value:

\begin{lstlisting}[language=galgas]
@MyList aList := [@myList listWithValue !"c" !3] ;
\end{lstlisting}


Using this constructor is equivalent to:

\begin{lstlisting}[language=galgas]
@MyList aList := [@MyList emptyList] ;
aList += !"c" !3 ;
\end{lstlisting}

\subsection{Adding elements}

\subsubsection{The \lstinline[language=galgas]!+=! operator}

The  \lstinline[language=galgas]!+=! operator adds a new element at the end of the list. The right side expressions should correspond to the attributes declared in the \lstinline[language=galgas]!list! declaration:\\

\begin{lstlisting}[language=galgas]
@MyList aList := ... ;
@string aString := ... ;
@bool aBool := ... ;
aList += !aString !aBool ;''
\end{lstlisting}


\subsubsection{The \lstinline[language=galgas]!.=! operator}

The \lstinline[language=galgas]!.=! operator concats a list at the end of the target list:

\begin{lstlisting}[language=galgas]
@MyList aList := ... ;
@MyList secondList := ... ;
aList .= secondList ;''
\end{lstlisting}



\subsubsection{The \lstinline[language=galgas]!prependValue! modifier}

The \lstinline[language=galgas]!prependValue! modifier adds a new element at the begining of the list. The right side expressions should correspond to the attributes declared in the  \lstinline[language=galgas]!list! declaration:

\begin{lstlisting}[language=galgas]
@MyList aList := ... ;
@string aString := ... ;
@bool aBool := ... ;
[!?aList prependValue !aString !aBool];
\end{lstlisting}

\subsubsection{The concatenation operator}

The «~\lstinline[language=galgas]!.!~» operator can be used fot concatenating two lists of the same type:


\begin{lstlisting}[language=galgas]
@MyList firstList := ... ;
@MyList secondList := ... ;
@MyList thirdList := firstList . secondList ;
\end{lstlisting}

\subsection{Removing elements}

\subsubsection{The \lstinline[language=galgas]!popFirst! modifier}


The \lstinline[language=galgas]!popFirst! modifier removes and returns the first element of the list. The right side expressions should correspond to the attributes declared in the \lstinline[language=galgas]!list! declaration:\\

\begin{lstlisting}[language=galgas]
@MyList aList := ... ;
@string aString ;
@bool aBool ;
[!?aList popFirst ?aString ?aBool];
\end{lstlisting}

If the list is empty when \lstinline[language=galgas]!popFirst! modifier is invoked, a run-time error is raised and the input arguments are not valuated.

\subsubsection{The \lstinline[language=galgas]!popLast! modifier}


The \lstinline[language=galgas]!popLast! modifier removes and returns the last element of the list. The right side expressions should correspond to the attributes declared in the \lstinline[language=galgas]!list! declaration:

\begin{lstlisting}[language=galgas]
@MyList aList := ... ;
@string aString ;
@bool aBool ;
[!?aList popLast ?aString ?aBool];
\end{lstlisting}

If the list is empty when \lstinline[language=galgas]!popLast! is invoked, a run-time error is raised and the input arguments are not valuated.

\subsection{Methods}

\subsubsection{The \lstinline[language=galgas]!first! method}

The \lstinline[language=galgas]!first! method returns the first element of the list. The element is not removed. The right side expressions should correspond to the attributes declared in the \lstinline[language=galgas]!list! declaration:

\begin{lstlisting}[language=galgas]
@MyList aList := ... ;
@string aString ;
@bool aBool ;
[aList first ?aString ?aBool];
\end{lstlisting}

If the list is empty when \lstinline[language=galgas]!first! is invoked, a run-time error is raised and the input arguments are not valuated.

\subsubsection{The \lstinline[language=galgas]!last! method}

The \lstinline[language=galgas]!last! method returns the last element of the list. The element is not removed. The right side expressions should correspond to the attributes declared in the \lstinline[language=galgas]!list! declaration:\\

\begin{lstlisting}[language=galgas]
@MyList aList := ... ;
@string aString ;
@bool aBool ;
[aList last ?aString ?aBool];
\end{lstlisting}


If the list is empty when \lstinline[language=galgas]!last! is invoked, a run-time error is raised and the input arguments are not valuated.








\subsection{Readers}

\subsubsection{The \lstinline[language=galgas]!length! reader}

\begin{lstlisting}[language=galgas]
reader length -> @uint ;
\end{lstlisting}

The \lstinline[language=galgas]!length! reader returns the number of elements in the receiver's value.




\subsubsection{The \lstinline[language=galgas]!subListFromIndex! reader}

\begin{lstlisting}[language=galgas]
reader subListFromIndex ?@uint inIndex -> @self
\end{lstlisting}

This reader returns a new list containing the elements of the receiver from the one at a given index to the end. The  \lstinline[language=galgas]!inIndex! value should be lower or equal to the length of the receiver's value. If \lstinline[language=galgas]!inIndex! is equal to the length of the receiver, the reader returns an empty list.





\subsubsection{The \lstinline[language=galgas]!subListWithRange! reader}

\begin{lstlisting}[language=galgas]
reader subListWithRange
  ?@uint inStartIndex
  ?@uint inCount
  -> @self
\end{lstlisting}

This reader returns a list containing the elements of the receiver that lie within a given range. The range must not exceed the length of the receiver's value, that is $inStartIndex + inCount \leqslant list\_length$. If \lstinline[language=galgas]!inCount! value is equal to zero, this reader returns an empty list.





\subsection{Enumerating a list with a foreach instruction}

The \lstinline[language=galgas]!foreach! instruction can be used for enumerating list objects. By default, lists are enumerated in the insertion order; enumeration in the reverse order is performed using the «~\lstinline[language=galgas]!>!~» qualifier.

There are two ways for accessing element values:
\begin{itemize}
\item using the implicitly declared constants that receive the current attribute values;
\item declare explicitly constants that receive the current attribute values.
\end{itemize}

Given the list declaration:

\begin{lstlisting}[language=galgas]
list @MyList {
  @string mFirstAttribute ;
  @bool mSecondAttribute ;
}
\end{lstlisting}

\subsubsection{Enumeration using the implicitly declared constants}

For every attribute, a constant of the same name is available in the \lstinline[language=galgas]!do! instruction list. Theses constants receive the value of the corresponding attribute of the current element.

\begin{lstlisting}[language=galgas]
foreach aList do
  # the mFirstAttribute constant receives the value
  # of the mFirstAttribute attribute of the current element,
  # and the mSecondAttribute constant receives the value
  # of the mSecondAttribute attribute of the current element.
end foreach ;
\end{lstlisting}

\subsubsection{Enumeration using the explicitly declared constants}

The \lstinline[language=galgas]!foreach! header declares a sequence of constants, corresponding to the attribute list of the \lstinline[language=galgas]!do! declaration. Theses constants receive the value of the corresponding attribute of the current element.


\begin{lstlisting}[language=galgas]
foreach aList (@string kString @bool kBool) do
  # the kString constant receives the value
  # of the mFirstAttribute attribute of the current element,
  # and the kBool constant receives the value
  # of the mSecondAttribute attribute of the current element.
end foreach ;
\end{lstlisting}

\subsubsection{Enumeration in the reverse order}

In GALGAS 1.7.3 and later, you can enumerate a list in the reverse order using the «~\lstinline[language=galgas]!>!~» qualifier:

\begin{lstlisting}[language=galgas]
foreach > aList (@string kString @bool kBool) do
  ...
end foreach ;
\end{lstlisting}




\subsection{Direct Access of an element attribute}

In GALGAS 1.7.5 and later, lists can be used as an array. Each element of a list is associated with an \nomType{uint} index, spanning from 0 to element count (value returned by \lstinline[language=galgas]!length! reader) minus one.

The element retrieved with \lstinline[language=galgas]!first! method is at index 0.

The element retrieved with \lstinline[language=galgas]!last! method is at index equal to element count minus one.

\subsubsection{Read Access}

By default and for every attribute, a reader is provided to retrieve the value of this attribute for an element at a given index. For example, for an attribute named \emph{name}, the \emph{nameAtIndex} reader is provided. It accepts one \nomType{uint} argument, the value of the index.

You can disable the default reader generation, by using the «~\lstinline[language=galgas]!feature nogetter!~» qualifier.

For example:
\begin{lstlisting}[language=galgas]
list @MyList {
  @string mFirstAttribute ;
  @bool mSecondAttribute feature nogetter ;
}
...
@MyList aList := ... ;
@string s := [aList mFirstAttributeAtIndex !1] ;
\end{lstlisting}

One reader is available: \lstinline[language=galgas]!mFirstAttributeAtIndex!; the \lstinline[language=galgas]!mSecondAttributeAtIndex! reader is not available.


\subsubsection{Write Access}

By default, no modifier is provided for performing a direct write access to an attribute at a given index. You should use the «~\lstinline[language=galgas]!feature setter!~» qualifier for enabling setter generation for a given attribute.

The modifier name is the name of the attribute with the first letter capitalized, prefixed by \emph{set} and suffixed by \emph{AtIndex}: for an attribute named \emph{name}, the modifier is named \emph{setNameAtIndex}. It accepts two arguments, the first one is the new attribute's value, the second one an \nomType{uint} argument, the value of the index.

For example:

\begin{lstlisting}[language=galgas]
list @MyList {
  @string mFirstAttribute feature setter ;
  @bool mSecondAttribute ;
}
...
@string s := ... ;
[!?aList setMFirstAttributeAtIndex !s !1] ;
\end{lstlisting}

One modifier is available: \lstinline[language=galgas]!setMFirstAttributeAtIndex!; the \lstinline[language=galgas]!setMSecondAttributeAtIndex! modifier is not available.

\subsubsection{Example of read and write accesses}

\begin{lstlisting}[language=galgas]
list @myList {
  @string name ;
}
...
@myList strList [emptyList] ;
strList += !"a" ;
strList += !"b" ;
strList += !"c" ;
strList += !"d" ;
@string s := [strList nameAtIndex !0] ;
log s ; # displays LOGGING s: <@string:"a">
s := [strList nameAtIndex !1] ;
log s ; # displays LOGGING s: <@string:"b">
s := [strList nameAtIndex !2] ;
log s ; # displays LOGGING s: <@string:"c">
s := [strList nameAtIndex !3] ;
log s ; # displays LOGGING s: <@string:"d">
[!?strList setNameAtIndex !"x" !0] ;
[!?strList setNameAtIndex !"y" !1] ;
[!?strList setNameAtIndex !"z" !2] ;
[!?strList setNameAtIndex !"t" !3] ;
s := [strList nameAtIndex !0] ;
log s ; # displays LOGGING s: <@string:"x">
s := [strList nameAtIndex !1] ;
log s ; # displays LOGGING s: <@string:"y">
s := [strList nameAtIndex !2] ;
log s ; # displays LOGGING s: <@string:"z">
s := [strList nameAtIndex !3] ;
log s ; # displays LOGGING s: <@string:"t">
\end{lstlisting}















\section{Sorted list type}


\section{Struct type}



\section{Class type}


\section{Map type}


\section{Map proxy type}




\section{Graph type}

\newpage
\section{Array type}

Le type \emph{array} est en projet, et pourra être implémenté dans la prochaine version de GALGAS. Il permet de réaliser des tableaux de dimensions et de types fixés à la compilation.

\subsection{Déclaration d'un type tableau}

La déclaration d'un type tableau contient les informations suivantes :
\begin{itemize}
  \item le type \lstinline[language=galgas]!@TypeElement! qui cite le type de l'élément de tableau ;
  \item la dimension du tableau, qui doit être un nombre entier strictement positif ;
  \item le type \lstinline[language=galgas]!@TypeTableau! qui est le nom donné au type de tableau.
\end{itemize}

La déclaration d'un type tableau a la syntaxe suivante :
\begin{lstlisting}[language=galgas]
array @TypeTableau : @TypeElement [dimension] ;
\end{lstlisting}

Par exemple :
\begin{lstlisting}[language=galgas]
array @monTableau : @string [3] ;
\end{lstlisting}


\subsection{Constructeur d'un type tableau}

Le seul constructeur d'un type tableau est le constructeur \lstinline[language=galgas]!new!. Il a pour but de fixer les dimensions initiales du tableau (il pourra ensuite être redimensionné). Il comporte \lstinline[language=galgas]!dimension! arguments de type \lstinline[language=galgas]!@uint!, qui fixent l'amplitude initiale de chaque axe.
Par exemple :
\begin{lstlisting}[language=galgas]
  @monTableau t [new !2 !3 !4] ;
\end{lstlisting}

Cette déclaration crée un tableau à $2*3*4$ éléments. Ces éléments sont par défaut \emph{invalides}, c'est à dire que leur lecture par le reader \lstinline[language=galgas]!valueAtIndex! déclenche une \emph{run-time error}. Pour être valide, un élément doit avoir été initialisé par un appel au modifier \lstinline[language=galgas]!setValueAtIndex!.

Il est valide d'affecter la valeur $0$ à un ou plusieurs axes. Le tableau ne contient alors aucun élément.


\subsection{Accès à un élément}

L'accès à la valeur d'un élément s'effectue par le reader \lstinline[language=galgas]!valueAtIndex!. La modification de la valeur d'un élément est réalisée par le modifier \lstinline[language=galgas]!setValueAtIndex! ou le modifier \lstinline[language=galgas]!forceValueAtIndex!.

\subsubsection{Le reader \lstinline[language=galgas]!valueAtIndex!}

Ce reader comporte \lstinline[language=galgas]!dimension! arguments de type \lstinline[language=galgas]!@uint!, qui précisent l'indice relatif à chaque axe. Les indices sont comptés à partir de zéro (comme en C).

Par exemple :
\begin{lstlisting}[language=galgas]
  @string s := [t valueAtIndex !1 !2 !2] ;
\end{lstlisting}


Une \emph{run-time error} est déclenchée si un indice dépasse sa borne correspondante, et la valeur retournée est \emph{invalide}. Si les indices ont des valeurs correctes, l'élément est retourné ; si cet élément est invalide, une \emph{run-time error} est déclenchée, et une valeur \emph{invalide} est retournée.






\subsubsection{Le modifier \lstinline[language=galgas]!setValueAtIndex!}

Ce modifier comporte (\lstinline[language=galgas]!dimension!+1) arguments :
\begin{itemize}
  \item le premier argument est type \lstinline[language=galgas]!@TypeElement!, et contient la valeur à écrire ;
  \item les \lstinline[language=galgas]!dimension! suivants arguments sont de type \lstinline[language=galgas]!@uint! et précisent l'indice relatif à chaque axe.
\end{itemize} 
  
Les indices sont comptés à partir de zéro (comme en C). Une \emph{run-time error} est déclenchée si un indice dépasse sa borne correspondante, et le tableau est alors non modifié.

Par exemple :
\begin{lstlisting}[language=galgas]
  @string s := ... ;
  [!?t setValueAtIndex !s !1 !2 !2] ;
\end{lstlisting}





\subsubsection{Le modifier \lstinline[language=galgas]!forceValueAtIndex!}

Ce modifier comporte (\lstinline[language=galgas]!dimension!+1) arguments :
\begin{itemize}
  \item le premier argument est type \lstinline[language=galgas]!@TypeElement!, et contient la valeur à écrire ;
  \item les \lstinline[language=galgas]!dimension! suivants arguments sont de type \lstinline[language=galgas]!@uint! et précisent l'indice relatif à chaque axe.
\end{itemize} 
  
Les indices sont comptés à partir de zéro (comme en C). Contrairement au modifier \lstinline[language=galgas]!setValueAtIndex!, aucune \emph{run-time error} n'est déclenchée si un indice dépasse sa borne correspondante : le tableau est d'abord agrandi, ce qui ajoute des éléments invalides, puis l'élément désigné par les indices est affecté.

Par exemple :
\begin{lstlisting}[language=galgas]
  @string s := ... ;
  [!?t forceValueAtIndex !s !5 !4 !4] ;
\end{lstlisting}





\subsection{Validité d'un élément}

Le reader \lstinline[language=galgas]!isValueValidAtIndex! permet de savoir si un élément est valide ou non, c'est à dire si sa lecture déclenchera une \emph{run-time error}. Le modifier \lstinline[language=galgas]!invalidateValueAtIndex! invalide un élément.

\subsubsection{Le reader \lstinline[language=galgas]!isValueValidAtIndex!}

Ce reader comporte \lstinline[language=galgas]!dimension! arguments de type \lstinline[language=galgas]!@uint!, qui précisent l'indice relatif à chaque axe. Les indices sont comptés à partir de zéro (comme en C). Une \emph{run-time error} est déclenchée si un indice dépasse sa borne correspondante, et la valeur retournée est \emph{invalide}. Il renvoie une valeur de type \lstinline[language=galgas]!@bool!, suivant que l'élément est valide ou non.

Par exemple :
\begin{lstlisting}[language=galgas]
  @bool b := [t isValueValidAtIndex !1 !2 !2] ;
\end{lstlisting}


\subsubsection{Le modifier \lstinline[language=galgas]!invalidateValueAtIndex!}

Ce modifier comporte \lstinline[language=galgas]!dimension! arguments de type \lstinline[language=galgas]!@uint!, qui précisent l'indice relatif à chaque axe. Les indices sont comptés à partir de zéro (comme en C). Une \emph{run-time error} est déclenchée si un indice dépasse sa borne correspondante. Il invalide l'élément correspondant, c'est dire qu'un appel au reader \lstinline[language=galgas]!valueAtIndex! pour lire cet élément déclenchera une \emph{run-time error}.

Par exemple :
\begin{lstlisting}[language=galgas]
  [!?t invalidateValueAtIndex !1 !2 !2] ;
\end{lstlisting}





\subsection{Contrôle des tailles des axes}

Le reader \lstinline[language=galgas]!axisCount! renvoie la dimension d'un tableau, c'est à dire le nombre de ces axes, le reader \lstinline[language=galgas]!sizeForAxis! renvoie la taille allouée à un axe particulier. Les modifiers \lstinline[language=galgas]!setSizeForAxis! et \lstinline[language=galgas]!setSize! permettent de modifier la taille d'un tableau.



\subsubsection{Le reader \lstinline[language=galgas]!axisCount!}

Ce reader sans argument renvoie un \lstinline[language=galgas]!@uint! qui contient le nombre d'axes d'un tableau. Comme ce nombre est fixé statiquement par la déclaration de type, la valeur retournée est toujours la même, pour toutes les objets d'un même type tableau.


Par exemple, pour la déclaration :
\begin{lstlisting}[language=galgas]
array @monTableau : @string [3] ;
\end{lstlisting}
Pour tous les objets de type \lstinline[language=galgas]!@monTableau!, l'appel au reader \lstinline[language=galgas]!axisCount! renvoie la valeur \lstinline[language=galgas]!3!.


\subsubsection{Le reader \lstinline[language=galgas]!sizeForAxis!}

Ce reader présente un argument de type \lstinline[language=galgas]!@uint! qui est l'indice de l'axe interrogé. Les axes sont numérotés à partir de zéro, c'est à dire que le premier axe a l'indice $0$, le deuxième l'indice $1$, \dots Une \emph{run-time error} est déclenchée si la valeur de l'argument est supérieure ou égale à la dimension du tableau, et la valeur renvoyée est invalide. Sinon, il renvoie un \lstinline[language=galgas]!@uint! qui contient la taille attribuée à l'axe correspondant.




\subsubsection{Le modifier \lstinline[language=galgas]!setSizeForAxis!}

Ce modifier permet de changer la taille d'un axe sans changer les tailles attribuées aux autres axes. Il présente deux arguments de type \lstinline[language=galgas]!@uint! :
\begin{itemize}
  \item le premier est la nouvelle taille ;
  \item le second est l'indice de l'axe concerné.
\end{itemize}

Les axes sont numérotés à partir de zéro, c'est à dire que le premier axe a l'indice $0$, le deuxième l'indice $1$, \dots Une \emph{run-time error} est déclenchée si la valeur de l'argument est supérieure ou égale à la dimension du tableau, et le tableau n'est pas modifié.
 
Diminuer la taille d'un axe fait disparaître des éléments, qui sont alors perdus. Si la nouvelle taille est zéro, le tableau est vidé de tous ses éléments.

Augmenter la taille fait apparaître de nouveaux éléments, qui sont invalides par défaut. Il faudra alors explicitement les initialiser individuellement par un appel au modifier \lstinline[language=galgas]!setValueAtIndex!.




\subsubsection{Le modifier \lstinline[language=galgas]!setSize!}

Ce modifier permet de changer les tailles de tous les axes. Il présente \lstinline[language=galgas]!@uint! arguments de type \lstinline[language=galgas]!@uint! qui contiennent les nouvelles tailles de chaque axe.

Diminuer la taille d'un axe fait disparaître des éléments, qui sont alors perdus. Si une des nouvelles tailles est zéro, le tableau est vidé de tous ses éléments.

Augmenter une taille fait apparaître de nouveaux éléments, qui sont invalides par défaut. Il faudra alors explicitement les initialiser individuellement par un appel au modifier \lstinline[language=galgas]!setValueAtIndex!.


\subsection{Comparaison}

Un type tableau implémente les opérateurs \lstinline[language=galgas]!=! et \lstinline[language=galgas]+!=+. L'égalité de deux tableaux est testé comme suit :
\begin{itemize}
  \item les tailles de chaque axe doivent être identiques ;
  \item les éléments doivent être identiques.
\end {itemize}

\newpage
\section{Predefined user types}

\subsectionTypePredefiniLabelIndex{lbool}

\subsectionTypePredefiniLabelIndex{lchar}

\subsectionTypePredefiniLabelIndex{ldouble}

\subsectionTypePredefiniLabelIndex{lsint}

\subsectionTypePredefiniLabelIndex{lsint64}

\subsectionTypePredefiniLabelIndex{lstring}

\subsectionTypePredefiniLabelIndex{luint}

\subsectionTypePredefiniLabelIndex{luint64}



%!TEX encoding = UTF-8 Unicode
%!TEX root = ../galgas-book.tex

%--------------------------------------------------------------
\chapter{Instructions sémantiques}
%-------------------------------------------------------------



%\sectionLabel{Cible}{ciblePour Instruction}
%
%\lstset{emph={cible}, emphstyle=\galgasEmphStyle}
%
%La notation \galgas{cible} apparaît dans plusieurs instructions :
%\begin{itemize}
%  \item l'instruction d'affectation (\refSectionPage{assignmentInstruction}) ;
%\end{itemize}
%
%Cette notation est décrite par le diagramme syntaxique de la \refFigure{}{diagrammeSyntaxiqueCible}.
%
%\begin{figure}[t]
%  \centering
%  \small
%  \begin{tikzpicture}[
%      point/.style={coordinate},
%      nonterminal/.style={rectangle, minimum size=6mm, very thick, draw=red!50!black!50, top color=white, bottom color=red!50!black!20, font=\it},
%      terminal/.style={rectangle, rounded corners=3mm, minimum size=6mm, very thick, draw=black!50, top color=white, bottom color=black!20, font=\ttfamily}
%    ]
%    \matrix[row sep=1mm,column sep=5mm] {
%      \node (a) [nonterminal] {cible} ; &
%      \node (start) {} ; &
%      \node (idf) [terminal] {identificateur} ; &
%      \node (p1)[point] {} ; &
%      \node (fleche) [terminal] {->} ; &
%      \node (idf2) [terminal] {identificateur} ; &
%      \node (p2)[point] {} ; &
%      \node (p3)[point] {} ; &
%      \node (end) {} ;\\
%    } ;
%    \draw [o-stealth] (start) -- (idf) ;
%    \draw [-stealth] (idf) -- (fleche) ;
%    \draw [-stealth] (fleche) -- (idf2) ;
%    \draw [-stealth] (idf2) -- (p2) ;
%    \draw [-stealth] (p1) -- +(0, 0.8) -| (p3) -- (end) ;
%    \draw [-stealth] (p2) -- +(0, -0.8) -| (p1) ;
%  \end{tikzpicture}
%  \caption{Diagramme syntaxique de \texttt{cible}}
%  \labelFigure{diagrammeSyntaxiqueCible}
%\end{figure}
%
%Un identificateur seul représente une variable. 

\section{Rôle du point-virgule «\texttt{;}»}

Le point-virgule n'est pas un terminateur d'instruction. Il représente une instruction vide. Aussi, il peut être utilisé en nombre quelconque entre deux instructions consécutives. Ainsi, on peut écrire :

\lstset{emph={instruction1, instruction2}, emphstyle=\galgasEmphStyle}
\begin{galgascode}
instruction1  instruction2
\end{galgascode}

Ou encore :
\begin{galgascode}
instruction1 ; instruction2
\end{galgascode}
\begin{galgascode}
instruction1 ;;;; instruction2
\end{galgascode}







\section{Instruction de déclaration de variable}

\lstset{emph={variable, expression, @type}, emphstyle=\galgasEmphStyle}

Une déclaration de variable peut citer une \galgas{expression} qui lui fournit sa valeur initiale. Dans le cas constraire, la variable est \emph{non construite} jusqu'à ce qu'une valeur lui soit affectée.

Les deux formes de déclaration sont donc :
\begin{itemize}
\item déclaration d'une variable \emph{non construite} : « \galgas{var @type variable} » ;
\item déclaration d'une variable \emph{construite} : « \galgas{var @type variable = expression} ».
\end{itemize}



La forme générale de déclaration d'une variable \emph{non construite} est :
\begin{galgascode}
var @type variable
\end{galgascode}


La forme générale de déclaration d'une variable \emph{construite} est :
\begin{galgascode}
var @type variable = expression
\end{galgascode}

\subsection{Déclaration « \texttt{var @type variable} »}

On peut alléger l'écriture en ommettant le mot clé \galgas{var} :
\begin{galgascode}
@type variable
\end{galgascode}


\subsectionLabel{Déclaration « \texttt{var @type variable = expression} »}{declarationVariableAvecExpression}


On peut alléger l'écriture en ommettant le mot clé \galgas{var} :
\begin{galgascode}
@type variable = expression
\end{galgascode}

On peut omettre le type (mais il faut alors garder le mot-clé \galgas{var}), à la condition que l'expression puisse fournir l'information de type. Par exemple, dans l'écriture suivante :
\begin{galgascode}
var @string s = "Hello"
\end{galgascode}

L'expression est une chaîne de caractères, dont le type est \galgas{@string}. On peut donc omettre l'annotation de type dans l'instruction :
\begin{galgascode}
var s = "Hello"
\end{galgascode}


Prenons un autre exemple, celui de l'initialisation d'une liste :
\begin{galgascode}
var @stringlist sl = @stringlist {}
\end{galgascode}

Le type est annoté de façon redondante, puisqu'il apparaît à la fois dans le membre de gauche et dans l'expression ; l'une des deux annotations peut être omise :
\begin{galgascode}
var sl = @stringlist {}
\end{galgascode}

Ou :
\begin{galgascode}
var @stringlist sl = {}
\end{galgascode}

Par contre, omettre les deux annotations ne permet pas de déduire le type de la variable : c'est donc une erreur détectée par le compilateur :
\begin{galgascode}
var sl = {} # Erreur : le type est indetermine
\end{galgascode}

Un dernier exemple, avec un constructeur :
\begin{galgascode}
var @location sl = @location.here
\end{galgascode}

On peut écrire :
\begin{galgascode}
var sl = @location.here
\end{galgascode}

Ou :
\begin{galgascode}
var @location sl = .here
\end{galgascode}

Mais en aucun cas :
\begin{galgascode}
var sl = .here # Erreur : le type est indetermine
\end{galgascode}





\section{Instruction de déclaration de constante}

La forme générale de cette instruction est la suivante :
\lstset{emph={nom, expression, @type}, emphstyle=\galgasEmphStyle}
\begin{galgascode}
let @type nom = expression
\end{galgascode}

Le mot clé \galgas{let} caractérise une déclaration de constante. L'annotation de type peut être omis si le type peut être déduit de l'expression, comme pour l'instruction de déclaration de variable (\refSubsectionPage{declarationVariableAvecExpression}). On peut donc reprendre les exemples de la section précédente :
\begin{galgascode}
let @string s = "Hello"
\end{galgascode}

L'expression est une chaîne de caractères, dont le type est \galgas{@string}. On peut donc omettre l'annotation de type dans l'instruction :
\begin{galgascode}
let s = "Hello"
\end{galgascode}


Prenons un autre exemple, celui de l'initialisation d'une liste :
\begin{galgascode}
let @stringlist sl = @stringlist {}
\end{galgascode}

Le type est annoté de façon redondante, puisqu'il apparaît à la fois dans le membre de gauche et dans l'expression ; l'une des deux annotations peut être omise :
\begin{galgascode}
let sl = @stringlist {}
\end{galgascode}

Ou :
\begin{galgascode}
let @stringlist sl = {}
\end{galgascode}

Par contre, omettre les deux annotations ne permet pas de déduire le type de la constante : c'est donc une erreur détectée par le compilateur :
\begin{galgascode}
let sl = {} # Erreur : le type est indetermine
\end{galgascode}

Un dernier exemple, avec un constructeur :
\begin{galgascode}
let @location sl = @location.here
\end{galgascode}

On peut écrire :
\begin{galgascode}
let sl = @location.here
\end{galgascode}

Ou :
\begin{galgascode}
let @location sl = .here
\end{galgascode}

Mais en aucun cas :
\begin{galgascode}
let sl = .here # Erreur : le type est indetermine
\end{galgascode}



















\sectionLabel{L'instruction d'affectation}{assignmentInstruction}

L'instruction d'affectation peut prendre plusieurs formes. La plus courante est :
\lstset{emph={expression, variable}, emphstyle=\galgasEmphStyle}
\begin{galgascode}
variable = expression
\end{galgascode}

Si \galgas{variable} est une instance de structure (\refChapterPage{type-structure}), on peut directement en modifier un champ en utilisant l'opérateur \galgas{.} :
\lstset{emph={expression, variable, champ}, emphstyle=\galgasEmphStyle}
\begin{galgascode}
variable.champ = expression
\end{galgascode}

Si \galgas{champ} est lui-même une structure, on peut accéder à l'un de ses champs (et ainsi de suite) :
\lstset{emph={expression, variable, champ, champ1}, emphstyle=\galgasEmphStyle}
\begin{galgascode}
variable.champ.champ1 = expression
\end{galgascode}













\sectionLabel{L'instruction \texttt{cast}}{instructionCast}

L'instruction \galgas{cast} permet simplement d'exprimer de manière élégante une série de tests de conversions polymorphiques inverses. Sa syntaxe est :

\lstset{emph={expression, nom1, nom2, conversion}, emphstyle=\galgasEmphStyle}
\begin{galgascode}
cast expression
case conversion @T1 nom1 :
  ...
case conversion @T2 nom2 :
  ...
else
  ...
end
\end{galgascode}

L'instruction accepte une ou plusieurs branches \galgas{case}, et zéro ou une branche \galgas{else}. \galgas{conversion} est soit \galgas{==}, soit \galgas{>=}. \galgas{nom1} et \galgas{nom2} sont des constantes dont le type est le type nommé dans la branche \galgas{case} qui la déclare, et dont la portée est limitée à cette branche \galgas{case}.

Lors de l'exécution, le type dynamique de \galgas{expression} est comparé successivement aux types (\galgas{@T1}, \galgas{@T2}) nommés dans les branches \galgas{case} ; dès que ce type dynamique est :
\begin{itemize}
  \item exactement la classe \galgas{@T} (\galgas{conversion} est \galgas{==}), 
  \item la classe \galgas{@T} ou de l'une de ses classes héritières (\galgas{conversion} est \galgas{>=}),
  \item une classe héritière de la classe \galgas{@T}, mais pas la classe \galgas{@T} (\galgas{conversion} est \galgas{>}),
\end{itemize}
alors la constante prend la valeur de \galgas{expression} et les instructions de la branche correspondante sont exécutées.

Si toutes les comparaisons échouent, la branche \galgas{else} est exécutée (si elle existe). La forme typique de cette instruction est donc :


\begin{galgascode}
cast expression
case >= @B v1 :
  ...
case >= @C v2 :
  ...
else
  message "conversion impossible"
end
\end{galgascode}

Note : si la variable \galgas{nom1} ou  \galgas{nom2} n'est pas utilisée dans la branche correspondante, une alerte est émise. Pour la supprimer, ne pas mentionner la variable en écrivant \galgas{case >= @T :}.









\sectionLabel{L'instruction d'ajout \texttt{+=}}{concatInstruction}

\lstset{emph={cible, expression, expression1, expressionN}, emphstyle=\galgasEmphStyle}
Cette instruction permet d'ajouter un élément à une collection ; elle présente deux syntaxes, suivant que l'élément à ajouter est défini par :
\begin{itemize}
  \item une expression : \galgas{cible += expression} ;
  \item une liste d'expressions : \galgas{cible += !expression1 ... !expressionN}.
\end{itemize}


La \galgas{cible} est une variable ou un champ de structure.

Cette instruction s'applique aux types suivants :
\begin{itemize}
  \item \galgas{@string} (\refSubsectionPage{instructionConcatString}) ;
  \item \galgas{@stringset} (\refSubsectionPage{instructionConcatStringset}) ;
  \item \galgas{list @T} (\refSubsectionPage{instructionConcatList}) ;
  \item \galgas{sortedlist @T} (\refSubsectionPage{instructionConcatSortedList}) ;
  \item \galgas{map @T} (\refSubsectionPage{instructionConcatMap}).
\end{itemize}


\subsectionLabel{Instruction d'ajout et le type \texttt{@string}}{instructionConcatString}

Sous la forme \galgas{cible += expression}, l'instruction permet de concaténer des chaînes de caractères :
\begin{galgascode}
var s = "abc"
s += "def" # s vaut "abcdef"
\end{galgascode}

La forme \galgas{cible += !expression1 ... !expressionN} n'est pas prise en compte pour le type \galgas{@string}.



\subsectionLabel{Instruction d'ajout et le type \texttt{@stringset}}{instructionConcatStringset}

Sous la forme \galgas{cible += expression}, l'instruction permet de concaténer d'effectuer l'union des ensembles de châines :
\begin{galgascode}
var strset1 = @stringset {!"a", !"b"} # Valeur : "a", "b"
var strset2 = @stringset {!"b", !"c"} # Valeur : "b", "c"
strset1 += strset2 # strset1 vaut "a", "b", "c"
\end{galgascode}

La forme \galgas{cible += !expression1 ... !expressionN} permet d'ajouter une chaîne à l'ensemble :
\begin{galgascode}
var strset1 = @stringset {!"a", !"b"} # Valeur : "a", "b"
strset1 += !"c" # strset1 vaut "a", "b", "c"
strset1 += !"b" # strset1 vaut "a", "b", "c"
\end{galgascode}



\subsectionLabel{Instruction d'ajout et les listes}{instructionConcatList}

Sous la forme \galgas{cible += expression}, l'instruction effectue une concaténation de listes : \galgas{cible} et \galgas{expression} doivent avoir le même type \galgas{list @T}, et l'\galgas{expression} est ajoutée à la fin de la \galgas{cible}.

\begin{galgascode}
var liste1 = @stringlist {!"a", !"b"}
var liste2 = @stringlist {!"c", !"d"}
liste1 += liste2 # liste1 vaut "a" "b" "c" "d"
\end{galgascode}



Sous la forme \galgas{cible += !expression1 ... !expressionN}, l'instruction ajoute un élément à la fin de \galgas{cible}. L'élément est défini par la liste des valeurs de ses champs.

Avec la liste :
\begin{galgascode}
list @maListe {
  @uint mProperty1
  @string mProperty2
}
\end{galgascode}

On a :

\begin{galgascode}
var liste = @maListe {}
liste += !2 !"a" # liste vaut contient un element 2, "a"
\end{galgascode}






\subsectionLabel{Instruction d'ajout et les listes triées}{instructionConcatSortedList}

Sous la forme \galgas{cible += expression}, l'instruction effectue une concaténation de listes : \galgas{cible} et \galgas{expression} doivent avoir le même type \galgas{list @T}, et chaque élément de \galgas{expression} est ajouté à la \galgas{cible} en respectant l'ordre de tri.

Avec la liste triée :
\begin{galgascode}
sortedlist @maListeTriee {
  @uint mProperty1
  @string mProperty2
}{
  mProperty1 <
}
\end{galgascode}

\begin{galgascode}
var liste1 = @maListeTriee {!3 !"a", !1 !"c"} # Valeur : (1, "c"), (3, "a")
var liste2 = @maListeTriee {!4 !"d", !2 !"b"} # Valeur : (2, "b"), (4, "d")
liste1 += liste2 # liste1 vaut (1, "c"), (2, "b"), (3, "a"), (4, "d")
\end{galgascode}



Sous la forme \galgas{cible += !expression1 ... !expressionN}, l'instruction ajoute un élément à la fin de \galgas{cible}. L'élément est défini par la liste des valeurs de ses champs.


On a :

\begin{galgascode}
var liste = @maListeTriee {}
liste += !2 !"a" # Valeur : (2, "a")
liste += !1 !"b" # Valeur : (1, "b"), (2, "a")
liste += !3 !"c" # Valeur : (1, "b"), (2, "a"), (3, "c")
\end{galgascode}







\subsectionLabel{Instruction d'ajout et les tables}{instructionConcatMap}

La forme \galgas{cible += expression} n'est pas prise en charge.

Sous la forme \galgas{cible += !expression1 ... !expressionN}, l'instruction ajoute un élément à la table \galgas{cible}. L'élément est défini par la clé et suivie de la liste des valeurs de ses champs.


Avec la table :
\begin{galgascode}
map @maTable {
  @uint mProperty1
  @string mProperty2
}
\end{galgascode}

on a :

\begin{galgascode}
var table = @maTable {}
@lstring clef = ...
table += !clef !2 !"a"
\end{galgascode}














\sectionLabel{Décrémentation \texttt{-{}-}}{decrementInstruction}

L'instruction de décrémentation s'applique aux types \refTypePredefini{sint}, \refTypePredefini{sint64}, \refTypePredefini{uint} et \refTypePredefini{uint64} ; sa syntaxe est la suivante :
\lstset{emph={variable, champ}, emphstyle=\galgasEmphStyle}
\begin{galgascode}
variable --
\end{galgascode}

Les champs de structure peuvent être décrémentés :
\begin{galgascode}
variable.champ --
\end{galgascode}

Une erreur d'exécution est déclenchée en cas de dépassement de capacité.






%---------------------------------------------------------------------------------------------------------------------------

\section{L'instruction \texttt{drop}}

La syntaxe de l'instruction \galgas{drop} est la suivante :
{\lstset{emph={variable}, emphstyle=\galgasEmphStyle}
\begin{galgascode}
drop variable, ...
\end{galgascode}
}

Chaque variable nommée est placée dans l'état \emph{non construit}.







%---------------------------------------------------------------------------------------------------------------------------

\section{L'instruction \texttt{error}}

L'instruction \galgas{error} permet de signaler une erreur à l'utilisateur. Elle est constituée de trois champs séparés par un double-point (\galgas{\:}) :

{\lstset{emph={localisation, message_erreur, variable1, variableN}, emphstyle=\galgasEmphStyle}
\begin{galgascode}
error localisation : message_erreur : variable1, ..., variableN
\end{galgascode}



Le champ \galgas{localisation} signale à l'utilisateur la position de l'erreur dans le texte source. C'est donc une expression de type \galgas{@location}, ou d'un type possédant un \emph{getter} sans argument nommé \galgas{location} et renvoyant un objet de type \galgas{@location} : c'est le cas de tous les types prédéfinis \refTypePredefini{luint}, \refTypePredefini{luint64}, \refTypePredefini{lsint}, \refTypePredefini{lsint64}, \refTypePredefini{lbool}, \refTypePredefini{lchar} et \refTypePredefini{lstring}. Le \galgas{message_erreur} est le message affiché à l'utilisateur : c'est donc une expression de type \galgas{@string}. Enfin, le troisième champ liste les variables \galgas{variable1}, ..., \galgas{variableN} qui ne peuvent être construites du fait de l'erreur. Si il n'y a pas de variable à citer, l'instruction s'écrit :
\begin{galgascode}
error localisation : message_erreur
\end{galgascode}

Par exemple :

\begin{galgascode}
$identifier$ ??@lstring nom
...
error nom.location : message_erreur
\end{galgascode}

Comme \galgas{nom} est de type \galgas{lstring} (voir ci-dessus), on peut simplement écrire :
\begin{galgascode}
$identifier$ ??@lstring nom
...
error nom : message_erreur
\end{galgascode}


Lister des variables qui ne peuvent pas être construites est indispensable dans certains cas. Examinons le code suivant (\underline{qui ne compile pas}) :
\begin{galgascode}
$identifier$ ??@lstring nom
@unType resultat
if ok then
  resultat = ...
else
  error nom : message_erreur
end # Erreur : 'resultat' est value par une branche
\end{galgascode}

En effet, une des branches du \galgas{if} donne une valeur à \galgas{resultat}, et l'autre pas. Or, en cas d'erreur, on veut que \galgas{resultat} ne soit pas valué à l'exécution. On écrit alors le texte suivant (qui compile) :
\begin{galgascode}
$identifier$ ??@lstring nom
@unType resultat
if ok then
  resultat = ...
else
  error nom : message_erreur : resultat
end
\end{galgascode}

Mentionner \galgas{resultat} à la fin de l'instruction \galgas{error} permet de faire croire au compilateur que \galgas{resultat} est valué.


%---------------------------------------------------------------------------------------------------------------------------

\sectionLabel{L'appel de procédure}{appelRoutine}

Cette instruction permet d'exécuter une procédure. Si par exemple celle-ci est définie par :
\begin{galgascode}
proc maRoutine !@uint a ?!@string b {
  ...
}
\end{galgascode}

L'instruction d'appel de cette routine est (il y a plusieurs variantes possibles pour le premier paramètre qui est en entrée) :
\begin{galgascode}
@string x = ...
maRoutine (?@uint y !?x)
\end{galgascode}

Note : les parenthèses sont obligatoires, même si il n'y a aucun argment.

La correspondance entre arguments formels et paramètres effectifs est décrite à la \refSectionPage{correspondanceArgFormelsParametresEffectifs}.



%---------------------------------------------------------------------------------------------------------------------------

\sectionLabel{L'instruction \texttt{for}}{instructionFor}

L'instruction \galgas{for} permet d'énumérer :
\begin{itemize}
  \item une collection ;
  \item plusieurs collections de manière synchrone.
\end{itemize}

Pour énumérer une collection, la syntaxe est la suivante :

\lstset{emph={expression, sens, condition, nom_index, instructions_before, instructions_between, instructions_after, instructions\_do, enumeration_collection, enumeration_collection1, enumeration_collection2}, emphstyle=\galgasEmphStyle}
\begin{galgascode}
for enumeration_collection
while condition # Optionnel
before instructions_before  # Optionnel
do 
  (nom_index) # Optionnel
  instructions_do
between instructions_between  # Optionnel
after instructions_after  # Optionnel
end
\end{galgascode}


Énumérer plusieurs collections s'exprime en séparant les différentes énumérations par une virgule :
\begin{galgascode}
for enumeration_collection1, enumeration_collection2, ...
while condition # Optionnel
before instructions_before  # Optionnel
do
  (nom_index) # Optionnel
  instructions_do
between instructions_between  # Optionnel
after instructions_after  # Optionnel
end
\end{galgascode}


\subsection{Les quatre formes d'une énumération}

Le \refTableau{quatreFormesEnumeration} liste les quatre façons d'exprimer l'énumération \galgas{enumeration_collection}.


\lstset{emph={enumeration_collection, expression, sens, prefixe, cst, cst1, cst2}, emphstyle=\galgasEmphStyle}
\begin{table}[t]
  \centering
  \begin{tabular}{lp{8.5cm}}
  \textbf{Énumération} & \textbf{Signification}\\
  \galgas{sens () in expression} & Utilisation de constantes définies implicitement qui représentent les champs de l'élément courant (\refSubsectionPage{enumerationImplicite}).\\
  \galgas{sens () prefixe in expression} & Utilisation de constantes préfixées définies implicitement qui représentent les champs de l'élément courant (\refSubsectionPage{enumerationImplicitePrefixee}).\\
  \galgas{sens cst in expression} & Déclaration d'une constante représentant l'élément courant (\refSubsectionPage{enumerationParConstante}).\\
  \galgas{sens (cst1 cst2 ...) in expression} & Déclaration de constantes représentant les champs de l'élément courant (\refSubsectionPage{enumerationParListeConstantes}).\\
  \end{tabular}
  \caption{Les quatre formes d'énumération de l'instruction \texttt{for}}
  \labelTableau{quatreFormesEnumeration}
  \ligne
\end{table}


\subsection{Types énumérables et ordre d'énumération}

Les types pouvant être énumérés sont listés dans le \refTableau{typesEnumerablesFor}, ainsi que leur ordre d'énumération par défaut. Si le champ \galgas{sens} est vide, c'est l'ordre par défaut qui est adopté. Utiliser \galgas{>} fixe le sens inverse.

\begin{table}[t]
  \centering
  \begin{tabular}{ll}
  \textbf{Type} & \textbf{Ordre d'énumération}\\
  \galgas{@data} & Ordre croissant des indices\\
  \galgas{list @T} & Ordre croissant des indices \\
  \galgas{map @T} & Ordre alphabétique des clés \\
  \galgas{listmap @T} & Ordre alphabétique des clés \\
  \galgas{sortedlist @T} & Ordre du tri \\
  \galgas{@stringset} & Ordre alphabétique \\
  \end{tabular}
  \caption{Types énumérables par l'instruction \texttt{for}}
  \labelTableau{typesEnumerablesFor}
  \ligne
\end{table}
















\subsectionLabel{Énumération « \texttt{() in expression} »}{enumerationImplicite}

Des constantes correspondants à chaque champ de l'élément courant sont implicitement déclarées (\refTableau{constantesImplicitementDeclarees}). 

\begin{table}[t]
  \centering
  \begin{tabular}{lp{12cm}}
  \textbf{Type} & \textbf{Constantes implicitement déclarées}\\
  \galgas{@data} & \galgas{data}, de type \galgas{@uint}\\
  \galgas{list @T} & À chaque champ de la liste, correspond une constante de même nom.\\
  \galgas{map @T} & \galgas{lkey}, de type \galgas{@lstring}, qui représente la clé, et à chaque champ de la table, correspond une constante de même nom.\\
  \galgas{listmap @T} & \galgas{key}, de type \galgas{@string}, qui représente la clé, et \galgas{mList}, qui représente la liste associée.\\
  \galgas{sortedlist @T} & À chaque champ de la liste, correspond une constante de même nom.\\
  \galgas{@stringset} & \galgas{key}, de type \galgas{@string} \\
  \end{tabular}
  \caption{Constantes implicitement déclarées par «\texttt{() in expression}»}
  \labelTableau{constantesImplicitementDeclarees}
  \ligne
\end{table}

Voici quelques exemples :
\begin{galgascode}
@data d = ...
for () in d do
  log data
end
\end{galgascode}



\begin{galgascode}
@stringset v = ...
for () in v do
  log key # Affichage des cles dans l'ordre alphabetique
end
\end{galgascode}

Avec la liste :
\begin{galgascode}
list @maListe {
  @uint mProperty1
  @string mProperty2
}
\end{galgascode}

On peut écrire :

\begin{galgascode}
@maListe lst = ...
for () in lst do
  log mProperty1, mProperty2
end
\end{galgascode}


Avec la table :
\begin{galgascode}
map @maTable {
  @uint mProperty1
  @string mProperty2
}
\end{galgascode}

On peut écrire :

\begin{galgascode}
@maTable tab = ...
for () in tab do
  log lkey, mProperty1, mProperty2
end
\end{galgascode}


\subsectionLabel{Énumération « \texttt{() prefixe in expression} »}{enumerationImplicitePrefixee}

Cette écriture est une extension de celle de la section précédente : \galgas{prefixe} est utilisé pour préfixer le nom des constantes implicitement déclarées. En reprenant les exemples de la section précédente :

\begin{galgascode}
@data d = ...
for () xyz_ in d do
  log xyz_data
end
\end{galgascode}



\begin{galgascode}
@stringset v = ...
for () pre in v do
  log prekey # Affichage des cles dans l'ordre alphabetique
end
\end{galgascode}


\begin{galgascode}
@maListe lst = ...
for () lst in lst do
  log lstmProperty1, lstmProperty2
end
\end{galgascode}


\begin{galgascode}
@maTable tab = ...
for () tb_ in tab do
  log tb_lkey, tb_mProperty1, tb_mProperty2
end
\end{galgascode}

Utiliser un préfixe permet de lever les collisions des noms des constantes implicites quand on énumère des collections ayant des champs de même nom :

\begin{galgascode}
@maListe v1 = ...
@maListe v2 = ...
for () in v1, () in v2 do # Erreur !
 ...
end
\end{galgascode}

Le compilateur GALGAS déclenche une erreur, car il y a ambiguïté sur la signification de \galgas{mProperty1} et de \galgas{mProperty2} à l'intérieur de la boucle : désigne-t'elle l'élément courant de \galgas{v1} ou l'élément courant de \galgas{v2} ?

Pour lever l'ambiguïté, on complète l'instruction en précisant un préfixe pour l'une des deux listes (par exemple la seconde) :
\begin{galgascode}
@maListe v1 = ...
@maListe v2 = ...
for () in v1, () l2_ in v2 do
  log mProperty1 # Designe sans ambiguite le champ de la premiere liste
  log l2_mProperty1 # Designe sans ambiguite le champ de la seconde liste
end
\end{galgascode}


\subsectionLabel{Énumération « \texttt{cst in expression} »}{enumerationParConstante}

Dans cette forme, une seule constante est déclarée (\galgas{cst}), et son type est donné par le \refTableau{enumerationParConstante}. Le type \galgas{@T-element} est implicitement déclaré avec la déclaration de la collection correspondante (liste, table), et est une structure : on accède donc à ses champs par l'opérateur \galgas{.}. 


\begin{table}[t]
  \centering
  \begin{tabular}{llp{7cm}}
  \textbf{Type} & \textbf{Type implicite de la constante}\\
  \galgas{@data} & \galgas{@uint}\\
  \galgas{list @T} & \galgas{@T-element}\\
  \galgas{map @T} & \galgas{@T-element}\\
  \galgas{listmap @T} & \galgas{@T-element}\\
  \galgas{sortedlist @T} & \galgas{@T-element}\\
  \galgas{@stringset} & \galgas{@string} \\
  \end{tabular}
  \caption{Type de la constante dans «\texttt{cst in expression} »}
  \labelTableau{enumerationParConstante}
  \ligne
\end{table}


En reprenant les exemples de la \refSubsectionPage{enumerationImplicite} :

\begin{galgascode}
@data d = ...
for v in d do
  log v
end
\end{galgascode}



\begin{galgascode}
@stringset v = ...
for s in v do
  log s
end
\end{galgascode}


\begin{galgascode}
@maListe lst = ...
for x in lst do
  log x.mProperty1, x.mProperty2
end
\end{galgascode}


\begin{galgascode}
@maTable tab = ...
for entry in tab do
  log entry.lkey, entry.mProperty1, entry.mProperty2
end
\end{galgascode}

\subsubsection{Type explicite}

On peut annoter le nom de la constante en la faisant précéder par un nom de type. Le compilateur GALGAS vérifie alors l'identité entre le type explicitement déclaré, et le type implicitement déduit du type de l'expression enumérée. Il est possible de déclarer explicitement le type de la constante en écrivant l'énumération sous la forme :

\begin{galgascode}
@unType cst in expression
\end{galgascode}

Les exemples précédents deviennent alors :

\begin{galgascode}
@data d = ...
for @uint v in d do
  log v
end
\end{galgascode}



\begin{galgascode}
@stringset v = ...
for @string s in v do
  log s
end
\end{galgascode}


\begin{galgascode}
@maListe lst = ...
for @maListe-element x in lst do
  log x.mProperty1, x.mProperty2
end
\end{galgascode}


\begin{galgascode}
@maListe tab = ...
for @maListe-element entry in tab do
  log entry.lkey, entry.mProperty1, entry.mProperty2
end
\end{galgascode}



\subsectionLabel{Énumération « \texttt{(cst1 cst2 ...) in expression} »}{enumerationParListeConstantes}

Le \refTableau{enumerationParListeConstantes} liste pour chaque type le nombre et la signification des constantes qui doivent être déclarées.

\begin{table}[t]
  \centering
  \begin{tabular}{lp{12cm}}
  \textbf{Type} & \textbf{Constantes à déclarer}\\
  \galgas{@data} & \emph{Ce type n'est pas pris en charge par cette forme.}\\
  \galgas{list @T} & Une constante pour chaque champ, dans l'ordre de déclaration.\\
  \galgas{map @T} & Une constante de type \galgas{@lstring}, qui représente la clé, suivi d'une constante pour chaque champ de la table, dans l'ordre de déclaration.\\
  \galgas{listmap @T} & Une constante de type \galgas{@string}, qui représente la clé, et une constante qui représente la liste associée.\\
  \galgas{sortedlist @T} & Une constante pour chaque champ, dans l'ordre de déclaration.\\
  \galgas{@stringset} & \emph{Ce type n'est pas pris en charge par cette forme.} \\
  \end{tabular}
  \caption{Constantes à déclarer pour «\texttt{(cst1 cst2 ...) in expression} »}
  \labelTableau{enumerationParListeConstantes}
  \ligne
\end{table}


Prenons comme exemple le type liste suivant :
\begin{galgascode}
list @maListe {
  @uint mProperty1
  @string mProperty2
  @char mProperty3
  @bool mProperty4
}
\end{galgascode}

L'énumération s'écrit :
\begin{galgascode}
@maListe uneListe = {!1 !"a" !'b' !false}
for (p1 p2 p3 p4) in uneListe do
  log p1, p2, p3, p4
end
\end{galgascode}

Plusieurs variantes sont possibles, et sont décrites ci-après.

\subsubsection{Type explicite}

Il est possible d'annoter chaque constante en précisant son type.

\begin{galgascode}
@maListe uneListe = {!1 !"a" !'b' !false}
for (@uint p1 @string p2 @char p3 @bool p4) in uneListe do
  log p1, p2, p3, p4
end
\end{galgascode}

Le compilateur vérifie alors que le type cité est égal au type déduit du type de l'expression énumérée.
\begin{galgascode}
@maListe uneListe = {!1 !"a" !'b' !false}
for (@uint p1 @string p2 @char p3 @bool p4) in uneListe do
  log p1, p2, p3, p4
end
\end{galgascode}


\subsubsection{Joker}

Si certaines constantes ne sont pas utiles, on peut les remplacer par un joker (\galgas{*}). Le nom du type ne doit alors pas figurer.
\begin{galgascode}
@maListe uneListe = {!1 !"a" !'b' !false}
for (@uint p1 * * @bool p4) in uneListe do
  log p1, p4
end
\end{galgascode}

Plusieurs jokers peuvent être rassemblés en mentionnant leur nombre d'occurrences.
\begin{galgascode}
@maListe uneListe = {!1 !"a" !'b' !false}
for (@uint p1 2* @bool p4) in uneListe do
  log p1, p4
end
\end{galgascode}



\subsubsection{Points de suspension}

Trois points consécutifs (\galgas{...}) peuvent être utilisés pour signifier que les dernières constantes ne sont pas utiles.

\begin{galgascode}
@maListe uneListe = {!1 !"a" !'b' !false}
for (@uint p1 ...) in uneListe do
  log p1
end
\end{galgascode}

Et si aucune constante n'est utile, on écrit :
\begin{galgascode}
@maListe uneListe = {!1 !"a" !'b' !false}
for (...) in uneListe do

end
\end{galgascode}






\subsection{Organigramme d'exécution}

L'organigramme illustrant l'exécution de l'instruction \galgas{for} est donné à la \refFigure{}{organigrammeFor}. Si plusieurs collections sont énumérées, l'instructions se termine dès que la collection la moins nombreuse est complètement enumérée.

\begin{figure}[t]
  \centering
  \small
  \begin{tikzpicture}[
      cloud/.style ={draw=red, thick, ellipse,fill=red!20, minimum height=2em},
      block/.style ={rectangle, draw=blue, thick, fill=green!20, align=center},
      decision/.style={chamfered rectangle, draw=blue, thick, fill=green!20},
      node distance=7mm
    ]
    \node [cloud] (start) {\textsc{begin}} ;
    \node [block] (init) [below=of start] {{\tt \emph{nom\_index}} = 0} ;
    \node [decision] (premierTest) [below=of init] {non empty \& {\tt \emph{while\_expression}} ?} ;
    \node [block] (before) [below=of premierTest] {\tt \emph{before\_instructions}} ;
    \node [block] (gotoFirst) [below=of before] {goto first element} ;
    \node [block] (doInstructions) [below=of gotoFirst] {\tt \emph{instructions\_do}} ;
    \node [block] (incLoopIndex) [below=of doInstructions] {{\tt \emph{nom\_index}} ++} ;
    \node [decision] (exp) [below=of incLoopIndex] {has next element \& {\tt \emph{while\_expression}} ?} ;
    \node [block] (after) [below=of exp] {\tt \emph{instructions\_after}} ;
    \node [cloud] (end) [below=of after] {\textsc{end}} ;
    \node [block] (between) [right=of doInstructions] {\tt \emph{instructions\_between}} ;
    \node [block] (gotoNext) [below=of between] {goto next element} ;
    
    \draw [-stealth, thick] (start) -- (init) ;
    \draw [-stealth, thick] (init) -- (premierTest) ;
    \draw [>-stealth, thick] (premierTest) to (before) ;
    \draw [o-stealth, thick] (premierTest.west) -- +(-1, 0) |- (end.west) ;
    \draw [-stealth, thick] (before) -- (gotoFirst) ;
    \draw [-stealth, thick] (gotoFirst) -- (doInstructions) ;
    \draw [-stealth, thick] (doInstructions) -- (incLoopIndex) ;
    \draw [-stealth, thick] (incLoopIndex) -- (exp) ;
    \draw [o-stealth, thick] (exp) to (after) ;
    \draw [>-stealth, thick] (exp.east) -| (gotoNext.south) ;
    \draw [-stealth, thick] (after) -- (end) ;
    \draw [-stealth, thick] (gotoNext) -- (between) ;
    \draw [-stealth, thick] (between) -- (doInstructions) ;
  \end{tikzpicture}
  \caption{Organigramme d'exécution d'une instruction \texttt{for}}
  \labelFigure{organigrammeFor}
  \ligne
\end{figure}


\subsection{Champs optionnels}

Plusieurs champs de l'instruction \galgas{for} sont optionnels.

\lstset{emph={expression, sens, instructions\_before, instructions\_do, instructions\_between, instructions\_after}, emphstyle=\galgasEmphStyle}

\galgas{sens}. Ce champ peut prendre trois valeurs, et fixe l'ordre dans lequel les éléments sont énumérés :
\begin{itemize}
  \item si le champ est vide, dans l'ordre indiqué par le \refTableau{typesEnumerablesFor} ;
  \item \galgas{>}, dans l'ordre inverse à celui indiqué par le \refTableau{typesEnumerablesFor}.
\end{itemize}


\galgas{(nom\_index)}. Vous pouvez mentionner un identificateur entre parenthèses après le mot réservé \galgas{do}. Cet identificateur est le nom d'une constante qui a implicitement le type \galgas{@uint} et qui est initialisée à 0 avant toute exécution de la boucle, et incrémentée après chaque exécution des \galgas{instructions\_do}, et avant l'exécution des \galgas{instructions\_between}. Sa visibilité est la branche \galgas{do}.

\galgas{while expression}. L'énumération est exécutée tant que l'\galgas{expression} est vraie. L'absence de cette construction est équivalent à \galgas{while true} et permet d'énumérer toutes les valeurs.


\galgas{before instructions\_before}. Ces instructions sont exécutées une seule fois, au début de l'exécution de l'instruction. Aucun accès aux objets énumérés n'est possible. Si l'énumération est vide, ces instructions ne sont pas exécutées.

\galgas{between instructions\_between}. Ces instructions sont exécutées entre deux exécutions consécutives des \galgas{instructions\_do}. Aucun accès aux objets énumérés n'est possible.

\galgas{after instructions\_after}. Ces instructions sont exécutées une seule fois, à la fin de l'exécution de l'instruction. Aucun accès aux objets énumérés n'est possible. Si l'énumération est vide, ces instructions ne sont pas exécutées.


\subsection{Modification de la collection}

Au début de l'exécution de l'instruction \galgas{for}, les valeurs des collections enumérées sont capturées et mémorisées. L'énumération s'effectue sur ces valeurs mémorisées. Aussi, il est possible de modifier la collection en cours d'énumération sans que cela affecte l'exécution :
\begin{galgascode}
@stringlist v = {}
v += !"A"
v += !"B"
v += !"C"
log v # "A", "B", "C"
for s in v do
  v += !s
end for
log v # "A", "B", "C", "A", "B", "C"
\end{galgascode}




















\sectionLabel{Instruction d'incrémentation}{incrementInstruction}

L'instruction d'incrémentation s'applique aux types  \refTypePredefini{sint},  \refTypePredefini{sint64}, \refTypePredefini{uint} et \refTypePredefini{uint64} ; sa syntaxe est la suivante :
\lstset{emph={variable, champ}, emphstyle=\galgasEmphStyle}
\begin{galgascode}
variable ++
\end{galgascode}

Les champs de structure peuvent être incrémentés :
\begin{galgascode}
variable.champ ++
\end{galgascode}

Une erreur d'exécution est déclenchée en cas de dépassement de capacité.









\section{L'instruction \texttt{if}}


Dans sa forme la plus générale, l'instruction \galgas{if} a la syntaxe suivante:
{\lstset{emph={condition, instructions, condition2, instructions2, instructions_else}, emphstyle=\galgasEmphStyle}
\begin{galgascode}
if condition then
  instructions
elsif condition2 then
  instructions2
...
else
  instructions_else
end
\end{galgascode}
}

Plus précisement, elle contient :
\begin{itemize}
\item zéro, une ou plusieurs branches \galgas{elsif} ;
\item zéro ou une branche \galgas{else}.
\end{itemize}


Aucune branche \galgas{else} est équivalent à une branche \galgas{else} sans aucune instruction.


Les branches \galgas{elsif} sont simplement du sucre syntaxique : il est sémantiquement équivalent d'utiliser des instructions \galgas{if} imbriquées. Par exemple :
{\lstset{emph={condition, instructions, condition2, instructions2, instructions_else}, emphstyle=\galgasEmphStyle}
\begin{galgascode}
if condition then
  instructions
elsif condition2 then
  instructions2
else
  instructions_else
end
\end{galgascode}
}
est équivalent à :
{\lstset{emph={condition, instructions, condition2, instructions2, instructions_else}, emphstyle=\galgasEmphStyle}
\begin{galgascode}
if condition then
  instructions
else
  if condition2 then
    instructions2
  else
    instructions_else
  end
end
\end{galgascode}
}

%So, for describing \emph{if} instruction static and dynamic semantics, we only need to describe an \emph{if} instruction without any \emph{elsif} branch and with an \emph{else} branch:
%{\lstset{emph={condition, instructions, else_instructions}, emphstyle=\galgasEmphStyle}
%\begin{galgascode}
%if condition then
%  instructions
%else
%  else_instructions
%end
%\end{galgascode}
%}

{\lstset{emph={condition, condition2}, emphstyle=\galgasEmphStyle}
Le langage permet que le type des \galgas{condition} et \galgas{condition2} soit différent du type \refTypePredefini{bool}, sous certaines conditions. La règle complète est que le type des \galgas{condition} et \galgas{condition2} est :
\begin{itemize}
\item soit le type \galgas{@bool} ;
\item soit un type \emph{structure}, possèdant une propriété nommée \galgas{bool}, dont le type est \galgas{@bool}; 
\item soit un type possédant un \emph{getter} sans argument nommé \galgas{bool} et renvoyant une valeur de type \galgas{@bool}.
\end{itemize}

Voici un exemple illustrant le deuxième cas ; le type \refTypePredefini{bool} est une structure possèdant une propriété nommée \galgas{bool}, dont le type est \galgas{@bool}. Aussi, écrire :
{\lstset{emph={expression, instructions, else_instructions}, emphstyle=\galgasEmphStyle}
\begin{galgascode}
@lbool variable = ...
if variable then
  instructions
else
  else_instructions
end
\end{galgascode}
}

est équivalent à :
{\lstset{emph={expression, instructions, else_instructions}, emphstyle=\galgasEmphStyle}
\begin{galgascode}
@lbool variable = ...
if variable.bool then
  instructions
else
  else_instructions
end
\end{galgascode}
}

Pour illustrer le troisième cas, on prend l'exemple de la classe suivante :
\begin{galgascode}
class @myClass { ... }

getter @myClass bool -> @bool outResult { ... }
\end{galgascode}

Ainsi, on peut écrire :
{\lstset{emph={expression, instructions, else_instructions}, emphstyle=\galgasEmphStyle}
\begin{galgascode}
@myClass myObject = ...
if myObject then
  instructions
else
  else_instructions
end
\end{galgascode}
}

Il est équivalent d'écrire :
{\lstset{emph={expression, instructions, else_instructions}, emphstyle=\galgasEmphStyle}
\begin{galgascode}
@myClass myObject = ...
if [myObject bool] then
  instructions
else
  else_instructions
end
\end{galgascode}
}


%\subsection{Dynamic semantics}
%
%According to the preceding section, we only need to describe the dynamic semantic of an \emph{if} instruction without any \emph{elsif} branch and with an \emph{else} branch:
%{\lstset{emph={expression, instructions, else_instructions}, emphstyle=\galgasEmphStyle}
%\begin{galgascode}
%if expression then
%  instructions
%else
%  else_instructions
%end  
%\end{galgascode}
%}
%
%
%
%The \emph{expression} is first computed :
%\begin{itemize}
%\item if the evaluation fails, neither the \emph{if} instructions, neither the \emph{else} instructions are executed;
%\item if the evaluation result is \emph{true}, the \emph{if} instructions are executed ;
%\item if the evaluation result is \emph{false}, the \emph{else} instructions are executed.
%\end{itemize}


\sectionLabel{L'instruction \texttt{grammar}}{instruction-grammar}

L'instruction \galgas{grammar} permet d'exécuter l'analyse d'un texte par une grammaire. Le texte peut être contenu dans un fichier ou dans une chaîne de caractères.

Si le texte est contenu dans un fichier :
\lstset{emph={nom\_grammaire, label\_grammaire, expression\_de\_type\_lstring, liste\_parametres,
              expression\_de\_type\_string, traduction\_dirigee\_par\_la\_syntaxe},
        emphstyle=\galgasEmphStyle
}
\begin{galgascode}
grammar
  nom_grammaire
  label_grammaire # Optionnel
  in expression_de_type_lstring # Chemin du fichier source
  liste_parametres
  traduction_dirigee_par_la_syntaxe # Optionnel
\end{galgascode}

Si le texte est contenu dans une chaîne de caractères :
\begin{galgascode}
grammar
  nom_grammaire
  label_grammaire # Optionnel
  on expression_de_type_string # Texte source
  liste_parametres
  traduction_dirigee_par_la_syntaxe # Optionnel
\end{galgascode}

Dans ces écritures :
\begin{itemize}
  \item \galgas{nom\_grammaire} est un identificateur nommant la grammaire, c'est le nom d'un composant grammaire du projet ;
  \item \galgas{label\_grammaire} est optionnel, et permet d'exécuter une variante des règles de production (voir \refSectionPage{analysePlusieursPhases}) ;
  \item \galgas{expression\_de\_type\_lstring} est une valeur de type \galgas{@lstring}, dont le champ \galgas{string} désigne un fichier source, par un chemin relatif ou absolu, et dont le champ \galgas{location} est la position de signalement d'erreur si le fichier source ne peut pas être lu ;
  \item \galgas{expression\_de\_type\_string} est directement la chaîne source à analyser ;
  \item \galgas{liste\_parametres} est une liste de paramètres effectifs (en entrée, sortie, ou sortie/entrée), en accord avec la liste des arguments formels de l'axiome de la grammaire ;
  \item \galgas{traduction\_dirigee\_par\_la\_syntaxe} est optionnel, et permet d'obtenir la chaîne source traduite lors d'une \emph{traduction dirigée par la syntaxe} (voir \refSectionPage{instructionGrammarEtTraductionDirigeeParLaSyntaxe}).
\end{itemize}


Prenons l'exemple d'un composant grammaire :
\begin{galgascode}
grammar maGrammaire "SLR" {
  syntax ...
   <start_symbol> ?!@declarationAST ioDeclarations
}
\end{galgascode}

L'instruction grammaire s'écrit alors :
\begin{galgascode}
grammar maGrammaire in fichierSource !?ioDeclarations
\end{galgascode}

Cette instruction est typiquement utilisé dans une règle d'analyse de fichier source (\refChapterPage{regleAnalyseFichierSource}) :

\begin{galgascode}
case . "monExtension"
message "un fichier source"
??@lstring inSourceFile {
  var declaration = @declarationList {}
  grammar maGrammaire in inSourceFile !?ioDeclarations
  ...
}
\end{galgascode}















\sectionLabel{L'instruction \texttt{log}}{instructionLog}

L'instruction \galgas{log} permet d'afficher le détail de la valeur d'une variable, d'une constante ou d'une expression :
\lstset{emph={expression, nom}, emphstyle=\galgasEmphStyle}
\begin{itemize}
  \item pour une variable ou une constante, \galgas{log nom} ;
  \item pour une expression, \galgas{log "message" : expression} ;
\end{itemize}

Par exemple :
\begin{galgascode}
let x = 2
log x # Affiche LOGGING x: <@uint:2>
log "valeur" : x * 2 # Affiche LOGGING valeur: <@uint:4>
\end{galgascode}

Plusieurs variables ou constantes peuvent être affichées par une même instruction \galgas{log}, en les séparant par une virgule :
\begin{galgascode}
let x = 2
log x, "valeur" : x * 2
\end{galgascode}














\section{L'instruction \texttt{loop}}


L'instruction \galgas{loop} a la syntaxe suivante :
\lstset{emph={expression, instructions_1, instructions_2, variant_expression}, emphstyle=\galgasEmphStyle}
\begin{galgascode}
loop (variant_expression)
  instructions_1
while expression do
  instructions_2
end
\end{galgascode}


Les \galgas{instructions\_1} et \galgas{instructions\_2} sont des listes d'instructions qui peuvent être vides.


Le \galgas{variant\_expression} est une expression de type \galgas{@uint} qui assure que la boucle n'est pas sans fin : elle est calculée au début de l'exécution de l'instruction, et décrémentée après chaque itération. Si sa valeur atteint zéro, une erreur d'exécution est déclenchée.

L'\galgas{expression} est une expression de type \galgas{@bool} qui exprime la continuation de l'exécution de la boucle.

L'exécution de l'instruction \galgas{loop} est illustrée par l'organigramme de la \refFigure{}{loopInstructionFlowchart}.


\begin{figure}[t]
  \centering
  \small
  \begin{tikzpicture}[
      cloud/.style ={draw=red, thick, ellipse,fill=red!20, minimum height=2em},
      block/.style ={rectangle, draw=blue, thick, fill=green!20, align=center},
      error/.style ={rectangle, draw=red, thick, fill=green!20, align=center},
      decision/.style={chamfered rectangle, draw=blue, thick, fill=green!20},
      node distance=7mm
    ]
    \node [cloud] (start) {\textsc{begin}} ;
    \node [block] (affectationVariant) [below=of start] {variant := {\tt \emph{variant\_expression}}} ;
    \node [decision] (premierTest) [below=of affectationVariant] {variant > 0 ?} ;
    \node [error] (loopVariantError) [right=of premierTest] {loop variant error} ;
    \node [block] (instructions1) [below=of premierTest] {\tt \emph{instructions\_1}} ;
    \node [decision] (expression) [below=of instructions1] {\tt \emph{expression} ?} ;
    \node [decision] (secondTest) [below=of expression] {variant > 0 ?} ;
    \node [error] (loopVariantError2) [right=of secondTest] {loop variant error} ;
    \node [block] (decVariant) [below=of secondTest] {variant {-}{-}} ;
    \node [block] (instructions2) [below=of decVariant] {\tt \emph{instructions\_2}} ;
    \node [cloud] (end) [below=of instructions2] {\textsc{end}} ;
    
    \draw [-stealth, thick] (start) -- (affectationVariant) ;
    \draw [-stealth, thick] (affectationVariant) -- (premierTest) ;
    \draw [o-stealth, thick] (premierTest) -- (loopVariantError) ;
    \draw [o-stealth, thick] (secondTest) -- (loopVariantError2) ;
    \draw [-stealth, thick] (loopVariantError.east) -- +(1, 0) |- (end.east) ;
    \draw [-stealth, thick] (loopVariantError2.east) -- +(1, 0) ;
    \draw [o-stealth, thick] (expression.east) -- +(4.1, 0) ;
    \draw [-stealth, thick] (instructions1) -- (expression) ;
    \draw [>-stealth, thick] (premierTest) -- (instructions1) ;
    \draw [>-stealth, thick] (expression) -- (secondTest) ;
    \draw [>-stealth, thick] (secondTest) -- (decVariant) ;
    \draw [-stealth, thick] (decVariant) -- (instructions2) ;
    \draw [-stealth, thick] (instructions2.west) -- +(-1, 0) |- (instructions1.west) ;
  \end{tikzpicture}
  \caption{Organigramme d'exécution d'une instruction \texttt{loop}}
  \labelFigure{loopInstructionFlowchart}
  \ligne
\end{figure}
















\sectionLabel{L'instruction d'appel de procédure}{instructionAppelProcedure}

La syntaxe de l'appel d'une procédure est :
\lstset{emph={nom_procedure, liste_parametres_effectifs}, emphstyle=\galgasEmphStyle}
\begin{galgascode}
nom_procedure (liste_parametres_effectifs)
\end{galgascode}

Les parenthèses sont obligatoires. La déclaration d'un procédure est présentée à la \refSectionPage{declarationProcedure}.

\galgas{nom_procedure} est le nom de la procédure, et \galgas{liste\_parametres\_effectifs} est la liste des paramètres effectifs de l'appel, en accord avec l'en-tête de la procédure.

Par exemple, la procedure suivante :
\begin{galgascode}
proc produit ?@uint a ?@uint b !@uint resultat {
  resultat = a * b
}
\end{galgascode}

Peut être appelée par :
\begin{galgascode}
produit (!2 !3 ??@uint resultat)
\end{galgascode}















\sectionLabel{L'instruction d'appel de méthode}{methodCallInstruction}

En GALGAS, une \emph{méthode} est un sous-programme qui s'applique à un objet, et qui ne modifie pas cet objet. La syntaxe de l'appel d'une méthode est :
\lstset{emph={nom_methode, expression, liste_parametres_effectifs}, emphstyle=\galgasEmphStyle}
\begin{galgascode}
[expression nom_methode liste_parametres_effectifs]
\end{galgascode}

La valeur d'\galgas{expression} est l'objet sur lequel la méthode est appelée, \galgas{nom\_methode} est le nom de la méthode, et \galgas{liste\_parametres\_effectifs} est la liste des paramètres effectifs, en accord avec l'en-tête de la méthode.

Avec l'option \texttt{-T}, un fichier HTML qui contient les caractéristiques de tous les types d'un projet est engendré dans le répertoire \texttt{build/helpers}. Ainsi, les en-têtes de toutes les méthodes sont listés. Par exemple, pour le type \galgas{@string}, la méthode \galgas{writeToExecutableFileWhenDifferentContents} est présentée comme suit :

\texttt{
\textbf{method} writeToExecutableFileWhenDifferentContents\\
\hspace*{1cm}??@string inFilePath\\
\hspace*{1cm}!@bool outFileModified
}

Aussi, cette méthode peut être appelée par :
\begin{galgascode}
@string contents = ...
@string filePath = ...
[contents writeToExecutableFileWhenDifferentContents
  !filePath
  ??@bool fileChanged
]
\end{galgascode}















\sectionLabel{L'instruction d'appel de procédure de classe}{classProcCallInstruction}

En GALGAS, une \emph{procédure de classe} est est une procédure définie dans un type. Au contraire d'une méthode, elle ne s'applique pas à un objet. La syntaxe de l'appel d'une procédure de classe est :
\lstset{emph={nom_procedure, liste_parametres_effectifs}, emphstyle=\galgasEmphStyle}
\begin{galgascode}
[@T nom_procedure liste_parametres_effectifs]
\end{galgascode}

\galgas{nom_procedure} est le nom de la procédure, et \galgas{liste\_parametres\_effectifs} est la liste des paramètres effectifs, en accord avec l'en-tête de la procédure.

Avec l'option \texttt{-T}, un fichier HTML qui contient les caractéristiques de tous les types d'un projet est engendré dans le répertoire \texttt{build/helpers}. Ainsi, les en-têtes de toutes les méthodes sont listés. Par exemple, pour le type \galgas{@string}, la procédure de classe \galgas{deleteFile} est présentée comme suit :

\texttt{
\textbf{proc} @string deleteFile\\
\hspace*{1cm}??@string inFilePath\\
}

Aussi, elle peut être appelée par :
\begin{galgascode}
@string filePath = ...
[@string deleteFile !filePath]
\end{galgascode}









\sectionLabel{L'instruction d'appel de \emph{setter}}{setterCallInstruction}


En GALGAS, un \emph{setter} est un sous-programme qui s'applique à un objet, et qui peut modifier cet objet. L'instruction d'appel accepte deux formes différentes.

\subsection{Appel simple}

La première syntaxe de l'appel d'un  \emph{setter} est :
\lstset{emph={nom_setter, cible, liste_parametres_effectifs}, emphstyle=\galgasEmphStyle}
\begin{galgascode}
[!?cible nom_setter liste_parametres_effectifs]
\end{galgascode}

Le délimiteur \galgast{!?} devant \galgas{cible} permet de distinguer syntaxiquement un appel de \emph{setter} d'un appel de méthode. \galgas{cible} désigne l'objet sur lequel le \emph{setter} est appelé, et peut être une variable, ou le champ d'une variable (\galgas{variable.champ}), ou le champ d'un champ d'une variable (\galgas{variable.champ.champ2}), ... \galgas{nom\_setter} est le nom du \emph{setter}, et \galgas{liste\_parametres\_effectifs} est la liste des paramètres effectifs, en accord avec l'en-tête du \emph{setter}.

Avec l'option \texttt{-T}, un fichier HTML qui contient les caractéristiques de tous les types d'un projet est engendré dans le répertoire \texttt{build/helpers}. Ainsi, les en-têtes de toutes les \emph{setters} sont listées. Par exemple, pour le type \galgas{@string}, le \emph{setter} \galgas{setCharacterAtIndex} est présenté comme suit :

\begin{galgascode}
setter setCharacterAtIndex
  ??@char inChar
  ??@uint inIndex
}
\end{galgascode}

Aussi, ce \emph{setter} peut être appelée par :
\begin{galgascode}
@string s = ...
[!?s setCharacterAtIndex !'a' !4]
\end{galgascode}


\subsection{Appel avec conversion de type}

La seconde syntaxe de l'appel d'un  \emph{setter} est :

\lstset{emph={nom_setter, cible, liste_parametres_effectifs}, emphstyle=\galgasEmphStyle}
\begin{galgascode}
[!?cible as @T nom_setter liste_parametres_effectifs]
\end{galgascode}

Statiquement, le type \galgas{@T} doit être un type héritier du type statique de \galgas{cible}.

À l'exécution, si le type dynamique est le type \galgas{@T} ou un de ses héritiers, l'instruction est exécutée. Sinon, une erreur d'exécution a lieu et le setter n'est pas appelé.

Par exemple, on considère les déclarations de classe et le setter suivant :

\begin{galgascode}
class @a { }

class @b : @a { }

setter @b aSetter { }
\end{galgascode}

Si on écrit :
\begin{galgascode}
@a unObjet = ...
[!?unObjet aSetter] # Erreur semantique
\end{galgascode}

L'instruction \galgas{[!?unObjet aSetter]} donne lieu à une erreur sémantique, puisque la classe \galgas{@a} ne définit pas le setter \galgas{aSetter}.

L'écriture suivante est acceptée par le compilateur car la classe \galgas{@b} définit le setter \galgas{aSetter} :
\begin{galgascode}
@a unObjet = ...
[!?unObjet as @b aSetter] # Compilation ok
\end{galgascode}

À l'exécution, le comportement dépend du type dynamique de \galgas{unObjet}. Si celui-ci est une instance de \galgas{@a}, une erreur d'exécution est déclenchée :
\begin{galgascode}
@a unObjet = @a.new
[!?unObjet as @b aSetter] # Compilation ok, erreur d'execution
\end{galgascode}


Par contre, si \galgas{unObject} est une instance de \galgas{@b}, l'appel du setter d'effectue :
\begin{galgascode}
@a unObjet = @b.new
[!?unObjet as @b aSetter] # Compilation ok, appel du setter
\end{galgascode}










\sectionLabel{L'instruction \texttt{switch}}{instructionSwitch}

L'instruction \galgas{switch} est dédiée aux types énumérés. Elle présente la syntaxe suivante :

\lstset{emph={constante, expression, liste_instructions}, emphstyle=\galgasEmphStyle}
\begin{galgascode}
switch expression
case constante, constante, ... :
  liste_instructions
case constante, constante, ... :
  liste_instructions
...
end
\end{galgascode}


Où \galgas{expression} est une expression d'un type énuméré. Toutes les constantes de ce type doivent être nommées dans les branches \galgas{case}, une et une seule fois.

Par exemple, avec la déclaration :

\begin{galgascode}
enum @feuTricolore {
  case vert
  case orange
  case rouge
}
\end{galgascode}

On peut écrire :

\begin{galgascode}
@feuTricolore feu = ... ;

switch feu
case vert, orange:
  ...
case rouge :
  ...
end
\end{galgascode}

Si des constantes déclarées dans l'énumération ont des valeurs associées, alors les branches \galgas{case} nommant ces constantes doivent adopter une syntaxe particulière. 

En prenant pour exemple une constante possédant deux valeurs associées, la forme la plus générale est :

\begin{galgascode}
switch expression
case constante (@type1 nom1 @type2 nom2) :
...
end
\end{galgascode}

\galgas{nom1} et \galgas{nom2} sont des constantes qui reçoivent les valeurs associées.

Si on n'est pas intéressé par une valeur, on peut substituer \galgas{*} au nom :

\begin{galgascode}
switch expression
case constante (@type1 nom1 @type2 *) :
...
end
\end{galgascode}

De même, l'annotation du type est optionnel : les types \galgas{@type1}, \galgas{@type2} peuvent être déduits de la déclaration de la valeur associée :

\begin{galgascode}
switch expression
case constante (nom1 *) :
...
end
\end{galgascode}

Ainsi, si l'on n'est pas intéressé par les valeurs associées, on peut écrire :
\begin{galgascode}
switch expression
case constante (* *) :
...
end
\end{galgascode}


Enfin, on peut mentionner dans le même branche \galgas{case} plusieurs constantes déclarant des valeurs associées, à la condition que ces valeurs associées soient de même nombre et de même type. Par exemple :

\begin{galgascode}
enum @erreur {
  case ok
  case erreur1 (@string unMessage)
  case erreur2 (@string autreMessage) 
}
\end{galgascode}

On peut écrire l'instruction \galgas{switch} correspondante :
\begin{galgascode}
@erreur erreur = ... ;

switch erreur
case ok:
  ...
case erreur1, erreur2 (@string m) :
  ...
end
\end{galgascode}




\section{L'instruction \texttt{warning}}

L'instruction \galgas{warning} permet de signaler une alerte à l'utilisateur. Elle est constituée de deux champs séparés par un double-point (\galgas{\:}) :

{\lstset{emph={localisation, message_alerte, variable1, variableN}, emphstyle=\galgasEmphStyle}
\begin{galgascode}
warning localisation : message_alerte
\end{galgascode}



Le champ \galgas{localisation} signale à l'utilisateur la position de l'erreur dans le texte source. C'est donc une expression de type \galgas{@location}, ou d'un type possédant un \emph{getter} sans argument nommé \galgas{location} et renvoyant un objet de type \galgas{@location} : c'est le cas de tous les types prédéfinis \refTypePredefini{luint}, \refTypePredefini{luint64}, \refTypePredefini{lsint}, \refTypePredefini{lsint64}, \refTypePredefini{lbool}, \refTypePredefini{lchar} et \refTypePredefini{lstring}. Le \galgas{message_alerte} est le message affiché à l'utilisateur : c'est donc une expression de type \galgas{@string}.

Par exemple :

\begin{galgascode}
$identifier$ ??@lstring nom
...
warning nom.location : message_alerte
\end{galgascode}

Comme \galgas{nom} est de type \galgas{lstring} (voir ci-dessus), on peut simplement écrire :
\begin{galgascode}
$identifier$ ??@lstring nom
...
warning nom : message_alerte
\end{galgascode}










\section{L'instruction \texttt{with}}

L'instruction \galgas{with} permet d'associer un test de recherche dans une table et l'accès aux champs correspondants si succès. Elle peut prendre quatre formes différentes, suivi que l'on veuille modifier la table ou non, et suivant que l'on veut tolérer l'échec de la recherche ou non.

\lstset{emph={expression_cle, prefixe_optionnel, expression_table, liste_instructions_do, liste_instructions_else, methode_recherche, cible_table}, emphstyle=\galgasEmphStyle}

Accès en lecture, tolérance de l'échec de la recherche (\refSubsectionPage{instructionWithEnLectureTolerante}) :
\begin{galgascode}
with expression_cle prefixe_optionnel in expression_table
do
  liste_instructions_do
else
  liste_instructions_else # Optionnel
end
\end{galgascode}

Accès en lecture, signalement d'erreur si échec de la recherche (\refSubsectionPage{instructionWithEnLectureSansTolerance}) :
\begin{galgascode}
with expression_cle prefixe_optionnel in expression_table
error message methode_recherche
do
  liste_instructions_do
end
\end{galgascode}



Accès en lecture/écriture, tolérance de l'échec de la recherche (\refSubsectionPage{instructionWithEnLectureEcritureTolerante}) :
\begin{galgascode}
with expression_cle prefixe_optionnel in !?cible_table
do
  liste_instructions_do
else
  liste_instructions_else # Optionnel
end
\end{galgascode}

Accès en lecture/écriture, signalement d'erreur si échec de la recherche (\refSubsectionPage{instructionWithEnLectureEcritureSansTolerance}) :
\begin{galgascode}
with expression_cle prefixe_optionnel in !?cible_table
error message methode_recherche
do
  liste_instructions_do
end
\end{galgascode}




\subsectionLabel{Accès en lecture tolérant l'échec de la recherche}{instructionWithEnLectureTolerante}


\begin{galgascode}
with expression_cle prefixe_optionnel in expression_table
do
  liste_instructions_do
else
  liste_instructions_else # Optionnel
end
\end{galgascode}

Où :
\begin{itemize}
  \item \galgas{expression\_cle} est une expression de type \galgas{@string} dont la valeur définit la clé ;
  \item \galgas{prefixe\_optionnel} est soit vide, soit est constitué d'un double-point \galgast{:} suivi d'un identificateur qui préfixe les noms des champs dans la liste d'instructions \galgas{liste\_instructions\_do} ;
  \item \galgas{expression\_table} est une expression qui désigne la table.
\end{itemize}

La clé désignée par la valeur de \galgas{expression\_cle} est recherchée dans la table désignée par la valeur de \galgas{expression\_table} :
\begin{itemize}
  \item en cas d'échec, les instructions \galgas{liste_instructions_else} sont exécutées ;
  \item en cas de succès, ce sont les instructions \galgas{liste_instructions_do} qui sont exécutées ; les propriétés de l'élément recherché sont alors disponibles en lecture, sous leur nom éventuellement préfixé ; la clé est disponible sous le nom \galgas{lkey} éventuellement préfixé, et est de type \galgas{@lstring}.
\end{itemize}

%Par exemple, on veut disposer d'une table qui implémente un \emph{counted set}, c'est à dire que l'on associe à la clé son nombre de citations. On déclare :
%\begin{galgascode}
%map @maTable {
%  @uint mOccurrenceCount
%  insert insertKey error message "entree deja presente"
%}
%\end{galgascode}
%
%Et on effectue des recherches / insertions de la façon suivante :
%\begin{galgascode}
%@lstring cle = ...
%with cle.string in !?table do
%  mOccurrenceCount ++
%else
%  [!?table insertKey !cle !1]
%end
%\end{galgascode}
%
%En utilisant un préfixe, le code devient :
%\begin{galgascode}
%@lstring cle = ...
%with cle.string : xyz_ in !?table do
%  mOccurrenceCount ++
%else
%  [!?table insertKey !cle !1]
%end
%\end{galgascode}


Par exemple, on veut disposer d'une table possédant une propriété identifiant de manière unique la clé. On déclare :
\begin{galgascode}
map @maTable {
  @uint mIndex
  insert insertKey error message "entree deja presente"
}
\end{galgascode}

Et on effectue des recherches / insertions de la façon suivante :
\begin{galgascode}
@uint idx ;
@lstring cle = ...
with cle.string in table do
  idx = mIndex
else
  idx = [table count]
  [!?table insertKey !cle !idx]
end
\end{galgascode}

En utilisant un préfixe, le code devient :
\begin{galgascode}
@uint idx ;
@lstring cle = ...
with cle.string : xyz_ in table do
  idx = xyz_mIndex
else
  idx = [table count]
  [!?table insertKey !cle !idx]
end
\end{galgascode}




\subsectionLabel{Accès en lecture, signalement d'erreur si échec de la recherche}{instructionWithEnLectureSansTolerance}


\begin{galgascode}
with expression_cle prefixe_optionnel in expression_table
error message methode_recherche
do
  liste_instructions_do
end
\end{galgascode}

Où :
\begin{itemize}
  \item \galgas{expression\_cle} est une expression de type \galgas{@lstring} dont la valeur définit la clé ;
  \item \galgas{prefixe\_optionnel} est soit vide, soit est constitué d'un double-point \galgast{:} suivi d'un identificateur qui préfixe les noms des champs dans la liste d'instructions \galgas{liste\_instructions\_do} ;
  \item \galgas{expression\_table} est une expression qui désigne la table ;
  \item \galgas{methode_recherche} est une méthode de recherche de la table (c'est-à-dire déclarée par \galgas{search}) dont le message associé sera utilisé pour le signalement d'erreur.
\end{itemize}

La clé désignée par la valeur de \galgas{expression\_cle} est recherchée dans la table désignée par la valeur de \galgas{expression\_table} :
\begin{itemize}
  \item en cas d'échec, une erreur est affichée, son message étant fourni par la méthode de recherche \galgas{methode_recherche}, et sa localisation par le champ \galgas{location} de l'\galgas{expression\_cle} ;
  \item en cas de succès, ce sont les instructions \galgas{liste_instructions_do} qui sont exécutées ; les propriétés de l'élément recherché sont alors disponibles en lecture, sous leur nom éventuellement préfixé ; la clé est disponible sous le nom \galgas{lkey} éventuellement préfixé, et est de type \galgas{@lstring}.
\end{itemize}

Par exemple, on veut disposer d'une table qui implémente un \emph{counted set}, c'est à dire que l'on associe à la clé son nombre de citations, et la doit avoir été entrée auparavant. On déclare :
\begin{galgascode}
map @maTable {
  @uint mOccurrenceCount
  search searchKey error message "entree %K absente"
}
\end{galgascode}

Et on effectue des recherches de la façon suivante :
\begin{galgascode}
@lstring cle = ...
@uint occurenceCount
with cle in table error message searchKey do
  occurenceCount = mOccurrenceCount ++
end
\end{galgascode}

En utilisant un préfixe, le code devient :
\begin{galgascode}
@lstring cle = ...
@uint occurenceCount
with cle : abc_ in table error message searchKey do
  occurenceCount = abc_mOccurrenceCount ++
end
\end{galgascode}





\subsectionLabel{Accès en lecture/écriture tolérant l'échec de la recherche}{instructionWithEnLectureEcritureTolerante}


\begin{galgascode}
with expression_cle prefixe_optionnel in !?cible_table
do
  liste_instructions_do
else
  liste_instructions_else # Optionnel
end
\end{galgascode}


Où :
\begin{itemize}
  \item \galgas{expression\_cle} est une expression de type \galgas{@string} dont la valeur définit la clé ;
  \item \galgas{prefixe\_optionnel} est soit vide, soit est constitué d'un double-point \galgast{:} suivi d'un identificateur qui préfixe les noms des champs dans la liste d'instructions \galgas{liste\_instructions\_do} ;
  \item \galgas{cible\_table} est de la forme \galgas{variable}, ou  \galgas{variable.champ}, ou \galgas{variable.champ.champ2}, ... et désigne la table ; la cible est accédée en lecture/écriture.
\end{itemize}

La clé désignée par la valeur de \galgas{expression\_cle} est recherchée dans la table désignée par la valeur de \galgas{cible_table} :
\begin{itemize}
  \item en cas d'échec, les instructions \galgas{liste_instructions_else} sont exécutées ;
  \item en cas de succès, ce sont les instructions \galgas{liste_instructions_do} qui sont exécutées ; les propriétés de l'élément recherché sont alors disponibles en lecture / écriture, sous leur nom éventuellement préfixé ; la clé est disponible en lecture seule sous le nom \galgas{lkey} éventuellement préfixé, est de type \galgas{@lstring}.
\end{itemize}

Par exemple, on veut disposer d'une table qui implémente un \emph{counted set}, c'est à dire que l'on associe à la clé son nombre de citations. On déclare :
\begin{galgascode}
map @maTable {
  @uint mOccurrenceCount
  insert insertKey error message "entree deja presente"
}
\end{galgascode}

Et on effectue des recherches / insertions de la façon suivante :
\begin{galgascode}
@lstring cle = ...
with cle.string in !?table do
  mOccurrenceCount ++
else
  [!?table insertKey !cle !1]
end
\end{galgascode}

En utilisant un préfixe, le code devient :
\begin{galgascode}
@lstring cle = ...
with cle.string : xyz_ in !?table do
  xyz_mOccurrenceCount ++
else
  [!?table insertKey !cle !1]
end
\end{galgascode}



\subsectionLabel{Accès en lecture/écriture, signalement d'erreur si échec de la recherche}{instructionWithEnLectureEcritureSansTolerance}


\begin{galgascode}
with expression_cle prefixe_optionnel in !?cible_table
error message methode_recherche
do
  liste_instructions_do
end
\end{galgascode}

Où :
\begin{itemize}
  \item \galgas{expression\_cle} est une expression de type \galgas{@lstring} dont la valeur définit la clé ;
  \item \galgas{prefixe\_optionnel} est soit vide, soit est constitué d'un double-point \galgast{:} suivi d'un identificateur qui préfixe les noms des champs dans la liste d'instructions \galgas{liste\_instructions\_do} ;
  \item \galgas{expression\_table} est une expression qui désigne la table ;
  \item \galgas{methode_recherche} est une méthode de recherche de la table (c'est-à-dire déclarée par \galgas{search}) dont le message associé sera utilisé pour le signalement d'erreur ;
  \item \galgas{cible\_table} est de la forme \galgas{variable}, ou  \galgas{variable.champ}, ou \galgas{variable.champ.champ2}, ... et désigne la table ; la cible est accédée en lecture/écriture.
\end{itemize}

La clé désignée par la valeur de \galgas{expression\_cle} est recherchée dans la table désignée par la valeur de \galgas{expression\_table} :
\begin{itemize}
  \item en cas d'échec, une erreur est affichée, son message étant fourni par la méthode de recherche \galgas{methode_recherche}, et sa localisation par le champ \galgas{location} de l'\galgas{expression\_cle} ;
  \item en cas de succès, ce sont les instructions \galgas{liste_instructions_do} qui sont exécutées ; les propriétés de l'élément recherché sont alors disponibles en lecture/écriture, sous leur nom éventuellement préfixé ; la clé est disponible en lecture seule sous le nom \galgas{lkey} éventuellement préfixé, et est de type \galgas{@lstring}.
\end{itemize}

Par exemple, on veut disposer d'une table qui implémente un \emph{counted set}, c'est à dire que l'on associe à la clé son nombre de citations, et la clé doit avoir été entrée auparavant. On déclare :
\begin{galgascode}
map @maTable {
  @uint mOccurrenceCount
  search searchKey error message "entree %K absente"
}
\end{galgascode}

Et on effectue des recherches de la façon suivante :
\begin{galgascode}
@lstring cle = ...
with cle in !?table error message searchKey do
  mOccurrenceCount ++
end
\end{galgascode}

En utilisant un préfixe, le code devient :
\begin{galgascode}
@lstring cle = ...
with cle : abc_ in !?table error message searchKey do
  abc_mOccurrenceCount ++
end
\end{galgascode}



%!TEX encoding = UTF-8 Unicode
%!TEX root = ../galgas-book.tex

%--------------------------------------------------------------
\chapter{Syntax and Grammar Components}
%-------------------------------------------------------------

\section {GALGAS and Context-Free Grammars}


\section{Writing a Syntax Component}\index{Component!Syntax}

\section{Syntax Instructions}

\subsection{Terminal Symbol Instruction}

\subsection{Non Terminal Symbol Instruction}


\subsection{Repeat Instruction}


\subsection{Select Instruction}



\subsection{Parse Instruction}

\subsubsection{Parse do ... Instruction}


\subsubsection{Parse loop ... Instruction}


\subsubsection{Parse when ... Instruction}


\section{Writing a Grammar Component}\index{Component!Grammar}



%!TEX encoding = UTF-8 Unicode
%!TEX root = ../galgas-book.tex

%--------------------------------------------------------------
\chapter{Graphic User Interface Component}\index{Component!Graphic User Interface}
%-------------------------------------------------------------


%!TEX encoding = UTF-8 Unicode
%!TEX root = ../galgas-book.tex

%--------------------------------------------------------------
\chapter{Program Component}\index{Component!Program}
%-------------------------------------------------------------



%-----------------------------------------------------------------------------------------------------------------------*
%                                                                                                                       *
%   I N D E X                                                                                                           *
%                                                                                                                       *
%-----------------------------------------------------------------------------------------------------------------------*

\cleardoublepage % Pour commencer à une page impaire
\phantomsection  % Pour faire correctement pointer l'hyperlien dans la table des matières

%--- Écrire l'index
{\small
\printindex
}

%-----------------------------------------------------------------------------------------------------------------------*
%                                                                                                                       *
%   F I N    D U    D O C U M E N T                                                                                     *
%                                                                                                                       *
%-----------------------------------------------------------------------------------------------------------------------*

\end{document}

%-----------------------------------------------------------------------------------------------------------------------*

