%!TEX TS-program = personallatex
%!TEX encoding = UTF-8 Unicode

% Le script utilisé pour compiler est :
%       xelatex --file-line-error --shell-escape --synctex=1

\documentclass[a4paper, 10pt, openany]{book}
\usepackage{verbatim}
\usepackage{filecontents}

%-----------------------------------------------------------------------------------------------------------------------*
%                                                                                                                       *
%   E N C O D A G E    D E S    S O U R C E S     :     U T F 8                                                         *
%                                                                                                                       *
%-----------------------------------------------------------------------------------------------------------------------*

%--------------------------- Pour compilation LaTex

%--- Ce paquetage permet d'effectuer certaines césures, et ainsi d'éviter les messages "Overfull \hbox"
%\usepackage[T1]{fontenc}

%--- Paquetage pour le codage des sources en UTF-8
%\usepackage[utf8x]{inputenc}

%--- Paquetage pour imposer les réglages français
%\usepackage[frenchb]{babel}

%--- Polices
%%\usepackage{fouriernc}
%%\usepackage[scaled=0.875]{helvet}

%\usepackage[scaled=0.85, default]{sourcecodepro}
%\usepackage[scaled=1.0, default]{sourcesanspro}


%--------------------------- Pour compilation XeLaTex
% http://www.tuteurs.ens.fr/logiciels/latex/xetex.html
\RequirePackage{etex}
\usepackage{fontspec}
\setmainfont{Verdana}[Ligatures=TeX]
\setsansfont{Courier}[Ligatures=NoCommon]
\setmonofont{Menlo}[Ligatures=NoCommon]
\usepackage{polyglossia}
\setdefaultlanguage{french}

%-----------------------------------------------------------------------------------------------------------------------*
%                                                                                                                       *
%   R É G L A G E S    « F R A N Ç A I S »                                                                              *
%                                                                                                                       *
%-----------------------------------------------------------------------------------------------------------------------*

%--- Latex demande ce paquetage pour mieux afficher le caractère "°" et \textquotesingle "'"
\usepackage{textcomp}

%--- Contrôle de l'indentation et de la séparation des paragraphes
\setlength{\parindent}{0pt} 
%\setlength{\parskip}{1.2ex} % Reporté avant les chapitres

%--- Ajouter une séparation à la fin des itemize
%\let\EndItemize\enditemize
%\def\enditemize{\EndItemize\vspace{1.2ex}}

%-----------------------------------------------------------------------------------------------------------------------*
%                                                                                                                       *
%   M I S E    E N    P A G E                                                                                           *
%                                                                                                                       *
%-----------------------------------------------------------------------------------------------------------------------*

%--- Interligne 1,5 ligne (voir http://tex.stackexchange.com/questions/819/double-line-spacing)
\usepackage{setspace}
\onehalfspacing

% Voir "Une courte introduction à Latex2e", § 6.4

%--- Marge gauche : 2,8 cm ; le paramètre \hoffset contient cette valeur, moins 1 pouce
%    \hoffset = 2,8 cm - 2,54 cm = 0,26 cm
\setlength{\hoffset}{0.26 cm}

%--- Marges supplémentaires, différenciées pour les pages gauches et droites ; ici, aucune.
\setlength{\oddsidemargin }{0 cm}
\setlength{\evensidemargin}{0 cm}

%--- Largeur du texte
%    \textwidth = 210 mm - 28 mm - 28 mm = 15,4 cm
\setlength{\textwidth}{15.4 cm}

%--- Marge haute : 2,8 cm ; le paramètre \voffset contient cette valeur, moins 1 pouce
%    \voffset = 2,8 cm - 2,54 cm = 0,26 cm
\setlength{\voffset}{0.26 cm}

%--- Distance entre la marge haute et l'en-tête : 0 cm
\setlength{\topmargin}{0 cm}

%--- Hauteur de l'en-tête de chaque page : 1 cm
\setlength{\headheight}{1 cm}

%--- Distance entre l'en-tête de chaque page et le corps : 0,5 cm
\setlength{\headsep}{0.5 cm}

%--- Hauteur du corps
%    \textheight = 29,7 cm - 2,8 cm - 2,8 cm - 1,5 cm = 22,6 cm
\setlength{\textheight}{22.6 cm}

%-----------------------------------------------------------------------------------------------------------------------*
%                                                                                                                       *
%   T R A C E R    U N E    L I G N E    H O R I Z O N T A L E                                                          *
%                                                                                                                       *
%-----------------------------------------------------------------------------------------------------------------------*

\newcommand\ligne{\hrulefill}

%-----------------------------------------------------------------------------------------------------------------------*
%                                                                                                                       *
%   E X T E N S I O N S    P O U R    L ' É C R I T U R E    D E S     F O R M U L E S    M A T H É M A T I Q U E S     *
%                                                                                                                       *
%-----------------------------------------------------------------------------------------------------------------------*

%--- Extensions pour l'écriture des formules mathématiques
\usepackage{amsmath}
\usepackage{amssymb}
\usepackage{amsfonts}

%--- Paquetage "IEEEtrantools"
% Pour créer des tableaux d'équations, bien alignées
% Voir courte-intro-latex.pdf, page §3.5.2 page 83
%\usepackage[retainorgcmds]{IEEEtrantools}

%-----------------------------------------------------------------------------------------------------------------------*
%                                                                                                                       *
%   E X T E N S I O N S    P O U R    P R É S E N T E R    L E S    T A B L E A U X                                     *
%                                                                                                                       *
%-----------------------------------------------------------------------------------------------------------------------*

%\usepackage{array}

%  http://en.wikibooks.org/wiki/LaTeX/Colors
\usepackage[dvipsnames, svgnames, table]{xcolor} % À placer avant \usepackage{listings}

%--- Ce paquetage permet de changer le style des légendes des tableaux et des figures (voir caption-eng.pdf) :
%  - l'étiquette est en italique gras
%  - le titre est en italique.
\usepackage[font=it, labelfont=bf]{caption}

%--- Par défaut, le caption de chaque tableau est préfixé par « Tab. ». La commande suivante impose le nom "Tableau"
% Voir http://fr.wikibooks.org/wiki/LaTeX/Éléments_flottants_et_figures
\addto\captionsfrench{\def\tablename{Tableau}}

%------------------------------------------------------------------------------------------ RÉFÉRENCES À UN TABLEAU
% La référence au tableau "nom-du-tableau" est définie par \labelTableau{nom-du-tableau}
\newcommand\labelTableau[1]{\label{tab:#1}}
% Latex autorise deux types d'appel à une référence \ref{tab:nom-du-tableau} et \pageref{tab:nom-du-tableau}

% \refTableau{}{nom-du-tableau} ---> "tableau x.y"   où x.y est le n° du tableau
\newcommand\refTableau[1]{\hyperref[tab:#1]{tableau \ref*{tab:#1}}}

% \refTableauSansPrefixe{}{nom-du-tableau} ---> "x.y"   où x.y est le n° du tableau
\newcommand\refTableauSansPrefixe[1]{\hyperref[tab:#1]{\ref*{tab:#1}}}

% \refTableauPage{}{nom-du-tableau} ---> "tableau x.y page n"   où x.y est le n° du tableau
\newcommand\refTableauPage[1]{\hyperref[tab:#1]{tableau \ref*{tab:#1} page \pageref{tab:#1}}}

% \refTableauPageSansPrefixe{}{nom-du-tableau} ---> "x.y page n"   où x.y est le n° du tableau
\newcommand\refTableauPageSansPrefixe[1]{\hyperref[tab:#1]{\ref*{tab:#1} page \pageref{tab:#1}}}

%-----------------------------------------------------------------------------------------------------------------------*
%                                                                                                                       *
%   E X T E N S I O N S    P O U R    P R É S E N T E R    L E S    F I G U R E S                                       *
%                                                                                                                       *
%-----------------------------------------------------------------------------------------------------------------------*

%--- Par défaut, le caption de chaque figure est préfixé par « Fig. ». La commande suivante impose le nom "Figure"
% Voir http://fr.wikibooks.org/wiki/LaTeX/Éléments_flottants_et_figures
\addto\captionsfrench{\def\figurename{Figure}}

% Grâce à ce paquetage, il est possible de placer plusieurs figures, tables, côte à côte.
% Voir http://en.wikibooks.org/wiki/LaTeX/Floats,_Figures_and_Captions
%\usepackage{subfig}

%-----------------------------------------------------------------------------------------------------------------------*
%                                                                                                                       *
%   P A Q U E T A G E S                                                                                                 *
%                                                                                                                       *
%-----------------------------------------------------------------------------------------------------------------------*

%--- Ce paquetage permet d'effectuer des tests : \ifthenelse{test}{bloc then}{bloc else}
\usepackage{ifthen}

%--- Affiche les sections dans le log
\usepackage{dprogress}

\usepackage{listings} % À placer après \usepackage[table]{xcolor}

\usepackage{lineno}

\usepackage{mdframed}

%-----------------------------------------------------------------------------------------------------------------------*
%                                                                                                                       *
%   T I K Z    -    P G F                                                                                               *
%                                                                                                                       *
%-----------------------------------------------------------------------------------------------------------------------*

\usepackage{tikz}
\usepackage{tkz-graph}
%\usepackage{pgfplots}
\usetikzlibrary{calc}
\usetikzlibrary{arrows}
\usetikzlibrary{decorations}
%\usetikzlibrary{decorations.pathmorphing}
%\usetikzlibrary{shapes.callouts}
\usetikzlibrary{shapes.misc}
\usetikzlibrary{automata}
\usetikzlibrary{positioning}
\usepgflibrary{shapes.geometric}

%-----------------------------------------------------------------------------------------------------------------------*
%                                                                                                                       *
%   I N C L U S I O N    D E S   M A C R O S   D E S T I N É E S   À   L ' É C R I T U R E   
%                                                                                                                       *
%                                        D E   C O D E   G A L G A S                                                    *
%                                                                                                                       *
%-----------------------------------------------------------------------------------------------------------------------*

%\usepackage{relsize}

%!TEX encoding = UTF-8 Unicode
%!TEX root = ../galgas-book.tex

%-----------------------------------------------------------------------------------------------------------------------*
%   A F F I C H A G E    D U    C O D E    G A L G A S                                                                  *
%-----------------------------------------------------------------------------------------------------------------------*

\newcommand\tpp[1]{\colorbox{gray!12}{\ttfamily #1}}

%-----------------------------------------------------------------------------------------------------------------------*
%   A F F I C H A G E    D U    C O D E    G A L G A S                                                                  *
%-----------------------------------------------------------------------------------------------------------------------*

\newcommand\keywordsStyleGalgas[1]{\textcolor{blue}{\textbf{#1}}}
\newcommand\delimitersStyleGalgas[1]{\textcolor{brown}{\textbf{#1}}}
\newcommand\selectorStyleGalgas[1]{\textcolor{orange}{#1}}
\newcommand\terminalStyleGalgas[1]{\textcolor{orange}{#1}}
\newcommand\nonTerminalStyleGalgas[1]{\textcolor{orange}{#1}}
\newcommand\integerStyleGalgas[1]{\textcolor{brown}{#1}}
\newcommand\floatStyleGalgas[1]{\textcolor{magenta}{#1}}
\newcommand\characterStyleGalgas[1]{\textcolor{cyan}{#1}}
\newcommand\stringStyleGalgas[1]{\textcolor{gray}{#1}}
\newcommand\typeNameStyleGalgas[1]{\textcolor{gray}{#1}}
\newcommand\attributeStyleGalgas[1]{\textcolor{brown}{#1}}
\newcommand\commentStyleGalgas[1]{\textcolor{red}{#1}}

%\newcommand\lexicalErrorGalgas{\textcolor{red}{\textbullet ERRLEX\textbullet}}

\newcommand\keywordsStylegalgas[1]{\textcolor{blue}{\textbf{#1}}}
\newcommand\delimitersStylegalgas[1]{\textcolor{brown}{\textbf{#1}}}
\newcommand\selectorStylegalgas[1]{\textcolor{orange}{#1}}
\newcommand\terminalStylegalgas[1]{\textcolor{orange}{#1}}
\newcommand\nonTerminalStylegalgas[1]{\textcolor{orange}{#1}}
\newcommand\integerStylegalgas[1]{\textcolor{brown}{#1}}
\newcommand\floatStylegalgas[1]{\textcolor{magenta}{#1}}
\newcommand\characterStylegalgas[1]{\textcolor{cyan}{#1}}
\newcommand\stringStylegalgas[1]{\textcolor{gray}{#1}}
\newcommand\typeNameStylegalgas[1]{\textcolor{gray}{#1}}
\newcommand\attributeStylegalgas[1]{\textcolor{brown}{#1}}
\newcommand\commentStylegalgas[1]{\textcolor{red}{#1}}

%\newcommand\lexicalErrorgalgas{\textcolor{red}{\textbullet ERRLEX\textbullet}}

\newmdenv[
  topline=false,
  bottomline=false,
  rightline=false,
%  skipabove=\topsep,
%  skipbelow=\topsep,
  linecolor=blue!25,
  linewidth=2pt
]{siderules}

\newwrite\tempfile

\makeatletter
\newenvironment{galgas}{%
  \begingroup
  \@bsphack
  \immediate\openout\tempfile=temp.galgas%
  \let\do\@makeother\dospecials
  \catcode`\^^M\active
  \verbatim@startline
  \verbatim@addtoline
  \verbatim@finish
  \def\verbatim@processline{\immediate\write\tempfile{\the\verbatim@line}}%
  \verbatim@start
}{
  \immediate\closeout\tempfile
  \@esphack
  \endgroup
  \immediate\write18{galgas --mode=latex:Galgas temp.galgas}
  {\singlespacing\begin{siderules}\ttfamily\input{temp.galgas.tex}\end{siderules}}
}
\makeatother

%-----------------------------------------------------------------------------------------------------------------------*
% COMMANDE \ggs : affichage de code en ligne galgas                                                                     *
%-----------------------------------------------------------------------------------------------------------------------*

\makeatletter
\newcommand*\ggs{%
  \@bsphack%
  \begingroup%
  \let\do\@makeother\dospecials%
  \let\do\do@noligs\verbatim@nolig@list%
  \catcode`\^^M=15\relax%
  \@vobeyspaces%
  \@ggs{\temporary}%
}%
\newcommand\@ggs[2]{%
  \catcode`-=12\relax%
  \catcode`<=12\relax%
  \catcode`>=12\relax%
  \catcode`,=12\relax%
  \catcode`'=12\relax%
  \catcode``=12\relax%
  \catcode`#2\active%
  \catcode`~\active%
  \lccode`\~`#2\relax%
  \begingroup%
  \lowercase{%
    \def\@tempa##1~{%
      \expandafter\endgroup%
      \expandafter\DeclareRobustCommand%
      \expandafter*%
      \expandafter#1%
      \expandafter{\@tempa}%
      \@esphack%
      \immediate\openout\tempfile=temp.galgas%
      \immediate\write\tempfile{##1}%
      \immediate\closeout\tempfile%
      \immediate\write18{galgas --mode=latex:Galgas temp.galgas}%
      \colorbox{gray!6}{\ttfamily\input{temp.galgas.tex}\unskip}%
    }%
  }%
  \ifnum`#2=`\~\else\@makeother\~\fi%
  \expandafter\endgroup%
  \@tempa%
}%
\makeatother

%-----------------------------------------------------------------------------------------------------------------------*
%                                                                                                                       *
% A F F I C H A G E    E T    C R O S S   R É F É R E N C E    D E S    T Y P E S    P R É D É F I N I S   G A L G A S  *
%                                                                                                                       *
%-----------------------------------------------------------------------------------------------------------------------*

%--- Les deux macros suivantes définissent une section et une sous-section :
%      - en formattant le titre
%      - en définissant un label pour cross référence ;
%      - en définissant une entrée dans l'index

% Exemple d'appel : \sectionTypePredefiniLabelIndex{bool}

\newcommand \chapitreTypePredefiniLabelIndex[1] {\chapter{Le type \texttt{@#1}}\label{type:#1}\index{Type!"@#1}}

\newcommand \sectionTypePredefiniLabelIndex[1] {\section{Le type \texttt{@#1}}\label{type:#1}\index{Type!"@#1}}

\newcommand \subsectionTypePredefiniLabelIndex[1] {\subsection{Le type \texttt{@#1}}\label{type:#1}\index{Type!"@#1}}

%--- Cette macro établit un hyperlien vers un type prédéfini
% Exemple d'appel : \refTypePredefini{bool} -- affiche --> @bool type (page xx)

\newcommand \refTypePredefini[1] {\hyperref[type:#1]{\texttt{@#1} (page \pageref{type:#1})}}

%-----------------------------------------------------------------------------------------------------------------------*
%   G E T T E R   C R O S S    R E F E R E N C I N G                                                                    *
%-----------------------------------------------------------------------------------------------------------------------*

\newcommand\subsectionGetter[2]{\subsection{Getter \texttt{#1}}\label{getter:#2:#1}\index{#1!"@#2 getter}}

%-----------------------------------------------------------------------------------------------------------------------*

% Exemple d'appel : \refGetterPage{bool}{string} -- affiche --> @bool string getter (page xx)
\newcommand\refGetterPage[2] {\hyperref[getter:#1:#2]{getter \texttt{@#1 #2} à la page \pageref{getter:#1:#2}}}

%-----------------------------------------------------------------------------------------------------------------------*
%   S E T T E R   C R O S S    R E F E R E N C I N G                                                                    *
%-----------------------------------------------------------------------------------------------------------------------*

\newcommand\subsectionSetter[2]{\subsection{Setter \texttt{#1}}\label{setter:#2:#1}\index{#1!"@#2 getter}}

%-----------------------------------------------------------------------------------------------------------------------*

% Exemple d'appel : \refSetterPage{bool}{string} -- affiche --> @bool string setter (page xx)
\newcommand\refSetterPage[2] {\hyperref[setter:#1:#2]{setter \texttt{@#1 #2} à la page \pageref{setter:#1:#2}}}


%-----------------------------------------------------------------------------------------------------------------------*
%   C O N S T R U C T O R   C R O S S    R E F E R E N C I N G                                                          *
%-----------------------------------------------------------------------------------------------------------------------*

\newcommand\subsectionConstructor[2]{\subsection{Constructeur \texttt{#1}}\label{constructor:#2:#1}\index{#1!"@#2 constructor}}

%-----------------------------------------------------------------------------------------------------------------------*

\newcommand \refConstructorPage[2] {\hyperref[constructor:#1:#2]{#2 constructor (page \pageref{constructor:#1:#2})}}

%-----------------------------------------------------------------------------------------------------------------------*


%-----------------------------------------------------------------------------------------------------------------------*
%                                                                                                                       *
%   P A Q U E T A G E    « M U L T I C O L »                                                                            *
%                                                                                                                       *
%-----------------------------------------------------------------------------------------------------------------------*

\usepackage{multicol}
\setlength{\columnsep}{30pt}
\setlength{\columnseprule}{1pt}

%-----------------------------------------------------------------------------------------------------------------------*
%                                                                                                                       *
%   P A Q U E T A G E    « L O N G T A B L E »                                                                          *
%                                                                                                                       *
%-----------------------------------------------------------------------------------------------------------------------*

%--- Pour afficher correctement des tables sur plusieurs pages
%\usepackage{longtable}

%-----------------------------------------------------------------------------------------------------------------------*
%                                                                                                                       *
%   E N - T Ê T E S    E T    P I E D S    D E    P A G E S                                                             *
%                                                                                                                       *
%-----------------------------------------------------------------------------------------------------------------------*

% Grâce au package "fancyhdr"
% voir http://www.exomatik.net/U-Latex/Personnaliser#toc2
%      http://www.trustonme.net/didactels/250.html
\usepackage{fancyhdr}
\pagestyle{fancy}
%--- Numéro de page : à gauche pages paires, à droite pages impaires
\fancyhead[EL,OR]{\thepage}
%--- Nom de chapitre : à droite page paires
\fancyhead[ER]{\leftmark}
%--- Nom de section : à gauche page impaires
\fancyhead[OL]{\rightmark}
%--- Version GALGAS : au milieu du pied de chaque page
\fancyfoot[C]{GALGAS, version GALGASBETAVERSION}
%--- filet en haut et en bas de chaque page
\renewcommand{\headrulewidth}{0.5 pt}
\renewcommand{\footrulewidth}{0.5 pt}

%\renewcommand{\chaptermark}[1]{\markboth{\bsc{\chaptername~\thechapter{}.} #1}{}}
%\renewcommand{\sectionmark}[1]{\markright{\bsc{\thesection{}.} #1}{}}

%-----------------------------------------------------------------------------------------------------------------------*
%                                                                                                                       *
%   C O N T R Ô L E    D E   L A   T A B L E   D E S   M A T I È R E S                                                  *
%                                                                                                                       *
%-----------------------------------------------------------------------------------------------------------------------*

% http://tex.stackexchange.com/questions/50471/question-about-indent-lengths-in-toc
\usepackage{tocloft}

%--- Gérer de l'indentation dans la table des matières
%\cftsetindents{part}{0.0em}{2.0em}
\cftsetindents{chapter}{0.0em}{2.0em}
\cftsetindents{section}{2.0em}{3.5em}
\cftsetindents{subsection}{5.5em}{4.5em}

%    Pour faire figurer la liste des tableaux (et la table des matières) dans la table des matières
\usepackage{tocbibind}

%--- Pour afficher dans la tables des matières jusqu'au niveau 2 (subsection)
\setcounter{tocdepth}{2}

%--- Par défaut, les parties ne sont pas numérotées dans la table des matières ; la commande suivante
%    ajoute une numération en chiffres romains
%    http://tex.stackexchange.com/questions/193809/add-part-string-to-part-list-in-toc
\makeatletter
\def\@part[#1]#2{\ifnum \c@secnumdepth >-2\relax
        \refstepcounter{part}%
        \addcontentsline{toc}{part}{\Roman{part} \thepart
        \hspace{1em}#1}\else
        \addcontentsline{toc}{part}{#1}\fi
   \markboth{}{}%
   {\centering
    \interlinepenalty \@M
    \ifnum \c@secnumdepth >-2\relax
      \huge\bf \partname~\thepart
    \par
    \vskip 20\p@\fi
    \Huge \bf
    #2\par}\@endpart}
\makeatother

%\renewcommand\cftpartpresnum{Part~}

%-----------------------------------------------------------------------------------------------------------------------*
%                                                                                                                       *
%   G E S T I O N    D E    L ' I N D E X                                                                               *
%                                                                                                                       *
%-----------------------------------------------------------------------------------------------------------------------*

% http://www.cuk.ch/articles/4097
% http://www.tuteurs.ens.fr/logiciels/latex/makeindex.html
% http://linux.die.net/man/1/makeindex
%
% Attention ! Les deux commandes suivantes, ainsi que le \printindex placé plus bas ne
% sont pas suffisants pour construire l'index : il faut utiliser l'utilitaire "makeIndex"
% Voir le fichier de commande "build.command"
\usepackage{makeidx}
\makeindex

%-----------------------------------------------------------------------------------------------------------------------*
%                                                                                                                       *
%   H Y P E R R E F                                                                                                     *
%                                                                                                                       *
%-----------------------------------------------------------------------------------------------------------------------*

%--- Pour les hyperliens, et le contrôle de la génération PDF 
\usepackage[xetex]{hyperref}
\usepackage{bookmark} % see http://tex.stackexchange.com/questions/176113/problem-with-in-pdf-bookmark-under-xelatex
\hypersetup{colorlinks=true}
\hypersetup{linkcolor=blue}
\hypersetup{breaklinks=true}

%-----------------------------------------------------------------------------------------------------------------------*
%                                                                                                                       *
%   R É F É R E N C E S                                                                                                 *
%                                                                                                                       *
%-----------------------------------------------------------------------------------------------------------------------*

% Au lieu d'écrire \chapter{titre-chapitre}, on écrit \chapterLabel{titre-chapitre}{label-chapitre}
\newcommand\chapterLabel[2]{\chapter{#1}\label{chapter:#2}}

% \refChapter{label-chapter} ---> "chapitre n"
\newcommand\refChapter[1]{\hyperref[chapter:#1]{chapitre \ref*{chapter:#1}}}

\newcommand\refChapterPage[1]{\hyperref[chapter:#1]{chapitre \ref*{chapter:#1} page \pageref{chapter:#1}}}

% Au lieu d'écrire \section{titre-section}, on écrit \sectionLabel{titre-section}{label-section}
\newcommand\sectionLabel[2]{\section{#1}\label{sec:#2}}

% \refSectionPage{label-section} ---> "section x.y page n"   où x.y est le n° de la section
\newcommand\refSectionPage[1]{\hyperref[sec:#1]{section \ref*{sec:#1} page \pageref{sec:#1}}}

%------------------------------------------------------------------------------------------ RÉFÉRENCES À UNE SUB-SECTION
% Au lieu d'écrire \subsection{titre-section}, on écrit \subsectionLabel{titre-section}{label-section}
\newcommand\subsectionLabel[2]{\subsection{#1}\label{subsec:#2}}


% \refSubsectionPage{label-section} ---> "section x.y page n"   où x.y est le n° de la sub-section
\newcommand\refSubsectionPage[1]{\hyperref[subsec:#1]{section \ref*{subsec:#1} page \pageref{subsec:#1}}}

%------------------------------------------------------------------------------------------ RÉFÉRENCES À UNE SUB-SUB-SECTION
% Au lieu d'écrire \subsection{titre-section}, on écrit \subsectionLabel{titre-section}{label-section}
\newcommand\subsubsectionLabel[2]{\subsubsection{#1}\label{subsubsec:#2}}

% \refSubsectionPage{label-section} ---> "section x.y page n"   où x.y est le n° de la sub-section
\newcommand\refSubsubsectionPage[1]{\hyperref[subsubsec:#1]{section \ref*{subsubsec:#1} page \pageref{subsubsec:#1}}}

%------------------------------------------------------------------------------------------ RÉFÉRENCES À UNE FIGURE
% La référence au tableau "nom-de-la-figure" est définie par \labelFigure{nom-de-la-figure}
\newcommand\labelFigure[1]{\label{fig:#1}}
% Latex autorise deux types d'appel à une référence \ref{fig:nom-de-la-figure} et \pageref{fig:nom-de-la-figure}

% \refFigure{}{nom-de-la-figure}   ---> "figure x.y"   où x.y est le n° de la figure
% \refFigure{z}{nom-de-la-figure}  ---> "figure x.y.z" où x.y est le n° de la figure
\newcommand\refFigure[2]{\hyperref[fig:#2]{figure \ref*{fig:#2}{\ifthenelse{\equal{#1}{}}{}{.#1}}}}

% \refFigureSansPrefixe{}{nom-de-la-figure}   ---> "x.y"   où x.y est le n° de la figure
% \refFigureSansPrefixe{z}{nom-de-la-figure}  ---> "x.y.z" où x.y est le n° de la figure
\newcommand\refFigureSansPrefixe[2]{\hyperref[fig:#2]{\ref*{fig:#2}{\ifthenelse{\equal{#1}{}}{}{.#1}}}}

% \refFigurePage{}{nom-de-la-figure}   ---> "figure x.y page n"   où x.y est le n° de la figure
% \refFigurePage{z}{nom-de-la-figure}  ---> "figure x.y.z page n" où x.y est le n° de la figure
\newcommand\refFigurePage[2]{\hyperref[fig:#2]{figure \ref*{fig:#2}{\ifthenelse{\equal{#1}{}}{}{.#1}} page \pageref{fig:#2}}}

% \refFigurePageSansPrefixe{}{nom-de-la-figure}   ---> "x.y page n"   où x.y est le n° de la figure
% \refFigurePageSansPrefixe{z}{nom-de-la-figure}  ---> "x.y.z page n" où x.y est le n° de la figure
\newcommand\refFigurePageSansPrefixe[2]{\hyperref[fig:#2]{\ref*{fig:#2}{\ifthenelse{\equal{#1}{}}{}{.#1}} page \pageref{fig:#2}}}

%-----------------------------------------------------------------------------------------------------------------------*
%                                                                                                                       *
%   AFFICHAGE D'UNE OPTION DE LA LIGNE DE COMMANDE                                                                      *
%                                                                                                                       *
%-----------------------------------------------------------------------------------------------------------------------*

\newcommand\optionGGS[1]{\colorbox{gray!10}{\small\bf\textbf{#1}}}

%-----------------------------------------------------------------------------------------------------------------------*
%                                                                                                                       *
%   D É B U T    D U    D O C U M E N T                                                                                 *
%                                                                                                                       *
%-----------------------------------------------------------------------------------------------------------------------*


\begin{document} 

%-----------------------------------------------------------------------------------------------------------------------*
%                                                                                                                       *
%   P A G E    D E    T I T R E                                                                                         *
%                                                                                                                       *
%-----------------------------------------------------------------------------------------------------------------------*

\title{\Huge{\textbf{GALGAS}}\\~\\ \normalsize{Version GALGASBETAVERSION}}
\author{Jean-Luc Béchennec\\Mikaël Briday\\Pierre Molinaro}
\date \today 

\maketitle

%-----------------------------------------------------------------------------------------------------------------------*
%                                                                                                                       *
%   T A B L E    D E S    M A T I È R E S                                                                               *
%                                                                                                                       *
%-----------------------------------------------------------------------------------------------------------------------*

\tableofcontents
 
%-----------------------------------------------------------------------------------------------------------------------*
%                                                                                                                       *
%   L I S T E    D E S    T A B L E A U X                                                                               *
%                                                                                                                       *
%-----------------------------------------------------------------------------------------------------------------------*

\listoftables
\addtocontents{lot}{\protect\thispagestyle{empty}\protect\pagestyle{empty}}

%-----------------------------------------------------------------------------------------------------------------------*
%                                                                                                                       *
%   L I S T E    D E S    F I G U R E S                                                                                 *
%                                                                                                                       *
%-----------------------------------------------------------------------------------------------------------------------*

\listoffigures
\addtocontents{lof}{\protect\thispagestyle{empty}\protect\pagestyle{empty}}

%-----------------------------------------------------------------------------------------------------------------------*
%                                                                                                                       *
%   L E S    C H A P I T R E S                                                                                          *
%                                                                                                                       *
%-----------------------------------------------------------------------------------------------------------------------*

%--- Contrôle de la séparation des paragraphes
%    On met cette définition ici, sinon elle affecte la table des matières, la liste des tableaux, ...
\setlength{\parskip}{1.2ex}

\part{Utilisation}
  %!TEX encoding = UTF-8 Unicode
%!TEX root = ../galgas-book.tex

%--------------------------------------------------------------
\chapter{Tutorial}
%-------------------------------------------------------------

Le but de ce tutorial est de construire en utilisant GALGAS un compilateur d’un langage inspiré de LOGO, qui fournit en sortie un fichier SVG contenant les tracés définis par un programme source LOGO.


Il est rédigé pour être réalisé sur Unix (Linux, Mac OS X).

La génération des fichiers SVG à partir de GALGAS sera faite par un template.

Vous trouverez des informations sur le format SVG sur la page :

\url{http://www.canarlake.org/index.cgi?theme=svg}

\section{Présentation du langage LOGO}

Vous trouverez une description précise du langage à la fin de cette section. Un programme LOGO décrit le déplacement d'une tortue dans un plan. Celle-ci peut effectuer des déplacements en ligne droite et des rotations sur place. La tortue est munie d'un crayon, qui peut être abaissé ou levé. Un déplacement provoque le tracé d'un segment de droite si le crayon est abaissé.

Un programme LOGO est contenu dans un fichier texte, d'extension \texttt{.logo}. Il comprend une liste (éventuellement vide) de sous-programmes, et une liste d'instructions. L'exécution du programme consiste à exécuter cette liste d'instructions, en appelant les sous-programmes qui y figurent. Initialement, la tortue est en (0, 0), sa direction est 0° (horizontale, vers la droite), et le crayon est levé.


\subsection{Quelques exemples}

Voici quelques exemples de programmes LOGO (\refTableau{carreEtoilePentagoneLogo} et \refTableauPage{hexagoneOctogoneLogo}). Le \refTableauPage{logoErreurSemantiques} liste des programmes présentant des erreurs sémantiques : le compilateur qui va être écrit décelera ces erreurs.

\begin{table}[t]
  \centering
  \small

\begin{multicols}{3}

\textbf{Dessin d'un carré}
\begin{lstlisting}
PROGRAM

  ROUTINE trace
  BEGIN
    FORWARD 50 ;
    ROTATE 90 ;
  END

BEGIN
  FORWARD 100 ;
  ROTATE 90 ;
  FORWARD 100 ;
  ROTATE 270 ;
  PEN DOWN ;
  CALL trace ;
  CALL trace ;
  CALL trace ;
  CALL trace ;
END.
\end{lstlisting}

\columnbreak

\textbf{Dessin d'une étoile}
\begin{lstlisting}
PROGRAM

  ROUTINE trace
  BEGIN
    FORWARD 70;
    ROTATE 160;
  END

  ROUTINE trace3
  BEGIN
    CALL trace;
    CALL trace;
    CALL trace;
  END

BEGIN
  FORWARD 200;
  ROTATE 90;
  FORWARD 300;
  ROTATE 270;
  PEN DOWN;
  CALL trace3;
  CALL trace3;
  CALL trace3;
END.
\end{lstlisting}

\columnbreak

\textbf{Dessin d'un pentagone}
\begin{lstlisting}
PROGRAM

  ROUTINE trace
  BEGIN
    FORWARD 70;
    ROTATE 72;
  END

BEGIN
  FORWARD 200;
  ROTATE 90;
  FORWARD 300;
  ROTATE 270;
  PEN DOWN;
  CALL trace;
  CALL trace;
  CALL trace;
  CALL trace;
  CALL trace;
END.
\end{lstlisting}

\end{multicols}

  \caption{Carré, étoile et pentagone en LOGO}
  \labelTableau{carreEtoilePentagoneLogo}
  \ligne
\end{table}


\begin{table}[t]
  \centering
  \small


\begin{multicols}{2}

\textbf{Dessin d'un hexagone}

\begin{lstlisting}
PROGRAM

  ROUTINE trace
  BEGIN
    FORWARD 70 ;
    ROTATE 60 ;
  END

BEGIN
  FORWARD 100 ;
  ROTATE 90;
  FORWARD 100;
  ROTATE 270;
  PEN DOWN;
  CALL trace;
  CALL trace;
  CALL trace;
  CALL trace;
  CALL trace;
  CALL trace;
END.
\end{lstlisting}


\columnbreak

\textbf{Dessin d'un octogone}

\begin{lstlisting}
PROGRAM

  ROUTINE trace
  BEGIN
    FORWARD 70;
    ROTATE 45;
  END

  ROUTINE trace1
  BEGIN
  CALL trace;
  CALL trace;
  END

  ROUTINE trace2
  BEGIN
  CALL trace1;
  CALL trace1;
  END

  ROUTINE trace3
  BEGIN
  CALL trace2;
  CALL trace2;
  END

BEGIN
  FORWARD 100;
  ROTATE 90;
  FORWARD 100;
  ROTATE 270;
  PEN DOWN;
  CALL trace3;
END.
\end{lstlisting}

\end{multicols}

  \caption{Hexagone et octogone en LOGO}
  \labelTableau{hexagoneOctogoneLogo}
  \ligne
\end{table}




\begin{table}[t]
  \centering
  \small

\begin{multicols}{3}

\textbf{Routine récursive}

\begin{lstlisting}
PROGRAM

  ROUTINE routine
  BEGIN
    CALL routine;
  END
  
BEGIN
END.
\end{lstlisting}

\columnbreak

\textbf{Routine indéfinie}

\begin{lstlisting}
PROGRAM

BEGIN
  CALL routine;
END.
\end{lstlisting}

\columnbreak
\textbf{Routine définie plusieurs fois}

\begin{lstlisting}
PROGRAM

  ROUTINE routine
  BEGIN
  END
  
  ROUTINE routine
  BEGIN
  END
  
BEGIN
END.
\end{lstlisting}

\end{multicols}

  \caption{Programmes LOGO contenant des erreurs sémantiques}
  \labelTableau{logoErreurSemantiques}
  \ligne
\end{table}

\subsection{Définition lexicale}

Les identificateurs sont constitués d'une séquence non vide de lettres minuscules ou majuscules. La casse est significative.

Les mots réservés sont les identificateurs suivants : \texttt{PROGRAM}, \texttt{ROUTINE}, \texttt{BEGIN}, \texttt{END}, \texttt{FORWARD}, \texttt{ROTATE}, \texttt{PEN}, \texttt{UP}, \texttt{DOWN} et \texttt{CALL}.

Les constantes littérales entières sont écrites en décimal (une séquence non vide de chiffres décimaux).

Les séparateurs sont tous les caractères compris entre le point de code Unicode \galgas{'\u0001'} et l’espace (point de code Unicode \galgas{'\u0020'}), ce qui inclut la tabulation horizontale et les différentes formes de la fin de ligne.

Les délimiteurs sont le point ('.') et le point virgule (';').

Un commentaire commence par le caractère dièse (\#) et s'étend jusqu'à la fin de la ligne courante.


\subsectionLabel{Définition syntaxique}{definitionSyntaxiqueLOGO}

Un programme LOGO commence le mot réservé \texttt{PROGRAM}, est suivi d'une liste éventuellement vide de définition de routines, du mot réservé \texttt{BEGIN}, d'une liste éventuellement vide d'instructions, et se termine par le mot réservé \texttt{END} suivi d'un point.

Une définition de routine est introduite par le mot réservé \texttt{ROUTINE}, est suivi d'un identificateur, du mot réservé \texttt{BEGIN}, d'une liste éventuellement vide d'instructions, et se termine par le mot réservé \texttt{END}.

Une instruction LOGO est une des séquences suivantes :
\begin{itemize}
  \item le mot réservé \texttt{FORWARD} suivi d'un entier littéral et d'un point virgule ;
  \item le mot réservé \texttt{ROTATE} suivi d'un entier littéral et d'un point virgule ;
  \item le mot réservé \texttt{PEN} suivi du mot réservé \texttt{UP} et d'un point virgule ;
  \item le mot réservé \texttt{PEN} suivi du mot réservé \texttt{DOWN} et d'un point virgule ;
  \item le mot réservé \texttt{CALL} suivi d'un identificateur et d'un point virgule.
\end{itemize}

\subsectionLabel{Sémantique statique}{semantiqueStatiqueLOGO}

Dans une définition de routine, l'identificateur qui suit le mot réservé \texttt{ROUTINE} est le nom de la routine définie. Dans une instruction \texttt{CALL}, l'identificateur est le nom de la routine appelée.

Contraintes (voir le \refTableau{logoErreurSemantiques} pour des exemples de programmes contenant des erreurs sémantiques) :
\begin{itemize}
  \item le nom d'une routine est unique (on n'a pas le droit de définir plusieurs routines de même nom) ;
  \item une instruction \texttt{CALL} ne peut nommer qu'une routine qui a été définie plus haut dans le texte ;
  \item la routine courante ne peut pas être appelée récursivement.
\end{itemize}

\subsectionLabel{Sémantique dynamique}{semantiqueDynamiqueLOGO}

L'espace de déplacement de la tortue est un plan, muni du repère orthonormé direct habituel. À un instant donné, l'état de la tortue est caractérisé par :
\begin{itemize}
  \item sa position dans le plan ;
  \item sa direction, mesuré en degrés à partir de l'axe horizontal, et dans le sens trigonométrique ;
  \item la position du crayon (levé ou abaissé).
\end{itemize}

Initialement, la position de la tortue est (0, 0), sa direction est 0°, et le crayon est levé.

L'exécution de chaque instruction a l'effet suivant :
\begin{itemize}
  \item l'instruction \texttt{FORWARD} avance la souris dans sa direction d'une longueur égale à la valeur de la constante entière contenue dans l'instruction ; si le crayon est abaissé, un segment de droite délimité par les positions de départ et d'arrivée de la tortue est tracé ;
  \item l'instruction \texttt{ROTATE} fait tourner la tortue dans le sens trigonométrique d'un nombre de degrés égal à la constante contenue dans l'instruction ; aucun tracé n'a lieu, quelque l'état du crayon.
  \item l'instruction \texttt{PEN UP} relève le crayon ;
  \item l'instruction \texttt{PEN DOWN} abaisse le crayon ;
  \item l'instruction \texttt{CALL} exécute le sous-programme nommé dans l'instruction.
\end{itemize}











\section{Installation de GALGAS}

Aller sur la page \url{http://galgas.rts-software.org/download/}

GALGAS est un utilitaire en ligne de commande (sauf sur Mac, pour lequel une application Cocoa est disponible). Vous pouvez :
\begin{itemize}
  \item soit télécharger le binaire correspondant à votre plateforme (pour l'installer, aller à la \refSubsectionPage{installerGalgas}) ;
  \item soit télécharger les sources et les recompiler.
\end{itemize}




\subsection{Téléchargement des sources et compilation}

Télécharger l’archive contenant les sources pour Unix et Mac.

Décompressez cette archive et placer le répertoire obtenu (galgas) dans un répertoire dont le chemin ne contient ni espace ni caractère accentué. C'est important car les chemins utilisés dans les makefile de GALGAS sont relatifs.

Dans la suite de la compilation GALGAS, tous les chemins sont indiqués relativement à ce répertoire, qui sera appelé \texttt{constructionGALGAS}.

Donc, vous devez obtenir à la suite de la décompression le répertoire \texttt{constructionGALGAS/galgas}.

Nous allons maintenant compiler GALGAS. Avec le terminal, sur Linux :
\begin{description}
  \item[ ] \texttt{cd constructionGALGAS/galgas/makefile-unix}
  \item[ ] \texttt{make}
\end{description}

Sur Mac :
\begin{description}
  \item[ ] \texttt{cd constructionGALGAS/galgas/makefile-macosx}
  \item[ ] \texttt{make}
\end{description}

La compilation de GALGAS peut prendre une dizaine de minutes. Deux exécutables sont produits :

\begin{itemize}
  \item \texttt{constructionGALGAS/galgas/makefile-unix/galgas} ;
  \item \texttt{constructionGALGAS/galgas/makefile-unix/galgas-debug}.
\end{itemize}

Les deux exécutables sont fonctionnellement identiques. Le premier est celui que vous utiliserez. Le second est la version debug du premier : il est exécuté avec de nombreuses vérifications internes, ce qui fait qu’il est beaucoup lent. Si le premier plante brutalement, on peut utiliser le second pour déceler si une erreur interne peut être mise en évidence.

La section suivante indique comment installer les binaires obtenus.

\subsectionLabel{Installation}{installerGalgas}


Pour pouvoir appeler les exécutables sans avoir besoin de mentionner un chemin, vous avez plusieurs possibilités :
\begin{itemize}
  \item le copier dans le répertoire \texttt{/bin} :  \texttt{sudo cp galgas /bin/}
  \item le copier dans votre répertoire local \texttt{bin} : \texttt{cp galgas $\sim$/bin/}
\end{itemize}

Attention, le répertoire \texttt{$\sim$/bin} n'existe peut-être pas pour votre compte : il faut alors le créer, et l'ajouter dans la variable \texttt{\$PATH}.

Sur Linux :
\begin{description}
  \item[ ] \texttt{mkdir $\sim$/bin/}
  \item[ ] \texttt{echo \textquotesingle export~PATH=\$PATH:$\sim$/bin\textquotesingle~\textgreater{}\textgreater~/home/user/.bashrc}
\end{description}

Sur Mac :
\begin{description}
  \item[ ] \texttt{mkdir $\sim$/bin/}
  \item[ ] \texttt{echo \textquotesingle export~PATH=\$PATH:$\sim$/bin\textquotesingle~\textgreater{}\textgreater~$\sim$/.bash\_profile}
\end{description}













\section{Création du squelette du compilateur LOGO}

Un projet GALGAS nécessite la mise en place de nombreux fichiers, de créer des makefile pour différentes plateformes, … 

Appeler galgas avec l'option \texttt{-{}-create-project} permet de créer automatique un projet prêt à être utilisé.

Pour tout le tutorial vous devez utiliser un répertoire dont le chemin ne contient ni espace ni caractère accentué. C'est important car les chemins utilisés dans les makefile de GALGAS sont relatifs.

Dans toute la suite de ce tutorial, les chemins sont indiqués relativement à ce répertoire, qui sera appelé \texttt{chezmoi}.

La création :
\begin{description}
  \item[ ] \texttt{cd chezmoi}
  \item[ ] \texttt{galgas -{}-create-project=logo}
\end{description}

Le message affiché par cette opération est :
\begin{description}
  \item[ ] \texttt{*** PERFORM PROJECT CREATION (-{}-create-project=logo option) ***}
  \item[ ] \texttt{*** DONE ***}
\end{description}

L’affichage de \texttt{DONE} indique que la création s’est effectuée avec succès : un répertoire nommé \texttt{logo} a été créé dans le répertoire \texttt{chezmoi}.

\subsection{Visite guidée du répertoire créé}

Dans le répertoire \texttt{chezmoi/logo} :
\begin{itemize}
  \item le fichier \texttt{+logo.galgasProject} est le fichier projet, c’est lui que vous compilerez ;
  \item le répertoire \texttt{galgas-sources} contient les fichiers sources que vous allez compléter tout au long de ce tutorial ; son contenu est indiqué dans le tableau suivant.
\end{itemize}

\begin{table}[t]
  \centering
  \begin{tabular}{ll}
    \textbf{Fichier} & \textbf{Description}\\
    \texttt{logo-lexique.galgas} & Définit l'analyseur lexical\\
    \texttt{logo-semantics.galgas} & Définit les types pour la sémantique\\
    \texttt{logo-syntax.galgas} & Définit les règles de production de la grammaire \\
    \texttt{logo-grammar.galgas} & Définit la grammaire (axiome, classe) \\
    \texttt{logo-program.galgas} & Définit la routine principale \\
    \texttt{logo-cocoa.galgas} & Définit l’interface pour Cocoa : utile uniquement sous Mac \\
    \texttt{logo-options.galgas} & Définit les options de la ligne de commande \\
  \end{tabular}
  \caption{Contenu des sous-répertoires de \texttt{logo} après compilation GALGAS}
  \labelTableau{tableauRepertoireLOGOapresCompGALGAS}
  \ligne
\end{table}






\subsection{Première compilation du projet}

Une compilation s'effectue en deux temps :
\begin{enumerate}
  \item d'abord une compilation GALGAS qui crée ou met à jour des fichiers C++ ;
  \item ensuite une compilation C++.
\end{enumerate}


\subsubsection{Compilation GALGAS}

Vous devez d'abord compiler les sources GALGAS :
\begin{description}
  \item[ ] \texttt{galgas -v chezmoi/logo/+logo.galgasProject}
\end{description}

L'option \texttt{-v} (verbose) provoque l'affichage de messages : observez ceux qui indiquent la création des fichiers C++. Ceux-ci sont rangés dans le répertoire \texttt{chezmoi/logo/build/output} et \texttt{chezmoi/logo/build/libpm}.

Le répertoire \texttt{logo} est complété par de nouveaux répertoires (\refFigure{}{figureRepertoireLOGOapresCompGALGAS} et \refTableau{tableauRepertoireLOGOapresCompGALGAS}).
\begin{figure}[t]
  \centering
  \includegraphics{partie-utilisation/repertoire-logo.pdf}
  \caption{Répertoire \texttt{logo} après compilation GALGAS}
  \labelFigure{figureRepertoireLOGOapresCompGALGAS}
  \ligne
\end{figure}


\begin{table}[t]
  \centering
  \begin{tabular}{ll}
    \textbf{Répertoire} & \textbf{Contenu} \\
    \texttt{makefile-macosx} & Makefile pour compiler sur Mac OS X \\
    \texttt{makefile-msys32-on-windows} & Makefile pour compiler sur Win 32 \\
    \texttt{makefile-unix} & Makefile pour compiler sur Unix \\
    \texttt{makefile-win32-on-macosx} & Makefile pour compiler sur Mac OS X pour Win 32 \\
    \texttt{makefile-x86linux32-on-macosx} & Makefile pour compiler sur Mac OS X pour x86 Linux 32 bits \\
    \texttt{makefile-x86linux64-on-macosx} & Makefile pour compiler sur Mac OS X pour x86 Linux 64 bits \\
    \texttt{xcode-project} & Projet Xcode pour compiler sur Mac OS X
  \end{tabular}
  \caption{Contenu des sous-répertoires de \texttt{logo} après compilation GALGAS}
  \labelTableau{tableauRepertoireLOGOapresCompGALGAS}
  \ligne
\end{table}



\subsubsection{Compilation C++}
Choisissez le répertoire correspondant à votre plateforme (\texttt{makefile-macosx} ou \texttt{makefile-unix}) et lancer le script de compilation \texttt{build.py} (soit via la ligne de commande, soit en double-cliquant).

Par exemple :
\begin{description}
  \item[ ] \texttt{chezmoi/logo/makefile-unix/build.py}
\end{description}

Vous obtenez deux exécutables :
\begin{description}
  \item[ ] \texttt{chezmoi/logo/makefile-unix/logo}
  \item[ ] \texttt{chezmoi/logo/makefile-unix/logo-debug}
\end{description}

Sous Mac, vous pouvez utiliser le projet Xcode engendré, et ainsi créer une application Cocoa nommée \texttt{CocoaLogo}.

Dans tous les cas, comme les analyseurs lexicaux et syntaxiques sont vides après la création, les exécutables ainsi obtenus ne sont pas exploitables.


\section {Analyseur lexical}


Dans cette partie, vous allez écrire l’analyseur lexical du langage LOGO. Pour cela, vous allez modifier le fichier \texttt{chezmoi/logo/galgas-sources/logo-lexique.galgas}.

Remarques préliminaires :
\begin{enumerate}
  \item en GALGAS, tous les symboles terminaux sont notés par une chaîne de caractères non vide délimitée par deux caractères \galgas{$} ; par exemple : \galgas{$identifier$}, \galgas{$integer$}, … 
  \item en GALGAS, un nom de type est un identificateur précédé du caractère \galgas{@} ; par exemple : \galgas{@string}, \galgas{@uint}, \galgas{@lstring}, \galgas{@luint}, … ;
  \item le type \galgas{@string} définit une valeur chaîne de caractères ;
  \item le type \galgas{@uint} définit une valeur entière non signée sur 32 bits ;
  \item le type \galgas{@lstring} définit une valeur composée d'une chaîne de caractères et d'une information de localisation sur la position de la chaîne dans le texte source ;
  \item le type \galgas{@luint} définit une valeur composée d'une valeur entière non signée et d'une information de localisation sur la position de la chaîne dans le texte source ;
  \item ces informations de localisation sont à la base du signalement d'erreur.
\end{enumerate}

\subsection{Analyse lexicale d'un identificateur et d'un mot réservé}

Par défaut, une analyse lexicale des identificateurs et une liste de mots réservés est présente. Tout ce que vous avez à faire est de modifier la liste des mots réservés pour y placer ceux du langage LOGO.

Voici les lignes correspondantes :

\begin{galgascode}
@string tokenString
style keywordsStyle -> "Keywords"
$identifier$ ! tokenString error message "an identifier"

list keyWordList style keywordsStyle error message "the '%K' keyword" {
  "begin",
  "end"
}

rule 'a'->'z' | 'A'->'Z' {
  repeat
    enterCharacterIntoString (!?tokenString !*)
  while 'a'->'z' | 'A'->'Z' | '_' | '0'->'9' :
  end
  send search tokenString in keyWordList default $identifier$
}
\end{galgascode}

Explications :
\begin{enumerate}
  \item \galgas{@string tokenString} déclare l’attribut lexical tokenString de type chaîne de caractères ; au début de l’analyse de chaque token, cet attribut est initialisé à la valeur chaîne vide ;
  \item \galgas{style keywordsStyle -> "Keywords"} déclare un style (uniquement utile pour l’application Cocoa engendrée, vous pouvez ignorer cette ligne) ;
  \item \galgas{$identifier$ ! tokenString error message "an identifier"} déclare le terminal \galgas{$identifier$} qui sera transmis à l’analyseur syntaxique accompagné de la valeur de \galgas{tokenString} ; le message d’erreur qui suit est celui qui est utilisé lors d’une erreur syntaxique ;
  \item \galgas{list keyWordList style keywordsStyle error ...} déclare une liste de mots réservés associés à un style d’affichage (pour l’application Cocoa sur Mac), un message d’erreur syntaxique ; telle qu’elle est présente, cette définition déclare les deux non terminaux \galgas{$begin$} et \galgas{$end$} ;
  \item enfin, \galgas{rule 'a'->'z' | 'A'->'Z' ...} effectue l’analyse lexicale des identificateurs en accumulant dans \galgas{tokenString} les caractères rencontrés ; la recherche d'un mot réservé est effectuée par\galgas{send search tokenString in keyWordList default $identifier$} : par défaut si la chaîne entrée n'est pas un mot réservé, un identificateur est retourné à l’analyseur syntaxique.
\end{enumerate}

\textbf{Travail à faire.} Modifier la liste des mots réservés en y plaçant ceux du langage LOGO.
%\begin{galgascode}
%list keyWordList style keywordsStyle error message "the '%K' keyword" {
%  "PROGRAM", "ROUTINE", "BEGIN", "END", "FORWARD",
%  "ROTATE",  "CALL",    "PEN",   "UP",  "DOWN"
%}
%\end{galgascode}

\subsection{Analyse lexicale d'une constante entière}

L’analyse lexicale d’une constante entière 32 bits non signée est présente par défaut, vous n’avez rien à ajouter.

Voici l'écriture correspondante :
\begin{galgascode}
style integerStyle -> "Integer Constants"
@uint uint32value
$integer$ !uint32value style integerStyle
               error message "a 32-bit unsigned decimal number"

message decimalNumberTooLarge : "decimal number too large"
message internalError : "internal error"

rule '0'->'9' {
  enterCharacterIntoString (!?tokenString !*)
  repeat
  while '0'->'9' :
    enterCharacterIntoString (!?tokenString !*)
  while '_' :
  end
  convertDecimalStringIntoUInt (
    !tokenString
    !?uint32value
    error decimalNumberTooLarge, internalError
  )
  send $integer$
}
\end{galgascode}


Explications :
\begin{enumerate}
  \item \galgas{style integerStyle -> "Integer Constants"} déclare un style (uniquement utile pour l’application Cocoa engendrée, vous pouvez ignorer cette ligne) ;
  \item \galgas{@uint uint32value} déclare l’attribut lexical \galgas{uint32value} de type entier 32 bits non signé ; au début de l’analyse de chaque token, cet attribut est initialisé à la valeur zéro ;
  \item \galgas{$integer$ !uint32value style integerStyle ...} déclare le terminal \galgas{$integer$} qui sera transmis à l’analyseur syntaxique accompagné de la valeur de \galgas{uint32value} ;  le message d’erreur qui suit est celui qui est utilisé lors d’une erreur syntaxique ;
  \item \galgas{message decimalNumberTooLarge : "decimal number too large"} déclare un message d’erreur ;
  \item enfin \galgas{rule '0'->'9' ...} définit l’analyse lexicale d’une contante entière non signé ; les caractères qui la composent sont accumulés dans \galgas{tokenString}, et la conversion de cette chaîne en entier est effectuée par la routine \galgas{convertDecimalStringIntoUInt} ; pour finir, \galgas{send $integer$} envoie le terminal vers l’analyseur syntaxique.
\end{enumerate}

\subsection{Analyse des délimiteurs}
Par défaut, un certain nombre de délimiteurs sont définis :

\begin{galgascode}
style delimitersStyle -> "Delimiters"
list delimitorsList style delimitersStyle error message "the '%K' delimitor" {
  ":",    ",",    ";",   "!",  "{",  "}", "->", "@", "*", "-"
}

rule list delimitorsList
\end{galgascode}

La règle \galgas{list delimitorsList style delimitersStyle error ...} déclare les terminaux \galgas{$:$}, \galgas{$,$}… Les messages d'erreur syntaxique sont définis en remplaçant la séquence \texttt{\%K} par l’épellation du délimiteur.

L'analyse des délimiteurs est définit par la règle \galgas{rule list delimitorsList}.

Travail à faire : remplacer la liste des délimiteurs par celle du langage LOGO.

\subsection{Analyse des chaînes de caractères}
Une analyse des chaînes de caractères est disponible par défaut. Comme le langage LOGO n’utilise pas de chaînes de caractères, vous pouvez supprimer les définitions suivantes :

\begin{galgascode}
style stringStyle -> "String Constants"
$literal_string$ ! tokenString style stringStyle %nonAtomicSelection
                   error message "a character string constant \"...\""

message incorrectStringEnd : "string does not end with '\"'"

rule '"' {
  repeat
  while ' ' | '!' | '#'-> '\uFFFD' :
    enterCharacterIntoString (!?tokenString !*)
  end
  select
  case '"' :
    send $literal_string$
  default
    error incorrectStringEnd
  end
}
\end{galgascode}

\subsection{Analyse des séparateurs}
C'est une règle très simple, qui accepte tout caractère de code ASCII compris entre \texttt{0x01} et \texttt{0x20} (l'espace). Comme il n'y a pas d'instruction send dans la règle lexicale, l'occurrence d'un séparateur est complètement ignorée par l'analyseur syntaxique.
\begin{galgascode}
rule '\u0001' -> ' ' {
}
\end{galgascode}
La séquence d'échappement \texttt{\textbackslash u} permet d'écrire directement des points de code Unicode sous la forme de quatre chiffres hexadécimaux.

\subsection{Analyse des commentaires}
C'est un peu plus compliqué, il faut repérer la fin de la ligne courante. Or, celle-ci peut être un seul caractère \texttt{LF} (fichier Unix), un seul caractère \texttt{CR} (fichier Mac Classic), une séquence \texttt{CRLF} (fichier Windows). D'autre part, une ligne de commentaire peut être la dernière ligne du fichier : notez que GALGAS rajoute automatiquement le caractère \texttt{\textquotesingle\textbackslash 0\textquotesingle} à la fin de la chaîne source. L'analyse d'un commentaire consiste donc, une fois le caractère initial \texttt{\textquotesingle\textbackslash\#\textquotesingle} repéré, à accepter silencieusement tous les caractères possibles, sauf \texttt{\textquotesingle\textbackslash u000A\textquotesingle} (LF), \texttt{\textquotesingle\textbackslash u000D\textquotesingle} (CR), \texttt{\textquotesingle\textbackslash0\textquotesingle}. L’écriture \galgas{drop $comment$} signifie que le terminal \galgas{$comment$} n’est pas transmis à l’analyseur syntaxique.

\begin{galgascode}
style commentStyle -> "Comments"
$comment$ style commentStyle %nonAtomicSelection error message "a comment"
rule '#' {
  repeat
  while '\u0001'->'\u0009' | '\u000B' | '\u000C' | '\u000E'->'\uFFFD':
  end
  drop $comment$
}
\end{galgascode}

Remarquez que pour un fichier Windows, le caractère \texttt{CR} marque la fin du commentaire, et que le caractère \texttt{LF} qui suit est silencieusement absorbé comme délimiteur.

Travail à faire
écrivez l'analyseur lexical du langage LOGO. Après compilations GALGAS et C++, les exécutables \texttt{logo} et \texttt{logo-debug} obtenus sont partiellement opérationnels (pas encore d’analyseur syntaxique) : avec l'option \texttt{-{}-mode=lexical-only}, vous pouvez faire afficher la liste des symboles terminaux produite par l'analyse lexicale du fichier source passé en argument.

Note : l'option \texttt{-{}-help} permet d'afficher la liste des options de l'exécutable. 

\section{Analyseur syntaxique}

Deux fichiers sont concernés :
\begin{itemize}
  \item \texttt{chezmoi/logo/galgas-sources/logo-syntax.galgas}, et
  \item \texttt{chezmoi/logo/galgas-sources/logo-grammar.galgas}.
\end{itemize}

Le fichier \texttt{logo-syntax.galgas} contient une liste de règles de production. Le fichier \texttt{logo-grammar.galgas} définit une grammaire.

Le fichier \texttt{logo-grammar.galgas} a la composition suivante :

\begin{galgascode}
grammar logo_grammar "LL1" {
  syntax logo_syntax
  <start_symbol>
}
\end{galgascode}

Explications :
\begin{enumerate}
  \item \galgas{"LL1"} est la classe de la grammaire ;
  \item \galgas{syntax logo_syntax} : les règles de productions sont dans le composant syntaxique \galgas{logo_syntax}, situé dans le fichier \texttt{logo-syntax.galgas} ; 
  \item \galgas{<start_symbol>} : l'axiome de la grammaire.
\end{enumerate}

A priori, vous n'avez pas besoin de modifier le fichier \texttt{logo-grammar.galgas} au cours de ce tutorial. Vous pouvez cependant modifier l'analyse effectuée en suivant les indications du \refTableauPage{tableauClasseGrammaire}\footnote{Rappel : toute grammaire LL(1) est SLR, toute grammaire SLR est LR(1).}.

\begin{table}[t]
  \centering
  \begin{tabular}{ll}
    \textbf{Chaîne} & \textbf{Analyse effectuée} \\
    \texttt{"LL1"} & Analyse LL (1) de la grammaire ; échoue si la grammaire n'est pas LL (1) \\
    \texttt{"SLR"} & Analyse SLR de la grammaire ; échoue si la grammaire n'est pas SLR \\
    \texttt{"LR1"} & Analyse LR (1) de la grammaire ; échoue si la grammaire n'est pas LR (1) \\
    \texttt{""} & Analyse LL (1) ; en cas d'échec, analyse SLR ; en cas de nouvel échec, analyse LR (1) \\
  \end{tabular}
  \caption{Spécification de l'analyse de la grammaire}
  \labelTableau{tableauClasseGrammaire}
  \ligne
\end{table}



Par défaut dans le fichier \galgas{logo_syntax.galgas}, seul le non terminal \galgas{<start_symbol>} est déclaré, et une règle de production vide est écrite.

C'est à vous d'écrire les règles de production qui définissent le langage LOGO (voir sa définition syntaxique \refSubsectionPage{definitionSyntaxiqueLOGO}).

Voici les indications qui vous permettront d'écrire ces règles :
\begin{itemize}
  \item vous pouvez déclarer autant de non terminaux que vous voulez ;
  \item la forme d'une règle de production est : \galgas{rule <mon_non_terminal> { partie droite }}
  \item la partie droite est une séquence éventuellement vide de :
  \begin{itemize}
    \item terminaux ;
    \item non-terminaux ;
    \item d'instructions de répétition syntaxique (\refSectionPage{instructionRepeatSyntaxique}) ;
    \item d'instruction de sélection syntaxique (\refSectionPage{instructionSelectSyntaxique}).
  \end{itemize}
  \item les règles de production peuvent apparaître dans un ordre quelquonque.
\end{itemize}

Pour vous aider, voici une écriture possible de la dérivation de l'axiome :

\begin{galgascode}
rule <start_symbol> {
#-- Definition des routines
  $PROGRAM$
  repeat
  while
    <routine_definition>
  end
#--- Programme principal
  $BEGIN$
  <instruction_list> 
  $END$
  $.$
}
\end{galgascode}

Et la règle de production \galgas{<routine_definition>} :

\begin{galgascode}
rule <routine_definition> {
  $ROUTINE$
  $identifier$ ?*
  $BEGIN$
  <instruction_list>
  $END$
}
\end{galgascode}

Noter l’écriture \galgas{$identifier$ ?*} : en effet, quand l’analyseur lexical envoie vers l’analyseur syntaxique le token \galgas{$identifier$}, celui-ci est accompagné d’une chaîne de caractères. On indique que la valeur de celle-ci n’est pas utilisée (pour le moment) par l’écriture \galgas{\?*}.

Il en est de même pour le token \galgas{$integer$} qui est accompagné d’une valeur entière.

Si des erreurs d'analyse de la grammaire surviennent, vous pouvez utiliser l'option \texttt{-{}-output-html-grammar} dans la ligne de commande : celle-ci provoque la génération du fichier : \texttt{chezmoi/logo/build/helpers/logo\_grammar.html} qui contient tous les détails de l'analyse de la grammaire.

À l'issue de ce travail, l'exécutable obtenu doit analyser correctement les programmes LOGO contenus sur le serveur pédagogique.

Vous pouvez utiliser l'option \texttt{-{}-mode=syntax-only} pour afficher la trace de l'analyse syntaxique.

\section{Sémantique statique}

Le but de cette étape est d'enrichir les fichiers GALGAS de façon à vérifier la sémantique statique du langage LOGO (\refSubsectionPage{semantiqueStatiqueLOGO}).

\subsection{Préliminaire : obtenir la valeur des identificateurs}

Dans l’analyseur syntaxique, pour chaque occurrence du token \galgas{$identifier$}, nous avons précédemment écrit \galgas{$identifier$ ?*} pour signifier que la valeur de la chaîne de caractères n’était pas utilisée.

À partir de maintenant, nous avons besoin de cette valeur. Celle-ci est récupérée en écrivant :

\begin{galgascode}
$identifier$ ?let @lstring unNom
\end{galgascode}


Cette écriture déclare une constante locale, nommée \galgas{unNom}, de type \galgas{@lstring}.

Notez que le type mentionné est \galgas{@lstring}, alors que dans l’analyseur lexical une valeur de type \galgas{@string} est associée au terminal \galgas{$identifier$}. Le type \galgas{@lstring} est une structure composée d’une valeur de type \galgas{@string} et d’une valeur de type \galgas{@location}. Cette dernière désigne un point dans le texte source analysé. Lors de la transmission des informations de l’analyseur lexical vers l’analyseur syntaxique, la valeur de type \galgas{@string} est associée à la position de l’identificateur dans le texte source. Ceci permet de construire facilement des messages d’erreur qui désigne l’endroit dans le texte source où une erreur est apparue.

Pour le moment, on ne modifie pas les terminaux \galgas{$integer$}.

Faire les modifications et recompiler. Comme les valeurs récupérées ne sont pas utilisées et perdues, vous obtenez un \emph{warning} pour chaque constante.

Vous pouvez afficher la valeur obtenue en ajoutant une instruction \galgas{log} à chacune des séquences précédentes :
\begin{galgascode}
$identifier$ ?let @lstring unNom
log unNom
\end{galgascode}

L'instruction \galgas{log} affiche la valeur d'une variable ou d’une constante. Elle est utilisable sur tous les types GALGAS.

\subsection{Principes d'écriture de la sémantique}

Le cadre général est celui des grammaires attribuées. Ceci revient à doter de paramètres formels les non terminaux de la partie gauche d'une règle, de la même façon que la définition d'une fonction C peut présenter des paramètres formels. En conséquence, un non-terminal apparaissant en partie droite d'une règle de production doit présenter des arguments effectifs, de la même façon qu'un appel de fonction doit citer des arguments effectifs en accord avec la déclaration du prototype de la fonction. Dès lors, vous pouvez établir les correspondances listées dans le \refTableauPage{tableauComparaisonArgumentsFormels}.

\begin{table}[t]
  \centering
  \begin{tabular}{p{8cm}p{6.7cm}}
    \textbf{En C} & \textbf{En GALGAS} \\
     Le prototype d'une fonction cite la liste des arguments formels & La déclaration d'un non-terminal cite la liste des attributs (au sens des grammaires attribuées) \\
     L'en tête de l'implémentation d'une fonction cite la liste des arguments formels & Le non terminal de gauche d'une règle de production cite la liste des attributs (au sens des grammaires attribuées) \\
     L'appel d'une fonction cite des paramètres effectifs & Un non terminal apparaissant dans la partie droite d'une règle de production cite une liste des attributs (au sens des grammaires attribuées) \\
  \end{tabular}
  \caption{Arguments formels, paramètres effectifs en C et en GALGAS}
  \labelTableau{tableauComparaisonArgumentsFormels}
  \ligne
\end{table}

En GALGAS, nous utilisons plutôt le vocabulaire des langages de programmation : \emph{argument formel}, \emph{paramètre effectif}.

\subsubsection{Arguments formels en GALGAS}
Un argument formel cite :
\begin{itemize}
  \item un délimiteur qui précise le sens de transmission de l'argument formel ;
  \item son type (par exemple \galgas{@lstring}, \galgas{@luint}, …) ;
  \item son nom.
\end{itemize}


Le sens de transmission d'un argument formel est défini dans le \refTableauPage{tableauSensTransmissionArgumentsFormels}. 

\begin{table}[t]
  \centering
  \begin{tabular}{ll}
    \textbf{Délimiteur} & \textbf{Sens de transmission} \\
      \texttt{?} & Entrée \\
      \texttt{?}\galgas{let} & Entrée constant \\
      \texttt{!} & Sortie \\
      \texttt{?!} & Entrée/sortie \\
  \end{tabular}
  \caption{Sens de transmission d'un argument formel}
  \labelTableau{tableauSensTransmissionArgumentsFormels}
  \ligne
\end{table}


\subsubsection{Paramètres effectifs en GALGAS}

Un paramètre effectif cite :
\begin{itemize}
  \item un délimiteur qui précise le sens de transmission du paramètre effectif ;
  \item une variable locale ou un argument formel de la règle de production.
\end{itemize}

Le sens de transmission d'un paramètre effectif est défini dans le t\refTableauPage{tableauSensTransmissionParametresEffectifs}. 

\begin{table}[t]
  \centering
  \begin{tabular}{lll}
    \textbf{Délimiteur} & \textbf{Sens de transmission} & \textbf{Argument formel correspondant} \\
      \texttt{?} & Entrée & \texttt{!} (argument formel en sortie) \\
      \texttt{!} & Sortie & \texttt{?} (argument formel en entrée) ou \\
                 &        & \texttt{?}\galgas{let} (argument formel en entrée constant) \\
      \texttt{!?} & Sortie/entrée & \texttt{?!} (argument formel en entrée/sortie) \\
  \end{tabular}
  \caption{Sens de transmission d'un paramètre effectif}
  \labelTableau{tableauSensTransmissionParametresEffectifs}
  \ligne
\end{table}

\subsubsection{Les types en GALGAS}

Il existe plusieurs sortes de types :
\begin{itemize}
  \item les types prédéfinis par le langage, comme \galgas{@lstring}, \galgas{@luint}, … ;
  \item les types définis par l'utilisateur, qui peuvent être :
  \begin{itemize}
    \item des types \emph{table} ;
    \item des types \emph{liste} ;
    \item des types \emph{classe}.
  \end{itemize}
\end{itemize}

\subsection{Écriture de la sémantique statique}
Pour décrire la sémantique statique (\refSubsectionPage{semantiqueStatiqueLOGO}), le plus simple est de créer un type table de symboles, dont une instance contiendra tous les noms de routines d'un programme LOGO.

\subsubsection{Ajout du type de table des routines}
éditez le fichier \texttt{chezmoi/logo/galgas-sources/logo-semantics.galgas} et ajouter la définition suivante :

\begin{galgascode}
map @routineMap {
  insert insertKey error message "the '%K' routine has been already declared"
  search searchKey error message "the '%K' routine is not declared"
}
\end{galgascode}

Ceci déclare le type \galgas{@routineMap}, avec une méthode d'insertion \galgas{insertKey} accompagnée de son message d'erreur, et une méthode de recherche \galgas{searchKey} accompagnée de son message d'erreur. Implicitement, la clé de la table est du type \galgas{@lstring}.

Cette définition sera complétée dans l'étape suivante afin de prendre en compte les instructions des routines (on n'en a pas besoin pour le moment).

À cet instant, vous pouvez recompiler le fichier \texttt{logo-semantics.galgas}.

Instructions sur les objets de type table
Voici quatre instructions relatives aux tables dont vous allez avoir besoin :
\begin{itemize}
  \item la déclaration d'un objet de type table ;
  \item l'initialisation d'un objet de type table ;
  \item l'instruction d'insertion dans une table ;
  \item l'instruction de recherche dans une table.
\end{itemize}

La déclaration d'un objet de type table se fait simplement en nommant le type puis l'objet ; par exemple :
\begin{galgascode}
@routineMap maTable
\end{galgascode}

L'initialisation d'un objet de type table s'effectue en créant une table vide :
\begin{galgascode}
maTable = {}
\end{galgascode}

Les deux écritures précédentes peuvent être condensées en une seule par :
\begin{galgascode}
@routineMap maTable = {}
\end{galgascode}

L'instruction d'insertion dans une table est :
\begin{galgascode}
[!?maTable insertKey !clef]
\end{galgascode}
où \galgas{insertKey} est le nom d'une méthode d'insertion déclarée dans le type table ; clef doit être une variable de type \galgas{@lstring} valuée. Si il existe déjà une entrée de même nom, le message d'erreur associé à la méthode d'insertion est affiché.

L'instruction de recherche dans une table est :
\begin{galgascode}
[maTable searchKey !clef]
\end{galgascode}
où \galgas{searchKey} est le nom d'une méthode de recherche déclarée dans le type table ; \galgas{clef} doit être une variable de type \galgas{@lstring} valuée. Si il n'existe pas d'entrée de même nom, le message d'erreur associé à la méthode de recherche est affiché.

\subsubsection{Ajout de la sémantique dans les règles de productions}
Éditer le fichier \texttt{chezmoi/logo/galgas-sources/logo-syntax.galgas} et modifiez la dérivation de l'axiome :

\begin{galgascode}
rule <start_symbol> {
  $PROGRAM$
  @routineMap tableRoutines = {}
  repeat
  while 
    <routine_definition> !?tableRoutines
  end
  $BEGIN$
  <instruction_list>
  $END$
  $.$
}
\end{galgascode}

L'appel du non terminal\galgas{<routine_definition>} impose que son en-tête doit être modifiée en conséquence :
\begin{galgascode}
rule <routine_definition> ?!@routineMap ioTableRoutines {
  ...
}
\end{galgascode}

\subsubsection{Travail à faire}
Maintenant, à vous de compléter les règles de façon à prendre en compte toutes les contraintes édictées à la \refSubsectionPage{semantiqueStatiqueLOGO}.

Vérifiez que votre analyseur détecte correctement les erreurs. Pour cela, vous pouvez utiliser les exemples du \refTableauPage{logoErreurSemantiques}.




\section{Sémantique dynamique}

Dans la sémantique dynamique (\refSubsectionPage{semantiqueDynamiqueLOGO}), nous allons prendre en compte la signification de l'exécution d'une instruction. En préliminaire, nous allons compléter l'analyseur lexical pour qu'il envoie la valeur d'une constante entière.

\subsubsection{Préliminaire : les constantes entières}

Modifier maintenant l’analyse syntaxique des constantes entières, à l’image de ce qui a été fait pour les identificateurs :
\begin{galgascode}
$integer$ ?let @luint unNom
\end{galgascode}

Le type \galgas{@luint} est une structure composée d’une valeur de type \galgas{@uint} et d’une valeur de type \galgas{@location}.

\subsubsection{Mise à plat de la liste des instructions}
Le but ultime est d'obtenir la liste des instructions du programme principal. Mais quelles sont les instructions qui devront apparaître dans cette liste ? A priori, toutes les instructions décrites dans la \refSubsectionPage{semantiqueDynamiqueLOGO}. En fait, vous pouvez vous passer de l'instruction \texttt{CALL} en insérant dans la liste des instructions non pas cette instruction, mais la liste des instructions de la routine correspondante. Il faut procéder de même lors de construction de la liste de chaque routine.

\subsubsection{Hiérarchie des classes des instructions}
Une solution classique pour ce type de situation est de définir une classe abstraite \galgas{@instruction}, et des classes concrètes \galgas{@penUp}, \galgas{@penDown}, \galgas{@rotate} et \galgas{@forward} qui héritent de cette classe abstraite :écriture des classes dans le composant sémantique éditez le fichier \texttt{chezmoi/logo/galgas-sources/logo-semantics.galgas} et insérer le texte suivant (n’importe où, dans n’importe quel ordre, GALGAS est indifférent à l’ordre des déclarations) :

\begin{galgascode}
abstract class @instruction {
}
class @penUp : @instruction {
}
class @penDown : @instruction {
}
class @forward : @instruction {
  @luint mLength
}
class @rotate : @instruction {
  @luint mAngle
}
\end{galgascode}


Les trois premières classes n'ont pas de propriété, et les deux dernières une propriété de type \galgas{@luint}.

\subsubsection{Instructions sur les objets de type \texttt{class}}

Vous avez besoin de deux instructions relatives aux classes :
\begin{itemize}
  \item la déclaration d'une variable de type classe ;
  \item l'instanciation d'un objet de type classe.
\end{itemize}

La déclaration d'un référence de type classe se fait simplement en nommant le type puis l'objet ; par exemple :
\begin{galgascode}
@instruction instruction
\end{galgascode}

L'instanciation d'un objet de type classe s'effectue en appelant le constructeur new d'une classe concrète avec les paramètres effectifs correspondants aux attributs de la classe, précédés des paramètres effectifs correspondants aux attributs des classes héritées :
\begin{galgascode}
instruction = @rotate.new {!valeurAngle}
\end{galgascode}

Les deux instructions peuvent réduites en :
\begin{galgascode}
@instruction instruction = @rotate.new {!valeurAngle}
\end{galgascode}

\subsubsection{Travail à faire}
Compléter les règles de productions pour chaque instruction (sauf l'instruction \texttt{CALL}).

\subsubsection{Le type liste d'instructions}
Pour construire la liste des instructions, il faut définir un nouveau type dont les valeurs sont des listes.

Éditez le fichier \texttt{chezmoi/logo/galgas-sources/logo-semantics.galgas} et insérer le texte suivant (n’importe où, GALGAS est indifférent à l’ordre des déclarations) :

\begin{galgascode}
list @instructionList {
  @instruction mInstruction
}
\end{galgascode}

Ceci déclare le type de liste \galgas{@instructionList}, dont chaque élément contient un objet instance d'une classe héritière de \galgas{@instruction}.

Instructions sur les objets de type liste
Voici trois instructions relatives aux listes dont vous allez avoir besoin :
\begin{itemize}
  \item la déclaration d'un objet de type liste ;
  \item l'initialisation d'un objet de type liste ;
  \item l'instruction d'ajout d'une valeur à une liste.
\end{itemize}

La déclaration d'un objet de type liste se fait simplement en nommant le type puis l'objet ; par exemple :
\begin{galgascode}
@instructionList maListe
\end{galgascode}

L'initialisation d'un objet de type liste s'effectue en créant une liste vide :
\begin{galgascode}
maListe = {}
\end{galgascode}

Les deux écritures précédentes peuvent condensées en une seule par :
\begin{galgascode}
@instructionList maListe = {}
\end{galgascode}

L'instruction d'ajout d'une valeur dans une liste est :
\begin{galgascode}
maListe += !instruction
\end{galgascode}
L'ajout s'effectue toujours à la fin de la liste.

\subsubsection{Travail à faire}
Compléter les règles de productions construire la liste des instructions d'une routine et la liste des instructions du programme principal (les instructions CALL sont toujours ignorées).

\subsubsection{L'instruction \texttt{CALL}}
Pour prendre en compte l’instruction \texttt{CALL}, nous allons procéder comme suit : d’abord, la définition du type table \galgas{@routineMap} va être modifier de façon à associer à chaque routine la liste mise à plat des instructions. Ensuite, nous prendrons en compte l’instruction \texttt{CALL} en extrayant de la table des routines la liste des instructions de la routine appelée, et en l’insérant à la fin de la liste courante des instructions.

\subsubsection{Modification du type table \texttt{@routineMap}}
Il faut maintenant modifier la définition du type table \galgas{@routineMap} de façon qu'à chaque nom de routine soit associée sa liste d'instructions :

\begin{galgascode}
map @routineMap {
  @instructionList mInstructionList
  insert insertKey  error message "the '%K' routine has been already declared"
  search searchKey error message "the '%K' routine is not declared
}
\end{galgascode}

Recompiler les sources GALGAS, et examiner les erreurs produites. Corrigez les en vous aidant des explications suivantes :
\begin{itemize}
  \item l'instruction d'insertion doit maintenant nommer un argument effectif en sortie supplémentaire, de type \galgas{@instructionList} :
  \begin{description}
    \item[ ] \galgas{[!?maTable insertKey !clef !maListe]}
  \end{description}
  \item l'instruction de recherche doit maintenant nommer un argument effectif en entrée, dont le type est \galgas{@instructionList} :
  \begin{description}
    \item[ ] \galgas{[maTable searchKey !clef ?maListe]}
  \end{description}
\end{itemize}

Prise en compte de l'instruction CALL
Il suffit d'ajouter à la liste courante des instructions toutes les instructions de la routine appelée par CALL :
\begin{galgascode}
...
[maTable searchKey !nomRoutine ?listeInstructionRoutine] ;
for i in listeInstructionRoutine do
  listeCouranteInstructions += !i.mInstruction
end
...
\end{galgascode}

L'instruction \galgas{for} permet d’énumérer un objet de type liste. Le corps de la boucle (entre \galgas{do} et \galgas{end}) est exécuté une fois pour chaque élément \galgas{i} de la liste.

\section{Génération de code}

Dans ce TP, la génération de code est divisée en deux étapes : d'abord, la succession des segments à tracer est simplement affichée sur le terminal ; dans un second temps, un fichier SVG est engendré au moyen d'un template.

L'allure du calcul des tracés est la suivante (à placer à la fin de la règle \galgas{<start_symbol>}) dans \texttt{logo-syntax.galgas} :

\begin{galgascode}
  ...
  @bool pendown = false
  @double x = 0.0
  @double y = 0.0
  @double angle = 0.0 # Angle en degres
  for i in instructionList do
    ...
  end
\end{galgascode}

Pour exprimer l'action à réaliser, des méthodes (définies et implémentées en dehors de leurs classes) vont être utilisées.

\subsubsection{Déclaration de la méthode abstraite}
Elle est nommée par exemple \galgas{codeDisplay} et doit être déclarée dans le fichier \texttt{chezmoi/logo/galgas-sources/logo-semantics.galgas}.

\begin{galgascode}
abstract method @instruction codeDisplay
  ?!@bool ioPenDown
  ?!@double ioX
  ?!@double ioY
  ?!@double ioAngle
\end{galgascode}

\subsubsection{Implémentation d'une héritière concrète}
Par exemple, pour la classe \galgas{@penUp}. Elle est implementée dans le fichier \texttt{chezmoi/logo/galgas-sources\-/logo-semantics.galgas}.

\begin{galgascode}
override method @penUp codeDisplay
  ?!@bool ioPenDown
  ?!@double unused ioX
  ?!@double unused ioY
  ?!@double unused ioAngle
{
  ioPenDown = false
}
\end{galgascode}

L'mplémentation de la méthode héritière concrète pour \galgas{@penDown} est élémentaire.

\subsubsection{Implémentation de l'héritière concrète pour \texttt{@rotate}}
Il faut accumuler l'angle de rotation dans l'argument \galgas{ioAngle}. Or l'attribut \galgas{mAngle} de la classe \galgas{@rotate} n'est pas du type \galgas{@uint}, mais du type \galgas{@luint}. Pour extraire la composante \galgas{@uint} d’un \galgas{@luint}, on écrit \galgas{[mAngle uint]}. Pour transformer un objet \galgas{unUint} de type \galgas{@uint} en \galgas{@double}, on écrit de la même façon \galgas{[unUint double]}. 


 Il faut donc écrire :
\begin{galgascode}
ioAngle = ioAngle + [[mAngle uint] double]
\end{galgascode}

\subsubsection{Implémentation de l'héritière concrète pour \texttt{@forward}}
La méthode complète est alors :

\begin{galgascode}
override method @forward codeDisplay
  ?!@bool ioPenDown
  ?!@double ioX
  ?!@double ioY
  ?!@double ioAngle
{
  let @double x = ioX + [mLength double] * [ioAngle cosDegree]
  let @double y = ioY + [mLength double] * [ioAngle sinDegree]
  if ioPenDown then
    message "[" + ioX + ", " + ioY + "] -> ["+ x + ", " + y + "]\n"
  end
  ioX = x
  ioY = y
}
\end{galgascode}

\subsubsection{Calcul des tracés}
Le calcul des tracés dans \texttt{logo-syntax.galgas} peut être complété par l'appel de la méhode \galgas{codeDisplay} pour chaque instruction.
\begin{galgascode}
  ...
  @bool pendown = false
  @double x = 0.0
  @double y = 0.0
  @double angle = 0.0 # Angle en degres
  for i in instructionList do
    [i.mInstruction codeDisplay !?penDown !?x !?y !?angle]
  end
\end{galgascode}

Maintenant vous pouvez effectuer la compilation GALGAS et la compilation C++.

\subsubsection{Exemple de fichier SVG}
Voici à titre d'exemple le fichier SVG qui doit être engendré par la compilation de l'exemple \texttt{carre.logo} :

\begin{galgascode}
<?xml version="1.0" standalone="no"?>
<!DOCTYPE svg PUBLIC "-//W3C//DTD SVG 1.1//EN"
                              "http://www.w3.org/Graphics/SVG/1.1/DTD/svg11.dtd">
<svg width="100%" height="100%" version="1.1" xmlns="http://www.w3.org/2000/svg">
<title>carre.logo</title>
<line x1="100" y1="100" x2="150" y2="100" style="stroke:#1F56D2" />
<line x1="150" y1="100" x2="150" y2="150" style="stroke:#1F56D2" />
<line x1="150" y1="150" x2="100" y2="150" style="stroke:#1F56D2" />
<line x1="100" y1="150" x2="100" y2="100" style="stroke:#1F56D2" />
</svg>
\end{galgascode}

\subsubsection{Template de génération du fichier SVG}
Créer un fichier \texttt{chezmoi/logo/galgas-sources/logo-svg.galgasTemplate} et y insérer le contenu suivant :

\texttt{<?xml version="1.0" standalone="no"?>}\\
\texttt{<!DOCTYPE svg PUBLIC "-//W3C//DTD SVG 1.1//EN"}\\
\texttt{\hspace*{5cm}"http://www.w3.org/Graphics/SVG/1.1/DTD/svg11.dtd">}\\
\texttt{<svg width="100\textbackslash\%" height="100\textbackslash\%" version="1.1" xmlns="http://www.w3.org/2000/svg">}\\
\texttt{<title>\%!TITLE\%</title>}\\
\texttt{\%!DRAWINGS\%</svg>}\\

Notez :
\begin{itemize}
  \item l'échappement des caractères \galgas{\%} à la ligne \texttt{<svg width ... >} ;
  \item les deux symboles \galgas{TITLE} et \galgas{DRAWINGS}.
\end{itemize}

\subsubsection{Déclarer un template en GALGAS}
Dans le fichier \texttt{chezmoi/logo/galgas-sources/logo-semantics.galgas}, insérer la déclaration du template :
\begin{galgascode}
filewrapper generationTemplate in "." {
} {
} {

template svg "logo-svg.galgasTemplate"
  ?@string TITLE
  ?@string DRAWINGS
}
\end{galgascode}

En GALGAS, un filewrapper est une structure de données qui est l'image d'un répertoire contenant des fichiers et des sous répertoires. Un fichier particulier est un template ; la déclaration mentionne les symboles (ici \galgas{TITLE} et \galgas{DRAWINGS}) comme arguments d'entrée, et le contenu est analysé par GALGAS de façon à vérifier qu'il est bien formé (usage correct des caractères \galgas{\%}).

\subsubsection{Construire la liste des instructions SVG}
La liste des instructions de tracé est accumulée dans une chaîne de caractères. Modifier toutes les méthodes \galgas{codeDisplay} de façon à construire cette chaîne en ajoutant un argument formel en entrée/sortie : \texttt{?!@string SVG}.

Il faut modifier la méthode \galgas{codeDisplay} de la classe \galgas{@forward} pour ajouter la génération de code :

\begin{galgascode}
override method @forward codeDisplay
  ?!@bool ioPenDown
  ?!@double ioX
  ?!@double ioY
  ?!@double ioAngle
  ?!@string SVG
{
  let @double x = ioX + [mLength double] * [ioAngle cosDegree]
  let @double y = ioY + [mLength double] * [ioAngle sinDegree]
  if ioPenDown then
    SVG += "<line x1=\"" + ioX + "\" y1=\"" + ioY + "\" x2=\""
        + x + "\" y2=\"" + y
        + "\" style=\"stroke:#1F56D2\" stroke-linecap=\"round\" />\n"
  end
  ioX = x
  ioY = y
}
\end{galgascode}

Pour terminer, voici le code complet de l’axiome \galgas{<start_symbol>}, qui enchaîne analyse syntaxique, sémantique et génération du fichier SVG :
\begin{galgascode}
rule <start_symbol> {
#-- Definition des routines
  $PROGRAM$
  @routineMap tableRoutines = {}
  @instructionList instructions = {}
  repeat
  while
    <routine_definition> !? tableRoutines
  end
#--- Programme principal
  $BEGIN$
  <instruction_list> !? tableRoutines !? instructions
  $END$
  $.$
#--- Compute SVG instructions
  @bool pendown = false
  @double x = 0.0
  @double y = 0.0
  @double angle = 0.0 # Angle en degres
  @string SVG = ""
  for i in instructions do
    [i.mInstruction codeDisplay !?pendown !?x !?y !?angle !?SVG]
  end
#--- Output file
  let @string sourceFilePath = @string.stringWithSourceFilePath
  let @string code = [filewrapper generationTemplate.svg
    ![sourceFilePath lastPathComponent]
    !SVG
  ]
  [code writeToFile ![sourceFilePath stringByDeletingPathExtension] + ".svg"]
}
\end{galgascode}

Compiler et essayer l'exécutable : un fichier SVG doit être produit lors de chaque exécution.

Le tutorial est terminé.

  \input{partie-utilisation/installation.tex}
  %!TEX encoding = UTF-8 Unicode
%!TEX root = ../galgas-book.tex

%--------------------------------------------------------------
\chapter{Using GALGAS}
%-------------------------------------------------------------


\section{Command Line Options}


\section{Creating a New Project}

  %!TEX encoding = UTF-8 Unicode
%!TEX root = ../galgas-book.tex

%--------------------------------------------------------------
\chapter{Lexical Elements}
%-------------------------------------------------------------

%Avant
%[\input{|"/bin/echo -n ab"}]
%Après
%
%
%(\immediate\write18{"/bin/echo -n ab"})
%
%\begin{filecontents}{myfile.tex}
%     This text gets written to \texttt{myfile.tex}.\\
%     Zis text gets written to \texttt{myfile.tex}.
%\end{filecontents}
%
%\input{myfile.tex}

  %!TEX encoding = UTF-8 Unicode
%!TEX root = ../galgas-book.tex

%--------------------------------------------------------------
\chapter{Formatage pour LaTeX}
%-------------------------------------------------------------

Si vous utilisez \LaTeX pour écrire la documentation de votre compilateur, vous êtes confronté sans doute au problème de la présentation des programmes sources. En effet, les paquetages classiques pour ce type de problème, comme par exemple \tpp{listings}, peuvent être trop rigides pour des règles lexicales particulières d'un langage.

Par exemple, en GALGAS, les constantes entières acceptent le caractère \tpp{\_}, comme dans \ggs+123_456+. Elles peuvent être préfixées par \tpp{0x}, et postfixées par \tpp{S}, \tpp{LS} pour indiquer leur type : \ggs+0x123_456S+, ou encore \ggs+0x_123_456_LS+. Le paquetage \tpp{listings} ne peut pas être paramétré pour afficher correctement les constantes entières de GALGAS. Il n'accepte pas non plus les caractères UTF-8 accentués dans les commentaires.

Comment faire ? Développer des commandes \LaTeX particulières pour faire ce travail. Elles s'appuient sur un mode particulier des compilateurs engendrés par GALGAS, qui permet de traduire un fichier source en un code compatible \LaTeX. C'est de cette façon que le code GALGAS est présenté dans ce document. Si les fichiers \tpp{.tex} sont codés en UTF-8, alors les caractères accentués peuvent être utilisés sans restriction, comme des caractères comme \tpp{æ} ou \tpp{Œ} (voir par exemple le \refGetterPage{char}{unicodeToLower}).

Dans la suite, nous allons progressivement présenter la démarche pour formatter un code source :
\begin{itemize}
  \item d'abord comment configurer votre compilateur pour qu'il engendre du code \LaTeX ;
  \item comment afficher ce code en utilisant le paquetage \tpp{f{}ilecontents} ;
  \item une amélioration de la solution précédente en définissant un environnement particulier (utilise le paquetage \tpp{verbatim}) ;
  \item définition d'une commande permettant d'afficher du code en ligne, appelable comme la commande \tpp{\textbackslash verb} (utilise le paquetage \tpp{verbatim}).
\end{itemize}

Dans tout ce chapitre, nous appliquons cette démarche au langage LOGO, défini à la \refSectionPage{presentation-logo}.





\section{Configuration de votre compilateur}

\subsection{option \texttt{-{}-mode=latex}}

Tout compilateur engendré par GALGAS possède un mode d'exécution particulier, le mode \emph{latex}. Il est activé par l'option \tpp{-{}-mode=latex}.

Dans ce mode, seule l'analyse lexicale est effectuée, aussi le fichier source doit être \emph{lexicalement correct}, mais n'a pas besoin d'être ni \emph{syntaxiquement correct}, ni \emph{sémantiquement correct}.

Le fichier de sortie a pour nom le fichier d'entrée postfixé par l'extension \tpp{.tex}. Il contient le texte source formatté pour \LaTeX.

Par exemple, si le fichier d'entrée est \tpp{test.logo} et contient :

\begin{lstlisting}
ROUTINE trace
BEGIN
  FORWARD 50;
  ROTATE 90;
END
\end{lstlisting}

En appelant le compilateur LOGO par la commande \tpp{logo -{}-mode=latex test.logo}, le fichier \tpp{test.logo.tex} est engendré et contient :

\begin{verbatim}
\keywordsStyle{R{}O{}U{}T{}I{}N{}E{}}\hspace*{.6em}t{}r{}a{}c{}e{} \\
\keywordsStyle{B{}E{}G{}I{}N{}} \\
\hspace*{.6em}\hspace*{.6em}\keywordsStyle{F{}O{}R{}W{}A{}R{}D{}}\hspace*{.6em}
                                       \integerStyle{5{}0{}}\delimitersStyle{;{}} \\
\hspace*{.6em}\hspace*{.6em}\keywordsStyle{R{}O{}T{}A{}T{}E{}}\hspace*{.6em}
                                       \integerStyle{9{}0{}}\delimitersStyle{;{}} \\
\keywordsStyle{E{}N{}D{}}
\end{verbatim}

Pour l'afficher, il suffit de définir les commandes \tpp{\textbackslash keywordsStyle}, \tpp{\textbackslash integerStyle} et \tpp{\textbackslash delimitersStyle}\footnote{Aucune commande n'est définie pour les identificateurs, car l'analyseur lexical ne définit pas de style pour ceux-ci (voir \refSubsectionPage{fonctionnement-mode-latex}).}, et de placer ce texte dans un environnement où une police à échappement fixe est activée :



\begin{verbatim}
\newcommand\keywordsStyle[1]{\textcolor{blue}{\textbf{#1}}}
\newcommand\delimitersStyle[1]{\textcolor{brown}{\textbf{#1}}}
\newcommand\integerStyle[1]{\textcolor{brown}{#1}}
\texttt{
\keywordsStyle{R{}O{}U{}T{}I{}N{}E{}}\hspace*{.6em}t{}r{}a{}c{}e{} \\
\keywordsStyle{B{}E{}G{}I{}N{}} \\
\hspace*{.6em}\hspace*{.6em}\keywordsStyle{F{}O{}R{}W{}A{}R{}D{}}\hspace*{.6em}
                                  \integerStyle{5{}0{}}\delimitersStyle{;{}} \\
\hspace*{.6em}\hspace*{.6em}\keywordsStyle{R{}O{}T{}A{}T{}E{}}\hspace*{.6em}
                                  \integerStyle{9{}0{}}\delimitersStyle{;{}} \\
\keywordsStyle{E{}N{}D{}}
}
\end{verbatim}

On obtient ainsi (noter l'identation de la première ligne) :

\texttt{
\textcolor{blue}{\bf ROUTINE}\hspace*{.6em}t{}r{}a{}c{}e{} \\
\textcolor{blue}{\bf BEGIN} \\
\hspace*{1.2em}\textcolor{blue}{\bf FORWARD}\hspace*{.6em}\textcolor{brown}{5{}0{}}\textcolor{brown}{\bf ;} \\
\hspace*{1.2em}\textcolor{blue}{\bf ROTATE}\hspace*{.6em}\textcolor{brown}{9{}0{}}\textcolor{brown}{\bf ;} \\
\textcolor{blue}{\bf END}
}


\subsection{option \texttt{-{}-mode:suffixe=latex}}

Si vous documentez plusieurs compilateurs, vous pouvez avoir une collision de nom de style. Une variante de l'option \tpp{-{}-mode=latex} est de préciser un suffixe : \tpp{-{}-mode:suffixe=latex}. Le \tpp{suffixe} doit être un nom uniquement constitué de lettres (minuscules ou majuscules). Ce suffixe est ajouté aux noms de style. En appelant le compilateur LOGO par la commande \tpp{logo -{}-mode=latex:Logo test.logo}, le fichier \tpp{test.logo.tex} est engendré et contient :

\begin{verbatim}
\keywordsStyleLogo{R{}O{}U{}T{}I{}N{}E{}}\hspace*{.6em}t{}r{}a{}c{}e{} \\
\keywordsStyleLogo{B{}E{}G{}I{}N{}} \\
\hspace*{.6em}\hspace*{.6em}\keywordsStyleLogo{F{}O{}R{}W{}A{}R{}D{}}\hspace*{.6em}
                              \integerStyleLogo{5{}0{}}\delimitersStyleLogo{;{}} \\
\hspace*{.6em}\hspace*{.6em}\keywordsStyleLogo{R{}O{}T{}A{}T{}E{}}\hspace*{.6em}
                              \integerStyleLogo{9{}0{}}\delimitersStyleLogo{;{}} \\
\keywordsStyleLogo{E{}N{}D{}}
\end{verbatim}




\subsection{Formatages complémentaires}

Il est possible de formatter l'affichage du code en utilisant des paquetages standard. Ci-après sont présentées deux possibilités avec les paquetages \tpp{lineo} et \tpp{mdframed}.

\subsubsection{Formatage avec le paquetage \texttt{lineno}}

Le paquetage \tpp{lineno} permet de numéroter les lignes sources :
\begin{verbatim}
\resetlinenumber
\begin{linenumbers}
\ttfamily
...
\end{linenumbers}
\end{verbatim}

Et on obtient :

\resetlinenumber
\begin{linenumbers}\singlespacing\ttfamily
\textcolor{blue}{\bf ROUTINE}\hspace*{.6em}t{}r{}a{}c{}e{} \\
\textcolor{blue}{\bf BEGIN} \\
\hspace*{1.2em}\textcolor{blue}{\bf FORWARD}\hspace*{.6em}\textcolor{brown}{5{}0{}}\textcolor{brown}{\bf ;} \\
\hspace*{1.2em}\textcolor{blue}{\bf ROTATE}\hspace*{.6em}\textcolor{brown}{9{}0{}}\textcolor{brown}{\bf ;} \\
\textcolor{blue}{\bf END}
\end{linenumbers}

\subsubsection{Formatage avec le paquetage \texttt{mdframed}}

Le paquetage \tpp{mdframed} permet (entre autres) d'afficher un trait vertical dans la marge gauche. Pour cela, il faut d'abord le configurer en créant un evironnement, ici \tpp{siderules} :

\begin{verbatim}
\newmdenv[
  topline=false,
  bottomline=false,
  rightline=false,
  linecolor=red!25,
  linewidth=2pt
]{siderules}
\end{verbatim}

En utilisant l'environnement \tpp{siderules} :

\begin{verbatim}
\begin{siderules}
\ttfamily
...
\end{siderules}
\end{verbatim}

On obtient :

{
\newmdenv[
  topline=false,
  bottomline=false,
  rightline=false,
  linecolor=red!25,
  linewidth=2pt
]{siderulesRed}

\begin{siderulesRed}
\ttfamily
\textcolor{blue}{\bf ROUTINE}\hspace*{.6em}t{}r{}a{}c{}e{} \\
\textcolor{blue}{\bf BEGIN} \\
\hspace*{1.2em}\textcolor{blue}{\bf FORWARD}\hspace*{.6em}\textcolor{brown}{5{}0{}}\textcolor{brown}{\bf ;} \\
\hspace*{1.2em}\textcolor{blue}{\bf ROTATE}\hspace*{.6em}\textcolor{brown}{9{}0{}}\textcolor{brown}{\bf ;} \\
\textcolor{blue}{\bf END}
\end{siderulesRed}
}

\subsection{Comment s'effectue la traduction en \LaTeX}

La traduction s'effectue comme suit :
\begin{itemize}
  \item à chaque \ggs+style+ défini dans l'analyseur lexical correspond une commande \LaTeX particulière : par exemple à \tpp{keywordsStyle} correspond \tpp{\textbackslash keywordsStyle} (\refSubsectionPage{fonctionnement-mode-latex}) ;
  \item les caractères possédant une signification particulière en \LaTeX sont échappés ou substitués selon le \refTableau{substitution-latex} ;
  \item après tout caractère non échappé ni substitué est ajoutée la séquence \tpp{\{\}}.
\end{itemize}

L'ajout de la séquence \tpp{\{\}} peut paraître superflue, mais elle permet de résoudre de manière systématique bien des difficultés d'affichage :
\begin{itemize}
  \item la séquence \tpp{-{}-} est affichée \tpp{--} : le compilateur produit \tpp{-\{\}-\{\}}, on a ainsi l'affichage souhaité \tpp{-{}-} ;
  \item de même \tpp{\textgreater{}\textgreater} est affiché \tpp{\textgreater\textgreater} et \tpp{\textless{}\textless} est affiché \tpp{\textless\textless} : le compilateur produit \tpp{\textgreater\{\}\textgreater\{\}} et  \tpp{\textless\{\}\textless\{\}} ;
  \item \LaTeX effectue par défaut des ligatures : \tpp{f{}i} est affiché \tpp{fi}, le compilateur produit \tpp{f\{\}i} pour obtenir  \tpp{f{}i}.
\end{itemize}



\begin{table}[t]
  \centering
  \begin{tabular}{rl}
    \textbf{Caractère source} & \textbf{Formattage pour\LaTeX}\\
    \texttt{\textquotesingle\textgreater\textquotesingle} & \texttt{\textbackslash textgreater\{\}} \\
    \texttt{\textquotesingle\textless\textquotesingle} & \texttt{\textbackslash textless\{\}} \\
    \texttt{\textquotesingle$\sim$\textquotesingle} & \texttt{\$\textbackslash sim\$} \\
    \texttt{\textquotesingle$\wedge$\textquotesingle} & \texttt{\$\textbackslash wedge\$} \\
    \texttt{\textquotesingle\&\textquotesingle} & \texttt{\textbackslash \&} \\
    \texttt{\textquotesingle\%\textquotesingle} & \texttt{\textbackslash \%} \\
    \texttt{\textquotesingle\#\textquotesingle} & \texttt{\textbackslash \#} \\
    \texttt{\textquotesingle\$\textquotesingle} & \texttt{\textbackslash \$} \\
    \texttt{\textquotesingle\`{}\textquotesingle} & \texttt{\textbackslash \`{}\{\}} \\
    \texttt{\textquotesingle~\textquotesingle} & \texttt{\textbackslash hspace*\{.6em\}} \\
    \texttt{\textquotesingle\textbackslash n\textquotesingle} & \texttt{\textvisiblespace\textbackslash\textbackslash\textbackslash n} \\
    \texttt{\textquotesingle\{\textquotesingle} & \texttt{\textbackslash \{} \\
    \texttt{\textquotesingle\}\textquotesingle} & \texttt{\textbackslash \}} \\
    \texttt{\textquotesingle\_\textquotesingle} & \texttt{\textbackslash \_} \\
    \texttt{\textquotesingle\textbackslash\textquotesingle} & \texttt{\textbackslash textbackslash\{\}} \\
    \texttt{\textquotesingle\textquotesingle\textquotesingle} & \texttt{\textbackslash textquotesingle\{\}} \\
    \texttt{\textquotesingle\textquotedbl\textquotesingle} & \texttt{\textbackslash textquotedbl\{\}} \\
    Autre caractère : \texttt{\textquotesingle}$c$\texttt{\textquotesingle} & $c$\texttt{\{\}} \\
  \end{tabular}
  \caption{Échappement et substitution des caractères pour formattage \LaTeX}
  \labelTableau{substitution-latex}
  \ligne
\end{table}



\subsectionLabel{Fonctionnement de l'option \texttt{-{}-mode=latex}}{fonctionnement-mode-latex}

L'option \tpp{-{}-mode=latex} utilise les noms de style définis dans l'analyseur lexical LOGO. Par exemple, l'extrait suivant indique que le style \tpp{integerStyle} est attaché au terminal \ggs+$integer$+ :

\begin{galgas}
style integerStyle -> "Integer Constants"
$integer$ !uint32value style integerStyle ...
\end{galgas}

À chaque style, correspond une commande \LaTeX obtenue en préfixant le nom du style par un anti-slash \tpp{\textbackslash}\footnote{Les chiffres et le caractère de soulignement \tpp{\_} sont interdits dans les noms de style.}. Si aucun style n'est défini par un terminal particulier, il est affiché avec le style par défaut : c'est le cas des identificateurs LOGO dans les listings précédents.

Noter que l'affichage des commentaires nécessite l'utilisation conjointe d'un style particulier et de l'instruction lexicale \ggs+drop+ (\refSubsectionPage{instructionLexicaleDrop}) ; pour le langage LOGO :

\begin{galgas}
style commentStyle -> "Comments"
$comment$ style commentStyle ...
rule '#' {
  repeat
  while '\u0001' -> '\u0009' | '\u000B' | '\u000C' | '\u000E' -> '\uFFFD' :
  end
  drop $comment$
}
\end{galgas}







\section{Affichage via le paquetage \texttt{f{}ilecontents}}

Insérer un texte en effectuant un copié/collé comme suggéré à la section précédente est très laborieux ! Le paquetage \tpp{f{}ilecontents} va permettre de simplifier l'écriture en utilisant l'environnement \tpp{f{}ilecontents*} :

\begin{verbatim}
\begin{filecontents*}{temp.logo}
ROUTINE trace
BEGIN
  FORWARD 50;
  ROTATE 90;
END
\end{filecontents*}
\immediate\write18{logo --mode=latex temp.logo}
\noindent{\ttfamily\input{temp.logo.tex}}
\end{verbatim}

L'environnement \tpp{f{}ilecontents*} écrit son contenu dans le fichier \tpp{temp.logo} du répertoire courant. La commande \tpp{\textbackslash immediate\textbackslash write18}\footnote{Penser à ajouter l'option \tpp{-shell-escape} lors de la compilation \LaTeX.} permet de lancer la commande shell \tpp{logo -{}-mode=latex temp.logo}\footnote{Le répertoire vers l'exécutable \tpp{logo} doit faire partie des chemins définis par la variable \tpp{\$PATH} du shell.}, qui a pour résultat d'écrire le fichier formatté \tpp{temp.logo.tex} dans le répertoire courant. Il suffit donc de l'inclure grâce à la commande \tpp{\textbackslash input} en sélectionnant une police à échappement fixe (\tpp{\textbackslash ttfamily}). \tpp{\textbackslash noindent} permet d'éliminer l'indentation de la première ligne.

Cette deuxième approche est plus satisfaisante car on peut faire figurer le texte source LOGO directement dans le fichier \LaTeX, mais nous allons voir dans la section suivante une meilleure solution.











\section{Définition d'un environnement d'affichage formatté}

Dans cette section, on va voir comment nous allons définir un environnement \tpp{logocode} qui permettra d'entrer et de formatter implicitement un texte LOGO :

\begin{verbatim}
\begin{logocode}
ROUTINE trace
BEGIN
  FORWARD 50;
  ROTATE 90;
END
\end{logocode}
\end{verbatim}

Ce qui permettra d'obtenir :

{\noindent\ttfamily
\textcolor{blue}{\bf ROUTINE}\hspace*{.6em}t{}r{}a{}c{}e{} \\
\textcolor{blue}{\bf BEGIN} \\
\hspace*{1.2em}\textcolor{blue}{\bf FORWARD}\hspace*{.6em}\textcolor{brown}{5{}0{}}\textcolor{brown}{\bf ;} \\
\hspace*{1.2em}\textcolor{blue}{\bf ROTATE}\hspace*{.6em}\textcolor{brown}{9{}0{}}\textcolor{brown}{\bf ;} \\
\textcolor{blue}{\bf END}
}


Pour cela, nous avons besoin du paquetage \tpp{verbatim}. Il est conseillé d'inclure ce paquetage juste après la déclaration \tpp{\textbackslash documentclass} :

{\singlespacing
\begin{verbatim}
\documentclass [...] {...}
\usepackage{verbatim}
...
\end{verbatim}
}

La définition de l'environnement \tpp{logocode} est la suivante :


%\begin{verbatim}
%\newwrite\tempfile
%\makeatletter
%\newenvironment{logocode}{%
%  \begingroup
%  \@bsphack
%  \immediate\openout\tempfile=temp.logo%
%  \let\do\@makeother\dospecials
%  \catcode`\^^M\active
%  \verbatim@startline
%  \verbatim@addtoline
%  \verbatim@finish
%  \def\verbatim@processline{\immediate\write\tempfile{\the\verbatim@line}}%
%  \verbatim@start
%}{
%  \immediate\closeout\tempfile
%  \@esphack
%  \endgroup
%  \immediate\write18{logo --mode=latex temp.logo}
%  \noindent{\ttfamily\input{temp.logo.tex}}
%}
%\makeatother
%\end{verbatim}

\resetlinenumber
\begin{linenumbers}\singlespacing\ttfamily
\textbackslash newwrite\textbackslash tempf{}ile\newline
\textbackslash makeatletter\newline
\textbackslash newenvironment\{\textcolor{blue}{logocode}\}\{\%\newline
\hspace*{1.2em}\textbackslash begingroup\newline
\hspace*{1.2em}\textbackslash @bsphack\newline
\hspace*{1.2em}\textbackslash immediate\textbackslash openout\textbackslash tempf{}ile=\textcolor{blue}{temp.logo}\%\newline
\hspace*{1.2em}\textbackslash let\textbackslash do\textbackslash @makeother\textbackslash dospecials\newline
\hspace*{1.2em}\textbackslash catcode\`{}\textbackslash$\wedge\wedge$M\textbackslash active\newline
\hspace*{1.2em}\textbackslash verbatim@startline\newline
\hspace*{1.2em}\textbackslash verbatim@addtoline\newline
\hspace*{1.2em}\textbackslash verbatim@f{}inish\newline
\hspace*{1.2em}\textbackslash def\textbackslash verbatim@processline\{\textbackslash immediate\textbackslash write\textbackslash tempf{}ile\{\textbackslash the\textbackslash verbatim@line\}\}\%\newline
\hspace*{1.2em}\textbackslash verbatim@start\newline
\}\{\newline
\hspace*{1.2em}\textbackslash immediate\textbackslash closeout\textbackslash tempf{}ile\newline
\hspace*{1.2em}\textbackslash @esphack\newline
\hspace*{1.2em}\textbackslash endgroup\newline
\hspace*{1.2em}\textbackslash immediate\textbackslash write18\{\textcolor{blue}{logo -{}-mode=latex temp.logo}\}\newline
\hspace*{1.2em}\{\textbackslash noindent\textbackslash ttfamily\textbackslash input\{\textcolor{blue}{temp.logo.tex}\}\}\newline
\}\newline
\textbackslash makeatother
\end{linenumbers}

Quelques commentaires :
\begin{itemize}
  \item ligne 1, la commande \tpp{\textbackslash newwrite\textbackslash tempf{}ile} est nécessaire pour l'écriture de fichier ; elle doit figurer une seule fois dans le texte source, si vous définissez plusieurs environnements d'affichage, veillez à ne pas la dupliquer ; 
  \item ligne 3, le nom d'environnement (en bleu) est défini : bien entendu, vous pouvez changer ce nom pour l'adapter au nom de votre compilateur ;
  \item ligne 8, attention, après la commande \tpp{\textbackslash catcode}, c'est un accent aigu \tpp{\`{}} ;
  \item ligne 19, l'affichage de la ligne traduite est effectuée ; à cet endroit, nous pouvez utiliser toutes les commandes de formattage, comme par exemple les paquetages \tpp{lineno} et \tpp{mdframed} cités plus haut.
\end{itemize}

Par exemple, à la place de la ligne 19, on peut utiliser l'environnement \tpp{siderules} (paquetage \tpp{mdframed}) et écrire :

\texttt{
\textbackslash noindent\textbackslash begin\{siderules\}\textbackslash ttfamily\textbackslash input\{temp.logo.tex\}\textbackslash end\{siderules\}
}












\section{Affichage du code en ligne}

Pour afficher du code en ligne, on va définir une commande \tpp{\textbackslash logo} qui s'utilise comme la commande verbatim en ligne \tpp{\textbackslash verb} ; si on écrit :

\begin{verbatim}
Les mots réservés de LOGO sont \logo+BEGIN+, \logo+END+, ..., les délimiteurs
sont \logo+;+ et \logo+.+.
\end{verbatim}

Le délimiteur utilisé ici est \tpp{+}, mais, comme pour \tpp{\textbackslash verb}, tout caractère peut être utilisé, à condition qu'il n'apparaisse pas dans la chaîne à formatter. On obtient donc :

Les mots réservés de LOGO sont \colorbox{gray!6}{\ttfamily\textcolor{blue}{\bf BEGIN}}, \colorbox{gray!6}{\ttfamily\textcolor{blue}{\bf END}}, ..., les délimiteurs sont \colorbox{gray!6}{\ttfamily\textcolor{brown}{\bf ;}} et \colorbox{gray!6}{\ttfamily\textcolor{brown}{\bf .}}.

Comme pour l'affichage d'un listing, nous avons besoin du paquetage \tpp{verbatim}. Rappelons qu'il est conseillé d'inclure ce paquetage juste après la déclaration \tpp{\textbackslash documentclass} :

{\singlespacing
\begin{verbatim}
\documentclass [...] {...}
\usepackage{verbatim}
...
\end{verbatim}
}

La définition de commande \tpp{\textbackslash logo} est la suivante :


%\begin{verbatim}
%\makeatletter
%\newcommand*\ggs{%
%  \@bsphack%
%  \begingroup%
%  \let\do\@makeother\dospecials%
%  \let\do\do@noligs\verbatim@nolig@list%
%  \catcode`\^^M=15\relax%
%  \@vobeyspaces%
%  \@ggs{\temporary}%
%}%
%\newcommand\@ggs[2]{%
%  \catcode`-=12\relax%
%  \catcode`<=12\relax%
%  \catcode`>=12\relax%
%  \catcode`,=12\relax%
%  \catcode`'=12\relax%
%  \catcode``=12\relax%
%  \catcode`#2\active%
%  \catcode`~\active%
%  \lccode`\~`#2\relax%
%  \lowercase{%
%    \begingroup%
%    \def\@tempa##1~{%
%      \edef\@tempa{%
%        \noexpand\@ifstar{%
%          \noexpand\@@ggs\noexpand\visiblespaces{##1}%
%        }{%
%          \noexpand\@@ggs\noexpand\whitespaces{##1}%
%        }%
%      }%
%      \expandafter\endgroup%
%      \expandafter\DeclareRobustCommand%
%      \expandafter*%
%      \expandafter#1%
%      \expandafter{%
%      \@tempa}%
%      \@esphack%
%      \immediate\openout\tempfile=temp.galgas%
%      \immediate\write\tempfile{##1}%
%      \immediate\closeout\tempfile%
%      \immediate\write18{galgas --mode=latex temp.galgas}%
%      \colorbox{gray!6}{\ttfamily\input{temp.galgas.tex}\unskip}%
%    }%
%  }%
%  \ifnum`#2=`\~\else\@makeother\~\fi%
%  \expandafter\endgroup%
%  \@tempa%
%}%
%\newcommand*\@@ggs[2]{%
%  \relax\ifmmode\hbox\else\leavevmode\null\fi%
%  \bgroup#1\frenchspacing\verbatim@font\verbtextstyle#2\egroup%
%}%
%\let\verbtextstyle\relax%
%\newcommand*\visiblespaces{\chardef\ 32\relax}%
%\newcommand*\whitespaces{\let\ \@@space}%
%\let\@@space\ %
%\makeatother
%\end{verbatim}




\resetlinenumber
\begin{linenumbers}\singlespacing\ttfamily
\textbackslash newwrite\textbackslash tempf{}ile\newline
\textbackslash makeatletter\newline
\textbackslash newcommand*\textbackslash \textcolor{blue}{logo}\{\%\newline
\hspace*{1.2em}\textbackslash @bsphack\%\newline
\hspace*{1.2em}\textbackslash begingroup\%\newline
\hspace*{1.2em}\textbackslash let\textbackslash do\textbackslash @makeother\textbackslash dospecials\%\newline
\hspace*{1.2em}\textbackslash let\textbackslash do\textbackslash do@noligs\textbackslash verbatim@nolig@list\%\newline
\hspace*{1.2em}\textbackslash catcode\`{}\textbackslash $\wedge\wedge$M=15\textbackslash relax\%\newline
\hspace*{1.2em}\textbackslash @vobeyspaces\%\newline
\hspace*{1.2em}\textbackslash @\textcolor{blue}{logo}\{\textbackslash temporary\}\%\newline
\}\%\newline
\textbackslash newcommand\textbackslash @\textcolor{blue}{logo}[2]\{\%\newline
\hspace*{1.2em}\textbackslash catcode\`{}-=12\textbackslash relax\%\newline
\hspace*{1.2em}\textbackslash catcode\`{}<=12\textbackslash relax\%\newline
\hspace*{1.2em}\textbackslash catcode\`{}>=12\textbackslash relax\%\newline
\hspace*{1.2em}\textbackslash catcode\`{},=12\textbackslash relax\%\newline
\hspace*{1.2em}\textbackslash catcode\`{}\textquotesingle=12\textbackslash relax\%\newline
\hspace*{1.2em}\textbackslash catcode\`{}\`{}=12\textbackslash relax\%\newline
\hspace*{1.2em}\textbackslash catcode\`{}\#2\textbackslash active\%\newline
\hspace*{1.2em}\textbackslash catcode\`{}$\sim$\textbackslash active\%\newline
\hspace*{1.2em}\textbackslash lccode\`{}\textbackslash $\sim$\`{}\#2\textbackslash relax\%\newline
\hspace*{1.2em}\textbackslash lowercase\{\%\newline
\hspace*{2.4em}\textbackslash begingroup\%\newline
\hspace*{2.4em}\textbackslash def\textbackslash @tempa\#\#1$\sim$\{\%\newline
%\hspace*{3.6em}\textbackslash edef\textbackslash @tempa\{\%\newline
%\hspace*{4.8em}\textbackslash noexpand\textbackslash @ifstar\{\%\newline
%\hspace*{6.0em}\textbackslash noexpand\textbackslash @@\textcolor{blue}{logo}\textbackslash noexpand\textbackslash visiblespaces\{\#\#1\}\%\newline
%\hspace*{4.8em}\}\{\%\newline
%\hspace*{6.0em}\textbackslash noexpand\textbackslash @@\textcolor{blue}{logo}\textbackslash noexpand\textbackslash whitespaces\{\#\#1\}\%\newline
%\hspace*{4.8em}\}\%\newline
%\hspace*{3.6em}\}\%\newline
\hspace*{3.6em}\textbackslash expandafter\textbackslash endgroup\%\newline
\hspace*{3.6em}\textbackslash expandafter\textbackslash DeclareRobustCommand\%\newline
\hspace*{3.6em}\textbackslash expandafter*\%\newline
\hspace*{3.6em}\textbackslash expandafter\#1\%\newline
\hspace*{3.6em}\textbackslash expandafter\{@tempa\}\%\newline
\hspace*{3.6em}\textbackslash @esphack\%\newline
\hspace*{3.6em}\textbackslash immediate\textbackslash openout\textbackslash \textcolor{blue}{tempf{}ile=temp.logo}\%\newline
\hspace*{3.6em}\textbackslash immediate\textbackslash write\textbackslash tempf{}ile\{\#\#1\}\%\newline
\hspace*{3.6em}\textbackslash immediate\textbackslash closeout\textbackslash tempf{}ile\%\newline
\hspace*{3.6em}\textbackslash immediate\textbackslash write18\{\textcolor{blue}{logo -{}-mode=latex temp.logo}\}\%\newline
\hspace*{3.6em}\textbackslash colorbox\{gray!6\}\{\textbackslash ttfamily\textbackslash input\{\textcolor{blue}{temp.logo.tex}\}\textbackslash unskip\}\%\newline
\hspace*{2.4em}\}\%\newline
\hspace*{1.2em}\}\%\newline
\hspace*{1.2em}\textbackslash ifnum\`{}\#2=\`{}\textbackslash $\sim$\textbackslash else\textbackslash @makeother\textbackslash $\sim$\textbackslash f{}i\%\newline
\hspace*{1.2em}\textbackslash expandafter\textbackslash endgroup\%\newline
\hspace*{1.2em}\textbackslash @tempa\%\newline
\}\%\newline
\textbackslash makeatother
\end{linenumbers}

Commentaires :
\begin{itemize}
  \item ligne 1, la commande \tpp{\textbackslash newwrite\textbackslash tempf{}ile} est nécessaire pour l'écriture de fichier ; elle doit figurer une seule fois dans le texte source, si vous définissez plusieurs environnements d'affichage, veillez à ne pas la dupliquer ; 
  \item ligne 8, 13 à 21 et 38 : attention, c'est un accent aigu \tpp{\`{}} ;
  \item ligne 3, 10 et 12 : le nom \tpp{logo} apparaît trois fois (en bleu pour être repéré plus facilement) : si vous changez le nom de la commande, veillez à en remplacer toutes les occurrences ;
  \item une difficulté est d'assurer que la commande n'insère aucune espace supplémentaire : c'est pour cela que toutes les lignes se terminent par \tpp{\%}\footnote{En fait, uniquement certaines lignes doivent être obligatoirement terminées par \tpp{\%} ; pour simplifier, on applique cette terminaison à toutes.} ;
    \item enfin le plus intéressant : ligne 31, le fichier \tpp{temp.logo} est ouvert en écriture ;
    \item ligne 32, le contenu de la commande est écrit dans ce fichier ;
    \item ligne 33, le fichier est fermé ;
    \item ligne 34, le compilateur est appelé pour effectuer la traduction en \LaTeX ; {\bf attention}, cette commande est un argument de \tpp{\textbackslash lowercase}\footnote{Aucune idée de son rôle, mais si on supprime \tpp{\textbackslash lowercase}, la compilation \LaTeX échoue.} (ligne 22), si bien que tous les caractères sont passés en minuscules : ainsi, si on écrit \tpp{logo -{}-mode=latex:Logo temp.logo}, c'est la commande \tpp{logo} \tpp{-{}-mode=latex:logo temp.logo} qui est exécutée ;
    \item ligne 35, le code traduit est affiché ; comme la commande \tpp{\textbackslash input} (ligne 35) insère toujours une espace après elle, on la supprime par \tpp{\textbackslash unskip}.
\end{itemize}

Noter bien que la ligne 35 est une commande générale d'affichage : ici on a choisi un fond gris, et une police à échappement fixe.

Enfin, la commande \tpp{\textbackslash logo} ne peut pas être utilisée dans les notes en bas de page (commande \tpp{\textbackslash footnote}), ni en argument d'une macro.



  %!TEX encoding = UTF-8 Unicode
%!TEX root = ../galgas-book.tex

%--------------------------------------------------------------
\chapterLabel{Traduction dirigée par la syntaxe}{chapitreTraductionDirigeeParSyntaxe}
%-------------------------------------------------------------

GALGAS permet de construire un \emph{traducteur dirigée par la syntaxe}. Ce type de traduction permet de transformer le texte source d'une grammaire en un autre texte source, tout en conservant les commentaires. C'est donc bien adapté pour mettre à jour des textes sources suite à un changement de syntaxe.

Mettre en place une traduction dirigée par la syntaxe en GALGAS fait appel aux constructions suivantes :
\begin{itemize}
  \item activer la traduction dirigée par la syntaxe pour chaque composant \ggs+syntax+ ;
  \item activer la traduction dirigée par la syntaxe pour le composant \ggs+grammar+ ;
  \item modifier l'instruction \ggs+grammar+, de façon à récupérer les informations de traduction ;
  \item modifier l'instruction d'appel de terminal, de façon à récupérer les informations relatives à l'occurrence du terminal ;
  \item modifier l'instruction d'appel de non terminal, de façon à récupérer la traduction du non terminal ;
  \item appeler l'instruction \galgas{send} pour insérer du texte dans la chaîne produite.
\end{itemize}








\section{Le programme d'exemple}

Pour illustrer les différentes possibilités, on prend pour exemple une grammaire qui analyse les expressions arithmétiques, dont les opérandes sont des identificateurs, et dont les deux opérateurs sont l'addition et la multiplication (l'exemple s'étend facilement à d'autres opérateurs). Les parenthèses sont utilisées pour forcer le groupement.

L'analyseur lexical -- non décrit -- définit les symboles terminaux \galgas{$idf$ !@lstring}, \galgas{$+$}, \galgas{$*$}, \galgas{$($} et \galgas{$)$}.

L'analyseur syntaxique est le suivant :
\begin{galgascode}
syntax expSyntax {
  rule <expression> {
    <terme>
    repeat while $+$ ; <terme> ; end
  }
  rule <terme> {
    <facteur>
    repeat while $*$ ; <facteur> ; end
  }
  rule <facteur> {
    $idf$ ?*
  }
  rule <facteur> {
    $($
    <expression>
    $)$
  }
}
\end{galgascode}

La grammaire :
\begin{galgascode}
grammar expGrammar "LL1" {
  syntax expSyntax
  <expression>
}
\end{galgascode}

La classe de la grammaire (ici \texttt{LL1}) n'a pas d'importance pour la traduction dirigée par la syntaxe : celle-ci fonctionne pour toutes les classes de grammaire. 

Enfin, le lien entre l'extension des fichiers source et l'analyseur est réalisé par le code suivant :
\begin{galgascode}
case . "expression"
message "an '.expression' source file"
??@lstring inSourceFile {
  grammar expGrammar in inSourceFile
}
\end{galgascode}








\section{Activer la traduction dirigée par la syntaxe}

Activer la traduction dirigée par la syntaxe indique à GALGAS d'engendrer le code supplémentaire qui prend en charge la traduction. L'activation doit être indiquée à la fois sur le composant \ggs+syntax+ et le composant \ggs+grammar+ en ajoutant la directive \ggs+%translate+ dans chaque en-tête\footnote{Dans le cas où les règles syntaxiques sont réparties dans plusieurs composants syntaxiques, l'activation doit être indiquée dans tous.}.

Pour le composant \ggs+syntax+ :
\begin{galgascode}
syntax expSyntax %translate {
  ...
\end{galgascode}

Et pour la grammaire :
\begin{galgascode}
grammar expGrammar "LL1" %translate {
  ...
\end{galgascode}

Quand la traduction est activée, l'analyse d'un fichier construit une chaîne de caractères, et par défaut celle-ci est identique à la chaîne source. Par défaut, la chaîne construite est perdue, la section suivante va montrer comment l'obtenir.








\sectionLabel{Obtenir la chaîne traduite}{instructionGrammarEtTraductionDirigeeParLaSyntaxe}

La chaîne traduite est obtenue en modifiant l'instruction \ggs+grammar+ (\refSectionPage{instruction-grammar}). Comme on l'a vu, celle-ci est : 
\begin{galgascode}
grammar expGrammar in inSourceFile
\end{galgascode}

Obtenir la chaine traduite s'exprime en utilisant l'opérateur  \galgas{\:>} :
\begin{galgascode}
grammar expGrammar in inSourceFile :> ?@string s
\end{galgascode}

L'instruction déclare une variable \galgas{s} de type \galgas{@string} et lui affecte la chaîne traduite \footnote{Il existe des variantes pour exprimer l'obtention de la chaîne traduite, voir la description de l'instruction \galgas{grammar} à la \refSectionPage{instruction-grammar}.}.

Par défaut, la chaîne traduite est identique à la chaîne source. Obtenir une chaîne différente est contrôlé par trois instructions :
\begin{itemize}
  \item l'instruction d'appel de terminal, de façon à récupérer les informations relatives à l'occurrence du terminal ;
  \item l'instruction d'appel de non terminal, de façon à récupérer la traduction du non terminal ;
  \item l'instruction \galgas{send} pour insérer du texte dans la chaîne produite.
\end{itemize}







\section{Modifier l'instruction d'appel de terminal}

Une instruction d'appel de terminal a l'allure suivante (par exemple pour \galgas{$idf$}) :
\begin{galgascode}
$idf$ ?*
\end{galgascode}

Par défaut, cette instruction recopie à l'identique dans la chaîne produite deux informations :
\begin{itemize}
  \item les séparateurs qui précèdent le terminal ;
  \item le terminal lui-même.
\end{itemize}

Prenons un exemple. On suppose que la chaîne source est : \galgas{@1@a+@2@b@3@}, les commentaires étant constitués des séquences \galgas{@...@}. Cet exemple considère des commentaires, mais il en est de même pour les séparateurs (espaces, retours à la ligne). La séquence des terminaux encontrés lors de l'analyse de cette phrase est :

\begin{center}
  \begin{tabular}{lllllll@{}}
  \textbf{Instruction} & \textbf{Séparateurs précédent le terminal}  & \textbf{Terminal} \\
  \hline
  \galgas{$idf$ ?*} & \galgas{@1@} & \galgas{a} \\
  \galgas{$+$} &  & \galgas{+} \\
  \galgas{$idf$ ?*} & \galgas{@2@} & \galgas{b} \\
  \hline
  \end{tabular}
\end{center}

Le dernier commentaire (\galgas{@3@}), placé après le dernier symbole non terminal, est toujours ajouté à la fin de la chaîne produite.

Pour obtenir les deux informations attachés à chaque terminal\footnote{Il existe d'autres variantes de cet opérateur, voir la description de l'\emph{instruction d'appel de terminal} à la \refSectionPage{instruction-appel-terminal}.}, on utilise l'opérateur \galgas{\:>} :
\begin{galgascode}
$idf$ ?* :> ?@string separateur ?@string token ;
\end{galgascode}

Cette écriture a pour effet que le séparateur précédent le terminal et le terminal lui-même ne sont plus transmis dans la chaîne traduite, mais affectés respectivement à \galgas{separateur} et à \galgas{token}.

On va prendre un exemple pour illustrer cette construction : produite une chaîne dont les identificateurs et les séparateurs qui les précèdent auront disparus. On modifie le composant \ggs+syntax+ comme suit\footnote{Il existe une expression plus simple de l'instruction \galgas{$idf$ ?* :> ?@string s ?@string t ;}, puisque \galgas{s} et \galgas{t} ne sont pas utilisés : c'est \galgas{$idf$ ?* :> ?* ?* ;}, décrite  à la \refSectionPage{instruction-appel-terminal}.} :
\begin{galgascode}
syntax expSyntax {
  rule <expression> {
    <terme> ;
    repeat while $+$ ; <terme> ; end
  }
  rule <terme> {
    <facteur> ;
    repeat while $*$ ; <facteur> ; end
  }
  rule <facteur> {
    $idf$ ?* :> ?@string s ?@string t ;
  }
  rule <facteur> {
    $($ ;
    <expression> ;
    $)$ ;
  }
}
\end{galgascode}

Si la chaîne source est \galgas{@1@a+@2@b@3@}, alors la chaîne produite est \galgas{+@3@}.

Cette première instruction permet donc de ne pas transmettre les informations attachées un terminal. L'instruction \galgas{send}, décrite à la section suivante, va montrer comment insérer du texte dans la chaîne produite.










\section{Insérer du texte : instruction \texttt{send}}

L'instruction \galgas{send} a la syntaxe suivante\footnote{L'instruction \galgas{send} est décrite à la \refSectionPage{instruction-send-syntaxique}.} :
\lstset{emph={exp}, emphstyle=\galgasEmphStyle}
\begin{galgascode}
send exp
\end{galgascode}

\galgas{exp} est une expression de type \galgas{@string}. Son comportement est simple : la valeur de l'expression chaîne de caractères est simplement transmise à la chaîne produite.

Par exemple, supposons que l'on veuille transformer les parenthèses en accolades ; on écrit le composant \ggs+syntax+ comme suit\footnote{Là encore, il existe une forme plus concise de l'instruction \galgas{$($ :> ?@string s ?@string t ;}, puisque \galgas{t} est inutilisé : c'est \galgas{$($ :> ?@string s ?* ;}, décrite  à la \refSectionPage{instruction-appel-terminal}.} :
\begin{galgascode}
syntax expSyntax {
  rule <expression> {
    <terme>
    repeat while $+$ <terme> end
  }
  rule <terme> {
    <facteur>
    repeat while $*$ <facteur> end
  }
  rule <facteur> {
    $idf$ ?*
  }
  rule <facteur> {
    $($ :> ?@string s ?@string t ; send s . "{"
    <expression>
    $)$ :> ?@string s ?@string t ; send s . "}"
  }
}
\end{galgascode}

Mentionner \galgas{s} dans l'instruction \galgas{send} permet de transmettre les séparateurs qui précèdent les parenthèses. Ainsi à partir de la chaîne source \galgas{(@1@a+@2@b)@3@}, on obtient \galgas{\{@1@a+@2@b\}@3@}.


L'instruction \galgas{send} permet de reconstituer le comportement par défaut de l'instruction d'appel de terminal : par exemple, \galgas{$($ \:> ?@string s ?@string t ; send s . t ;} a le même effet que \galgas{$($ ;}.


Attention, l'instruction \galgas{send} est une instruction syntaxique. Cela signifie que le code suivant est incorrect :
\lstset{emph={condition, A, B}, emphstyle=\galgasEmphStyle}
\begin{galgascode}
if condition then
  send A # Erreur
else
  send B # Erreur
end
\end{galgascode}

L'analyse des instructions \galgas{send A ;} et  \galgas{send B ;} déclenche une erreur ; en effet, les branches d'une instruction \galgas{if} ne peuvent contenir que des instructions sémantiques. Les instructions \galgas{send} ne peuvent figurer que directement dans des règles de production, soient dans les branches des instructions \galgas{select}, \galgas{repeat} ou \galgas{parse}. Pour contourner cette interdiction, écrire :
\begin{galgascode}
@string s
if condition then
  s = A
else
  s = B
end
send s
\end{galgascode}



\section{Modifier l'instruction d'appel de non-terminal}

L'instruction d'appel de non terminal capture la chaîne obtenue par la dérivation de ce non terminal :
\begin{galgascode}
<expression>
\end{galgascode}


Par défaut, cette chaîne est ajoutée à la chaîne produite.

Là encore, l'opérateur \galgas{\:>} permet d'effectuer une interception. On écrit :
\begin{galgascode}
<expression> :> ?@string e
\end{galgascode}

La chaîne obtenue par la dérivation du non terminal \galgas{<expression>} n'est pas ajoutée à la chaîne produite, mais affectée à la variable \galgas{e}. D'une manière analogue à l'instruction d'appel de terminal, l'instruction \galgas{send} permet de retrouver le comportement par défaut :
\begin{galgascode}
<expression> :> ?@string e ; send e
\end{galgascode}

On utilise souvent cette construction pour ne pas transmettre la chaîne obtenue par la dérivation d'un non terminal ; par exemple, si on ne veut pas transmettre les expressions entre parenthèses, on modifie la dernière règle \galgas{facteur} en\footnote{Ou encore : \galgas{<expression> \:> ?*}.} :
\begin{galgascode}
syntax expSyntax {
  ...
  rule <facteur> {
    $($
    <expression> :> ?@string e
    $)$
  }
}
\end{galgascode}









\part{Composants}

%!TEX encoding = UTF-8 Unicode
%!TEX root = ../galgas-book.tex

%--------------------------------------------------------------
\chapter{Project Component}\index{Component!Project}
%-------------------------------------------------------------


\section{Generated Cocoa Application}

When a project component is compiled with a Xcode project target, a \texttt{project\_xcode} directory is created. This directory contains:
\begin{itemize}
\item the Xcode project file;
\item a \texttt{build.command} file ;
\item an \texttt{Info.plist} file ;
\item an \texttt{English.lproj} directory ;
\item an empty \texttt{userResources} directory.
\end{itemize}

The \texttt{Info.plist}, the \texttt{English.lproj} directory and the \texttt{userResources} directory are used by the Cocoa target of the Xcode project. The \texttt{build.command} file is a command file that builds the Xcode project.

All files you put in the \texttt{userResources} directory are added to the Cocoa target of the Xcode project when the GALGAS Project component is compiled. When the Cocoa target of the Xcode project is compiled, theses files are put in the \texttt{Resources} directory within the application bundle.

Adding files to the \texttt{userResources} directory is the way of customizing the Cocoa Application:
\begin{itemize}
\item adding icons to your Application (\refSubsectionPage{addingIconsCocoaApplication});
\item customizing syntax coloring (\refSubsectionPage{customizingSyntaxColoring}). 
\end{itemize}




\subsectionLabel{Adding Icons to your Cocoa Application}{addingIconsCocoaApplication}




\subsectionLabel{Customizing Syntax Coloring}{customizingSyntaxColoring}

%!TEX encoding = UTF-8 Unicode
%!TEX root = ../galgas-book.tex

%--------------------------------------------------------------
\chapter{Projet \texttt{Xcode} et application Cocoa}
%-------------------------------------------------------------

Vous pouvez demander à GALGAS d'engendrer un projet \texttt{Xcode}, qui contiendra :
\begin{itemize}
  \item le compilateur en version \emph{release} sous la forme d'un utilitaire en ligne de commande ; 
  \item le compilateur en version \emph{debug} sous la forme d'un utilitaire en ligne de commande ; 
  \item une application Cocoa permettant d'appeler les deux utilitaires.
\end{itemize}






\section{Paramétrage du projet GALGAS}

Pour engendrer un projet \texttt{Xcode}, il vous suffit d'ajouter une déclaration telle que \ggs+%Mavericks+ dans votre fichier projet (d'extension \tpp{.galgasProject}). Par exemple :

\begin{galgas}
project (0:0:1) -> "logo" {
  %Mavericks
  %applicationBundleBase : "fr.what"
  ...
\end{galgas}

Un projet \texttt{Xcode} définit la version de Mac OS pour laquelle il va être compilé : évidemment, \ggs+%Mavericks+ définit la version \texttt{Mavericks} (10.9). Le \refTableau{options-pour-xcode} liste les différents options possibles. GALGAS fixe la version indiquée dans le projet \texttt{Xcode}, et il faut ensuite que la version de \texttt{Xcode} utilisée soit compatible avec ce réglage. L'option \ggs+%LatestMacOS+ correspond au réglage correspondant du projet \texttt{Xcode} engendré.

Il y a une seconde option à ajouter dans le projet GALGAS : \ggs+%applicationBundleBase+. Celle-ci fixe le \emph{Bundle Identifier} de l'application Cocoa. À la chaîne définie dans l'option (ici \ggs+"fr.what"+) est ajouté le nom du projet (défini dans l'en-tête, ici \ggs+"logo"+), précédé par un point : le \emph{Bundle Identifier} est donc \tpp{fr.what.logo}.


\begin{table}[t]
  \centering
  \begin{tabular}{rl}
    \textbf{Option} & \textbf{Version Mac OS correspondante}\\
    \ggs+%SnowLeopard+ & \texttt{SnowLeopard} (10.6) \\
    \ggs+%Lion+ & \texttt{Lion} (10.7) \\
    \ggs+%MountainLion+ & \texttt{Mountain Lion} (10.8) \\
    \ggs+%Mavericks+ & \texttt{Mavericks} (10.9) \\
    \ggs+%Yosemite+ & \texttt{Yosemite} (10.10) \\
    \ggs+%LatestMacOS+ & Dernière version Mac OS supportée par \texttt{Xcode} \\
  \end{tabular}
  \caption{Options du projet GALGAS indiquant la version Mac OS}
  \labelTableau{options-pour-xcode}
  \ligne
\end{table}







\section{Projet \texttt{Xcode} engendré}


Quand le projet GALGAS est compilé, un répertoire \tpp{xcode-project} directory est créé, et contient :
\begin{itemize}
\item le fichier projet \texttt{Xcode} ;
\item un fichier \tpp{build.command} ;
\item un fichier \tpp{Info.plist} ;
\item un répertoire \tpp{English.lproj} ;
\item un répertoire \tpp{userResources}.
\end{itemize}

Le rôle de chacun est précisé par le \refTableau{fichiers-repertoires-xcode}. Ne pas modifier ces fichiers et répertoires à la main, une compilation GALGAS supprimerait vos changements. La seule exception est le contenu du répertoire \tpp{userResources} qui n'est pas modifié par les compilations GALGAS.

\begin{table}[t]
  \centering
  \begin{tabular}{rp{11cm}}
    \textbf{Fichier ou répertoire} & \textbf{Rôle}\\
    \tpp{build.command} & Effectue la compilation Xcode, appelable via une commande \emph{Shell} \\
    \tpp{Info.plist}    & Informations pour l'application Cocoa \\
    \tpp{English.lproj} & Informations pour l'application Cocoa \\
    \tpp{userResources} & Permet d'associer des icônes aux fichiers sources de votre compilateur, ainsi qu'à l'application Cocoa engendrée (voir \refSectionPage{ajouterIconesAppliCocoa}) \\
  \end{tabular}
  \caption{Fichiers et répertoires relatifs au projet Xcode}
  \labelTableau{fichiers-repertoires-xcode}
  \ligne
\end{table}





\sectionLabel{Définir des icônes pour votre application Cocoa}{ajouterIconesAppliCocoa}

Vous pouvez définir :
\begin{itemize}
  \item une icône pour l'application Cocoa ;
  \item une icône particulière pour chaque type de fichier source.
\end{itemize}

Le nom de chaque fichier d'icône fixe son rôle :
\begin{itemize}
  \item pour l'application Cocoa, le fichier d'icône doit s'appeler \tpp{application\_icns.icns} ;
  \item pour chaque type de fichier source, le nom est basé sur l'extension du fichier : si celui-ci est par exemple \tpp{.logo}, le fichier d'icônes doit s'appeler \tpp{logo\_icns.icns}.
\end{itemize}

Ces fichiers d'icônes doivent être placés dans le répertoire \tpp{userResources}, et il faut ensuite refaire une compilation GALGAS pour que ces fichiers soient ajoutés au projet \texttt{Xcode}.

En résumé :
\begin{enumerate}
  \item concevoir les fichiers d'icônes, en fixant leur nom comme indiqué ci-dessus ;
  \item placer ces icônes dans le répertoire \tpp{userResources} ;
  \item effectuer une compilation GALGAS : celle-ci met à jour le projet \texttt{Xcode}, en ajoutant les fichiers d'icônes au \emph{target} Cocoa ;
  \item recompiler le \emph{target} Cocoa du projet \texttt{Xcode} : les icônes sont prises en compte.
\end{enumerate}











%\sectionLabel{Customizing Syntax Coloring}{customizingSyntaxColoring}
%
%This feature enables to set particular display attributes to a given list of tokens. This list is defined by a plist file located in the \emph{Resources} directory of the application bundle.
%
%{1} Edit the GALGAS lexique component, and add one (or more) \ggs+style+ entries. For example:
%
%\begin{galgas}
%lexique my_lexique :
%  ...
%style mySpecificStyle -> "My Style" ;
%  ...
%end lexique ;
%\end{galgas}
%
%This new style's feature can be edited as other styles, by the Preferences setting of your Cocoa application.
%
%
%{2} Create a plist file with the \emph{Property List Editor} application. This file should be named with the lexique component name, suffixed by \tpp{-syntax-coloring-adds}: so, for the example, the file name is \tpp{my\_lexique-syntax-coloring-adds.plist}. Put this file in the \emph{userResources} directory: so when the GALGAS project document is compiled, this file is added to the Cocoa Target of the Xcode project. 
%
%{3} Edit the \tpp{my\_lexique-syntax-coloring-adds.plist} with the \emph{Property List Editor} application or Xcode. Add one entry for every custom syntax coloring case: the \emph{key} is the terminal spelling, the \emph{value} has the \emph{String} type, and the specific style name. For example, the \refFigure{}{customSyntaxColoringPropertyList} shows the assignment of the terminal which spelling is \tpp{begin} by the \tpp{mySpecificStyle} style.
%
%\begin{figure}[t]
%  \centering
%  \includegraphics[width=15cm]{chapter-cocoa-features/custom-syntax-coloring-property-list-edition.pdf}
%  \caption{Example of a syntax coloring property list}
%  \labelFigure{customSyntaxColoringPropertyList}
%  \ligne
%\end{figure}
%
%If your provides an undefined style name, you will be warned every time you open a document by a beep and a small explanation window.






\sectionLabel{Indexation des fichiers sources}{indexingYourSourceFiles}

Vous pouvez configurer votre projet GALGAS pour que l'application Cocoa engendrée établisse une indexation et des références croisées : un \tpp{cmd-click} affiche un menu contextuel. Cette indexation est basée sur l'analyse syntaxique. C'est ce qui a été fait pour l'application \texttt{CocoaGalgas} (\refFigure{}{indexingUnderCocoaGALGAS}). On voit dans le menu contextuel trois classes d'index : \tpp{Class Definition}, \tpp{Class Reference as Superclass} et \tpp{Abstract Category Method Definition} ; au dessous, les références croisées correspondantes.


%You can configure your project for enabling cross-referencing entities with your Cocoa application. This has been done in GALGAS, providing such feature (\refFigure{}{indexingUnderCocoaGALGAS}). The contextual menu is displayed with a \texttt{cmd-click}.

\begin{figure}[t]
  \centering
  \includegraphics[width=16cm]{chapter-cocoa-features/indexing-sample.png}
  \caption{Indexation et références croisées dans l'application CocoaGalgas}
  \labelFigure{indexingUnderCocoaGALGAS}
  \ligne
\end{figure}

Pour configurer votre projet, vous avez à modifier le composant \emph{lexique}, le composant \emph{syntax}, le composant \emph{grammar}, et la règle d'analyse du fichier source. Les cinq modifications sont présentées successivement ci-après, en prenant comme exemple le langage LOGO (\refSectionPage{presentation-logo}).





\subsection{En tête du composant \texttt{lexique}}

Il faut modifier l'en-tête, en ajoutant la déclaration  \ggs+indexing in+ :


\begin{galgas}
lexique logo_lexique indexing in "INDEXING" {
  ...
\end{galgas}

La chaîne \ggs+"INDEXING"+ définit le nom du répertoire qui contient les fichiers cache de l'indexation. Ce répertoire est relatif au répertoire qui contient le fichier source.

Note : si vous effectuez maintenant la compilation GALGAS, vous obtiendrez une erreur sur la définition de la grammaire, indiquant qu'elle doit aussi indiquer la prise en compte de l'indexation.




\subsection{En tête du composant \texttt{grammar}}

Il suffit de préfixer par \ggs+indexing+ l'en-tête du composant \ggs+grammar+ :

\begin{galgas}
indexing grammar logo_grammar ... {
  ...
\end{galgas}

Note : maintenant, la compilation GALGAS s'effectue sans erreur.




\subsection{Règle d'analyse des fichiers sources}

La règle d'analyse des fichiers source doit mentionner dans l'en-tête la grammaire utilisée pour l'analyse (pour l'exemple du langage LOGO, la troisième ligne \ggs+grammar logo_grammar+ remplit ce rôle).

\begin{galgas}
case . "logo"
message "a source text file with the .logo extension"
grammar logo_grammar
?sourceFilePath:@lstring inSourceFile {
  grammar logo_grammar in inSourceFile
}
\end{galgas}

Quand le mode d'exécution (absence de l'option \tpp{-{}-mode}) est le mode par défaut, les instructions de la règle sont exécutées. Ci-dessus, la seule instruction est \ggs+grammar logo_grammar in inSourceFile+ (ligne 5).

Quand le mode d'exécution (présence de l'option \tpp{-{}-mode}) n'est pas le mode par défaut, les instructions de la règle ne sont pas exécutées, et les opérations sont guidées par la grammaire indiquée ligne 3. Dans le cas de l'indexation, l'exécution construit l'indexation du fichier source.









\subsection{Déclaration des classes d'index}

La déclaration des classes d'index s'effectue dans l'analyseur lexical. Dans la cadre du langage d'exemple LOGO, on veut simplement indéxer les routines, plus précisément l'endroit de leur définition, et les endroits où elles sont appelées. On définit donc deux classes d'index \ggs+routineDefinition+ et \ggs+routineCall+. À chaque déclaration est associée une chaîne de caractères, qui sera le titre affiché dans le menu contextuel. 


\begin{galgas}
lexique logo_lexique indexing in "INDEXING" {
  ...
indexing routineDefinition : "Routine Definition"
  ...
indexing routineCall : "Routine call"
  ...
\end{galgas}


Ces définitions peuvent être placées à tout endroit dans la définition de l'analyseur lexical.








\subsection{Définition des entrées indexées}

L'analyseur syntaxique va être complété de façon à définir les symboles qui seront indéxés. Plus précisement, c'est l'instruction d'analyse de symbole terminal qui est modifiée.

Considérons d'abord la déclaration de routine. La règle de l'analyseur syntaxique qui définit cette analyse est :

\begin{galgas}
rule <routine_definition> {
  $ROUTINE$
  $identifier$ ?let @lstring routineName
  $BEGIN$
  <instruction_list>
  $END$
}
\end{galgas}

Le nom de la routine est défini par l'instruction \ggs+$identifier$ ?let @lstring routineName+ : on la modifie alors de façon à signifier que l'indentificateur doit être indéxé comme une définition de routine :

\begin{galgas}
rule <routine_definition> {
  $ROUTINE$
  $identifier$ ?let @lstring routineName indexing routineDefinition
  $BEGIN$
  <instruction_list>
  $END$
}
\end{galgas}

Maintenant, l'instruction d'appel de routine :

\begin{galgas}
rule <instruction> {
  select
    $CALL$
    $identifier$ ?let @lstring routineName
    $;$
  or
    ...
  end
}
\end{galgas}

On modifie de manière analogue l'instruction \ggs+$identifier$ ?let @lstring routineName+ :

\begin{galgas}
rule <instruction> {
  select
    $CALL$
    $identifier$ ?let @lstring routineName indexing routineCall
    $;$
  or
    ...
  end
}
\end{galgas}



\subsection{Compilation et essai}

Les modifications sont terminées. Vous pouvez recompiler votre projet (compilation GALGAS puis compilation de la cible Cocoa du projet \texttt{Xcode}). La \refFigure{}{exemple-indexation-logo} montre le résultat obtenu en effectuant un \tpp{cmd-click} sur le nom de la routine.

\begin{figure}[t]
  \centering
  \includegraphics[width=8cm]{chapter-cocoa-features/exemple-indexation-logo.png}
  \caption{Exemple d'indexation en LOGO}
  \labelFigure{exemple-indexation-logo}
  \ligne
\end{figure}



%\noindent{4} \textbf{Program component configuration.} Insert the \ggs+grammar+ declaration after the « \texttt{message ...} » declaration in every program rule concerned by indexing:
%
%\begin{galgas}
%case ...
%message ...
%grammar my_grammar
%?@lstring inSourceFile {
%  ...
%\end{galgas}
%
%
%
%
%
%
%\noindent{5} \textbf{Define indexing entries.} The indexing entries are defined within the rules of syntax components. The \emph{terminal check} instruction is the unique way for definition, by naming an index class name:
%
%\begin{galgas}
%syntax ... ("my_lexique.gLexique") :
%  ...
%rule ... :
%  ...
%  $identifier$ ? ... indexing myIndexClass1 ;
%  ...
%end rule ;
%  ...
%\end{galgas}
%
%Any kind of terminal symbol accepts an « \texttt{indexing} » attribute : keywords, delimiters, literal string, integers, identifiers, \dots
%
%Several index class names can be named, using a comma as separator:
%\begin{galgas}
%  ...
%  $identifier$ ? ... indexing myIndexClass1, myIndexClass2 ;
%  ...
%\end{galgas}
%
%
%
%
%
%
%\noindent{6} \textbf{Compile and play.} Now, you can compile and run the Cocoa Application. With a \texttt{cmd}-click on an indexed terminal symbol, the contextual menu is displayed. You can delete the indexing directory at any moment, it will be rebuilt as needed.








%!TEX encoding = UTF-8 Unicode
%!TEX root = ../galgas-book.tex

%--------------------------------------------------------------
\chapter{Le composant \texttt{lexique}}
%-------------------------------------------------------------

Le rôle d'un analyseur lexical est de grouper les caractères de la chaîne d'entrée en \emph{symboles terminaux}, ou encore \emph{terminaux}, en écartant les séparateurs comment les espaces ou les commentaires. 

En GALGAS, un analyseur lexical est défini par un composant \ggs+lexique+. Les composants \ggs+syntax+, qui définissent un ensemble de règles de production, font référence à un composant \ggs+lexique+.









\section{Définition d'un composant \texttt{lexique}}


En GALGAS, un composant \ggs+lexique+ a la structure suivante :

\begin{galgas}
lexique nom {
  declarations
}
\end{galgas}

Le \ggs+nom+ est le nom donné au composant ; il est utilisé pour référencer le composant \ggs+lexique+ dans un composant \ggs+syntax+.


Dans un composant \ggs+lexique+, cinq types de déclarations sont définis :
\begin{itemize}
  \item déclaration d'attribut lexical ;
  \item déclaration d'un symbole terminal ;
  \item déclaration d'une liste de symboles terminaux ;
  \item déclaration d'un message d'erreur lexical ;
  \item déclaration d'un style ;
  \item déclaration de règles d'analyse.
\end{itemize}

A //lexical attribute// carries the value associated with a terminal symbol: for example, the integer value of a literal integer constant, the string value of a character string constant, ...

In GALGAS, all terminal symbols must be declared either by a //single terminal symbol declaration//, either by a //list of terminal symbols declaration//. This defines the set of defined terminal symbols of your grammar.

Lexical error messages need also to be explicitly declared by //lexical error message declaration//. 

A //style declaration// declares a style identifier, for defining automatic coloring in a text editor. Currently, coloring is only available for Mac OS X Cocoa applications.

The order of declarations is not significant, but any entity must be declared before being used.

==== Lexical Rules Overview ====
The //lexical rules// define the executable part of a lexical component. Every lexical rule define //matching strings// that are are tested against substring from current location in input string. A matching string has a one character or more.

%\section{Fichiers engendrés}
%
%A lexical component description is translated in C++ code; for every lexical component, GALGAS generates a specific C++ class:
%  * the name of the class is the name of the \ggs+lexique+ component;
%  * this class is declared in a header file that is named the name of the \ggs+lexique+ component with the ''\textquotesingle.h\textquotesingle'' extension;
%  * this class is implemented in a file that is named the name of the \ggs+lexique+ component with the ''\textquotesingle.cpp\textquotesingle'' extension;
%  * this class inherits from ''C\_Lexique'' class (declared in ''libpm/galgas/C\_Lexique.h'' and implemented in ''libpm/galgas/C\_Lexique.cpp'').
%
%The two generated files are generated according the [[generated\_files|GALGAS file generation process]].


\section{Comment opère un analyseur lexical}

You can consider the lexical analyzer as an autonomous thread which analyzes the input string and which sends the sequence of the terminal symbols to the parser. Of course, for efficiency, the lexical analyzer is actually a parser subroutine.

The flowchart of a GALGAS lexical analyzer execution is:

{{ how\_works\_a\_lexical\_analyzer.png }}

When the input string is loaded from source file, a ''NUL'' character is appended as End Of String (eos) mark.

During execution, the lexical analyzer maintains a //current location// that designates the next character of the input string to be analyzed. Initially, current location points out the first character of the input string.

The lexical analyzer loops until the end of input string is reached. At the beginning of every loop, lexical attributes are reset to their default value.

Then, the first lexical rule matching expressions are tested against substring at current location in input string:
  * on match success, the first lexical rule is executed; usually, this execution sends a terminal symbol to the parser; however, in some cases as parsing a delimitor or a comment, no terminal symbol is sent;
  * on match failure, the lexical analyzer tries to find a match with the second lexical rule, and so on.

If no lexical rule matches, the character at current location is tested against eos character. On match success, the lexical analyzer sends once a predefined terminal symbol (denoted by ''\\$\\$'') to the parser, for telling it the end of input string is reached. On match failure, the //unknow character// lexical error is raised. The character at current location is discarded, that is the current location points out the next character of the input string.

\section{Ambiguïtés lexicales}

**GALGAS does not currently check that the set of lexical rules is unambiguous.** So, if the set is unambiguous, the rule order is not significant; if two or more rules introduce an ambiguity, the first defined one is used. 

\section{Un exemple}

This is very simple scanner, from ''galgas/samples/notSLRgrammar.ggs'':

|''**lexique** my\_scanner\_for\_not\_SLR\_grammar:\\ 
\#--- Identifiers\\ 
\\$id\\$ error **message** %%"%%an identifier%%"%% ;\\ 
**rule** \textquotesingle{a}\textquotesingle -> \textquotesingle{z}\textquotesingle | \textquotesingle{A}\textquotesingle -> \textquotesingle{Z}\textquotesingle :\\ 
 send \\$id\\$ ;\\ **end** **rule** ;\\ 
\#--- Delimitors\\ 
**list** delimitorsList error **message** %%"%%the %%'"%% . * . %%"'%% delimitor%%"%%: %%"%%=%%"%% , %%"%%*%%"%% ;\\ 
**rule** **list** delimitorsList ;\\ 
\#--- Separators\\ 
**rule** \textquotesingle\textbackslash{1}\textquotesingle -> %%' '%%:\\ 
**end** **rule** ;\\ 
**end** **lexique** ;''|

This \ggs+lexique+ component defines the following set of terminal symbols: ''\\$id\\$'' (explicitly declared), ''\\$=\\$'' and ''\\$*\\$'' (declared  by ''delimitorsList'' list.

The first rule sends the ''\\$id\\$'' terminal symbol each time a lower case or upper case character is found. The second rule names the ''delimitorsList'' list and sends the ''\\$=\\$'' or ''\\$*\\$'' terminal symbol each time the corresponding character is found. The last rule discards silently the space character and any control character.

Note that this scanner considers identifiers of only one character: ''ab'' is scanned as two consecutive identifiers.

===== Finding Sample Code =====

You can find examples of \ggs+lexique+ components in:
  * ''galgas/sample/alt\_sample.ggs'' file; this is a very basic scanner that handles one-letter identifier and four delimitors;
  * ''galgas/sample/arith\_expression.ggs'' file (for scanning literal integers); 
  * ''galgas/sample/test\_LR1\_grammar.ggs'' file gives an example of a small scanner for "toy" parser;
  * ''galgas/galgas/galgas\_sources/galgas\_scanner.ggs'' file: this is the actual scanner of the GALGAS language, and scans identifiers, keywords, delimiters, literal integers, literal characters, literal character strings, galgas type names (the '@' character followed by a sequence of letters), comments, ...   

\section{Déclarations lexicales}

\subsection{Déclaration d'un symbole terminal}

The //single terminal symbol declaration// declares a name used for naming a terminal symbol. This declaration just performs declaration, not scanning. For sending this terminal symbol to the parser, it must be named in a ''send'' lexical instruction within a lexical rule.

The declaration associates to the terminal symbol a possibly empty list of lexical attributes and a syntax error message (not a //lexical// error message), defined by a character string.

First example:

|''\$literal\_integer\$ error **message** %%"%%a decimal number%%"%%;''|

This declaration names no lexical attribute. Consequently, when the lexical send instruction ''send \$literal\_integer\$;'' will be called from a lexical rule, only the terminal symbol will be sent to the parser, but not the literal integer value. The parser has no way to get the actual value: all integer values share the same terminal symbol. It is sufficient for a pure parser, however a real compiler needs the actual value.

Second example:

|''@uint unsignedValueAttribute;\\ 
\$literal\_integer\$ !unsignedValueAttribute error **message** %%"%%a decimal number%%"%%;''|

In this declaration, the ''unsignedValueAttribute'' attribute is named in the terminal symbol declaration. So, when the lexical send instruction ''send \$literal\_integer\$;'' will be called from a lexical rule, the terminal symbol will be sent to the parser together with the unsigned value of the ''unsignedValueAttribute'' attribute, enabling the semantic instructions to catch it.

\subsection{Déclaration d'une liste de symboles terminaux}

The //list of terminal symbol declaration// associates to a name a list of terminal symbols with a generic syntax error message. It is typically used for declaring the keywords and the delimiters.

An example of key words declaration:

| ''**list** keywordList error **message** %%"%%the '%K' key word%%"%%: %%"%%if%%"%%, %%"%%then%%"%%, %%"%%else%%"%% ;'' |

The declared terminal symbols are: ''\$if\$'', ''\$then\$'', ''\$else\$''. The actual syntax error message is built from generic error message by replacing ''%K'' with terminal symbol string (for outputing a single ''%'', write ''%''''%''). So the syntax error message associated to the ''\$if\$'' terminal symbol is: "''the 'if' key word''".

An other example is a delimitor list declaration:

|''**list** delimitorList error **message** %%"%%the '%K' delimitor%%"%%: %%"%%.%%"%%, %%"%%;%%"%%, %%"%%(%%"%%, %%"%%)%%"%% ;''|

Actual scanning of a delimitor is done by a ''**rule** **list**'' lexical instruction.

\subsection{Déclaration d'un attribut terminal}

Lexical attributes carry values associated with terminal symbol. GALGAS handles string, unsigned, character, float lexical attributes. Every lexical attribute needs to be declared and its declaration names a GALGAS type name.


 The following table summerizes the attributes features and type notation:

%\^ Attribute Type \^ Type Name \^ Default Value \^ Corresponding C++ type \^
| ASCII String | ''@string'' | ''%%""%%'' (the empty string) | ''C\_String'' |
| ASCII Character | ''@char'' | ''%%'\0'%%'' | ''char'' |
| 32-bit Unsigned Integer | ''@uint'' | ''0'' | ''uint32'' |
| 32-bit Signed Integer | ''@sint'' | ''0'' | ''sint32'' |
| Float | ''@double'' | ''0.0'' | ''double'' |

In GALGAS, type names are identifiers prefixed by a ''@'' character.

An ''@string'', ''@char'', ''@uint'', ''@sint'', ''@double'' lexical attribute carry a string, character, unsigned, signed, double value.

In a ''**syntax**'' component, information that defines the location of the scanned terminal symbol in the input string is added to attribute value: so an ''@string'' object in the lexique component corresponds to an ''@lstring'' object in the syntax component. Location information is used by the parser and the semantic instructions for building syntax and semantic error messages that indicates //where// the error is located.

The //default value// is the one used at the beginning of every scanning loop for resetting lexical attribute.

The //corresponding C type// is useful if you want to write your own lexical actions (in C++). Please note that this correspondance is **only** available for lexical actions, and not for semantic action. The ''C\_String'' type is a C++ class that handles mutable character strings, without being worried about memory management. It is declared in the ''libpm/strings/C\_string.h'' file. The ''uint32'' type is the 32-bit unsigned integer type, and the ''sint32'' type is the 32-bit signed integer type. 
 

\subsection{Déclaration d'un message d'erreur lexicale}

The //lexical error message declaration// associates a name to a string. These error messages are used in lexical actions, and define the message that are displayed when a lexical error occurs.

|  ''**message** decimalNumberTooLarge: %%"%%decimal number too large%%"%%;'' |

 

\section{Règles lexicales}

There are two kinds of //lexical rules//:
  - the //list lexical rule//;
  - the //single lexical rule//.

\subsection{Règle s'appuyant sur une liste}

This is the simpliest form: it just names a previously defined list of terminal symbols; for example:

|''**rule** **list** delimitorList;''|

//Matching expressions// are the set of strings defined by the list. This rule tries to find a substring from input string at current location that matches a terminal symbol string defined in the list, sorted by decreasing length (so longest strings are tested first). On match success, //executing the rule// consists of sending the corresponding terminal symbol.

This kind of rule is typically used for scanning for a delimitor.

\subsection{Simple règle}

A //single lexical rule// has the following form:

|''**rule** //matching\_expression//:\\  //lexical\_instructions//\\ **end** **rule**;''|

The //matching expression// defines a set of matching strings, that are tested against the substring from input string at current location. On match, the //lexical instructions// are executed.

A matching expression can be:
  - a one-character string (for example, ''\textquotesingle{a}\textquotesingle'' matches the ''a'' character);
  - an union of one-character strings, defined by a character subrange (for example, ''\textquotesingle{a}\textquotesingle -> \textquotesingle{z}\textquotesingle'' matches a lower case letter);
  - a one or more characters string (for example, ''%%"%%=%%"%%'' matches the corresponding string);
  - an union of above (for example: ''\textquotesingle{A}\textquotesingle -> \textquotesingle{Z}\textquotesingle | \textquotesingle{a}\textquotesingle -> \textquotesingle{z}\textquotesingle'' matches a lower or upper case letter).

On match success, the current location is moved to designate the character after the matching string.

\section{Instructions lexicales}


\subsectionLabel{Instruction lexicale \texttt{select}}{instructionSelectLexical}

The //lexical select instruction// is the following:

|''**select**\\ **when** //matching\_expression\_1\_in\_select//: //lexical\_instructions\_1//\\ **when** //matching\_expression\_2\_in\_select//: //lexical\_instructions\_2//\\ ...\\ default //default\_lexical\_instructions//\\ **end** **select**;''|

A //lexical select instruction// has one or more ''**when**'' branches.

//matching expression\_1\_in\_select//, //matching expression\_2\_in\_select// conform to the defined above //matching\_expression//.

This instruction tries to match the different //matching expressions// until a matching success is found. In such case, the corresponding //lexical instructions// are executed. If all matching fail, the //default lexical instructions// are executed.

\subsectionLabel{Instruction lexicale \texttt{repeat}}{instructionRepeatLexical}

The //lexical repeat instruction// is the following:

|''**repeat**\\  //lexical\_instructions\_0//\\ **while** //matching\_expression\_1\_in\_repeat//: //lexical\_instructions\_1//\\ **while** //matching\_expression\_2\_in\_repeat//: //lexical\_instructions\_2//\\ ...\\ **end** **repeat**;''|

A //lexical while instruction// has one or more ''**while**'' branches.

//matching expression\_1\_in\_repeat//, //matching expression\_2\_in\_repeat// can be:
  - an expression conform to the defined above //matching\_expression//;
  - the ''~ //string//'' construct: the match succeeds when the //string// **is not** the current string;
  - the ''~ //string1//, //string2//, ...'' construct: the match succeeds when neither of //string1//, //string2//, ... are the current string.

This instruction first executes the //lexical instructions 0//. Then, it tries to match the different //matching expressions// until a matching success is found. In such case, the corresponding //lexical instructions// are executed, then the instruction is executed again (from //lexical instructions 0//). If all matching fail, execution of this instruction is complete (excution goes on the next instruction).

\subsection{Appel d'une action lexicale}

The //lexical action call instruction// calls a C++ defined method for performing computation and checking on lexical attributes. Its syntax is the following:

|''lexical\_action\_name (parameter, ...) ;''|

or

|''lexical\_action\_name (parameter, ...) error message\_name, ... ;''|

A lexical action is designated by its name. It accepts one or more parameters, and zero, one or more messages names.

A parameter is:
  - either a lexical attribute,
  - either a lexical function call;
  - either the joker character ''\textquotesingle*\textquotesingle'' that represents the character at current location.

A lexical action can be predefined or defined by the user. Predefined lexical actions are actually methods of ''C\_Lexique'' class (the generated scanner is a class that inherits from this class). User defined lexical actions must be implemented as methods of the generated scanner class.

**Note that no parameter type checking, no error message count checking is performed by GALGAS. ** A parameter type error or a message count error is detected at C++ compilation stage.
 
\subsection{Appel d'une fonction lexicale}

The //lexical function call// calls a C++ defined method for performing computation on lexical attributes. It can only appear as parameter of a lexical action call or a parameter of an other lexical function call. Its syntax is the following:

|''lexical\_function\_name (parameter, ...) ;''|

A lexical function is designated by its name. It accepts one or more parameters.

A lexical function parameter is:
  - either a lexical attribute,
  - either a lexical function call;
  - either the joker character ''\textquotesingle*\textquotesingle'' that represents the character at current location.

A lexical function can be predefined or defined by the user. Predefined lexical actions are actually methods of ''C\_Lexique'' class (the generated scanner is a class that inherits from this class). User defined lexical functions must be implemented as methods of the generated scanner class.

**Note that no parameter type checking is performed by GALGAS. ** A parameter type error is detected at C++ compilation stage.
 
\subsection{Instruction lexicale \texttt{error}}

The //lexical error instruction// raises a lexical error. Its syntax is:

|''error message\_name ;''|

The //message name// is the name of a previously declared lexical error message.

\subsection{Instruction lexicale \texttt{send}}

The //lexical send instruction// sends a terminal symbol to the parser. It has several forms:

=== First Form ===

|''send terminal\_symbol ;''|

This instruction sends inconditionnaly the //terminal symbol// to the parser.

=== Second Form ===

|''send search //attribute\_name// in //lexical\_list// default terminal\_symbol ;''|

This instruction first search for //attribute name// value in the //lexical list//. If found, the corresponding terminal symbol is sent to the parser. If not found, the default //terminal symbol// is sent.

Several consecutive ''search'' are accepted, allowing sequential searching in different lists:

|''send search //attribute\_name\_1// in //lexical\_list\_1// default search //attribute\_name\_2// in //lexical\_list\_2// default terminal\_symbol ;''|

\subsectionLabel{Instruction lexicale \texttt{drop}}{instructionLexicaleDrop}

|Available in GALGAS 1.5.6 and later.|


The //lexical drop instruction// does not send any terminal symbol to the parser. It is only significant for lexical coloring (see [[\#coloring\_comments|coloring comments]]).

This instruction names a terminal symbol:
|''**drop** //terminal\_symbol// ;''|


\subsection{Instruction lexicale \texttt{tag}}

|Available in GALGAS 1.5.6 and later.|

This instruction declares a new //tag identifier//.

|''**tag** //tag\_identifier// ;''|

A ''**tag**'' instruction records a location in the scanned file. The only way to use the declared tag identifier is the [[\#lexical\_rewind\_instruction|lexical rewind instruction]].

\subsection{Instruction lexicale \texttt{rewind}}

|Available in GALGAS 1.5.6 and later.|

|''**rewind** //tag\_identifier// send //terminal\_symbol//;''|

This instruction rewinds the scanned location from the tag identifier value, and sends the terminal symbol to the parser.








\section{Routines lexicales prédéfinies}

Lexical routine calls are instructions. Lexical function calls can appear as actual output parameters of routine calls and function calls. GALGAS predefines several lexical routines and several lexical functions (listed below).

A lexical routine accepts:
  * zero, one or more input/output or input formal arguments;
  * zero, one or more error messages.

Running the \texttt{-{}-print-predefined-lexical-actions} command line option lists all predefined routines and functions prototype.

\subsection{Routine \texttt{codePointToUnicode}}

\begin{galgas}
codePointToUnicode !@string inCodePointString
                   ?!@string ioString
\end{galgas}

\subsection{Routine \texttt{convertDecimalStringIntoSInt}}

\begin{galgas}
convertDecimalStringIntoSInt !@string inString
                             ?!@sint ioSignedNumber
                             error inNumberTooLargeError,
                                   inCharacterIsNotDecimalDigitError
\end{galgas}

\subsection{Routine \texttt{convertDecimalStringIntoSInt64}}

\begin{galgas}
convertDecimalStringIntoSInt64 !@string inString
                               ?!@sint64 ioSignedNumber
                               error inNumberTooLargeError,
                                     inCharacterIsNotDecimalDigitError
\end{galgas}

\subsection{Routine \texttt{convertDecimalStringIntoUInt}}

\begin{galgas}
convertDecimalStringIntoUInt !@string inString
                             ?!@uint ioUnsignedNumber
                             error inNumberTooLargeError,
                                   inCharacterIsNotDecimalDigitError
\end{galgas}

\subsection{Routine \texttt{convertDecimalStringIntoUInt64}}

\begin{galgas}
convertDecimalStringIntoUInt64 !@string inString
                               ?!@uint64 ioUnsignedNumber
                               error inNumberTooLargeError,
                                     inCharacterIsNotDecimalDigitError
\end{galgas}

\subsection{Routine \texttt{convertHTMLSequenceToUnicodeCharacter}}

\begin{galgas}
convertHTMLSequenceToUnicodeCharacter ?!@string inString
                                      ?!@char ioUnicodeCharacter
                                      error inUnassignedHTMLSequenceError
\end{galgas}

\subsection{Routine \texttt{convertHexStringIntoSInt}}

\begin{galgas}
convertHexStringIntoSInt !@string inString
                         ?!@sint ioSignedNumber
                         error inNumberTooLargeError,
                               inCharacterIsNotHexDigitError
\end{galgas}

\subsection{Routine \texttt{convertHexStringIntoSInt64}}

\begin{galgas}
convertHexStringIntoSInt64 !@string inString
                           ?!@sint64 ioSignedNumber
                           error inNumberTooLargeError,
                                 inCharacterIsNotHexDigitError
\end{galgas}

\subsection{Routine \texttt{convertHexStringIntoUInt}}

\begin{galgas}
convertHexStringIntoUInt !@string inString
                         ?!@uint ioUnsignedNumber
                         error inNumberTooLargeError,
                               inCharacterIsNotHexDigitError
\end{galgas}

\subsection{Routine \texttt{convertHexStringIntoUInt64}}

\begin{galgas}
convertHexStringIntoUInt64 !@string inString
                           ?!@uint64 ioUnsignedNumber
                           error inNumberTooLargeError,
                                 inCharacterIsNotHexDigitError
\end{galgas}

\subsection{Routine \texttt{convertStringToDouble}}

\begin{galgas}
convertStringToDouble !@string inString
                      ?!@double ioDouble
                      error inConversionError
\end{galgas}

This action tries to convert the string value of the first argument into a double value. On success, the resulting double is set to the second argument. The conversion error message is displayed on conversion error.

\subsection{Routine \texttt{convertUInt64ToSInt64}}

\begin{galgas}
convertUInt64ToSInt64 !@uint64 inUnsignedNumber
                      ?!@sint64 ioSignedNumber
                      error inNumberTooLargeError
\end{galgas}

If the unsigned value of the ''inUnsignedNumber'' argument is greater than ''2<sup>63</sup>-1'', the error is raised. Otherwise, the value is assigned to the ''ioSignedNumber'' argument.

\subsection{Routine \texttt{convertUIntToSInt}}

\begin{galgas}
convertUIntToSInt !@uint inUnsignedNumber
                  ?!@sint ioSignedNumber
                  error inNumberTooLargeError
\end{galgas}

If the unsigned value of the ''inUnsignedNumber'' argument is greater than ''2<sup>31</sup>-1'', the error is raised. Otherwise, the value is assigned to the ''ioSignedNumber'' argument.

\subsection{Routine \texttt{convertUnsignedNumberToUnicodeChar}}

\begin{galgas}
convertUnsignedNumberToUnicodeChar ?!@uint inUnsignedNumber
                                   ?!@char ioUnicodeCharacter
                                   error inUnassignedUnicodeValueError
\end{galgas}

\subsection{Routine \texttt{enterBinDigitIntoUInt}}

\begin{galgas}
enterBinDigitIntoUInt !@char inCharacter
                      ?!@uint ioUnsignedNumber
                      error inNumberTooLargeError,
                            inCharacterIsNotBinDigitError
\end{galgas}

\subsection{Routine \texttt{enterBinDigitIntoUInt64}}

\begin{galgas}
enterBinDigitIntoUInt64 !@char inCharacter
                        ?!@uint64 ioUnsignedNumber
                        error inNumberTooLargeError,
                              inCharacterIsNotBinDigitError
\end{galgas}

\subsection{Routine \texttt{enterCharacterIntoCharacter}}

\begin{galgas}
enterCharacterIntoCharacter ?!@char ioCharacter
                            !@char inCharacter
\end{galgas}

This routine performs ''ioCharacter = inCharacter'' assignment.

\subsection{Routine \texttt{enterCharacterIntoString}}

\begin{galgas}
enterCharacterIntoString ?!@string ioString
                         !@char inCharacter
\end{galgas}

Appends the character value of the second argument to the string value of the first argument. The resulting string is set to the first argument.

\subsection{Routine \texttt{enterDigitIntoASCIIcharacter}}

\begin{galgas}
enterDigitIntoASCIIcharacter ?!@char ioASCIICharacter
                             !@char inDecimalDigitCharacter
                             error inErrorCodeGreaterThan255,
                                   inErrorNotDecimalDigitCharacter
\end{galgas}

Build an ASCII character from its decimal definition.

First, the character value of the ''inDecimalDigitCharacter'' argument is tested to be a valid decimal digit, that is in one range ''[\textquotesingle0\textquotesingle, \textquotesingle9\textquotesingle]''. On failure, the ''inErrorNotDecimalDigitCharacter'' error message is displayed. On success, the unsigned value of the ''ioASCIICharacter'' argument is multiplied by ten, and is added the decimal value corresponding to second argument. If the result is lower or equal to ''2<sup>8</sup>-1'', it is set to the ''ioASCIICharacter'' argument. Otherwise, the ''inErrorCodeGreaterThan255'' error is raised.

Note: this lexical action treats characters as unsigned values.

\subsection{Routine \texttt{enterDigitIntoUInt}}

\begin{galgas}
enterDigitIntoUInt !@char inDecimalDigitCharacter
                   ?!@uint ioUnsignedNumber
                   error inNumberTooLargeError,
                         inCharacterIsNotDecimalDigitError
\end{galgas}

First, the value of ''inDecimalDigitCharacter'' argument is tested to be in the range ''[\textquotesingle0\textquotesingle, \textquotesingle9\textquotesingle]''. On failure, the ''inCharacterIsNotDecimalDigitError'' error message is displayed. On success, the unsigned value of the first argument is multiplied by ten, and is added the decimal value corresponding to the ''ioUnsignedNumber'' argument. If the result is lower or equal to ''2<sup>32</sup>-1'', it is set to the ''ioUnsignedNumber'' argument. Otherwise, the ''inNumberTooLargeError'' error is raised.

\subsection{Routine \texttt{enterDigitIntoUInt64}}

\begin{galgas}
enterDigitIntoUInt64 !@char inDecimalDigitCharacter
                     ?!@uint64 ioUnsignedNumber
                     error inNumberTooLargeError,
                           inCharacterIsNotDecimalDigitError
\end{galgas}

First, the value of ''inDecimalDigitCharacter'' argument is tested to be in the range ''[\textquotesingle0\textquotesingle, \textquotesingle9\textquotesingle]''. On failure, the ''inCharacterIsNotDecimalDigitError'' error message is displayed. On success, the unsigned value of the first argument is multiplied by ten, and is added the decimal value corresponding to the ''ioUnsignedNumber'' argument. If the result is lower or equal to ''2<sup>64</sup>-1'', it is set to the ''ioUnsignedNumber'' argument. Otherwise, the ''inNumberTooLargeError'' error is raised.

\subsection{Routine \texttt{enterHexDigitIntoASCIIcharacter}}

\begin{galgas}
enterHexDigitIntoASCIIcharacter ?!@char ioASCIICharacter
                                !@char inHexDigitCharacter
                                error inErrorCodeGreaterThan255,
                                      inErrorNotHexDigitCharacter
\end{galgas}

Build an ASCII character from its hexadecimal definition.

First, the character value of the ''inHexDigitCharacter'' argument is tested to be a valid hexadecimal digit, that is in one of the ranges ''[\textquotesingle0\textquotesingle, \textquotesingle9\textquotesingle]'', ''[\textquotesingle{a}\textquotesingle, \textquotesingle{f}\textquotesingle]'', ''[\textquotesingle{A}\textquotesingle, \textquotesingle{F}\textquotesingle]''. On failure, the ''inErrorNotHexDigitCharacter'' error message is displayed. On success, the unsigned value of the first argument is multiplied by sixteen, and is added the hexadecimal value corresponding to ''ioASCIICharacter'' argument. If the result is lower or equal to ''2<sup>8</sup>-1'', it is set to the ''ioASCIICharacter'' argument. Otherwise, the ''inErrorCodeGreaterThan255'' error is raised.

Note: this lexical action treats characters as unsigned values.

\subsection{Routine \texttt{enterHexDigitIntoUInt}}

\begin{galgas}
enterHexDigitIntoUInt !@char inHexDigitCharacter
                      ?!@uint ioUnsignedNumber
                      error inNumberTooLargeError,
                            inCharacterIsNotHexDigitError
\end{galgas}

First, the character value of the ''inHexDigitCharacter'' argument is tested to be a valid hexadecimal digit, that in one of the the ranges ''[\textquotesingle0\textquotesingle, \textquotesingle9\textquotesingle]'', ''[\textquotesingle{a}\textquotesingle, \textquotesingle{f}\textquotesingle]'', ''[\textquotesingle{A}\textquotesingle, \textquotesingle{F}\textquotesingle]''. On failure, the ''inCharacterIsNotHexDigitError'' error message is displayed. On success, the unsigned value of the ''ioUnsignedNumber'' argument is multiplied by sixteen, and is added the hexadecimal value corresponding to second argument. If the result is lower or equal to ''2<sup>32</sup>-1'', it is set to the ''ioUnsignedNumber'' argument. Otherwise, the first error is raised.

\subsection{Routine \texttt{enterHexDigitIntoUInt64}}

\begin{galgas}
enterHexDigitIntoUInt64 !@char inHexDigitCharacter
                        ?!@uint64 ioUnsignedNumber
                        error inNumberTooLargeError,
                              inCharacterIsNotHexDigitError
\end{galgas}

First, the character value of the ''inHexDigitCharacter'' argument is tested to be a valid hexadecimal digit, that in one of the the ranges ''[\textquotesingle0\textquotesingle, \textquotesingle9\textquotesingle]'', ''[\textquotesingle{a}\textquotesingle, \textquotesingle{f}\textquotesingle]'', ''[\textquotesingle{A}\textquotesingle, \textquotesingle{F}\textquotesingle]''. On failure, the ''inCharacterIsNotHexDigitError'' error message is displayed. On success, the unsigned value of the ''ioUnsignedNumber'' argument is multiplied by sixteen, and is added the hexadecimal value corresponding to second argument. If the result is lower or equal to ''2<sup>64</sup>-1'', it is set to the ''ioUnsignedNumber'' argument. Otherwise, the first error is raised.

\subsection{Routine \texttt{enterOctDigitIntoUInt}}

\begin{galgas}
enterOctDigitIntoUInt !@char inString
                      ?!@uint ioUnsignedNumber
                      error inNumberTooLargeError,
                            inCharacterIsNotOctDigitError
\end{galgas}

\subsection{Routine \texttt{enterOctDigitIntoUInt64}}

\begin{galgas}
enterOctDigitIntoUInt64 !@char inString
                        ?!@uint64 ioUnsignedNumber
                        error inNumberTooLargeError,
                              inCharacterIsNotOctDigitError
\end{galgas}

\subsection{Routine \texttt{multiplyUInt}}

\begin{galgas}
multiplyUInt !@uint inUnsignedNumber
             ?!@uint ioUnsignedNumber
             error inResultTooLargeError
\end{galgas}

Multiply the ''ioUnsignedNumber'' value by ''inUnsignedNumber'' value. Detection of overflow is performed.

\subsection{Routine \texttt{multiplyUInt64}}

\begin{galgas}
multiplyUInt64 !@uint inUnsignedNumber
               ?!@uint64 ioUnsignedNumber
               error inResultTooLargeError
\end{galgas}

Multiply the ''ioUnsignedNumber'' value by ''inUnsignedNumber'' value. Detection of overflow is performed.

\subsection{Routine \texttt{negateSInt}}

\begin{galgas}
negateSInt ?!@sint ioNumber
           error inNumberTooLargeError
\end{galgas}

\subsection{Routine \texttt{negateSInt64}}

\begin{galgas}
negateSInt64 ?!@sint64 ioNumber
             error inNumberTooLargeError
\end{galgas}


\subsection{Routine \texttt{resetString}}

\begin{galgas}
resetString ?!@string ioString
\end{galgas}








\section{Fonctions lexicales prédéfinies}


A lexical function accepts:
  * zero, one or more input formal arguments.

Running the \texttt{-{}-print-predefined-lexical-actions} command line option lists all predefined routines and functions prototype.

\subsection{Fonction \texttt{toLower}}

\begin{galgas}
toLower ?@char inCharacter -> @char
\end{galgas}

If the character value of the argument is an upper case letter, this function returns the corresponding lower case letter. Otherwise, it returns the unchanged character value of the argument.

\subsection{Fonction \texttt{toUpper}}

\begin{galgas}
toUpper ?@char inCharacter -> @char
\end{galgas}


If the character value of the argument is an lower case letter, this function returns the corresponding upper case letter. Otherwise, it returns the unchanged character value of the argument.



\section{Définir vos propres actions et fonctions lexicales}

You can define your own lexical actions and functions in C++ and make them available to called by lexical action call instructions.

\subsection{Où ?}

You must define your lexical actions and functions as a method of the C++ class generated by compilation of the \ggs+lexique+ component. You need to modify the generated code, adding method prototype declaration in class declaration.

**So that the method declaration that you added is not deleted at the time of a future compilation, define it in user zone 2 of the generated header file.** For more details, see [[generated\_files |file generation process page]].

For implementing your method, you can insert it in user zone 2 of the generated implementation file (for more details, see [[generated\_files |file generation process page]]). Alternatively, you can implement it in any other file, provided you include the needed header files.

\subsection{Correspondance entre les appels d'actions GALGAS et C++}

This table gives the correspondance between lexical argument types and C++ types. **Note this correspondance is only available for lexical arguments**.

%\^Lexical Formal Argument Type  \^C++ Type  \^
|''? @string''  |''**const** C\_String \&''|
|''?! @string''  |''C\_String \&''|
|''? @char''  |''**const** **char**''|
|''?! @char''  |''**char** \&''|
|''? @uint''  |''**const** uint32''|
|''?! @uint''  |''uint32 \&''|
|''? @sint''  |''**const** sint32''|
|''?! @sint''  |''sint32 \&''|
|''? @double''  |''**const** **double**''|
|''?! @double''  |''**double** \&''|

''?'' means the formal argument has input passing mode: it cannot be modified by the lexical action. ''?!'' means the formal argument has in/out passing mode: its value is got from the caller, can modified by the lexical action and is returned to the caller.

An error message argument corresponds to the C++ type ''**const** **char** *''.

In C++ generated code, the method call instruction generated by lexical action call names the lexical action name, prefixed by ''scanner\_routine\_''.

For example, consider the ''convertStringToDouble'' lexical action described below. This corresponds to the following method prototype:

''**void** scanner\_routine\_convertStringToDouble (**const** C\_String \&, **double** \&, **const char** *) ;''
==== Defining Action and Function Prototype ====

The prototype must conform to the rules presented in the [[\#Correspondance between Lexical Action Calls and C++ Called Methods|above]] section.

%\^Remember that GALGAS does not perform any checking on lexical action calls. Errors are detected at C++ compilation stage.\^

\section{Exemples d'analyseurs lexicaux}

\subsection{Analyser des identificateurs}

|''@string identifierString;\\ 
\$identifier\$ !identifierString error **message** %%"%%an identifier%%"%%;\\ 
**rule** %%'a'->'z' | 'A'->'Z'%%:\\ 
 **repeat**\\ 
  enterCharacterIntoString !?identifierString !* ;\\ 
 **while** %%'a'->'z' | 'A'->'Z' | '\_' | '0'->'9'%%:\\ 
 **end** **repeat** ;\\ 
 send \$identifier\$ ;\\
**end** **rule** ;''|

|''@string identifierString;\\ 
\$identifier\$ !identifierString error **message** %%"%%an identifier%%"%%;\\ 
**rule** %%'a'->'z' | 'A'->'Z'%%:\\ 
 **repeat**\\ 
  enterCharacterIntoString !?identifierString !toLower (!*) ;\\ 
 **while** %%'a'->'z' | 'A'->'Z' | '\_' | '0'->'9'%%:\\ 
 **end** **repeat** ;\\ 
 send \$identifier\$ ;\\
**end** **rule** ;''|

\subsection{Analyser des identificateurs et des mots-clés}

|''@string identifierString;\\ 
\\ 
\$identifier\$ !identifierString error **message** %%"%%an identifier%%"%%;\\ 
\\ 
**list** keywordList error **message** %%"the '%K' key word": "begin", "else", "end"%%;\\
\\ 
**rule** %%'a'->'z' | 'A'->'Z'%%:\\ 
 **repeat**\\ 
  enterCharacterIntoString !?identifierString !* ;\\ 
 **while** %%'a'->'z' | 'A'->'Z' | '\_' | '0'->'9'%%:\\ 
 **end** **repeat** ;\\ 
 send search identifierString in keywordList  default \$identifier\$ ;\\
**end** **rule** ;''|

\subsection{Analyser des délimiteurs}

|''**list** galgasDelimitorsList **error message** %%"the '%K' delimitor"%%:\\ 
 %%"*",  "|", ",",  ".",  "<>", "::", ">",  "<",  ";",  ":",%%\\ 
 %%"-",  "(", ")",  "->", "?", "==", "?", "!",  "=", "...",%%\\ 
 %%"[",  "]", "+=", "?!", "!?", "/",  "!=", "<=", ">=", "\&",%%\\ 
 %%"++", "{", "}"%% ;\\ 
\\ 
**rule list** galgasDelimitorsList ;''|

\subsection{Analyser des séparateurs}

|''**rule** %%'\u0001' -> ' '%% :\\ 
**end rule** ;''|

\subsection{Analyser des commentaires}

|''**rule** '\#' :\\ 
 **repeat**\\ 
 **while** %%'\u0001' -> '\u0009' | '\u000B' -> '\uFFFD'%% :\\ 
 **end repeat** ;\\ 
**end rule** ;''|

\subsection{Analyser des entiers décimaux non signés}

|''\$unsigned\_literal\_integer\$ !ulongValue **error message** %%"a decimal number"%% ;\\ 
\$signed\_literal\_integer\$ !longValue error **message** %%"a signed decimal number"%% ;\\ 
\\ 
**message** decimalNumberTooLarge : %%"decimal number too large"%% ;\\ 
\\ 
**message** internalError : %%"internal error"%% ;\\ 
\\ 
**rule** %%'0'->'9'%% :\\ 
 enterDigitIntoUlong !?ulongValue !* error decimalNumberTooLarge, internalError ;\\ 
 **repeat**\\ 
 **while** %%'0'->'9'%% :\\ 
  enterDigitIntoUlong !?ulongValue !* error decimalNumberTooLarge, internalError ;\\ 
 **while** %%'\_'%% :\\ 
 **end repeat** ;\\ 
 **select**\\ 
 **when** %%'S' | 's'%% :\\ 
  convertUlongToLong !?longValue !ulongValue %%error%% decimalNumberTooLarge ;\\ 
  send \$signed\_literal\_integer\$ ;\\ 
 default\\ 
  send \$unsigned\_literal\_integer\$ ;\\ 
 **end select** ;\\ 
**end rule** ;''|

\subsection{Analyser des entiers hexadécimaux non signés}

\subsection{Analyser des constantes caractère}

|''\$literal\_char\$ ! charValue **error message** %%"a character constant"%% ;\\ 
\\ 
**message** incorrectCharConstant : %%"incorrect literal character"%% ;\\ 
\\ 
**message** ASCIIcodeTooLargeError : %%"ASCII code > 255"%% ;\\ 
\\ 
**rule** %%'\''%% :\\ 
 **select**\\ 
 **when** %%'\\'%% :\\ 
  **select**\\ 
  **when** %%'f'%% :\\ 
   enterCharacterIntoCharacter !?charValue !%%'\f'%% ;\\ 
  **when** %%'n'%% :\\ 
   enterCharacterIntoCharacter !?charValue !%%'\n'%% ;\\ 
  **when** %%'r'%% :\\ 
   enterCharacterIntoCharacter !?charValue !%%'\r'%% ;\\ 
  **when** %%'t'%% :\\ 
   enterCharacterIntoCharacter !?charValue !%%'\t'%% ;\\ 
  **when** %%'v'%% :\\ 
   enterCharacterIntoCharacter !?charValue !%%'\v'%% ;\\ 
  **when** %%'\\'%% :\\ 
   enterCharacterIntoCharacter !?charValue !%%'\\'%% ;\\ 
  **when** %%'0'%% :\\ 
   enterCharacterIntoCharacter !?charValue !%%'\0'%% ;\\ 
  **when** %%'\''%% :\\ 
   enterCharacterIntoCharacter !?charValue !%%'\''%% ;\\ 
  **when** %%'0' -> '9'%% :\\ 
   **repeat**\\ 
    enterHexDigitIntoASCIIcharacter !?charValue !* error ASCIIcodeTooLargeError, internalError ;\\ 
   **while** %%'0' -> '9'%% :\\ 
   **end repeat** ;\\ 
  default\\ 
   error incorrectCharConstant ;\\ 
  **end select** ;\\ 
 **when** %%' ' -> '\uFFFD'%% :\\ 
  enterCharacterIntoCharacter !?charValue !* ;\\ 
 default\\ 
  error incorrectCharConstant ;\\ 
 **end select** ;\\ 
 **select**\\ 
 **when** %%'\''%% :\\ 
  send \$literal\_char\$ ;\\ 
 default\\ 
  error incorrectCharConstant ;\\ 
 **end select** ;\\ 
**end rule** ;''|

\subsection{Analyser des constantes chaîne de caractères}

\subsection{Analyser des constantes flottantes}

|''\$literal\_double\$ !floatValue !tokenString **error message** %%"a float number"%%;\\ 
\\ 
\$.\$ **error message** %%"the '.' delimitor"%%;\\ 
\\ 
**message** floatNumberConversionError : %%"invalid float number"%% ;\\ 
\\ 
**rule** %%'.'%% :\\ 
 **select**\\ 
 **when** %%'0'->'9'%% :\\ 
  enterCharacterIntoString !?tokenString !%%'0'%% ;\\ 
  enterCharacterIntoString !?tokenString !%%'.'%% ;\\ 
  enterCharacterIntoString !?tokenString !* ;\\ 
  **repeat**\\ 
  **while** %%'0'->'9'%% :\\ 
   enterCharacterIntoString !?tokenString !* ;\\ 
  **while** %%'\_'%% :\\ 
  **end repeat** ;\\ 
  convertStringToDouble !tokenString !?floatValue error floatNumberConversionError ;\\ 
  send \$literal\_double\$ ;\\ 
 default\\ 
  send \$.\$ ;\\ 
 **end select** ;\\
**end rule** ;''|

\section{\emph{Back tracking} avec les instructions \texttt{tag} et \texttt{rewind}}

|Available in GALGAS 1.5.6 and later.|

The ''**tag**'' and ''**rewind**'' instructions can be used for performing back tracking.

The first example is the way the non terminal symbols are scanned in GALGAS 1.5.6 (and later).

A non terminal is composed of a single '<' character, followed by a letter, zero, one or more letters, digits or underscore characters, is ended by a single '>' character. For example ''<abcdef>'' is a valid non terminal. However, ''<abcdef >'' is //not// a valid non terminal (because of the space before the final '>' character): it is considered as a '<' delimitor, followed by the ''abcdef'' identifier and by the '>' delimitor.

In the file ''galgas/galgas\_sources/galgas\_scanner.ggs'', the three delimitors befgging with a '<' character and the non terminal symbols are scanned by the following code:

''\$<\$ **error message** "the '<' delimitor" **style** delimitersStyle ;''\\
''%%\$<=\$%% **error message** "the '<=' delimitor" **style** delimitersStyle ;''\\
''%%\$<<\$%% **error message** "the '<<' delimitor" **style** delimitersStyle ;''\\
''\$non\_terminal\_symbol\$ ! tokenString **error message** "a non terminal symbol <...>" **style** nonTerminalStyle ;''\\

''**rule** '<' :''\\
'' **tag** onlyInfDelimiter ;''\\
'' **select**''\\
'' **when** '=' :''\\
'' send %%\$<=\$%% ;''\\
'' **when** '<' :''\\
''  send %%\$<<\$%% ;''\\
'' **when** %%'a' -> 'z' | 'A' ->'Z'%% :''\\
''  **repeat**''\\
''   enterCharacterIntoString !?tokenString !* ;''\\
''  **while** %%'a' -> 'z' | 'A' ->'Z' | '0' -> '9' | '\_'%% :''\\
''  **end repeat** ;''\\
''  **select**''\\
''  **when** '>' :''\\
''   send \$non\_terminal\_symbol\$ ;''\\
''  default''\\
''   **rewind** onlyInfDelimiter send \$<\$ ;''\\
''  **end select** ;''\\
'' default''\\
''  send \$<\$ ;''\\
'' **end select** ;''\\
''**end rule** ;''\\

The ''**tag**'' instruction records a scanning location. When the final '>' character is not found, the scanner is rewinded at the character following the '<' character, and the ''\$<\$'' terminal is sent. On next scanning, an identifier (or a key word) will be found.

The second examples shows how to scan for integer constants, float constants, and array bounds in Pascal :
  * an integer constant is a (non empty) sequence of digits ;
  * a float constant is a (non empty) sequence of digits, following by a dot and a (possibly empty) sequence of digits;
  * an array bound is an integer constant, followed by the '..' delimitor (two dots) and an integer constant.

The problem is that ''1..2'' should not be scanned as a float constant, a single dot delimitor, and an integer constant.

This can be achieved by the following code:

''**rule** %%'0' -> '9'%% :''\\
'' **repeat**''\\
'' **while** %%'0' -> '9'%% :''\\
'' **end repeat** ;''\\
'' **tag** endOfIntegerConstant ;''\\
'' **select**''\\
'' **when** %%'.'%% :''\\
''  **select**''\\
''  **when** %%'.'%% :''\\
''   **rewind** endOfIntegerConstant send \$integer\_constant\$ ;''\\
''  **when** %%'0' -> '9'%% :''\\
''   **repeat**''\\
''   **while** %%'0' -> '9'%% :''\\
''   **end repeat** ;''\\
''   send \$float\_constant\$ ;''\\
''  default''\\
''   send \$float\_constant\$ ;''\\
''  **end select** ;''\\
'' default''\\
''  send \$integer\_constant\$ ;''\\
'' **end select** ;''\\
''**end rule** ;''\\


\section{Ajouter la coloration lexicale (sur Mac uniquement)}

With GALGAS, you can easily embbed your compiler in a GUI application (currently available only for Mac OS X). This application has a built-in text editor, from which you can modify, save and compile source file. With //style declarations//, you can add automatic coloring in the built-in text editor.

A //style declaration// associates a message to a style identifier. For example:

|''**style** keywordsStyle -> %%"%%Keywords:%%"%% ;''|

The associated message is used in application preferences window as a comment of each color selection item.

A //style declaration// does not link a style identifier to any terminal symbol. You need to add this information to //single terminal symbol declaration// and //list of terminal symbols declaration// by naming the style identifier after the syntax error message:

|''\$literal\_integer\$ error **message** %%"%%a decimal number%%"%% **style** integerStyle;''|

|''**list** delimitorList error **message** %%"%%the '%%"%% . * . %%"%%' delimitor%%"%% **style** keywordsStyle: %%"%%.%%"%%, %%"%%;%%"%%, %%"%%(%%"%%, %%"%%)%%"%%;''|

\subsection{Exemple : les styles de l'analyseur lexical GALGAS}

As an example, you can take a look on GALGAS scanner, in ''galgas/galgas\_sources/galgas\_scanner.ggs'' file. The style declarations are the following:

|''**style** keywordsStyle -> %%"%%Keywords:%%"%% ;\\ **style** delimitersStyle -> %%"%%Delimiters:%%"%% ;\\ **style** terminalStyle -> %%"%%Terminal symbols:%%"%% ;\\ **style** integerStyle -> %%"%%Integer constants:%%"%% ;\\ **style** characterStyle -> %%"%%Character constants:%%"%% ;\\ **style** stringStyle -> %%"%%String constants:%%"%% ;\\ **style** typeNameStyle -> %%"%%Type names (@...):%%"%%'';|

You can search for the occurrence of style identifiers, to see how they are used.

In Cocoa GALGAS application, the Color tab of the Preferences window lists all style comments, each of them being associated to a ''NSColorWell'' for color selection:

{{cocoa\_galgas\_color\_styles.png}}

Note that no default color is defined in style declaration. Until you define yourself a color from Preference window, it defaults to black color.

\subsection{Appliquer un style aux commentaires}
|Available in GALGAS 1.5.6 and later.|

In GALGAS 1.5.6 and later, you can define a color for comments. Proceed as follows:
  - declare a new terminal symbol, for example ''\$comment\$'';
  - declare a style for this new terminal symbol;
  - when a comment is scanned, use the ''**drop**'' instruction for naming the new terminal symbol (instead of the usual ''send'' instruction).

The ''**drop**'' instruction is only significant for syntax coloring.

For example, GALGAS comments are defined in ''galgas/galgas\_sources/galgas\_scanner.ggs'' in this way:

''**style** commentStyle -> "Comments:" ;''\\
''...''\\
''\$comment\$ error **message** %%"%%a comment%%"%% **style** commentStyle ;''\\
''**rule** %%'\#'%% :''\\
'' **repeat**''\\
'' **while** %%'\u0001' -> '\u0009' | '\u000B' | '\u000C' | '\u000E' -> '\uFFFD'%% :''\\
'' **end repeat** ;''\\
'' **drop** \$comment\$ ;''\\
''**end rule** ;''\\

%!TEX encoding = UTF-8 Unicode
%!TEX root = ../galgas-book.tex

%--------------------------------------------------------------
\chapter{Syntax and Grammar Components}
%-------------------------------------------------------------

\section {GALGAS and Context-Free Grammars}


\section{Writing a Syntax Component}\index{Component!Syntax}

\section{Syntax Instructions}

\subsection{Terminal Symbol Instruction}

\subsection{Non Terminal Symbol Instruction}


\subsection{Repeat Instruction}


\subsection{Select Instruction}



\subsection{Parse Instruction}

\subsubsection{Parse do ... Instruction}


\subsubsection{Parse loop ... Instruction}


\subsubsection{Parse when ... Instruction}


\section{Writing a Grammar Component}\index{Component!Grammar}


%!TEX encoding = UTF-8 Unicode
%!TEX root = ../galgas-book.tex

%--------------------------------------------------------------
\chapter{Graphic User Interface Component}\index{Component!Graphic User Interface}
%-------------------------------------------------------------

%!TEX encoding = UTF-8 Unicode
%!TEX root = ../galgas-book.tex

%--------------------------------------------------------------
\chapterLabel{Le composant \texttt{option}}{composantOption}
%-------------------------------------------------------------


Le composant \galgas{option} permet de définir des options qui sont appelables à partir de la ligne de commande. Dans le code, la valeur d'une option est obtenue à partir de l'opérande \emph{appel d'une option}, décrit dans la \refSubsectionPage{appelOption}.

Voici l'exemple d'un composant \galgas{option} qui déclare une option (évidement, un composant \galgas{option} peut déclarer un nombre quelconque d'options) :
\begin{galgascode}
option nom_composant {
  @bool nom_option : 'S', "asm" -> "Extract assembly code"
}
\end{galgascode}


\section{Déclaration d'une option}

La déclaration d'une option présente le syntaxe suivante :
\begin{galgascode}
  @T nom_option : caractere, chaine -> description
\end{galgascode}

Les cinq champs qui définissent une option sont :
\begin{itemize}
  \item \galgas{@T} : le type de l'option ; trois types sont autorisés : \galgas{@bool}, \galgas{@uint} et \galgas{@string} ;
  \item \galgas{nom_option} : c'est le nom, interne à GALGAS, qui permettra de désigner l'option dans l'\emph{appel d'une option} (\refSubsectionPage{appelOption}) ; 
  \item \galgas{caractere} : le caractère qui activera l'option dans la ligne de commande ; par exemple, en écrivant \galgas{'A'}, l'option sera activée par \texttt{-A} dans la ligne de commande ; si vous ne voulez pas d'activation par un caractère, écrivez \galgas{'\\0'} ;
  \item \galgas{chaine} : la chaîne de caractères qui activera l'option dans la ligne de commande ; par exemple, en écrivant \galgas{"ABEDEF"}, l'option sera activée par \texttt{-{}-ABCDEF} dans la ligne de commande ; si vous ne voulez pas d'activation par une chaîne, écrivez \galgas{""} ;
  \item \galgas{description} : une chaîne de caractère qui contient une description de l'option, qui sera affichée par l'option \texttt{-{}-help} de votre compilateur.
\end{itemize}








\section{Option booléenne}

Le champ qui définit le type de l'option est \galgas{@bool} ; par exemple :
\begin{galgascode}
  @bool nom_option : 'S', "asm" -> "Extract assembly code"
\end{galgascode}

Dans la ligne de commande, l'option est activée par \texttt{-A} ou \texttt{-{}-asm}.

Par défaut, l'option n'est pas activée, et sa valeur associée est \galgas{false}. Quand l'option est activée dans la ligne de commande, sa valeur associée est \galgas{true}.








\section{Option entière}

Le champ qui définit le type de l'option est \galgas{@uint} ; par exemple :
\begin{galgascode}
  @uint nom_option : 'M', "max-iterations-count" -> "Max of iteration count"
\end{galgascode}

Dans la ligne de commande, l'option est activée par \texttt{-N=xxx} ou \texttt{-{}-max-iterations-count=xxx}, où \texttt{xxx} est un nombre entier positif ou nul (et inférieur ou égal à $2^{32}-1$).

Par défaut, l'option n'est pas activée, et sa valeur associée est $0$. Quand l'option est activée dans la ligne de commande, sa valeur associée est la valeur \texttt{xxx}. Ainsi, l'option \texttt{-N=0}, comme l'option \texttt{-{}-max-iterations-count=0} n'a aucun effet.










\section{Option chaîne de caractères}

Le champ qui définit le type de l'option est \galgas{@string} ; par exemple :
\begin{galgascode}
  @string nom_option : 'F', "file-name" -> "File name"
\end{galgascode}

Dans la ligne de commande, l'option est activée par \texttt{-F=abc} ou \texttt{-{}-file-name=abc}, où \texttt{abc} est une chaîne de caractères sans espaces. Si vous voulez entrer une chaîne de caractères qui comprend des espaces, écrivez : \texttt{"-F=abc"} ou \texttt{"-{}-file-name=abc"}.

Par défaut, l'option n'est pas activée, et sa valeur associée est la chaîne vide. Quand l'option est activée dans la ligne de commande, sa valeur associée est la chaîne \texttt{abc}. Ainsi, l'option \texttt{-F=}, comme l'option \texttt{-{}-file-name=} n'a aucun effet.



%!TEX encoding = UTF-8 Unicode
%!TEX root = ../galgas-book.tex

%--------------------------------------------------------------
\chapter{Program Component}\index{Component!Program}
%-------------------------------------------------------------









\part{Le système de types}
  %!TEX encoding = UTF-8 Unicode
%!TEX root = ../galgas-book.tex

\chapter{Présentation du système de types}






\section{Opérations définies pour tous les types}

Tout type implémente implicitement :
\begin{itemize}
  \item l'opérateur \galgas{==} ;
  \item l'opérateur \galgas{\!=} ;
  \item le \emph{reader} \galgas{description} ;
  \item le \emph{reader} \galgas{dynamicType} ;
  \item le \emph{reader} \galgas{object}.
\end{itemize}

La plupart des types implémentent le constructeur par défaut \galgas{default} (voir \refSectionPage{constructeurParDefaut}). 


\subsection{L'opérateur \texttt{==}}

\begin{galgascode}
operator @T == -> @bool ;
\end{galgascode}

Cet opérateur permet de tester l'identité entre de deux objets de même type. 

\subsection{L'opérateur \texttt{!=}}

\begin{galgascode}
operator @T != -> @bool ;
\end{galgascode}

Cet opérateur permet de tester la non identité entre de deux objets de même type. Il renvoie le complément logique du résultat de l'application de l'opérateur \galgas{==}.





\subsection{Le reader \texttt{description}}

\begin{galgascode}
reader @T description -> @string ;
\end{galgascode}

Le \emph{reader} \galgas{description} retourne une description textuelle du receveur, la même que celle affichée par l'instruction \galgas{log} (\refSectionPage{instructionLog}).



\subsection{Le reader \texttt{dynamicType}}

\begin{galgascode}
reader @T dynamicType -> @type ;
\end{galgascode}

Le \emph{reader} \galgas{dynamicType} retourne un objet de type \galgas{@type}, dont la valeur représente le type dynamique du receveur (voir aussi la définition du \refTypePredefini{type}).

Pour tous les types sauf les classes, leurs instances sont du même type que le type statique :

\begin{galgascode}
@uint n := 2 ;
@type t := [n dynamicType] ;
log t ; # Affiche @uint
\end{galgascode}

Pour les instances de classes, le jeu des affectations polymorphiques peut entraîner que le type dynamique soit une classe héritière du type statique.

Par exemple, en déclarant :
\begin{galgascode}
class @A { }
class @B extends @A { }
\end{galgascode}

Et avec la séquence d'instructions suivante :
\begin{galgascode}
@B b [new] ;
@type t := [b dynamicType] ;
log t ; # Affiche @B, type statique de b : @B
@A a := b ; # Affectation polymorphique
t := [a dynamicType] ;
log t ; # Affiche @B, type statique de a : @A
\end{galgascode}





\subsection{Le reader \texttt{object}}

\begin{galgascode}
reader @T object -> @object ;
\end{galgascode}


Le \emph{reader} \galgas{object} retourne un objet de type \galgas{@object}. Une variable de \refTypePredefini{object} peut encapsuler tout type de valeur.

%====== Readers ======
%===== description =====
%
%''**reader** description %%->%% @string ;''\\
%
%This reader returns a string representation of the receiver's value.
%
%===== dynamicType =====
%
%|Available on GALGAS 1.9.5 and later|
%
%''**reader** dynamicType %%->%% @type ;''\\
%
%This reader returns the dynamic type of the receiver's value.
%===== object =====
%
%|Available on GALGAS 1.9.5 and later|
%
%''**reader** object %%->%% @object ;''\\
%
%This reader returns an ''@object'' instance that embeds the receiver's value.











\sectionLabel{Constructeur par défaut}{constructeurParDefaut}

Pour la plupart des types, un constructeur par défaut est implicitement défini (voir la définition précise \refSubsectionPage{constructeurParDefautPourChaqueType}). Celui-ci est invoqué par le mot réservé \galgas{default}.

Le constructeur par défaut peut être utilisé dans deux constructions :
\begin{itemize}
  \item la déclaration d'une variable ou d'une constante ;
  \item dans une expression.
\end{itemize}

\subsection{Intérêt du constructeur par défaut}


L'intérêt du constructeur par défaut est qu'il allège l'écriture de l'initialisation des variables de certains types. Ce n'est pas une construction qui apporte de l'expressivité au langage (on peut très bien se passer d'appeler des constructeurs par défaut).

Pour un type comme \galgas{@uint}, écrire \galgas{@uint v [default] ;} est sémantiquement équivalent à écrire \galgas{@uint v := 0 ;}. On voit que le constructeur par défaut présente peu d'utilité ici.

Par contre, si l'on a un type structure :

\begin{galgascode}
struct @T {
  @uneMap mMap ;
  @uneListe mList ;
  @stringlist mStringList ;
  @stringset mStringSet ;
}
\end{galgascode}

Déclarer et initialiser une variable de ce type s'écrit :

\begin{galgascode}
@T variable [new
  ![@uneMap emptyMap]
  ![@uneListe emptyList]
  ![@stringlist emptyList]
  ![@stringset emptySet]
] ;
\end{galgascode}

Avec le constructeur par défaut, cette instruction s'écrit simplement :

\begin{galgascode}
@T variable [default] ;
\end{galgascode}

Pour une structure, comme on va le voir plus bas, le constructeur par défaut appelle le constructeur par défaut pour chaque champ ; le constructeur par défaut d'une \galgas{map} est équivalent à \galgas{emptyMap}, celui d'une \galgas{list}  équivalent à \galgas{emptyList}, et celui d'un \galgas{@stringset}  équivalent à \galgas{emptySet}.


\subsection{Appel dans la déclaration d'une variable ou d'une constante}

\begin{galgascode}
@T variable [default] ;
\end{galgascode}

Ceci déclare une variable de type \galgas{@T} et l'initialise avec le constructeur par défaut. Pour une constante, la syntaxe est :

\begin{galgascode}
const @T constante [default] ;
\end{galgascode}


\subsection{Appel dans une expression}

L'expression \galgas{[@T default]} invoque le constructeur par défaut du type \galgas{@T} et renvoie un objet initialisé du type \galgas{@T}.

\subsectionLabel{Les constructeurs par défaut pour chaque type}{constructeurParDefautPourChaqueType}

Le \refTableau{constructeurParDefaut} précise par chaque type l'existence du constructeur par défaut.


\begin{table}[t]
  \centering
%  \rowcolors{2}{\fondTableau}{}
  \begin{tabular}{@{}lllllll@{}}
  \textbf{Type} & \textbf{Constructeur par défaut} \\
  \galgas{abstract class @T} & \emph{Pas de constructeur par défaut} \\
  \galgas{@bool} & Initialisation à \galgas{false} \\
  \galgas{@application} & \emph{Pas de constructeur par défaut} \\
  \galgas{array @T} & \emph{Pas de constructeur par défaut} \\
  \galgas{@char} & Initialisation au caractère \texttt{NULL} \\
  \galgas{class @T} & Oui, si tous les attributs possèdent un constructeur par défaut \\
  \galgas{@data} & Équivalent au constructeur \galgas{emptyData} \\
  \galgas{@double} & Initialisation à \texttt{0.0} \\
  \galgas{@filewrapper} & \emph{Pas de constructeur par défaut} \\
  \galgas{@function} & \emph{Pas de constructeur par défaut} \\
  \galgas{graph @T} & Équivalent au constructeur \galgas{emptyGraph} \\
  \galgas{list @T} & Équivalent au constructeur \galgas{emptyList} \\
  \galgas{map @T} & Équivalent au constructeur \galgas{emptyMap} \\
  \galgas{listmap @T} & Équivalent au constructeur \galgas{emptyMap} \\
  \galgas{@object} & \emph{Pas de constructeur par défaut} \\
  \galgas{@sint} & Initialisation à \galgas{0S} \\
  \galgas{@sint64} & Initialisation à \galgas{0LS} \\
  \galgas{sortedlist @T} & Équivalent au constructeur \galgas{emptySortedList} \\
  \galgas{@string} & Initialisation à chaîne vide \galgas{""} \\
  \galgas{@stringset} & Équivalent au constructeur \galgas{emptySet} \\
  \galgas{struct @T} & Oui, si tous les attributs possèdent un constructeur par défaut \\
  \galgas{@type} & \emph{Pas de constructeur par défaut} \\
  \galgas{@uint} & Initialisation à \galgas{0} \\
  \galgas{@uint64} & Initialisation à \galgas{0L} \\
  \end{tabular}
  \caption{Constructeur par défaut pour chaque type}
  \labelTableau{constructeurParDefaut}
  \ligne
\end{table}

Remarques :
\begin{itemize}
  \item une classe abstraite ne possède pas de constructeur par défaut ;
  \item une classe concrète possède un constructeur par défaut si tous les attributs (ceux déclarés dans la classe et les super classes) en possèdent un ; la valeur par défaut est celle définie par l'appel du constructeur par défaut sur tous ces attributs ;
  \item une structure possède un constructeur par défaut si tous ces champs en possèdent un ; la valeur par défaut est celle définie par l'appel du constructeur par défaut sur tous les champs.
\end{itemize}


  %!TEX encoding = UTF-8 Unicode
%!TEX root = ../galgas-book.tex

\chapitreTypePredefiniLabelIndex{application}

\tableDesMatieresLocaleDeProfondeurRelative{1}


Le type \ggst+@application+ ne définit que des constructeurs et des procédures de type qui permettent d'obtenir des informations sur le programme courant et son exécution.


\section{Numéros de version}

\subsectionConstructor{galgasVersionString}{application}

\begin{galgas3box}
constructor @application galgasVersionString -> @string
\end{galgas3box}

Ce constructeur renvoie la version du compilateur GALGAS qui a engendré cet exécutable. Pour le compilateur correspondant à cette documentation, la chaîne renvoyée est \ggst+"GALGASBETAVERSION"+ :

\begin{galgas3}
let s = @application.galgasVersionString # "GALGASBETAVERSION"
\end{galgas3}








\subsectionConstructor{projectVersionString}{application}

\begin{galgas3box}
constructor @application projectVersionString -> @string
\end{galgas3box}

Ce constructeur renvoie la version du projet GALGAS dont la compilation fournit cet exécutable. C'est l'information qui apparaît après le mot réservé \ggst+project+ (voir \refSubsectionPage{versionProjet}), en utilisant le point « . » comme séparateur. Par exemple, si l'en-tête du projet est :

\begin{galgas3}
project (1:2:3) -> "logo" {
  ...
}
\end{galgas3}

La chaîne renvoyée est \ggst+"1.2.3"+:
\begin{galgas3}
let s = @application.projectVersionString # "1.2.3"
\end{galgas3}










\section{Arguments de la ligne de commande}


\subsectionConstructor{commandLineArgumentCount}{application}

\begin{galgas3box}
constructor @application commandLineArgumentCount -> @uint
\end{galgas3box}

Ce constructeur renvoie le nombre d'arguments de la ligne de commande.


\subsectionConstructor{commandLineArgumentAtIndex}{application}

\begin{galgas3box}
constructor @application commandLineArgumentAtIndex ?@uint inIndex
  -> @string
\end{galgas3box}

Ce constructeur renvoie l'argument d'indice \ggst+inIndex+ de la ligne de commande. Les arguments sont indexés à partir de zéro, aussi la valeur de \ggst+inIndex+ doit être strictement inférieur à la valeur retournée par \ggst+@application.commandLineArgumentCount+. Une erreur d'exécution est déclenchée dans le cas contraire.

À titre d'exemple, voici comment imprimer tous les arguments de la ligne de commande~:
\begin{galgas3}
for idx in 0 ..< @application.commandLineArgumentCount do
  message "Argument " + idx + ": '"
    + @application.commandLineArgumentAtIndex {!idx} + "'\n"
end
\end{galgas3}










\section{Options booléennes de la ligne de commande}

\subsectionConstructor{boolOptionNameList}{application}

\begin{galgas3box}
constructor @application boolOptionNameList -> @2stringlist
\end{galgas3box}

Ce constructeur renvoie la liste des options booléennes définie par l'application, que ces options soient nommées dans la ligne de commande ou non. Chaque option est définie par un couple, son nom de domaine et son identificateur. À titre d'exemple, voici comment imprimer la liste des options booléennes :
\begin{galgas3}
for (domain identifier) in @application.boolOptionNameList do
  message "Domain: '" + domain + "', identifier: '" + identifier + "'\n"
end
\end{galgas3}


\subsectionConstructor{boolOptionCommentString}{application}

\begin{galgas3box}
constructor @application boolOptionCommentString
    ?@string inDomainName
    ?@string inOptionIdentifier -> @string
\end{galgas3box}

Ce constructeur renvoie la chaîne de commentaires associée à l'option booléenne spécifiée par son nom de domaine et son identificateur. Par exemple :
\begin{galgas3}
for (domain identifier) in @application.boolOptionNameList do
  message "Domain: '" + domain + "', identifier: '" + identifier + "'\n"
  message "Comment: '"
    + @application.boolOptionCommentString {!domain !identifier} + "'\n"
end
\end{galgas3}

Une erreur d'exécution est déclenchée si l'option n'existe pas, et la chaîne renvoyée n'est pas construite.


\subsectionConstructor{boolOptionInvocationCharacter}{application}

\begin{galgas3box}
constructor @application boolOptionInvocationCharacter
    ?@string inDomainName
    ?@string inOptionIdentifier -> @char
\end{galgas3box}

Ce constructeur renvoie le caractère d'activation associé à l'option booléenne spécifiée par son nom de domaine et son identificateur.

Le caractère d'activation est le caractère qui, précédé de « \texttt{-} » permet l'activation de l'option sur la ligne de commande. Si l'option ne définit pas de caractère d'activation, la valeur renvoyée est \texttt{NUL}.

 Par exemple :
\begin{galgas3}
for (domain identifier) in @application.boolOptionNameList do
  message "Domain: '" + domain + "', identifier: '" + identifier + "'\n"
  message "Invocation character: '"
    + @application.boolOptionInvocationCharacter {!domain !identifier} + "'\n"
end
\end{galgas3}

Une erreur d'exécution est déclenchée si l'option n'existe pas, et le caractère renvoyé n'est pas construit.


\subsectionConstructor{boolOptionInvocationString}{application}

\begin{galgas3box}
constructor @application boolOptionInvocationString
    ?@string inDomainName
    ?@string inOptionIdentifier -> @string
\end{galgas3box}

Ce constructeur renvoie la chaîne d'activation associée à l'option booléenne spécifiée par son nom de domaine et son identificateur.

La chaîne d'activation est la chaîne qui, précédée de « \texttt{-{}-} » permet l'activation de l'option sur la ligne de commande. Si l'option ne définit pas de chaîne d'activation, la valeur renvoyée est la chaîne vide.

 Par exemple :
\begin{galgas3}
for (domain identifier) in @application.boolOptionNameList do
  message "Domain: '" + domain + "', identifier: '" + identifier + "'\n"
  message "Invocation string: '"
    + @application.boolOptionInvocationString {!domain !identifier} + "'\n"
end
\end{galgas3}

Une erreur d'exécution est déclenchée si l'option n'existe pas, et la chaîne renvoyée n'est pas construite.






\subsectionConstructor{boolOptionValue}{application}

\begin{galgas3box}
constructor @application boolOptionValue
    ?@string inDomainName
    ?@string inOptionIdentifier -> @bool
\end{galgas3box}

Ce constructeur renvoie la valeur associée à l'option booléenne spécifiée par son nom de domaine et son identificateur. Si l'option n'existe pas, le résultat n'est pas construit.




\subsectionStaticProc{setBoolOptionValue}{application}


\begin{galgas3box}
proc @application setBoolOptionValue
    ?@string inDomainName
    ?@string inOptionIdentifier
    ?@bool inValue
\end{galgas3box}

Ce procédure de type affecte la valeur de \ggst=inValue= à l'option booléenne spécifiée par son nom de domaine et son identificateur. Si l'option n'existe pas, cette fonction est sans effet.







\section{Options entières de la ligne de commande}

\subsectionConstructor{uintOptionNameList}{application}

\begin{galgas3box}
constructor @application uintOptionNameList -> @2stringlist
\end{galgas3box}

Ce constructeur renvoie la liste des options entières définie par l'application, que ces options soient nommées dans la ligne de commande ou non. Chaque option est définie par un couple, son nom de domaine et son identificateur. Par d'exemple :
\begin{galgas3}
for (domain identifier) in @application.uintOptionNameList do
  message "Domain: '" + domain + "', identifier: '" + identifier + "'\n"
end
\end{galgas3}


\subsectionConstructor{uintOptionCommentString}{application}

\begin{galgas3box}
constructor @application uintOptionCommentString
    ?@string inDomainName
    ?@string inOptionIdentifier -> @string
\end{galgas3box}

Ce constructeur renvoie la chaîne de commentaires associée à l'option entière spécifiée par son nom de domaine et son identificateur. Par exemple :
\begin{galgas3}
for (domain identifier) in @application.uintOptionNameList do
  message "Domain: '" + domain + "', identifier: '" + identifier + "'\n"
  message "Comment: '"
    + @application.uintOptionCommentString {!domain !identifier} + "'\n"
end
\end{galgas3}

Une erreur d'exécution est déclenchée si l'option n'existe pas, et la chaîne renvoyée n'est pas construite.


\subsectionConstructor{uintOptionInvocationCharacter}{application}

\begin{galgas3box}
constructor @application uintOptionInvocationCharacter
    ?@string inDomainName
    ?@string inOptionIdentifier -> @char
\end{galgas3box}

Ce constructeur renvoie le caractère d'activation associé à l'option entière spécifiée par son nom de domaine et son identificateur.

Le caractère d'activation est le caractère qui, précédé de « \texttt{-} » permet l'activation de l'option sur la ligne de commande. Si l'option ne définit pas de caractère d'activation, la valeur renvoyée est \texttt{NUL}.

 Par exemple :
\begin{galgas3}
for (domain identifier) in @application.uintOptionNameList do
  message "Domain: '" + domain + "', identifier: '" + identifier + "'\n"
  message "Invocation character: '"
    + @application.uintOptionInvocationCharacter {!domain !identifier} + "'\n"
end
\end{galgas3}

Une erreur d'exécution est déclenchée si l'option n'existe pas, et le caractère renvoyé n'est pas construit.


\subsectionConstructor{uintOptionInvocationString}{application}

\begin{galgas3box}
constructor @application uintOptionInvocationString
    ?@string inDomainName
    ?@string inOptionIdentifier -> @string
\end{galgas3box}

Ce constructeur renvoie la chaîne d'activation associée à l'option entière spécifiée par son nom de domaine et son identificateur.

La chaîne d'activation est la chaîne qui, précédée de « \texttt{-{}-} » permet l'activation de l'option sur la ligne de commande. Si l'option ne définit pas de chaîne d'activation, la valeur renvoyée est la chaîne vide.

 Par exemple :
\begin{galgas3}
for (domain identifier) in @application.uintOptionNameList do
  message "Domain: '" + domain + "', identifier: '" + identifier + "'\n"
  message "Invocation string: '"
    + @application.uintOptionInvocationString {!domain !identifier} + "'\n"
end
\end{galgas3}

Une erreur d'exécution est déclenchée si l'option n'existe pas, et la chaîne renvoyée n'est pas construite.


\subsectionConstructor{uintOptionValue}{application}

\begin{galgas3box}
constructor @application uintOptionValue
    ?@string inDomainName
    ?@string inOptionIdentifier -> @uint
\end{galgas3box}

Ce constructeur renvoie la valeur associée à l'option entière spécifiée par son nom de domaine et son identificateur. Si l'option n'existe pas, le résultat n'est pas construit.




\subsectionStaticProc{setUIntOptionValue}{application}


\begin{galgas3box}
proc @application setUIntOptionValue
    ?@string inDomainName
    ?@string inOptionIdentifier
    ?@uint inValue
\end{galgas3box}

Ce procédure de type affecte la valeur de \ggst=inValue= à l'option entière spécifiée par son nom de domaine et son identificateur. Si l'option n'existe pas, cette fonction est sans effet.

















\section{Options chaînes de caractères de la ligne de commande}

\subsectionConstructor{stringOptionNameList}{application}

\begin{galgas3box}
constructor @application stringOptionNameList -> @2stringlist
\end{galgas3box}

Ce constructeur renvoie la liste des options chaînes de caractères définie par l'application, que ces options soient nommées dans la ligne de commande ou non. Chaque option est définie par un couple, son nom de domaine et son identificateur. Par d'exemple :
\begin{galgas3}
for (domain identifier) in @application.stringOptionNameList do
  message "Domain: '" + domain + "', identifier: '" + identifier + "'\n"
end
\end{galgas3}


\subsectionConstructor{stringOptionCommentString}{application}

\begin{galgas3box}
constructor @application stringOptionCommentString
    ?@string inDomainName
    ?@string inOptionIdentifier -> @string
\end{galgas3box}

Ce constructeur renvoie la chaîne de commentaires associée à l'option chaîne de caractères spécifiée par son nom de domaine et son identificateur. Par exemple :
\begin{galgas3}
for (domain identifier) in @application.stringOptionNameList do
  message "Domain: '" + domain + "', identifier: '" + identifier + "'\n"
  message "Comment: '"
    + @application.stringOptionCommentString {!domain !identifier} + "'\n"
end
\end{galgas3}

Une erreur d'exécution est déclenchée si l'option n'existe pas, et la chaîne renvoyée n'est pas construite.


\subsectionConstructor{stringOptionInvocationCharacter}{application}

\begin{galgas3box}
constructor @application stringOptionInvocationCharacter
    ?@string inDomainName
    ?@string inOptionIdentifier -> @char
\end{galgas3box}

Ce constructeur renvoie le caractère d'activation associé à l'option chaîne de caractères spécifiée par son nom de domaine et son identificateur.

Le caractère d'activation est le caractère qui, précédé de « \texttt{-} » permet l'activation de l'option sur la ligne de commande. Si l'option ne définit pas de caractère d'activation, la valeur renvoyée est \texttt{NUL}.

 Par exemple :
\begin{galgas3}
for (domain identifier) in @application.stringOptionNameList do
  message "Domain: '" + domain + "', identifier: '" + identifier + "'\n"
  message "Invocation character: '"
    + @application.uintOptionInvocationCharacter {!domain !identifier} + "'\n"
end
\end{galgas3}

Une erreur d'exécution est déclenchée si l'option n'existe pas, et le caractère renvoyé n'est pas construit.


\subsectionConstructor{stringOptionInvocationString}{application}

\begin{galgas3box}
constructor @application stringOptionInvocationString
    ?@string inDomainName
    ?@string inOptionIdentifier -> @string
\end{galgas3box}

Ce constructeur renvoie la chaîne d'activation associée à l'option chaîne de caractères spécifiée par son nom de domaine et son identificateur.

La chaîne d'activation est la chaîne qui, précédée de « \texttt{-{}-} » permet l'activation de l'option sur la ligne de commande. Si l'option ne définit pas de chaîne d'activation, la valeur renvoyée est la chaîne vide.

 Par exemple :
\begin{galgas3}
for (domain identifier) in @application.stringOptionNameList do
  message "Domain: '" + domain + "', identifier: '" + identifier + "'\n"
  message "Invocation string: '"
    + @application.stringOptionInvocationString {!domain !identifier} + "'\n"
end
\end{galgas3}

Une erreur d'exécution est déclenchée si l'option n'existe pas, et la chaîne renvoyée n'est pas construite.




\subsectionConstructor{stringOptionValue}{application}

\begin{galgas3box}
constructor @application stringOptionValue
    ?@string inDomainName
    ?@string inOptionIdentifier -> @string
\end{galgas3box}

Ce constructeur renvoie la valeur associée à l'option chaîne de caractères spécifiée par son nom de domaine et son identificateur. Si l'option n'existe pas, le résultat n'est pas construit.




\subsectionStaticProc{setStringOptionValue}{application}


\begin{galgas3box}
proc @application setStringOptionValue
    ?@string inDomainName
    ?@string inOptionIdentifier
    ?@string inValue
\end{galgas3box}

Cette procédure de type affecte la valeur de \ggst=inValue= à l'option chaîne de caractères spécifiée par son nom de domaine et son identificateur. Si l'option n'existe pas, cette fonction est sans effet.











\sectionConstructor{system}{application}

\begin{galgas3box}
constructor @application system -> @string
\end{galgas3box}

Ce constructeur permet de savoir sur quel type de système l'application tourne en renvoyant la chaîne :
\begin{itemize}
  \item \ggst!"unix"! sur Unix, par exemple OSX ou Linux ;
  \item \ggst!"windows"! sur Windows.
\end{itemize}




\sectionStaticProc{exit}{application}


\begin{galgas3box}
proc @application exit ?@uint inErrorCode
\end{galgas3box}

L'exécution de cette procédure de type avorte immédiatement l'exécution (la fonction C \texttt{exit} est appelée). L'argument est le code d'erreur associé. Si il n'est pas construit, la valeur 1 est utilisée.









\sectionConstructor{verboseOutput}{application}

\begin{galgas3box}
constructor @application verboseOutput -> @bool
\end{galgas3box}

Ce constructeur permet de savoir si l'indicateur de sortie verbeuse est activé ou non.

La sortie verbeuse est controllée par les options de la ligne de commande \emph{quiet} et \emph{verbose} (\refSubsectionPage{optionsQuietVerbose}) ; leur présence dans le compilateur engendré dépend de la présence de la déclaration \ggst!%quietOutputByDefault!\index{\%quietOutputByDefault} parmi les déclarations du fichier projet (\refSectionPage{projetDeclarationQuietOutputByDefault}).

Les deux options de la ligne de commande \emph{quiet} et \emph{verbose} s'excluent et ne peuvent pas être appelées par la construction \ggst+[option nom_composant_option.nom_option nom_info]+ (voir \refSubsectionPage{appelOption}) : c'est ce constructeur, qui s'adapte à la configuration du compilateur, qu'il faut appeler.

Par exemple :
\begin{galgas3}
if @application.verboseOutput then
  # impressions de la sortie verbeuse
end
\end{galgas3}














\section{Instrospection des composants lexique}

\subsectionConstructor{keywordIdentifierSet}{application}

\begin{galgas3box}
constructor @application keywordIdentifierSet -> @stringset
\end{galgas3box}


Ce constructeur renvoie l'ensemble des identificateurs des listes de mots réservés définies dans les composants lexiques du projet. Un identificateur est composé du nom du lexique, suivi de «~\texttt{:}~», et du nom de la liste des mots réservés.


Si par exemple un projet définit le composant lexique suivant :

\begin{galgas3}
lexique monLexique {
  ...
  list mots1 ... { ... }
  ...
  list mots2 ... { ... }
  ...
}
\end{galgas3}

Alors :
\begin{galgas3}
let theList = @application.keywordIdentifierSet
log theList # "monLexique:mots1", "monLexique:mots2"
\end{galgas3}





\subsectionConstructor{keywordListForIdentifier}{application}

\begin{galgas3box}
constructor @application keywordListForIdentifier
  ?@string inIdentifier
  -> @stringlist
\end{galgas3box}


Ce constructeur renvoie le contenu de la liste désignée par \ggst!inIdentifier!. Si \ggst!inIdentifier! n'est pas une des valeurs renvoyées par le \refConstructorPage{application}{keywordIdentifierSet}, la liste retournée est vide.


Si par exemple un projet définit le composant lexique suivant :

\begin{galgas3}
lexique monLexique {
  ...
  list mots ... { "a", "b", "c" }
  ...
}
\end{galgas3}

Alors :
\begin{galgas3}
let theList = @application.keywordListForIdentifier {!"monLexique:mots"}
log theList # "a", "b", "c"
\end{galgas3}



  %!TEX encoding = UTF-8 Unicode
%!TEX root = ../galgas-book.tex

\chapitreTypePredefiniLabelIndex{bigint}

Le \ggs+@bigint+ définit les entiers signés d'une taille quelconque, seulement limitée par la mémoire disponible. Ce type est simplement une interface des entiers de la librairie GMP\footnote{\url{http://www.gmplib.org}.}.

\section{Constante littérale}

Utiliser le suffixe « \texttt{G} » pour définir une constante littérale de type \ggs!@bigint! :
\begin{galgas}
@bigint a = 1234567890_1234567890_1234567890_G
message [a string] + "\n" # 123456789012345678901234567890
\end{galgas}

Vous pouvez utiliser le caractère de soulignement « \texttt{\_} » pour séparer les chiffres.

Avec le préfixe « \texttt{0x} », vous pouvez écrire les nombres en héxadécimal :
\begin{galgas}
@bigint a = 0x123456789ABCDEF0_123456789abcdefG
message [a hexString] + "\n" # 0x123456789ABCDEF0_123456789ABCDEF
\end{galgas}

Les lettres minuscules et majuscules sont utilisables.

\section{Constructeurs}

\subsectionConstructor{zero}{bigint}

Le constructeur \ggs!zero! renvoie un \ggs!@bigint! initialisé à zéro :
\begin{galgas}
@bigint a = .zero
message [a string] + "\n" # 0
\end{galgas}


\subsectionConstructor{default}{bigint}


Le constructeur \ggs!default!, comme le constructeur \ggs!zero!, renvoie un \ggs!@bigint! initialisé à zéro :
\begin{galgas}
@bigint a = .default
message [a string] + "\n" # 0
\end{galgas}

\subsectionConstructor{sint}{bigint}

Le constructeur \ggs!sint! permet de construire un \ggs!@bigint! à partir d'une valeur de type \ggs!@sint! :
\begin{galgas}
@bigint a = .sint {!-678S}
message [a string] + "\n" # -678
\end{galgas}


\subsectionConstructor{sint64}{bigint}

Le constructeur \ggs!sint64! permet de construire un \ggs!@bigint! à partir d'une valeur de type \ggs!@sint64! :
\begin{galgas}
@bigint a = .sint64 {!-678LS}
message [a string] + "\n" # -678
\end{galgas}




\subsectionConstructor{uint}{bigint}

Le constructeur \ggs!uint! permet de construire un \ggs!@bigint! à partir d'une valeur de type \ggs!@uint! :
\begin{galgas}
@bigint a = .uint {!678}
message [a string] + "\n" # 678
\end{galgas}




\subsectionConstructor{uint64}{bigint}

Le constructeur \ggs!uint64! permet de construire un \ggs!@bigint! à partir d'une valeur de type \ggs!@uint64! :
\begin{galgas}
@bigint a = .uint64 {!678L}
message [a string] + "\n" # 678
\end{galgas}










\section{Comparaison}

Le type \ggs!@bigint! implémente les six opérateurs de comparaison \ggs!==!, \ggs+!=+, \ggs!<!, \ggs!<=!, \ggs!>! et \ggs!>=!. Il  implémente aussi les \emph{getters} \ggs!isZero! et \ggs!sign! qui permettent de comparer un \ggs!@bigint! avec zéro.


\subsectionGetter{isZero}{bigint}

Ce \emph{getter} renvoie \ggs!true! si valeur du récepteur est nulle, et \ggs!false! dans le cas contraire.

\begin{galgas}
message [[0G isZero] ocString] + "\n" # YES
message [[1G isZero] ocString] + "\n" # NO
\end{galgas}



\subsectionGetter{sign}{bigint}

Ce \emph{getter} renvoie :
\begin{itemize}
\item \ggs!-1S! si la valeur du récepteur est strictement négative ;
\item \ggs!0S! si la valeur du récepteur est nulle ;
\item \ggs!1S! si la valeur du récepteur est strictement positive.
\end{itemize}

\begin{galgas}
message [[0G sign] >= 0S ocString] + "\n" # YES
message [[1G sign] < 0S ocString] + "\n" # NO
\end{galgas}











\section{Conversions}

\subsectionGetter{bitCountForSignedRepresentation}{uint}

Ce \emph{getter} permet de connaître le nombre de bits nécessaires pour écrire la valeur du récepteur dans la représentation binaire \emph{complément à deux}.

\begin{galgas}
message [[0G bitCountForSignedRepresentation] string] + "\n" # 1
message [[1G bitCountForSignedRepresentation] string] + "\n" # 2
message [[-1G bitCountForSignedRepresentation] string] + "\n" # 1
message [[0x8000G bitCountForSignedRepresentation] string] + "\n" # 17
message [[-0x8000G bitCountForSignedRepresentation] string] + "\n" # 16
\end{galgas}


Pour connaître le nombre d'octets nécessaires pour représenter la valeur du récepteur dans la représentation binaire \emph{complément à deux}, on calcule :
\begin{galgas}
([bigint bitCountForSignedRepresentation] - 1) / 8 + 1 
\end{galgas}

Et pour le nombre de mots de 32 bits :
\begin{galgas}
([bigint bitCountForSignedRepresentation] - 1) / 32 + 1 
\end{galgas}



\subsectionGetter{bitCountForUnsignedRepresentation}{uint}

Ce \emph{getter} permet de connaître le nombre de bits nécessaires pour écrire la valeur absolue du récepteur dans la représentation binaire \emph{naturelle}. 

\begin{galgas}
message [[0G bitCountForUnsignedRepresentation] string] + "\n" # 1
message [[1G bitCountForUnsignedRepresentation] string] + "\n" # 1
message [[-1G bitCountForUnsignedRepresentation] string] + "\n" # 1
message [[0x8000G bitCountForUnsignedRepresentation] string] + "\n" # 16
message [[-0x8000G bitCountForUnsignedRepresentation] string] + "\n" # 16
\end{galgas}

Comme c'est la valeur absolue qui est prise en compte, le signe n'intervient pas.

Pour connaître le nombre d'octets nécessaires pour représenter la valeur absolue du récepteur dans la représentation binaire \emph{naturelle}, on calcule :
\begin{galgas}
([bigint bitCountForUnsignedRepresentation] - 1) / 8 + 1 
\end{galgas}

Et pour le nombre de mots de 32 bits :
\begin{galgas}
([bigint bitCountForUnsignedRepresentation] - 1) / 32 + 1 
\end{galgas}

\subsectionGetter{fitsInSInt}{bigint}

Ce \emph{getter} permet de savoir si le récepteur peut être converti en \ggs!@sint!.

\begin{galgas}
message [[0x1234_5678G fitsInSInt] ocString] + "\n" # YES
message [[0x7FFF_FFFFG fitsInSInt] ocString] + "\n" # YES
message [[0x8000_0000G fitsInSInt] ocString] + "\n" # NO
message [[-0x8000_0000G fitsInSInt] ocString] + "\n" # YES
message [[-0x8000_0001G fitsInSInt] ocString] + "\n" # NO
\end{galgas}




\subsectionGetter{fitsInSInt64}{bigint}

Ce \emph{getter} permet de savoir si le récepteur peut être converti en \ggs!@sint64!.

\begin{galgas}
message [[0x1234_5678_9ABC_DEF0G fitsInSInt64] ocString] + "\n" # YES
message [[0x7FFF_FFFF_FFFF_FFFFG fitsInSInt64] ocString] + "\n" # YES
message [[0x8000_0000_0000_0000G fitsInSInt64] ocString] + "\n" # NO
message [[-0x8000_0000_0000_0000G fitsInSInt64] ocString] + "\n" # YES
message [[-0x8000_0000_0000_0001G fitsInSInt64] ocString] + "\n" # NO
\end{galgas}




\subsectionGetter{fitsInUInt}{bigint}

Ce \emph{getter} permet de savoir si le récepteur peut être converti en \ggs!@uint!.

\begin{galgas}
message [[0x1234_5678G fitsInUInt] ocString] + "\n" # YES
message [[0x1234_5678_9G fitsInUInt] ocString] + "\n" # NO
message [[-1G fitsInUInt] ocString] + "\n" # NO
\end{galgas}






\subsectionGetter{fitsInUInt64}{bigint}

Ce \emph{getter} permet de savoir si le récepteur peut être converti en \ggs!@uint64!.

\begin{galgas}
message [[0x1234_5678_9ABC_DEF0G fitsInUInt64] ocString] + "\n" # YES
message [[0x1234_5678_9ABC_DEF0_1G fitsInUInt64] ocString] + "\n" # NO
message [[-1G fitsInUInt64] ocString] + "\n" # NO
\end{galgas}


\subsectionGetter{sint}{bigint}

Ce \emph{getter} permet de convertir le récepteur en \ggs!@sint!. Si la conversion n'est pas possible, un message d'erreur est affiché et la valeur renvoyée n'est pas construite. On peut tester si la conversion est possible en appelant le \refGetterPage{bigint}{fitsInSInt}.

\begin{galgas}
message [[-0x1234_5678G sint] hexString] + "\n" # 0xEDCBA988
\end{galgas}




\subsectionGetter{sint64}{bigint}

Ce \emph{getter} permet de convertir le récepteur en \ggs!@sint64!. Si la conversion n'est pas possible, un message d'erreur est affiché et la valeur renvoyée n'est pas construite. On peut tester si la conversion est possible en appelant le \refGetterPage{bigint}{fitsInSInt64}.

\begin{galgas}
message [[-0x1234_5678_9ABC_DEF0G sint64] hexString] + "\n" # 0xEDCBA98765432110
\end{galgas}


\subsectionGetter{uint}{bigint}

Ce \emph{getter} permet de convertir le récepteur en \ggs!@uint!. Si la conversion n'est pas possible, un message d'erreur est affiché et la valeur renvoyée n'est pas construite. On peut tester si la conversion est possible en appelant le \refGetterPage{bigint}{fitsInUInt}.

\begin{galgas}
message [[0x1234_5678G uint] hexString] + "\n" # 0x12345678
\end{galgas}


\subsectionGetter{uint64}{bigint}

Ce \emph{getter} permet de convertir le récepteur en \ggs!@uint64!. Si la conversion n'est pas possible, un message d'erreur est affiché et la valeur renvoyée n'est pas construite. On peut tester si la conversion est possible en appelant le \refGetterPage{bigint}{fitsInUInt64}.

\begin{galgas}
message [[0x1234_5678_9ABC_DEFG uint64] hexString] + "\n" # 0x123456789ABCDEF
\end{galgas}







\section{Conversions en chaîne de caractères}

\subsectionGetter{string}{bigint}

Ce getter renvoie la valeur du récepteur sous la forme d'une chaîne de caractères décimaux (de \ggs!0! à \ggs!9!). Si cette valeur est négative, le premier caractère est un signe \ggs!-!. Par exemple :

\begin{galgas}
@bigint a = -1234567890_1234567890_1234567890_G
message [a string] + "\n" # -123456789012345678901234567890
\end{galgas}





\subsectionGetter{hexString}{bigint}

Ce getter renvoie la valeur du récepteur sous la forme d'une chaîne de caractères héxadécimaux (\ggs!0! à \ggs!9!, \ggs!A! à \ggs!F!). La valeur retournée est préfixée par « \texttt{0x} », qui est placé après un éventuel signe « \texttt{-} ». Exemple :

\begin{galgas}
@bigint a = -1234567890_1234567890_1234567890_G
message [a hexString] + "\n" # -0x18EE90FF6C373E0EE4E3F0AD2
\end{galgas}








\subsectionGetter{xString}{bigint}

Ce getter renvoie la valeur du récepteur sous la forme d'une chaîne de caractères héxadécimaux (\ggs!0! à \ggs!9!, \ggs!A! à \ggs!F!). Si cette valeur est négative, le premier caractère est un signe \ggs!-!. Il n'y a pas de préfixe « \texttt{0x} ». Exemple :

\begin{galgas}
@bigint a = -1234567890_1234567890_1234567890_G
message [a xString] + "\n" # -18EE90FF6C373E0EE4E3F0AD2
\end{galgas}









\section{Extraction}

Six \emph{getters} d'extraction sont définis. Ils permettent d'obtenir la valeur d'un \ggs!@bigint! sous la forme d'un \ggs!@uintlist! ou d'un \ggs!@uint64list!. Ces getters sont :
\begin{itemize}
  \item \refGetterPage{bigint}{extract8ForUnsignedRepresentation} ;
  \item \refGetterPage{bigint}{extract8ForSignedRepresentation} ;
  \item \refGetterPage{bigint}{extract32ForUnsignedRepresentation} ;
  \item \refGetterPage{bigint}{extract32ForSignedRepresentation} ;
  \item \refGetterPage{bigint}{extract64ForUnsignedRepresentation} ;
  \item \refGetterPage{bigint}{extract64ForSignedRepresentation}.
\end{itemize}

L'extraction permet d'obtenir des mots de $8$, $32$ et $64$ bits. Les \emph{getters} « \texttt{…Unsigned…} » extraient la valeur absolue du nombre, et retournent une représentation \emph{binaire naturelle}. Les \emph{getters} « \texttt{…Signed…} » extraient la valeur du nombre en tenant compte de son signe, et retournent une représentation \emph{complément à deux}.

\subsectionGetter{extract8ForUnsignedRepresentation}{bigint}

Ce \emph{getter} permet d'obtenir la représentation binaire \emph{naturelle} de la valeur absolue du récepteur sous la forme d'un \ggs!@uintlist!, dont la valeur de chaque élément est comprise entre $0$ et $255$. L'octet de poids faible est à l'indice $0$, et l'octet de poids fort au dernier indice. Suivant le sens de parcours de la liste, on peut construire une représentation \emph{little endian} ou \emph{big endian}.

\begin{galgas}
# Parcours dans le sens des indices croissants : little endian
@uintlist a = [0xFF_EEDD_CCBB_AA99_8877_6655_4433_2211G
  extract8ForUnsignedRepresentation
]
var s = ""
for (n) in a
  do s += [n hexString]
  between s += " "
end
message s + "\n" # 0x11 0x22 0x33 0x44 . . . 0xAA 0xBB 0xCC 0xDD 0xEE 0xFF 
# Parcours dans le sens des indices décroissants : big endian
s = ""
for > (n) in a
  do s += [n hexString]
  between s += " "
end
message s + "\n" # 0xFF 0xEE 0xDD 0xCC 0xBB 0xAA . . . 0x44 0x33 0x22 0x11
\end{galgas}

Si le récepteur est nul, le vecteur retourné comprend un seul élément de valeur $0$.

\begin{galgas}
@uintlist a = [0G extract8ForUnsignedRepresentation]
var s = ""
for (n) in a
  do s += [n hexString]
  between s += " "
end
message s + "\n" # 0x0
\end{galgas}

\subsectionGetter{extract8ForSignedRepresentation}{bigint}

Ce \emph{getter} permet d'obtenir la représentation binaire \emph{complément à deux} de la valeur du récepteur sous la forme d'un \ggs!@uintlist!, dont la valeur de chaque élément est comprise entre $0$ et $255$. L'octet de poids faible est à l'indice $0$, et l'octet de poids fort au dernier indice. Suivant le sens de parcours de la liste, on peut construire une représentation \emph{little endian} ou \emph{big endian}.

Si la valeur du récepteur est positive, alors son bit de poids fort est zéro. Ce bit est le bit le plus significatif du dernier élément de la liste renvoyée. Dans l'exemple ci-dessus, c'est la valeur \texttt{0xFF\_EEDD\_...\_2211G} qui est utilisée, comme pour le premier exemple du \emph{getter} \ggs!extract8ForUnsignedRepresentation!. Comme le bit de poids fort de ce nombre est $1$, l'extraction en \emph{signé} retourne un élément de plus que l'extraction en \emph{non signé}, élément dont la valeur est $0$.

\begin{galgas}
# Parcours dans le sens des indices croissants : little endian
@uintlist a = [0xFF_EEDD_CCBB_AA99_8877_6655_4433_2211G
  extract8ForSignedRepresentation
]
var s = ""
for (n) in a
  do s += [n hexString]
  between s += " "
end
message s + "\n" # 0x11 0x22 0x33 0x44 . . . 0xAA 0xBB 0xCC 0xDD 0xEE 0xFF 0x00
# Parcours dans le sens des indices décroissants : big endian
s = ""
for > (n) in a
  do s += [n hexString]
  between s += " "
end
message s + "\n" # 0x00 0xFF 0xEE 0xDD 0xCC 0xBB 0xAA . . . 0x44 0x33 0x22 0x11
\end{galgas}

Un nombre négatif est représenté sous la forme de son complément à deux, son bit de poids fort est toujours un $1$ : 

\begin{galgas}
# Parcours dans le sens des indices croissants : little endian
@uintlist a = [-0x4433_2211G extract8ForSignedRepresentation]
var s = ""
for (n) in a
  do s += [n hexString]
  between s += " "
end
message s + "\n" # 0xEF 0xDD 0xCC 0xBB
# Parcours dans le sens des indices décroissants : big endian
s = ""
for > (n) in a
  do s += [n hexString]
  between s += " "
end
message s + "\n" # 0xBB 0xCC 0xDD 0xEF
\end{galgas}

\subsectionGetter{extract32ForUnsignedRepresentation}{bigint}

Ce \emph{getter} permet d'obtenir la représentation binaire \emph{naturelle} de la valeur absolue du récepteur sous la forme d'un \ggs!@uintlist!. Le mot de poids faible est à l'indice $0$, et le mot de poids fort au dernier indice. Suivant le sens de parcours de la liste, on peut construire une représentation \emph{little endian} ou \emph{big endian}.

\begin{galgas}
# Parcours dans le sens des indices croissants : little endian
@uintlist a = [0xFF_EEDD_CCBB_AA99_8877_6655_4433_2211G
  extract32ForUnsignedRepresentation
]
var s = ""
for (n) in a
  do s += [n hexString]
  between s += " "
end
message s + "\n" # 0x44332211 0x88776655 0xCCBBAA99 0x00FFEEDD 
# Parcours dans le sens des indices décroissants : big endian
s = ""
for > (n) in a
  do s += [n hexString]
  between s += " "
end
message s + "\n" # 0x00FFEEDD 0xCCBBAA99 0x88776655 0x44332211
\end{galgas}

Si le récepteur est nul, le vecteur retourné comprend un seul élément de valeur $0$.

\begin{galgas}
@uintlist a = [0G extract32ForUnsignedRepresentation]
var s = ""
for (n) in a
  do s += [n hexString]
  between s += " "
end
message s + "\n" # 0x0
\end{galgas}




\subsectionGetter{extract32ForSignedRepresentation}{bigint}

Ce \emph{getter} permet d'obtenir la représentation binaire \emph{complément à deux} de la valeur du récepteur sous la forme d'un \ggs!@uintlist!. L'octet de poids faible est à l'indice $0$, et l'octet de poids fort au dernier indice. Suivant le sens de parcours de la liste, on peut construire une représentation \emph{little endian} ou \emph{big endian}.

Si la valeur du récepteur est positive, alors son bit de poids fort est zéro. Ce bit est le bit le plus significatif du dernier élément de la liste renvoyée.

\begin{galgas}
# Parcours dans le sens des indices croissants : little endian
@uintlist a = [0xFF_EEDD_CCBB_AA99_8877_6655_4433_2211G
  extract32ForSignedRepresentation
]
var s = ""
for (n) in a
  do s += [n hexString]
  between s += " "
end
message s + "\n" # 0x44332211 0x88776655 0xCCBBAA99 0x00FFEEDD
# Parcours dans le sens des indices décroissants : big endian
s = ""
for > (n) in a
  do s += [n hexString]
  between s += " "
end
message s + "\n" # 0x00FFEEDD 0xCCBBAA99 0x88776655 0x44332211
\end{galgas}

Un nombre négatif est représenté sous la forme de son complément à deux, son bit de poids fort est toujours un $1$ : 

\begin{galgas}
# Parcours dans le sens des indices croissants : little endian
@uintlist a = [-0x55_4433_2211G extract32ForSignedRepresentation]
var s = ""
for (n) in a
  do s += [n hexString]
  between s += " "
end
message s + "\n" # 0xBBCCDDEF 0xFFFFFFAA
# Parcours dans le sens des indices décroissants : big endian
s = ""
for > (n) in a
  do s += [n hexString]
  between s += " "
end
message s + "\n" # 0xFFFFFFAA 0xBBCCDDEF
\end{galgas}





\subsectionGetter{extract64ForUnsignedRepresentation}{bigint}

Ce \emph{getter} permet d'obtenir la représentation binaire \emph{naturelle} de la valeur absolue du récepteur sous la forme d'un \ggs!@uint64list!. Le mot de poids faible est à l'indice $0$, et le mot de poids fort au dernier indice. Suivant le sens de parcours de la liste, on peut construire une représentation \emph{little endian} ou \emph{big endian}.

\begin{galgas}
# Parcours dans le sens des indices croissants : little endian
@uint64list a = [0xFF_EEDD_CCBB_AA99_8877_6655_4433_2211G
  extract64ForUnsignedRepresentation
]
var s = ""
for (n) in a
  do s += [n hexString]
  between s += " "
end
message s + "\n" # 0x8877665544332211 0xFFEEDDCCBBAA99
# Parcours dans le sens des indices décroissants : big endian
s = ""
for > (n) in a
  do s += [n hexString]
  between s += " "
end
message s + "\n" # 0xFFEEDDCCBBAA99 0x8877665544332211
\end{galgas}

Si le récepteur est nul, le vecteur retourné comprend un seul élément de valeur $0$.

\begin{galgas}
@uint64list a = [0G extract64ForUnsignedRepresentation]
var s = ""
for (n) in a
  do s += [n hexString]
  between s += " "
end
message s + "\n" # 0x0
\end{galgas}




\subsectionGetter{extract64ForSignedRepresentation}{bigint}

Ce \emph{getter} permet d'obtenir la représentation binaire \emph{complément à deux} de la valeur du récepteur sous la forme d'un \ggs!@uintlist!. L'octet de poids faible est à l'indice $0$, et l'octet de poids fort au dernier indice. Suivant le sens de parcours de la liste, on peut construire une représentation \emph{little endian} ou \emph{big endian}.

Si la valeur du récepteur est positive, alors son bit de poids fort est zéro. Ce bit est le bit le plus significatif du dernier élément de la liste renvoyée.

\begin{galgas}
# Parcours dans le sens des indices croissants : little endian
@uint64list a = [0xFF_EEDD_CCBB_AA99_8877_6655_4433_2211G
  extract64ForSignedRepresentation
]
var s = ""
for (n) in a
  do s += [n hexString]
  between s += " "
end
message s + "\n" # 0x8877665544332211 0xFFEEDDCCBBAA99
# Parcours dans le sens des indices décroissants : big endian
s = ""
for > (n) in a
  do s += [n hexString]
  between s += " "
end
message s + "\n" # 0xFFEEDDCCBBAA99 0x8877665544332211
\end{galgas}

Un nombre négatif est représenté sous la forme de son complément à deux, son bit de poids fort est toujours un $1$ : 

\begin{galgas}
# Parcours dans le sens des indices croissants : little endian
@uint64list a = [-0x55_4433_2211G extract64ForSignedRepresentation]
var s = ""
for (n) in a
  do s += [n hexString]
  between s += " "
end
message s + "\n" # 0xFFFFFFAABBCCDDEF
# Parcours dans le sens des indices décroissants : big endian
s = ""
for > (n) in a
  do s += [n hexString]
  between s += " "
end
message s + "\n" # 0xFFFFFFAABBCCDDEF
\end{galgas}














\section{Arithmétique}


\subsection{Opérateurs \texttt{+} et \texttt{-} préfixés}

L'opérateur « \texttt{-} » préfixé effectue la négation de l'expression qui le suit. L'opérateur « \texttt{+} » préfixé n'a aucun effet, il retourne la valeur de l'expression.

\begin{galgas}
@bigint a = +1234567890_1234567890_1234567890_G
message [a string] + "\n" # 123456789012345678901234567890
\end{galgas}









\subsectionGetter{abs}{bigint}

Le \emph{getter} \ggs!abs! retourne la valeur absolue.

\begin{galgas}
@bigint a = [-1234567890_1234567890_1234567890_G abs]
message [a string] + "\n" # 123456789012345678901234567890
\end{galgas}






\subsection{Addition et soustraction}

Les opérateurs « \texttt{+} » et « \texttt{-} » infixés effectuent respectivement la somme et la différence de leurs opérandes. Comme la taille des \ggs!@bigint! est non limitée, aucun débordement n'a lieu.


\subsection{Incrémentation et décrémentation}

Le type \ggs!@bigint! accepte les opérateurs d'incrémentation \ggs!++! et de décrémentation \ggs!--!. Aucun débordement n'a lieu.

\subsection{Multiplication}

L'opérateur \ggs!*! infixé effectue le produit de ses opérandes. Comme la taille des \ggs!@bigint! est non limitée, aucun débordement n'a lieu.




\section{Division}



La division d'un entier $n$ par un diviseur $d$ retourne un quotient $q$ et un reste $r$ :
\begin{equation*}
n = q * d + r\text{, avec 0 } \leqslant \mid r\mid < \mid d\mid
\end{equation*}

Trois opérations différentes sont possibles, suivant que l'on veuille obtenir un quotient arrondi :
\begin{itemize}
\item \emph{vers $+\infty$}, et $r$ a un signe opposé à $d$ ;
\item \emph{vers $-\infty$}, et $r$ a le même signe que $d$ ;
\item \emph{vers zéro}, et $r$ a le même signe que $n$.
\end{itemize}

En C, les opérateurs de division (« \texttt{/} »), et de calcul du reste (« \texttt{\%} ») utilisent un quotient arrondi \emph{vers zéro}. L'opérateur de décalage à droite (« \texttt{>{}>} ») de $n$ bits renvoie le quotient arrondi vers \emph{vers $-\infty$} de la division par $2^n$. En GALGAS, les opérateurs correspondants sur les types \ggs!@uint!, \ggs!@sint!, \ggs!@uint64! et \ggs!@sint64! sont conformes à ce comportement.

Le type \ggs!@bigint! obéit aux mêmes règles :
\begin{itemize}
\item les opérateurs \ggs!/! et \ggs!mod! infixés effectuent la division qui calcule le quotient arrondi \emph{vers zéro} ;
  \item l'opérateur \ggs!>>! infixé calcule le quotient arrondi \emph{vers $-\infty$} de la division par $2^n$ ;
\end{itemize}

De plus, trois méthodes sont disponibles, qui retournent quotient et reste de la division :
\begin{itemize}
  \item la méthode \ggs!divideBy! retourne le le quotient arrondi \emph{vers zéro} et le reste correspondant ;
  \item la méthode \ggs!floorDivideBy! retourne le le quotient arrondi \emph{vers $-\infty$} et le reste correspondant ;
  \item la méthode \ggs!ceilDivideBy! retourne le le quotient arrondi \emph{vers $+\infty$} et le reste correspondant.
\end{itemize}


\subsection{Opérateur « \texttt{/} » infixé}
Il effectue la division entière de l'expression de gauche par l'expression de droite et renvoie le quotient. Si l'expression de gauche est nulle, alors un message d'erreur est affiché et le résultat n'est pas construit.

\begin{galgas}
  message [(-7S) / 2S string] + "\n" # -3
  message [(-7G) / 2G string] + "\n" # -3
  message [(-7S) / (-2S) string] + "\n" # 3
  message [(-7G) / (-2G) string] + "\n" # 3
  message [7S / (-2S) string] + "\n" # -3
  message [7G / (-2G) string] + "\n" # -3
\end{galgas}



\subsection{Opérateur « \texttt{mod} » infixé}
Il renvoie le reste de la division entière de l'expression de gauche par l'expression de droite, telle que décrite au dessus. Si cette dernière est nulle, alors un message d'erreur est affiché et le résultat n'est pas construit.

\begin{galgas}
  message [9876543210G mod 1234567890G string] + "\n" # 90
  message [(-9876543210G) mod 1234567890G string] + "\n" # -90
  message [(-9876543210G) mod (-1234567890G) string] + "\n"  # -90
  message [9876543210G mod (-1234567890G) string] + "\n"  # 90
  message [2000S mod 183S string] + "\n" # 170
  message [(-2000S) mod 183S string] + "\n" # -170
  message [(-2000S) mod (-183S) string] + "\n" # -170
  message [2000S mod (-183S) string] + "\n" # 170
\end{galgas}




\subsectionMethod{divideBy}{bigint}
Elle effectue la division dont le quotient arrondi \emph{vers zéro}, c'est-à-dire elle combine les opérateurs « \ggs!/! » et « \ggs!mod! » en une seule opération pour retourner quotient et reste.

\begin{galgas}
  @bigint quotient
  @bigint remainder
  [9876543210_9876543210G divideBy
    !1234567890G
    ?quotient:quotient
    ?remainder:remainder
  ]
  message [quotient string] + " " + remainder + "\n" # 80000000737 8280
  [-9876543210_9876543210G divideBy
    !1234567890G
    ?quotient:quotient
    ?remainder:remainder
  ]
  message [quotient string] + " " + remainder + "\n" # -80000000737 -8280
  [-9876543210_9876543210G divideBy
    !-1234567890G
    ?quotient:quotient
    ?remainder:remainder
  ]
  message [quotient string] + " " + remainder + "\n" # 80000000737 -8280
  [9876543210_9876543210G divideBy
    !-1234567890G
    ?quotient:quotient
    ?remainder:remainder
  ]
  message [quotient string] + " " + remainder + "\n" # -80000000737 8280
\end{galgas}




\subsectionMethod{floorDivideBy}{bigint}
Elle effectue toujours la division dont le quotient arrondi \emph{vers $-\infty$}.

\begin{galgas}
  @bigint quotient
  @bigint remainder
  [9876543210_9876543210G floorDivideBy
    !1234567890G
    ?quotient:quotient
    ?remainder:remainder
  ]
  message [quotient string] + " " + remainder + "\n" # 80000000737 8280
  [-9876543210_9876543210G floorDivideBy
    !1234567890G
    ?quotient:quotient
    ?remainder:remainder
  ]
  message [quotient string] + " " + remainder + "\n" # -80000000738 1234559610
  [-9876543210_9876543210G floorDivideBy
    !-1234567890G
    ?quotient:quotient
    ?remainder:remainder
  ]
  message [quotient string] + " " + remainder + "\n" # 80000000737 -8280
  [9876543210_9876543210G floorDivideBy
    !-1234567890G
    ?quotient:quotient
    ?remainder:remainder
  ]
  message [quotient string] + " " + remainder + "\n" # -80000000738 -1234559610
\end{galgas}





\subsectionMethod{ceilDivideBy}{bigint}
Elle effectue toujours la division dont le quotient arrondi \emph{vers $+\infty$}.

\begin{galgas}
  @bigint quotient
  @bigint remainder
  [9876543210_9876543210G ceilDivideBy
    !1234567890G
    ?quotient:quotient
    ?remainder:remainder
  ]
  message [quotient string] + " " + remainder + "\n" # 80000000738 -1234559610
  [-9876543210_9876543210G ceilDivideBy
    !1234567890G
    ?quotient:quotient
    ?remainder:remainder
  ]
  message [quotient string] + " " + remainder + "\n" # -80000000737 -8280
  [-9876543210_9876543210G ceilDivideBy
    !-1234567890G
    ?quotient:quotient
    ?remainder:remainder
  ]
  message [quotient string] + " " + remainder + "\n" # 80000000738 1234559610
  [9876543210_9876543210G ceilDivideBy
    !-1234567890G
    ?quotient:quotient
    ?remainder:remainder
  ]
  message [quotient string] + " " + remainder + "\n" # -80000000737 8280
\end{galgas}








\section{Décalages}

\subsection{Opérateur \texttt{<{}<}}

L'opérateur « \ggs!<<! » infixé effectue un décalage à gauche. L'expression de droite est toujours un \ggs!@uint!. Un décalage à gauche de $n$ bits est sémantiquement équivalent à une multiplication par $2^n$, que le nombre auquel s'applique le décalage soit signé ou non. C'est la sémantique des décalages à gauche des types \ggs!@sint! et \ggs!@sint64! :

\begin{galgas}
  message [0x1234567890G << 12 hexString] + "\n" # 0x1234567890000
  message [(-0x1234567890G) << 12 hexString] + "\n" # -0x1234567890000
  message [2000S << 2 string] + "\n" # 8000
  message [(-2000S) << 2 string] + "\n" # -8000
\end{galgas}

\subsection{Opérateur \texttt{>{}>}}

L'opérateur « \ggs!>>! » infixé effectue un décalage à droite. L'expression de droite est toujours un \ggs!@uint! :
\begin{galgas}
  message [0x1234567890G >> 12 hexString] + "\n" # 0x1234567
  message [(-0x1234567890G) >> 12 hexString] + "\n" # -0x1234567
  message [2000S >> 2 string] + "\n" # 500
  message [(-2000S) >> 2 string] + "\n" # -500
\end{galgas}

Un décalage à droite de $n$ bits d'un nombre posifif ou négatif est sémantiquement équivalent au quotient \emph{par défaut} d'une division par $2^n$, c'est-à-dire que le reste est toujours positif ou nul.

Quelques exemples de décalage à droite de nombres positifs :

\begin{galgas}
  message [9G >> 1 string] + "\n" # 4
  message [9S >> 1 string] + "\n" # 4
  message [7G >> 1 string] + "\n" # 3
  message [7S >> 1 string] + "\n" # 3
  message [3G >> 1 string] + "\n" # 1
  message [3S >> 1 string] + "\n" # 1
  message [1G >> 1 string] + "\n" # 0
  message [1S >> 1 string] + "\n" # 0
\end{galgas}


Et pour des nombres négatifs :

\begin{galgas}
  message [-9G >> 1 string] + "\n" # -5
  message [-9S >> 1 string] + "\n" # -5
  message [-7G >> 1 string] + "\n" # -4
  message [-7S >> 1 string] + "\n" # -4
  message [-3G >> 1 string] + "\n" # -2
  message [-3S >> 1 string] + "\n" # -2
  message [-1G >> 1 string] + "\n" # -1
  message [-1S >> 1 string] + "\n" # -1
\end{galgas}

Dans tous les cas, la sémantique du décalage à droite du type \ggs!@bigint! est la même que celles des types \ggs!@sint! et \ggs!@sint64!.










\section{Opérations logiques}

Le type \ggs!@bigint! implémente les opérations logiques \ggs!&! (\emph{et logique}), \ggs!|! (\emph{ou logique}), \ggs!^! (\emph{ou exclusif logique}) et \ggs!~! (\emph{négation logique}). Si les opérandes sont positifs ou nuls, le comportement de ces opérateurs est celui attendu. Pour comprendre le comportement avec des opérandes négatifs, ou de signe contraire, il faut considérer que la représentation des \ggs!@bigint! est la suivante :
\begin{itemize}
  \item la valeur d'un nombre positif ou nul est préfixée par une infinité de zéros ;
  \item la valeur d'un nombre strictement négatif est préfixée par une infinité de uns.
\end{itemize}

Par exemple :
\begin{itemize}
  \item \texttt{0x1234} est représenté par \texttt{0x…01234} ;
  \item \texttt{-0x1234} est représenté par \texttt{0x…FEDCC}.
\end{itemize}


\subsection{Opérateur \texttt{\&} infixé}

L'opérateur \ggs!&! infixé réalise un « \emph{et logique} » entre ses opérandes. Le résultat est positif ou nul dès qu'un des deux opérandes est positif.

\begin{galgas}
message [0x1234G & 0x4321G hexString] + "\n" # 0x220
message [-0x1234G & 0x4321G hexString] + "\n" # 0x4100
message [-0x80G & 0xFFG hexString] + "\n" # 0x80
\end{galgas}

Considérons le deuxième exemple et voyons comment le résultat est obtenu :

\begin{tabular}{llll}
Premier opérande & \texttt{0x…FEDCC} & représentation théorique de \texttt{-0x1234}\\
Second opérande  & \texttt{0x…04321} & représentation théorique de \texttt{0x4321} \\
Résultat & \texttt{0x…04100} & représentation théorique de \texttt{0x4100} \\
\end{tabular}

\subsection{Opérateur \texttt{|} infixé}


L'opérateur \ggs!|! infixé réalise un « \emph{ou logique} » entre ses opérandes. Le résultat est négatif dès qu'un des deux opérandes est négatif.

\begin{galgas}
message [0x1234G | 0x4321G hexString] + "\n" # 0x5335
message [-0x1234G | 0x4321G hexString] + "\n" # -0x1013
message [-0x80G | 0xFFG hexString] + "\n" # -0x1
\end{galgas}

Considérons le deuxième exemple et voyons comment le résultat est obtenu :

\begin{tabular}{llll}
Premier opérande & \texttt{0x…FEDCC} & représentation théorique de \texttt{-0x1234}\\
Second opérande  & \texttt{0x…04321} & représentation théorique de \texttt{0x4321} \\
Résultat & \texttt{0x…FEFED} & représentation théorique de \texttt{-0x1013} \\
\end{tabular}


\subsection{Opérateur $\wedge$ infixé}



L'opérateur \ggs!^! infixé réalise un « \emph{ou exclusif logique} » entre ses opérandes. Le résultat est négatif quand les deux opérandes sont de signe contraire, et positif si ils sont de même signe.

\begin{galgas}
message [0x1234G ^ 0x4321G hexString] + "\n" # 0x5115
message [-0x1234G ^ 0x4321G hexString] + "\n" # -0x5113
message [-0x80G ^ 0xFFG hexString] + "\n" # -0x81
message [-0x80G ^ -0xFFG hexString] + "\n" # 0x81
\end{galgas}

Considérons le deuxième exemple et voyons comment le résultat est obtenu :

\begin{tabular}{llll}
Premier opérande & \texttt{0x…FEDCC} & représentation théorique de \texttt{-0x1234}\\
Second opérande  & \texttt{0x…04321} & représentation théorique de \texttt{0x4321} \\
Résultat & \texttt{0x…FAEED} & représentation théorique de \texttt{-0x5113} \\
\end{tabular}


\subsection{Opérateur $\sim$ préfixé}

L'opérateur \ggs!~! préfixé réalise la complémentation logique de son opérande. Le résultat est négatif si l'opérande est positif ou nul, et positif si il est négatif.

\begin{galgas}
message [~  0x1234G hexString] + "\n" # -0x1235
message [~ -0x1234G hexString] + "\n" # 0x1233
\end{galgas}

Considérons le second exemple et voyons comment le résultat est obtenu :

\begin{tabular}{llll}
Opérande & \texttt{0x…FEDCC} & représentation théorique de \texttt{-0x1234}\\
Résultat & \texttt{0x…01233} & représentation théorique de \texttt{0x1233} \\
\end{tabular}











\section{Manipulation de bits}

Les constructions suivantes permettent d'accéder à un bit particulier de la représentation signée en \emph{complément à deux} de la valeur d'un \ggs!@bitint!.

Pour comprendre le comportement avec un récepteur négatif, il faut considérer, comme pour les opérateurs logiques, que la représentation des \ggs!@bigint! est la suivante :
\begin{itemize}
  \item la valeur d'un nombre positif ou nul est préfixée par une infinité de zéros ;
  \item la valeur d'un nombre strictement négatif est préfixée par une infinité de uns.
\end{itemize}

Par exemple :
\begin{itemize}
  \item \texttt{0x1234} est représenté par \texttt{0x…01234} ;
  \item \texttt{-0x1234} est représenté par \texttt{0x…FEDCC}.
\end{itemize}

\subsectionGetter{bitAtIndex}{bigint}

Ce \emph{getter} permet d'obtenir la valeur d'un bit particulier de la représentation signée en \emph{complément à deux} du récepteur. À partir d'un certain rang, la valeur obtenue pour un nombre positif est toujours \ggs!false!, et pour un nombre négatif toujours \ggs!true!.

\begin{galgas}
message [[0x1234G bitAtIndex !7] ocString] + "\n" # NO
message [[0x1234G bitAtIndex !5] ocString] + "\n" # YES
message [[0x1234G bitAtIndex !25] ocString] + "\n" # NO
message [[-0x1234G bitAtIndex !7] ocString] + "\n" # YES
message [[-0x1234G bitAtIndex !5] ocString] + "\n" # NO
message [[-0x1234G bitAtIndex !25] ocString] + "\n" # YES
\end{galgas}

\subsectionSetter{setBitAtIndex}{bigint}

Ce \emph{setter} permet de mettre à zéro ou à un bit particulier de la représentation signée en \emph{complément à deux} du récepteur. Noter que cette opération ne change jamais le signe d'un nombre.

\begin{galgas}
var a = 0x1234G
[!?a setBitAtIndex !true !14]
message [a hexString] + "\n" # 0x5234
[!?a setBitAtIndex !true !40]
message [a hexString] + "\n" # 0x10000005234
a = -0x1234G
[!?a setBitAtIndex !false !14]
message [a hexString] + "\n" # -0x5234
[!?a setBitAtIndex !false !40] # -0x10000005234
message [a hexString] + "\n"
\end{galgas}

Considérons le dernier exemple et voyons comment le résultat est obtenu :

\begin{tabular}{llll}
Récepteur & \texttt{0x…FFFF\_FFFF\_EDCC} & représentation théorique de \texttt{-0x1234}\\
Valeur de $2^{40}$ & \texttt{0x…0100\_0000\_0000} & représentation théorique de $2^{40}$ \\
Valeur de $\sim2^{40}$ & \texttt{0x…FEFF\_FFFF\_FFFF} & représentation théorique de $\sim2^{40}$ \\
Résultat & \texttt{0x…FEFF\_FFFF\_EDCC} & représentation théorique de \texttt{-0x10000005234} \\
\end{tabular}

Le résultat est un \emph{et logique} entre la valeur du récepteur et  $\sim2^{40}$.

\subsectionSetter{complementBitAtIndex}{bigint}

Ce \emph{setter} permet de complémenter un bit particulier de la représentation signée en \emph{complément à deux} du récepteur. Noter que cette opération ne change jamais le signe d'un nombre.

\begin{galgas}
var a = 0x1234G
[!?a complementBitAtIndex !14]
message [a hexString] + "\n" # 0x5234
a = -0x1234G
[!?a complementBitAtIndex !40]
message [a hexString] + "\n" # -0x10000005234
\end{galgas}

Considérons le dernier exemple et voyons comment le résultat est obtenu :

\begin{tabular}{llll}
Récepteur & \texttt{0x…FFFF\_FFFF\_EDCC} & représentation théorique de \texttt{-0x1234}\\
Résultat & \texttt{0x…FEFF\_FFFF\_EDCC} & représentation théorique de \texttt{-0x10000005234} \\
\end{tabular}


  %!TEX encoding = UTF-8 Unicode
%!TEX root = ../galgas-book.tex

\chapitreTypePredefiniLabelIndex{binaryset}

\tableDesMatieresLocaleDeProfondeurRelative{1}


Le type \ggst+@binaryset+ encode des ensembles, des relations binaires, des expressions booléennes. Il est implémenté par des BDD (Binary Decision Diagrams).


\section{Constructeurs}

\subsectionConstructor{binarySetWithBit}{binaryset}

\begin{galgas3box}
constructor binarySetWithBit ?@uint inBitIndex -> @binaryset
\end{galgas3box}


Retourne un \ggst+@binaryset+ dont le bit \ggst+inBitIndex+ est égal à 1.


\textbf{Exemple :}
\begin{galgas3}
@binaryset s = .binarySetWithBit {!2}
log s # Affiche <@binaryset: 1XX>
\end{galgas3}


\subsectionConstructor{binarySetWithEqualComparison}{binaryset}

\begin{galgas3box}
constructor binarySetWithEqualComparison
  ?@uint inLeftFirstIndex
  ?@uint inBitCount
  ?@uint inRightFirstIndex
  -> @binaryset
\end{galgas3box}




Retourne un \ggst+@binaryset+ qui encode la relation d'égalité entre deux variables.

Ce constructeur retourne un binary set qui encode la relation \emph{a~==~b}, où \emph{a} est encodé à partir du bit d'indice \emph{inLeftFirstIndex} jusqu'au bit d'indice \emph{inLeftFirstIndex  + inBitCount - 1}, et \emph{b} est encodé à partir du bit d'indice bit \emph{inRightFirstIndex} jusqu'au bit d'indice \emph{inRightFirstIndex + inBitCount - 1}.

\textbf{Exemple :}
\begin{galgas3}
@binaryset s = .binarySetWithEqualComparison {!0 !2 !3}
log s # Affiche <@binaryset: 00x00, 01X01, 10X10, 11X11>
\end{galgas3}




\subsectionConstructor{binarySetWithEqualToConstant}{binaryset}

\begin{galgas3box}
constructor binarySetWithEqualToConstant
  ?@uint inLeftFirstIndex
  ?@uint inBitCount
  ?@uint64 inConstant
  -> @binaryset
\end{galgas3box}


Retourne un \ggst+@binaryset+ object that encodes a equality relation between a variable and a constant.

Ce constructeur retourne un objet qui encode la relation the \emph{a~==~cst}, où \emph {a} est encodé à partir du bit d'indice \emph{inBitIndex} jusqu'au bit d'indice \emph{inBitIndex  + inBitCount - 1}, et \emph{cst} est défini par l'argument \emph{inConstant}.

\textbf{Exemple :}
\begin{galgas3}
@binaryset s = .binarySetWithEqualToConstant {!0 !6 !23L}
log s # Affiche <@binaryset: 10111>
\end{galgas3}




\subsectionConstructor{binarySetWithGreaterOrEqualComparison}{binaryset}

\begin{galgas3box}
constructor binarySetWithGreaterOrEqualComparison
  ?@uint inLeftFirstIndex
  ?@uint inBitCount
  ?@uint inRightFirstIndex
  -> @binaryset
\end{galgas3box}


Retourne un \ggst+@binaryset+ object qui encode la relation \emph{supérieur ou égal} entre deux variables.

Ce constructeur retourne un binary set qui encode la relation \emph{a~>=~b}, où \emph{a} est encodé à partir du bit d'indice \emph{inLeftFirstIndex} jusqu'au bit d'indice \emph{inLeftFirstIndex  + inBitCount - 1}, et \emph{b} est encodé à partir du bit d'indice bit \emph{inRightFirstIndex} jusqu'au bit d'indice \emph{inRightFirstIndex + inBitCount - 1}.

\textbf{Exemple :}
\begin{galgas3}
@binaryset s = .binarySetWithGreaterOrEqualComparison {!0 !2 !3}
log s # Affiche <@binaryset: 00XXX, 01X01, 01X1X, 10X1X, 11X11>
\end{galgas3}



\subsectionConstructor{binarySetWithGreaterOrEqualToConstant}{binaryset}

\begin{galgas3box}
constructor binarySetWithGreaterOrEqualToConstant
  ?@uint inLeftFirstIndex
  ?@uint inBitCount
  ?@uint64 inConstant
  -> @binaryset
\end{galgas3box}



Retourne un \ggst+@binaryset+ object that encodes a greater or equal relation between a variable and a constant.

The constructor returns a binary set that encodes the \emph{a~>=~cst} relation, where \emph {a} est encodé à partir du bit d'indice \emph{inBitIndex} jusqu'au bit d'indice \emph{inBitIndex  + inBitCount - 1}, and \emph{cst} is defined by the \emph{inConstant} argument.




\subsectionConstructor{binarySetWithLowerOrEqualComparison}{binaryset}

\begin{galgas3box}
constructor binarySetWithLowerOrEqualComparison
  ?@uint inLeftFirstIndex
  ?@uint inBitCount
  ?@uint inRightFirstIndex
  -> @binaryset
\end{galgas3box}


Retourne un \ggst+@binaryset+ object that encodes a lower or equal relation between two variables.

The constructor returns a binary set that encodes the \emph{a~<=~b} relation, where \emph{a} est encodé à partir du bit d'indice \emph{inLeftFirstIndex} jusqu'au bit d'indice \emph{inLeftFirstIndex  + inBitCount - 1}, and \emph{b} est encodé à partir du bit d'indice \emph{inRightFirstIndex} to \emph{inRightFirstIndex + inBitCount - 1}.

\textbf{Exemple :}
\begin{galgas3}
@binaryset s = .binarySetWithLowerOrEqualComparison !0 !2 !3]
log s # Affiche <@binaryset: 00X00, 01X0X, 10X0X, 10X10, 11XXX>
\end{galgas3}




\subsectionConstructor{binarySetWithLowerOrEqualToConstant}{binaryset}

\begin{galgas3box}
constructor binarySetWithLowerOrEqualToConstant
  ?@uint inLeftFirstIndex
  ?@uint inBitCount
  ?@uint64 inConstant
  -> @binaryset
\end{galgas3box}


Retourne un \ggst+@binaryset+ object that encodes a lower or equal relation between a variable and a constant.

The constructor returns a binary set that encodes the \emph{a~<=~cst} relation, where \emph {a} est encodé à partir du bit d'indice \emph{inBitIndex} jusqu'au bit d'indice \emph{inBitIndex  + inBitCount - 1}, and \emph{cst} is defined by the \emph{inConstant} argument.




\subsectionConstructor{binarySetWithNotEqualComparison}{binaryset}

\begin{galgas3box}
constructor binarySetWithNotEqualComparison
  ?@uint inLeftFirstIndex
  ?@uint inBitCount
  ?@uint inRightFirstIndex
  -> @binaryset
\end{galgas3box}



Retourne un \ggst+@binaryset+ object that encodes an inequality relation between two variables.

The constructor returns a binary set that encodes the \emph{a~!=~b} relation, where \emph{a} est encodé à partir du bit d'indice \emph{inLeftFirstIndex} jusqu'au bit d'indice \emph{inLeftFirstIndex  + inBitCount - 1}, and \emph{b} est encodé à partir du bit d'indice \emph{inRightFirstIndex} to \emph{inRightFirstIndex + inBitCount - 1}.

\textbf{Exemple :}
\begin{galgas3}
@binaryset s = .binarySetWithNotEqualComparison !0 !2 !3]
log s # Affiche <@binaryset: 00X01, 00X1X, 01X00, 01X1X, 10X0X, 10X11, 11X0X, 11X10>
\end{galgas3}




\subsectionConstructor{binarySetWithNotEqualToConstant}{binaryset}

\begin{galgas3box}
constructor binarySetWithNotEqualToConstant
  ?@uint inLeftFirstIndex
  ?@uint inBitCount
  ?@uint64 inConstant
  -> @binaryset
\end{galgas3box}


Retourne un \ggst+@binaryset+ object that encodes an inequality relation between a variable and a constant.

The constructor returns a binary set that encodes the \emph{a~!=~cst} relation, where \emph {a} est encodé à partir du bit d'indice \emph{inBitIndex} jusqu'au bit d'indice \emph{inBitIndex  + inBitCount - 1}, and \emph{cst} is defined by the \emph{inConstant} argument.







\subsectionConstructor{binarySetWithPredicateString}{binaryset}

\begin{galgas3box}
constructor binarySetWithPredicateString ?@string inPredicateString -> @binaryset
\end{galgas3box}

Returns the \ggst+@binaryset+ object described by the \emph{inPredicateString} argument.

The \emph{inBitString} argument string encodes a predicate string, such as those returned by \refGetterPage{binaryset}{predicateStringValue}.

\begin{description}
\item The \emph{inBitString} argument string characters should have one of the five following values:
\begin{itemize}
\item \texttt{\textquotesingle 0\textquotesingle}: a bit set to zero;
\item \texttt{\textquotesingle 1\textquotesingle}: a bit set to one;
\item \texttt{\textquotesingle X\textquotesingle}: a don't care bit;
\item \texttt{\textquotesingle~\textquotesingle}: a separator (non significant character);
\item \texttt{\textquotesingle\textbar\textquotesingle}: the boolean \emph{or} operation (in infix notation).
\end{itemize}
\end{description}


\textbf{Exemple :}
An empty predicate string (or a string containing only spaces) provides an empty binary set:
\begin{galgas3}
@binaryset s = .binarySetWithPredicateString !" "]
@bool b = = .s isEmptySet]; # b is true
\end{galgas3}


A predicate string containing only 'X' characters (at least one) provides an full binary set:
\begin{galgas3}
@binaryset s = .binarySetWithPredicateString !" X X"] # Spaces are non significant
@bool b = [s isFullSet]; # b is true
\end{galgas3}


A predicate string can encode a binary value (MSB first):
\begin{galgas3}
@binaryset s [binarySetWithPredicateString !"1100"] # 12 in decimal
log s # Affiche <@binaryset: 1100>
\end{galgas3}


You can use the boolean '|' operator for providing an or'ed values:
\begin{galgas3}
@binaryset s [binarySetWithPredicateString !" 1100 | 1101"]
log s # Affiche <@binaryset: 110X>
\end{galgas3}



You can use you can use don't care bits and '|' operator together:
\begin{galgas3}
@binaryset s [binarySetWithPredicateString !"11X00X0 | 111XXX"]
log s # Affiche <@binaryset: 1100X0, 111XXX>
\end{galgas3}




\subsectionConstructor{binarySetWithStrictGreaterComparison}{binaryset}

\begin{galgas3box}
constructor binarySetWithStrictGreaterComparison
  ?@uint inLeftFirstIndex
  ?@uint inBitCount
  ?@uint inRightFirstIndex
  -> @binaryset
\end{galgas3box}


Retourne un \ggst+@binaryset+ object that encodes a strict greater than relation between two variables.

The constructor returns a binary set that encodes the \emph{a~>~b} relation, where \emph{a} est encodé à partir du bit d'indice \emph{inLeftFirstIndex} jusqu'au bit d'indice \emph{inLeftFirstIndex  + inBitCount - 1}, and \emph{b} est encodé à partir du bit d'indice \emph{inRightFirstIndex} to \emph{inRightFirstIndex + inBitCount - 1}.

\textbf{Exemple :}
\begin{galgas3}
@binaryset s [binarySetWithStrictGreaterComparison !0 !2 !3]
log s # Affiche <@binaryset: 00X01, 00X1X, 01X1X, 10X11>
\end{galgas3}




\subsectionConstructor{binarySetWithStrictGreaterThanConstant}{binaryset}

\begin{galgas3box}
constructor binarySetWithStrictGreaterThanConstant
  ?@uint inLeftFirstIndex
  ?@uint inBitCount
  ?@uint64 inConstant
  -> @binaryset
\end{galgas3box}


Retourne un \ggst+@binaryset+ object that encodes a strict greater than relation between a variable and a constant.

The constructor returns a binary set that encodes the \emph{a~>~cst} relation, where \emph {a} est encodé à partir du bit d'indice \emph{inBitIndex} jusqu'au bit d'indice \emph{inBitIndex  + inBitCount - 1}, and \emph{cst} is defined by the \emph{inConstant} argument.




\subsectionConstructor{binarySetWithStrictLowerComparison}{binaryset}

\begin{galgas3box}
constructor binarySetWithStrictLowerComparison
  ?@uint inLeftFirstIndex
  ?@uint inBitCount
  ?@uint inRightFirstIndex
  -> @binaryset
\end{galgas3box}


Retourne un \ggst+@binaryset+ object that encodes a strict lower than relation between two variables.

The constructor returns a binary set that encodes the \emph{a~<~b} relation, where \emph{a} est encodé à partir du bit d'indice \emph{inLeftFirstIndex} jusqu'au bit d'indice \emph{inLeftFirstIndex  + inBitCount - 1}, and \emph{b} est encodé à partir du bit d'indice \emph{inRightFirstIndex} to \emph{inRightFirstIndex + inBitCount - 1}.

\textbf{Exemple :}
\begin{galgas3}
@binaryset s [binarySetWithStrictLowerComparison !0 !2 !3]
log s # Affiche <@binaryset: 01X00, 10X0X, 11X0X, 11X10>
\end{galgas3}




\subsectionConstructor{binarySetWithStrictLowerThanConstant}{binaryset}

\begin{galgas3box}
constructor binarySetWithStrictLowerThanConstant
  ?@uint inLeftFirstIndex
  ?@uint inBitCount
  ?@uint64 inConstant
  -> @binaryset
\end{galgas3box}


Retourne un \ggst+@binaryset+ object that encodes a strict lower than relation between a variable and a constant.

The constructor returns a binary set that encodes the \emph{a~<~cst} relation, where \emph {a} est encodé à partir du bit d'indice \emph{inBitIndex} jusqu'au bit d'indice \emph{inBitIndex  + inBitCount - 1}, and \emph{cst} is defined by the \emph{inConstant} argument.




\subsectionConstructor{emptyBinarySet}{binaryset}

\begin{galgas3box}
constructor emptyBinarySet -> @binaryset
\end{galgas3box}


Retourne un empty \ggst+@binaryset+ object.





\subsectionConstructor{fullBinarySet}{binaryset}

\begin{galgas3box}
constructor fullBinarySet -> @binaryset
\end{galgas3box}

Returns a full \ggst+@binaryset+ object.


\section{Getters}



\subsectionGetter{accessibleStates}{binaryset}

\begin{galgas3box}
getter accessibleStates -> @binaryset
\end{galgas3box}

Returns the set of accessible states from an initial state set. It computes the set of accessible states from the \emph{inInitialStateSet} state set using the accessibility relation encoded by the receiver.

\textbf{Exemple :}
\begin{galgas3}
@binaryset gr [binarySetWithPredicateString !"0001 0000"] # Edge 0 -> 1
gr = gr | [@binaryset binarySetWithPredicateString !"0010 0001"] # Edge 1 -> 2
gr = gr | [@binaryset binarySetWithPredicateString !"0011 0010"] # Edge 2 -> 3
gr = gr | [@binaryset binarySetWithPredicateString !"0100 0011"] # Edge 3 -> 4
gr = gr | [@binaryset binarySetWithPredicateString !"0101 0100"] # Edge 4 -> 5
@binaryset initialState [binarySetWithPredicateString !"0000"] # 0 is the initial state
@binaryset accessibleStates = [gr accessibleStates !initialState !4]
message " Accessible:"
@uint64list valueList = [accessibleStates uint64ValueList !4]
foreach valueList do
  message " " . [mValue string]
end foreach
message "\n"
\end{galgas3}


This program Affiche: \texttt{Accessible: 0 1 2 3 4 5}.



\subsectionGetter{binarySetByTranslatingFromIndex}{binaryset}

\begin{galgas3box}
getter binarySetByTranslatingFromIndex ?@uint inFirstIndex ?@uint inTranslation -> @string
\end{galgas3box}


Returns a \ggst+@binaryset+ value computed by translating the receiver's value by \emph{inTranslation} bits from index \emph{inFirstIndex}.



\subsectionGetter{compressedValueCount}{binaryset}

\begin{galgas3box}
getter compressedValueCount -> @uint64
\end{galgas3box}

Returns in an \ggst+@uint64+ value the number of different compressed string values encoded by receiver's value.



\subsectionGetter{compressedStringValueList}{binaryset}

\begin{galgas3box}
getter compressedStringValueList ?@uint inBitCount -> @stringlist
\end{galgas3box}

Returns the list of compressed string values corresponding to receiver's value, considering it uses \emph{inBitCount} bits.










\subsectionGetter{containsValue}{binaryset}

\begin{galgas3box}
getter containsValue ?@uint inFirstBit ?@uint inBitCount -> @bool
\end{galgas3box}


Retourne un \ggst+@bool+ value indicating whether the receiver'value contains a given value: \ggst+true+ if the receiver's contains a value, and \ggst+false+ otherwise; this value is computed from the \emph{inBitCount} first bits of \emph{inValue} value, shifted left by \emph{inFirstBit}.


\textbf{Exemple :}
\begin{galgas3}
var s = @binaryset.binarySetWithPredicateString {!"0 00XX X111| 1 1111 1111"}
log s # Affiche <@binaryset: 000XXX111, 111111111>
@bool b = [s containsValue !127L !0 !7]
log b # Affiche <@bool:true>
b = [s containsValue !31L !1 !7]
log b # Affiche <@bool:true>
b = [s containsValue !63L !1 !8]
log b # Affiche <@bool:false>
b = [s containsValue !7L !0 !9]
log b # Affiche <@bool:true>
b = [s containsValue !7L !0 !10]
log b # Affiche <@bool:true>
b = [s containsValue !32767L !1 !12]
log b # Affiche <@bool:true>
\end{galgas3}








\subsectionGetter{equalTo}{binaryset}

\begin{galgas3box}
getter equalTo ?@binaryset inOperand -> @binaryset
\end{galgas3box}

Returns the complement of the exclusive or between the receiver's value and the operand's value.

Note that \ggst+[a equalTo !b]+ is equivalent to \texttt{$\sim$ (a $\wedge$ b)}.

This operation returns un \ggst+@binaryset+ value; do not confuse with \ggst+==+ operator that Retourne un \ggst+@bool+ value.







\subsectionGetter{existOnBitIndex}{binaryset}

\begin{galgas3box}
getter existOnBitIndex ?@uint inBitIndex -> @binaryset
\end{galgas3box}

Returns the binary computed by applying the \emph{exist} operator on the \emph{inBitIndex} bit of the receiver's value.






\subsectionGetter{existsOnBitRange}{binaryset}

\begin{galgas3box}
getter existsOnBitRange ?@uint inFirstBitIndex ?@uint inBitCount -> @bool
\end{galgas3box}


Returns the binary computed by applying the \emph{exist} operator on the receiver's value, from \emph{inFirstBitIndex} bit index until the \emph{inFirstBitIndex + inBitCount - 1} bit index.


\textbf{Exemple :}
\begin{galgas3}
@binaryset s [binarySetWithPredicateString !"01110010"]
log s # Affiche <@binaryset: 01110010>
@binaryset ss = [s existsOnBitRange !2 !3]
log s # Affiche <@binaryset: 011XXX10>
\end{galgas3}







\subsectionGetter{existOnBitIndexAndBeyond}{binaryset}

\begin{galgas3box}
getter existOnBitIndexAndBeyond ?@uint inBitIndex -> @binaryset
\end{galgas3box}

Returns the binary set computed by applying the \emph{exist} operator on all bits from \emph{inFirstBitIndex} bit index of the receiver's value.







\subsectionGetter{forAllOnBitIndex}{binaryset}

\begin{galgas3box}
getter forAllOnBitIndex ?@uint inBitIndex -> @binaryset
\end{galgas3box}

Returns the binary set computed by applying the \emph{for all} operator on the \emph{inFirstBitIndex} bit index of the receiver's value.







\subsectionGetter{forAllOnBitIndexAndBeyond}{binaryset}

\begin{galgas3box}
getter forAllOnBitIndexAndBeyond ?@uint inBitIndex -> @binaryset
\end{galgas3box}


Returns the binary computed by applying the \emph{for all} operator on all bits from \emph{inFirstBitIndex} bit index of the receiver's value.








\subsectionGetter{greaterOrEqualTo}{binaryset}

\begin{galgas3box}
getter greaterOrEqualTo ?@binaryset inOperand -> @binaryset
\end{galgas3box}


Returns the complement of the exclusive or between the receiver's value and the operand's value.

Note that \ggst+[a greaterOrEqualTo !b]+ is equivalent to \texttt{(a \textbar ~$\sim$b)}.








\subsectionGetter{isEmpty}{binaryset}

\begin{galgas3box}
getter isEmpty -> @bool
\end{galgas3box}

Returns a \ggst+@bool+ value that indicates whether the receiver's value is the empty set :  \ggst+true+ if receiver's value is the empty set, and \ggst+false+ otherwise.







\subsectionGetter{isFull}{binaryset}

\begin{galgas3box}
getter isFull -> @bool
\end{galgas3box}

Returns a \ggst+@bool+ value that indicates whether the receiver's value is the full set : \ggst+true+ if receiver's value is the full set, and \ggst+false+ otherwise.







\subsectionGetter{ITE}{binaryset}

\begin{galgas3box}
getter ITE ?@binaryset inThenOperand ?@binaryset inElseOperand -> @binaryset
\end{galgas3box}


Returns the binary set computed by applying the \emph{ite} operator on the receiver's value, the \emph{inThenOperand} argument, and the  \emph{inElseOperand} argument.

{\texttt{ite (x, y, z)} is \texttt{(x \& y) \textbar ($\sim$x \& z)}.}







\subsectionGetter{lowerOrEqualTo}{binaryset}

\begin{galgas3box}
getter lowerOrEqualTo ?@binaryset inOperand -> @binaryset
\end{galgas3box}


Returns the binary set computed by applying the \emph{lower or equal} operator on the receiver's value and the \emph{inOperand} argument.
{\texttt{[a lowerOrEqualTo !b]} is \texttt{(($\sim$x) \textbar y)}.}







\subsectionGetter{notEqualTo}{binaryset}

\begin{galgas3box}
getter notEqualTo ?@binaryset inOperand -> @binaryset
\end{galgas3box}


Returns the binary set computed by applying the \emph{not equal} operator on the receiver's value and the \emph{inOperand} argument.
{\texttt{[a notEqualTo !b]} is \texttt{(x $\wedge$ y)}.}







\subsectionGetter{predicateStringValue}{binaryset}

\begin{galgas3box}
getter predicateStringValue -> @string
\end{galgas3box}

Returns a string representation of the receiver's value. The returned string is compatible with the \refConstructorPage{binaryset}{binarySetWithPredicateString}.







\subsectionGetter{strictGreaterThan}{binaryset}

\begin{galgas3box}
getter strictGreaterThan ?@binaryset inOperand -> @binaryset
\end{galgas3box}

Returns the binary set computed by applying the \emph{strict greater} operator on the receiver's value and the \emph{inOperand} argument.
{\texttt{[a strictGreaterThan !b]} is \texttt{(x \& $\sim$y)}.}







\subsectionGetter{strictLowerThan}{binaryset}

\begin{galgas3box}
getter strictLowerThan ?@binaryset inOperand -> @binaryset
\end{galgas3box}

Returns the binary set computed by applying the \emph{strict lower} operator on the receiver's value and the \emph{inOperand} argument.
{\texttt{[a strictLowerThan !b]} is \texttt{($\sim$x \& y)}.}







\subsectionGetter{stringValueList}{binaryset}

\begin{galgas3box}
getter stringValueList ?@uint inBitCount -> @stringlist
\end{galgas3box}

Returns the list of string values corresponding to receiver's value, considering it uses \emph{inBitCount} bits.







\subsectionGetter{stringValueListWithNameList}{binaryset}

\begin{galgas3box}
getter stringValueListWithNameList
  ?@uint inBitCount
  ?@stringlist inNameList
  -> @stringlist
\end{galgas3box}


Returns the list of named values corresponding to receiver's value, considering it uses \emph{inBitCount} bits. First, the receiver is enumerated, considering it uses \emph{inBitCount} bits. Each enumerated value is used as an index of \emph{inNameList}, and the string value at this index is appended at the end of the returned value.







\subsectionGetter{swap021}{binaryset}

\begin{galgas3box}
getter swap021
  ?@uint inBitCount1
  ?@uint inBitCount2
  ?@uint inBitCount3
  -> @binaryset
\end{galgas3box}



Returns the transposed \emph{(x, z, y)} relation.

This getter considers that the receiver encodes an \emph{(x, y, z)} relation, where \emph{x} is defined by bits index \emph{0} to \emph{inBitCount1  - 1}, \emph{y} is defined by bits index \emph{inBitCount1} to \emph{inBitCount1 + inBitCount2 - 1} and  \emph{z} is defined by bits index \emph{inBitCount1 + inBitCount2} to \emph{inBitCount1 + inBitCount2 + inBitCount3 - 1}.







\subsectionGetter{swap01}{binaryset}

\begin{galgas3box}
getter swap01 ?@uint inBitCount1 ?@uint inBitCount2 -> @binaryset
\end{galgas3box}


Returns the transposed \emph{(y, x)} relation.

This getter considers that the receiver encodes an \emph{(x, y)} relation, where \emph{x} is defined by bits index \emph{0} to \emph{inBitCount1  - 1}, \emph{y} is defined by bits index \emph{inBitCount1} to \emph{inBitCount1 + inBitCount2 - 1}.





\subsectionGetter{swap102}{binaryset}

\begin{galgas3box}
getter swap102
  ?@uint inBitCount1
  ?@uint inBitCount2
  ?@uint inBitCount3
  -> @binaryset
\end{galgas3box}

Returns the transposed \emph{(y, x, z)} relation.

This getter considers that the receiver encodes an \emph{(x, y, z)} relation, where \emph{x} is defined by bits index \emph{0} to \emph{inBitCount1  - 1}, \emph{y} is defined by bits index \emph{inBitCount1} to \emph{inBitCount1 + inBitCount2 - 1} and  \emph{z} is defined by bits index \emph{inBitCount1 + inBitCount2} to \emph{inBitCount1 + inBitCount2 + inBitCount3 - 1}.






\subsectionGetter{swap120}{binaryset}

\begin{galgas3box}
getter swap120
  ?@uint inBitCount1
  ?@uint inBitCount2
  ?@uint inBitCount3
  -> @binaryset
\end{galgas3box}

Returns the transposed \emph{(y, z, x)} relation.

This getter considers that the receiver encodes an \emph{(x, y, z)} relation, where \emph{x} is defined by bits index \emph{0} to \emph{inBitCount1  - 1}, \emph{y} is defined by bits index \emph{inBitCount1} to \emph{inBitCount1 + inBitCount2 - 1} and  \emph{z} is defined by bits index \emph{inBitCount1 + inBitCount2} to \emph{inBitCount1 + inBitCount2 + inBitCount3 - 1}.






\subsectionGetter{swap201}{binaryset}

\begin{galgas3box}
getter swap201
  ?@uint inBitCount1
  ?@uint inBitCount2
  ?@uint inBitCount3
  -> @binaryset
\end{galgas3box}

Returns the transposed \emph{(z, x, y)} relation.

This getter considers that the receiver encodes an \emph{(x, y, z)} relation, where \emph{x} is defined by bits index \emph{0} to \emph{inBitCount1  - 1}, \emph{y} is defined by bits index \emph{inBitCount1} to \emph{inBitCount1 + inBitCount2 - 1} and  \emph{z} is defined by bits index \emph{inBitCount1 + inBitCount2} to \emph{inBitCount1 + inBitCount2 + inBitCount3 - 1}.






\subsectionGetter{swap210}{binaryset}

\begin{galgas3box}
getter swap210
  ?@uint inBitCount1
  ?@uint inBitCount2
  ?@uint inBitCount3
  -> @binaryset
\end{galgas3box}

Returns the transposed \emph{(z, y, x)} relation.

This getter considers that the receiver encodes an \emph{(x, y, z)} relation, where \emph{x} is defined by bits index \emph{0} to \emph{inBitCount1  - 1}, \emph{y} is defined by bits index \emph{inBitCount1} to \emph{inBitCount1 + inBitCount2 - 1} and  \emph{z} is defined by bits index \emph{inBitCount1 + inBitCount2} to \emph{inBitCount1 + inBitCount2 + inBitCount3 - 1}.








\subsectionGetter{transitiveClosure}{binaryset}

\begin{galgas3box}
getter transitiveClosure ?@uint inBitCount -> @binaryset
\end{galgas3box}


Returns the transitive closure of the relation encoded by the receiver.

This getter considers that the receiver encodes an \emph{(x, y)} relation, where \emph{x} is defined by bits index \emph{0} to \emph{inBitCount  - 1}, \emph{y} is defined by bits index \emph{inBitCount} to \emph{2 * inBitCount - 1}.






\subsectionGetter{transposedBy}{binaryset}

\begin{galgas3box}
getter transposedBy ?@uintlist inVector -> @binaryset
\end{galgas3box}

Retourne la valeur transposée du récepteur. L'argument \ggst+inVector+ spécifie comment la transposition s'opère : la valeur à l'indice $i$ est l'indice de destination du bit $i$ dans le \emph{binaryset} renvoyé.

{\bf 1\textsuperscript{er} exemple.} Si on veut échanger les bits $0$ et $1$, on écrit :
\begin{galgas3}
let vector = @uintlist {!1, !0}
let result = [myBinarySet transposedBy !vector]
\end{galgas3}

{\bf 2\textsuperscript{e} exemple.}
\begin{galgas3}
  let b = @binaryset.binarySetWithStrictGreaterComparison {!0 !2 !4}
    & @binaryset.binarySetWithEqualToConstant {!2 !2 !0}
  log b # <@binaryset: 000001, 00001X, 01001X, 100011>
  let vr = @uintlist {!0, !1, !4, !5, !2, !3}
  let r = [b transposedBy !vr]
  log r # <@binaryset: 000001, 00001X, 00011X, 001011>
  let vs = @uintlist {!4, !5, !0, !1, !2, !3}
  let s = [b transposedBy !vs]
  log s # <@binaryset: 010000, 100X00, 110X00, 111000>
\end{galgas3}

La constante \ggst=b= encode la relation $A > B$, où $A$ est encodé par les bits 0 et 1, et $B$ par les bits 4 et 5. Les bits 2 et 3 sont fixés à 0. Dans le résultat \ggst=r=, $A$ est encodé par les bits 0 et 1 (inchangés), $B$ par les bits 2 et 3, et maintenant les bits 4 et 5 sont fixés à 0. Dans le résultat \ggst=s=, $A$ est encodé par les bits 4 et 5, $B$ par les bits 2 et 3, et les bits 0 et 1 sont fixés à 0.












\subsectionGetter{uint64ValueList}{binaryset}

\begin{galgas3box}
getter uint64ValueList ?@uint inBitCount -> @uint64list
\end{galgas3box}


Returns the list of \ggst+@uint64+ values corresponding to receiver's value, considering it uses \emph{inBitCount} bits.








\subsectionGetter{valueCount}{binaryset}

\begin{galgas3box}
getter valueCount ?@uint inBitCount -> @uint64
\end{galgas3box}


Returns in an \ggst+@uint64+ object the number of different values encoded by receiver, considering it uses \emph{inBitCount} bits. No overflow test in performed.







%-------------------------------

\section{Logical Operators}

The \ggst+@binaryset+ type supports the three logical operators:\newline

\begin{tabular}{|c|c|}
\hline
\texttt{$\&$} & Logical And, intersection \\
\hline
\texttt{\textbar} & Logical Or, union \\
\hline
\texttt{$\wedge$}  & Exclusive or \\
\hline
\end{tabular}

Theses operators require both arguments to be \ggst+@binaryset+ objects and return an \ggst+@binaryset+ object.\newline


The \ggst+@binaryset+ type supports the logical unary operator:\newline

\begin{tabular}{|c|c|}
\hline
$\sim$ & Negation, Complementation \\
\hline
\end{tabular}

This operator Retourne un \ggst+@binaryset+ object.







\section{Comparison Operators}

The \ggst+@binaryset+ type supports the two comparison operators:\newline

\begin{tabular}{|c|c|}
\hline
$=$ & Equality \\
\hline
$!=$ & Non Equality \\
\hline
\end{tabular}

Theses operators require both arguments to be \ggst+@binaryset+ objects, and return a \ggst+@bool+ object. These operations are very fast and are performed in a constant time (integer equality comparison).

Do not confuse with \refGetterPage{binaryset}{equalTo} and \refGetterPage{binaryset}{notEqualTo} that return a \ggst+@binaryset+ object.







\section{Shift Operators}

The \ggst+@binaryset+ type supports the two shift operators:\newline

\begin{tabular}{|c|c|}
\hline
$<<$ & Left Shift \\
\hline
$>>$ & Right Shift \\
\hline
\end{tabular}

\textbf{Exemple :}
\begin{galgas3}
@binaryset b [binarySetWithPredicateString !"1010"]
log b # Affiche: <@binaryset: 1010>
@binaryset bb = b << 3
log bb # Affiche: <@binaryset: 1010XXX>
\end{galgas3}


  %!TEX encoding = UTF-8 Unicode
%!TEX root = ../galgas-book.tex

\chapitreTypePredefiniLabelIndex{bool}

\tableDesMatieresLocaleDeProfondeurRelative{1}


Le type \ggst+@bool+ est le type booléen. Les deux mots réservés \ggst+true+ et \ggst+false+ sont du type \ggst+@bool+ type, et dénote les valeurs \emph{vari} et \emph{faux}. Le seul constructeur du \ggst+@bool+ type est le constructeur \ggst!default!, qui initialise un booléen à \ggst+false+.


\section{Conversion en chaîne de caractères}

\subsectionGetter{cString}{bool}

\begin{galgas3box}
getter cString -> @string
\end{galgas3box}

Retourne la chaîne \ggst!"true"! si le booléen est vrai, et la chaîne \ggst!"false"! dans le cas contraire.







\subsectionGetter{ocString}{bool}

\begin{galgas3box}
getter ocString -> @string
\end{galgas3box}

Retourne la chaîne \ggst!"YES"! si le booléen est vrai, et la chaîne \ggst!"NO"! dans le cas contraire.




\section{Conversion en entier}


\subsectionGetter{bigint}{bool}

\begin{galgas3box}
getter bigint -> @bigint
\end{galgas3box}

Retourne l'entier \ggst!1G! si le booléen est vrai, et l'entier \ggst!0G! dans le cas contraire.

\begin{galgas3}
  message [[false bigint] string] + "\n" # 0
  message [[true bigint] string] + "\n" # 1
\end{galgas3}


\subsectionGetter{sint}{bool}

\begin{galgas3box}
getter sint -> @sint
\end{galgas3box}

Retourne l'entier \ggst!1S! si le booléen est vrai, et l'entier \ggst!0S! dans le cas contraire.




\subsectionGetter{sint64}{bool}

\begin{galgas3box}
getter sint64 -> @sint64
\end{galgas3box}

Retourne l'entier \ggst!1LS! si le booléen est vrai, et l'entier \ggst!0LS! dans le cas contraire.




\subsectionGetter{uint}{bool}

\begin{galgas3box}
getter uint -> @uint
\end{galgas3box}

Retourne l'entier \ggst!1! si le booléen est vrai, et l'entier \ggst!0! dans le cas contraire.




\subsectionGetter{uint64}{bool}

\begin{galgas3box}
getter uint64 -> @uint64
\end{galgas3box}

Retourne l'entier \ggst!1L! si le booléen est vrai, et l'entier \ggst!0L! dans le cas contraire.




\section{Opérateurs logiques}

\begin{galgas3box}
operator @bool & @bool -> @bool
operator @bool | @bool -> @bool
operator @bool ^ @bool -> @bool
operator not @bool -> @bool
\end{galgas3box}

Le type \ggst+@bool+ accepte les trois opérateurs suivants
\begin{itemize}
\item l'opérateur \ggst!&! infixé qui effectue un \emph{et logique} ;
\item l'opérateur \ggst!|! infixé qui effectue un \emph{ou logique} ;
\item l'opérateur \ggst!^! infixé qui effectue un \emph{ou exclusif logique} ;
\item l'opérateur \ggst!not! infixe qui effectue la\emph{négation logique}.
\end{itemize}








\section{Comparaison}

Le type \ggst!@bool! implémente les six opérateurs de comparaison \ggst!==!, \ggst+!=+, \ggst!<!, \ggst!<=!, \ggst!>! et \ggst!>=!, avec \ggst!false < true!.

  %!TEX encoding = UTF-8 Unicode
%!TEX root = ../galgas-book.tex

\chapitreTypePredefiniLabelIndex{char}

An \galgas{@char} object value is an Unicode character. You can initialize an \galgas{@char} object from a character constant:

\begin{galgascode}
@char myCharacter = 'A' ;
\end{galgascode}


You have several ways for writing a literal character constant. In any case, it should define an assigned Unicode character. A compile-time error is raised if it does not.


A literal character constant is a single character or an escape sequence enclosed by single quotes (\galgas{'}).

For an ASCII printable character:

\begin{galgascode}
@char myCharacter = 'a' ;
\end{galgascode}


If you want to get ASCII source text file, any character that does not correspond to an ASCII printable character should be expressed with an escape sequence.

Otherwise, for any printable Unicode character, you can write it directly, without escape sequence, provided your text file encoding supports this character:\\

\texttt{@char myCharacter = \textquotesingle\ae\textquotesingle ;}\\

The following escape sequences are defined (they begin with a '\textquotesingle').

\begin{tabular}{|c|c|}
\hline
Character Constant & Meaning \\
\hline
\texttt{\textquotesingle\textbackslash f\textquotesingle} & A Form Feed Character \\
\hline
\texttt{\textquotesingle\textbackslash n\textquotesingle} & A New Line Character \\
\hline
\texttt{\textquotesingle\textbackslash r\textquotesingle} & A Carriage Return Character \\
\hline
\texttt{\textquotesingle\textbackslash v\textquotesingle} & A Vertical Tabulation Character \\
\hline
\texttt{\textquotesingle\textbackslash\textbackslash\textquotesingle} & A back slash Character \\
\hline
\texttt{\textquotesingle\textbackslash 0\textquotesingle} & A Nul Character \\
\hline
\texttt{\textquotesingle\textbackslash\textquotesingle\textquotesingle} & A Single Quote Character \\
\hline
\end{tabular}


\begin{tabular}{|c|c|}
\hline
Character Constant & Meaning \\
\hline
\texttt{\textquotesingle\textbackslash uABCD\textquotesingle} & An Unicode Character \\
\hline
\end{tabular}

Where \emph{ABCD} is a four digit hexadecimal number that represents an assigned Unicode point code. For example:

\texttt{@char myCharacter = \textquotesingle\textbackslash u03A0\textquotesingle ; \# The '$\Sigma$' character}\\

Note: an unassigned point code raises a compile-time error:\\
\texttt{@char myCharacter = \textquotesingle\textbackslash uFFFF\textquotesingle ; \# The \textbackslash uFFFF point code is not assigned}\\


\begin{tabular}{|c|c|}
\hline
Character Constant & Meaning \\
\hline
\texttt{\textquotesingle\textbackslash Uabcdxyzt\textquotesingle} & An Unicode Character \\
\hline
\end{tabular}

Where \emph{abcdxyzt} is a eight digit hexadecimal number that represents an assigned Unicode point code. For example:

\texttt{@char myCharacter = \textquotesingle\textbackslash U0010170\textquotesingle ; \# The 'GREEK ACROPHONIC NAXIAN FIVE HUNDRED' character}\\

Note: an unassigned point code raises a compile-time error:\\

\texttt{@char myCharacter = \textquotesingle\textbackslash U0000FFFF\textquotesingle ; \# Raises a compile-time error: \textbackslash U0000FFFF is not assigned.}\\

Any point code beyond \textbackslash U0010FFFF is invalid and not assigned.




\section{Constructors}



\constructeurSansArgument{replacementCharacter}
{char}
{1.8.3}
{char}
{Returns an \galgas{@char} object corresponding to Unicode replacement character (\texttt{\textquotesingle\textbackslash uFFFD}.}
{}







\constructeurUnArgument{unicodeCharacterWithUnsigned}
{char}
{1.8.3}
{char}
{@uint inValue}
{Returns an \galgas{@char} object from an Unicode code point.}
{A run-time error is raised if the \emph{inValue} value does not represent an assigned Unicode value. You can check if an \galgas{@uint} value represents an assigned Unicode value with the \refGetterPage{uint}{isUnicodeValueAssigned}.}


\section{Getters}


\subsectionGetter{isalnum}{char}

\begin{galgascode}
getter isalnum -> @bool
\end{galgascode}

Returns an \galgas{@bool} value indicating whether the receiver'value represents an ASCII letter or an ASCII digit: \galgas{true} if the receiver'value represents an ASCII letter or an ASCII digit (between \ggs+'A'+ and \ggs+'Z'+, or between \ggs'a'+ and \ggs+'z'+, or between \ggs+'0'+ and \ggs+'9'+, and \galgas{false} otherwise.




\subsectionGetter{isalpha}{char}

\begin{galgascode}
getter isalpha -> @bool
\end{galgascode}

Returns an \galgas{@bool} value indicating whether the receiver'value represents an ASCII letter: \galgas{true} if the receiver'value represents an ASCII letter (between \ggs+'A'+ and \ggs+'Z'+, or between \ggs+'a'+ and \ggs+'z'+), and \galgas{false} otherwise.




\subsectionGetter{iscntrl}{char}

\begin{galgascode}
getter iscntrl -> @bool
\end{galgascode}

Returns an \galgas{@bool} value indicating whether the receiver'value represents an ASCII control character: \galgas{true} if the receiver'value represents an ASCII control character (strictly before the \emph{SPACE} character), and \galgas{false} otherwise.





\subsectionGetter{isdigit}{char}

\begin{galgascode}
getter isdigit -> @bool
\end{galgascode}

Returns an \galgas{@bool} value indicating whether the receiver'value represents an ASCII digit: \galgas{true} if the receiver'value represents an ASCII digit (between \ggs+'0'+ and \ggs+'9'+), and \galgas{false} otherwise.





\subsectionGetter{islower}{char}

\begin{galgascode}
getter islower -> @bool
\end{galgascode}

Returns an \galgas{@bool} value indicating whether the receiver'value represents an ASCII lowercase ASCII letter: \galgas{true} if the receiver'value represents an ASCII lowercase letter (between \ggs+'a'+ and \ggs+'z'+), and \galgas{false} otherwise.






\subsectionGetter{isUnicodeCommand}{char}

\begin{galgascode}
getter isUnicodeCommand -> @bool
\end{galgascode}

Returns an \galgas{@bool} value indicating whether the receiver'value represents an Unicode command: \galgas{true} if the receiver'value represents an Unicode command, and \galgas{false} otherwise.






\subsectionGetter{isUnicodeLetter}{char}

\begin{galgascode}
getter isUnicodeLetter -> @bool
\end{galgascode}

Returns an \galgas{@bool} value indicating whether the receiver'value represents an Unicode letter: \galgas{true} if the receiver'value represents an Unicode letter, and \galgas{false} otherwise.






\subsectionGetter{isUnicodeMark}{char}

\begin{galgascode}
getter isUnicodeMark -> @bool
\end{galgascode}

Returns an \galgas{@bool} value indicating whether the receiver'value represents an Unicode mark character: \galgas{true} if the receiver'value represents an Unicode mark character, and \galgas{false} otherwise.






\subsectionGetter{isUnicodePunctuation}{char}

\begin{galgascode}
getter isUnicodePunctuation -> @bool
\end{galgascode}

Returns an \galgas{@bool} value indicating whether the receiver'value represents an Unicode punctuation character: \galgas{true} if the receiver'value represents an Unicode punctuation character, and \galgas{false} otherwise.






\subsectionGetter{isUnicodeSeparator}{char}

\begin{galgascode}
getter isUnicodeSeparator -> @bool
\end{galgascode}

Returns an \galgas{@bool} value indicating whether the receiver'value represents an Unicode separator character: \galgas{true} if the receiver'value represents an Unicode separator character, and \galgas{false} otherwise.






\subsectionGetter{isUnicodeSymbol}{char}

\begin{galgascode}
getter isUnicodeSymbol -> @bool
\end{galgascode}

Returns an \galgas{@bool} value indicating whether the receiver'value represents an Unicode symbol character: \galgas{true} if the receiver'value represents an Unicode symbol character, and \galgas{false} otherwise.









\subsectionGetter{isupper}{char}

\begin{galgascode}
getter isupper -> @bool
\end{galgascode}

Returns an \galgas{@bool} value indicating whether the receiver'value represents an ASCII uppercase ASCII letter: \galgas{true} if the receiver'value represents an ASCII uppercase letter (between \ggs+'A'+ and \ggs+'Z'+, and \galgas{false} otherwise.





\subsectionGetter{string}{char}

\begin{galgascode}
getter string -> @string
\end{galgascode}

Returns returns a string representation of the receiver's value: a one character \galgas{@string} object, containing the receiver's value.




\subsectionGetter{uint}{char}

\begin{galgascode}
getter uint -> @uint
\end{galgascode}

Returns an \galgas{@uint} object representing the Unicode code point of the receiver's value.




\subsectionGetter{unicodeName}{char}

\begin{galgascode}
getter unicodeName -> @string
\end{galgascode}

Returns the unicode name of the receiver's value: for an decimal string representation of the receiver's value, see the \refGetterPage{uint}{hexString}; for a decimal string representation of the receiver's value, see the \refGetterPage{uint}{string}.

\textbf{Exemple :}
\begin{galgascode}
['\AE' unicodeName] # returns "LATIN CAPITAL LETTER AE"
\end{galgascode}




\subsectionGetter{unicodeToLower}{char}

\begin{galgascode}
getter unicodeToLower -> @char
\end{galgascode}

Returns the lowercase character corresponding to the receiver's value: if the receiver's value is an Unicode uppercase character, this getter returns the corresponding lowercase character. Otherwise, it returns the receiver's value.

\textbf{Exemple :}
\begin{galgascode}
['Æ' unicodeToLower] # returns 'æ'
['æ' unicodeToLower] # returns 'æ'
\end{galgascode}




\subsectionGetter{unicodeToUpper}{char}

\begin{galgascode}
getter unicodeToUpper -> @char
\end{galgascode}

Returns the uppercase character corresponding to the receiver's value: if the receiver's value is an Unicode lowercase character, this getter returns the corresponding uppercase character. Otherwise, it returns the receiver's value.

\textbf{Exemple :}
\begin{galgascode}
['Æ' unicodeToUpper] # returns 'Æ'
['æ' unicodeToUpper] # returns 'Æ'
\end{galgascode}





\section{Comparison Operators}

The \galgas{@char} type supports the six comparison operators:\newline

\begin{tabular}{|c|c|}
\hline
$=$ & Equality \\
\hline
$!=$ & Non Equality \\
\hline
$<$  & Strict Lower Than \\
\hline
$<=$  & Lower or Equal \\
\hline
$>$  & Strict Greater Than \\
\hline
$>=$  & Greater or Equal \\
\hline
\end{tabular}

Theses operators require both arguments to be \galgas{@char} objects, and return a \ggs+@bool+ object. Comparison is done by comparing of the Unicode code point's value.



  %!TEX encoding = UTF-8 Unicode
%!TEX root = ../galgas-book.tex

\chapitreTypePredefiniLabelIndex{data}

Le type \ggs=@data= est un buffer d'octets. Il peut être utilisé pour lire et écrire des fichiers binaires.




\section{Constructeurs}


\subsectionConstructor{dataWithContentsOfFile}{data}

\begin{galgas}
constructor dataWithContentsOfFile ?@string inFilePath -> @data
\end{galgas}

Ce constructeur instancie un objet \ggs=@data= avec le contenu du fichier désigné par \ggs=inFilePath=. Si le fichier n'existe pas, une erreur d'exécution est déclenchée et le constructeur renvoie une valeur poison.




\subsectionConstructor{emptyData}{data}

\begin{galgas}
constructor emptyData -> @data
\end{galgas}

Ce constructeur instancie un objet \ggs=@data= vide.








\section{Getters}


\subsectionGetter{cStringRepresentation}{data}

\begin{galgas}
getter cStringRepresentation -> @string
\end{galgas}

Ce \emph{getter} renvoie la valeur du récepteur sous la forme d'une liste d'octets séparés par des virgules. Chaque octet est écrit en décimal. Toutes les 16 valeurs, un retour-chariot est inséré.



\subsectionGetter{length}{data}

\begin{galgas}
getter length -> @uint
\end{galgas}

Ce \emph{getter} renvoie le nombre d'octets du récepteur.





\section{Méthodes}


\subsectionMethod{writeToExecutableFile}{data}

\begin{galgas}
method writeToExecutableFile ?@string inFilePath
\end{galgas}

Cette méthode écrit le contenu du récepteur dans le fichier désigné par \ggs=inFilePath=, et rend ce fichier exécutable.




\subsectionMethod{writeToFile}{data}

\begin{galgas}
method writeToFile ?@string inFilePath
\end{galgas}

Cette méthode écrit le contenu du récepteur dans le fichier désigné par \ggs=inFilePath=.




\subsectionMethod{writeToFileWhenDifferentContents}{data}

\begin{galgas}
method writeToFileWhenDifferentContents
  ?@string inFilePath
  !@bool outFileModified
\end{galgas}

Cette méthode écrit le contenu du récepteur dans le fichier désigné par \ggs=inFilePath=, uniquement si la valeur du récepteur est différente du contenu du fihier. La variable \ggs=outFileModified= est retournée à l'appelant, et permet de savoir si le fichier a été modifié ou non.







\section{Setters}


\subsectionSetter{appendByte}{data}

\begin{galgas}
setter appendByte ?@uint inValue
\end{galgas}

Ce \emph{setter} ajoute la valeur de \ggs=inValue= à la fin du récepteur. Comme un objet de \ggs=@data= est un tableau d'octets, \ggs=inValue= doit être compris entre $0$ et $255$. Si il est supérieur à $255$, une erreur d'exécution est déclenchée.



\subsectionSetter{appendData}{data}

\begin{galgas}
setter appendData ?@data inData
\end{galgas}

Ce \emph{setter} ajoute la valeur de \ggs=inData= à la fin du récepteur.





\subsectionSetter{appendShortBE}{data}

\begin{galgas}
setter appendShortBE ?@uint inValue
\end{galgas}

Pour ce \emph{setter}, \ggs=inValue= doit être compris entre $0$ et $2^{16}-1$, c'est-à-dire réprésentable par un entier non signé sur deux octets. Si ce n'est pas le cas, une erreur d'exécution est déclenchée. Si c'est le cas, deux octets sont ajoutés à la fin du récepteur, d'abord l'octet de poids fort, puis l'octet de poids faible.







\subsectionSetter{appendShortLE}{data}

\begin{galgas}
setter appendShortLE ?@uint inValue
\end{galgas}

Pour ce \emph{setter}, \ggs=inValue= doit être compris entre $0$ et $2^{16}-1$, c'est-à-dire réprésentable par un entier non signé sur deux octets. Si ce n'est pas le cas, une erreur d'exécution est déclenchée. Si c'est le cas, deux octets sont ajoutés à la fin du récepteur, d'abord l'octet de poids faible, puis l'octet de poids fort.








\subsectionSetter{appendUIntBE}{data}

\begin{galgas}
setter appendUIntBE ?@uint inValue
\end{galgas}

Ce \emph{setter} ajoute la valeur de \ggs=inValue= à la fin du récepteur, sous la forme de quatre octets, en commençant par l'octet de poids fort.









\subsectionSetter{appendUIntLE}{data}

\begin{galgas}
setter appendUIntLE ?@uint inValue
\end{galgas}

Ce \emph{setter} ajoute la valeur de \ggs=inValue= à la fin du récepteur, sous la forme de quatre octets, en commençant par l'octet de poids faible.











\subsectionSetter{appendUTF8String}{data}

\begin{galgas}
setter appendUTF8String ?@string inValue
\end{galgas}

Ce \emph{setter} ajoute la valeur de \ggs=inValue= à la fin du récepteur, sous la forme d'une chaîne de caractères UTF-8, y compris le zéro final.


\section{Énumération des valeurs}

Un objet de type \ggs=@data= est énumérable par une instruction \ggs=for= (\refSectionPage{instructionFor}).

  %!TEX encoding = UTF-8 Unicode
%!TEX root = ../galgas-book.tex

\chapitreTypePredefiniLabelIndex{double}

The \ggs+@double+ object values correspond to the C type \ggs+@double+ values. You can initialize an \ggs+@double+ object from a float constant:

\begin{galgascode}
@double myDouble = 123.456
\end{galgascode}

Note that a \ggs+@double+ constant is characterized by the occurrence of the decimal point (.)

\section{Constructor}

\subsectionConstructor{doubleWithBinaryImage}{double}

\begin{galgascode}
constructor doubleWithBinaryImage ?@uint inValue -> @double
\end{galgascode}


Returns a double object from the binary image of the argument.



\subsectionConstructor{pi}{double}

\begin{galgascode}
constructor pi -> @double
\end{galgascode}



Returns an approximation of the $\pi$ constant value (\ggs+3.14159265358979323846264338327950288+).

\section{Getters}

\subsectionGetter{binaryImage}{double}

\begin{galgascode}
getter binaryImage -> @uint64
\end{galgascode}

Returns the binary image of the value of receiver's value.




\subsectionGetter{cos}{double}

\begin{galgascode}
getter cos -> @double
\end{galgascode}

Returns the \emph{cosine} value of receiver's value, expressed in radian.




\subsectionGetter{sin}{double}

\begin{galgascode}
getter sint -> @double
\end{galgascode}

Returns the \emph{sine} value of receiver's value, expressed in radian.




\subsectionGetter{sint}{double}

\begin{galgascode}
getter sint -> @sint
\end{galgascode}

Returns the receiver's value in an \refTypePredefini{sint} (32-bit signed integer) object: if receiver's value is outside \ggs+@sint+ bounds, a runtime error is raised.



\subsectionGetter{sint64}{double}

\begin{galgascode}
getter sint64 -> @sint64
\end{galgascode}

Returns the receiver's value in an \refTypePredefini{sint64} (64-bit signed integer) object: if receiver's value is outside \ggs+@sint64+ bounds, a runtime error is raised.




\subsectionGetter{string}{double}

\begin{galgascode}
getter string -> @string
\end{galgascode}

Returns a decimal string representation of the receiver's value (this getter never fails).




\subsectionGetter{tan}{double}

\begin{galgascode}
getter tan -> @double
\end{galgascode}

Returns the \emph{tangent} value of receiver's value, expressed in radian.







\subsectionGetter{uint}{double}

\begin{galgascode}
getter uint -> @uint
\end{galgascode}

Returns the receiver's value in an \refTypePredefini{uint} (32-bit unsigned integer) object: if receiver's value is outside \ggs+@uint+ bounds, a runtime error is raised.





\subsectionGetter{uint64}{double}

\begin{galgascode}
getter uint64 -> @uint64
\end{galgascode}

Returns the receiver's value in an \refTypePredefini{uint64} (64-bit unsigned integer) object: if receiver's value is outside \ggs+@uint64+ bounds, a runtime error is raised.




\section{Arithmetic Operators}

The \ggs+@double+ type supports the five arithmetic diadic operators:\newline

\begin{tabular}{|c|c|}
\hline
$+$ & Addition \\
\hline
$-$ & Substraction \\
\hline
$*$ & Multiplication \\
\hline
$/$ & Division \\
\hline
\ggs+mod+ & Modulo \\
\hline
\end{tabular}

Theses operators require both arguments to be \ggs+@double+ objects.\newline

A run-time error is raised if the operation leads to an overflow.

The \ggs+@double+ type supports the following arithmetic unary operators:\newline

\begin{tabular}{|c|c|}
\hline
$+$ & No operation \\
$-$ & Negate \\
\hline
\end{tabular}

This operator returns the receiver's value (an \ggs+@double+ object).






\section{Comparison Operators}

The \ggs+@double+ type supports the six comparison operators:\newline

\begin{tabular}{|c|c|}
\hline
$=$ & Equality \\
\hline
$!=$ & Non Equality \\
\hline
$<$  & Strict Lower Than \\
\hline
$<=$  & Lower or Equal \\
\hline
$>$  & Strict Greater Than \\
\hline
$>=$  & Greater or Equal \\
\hline
\end{tabular}

Theses operators require both arguments to be \ggs+@double+ objects, and return a \ggs+@bool+ object.



  %!TEX encoding = UTF-8 Unicode
%!TEX root = ../galgas-book.tex

\chapitreTypePredefiniLabelIndex{filewrapper}

Le type \galgas{@filewrapper} permet d'accéder à un \emph{filewrapper}, c'est à dire à des fichiers embarqués dans l'exécutable (voir \refChapterPage{filewrapper}).

\section{Constructor}

\section{Modifier}

\subsection{Modifier \texttt{setCurrentDirectory}}

\begin{galgascode}
modifier setCurrentDirectory ??@string inDirectory ;
\end{galgascode}


\section{Readers}




\subsection{Reader \texttt{allTextFilePathes}}

\begin{galgascode}
reader allTextFilePathes -> @stringlist ;
\end{galgascode}






\subsection{Reader \texttt{allDirectoryPathes}}

\begin{galgascode}
reader allDirectoryPathes -> @stringlist ;
\end{galgascode}






\subsection{Reader \texttt{currentDirectory}}

\begin{galgascode}
reader currentDirectory -> @string ;
\end{galgascode}








\subsection{Reader \texttt{allFilePathesWithExtension}}

\begin{galgascode}
reader allFilePathesWithExtension ??@string inExtension -> @stringlist ;
\end{galgascode}










\subsection{Reader \texttt{directoryExistsAtPath}}

\begin{galgascode}
reader directoryExistsAtPath ??@string inPath -> @bool ;
\end{galgascode}










\subsection{Reader \texttt{fileExistsAtPath}}

\begin{galgascode}
reader fileExistsAtPath ??@string inPath -> @bool ;
\end{galgascode}










\subsection{Reader \texttt{textFileContentsAtPath}}

\begin{galgascode}
reader textFileContentsAtPath ??@string inPath -> @string ;
\end{galgascode}










\subsection{Reader \texttt{binaryFileContentsAtPath}}

\begin{galgascode}
reader binaryFileContentsAtPath ??@string inPath -> @data ;
\end{galgascode}










\subsection{Reader \texttt{absolutePathForPath}}

\begin{galgascode}
reader absolutePathForPath ??@string inPath -> @string ;
\end{galgascode}



  %!TEX encoding = UTF-8 Unicode
%!TEX root = ../galgas-book.tex

\chapitreTypePredefiniLabelIndex{location}

An \galgas{@location} object value is a location in a source file. Objects of this type are useful for pointing out an error or a warning location.

\section{The \texttt{here} Keyword}

The \galgas{here} keyword indicates the current parsing location is the current source file. Assigning an \galgas{@location} object from the \galgas{here} keyword is a way for initializing an \galgas{@location} object:\newline

\texttt{@location currentLocation := here ;}

\section{Constructor}

\constructeurSansArgument{nowhere}
{location}
{2.1.2}
{location}
{Returns an \galgas{@location} that does not points out any location.}
{The returned object responds \galgas{true} to the \refReaderPage{location}{isNowhere}.}

\section{Readers}

\readerSansArgument{column}
{location}
{1.8.2}
{uint}
{Returns an \galgas{@uint} value containing the column of the receiver's value.}
{this reader raises a run-time error if the receiver's value responds \galgas{true} to the \refReaderPage{location}{isNowhere}.}


\readerSansArgument{isNowhere}
{location}
{2.1.2}
{bool}
{Returns an \galgas{@bool} value indicating whether the receiver'value points out a source location or does not.}
{this reader returns \galgas{true} if the receiver's value does not point out an actual location in a text source (i.e. it has been constructed using the nowhere constructor), and \galgas{false} if the receiver's value points out an actual location in a text source (i.e. it has been constructed using the \galgas{here} keyword.}


\readerSansArgument{line}
{location}
{1.8.2}
{uint}
{Returns an \galgas{@uint} value containing the line of the receiver's value.}
{this reader raises a run-time error if the receiver's value responds \galgas{true} to the \refReaderPage{location}{isNowhere}.}


\readerSansArgument{locationIndex}
{location}
{1.8.2}
{uint}
{Returns an \galgas{@uint} value containing the the offset from the the beginning of the source of the location defined by receiver's value.}
{this reader raises a run-time error if the receiver's value responds \galgas{true} to the \refReaderPage{location}{isNowhere}.}


\readerSansArgument{locationString}
{location}
{1.8.2}
{string}
{returns an \galgas{@string} object that contains a string representation of the location defined by receiver's value.}
{this reader raises a run-time error if the receiver's value responds \galgas{true} to the \refReaderPage{location}{isNowhere}.}

  %!TEX encoding = UTF-8 Unicode
%!TEX root = ../galgas-book.tex

\chapitreTypePredefiniLabelIndex{function}

\tableDesMatieresLocaleDeProfondeurRelative{1}


Le type \ggst=@function= permet de faire l'inventaire des fonctions définies dans votre projet GALGAS et de les appeler de manière indirecte. Un objet de type \ggst=@function= est une référence à une fonction du projet GALGAS, et permet de l'appeler de manière indirecte.

Pour faire l'inventaire des fonctions : \refConstructorPage{function}{functionList}.

Pour savoir si une fonction d'un certain nom existe : \refConstructorPage{function}{isFunctionDefined}.

Pour instancier un objet \ggst=@function= qui référence une fonction : \refConstructorPage{function}{functionWithName}, ou exploiter la liste retournée par le \refConstructorPage{function}{functionList}.

Pour connaître le type des arguments et le type retourné par une fonction :  \refGetterPage{function}{formalParameterTypeList} et  \refGetterPage{function}{resultType}.

Pour appeler une fonction :  \refGetterPage{function}{invoke}.






\section{Constructeurs}


\subsectionConstructor{functionList}{function}

\begin{galgas3}
constructor functionList -> @functionlist
\end{galgas3}

Ce constructeur renvoie la liste de toute les fonctions définies dans le projet GALGAS.



\subsectionConstructor{functionWithName}{function}

\begin{galgas3}
constructor functionWithName ?let @string inFunctionName -> @function
\end{galgas3}

Ce constructeur renvoie un objet de type \ggst=@function= permettant d'appeler de manière indirecte la fonction dont le nom est \ggst=inFunctionName=. Si il n'y a pas de fonction de ce nom, une erreur d'exécution est déclenchée, et une valeur \emph{poison} est retournée. Pour savoir si une fonction existe, utiliser le \refConstructorPage{function}{isFunctionDefined}.





\subsectionConstructor{isFunctionDefined}{function}

\begin{galgas3}
constructor isFunctionDefined ?let @string inFunctionName -> @bool
\end{galgas3}

Ce constructeur permet de savoir si une fonction dont le nom est \ggst=inFunctionName= existe.






\section{Getters}


\subsectionGetter{formalParameterTypeList}{function}

\begin{galgas3}
getter formalParameterTypeList -> @typelist
\end{galgas3}

Ce \emph{getter} renvoie la liste des types des arguments formels de la fonction désignée par le récepteur. Une fonction n'admet que des arguments formels en entrée, aussi le mode de passage est connu et n'est pas renvoyé par ce \emph{getter}.




\subsectionGetter{invoke}{function}

\begin{galgas3}
getter invoke ?@objectlist inParameters
              ?@location inErrorLocation -> @object
\end{galgas3}

Ce \emph{getter} appelle la fonction désignée par le récepteur avec la liste de paramètres effectifs \ggst=inParameters=. La valeur renvoyée par ce \emph{getter} est la valeur renvoyée par la fonction appelée. Si liste de paramètres effectifs \ggst=inParameters= est invalide (nombre incorrect d'éléments, type des arguments ne correspondant pas), une erreur d'exécution est déclenchée, en signalant la position de l'erreur grâce à \ggst=inErrorLocation=.





\subsectionGetter{resultType}{function}

\begin{galgas3}
getter resultType -> @type
\end{galgas3}

Ce \emph{getter} renvoie le type de la valeur retournée par la fonction désignée par le récepteur.






  \input{partie-types/type-object.tex}
  %!TEX encoding = UTF-8 Unicode
%!TEX root = ../galgas-book.tex

\sectionTypePredefiniLabelIndex{sint}

An \nomType{sint} object value is a 32-bit signed integer value. You can initialize an \nomType{sint} object from an 32-bit signed integer constant:\\

\texttt{@sint mySignedInteger := 123\_456S ;}

Note that a 32-bit signed integer constant is characterized by the 'S' suffix.




\constructeurSansArgument{min}
{@sint}
{1.3.0}
{@sint}
{Returns an \nomType{sint} object that the minimum value of the 32-bit signed range.}
{the returned value is $-2^{31}$.}





\constructeurSansArgument{max}
{@sint}
{1.3.0}
{@sint}
{Returns an \nomType{sint} object that the maximum value of the 32-bit signed range.}
{the returned value is $2^{31}-1$.}





\readerSansArgument{double}
{@sint}
{1.9.8}
{@double}
{Returns the receiver's value converted in a \nomType{double} object.}
{as a 32-bit integer value can always be converted in a \nomType{double} value, this reader never fails.}





\readerSansArgument{sint64}
{@sint}
{1.6.12}
{@sint64}
{Returns the receiver's value in an \refTypePredefini{sint64} (64-bit signed integer) object.}
{as a 32-bit signed value can always be converted in a 64-bit signed value, this reader never fails.}

This reader is the only way to convert an \refTypePredefini{sint} object into an \refTypePredefini{sint64} object.





\readerSansArgument{string}
{@sint}
{1.6.12}
{@string}
{Returns a decimal string representation of the receiver's value.}
{for an hexadecimal string representation of the receiver's value, see \refReaderPage{uint}{hexString} and \refReaderPage{uint}{xString}.}







\readerSansArgument{uint}
{@sint}
{1.3.0}
{@uint}
{Returns the receiver's value in an \refTypePredefini{uint} (32-bit unsigned integer) object.}
{an error is raised is receiver's value is negative.}

This reader is the only way to convert an \refTypePredefini{sint} object into an \refTypePredefini{uint} object.




\readerSansArgument{uint64}
{@sint}
{1.3.0}
{@uint64}
{Returns the receiver's value in an \refTypePredefini{uint64} (64-bit unsigned integer) object.}
{an error is raised is receiver's value is negative.}

This reader is the only way to convert an \refTypePredefini{sint} object into an \refTypePredefini{uint64} object.





\subsection{Incrementation and decrementation}

The \refTypePredefini{sint} supports incrementation and decrementation instructions.

\texttt{@sint n := ... ; n ++ ; \# Incrementation}

\texttt{@sint p := ... ; p -- ; \# Decrementation}\newline

The incrementation instruction raises an error if receiver's value is equal to $2^{31}-1$.\newline

The decrementation instruction raises an error if receiver's value is equal to $-2^{31}$.\newline

Note that incrementation and decrementation are not available within an expression.




\subsection{Arithmetic Operators}

The \nomType{sint} type supports the five arithmetic diadic operators:\newline

\begin{tabular}{|c|c|}
\hline
$+$ & Addition \\
\hline
$-$ & Substraction \\
\hline
$*$ & Multiplication \\
\hline
$/$ & Division \\
\hline
$\%$ & Modulo \\
\hline
\end{tabular}

Theses operators require both arguments to be \nomType{sint} objects.\newline

A run-time error is raised if the operation leads to a 32-bit signed overflow.

The \nomType{sint} type supports the following arithmetic unary operators:\newline

\begin{tabular}{|c|c|}
\hline
$+$ & No operation \\
\hline
$-$ & Negate \\
\hline
\end{tabular}

This operator returns the receiver's value (an \nomType{sint} object). A run-time error is raised if "-" operator is invoked on an object whose value is $-2^{31}$.






\subsection{Shift Operators}


The \nomType{sint} type supports right and left shift operators:\newline

\begin{tabular}{|c|c|}
\hline
$<<$ & Left shift \\
\hline
$>>$ & Right shift \\
\hline
\end{tabular}

Theses operators require the right argument to be \nomType{sint} object, and the left argument to be \nomType{uint} object.\newline

Note the right shift inserts a zero bit in the most significant bit location if the receiver's value is negative, and a one bit otherwise (it is a arithmetic right shift).\newline

The actual amount of the shift is the value of the right-hand operand masked by 31, i.e. the shift distance is always between 0 and 31.




\subsection{Logical Operators}

The \nomType{sint} type supports the three bit-wise logical operators:\newline

\begin{tabular}{|c|c|}
\hline
$\&$ & Bit-wise and \\
\hline
\textbar & Bit-wise or \\
\hline
\^\  & Bit-wise exclusive or \\
\hline
\end{tabular}

Theses operators require both arguments to be \nomType{sint} objects.\newline


The \nomType{sint} type supports the bit-wise logical unary operator:\newline

\begin{tabular}{|c|c|}
\hline
$\sim$ & Bit-wise complementation \\
\hline
\end{tabular}

This operator returns an \nomType{sint} object.







\subsection{Comparison Operators}

The \nomType{sint} type supports the six comparison operators:\newline

\begin{tabular}{|c|c|}
\hline
$=$ & Equality \\
\hline
$!=$ & Non Equality \\
\hline
$<$  & Strict Lower Than \\
\hline
$<=$  & Lower or Equal \\
\hline
$>$  & Strict Greater Than \\
\hline
$>=$  & Greater or Equal \\
\hline
\end{tabular}

Theses operators require both arguments to be \nomType{sint} objects, and return a \nomType{bool} object.



  %!TEX encoding = UTF-8 Unicode
%!TEX root = ../galgas-book.tex

\chapitreTypePredefiniLabelIndex{sint64}

\tableDesMatieresLocaleDeProfondeurRelative{1}


An \ggst+@sint64+ object value is a 64-bit signed integer value. You can initialize an \ggst+@sint64+ object from an 64-bit signed integer constant:\\

\texttt{@sint64 mySignedInteger = 123\_456LS ;}

Note that a 64-bit signed integer constant is characterized by the 'LS' suffix.

\section{Constructors}


\subsectionConstructor{min}{sint64}

\begin{galgas3}
constructor min -> @sint64
\end{galgas3}

Returns an \ggst+@sint64+ object that the minimum value of the 64-bit signed range ($-2^{63}$).





\subsectionConstructor{max}{sint64}

\begin{galgas3}
constructor max -> @sint64
\end{galgas3}

Returns an \ggst+@sint64+ object that the maximum value of the 64-bit signed range ($2^{63}-1$).


\section{Getters}


\subsectionGetter{bigint}{sint64}

Ce \emph{getter} permet de convertir un \ggst!@sint64! en \ggst!@bigint!. Comme la plage des valeurs des \ggst!bigint! n'est limitée que par la mémoire disponible, il n'échoue jamais.

\begin{galgas3}
  message [[-1234LS bigint] string] + "\n" # -1234
\end{galgas3}



\subsectionGetter{double}{sint64}

\begin{galgas3}
getter double -> @double
\end{galgas3}

Returns the receiver's value converted in a \ggst+@double+ object. As a 64-bit integer value can always be converted in a \ggst+@double+ value, this getter never fails.



\subsectionGetter{hexStringSeparatedBy}{sint64}

\begin{galgas3}
getter hexStringSeparatedBy ?@char inSeparator ?@uint inGroup -> @string
\end{galgas3}

Returns the an hexadecimal string representation of the receiver value, prefixed by the string \texttt{0x}. Groups of \ggst=inGroup= digits are separated by the \ggst=inSeparator= character.

If \ggst=inGroup= is equal to zero, a run-time error is raised.

For example:
\begin{galgas3}
let s = [0x123456789ABCDEF0LS hexStringSeparatedBy !'_' !3] # 0x1_234_567_89A_BCD_EF0
\end{galgas3}




\subsectionGetter{sint}{sint64}

\begin{galgas3}
getter sint -> @sint
\end{galgas3}

Returns the receiver's value in an \refTypePredefini{sint} (32-bit signed integer) object. An error is raised is receiver's value is lower than $-2^{31}$ or greater than $2^{31}-1$.

This getter is the only way to convert an \refTypePredefini{sint64} object into an \refTypePredefini{sint} object.





\subsectionGetter{string}{sint64}

\begin{galgas3}
getter string -> @string
\end{galgas3}

Returns a decimal string representation of the receiver's value. This getter never fails.








\subsectionGetter{uint}{sint64}

\begin{galgas3}
getter uint -> @uint
\end{galgas3}

Returns the receiver's value in an \refTypePredefini{uint} (32-bit unsigned integer) object. An error is raised is receiver's value is negative or greater than $2^{32}-1$.

This getter is the only way to convert an \refTypePredefini{sint64} object into an \refTypePredefini{uint} object.





\subsectionGetter{uint64}{sint64}

\begin{galgas3}
getter uint64 -> @uint64
\end{galgas3}

Returns the receiver's value in an \refTypePredefini{uint64} (64-bit unsigned integer) object. This getter raises a run-time error if the receiver's value is negative.

This getter is the only way to convert an \refTypePredefini{sint64} object into an \refTypePredefini{uint64} object.








\section{Arithmétique}

\subsection{Opérateurs infixés}

Le type \ggst+@sint64+ accepte les opérateurs arithmétiques infixés suivants :
\begin{itemize}
  \item \ggst!+!, addition, une erreur d'exécution est déclenchée en cas de débordement ;
  \item \ggst!-!, soustraction, une erreur d'exécution est déclenchée en cas de débordement ;
  \item \ggst!*!, multiplication, une erreur d'exécution est déclenchée en cas de débordement ;
  \item \ggst!/!, division, une erreur d'exécution est déclenchée si le diviseur est nul ;
  \item \ggst!mod!, calcul du reste, une erreur d'exécution est déclenchée si le diviseur est nul ;
  \item \ggst!&+!, addition, le résultat étant silencieusement tronqué sur 64 bits ;
  \item \ggst!&-!, soustraction, le résultat étant silencieusement tronqué sur 64 bits ;
  \item \ggst!&*!, multiplication, le résultat étant silencieusement tronqué sur 64 bits ;
  \item \ggst!&/!, division, qui retourne zéro si le diviseur est nul.
\end{itemize}

Ces opérateurs exigent que les deux opérandes soient des objets du même type \ggst+@sint64+.

\subsection{Opérateurs préfixés}
Le type \ggst+@sint64+ accepte les opérateurs arithmétiques préfixés suivants :
\begin{itemize}
  \item \ggst!+!, qui retourne simplement la valeur de l'opérande ;
  \item \ggst!-!, négation arithmétique, une erreur d'exécution est déclenchée si l'opérande est égal à $-2^{63}$ ;
  \item \ggst!&-!, négation arithmétique, sans détection de débordement : la négation de $-2^{63}$ est $-2^{63}$.
\end{itemize}

La valeur renvoyée est du même type  \ggst+@sint64+.


\subsectionLabel{Instructions}{instructionsSINT64}

Le type \ggst+@sint64+ accepte les instructions arithmétiques suivantes :
\begin{itemize}
  \item \ggst!+=!, addition, une erreur d'exécution est déclenchée en cas de débordement ;
  \item \ggst!-=!, soustraction, une erreur d'exécution est déclenchée en cas de débordement ;
  \item \ggst!*=!, multiplication, une erreur d'exécution est déclenchée en cas de débordement ;
  \item \ggst!/=!, division, une erreur d'exécution est déclenchée en cas division par zéro ;
  \item \ggst!++!, incrémentation, une erreur d'exécution est déclenchée en cas de débordement ;
  \item \ggst!--!, décrémentation, une erreur d'exécution est déclenchée en cas de débordement ;
  \item \ggst!&++!, incrémentation, le résultat étant silencieusement tronqué sur 64 bits ;
  \item \ggst!&--!, décrémentation, le résultat étant silencieusement tronqué sur 64 bits.
\end{itemize}

\ggst!x+=y! est équivalent à \ggst!x=x+y! ; \ggst!x-=y! est équivalent à \ggst!x=x-y!.
La variable cible \ggst!x!, comme l'expression source \ggst!y! doivent être du même type \ggst+@sint64+.

Incrémentation et décrémentation sont des instructions, et ne peuvent pas apparaître des expressions.
\begin{galgas3}
@sint64 n = ... ; n ++ # Incrémentation
\end{galgas3}

\begin{galgas3}
@sint64 n = ... ; n -- # Décrémentation
\end{galgas3}







\section{Shift Operators}


The \ggst+@sint64+ type supports right and left shift operators:\newline

\begin{tabular}{|c|c|}
\hline
$<<$ & Left shift \\
\hline
$>>$ & Right shift \\
\hline
\end{tabular}

Theses operators require the right argument to be \ggst+@sint64+ object, and the left argument to be \ggst+@uint+ object.\newline

Note the right shift inserts a zero bit in the most significant bit location if the receiver's value is negative, and a one bit otherwise (it is a arithmetic right shift).\newline

The actual amount of the shift is the value of the right-hand operand masked by 63, i.e. the shift distance is always between 0 and 63.




\section{Logical Operators}

The \ggst+@sint64+ type supports the three bit-wise logical operators:\newline

\begin{tabular}{|c|c|}
\hline
$\&$ & Bit-wise and \\
\hline
\textbar & Bit-wise or \\
\hline
\^\  & Bit-wise exclusive or \\
\hline
\end{tabular}

Theses operators require both arguments to be \ggst+@sint64+ objects.\newline


The \ggst+@sint64+ type supports the bit-wise logical unary operator:\newline

\begin{tabular}{|c|c|}
\hline
$\sim$ & Bit-wise complementation \\
\hline
\end{tabular}

This operator returns an \ggst+@sint64+ object.







\section{Comparison Operators}

The \ggst+@sint64+ type supports the six comparison operators:\newline

\begin{tabular}{|c|c|}
\hline
$=$ & Equality \\
\hline
$!=$ & Non Equality \\
\hline
$<$  & Strict Lower Than \\
\hline
$<=$  & Lower or Equal \\
\hline
$>$  & Strict Greater Than \\
\hline
$>=$  & Greater or Equal \\
\hline
\end{tabular}

Theses operators require both arguments to be \ggst+@sint64+ objects, and return a \ggst+@bool+ object.



  %!TEX encoding = UTF-8 Unicode
%!TEX root = ../galgas-book.tex

\chapitreTypePredefiniLabelIndex{string}

A \galgas{@string} object value is an Unicode character string value. The @string type defines several constructors, getters constant methods and setters, described below.

\paragraph{Literal String Constants.}

Characters strings are written enclosed within quotation marks (") characters, as in many languages. For example: "a string". Note that a literal string constant is an actual @string object, so a getter can be used on it. For example: \lstinline[language=galgas]{["ae" uppercaseString]} returns the "AE" string.

\section{Getters}

\readerUnArgument{containsCharacter}
{string}
{2.5.0}
{bool}
{@char inCharacter}
{Returns true if the receiver contains the given charactezr, and false oteherwise.}
{}

\begin{galgascode}
@string s := "abcdef";
@string s2 := [s rightSubString!3]; # The value of s2 is "def"
\end{galgascode}

\readerDeuxArguments{subString}
{string}
{1.7.8}
{string}
{@uint inStart}
{@uint inLength}
{Creates and returns the string built with the \emph{inLength} last characters of the receiver. If the receiver contains less than inLength characters, the receiver’s value is returned.}
{}


%\constructeurSansArgument{emptySet}
%{@stringset}
%{1.3.0}
%{@stringset}
%{Creates and returns an empty \galgas{@stringset} object.}
%{}
%
%\constructeurUnArgument{setWithString}
%{@stringset}
%{1.3.0}
%{@stringset}
%{@string inString}
%{Creates and returns an \galgas{@stringset} object that contains the value of the \emph{inString} argument object.}
%{}
%
%
%\readerSansArgument{count}
%{@stringset}
%{1.3.0}
%{@uint}
%{Returns the number of strings in the set.}
%{}
%
%
%
%\readerUnArgument{hasKey}
%{@stringset}
%{1.3.0}
%{@bool}
%{@string inString}
%{Returns a boolean value that indicates whether the value of \emph{inString} argument is present in the set.}
%{returns \motCle{true} if the value of \emph{inString} argument is present in the set, \motCle{false} otherwise.}
%
%
%
%
%\modifierUnArgument{removeKey}
%{@stringset}
%{1.3.0}
%{@string inString}
%{Removes the value of \emph{inString} argument from the receiver's value.}
%{if the receiver's value does not contain the value of \emph{inString} argument, this setter leaves the receiver's value unchanged.}
%
%
%
%
%
%
%\subsection{the \emph{+=} Operator}
%
%The \emph{+=} operator adds a string value to the receiver. If the receiver's value already contains the added value, this operator has no effect.
%
%\exempleTroisLignes
%{}
%{@string aString := ... ;}
%{@stringset aStringSet := ... ;}
%{aStringSet += !aString ;}
%
%
%
%
%\subsection{the \emph{$\&$} Operator}
%
%The \emph{$\&$} operator returns the intersection of its operand values.
%
%\exempleTroisLignes
%{}
%{@stringset s1 := ... ;}
%{@stringset s2 := ... ;}
%{@stringset s := s1 \& s2 ; \# s is the intersection of s1 and s2}
%
%
%
%
%
%
%\subsection{the \emph{$\textbar$} Operator}
%
%The \emph{$\textbar$} operator returns the union of its operand values.
%
%\exempleTroisLignes
%{}
%{@stringset s1 := ... ;}
%{@stringset s2 := ... ;}
%{@stringset s := s1 \textbar s2 ; \# s is the union of s1 and s2}
%
%
%
%
%
%
%\subsection{the \emph{$-$} Operator}
%
%The \emph{$-$} operator returns the difference of its operand values.
%
%\exempleTroisLignes
%{}
%{@stringset s1 := ... ;}
%{@stringset s2 := ... ;}
%{@stringset s := s1 - s2 ; \# s is the difference of s1 and s2}
%
%
%
%
%
%
%
%
%\subsection{Enumerating \galgas{@stringset} objects}
%
%
%The \motCle{foreach} instruction can be used for enumerating \galgas{@stringset} values; enumeration is performed in the ascending order, or in the reverse alphabetical order using the '>' qualifier.
%
%\texttt{@stringset s := ... ;}\newline
%\textbf{foreach} \texttt {s} \textbf {do}\newline
%\texttt{\# the \emph{key} constant has the value of current entry of \emph{s} stringset}\newline
%\textbf{end foreach} \texttt{;}
%
%
%
%
%
%
%
%\subsection{Comparison Operators}
%
%The \galgas{@stringset} type supports the six comparison operators:\newline
%
%\begin{tabular}{|c|c|}
%\hline
%$=$ & Equality \\
%\hline
%$!=$ & Non Equality \\
%\hline
%$<$  & Strict Inclusion \\
%\hline
%$<=$  & Inclusion or Equality \\
%\hline
%$>$  & Strict Greater \\
%\hline
%$>=$  & Greater or Equality \\
%\hline
%\end{tabular}
%
%Theses operators require both arguments to be \galgas{@stringset} objects, and return a \galgas{@stringset} object.
%
%

  %!TEX encoding = UTF-8 Unicode
%!TEX root = ../galgas-book.tex

\chapitreTypePredefiniLabelIndex{stringset}

An \ggs+@stringset+ object value is a set of \ggs+@string+ values.\\

\section{Constructors}

\subsectionConstructor{emptySet}{stringset}

\begin{galgascode}
constructor emptySet -> @stringset
\end{galgascode}


Creates and returns an empty \ggs+@stringset+ object.

\subsectionConstructor{setWithString}{stringset}

\begin{galgascode}
constructor setWithString ?@string inString -> @stringset
\end{galgascode}


Creates and returns an \ggs+@stringset+ object that contains the value of the \emph{inString} argument object.

\section{Getters}

\subsectionGetter{count}{stringset}

\begin{galgascode}
getter count -> @uint
\end{galgascode}

Returns the number of strings in the set.



\subsectionGetter{hasKey}{stringset}

\begin{galgascode}
getter hasKey ?@string inString -> @bool
\end{galgascode}

Returns a boolean value that indicates whether the value of \emph{inString} argument is present in the set: \ggs+true+ if the value of \emph{inString} argument is present in the set, \ggs+false+ otherwise.


\subsectionGetter{anyString}{stringset}

\begin{galgascode}
getter anyString -> @string
\end{galgascode}

Retourne une des chaînes de caractères contenue dans le récepteur. Si le récepteur est vide, une erreur d'exécution est déclenchée.




\section{Setter}

\subsectionSetter{removeKey}{stringset}

\begin{galgascode}
setter removeKey ?@string inString
\end{galgascode}


Removes the value of \emph{inString} argument from the receiver's value. If the receiver's value does not contain the value of \emph{inString} argument, this setter leaves the receiver's value unchanged.






\section{the \texttt{+=} Operator}

The \emph{+=} operator adds a string value to the receiver. If the receiver's value already contains the added value, this operator has no effect.

\textbf{exemple :}
\begin{galgascode}
@string aString = ... ;
@stringset aStringSet = ... ;
aStringSet += !aString ;
\end{galgascode}




\section{the \emph{$\&$} Operator}

The \emph{$\&$} operator returns the intersection of its operand values.

\textbf{exemple :}
\begin{galgascode}
@stringset s1 = ... ;
@stringset s2 = ... ;
@stringset s = s1 & s2 ; # s is the intersection of s1 and s2
\end{galgascode}






\section{the \emph{$\textbar$} Operator}

The \emph{$\textbar$} operator returns the union of its operand values.

\textbf{exemple :}
\begin{galgascode}
@stringset s1 = ... ;
@stringset s2 = ... ;
@stringset s = s1 | s2 ; # s is the union of s1 and s2
\end{galgascode}






\section{the \emph{$-$} Operator}

The \emph{$-$} operator returns the difference of its operand values.

\textbf{exemple :}
\begin{galgascode}
@stringset s1 = ... ;
@stringset s2 = ... ;
@stringset s = s1 - s2 ; \# s is the difference of s1 and s2
\end{galgascode}








\section{Enumerating \texttt{@stringset} objects}


The \ggs+for+ instruction can be used for enumerating \ggs+@stringset+ values; enumeration is performed in the ascending order, or in the reverse alphabetical order using the '>' qualifier.

\texttt{@stringset s = ... ;}\newline
\textbf{foreach} \texttt {s} \textbf {do}\newline
\texttt{\# the \emph{key} constant has the value of current entry of \emph{s} stringset}\newline
\textbf{end foreach} \texttt{;}







\section{Comparison Operators}

The \ggs+@stringset+ type supports the six comparison operators:\newline

\begin{tabular}{|c|c|}
\hline
$=$ & Equality \\
\hline
$!=$ & Non Equality \\
\hline
$<$  & Strict Inclusion \\
\hline
$<=$  & Inclusion or Equality \\
\hline
$>$  & Strict Greater \\
\hline
$>=$  & Greater or Equality \\
\hline
\end{tabular}

Theses operators require both arguments to be \ggs+@stringset+ objects, and return a \ggs+@stringset+ object.



  %!TEX encoding = UTF-8 Unicode
%!TEX root = ../galgas-book.tex

\chapitreTypePredefiniLabelIndex{timer}

\tableDesMatieresLocaleDeProfondeurRelative{1}


Le type \ggs!@timer! permet de mesurer des durées d'exécution de portions de code ; une utilisation typique est :

\begin{galgas}
var @timer t = .start
  # instructions
message "Durée : " + [t string] + "\n" # Affiche la durée d'exécution des instructions
\end{galgas}


\section{Constructeurs}

Le type \ggs!@timer! accepte deux constructeurs :
\begin{itemize}
  \item le contructeur \ggs!start! ;
  \item le constructeur \ggs!default! (\refSubsectionPage{constructeurParDefaut}), qui a le même effet que le constructeur \ggs!start!.
\end{itemize}


\subsectionConstructor{start}{timer}

\begin{galgasbox}
constructor @timer start -> @timer
\end{galgasbox}

Appeler le constructeur \ggs!start! est la seule façon d'instancier un objet \ggs!@timer!. Le chronomètre est enclenché, c'est-à-dire qu'il compte la durée à partir de laquelle le constructeur \ggs!start! a été appelé.







\section{Setters}

Le type \ggs!@timer! accepte deux \emph{setters} :
\begin{itemize}
  \item le \emph{setter} \ggs!resume! ;
  \item le \emph{setter} \ggs!stop!.
\end{itemize}

\subsectionSetter{resume}{timer}

\begin{galgasbox}
setter @timer resume
\end{galgasbox}

Le \emph{setter} \ggs!resume! redémarre le chronomètre si il est arrêté, et le réinitialise si il est en marche.
\subsectionSetter{stop}{timer}

\begin{galgasbox}
setter @timer stop
\end{galgasbox}

Le \emph{setter} \ggs!stop! arrête le chronomètre. Si il est déjà arrêté, appeler ce \emph{setter} n'a aucun effet.







\section{Getters}

Le type \ggs!@timer! accepte trois \emph{getters} :
\begin{itemize}
  \item le \emph{getter} \ggs!isRunning! ;
  \item le \emph{getter} \ggs!msFromStart! ;
  \item le \emph{getter} \ggs!string!.
\end{itemize}

\subsectionGetter{isRunning}{timer}

\begin{galgasbox}
getter @timer isRunning -> @bool
\end{galgasbox}

Ce \emph{getter} renvoie  \ggs!true! si le récepteur décompte le temps, ou \ggs!false! si il a été arrêté par un appel au \emph{setter} \ggs!stop!.


\subsectionGetter{msFromStart}{timer}

\begin{galgasbox}
getter @timer msFromStart -> @uint
\end{galgasbox}

La valeur obtenue par le \emph{getter} \ggs!msFromStart! est la durée écoulée depuis son instanciation (par le constructeur \ggs!start!) ou depuis le dernier appel au \emph{setter} \ggs!resume!. La durée est exprimée en millisecondes.


\subsectionGetter{string}{timer}

\begin{galgasbox}
getter @timer string -> @string
\end{galgasbox}

La valeur obtenue par le \emph{getter} \ggs!string! est la durée écoulée depuis son instanciation (par le constructeur \ggs!start!) ou depuis le dernier appel au \emph{setter} \ggs!resume!. La durée est exprimée sous la forme d'une chaîne de caractères.


  %!TEX encoding = UTF-8 Unicode
%!TEX root = ../galgas-book.tex

\chapitreTypePredefiniLabelIndex{type}




  %!TEX encoding = UTF-8 Unicode
%!TEX root = ../galgas-book.tex

\chapitreTypePredefiniLabelIndex{uint}

\tableDesMatieresLocaleDeProfondeurRelative{1}


An \ggst+@uint+ object value is a 32-bit unsigned integer value. You can initialize an \ggst+@uint+ object from an unsigned integer constant:\\

\begin{galgas3}
@uint myUnsignedInteger = 123_456 ;
\end{galgas3}

Note that a 32-bit unsigned integer constant is characterized by no suffix.

\section{Constructors}

\subsectionConstructor{errorCount}{uint}

\begin{galgas3}
constructor errorCount -> @uint
\end{galgas3}


Returns an \ggst+@uint+ object that contains the number of errors. The returned value is the cumulative count of errors from the beginning of execution.

\textbf{Exemple :}
\begin{galgas3}
@uint x = [@uint errorCount] ;
\end{galgas3}




\subsectionConstructor{max}{uint}

\begin{galgas3}
constructor max -> @uint
\end{galgas3}

Returns an \ggst+@uint+ object that the maximum value of the 32-bit unsigned range ($2^{32}-1$).






\subsectionConstructor{random}{uint}

\begin{galgas3}
constructor random -> @uint
\end{galgas3}

Retourne une valeur aléatoire de type \ggst+@uint+. La procédure de type \refStaticProcPage{uint}{setRandomSeed} permet d'en fixer la valeur initiale.

\begin{galgas3}
  let v = @uint.random
\end{galgas3}


{\bf Note. } Sur Unix, la valeur renvoyée est la valeur renvoyée par l'appel de la fonction \texttt{random} de la librairie \texttt{libc}. Sur Windows, c'est la fonction \texttt{rand} qui est appelée.


\subsectionConstructor{valueWithMask}{uint}

\begin{galgas3}
constructor valueWithMask ?@uint inLowerIndex ?@uint inUpperIndex -> @uint
\end{galgas3}


Returns an \ggst+@uint+ object with bits from \emph{inLowerIndex} to \emph{inUpperIndex} equal to 1.

A run-time error is raised if \emph{inLowerIndex $>$ inUpperIndex} or if \emph{inUpperIndex $>$ 31}.



\textbf{Exemple :}
\begin{galgas3}
@uint x = [@uint valueWithMask !2 !4] ; # x is equal to 28 (0b1_1100)
\end{galgas3}




\subsectionConstructor{warningCount}{uint}

\begin{galgas3}
constructor warningCount -> @uint
\end{galgas3}


Returns an \ggst+@uint+ object that contains the number of warnings. The returned value is the cumulative count of warnings from the beginning of execution.





\section{Procédure de type}


\subsectionStaticProc{setRandomSeed}{uint}


\begin{galgas3box}
proc @uint setRandomSeed ?@uint inSeed
\end{galgas3box}

Affecte la valeur initiale utilisée par le générateur de nombres aléatoires (voir le \refConstructorPage{uint}{random}) Par exemple~:

\begin{galgas3}
  [@uint setRandomSeed !0]
\end{galgas3}






\section{Getters}

\subsectionGetter{alphaString}{uint}

Ce \emph{getter} permet de convertir un \ggst!@uint! en une chaîne de caractères, telle que l'ordre des entiers est conservé sur la chaîne obtenue.

La chaîne obtenue comporte exactement 7 lettres minuscules. C'est en fait une conversion en base 26, la lettre \ggst=a= ayant la valeur $0$, et la lettre \ggst=z= la valeur $25$.


\begin{galgas3}
  message [0 alphaString] + "\n"         # aaaaaaa
  message [12_345 alphaString] + "\n"    # aaaasgv
  message [@uint.max alphaString] + "\n" # nxmrlxv
\end{galgas3}



\subsectionGetter{bigint}{uint}

Ce \emph{getter} permet de convertir un \ggst!@uint! en \ggst!@bigint!. Comme la plage des valeurs des \ggst!bigint! n'est limitée que par la mémoire disponible, il n'échoue jamais.

\begin{galgas3}
  message [[1234 bigint] string] + "\n" # 1234
\end{galgas3}


\subsectionGetter{double}{uint}

\begin{galgas3}
getter double -> @double
\end{galgas3}

Returns the receiver's value converted in a \ggst+@double+ object. As a 32-bit integer value can always be converted in a \ggst+@double+ value, this getter never fails.



\subsectionGetter{hexString}{uint}

\begin{galgas3}
getter hexString -> @string
\end{galgas3}

Returns the an hexadecimal string representation of the receiver value, prefixed by the string \texttt{0x}. For getting an hexadecimal representation string without any prefix, see \refGetterPage{uint}{xString}.



\subsectionGetter{hexStringSeparatedBy}{uint}

\begin{galgas3}
getter hexStringSeparatedBy ?@char inSeparator ?@uint inGroup -> @string
\end{galgas3}

Returns the an hexadecimal string representation of the receiver value, prefixed by the string \texttt{0x}. Groups of \ggst=inGroup= digits are separated by the \ggst=inSeparator= character.

If \ggst=inGroup= is equal to zero, a run-time error is raised.

For example:
\begin{galgas3}
let s = [0x12345678 hexStringSeparatedBy !'_' !2] # 0x12_34_56_78
\end{galgas3}



\subsectionGetter{isInRange}{uint}

\begin{galgas3}
getter isInRange ?@range inRange -> @bool
\end{galgas3}

{Returns an \ggst+@bool+ value indicating whether the receiver'value belongs to \ggst+inRange+ range : for a receiver's value equal to $v$ and a range of length $length$ starting at $start$, it returns \ggst+true+ if $((v \geqslant start)~and~(v<(start+length)))$, and \ggst+false+ otherwise.



\subsectionGetter{isUnicodeValueAssigned}{uint}

\begin{galgas3}
getter isUnicodeValueAssigned -> @bool
\end{galgas3}

Returns an \ggst+@bool+ value indicating whether the receiver'value represents an assigned Unicode character. It returns \ggst+true+ if the receiver value represents an assigned Unicode character, \ggst+false+ and otherwise.

\textbf{Exemple :}
\begin{galgas3}
[0xFFFF isUnicodeValueAssigned] # is false, as \uFFFF is not assigned.
[0x41 isUnicodeValueAssigned] # is true, as \u0041 is assigned (LATIN CAPITAL LETTER A).
\end{galgas3}



\subsectionGetter{lsbIndex}{uint}

\begin{galgas3}
getter lsbIndex -> @uint
\end{galgas3}

Returns an \ggst+@uint+ value of the index of the most significant bit of the receiver value. It raises a run-time error if the receiver value is zero.

\textbf{Exemple :}
\begin{galgas3}
@uint value = 192 ; # 192 is ...011000000 in binary
@uint x = [value lsbIndex] ; # x is equal to 7
\end{galgas3}

The most significant bit of 192 is the 7th bit.




\subsectionGetter{significantBitCount}{uint}

\begin{galgas3}
getter significantBitCount -> @uint
\end{galgas3}

Returns the number of bits needed to express the receiver value. If the receiver value is zero, it returns 0 ; otherwise, it returns the most significant bit index plus one.

\textbf{Exemple :}
\begin{galgas3}
@uint value = 145 ; # 145 is 10010001 in binary
@uint x = [value significantBitCount] ; # x is equal to 8
\end{galgas3}






\subsectionGetter{sint}{uint}

\begin{galgas3}
getter sint -> @sint
\end{galgas3}

Returns the receiver's value in an \refTypePredefini{sint} (32-bit signed integer) object. An error is raised is receiver's value is greater than $2^{31}-1$.

This getter is the only way to convert an \refTypePredefini{uint} object into an \refTypePredefini{sint} object.




\subsectionGetter{sint64}{uint}

\begin{galgas3}
getter sint64 -> @sint64
\end{galgas3}

Returns the receiver's value in an \refTypePredefini{sint64} (64-bit signed integer) object. As a 32-bit unsigned value can always be converted in a 64-bit signed value, this getter never fails.

This getter is the only way to convert an \refTypePredefini{uint} object into an \refTypePredefini{sint64} object.


\subsectionGetter{string}{uint}

\begin{galgas3}
getter string -> @string
\end{galgas3}

Returns a decimal string representation of the receiver's value. For an hexadecimal string representation of the receiver's value, see \refGetterPage{uint}{hexString} and \refGetterPage{uint}{xString}.




\subsectionGetter{uint64}{uint}

\begin{galgas3}
getter uint64 -> @uint64
\end{galgas3}

Returns the receiver's value in an \refTypePredefini{uint64} (64-bit unsigned integer) object. As a 32-bit unsigned value can always be converted in a 64-bit unsigned value, this getter never fails.

This getter is the only way to convert an \refTypePredefini{uint} object into an \refTypePredefini{uint64} object.




\subsectionGetter{xString}{uint}

\begin{galgas3}
getter xString -> @string
\end{galgas3}

Returns an hexadecimal string representation of the receiver's value (without any prefix). For an decimal string representation of the receiver's value, see the \refGetterPage{uint}{hexString}; for a decimal string representation of the receiver's value, see the \refGetterPage{uint}{string}.







\section{Arithmétique}

\subsection{Opérateurs infixés}

Le type \ggst+@uint+ accepte les opérateurs arithmétiques infixés suivants :
\begin{itemize}
  \item \ggst!+!, addition, une erreur d'exécution est déclenchée en cas de débordement ;
  \item \ggst!-!, soustraction, une erreur d'exécution est déclenchée en cas de débordement ;
  \item \ggst!*!, multiplication, une erreur d'exécution est déclenchée en cas de débordement ;
  \item \ggst!/!, division, une erreur d'exécution est déclenchée si le diviseur est nul ;
  \item \ggst!mod!, calcul du reste, une erreur d'exécution est déclenchée si le diviseur est nul ;
  \item \ggst!&+!, addition, le résultat étant silencieusement tronqué sur 32 bits ;
  \item \ggst!&-!, soustraction, le résultat étant silencieusement tronqué sur 32 bits ;
  \item \ggst!&*!, multiplication, le résultat étant silencieusement tronqué sur 32 bits ;
  \item \ggst!&/!, division, qui retourne zéro si le diviseur est nul.
\end{itemize}

Ces opérateurs exigent que les deux opérandes soient des objets du même type \ggst+@uint+.

\subsection{Opérateur préfixé}
Le type \ggst+@uint+ accepte un opérateur arithmétique préfixé :
\begin{itemize}
  \item \ggst!+!, qui retourne simplement la valeur de l'opérande.
\end{itemize}

\subsectionLabel{Instructions}{instructionsUINT}

Le type \ggst+@uint+ accepte les deux instructions arithmétiques suivantes :
\begin{itemize}
  \item \ggst!+=!, addition, une erreur d'exécution est déclenchée en cas de débordement ;
  \item \ggst!-=!, soustraction, une erreur d'exécution est déclenchée en cas de débordement ;
  \item \ggst!*=!, multiplication, une erreur d'exécution est déclenchée en cas de débordement ;
  \item \ggst!/=!, division, une erreur d'exécution est déclenchée en cas division par zéro ;
  \item \ggst!++!, incrémentation, une erreur d'exécution est déclenchée en cas de débordement ;
  \item \ggst!--!, décrémentation, une erreur d'exécution est déclenchée en cas de débordement ;
  \item \ggst!&++!, incrémentation, le résultat étant silencieusement tronqué sur 32 bits ;
  \item \ggst!&--!, décrémentation, le résultat étant silencieusement tronqué sur 32 bits.
\end{itemize}

\ggst!x+=y! est équivalent à \ggst!x=x+y! ; \ggst!x-=y! est équivalent à \ggst!x=x-y!.
La variable cible \ggst!x!, comme l'expression source \ggst!y! doivent être du même type \ggst+@uint+.

Incrémentation et décrémentation sont des instructions, et ne peuvent pas apparaître des expressions.
\begin{galgas3}
@uint n = ... ; n ++ # Incrémentation
\end{galgas3}

\begin{galgas3}
@uint n = ... ; n -- # Décrémentation
\end{galgas3}




\section{Shift Operators}


The \ggst+@uint+ type supports right and left shift operators:\newline

\begin{tabular}{|c|c|}
\hline
$<<$ & Left shift \\
\hline
$>>$ & Right shift \\
\hline
\end{tabular}

Theses operators require both arguments to be \ggst+@uint+ objects.\newline

Note the right shift inserts always a zero bit in the most significant bit location (it is a logical right shift).\newline

The actual amount of the shift is the value of the right-hand operand masked by 31, i.e. the shift distance is always between 0 and 31.




\section{Logical Operators}

The \ggst+@uint+ type supports the three bit-wise logical operators:\newline

\begin{tabular}{|c|c|}
\hline
$\&$ & Bit-wise and \\
\hline
\textbar & Bit-wise or \\
\hline
\^\  & Bit-wise exclusive or \\
\hline
\end{tabular}

Theses operators require both arguments to be \ggst+@uint+ objects.\newline


The \ggst+@uint+ type supports the bit-wise logical unary operator:\newline

\begin{tabular}{|c|c|}
\hline
$\sim$ & Bit-wise complementation \\
\hline
\end{tabular}

This operator returns an \ggst+@uint+ object.







\section{Comparison Operators}

The \ggst+@uint+ type supports the six comparison operators:\newline

\begin{tabular}{|c|c|}
\hline
$=$ & Equality \\
\hline
$!=$ & Non Equality \\
\hline
$<$  & Strict Lower Than \\
\hline
$<=$  & Lower or Equal \\
\hline
$>$  & Strict Greater Than \\
\hline
$>=$  & Greater or Equal \\
\hline
\end{tabular}

\vspace{2mm}
Theses operators require both arguments to be \ggst+@uint+ objects, and return a \ggst+@bool+ object.



  %!TEX encoding = UTF-8 Unicode
%!TEX root = ../galgas-book.tex

\chapitreTypePredefiniLabelIndex{uint64}

An \galgas{@uint64} object value is a 64-bit unsigned integer value. You can initialize an \galgas{@uint64} object from a 64-bit unsigned integer constant:\\

\texttt{@uint64 myUnsignedInteger := 123\_456L ;}\newline

Note the 'L' suffix is required for a 64-bit unsigned integer constant.

\section{Constructeurs}

\constructeurSansArgument{max}
{uint64}
{1.3.0}
{uint64}
{Returns an \galgas{@uint64} object that the maximum value of the 64-bit unsigned range.}
{The returned value is $2^{64}-1$.}


\constructeurUnArgument{uint64BaseValueWithCompressedBitString}
{uint64}
{1.6.4}
{uint64}
{@string inBitString}
{Returns an \galgas{@uint64} object computed from a string containing '0', '1' or 'X' characters, replacing all occurrences of 'X' by '0'.}
{the inBitString argument should contain only '0', '1' or 'X' characters. A run time exception is raised if an other character appears.

This constructor considers the \emph{inBitString} argument value as a binary encoding of an integer value. First, it internally replaces all 'X's by '0's, and then converts the resulting string into an integer value that is the one returned by this constructor.

Note that the first character of the \emph{inBitString} argument value corresponds to the most significant bit of the converted value.}


\textbf{Exemple :}
\begin{galgascode}
@uint64 v [uint64BaseValueWithCompressedBitString !"01XX10"] ;
log v ; # Displays <@uint64:18> ;
\end{galgascode}





\constructeurUnArgument{uint64MaskWithCompressedBitString}
{uint64}
{1.6.4}
{uint64}
{@string inBitString}
{Returns an \galgas{@uint64} object computed from a string containing '0', '1' or 'X' characters, replacing all occurrences of '0' by '1' and all occurrences of 'X' by '0'.}
{the \emph{inBitString} argument should contain only '0', '1' and 'X' characters. A run time exception is raised if an other character appears.

This constructor considers the \emph{inBitString} argument value as a binary encoding of an integer value. First, it internally replaces all '0's by '1's and all 'X's by '0's, and then converts the resulting string into an integer value that is the one returned by this constructor.

Note that the first '0' or '1' character of the \emph{inBitString} argument value corresponds to the most significant Bit of the converted value.}

\textbf{Exemple :}
\begin{galgascode}
@uint64 v [uint64MaskWithCompressedBitString !"01XX10"] ;
log v ; \# Displays <@uint64:51> ;
\end{galgascode}



\constructeurUnArgument{uint64WithBitString}
{uint64}
{1.6.4}
{uint64}
{@string inBitString}
{Returns an \galgas{@uint64} object computed from a string containing '0' or '1' characters.}
{the \emph{inBitString} argument should contain only '0' and '1' characters. A run time exception is raised if an other character appears.

This constructor considers the \emph{inBitString} argument value as a binary encoding of an integer value. It returns an \galgas{@uint64} object containing the converted value.

Note that the first '1' character of the \emph{inBitString} argument value corresponds to the most significant bit of the converted value.}

\textbf{Exemple :}
\begin{galgascode}
@uint64 v [uint64WithBitString !"0101"]] ;
log v ; # Displays <@uint64:5> ;
\end{galgascode}


\section{Readers}

\readerSansArgument{double}
{uint64}
{1.9.8}
{double}
{Returns the receiver's value converted in a \galgas{@double} object.}
{as a 64-bit integer value can always be converted in a \galgas{@double} value, this reader never fails.}



\readerSansArgument{hexString}
{uint64}
{1.5.2}
{string}
{Returns the an hexadecimal string representation of the receiver value, prefixed by the string "0x".}
{for getting an hexadecimal representation string without "0x" prefix, see \refReaderPage{uint64}{xString}.}





\readerSansArgument{sint}
{uint64}
{1.6.12}
{sint}
{Returns the receiver's value in an \refTypePredefini{sint} (32-bit signed integer) object.}
{an error is raised is receiver's value is greater than $2^{31}-1$.}

This reader is the only way to convert an \refTypePredefini{uint64} object into an \refTypePredefini{sint} object.




\readerSansArgument{sint64}
{uint64}
{1.6.12}
{sint64}
{Returns the receiver's value in an \refTypePredefini{sint64} (64-bit signed integer) object.}
{an error is raised is receiver's value is greater than $2^{63}-1$.}

This reader is the only way to convert an \refTypePredefini{uint64} object into an \refTypePredefini{sint64} object.


\readerSansArgument{string}
{uint64}
{1.6.12}
{string}
{Returns a decimal string representation of the receiver's value.}
{for an hexadecimal string representation of the receiver's value, see \refReaderPage{uint64}{hexString} and \refReaderPage{uint64}{xString}.}



\readerSansArgument{uint}
{uint64}
{1.6.12}
{uint}
{Returns the receiver's value in an \refTypePredefini{uint} (32-bit unsigned integer) object.}
{an error is raised is receiver's value is greater than $2^{32}-1$.}

This reader is the only way to convert an \refTypePredefini{uint64} object into an \refTypePredefini{uint} object.


\readerDeuxArguments{uintSlice}
{uint64}
{1.6.0}
{uint}
{@uint inStartBit}
{@uint inBitCount}
{Returns an \refTypePredefini{uint} value, extracted from a bit slice of the receiver's value.}
{the receiver's value is right shifted by \emph{inStartBit}, and the resulted value is and'ed with a mask equal to $2^{inBitCount}-1$.}


\textbf{Exemple :}
\begin{galgascode}
@uint64 v := 0x1234_5678_9ABC_DEF0L ;
@uint result := [v uintSlice !4 !5] ; # The result value is 0x8_9ABC
\end{galgascode}




%\defReaderSansArgument{xString}{uint64}
%{1.9.10}
%{string}
%{Returns an hexadecimal string representation of the receiver's value (without any prefix).}
%{for an decimal string representation of the receiver's value, see the \refReaderPage{uint64}{hexString}; for a decimal string representation of the receiver's value, see the \refReaderPage{uint64}{string}.}
%{}{}

\readerSansArgument{xString}
{uint64}
{1.9.10}
{string}
{Returns an hexadecimal string representation of the receiver's value (without any prefix).}
{for an decimal string representation of the receiver's value, see the \refReaderPage{uint64}{hexString}; for a decimal string representation of the receiver's value, see the \refReaderPage{uint64}{string}.}






\section{Incrementation and decrementation}

The \refTypePredefini{uint64} supports incrementation and decrementation instructions.

\texttt{@uint64 n := ... ; n ++ ; \# Incrementation}

\texttt{@uint64 p := ... ; p -- ; \# Decrementation}\newline

The incrementation instruction raises an error if receiver's value is equal to $2^{64}-1$.\newline

The incrementation instruction raises an error if receiver's value is equal to 0.\newline

Note that incrementation and decrementation are not available within an expression.




\section{Arithmetic Operators}

The \galgas{@uint64} type supports the five arithmetic diadic operators:\newline

\begin{tabular}{|c|c|}
\hline
$+$ & Addition \\
\hline
$-$ & Substraction \\
\hline
$*$ & Multiplication \\
\hline
$/$ & Division \\
\hline
\galgas{mod} & Modulo \\
\hline
\end{tabular}

Theses operators require both arguments to be \galgas{@uint64} objects.\newline

A run-time error is raised if the operation leads to a 64-bit unsigned overflow.

The \galgas{@uint64} type supports the following arithmetic unary operator:\newline

\begin{tabular}{|c|c|}
\hline
$+$ & No operation \\
\hline
\end{tabular}

This operator returns the receiver's value (an  \galgas{@uint64} object).




\section{Shift Operators}


The \galgas{@uint} type supports right and left shift operators:\newline

\begin{tabular}{|c|c|}
\hline
$<<$ & Left shift \\
\hline
$>>$ & Right shift \\
\hline
\end{tabular}

Theses operators require the left argument to be \galgas{@uint64} object, and  the right argument to be \galgas{@uint} object.\newline

Note the right shift inserts always a zero bit in the most significant bit location (it is a logical right shift).\newline

The actual amount of the shift is the value of the right-hand operand masked by 63, i.e. the shift distance is always between 0 and 63.




\section{Logical Operators}

The \galgas{@uint64} type supports the three bit-wise logical diadic operators:

\begin{tabular}{|c|c|}
\hline
$\&$ & Bit-wise and \\
\hline
\textbar & Bit-wise or \\
\hline
\^\  & Bit-wise exclusive or \\
\hline
\end{tabular}

Theses operators require both arguments to be \galgas{@uint64} objects.\newline


The \galgas{@uint64} type supports the bit-wise logical unary operator:

\begin{tabular}{|c|c|}
  \hline
  $\sim$ & Bit-wise complementation \\
  \hline
\end{tabular}

This operator returns an \galgas{@uint64} object.




\section{Comparison Operators}

The \galgas{@uint64} type supports the six comparison operators:

\begin{tabular}{|c|c|}
\hline
$=$ & Equality \\
\hline
$!=$ & Non Equality \\
\hline
$<$  & Strict Lower Than \\
\hline
$<=$  & Lower or Equal \\
\hline
$>$  & Strict Greater Than \\
\hline
$>=$  & Greater or Equal \\
\hline
\end{tabular}

Theses operators require both arguments to be \galgas{@uint64} objects, and return a \galgas{@bool} object.

  %!TEX encoding = UTF-8 Unicode
%!TEX root = ../galgas-book.tex

%--------------------------------------------------------------
\chapter{List Type}
%-------------------------------------------------------------

\section{List Type Declaration}

A \lstinline[language=galgas]!list! type declaration names all attributes of the list elements:

\begin{lstlisting}[language=galgas]
list @MyList {
  @string mFirstAttribute ;
  @bool mSecondAttribute ;
}
\end{lstlisting}

\section{Constructors}

\subsection{The \lstinline[language=galgas]!emptyList! constructor}

For every list, an \lstinline[language=galgas]!emptyList! constructor is implicitly declared. It returns an empty list:

\begin{lstlisting}[language=galgas]
@MyList aList := [@MyList emptyList] ;
\end{lstlisting}


\subsection{The \lstinline[language=galgas]!listWithValue! constructor}

A list can be constructed directly with one value:

\begin{lstlisting}[language=galgas]
@MyList aList := [@myList listWithValue !"c" !3] ;
\end{lstlisting}


Using this constructor is equivalent to:

\begin{lstlisting}[language=galgas]
@MyList aList := [@MyList emptyList] ;
aList += !"c" !3 ;
\end{lstlisting}

\section{Adding elements}

\subsection{The \lstinline[language=galgas]!+=! operator}

The  \lstinline[language=galgas]!+=! operator adds a new element at the end of the list. The right side expressions should correspond to the attributes declared in the \lstinline[language=galgas]!list! declaration:\\

\begin{lstlisting}[language=galgas]
@MyList aList := ... ;
@string aString := ... ;
@bool aBool := ... ;
aList += !aString !aBool ;''
\end{lstlisting}


\subsection{The \lstinline[language=galgas]!.=! operator}

The \lstinline[language=galgas]!.=! operator concats a list at the end of the target list:

\begin{lstlisting}[language=galgas]
@MyList aList := ... ;
@MyList secondList := ... ;
aList .= secondList ;''
\end{lstlisting}



\subsection{The \lstinline[language=galgas]!prependValue! modifier}

The \lstinline[language=galgas]!prependValue! modifier adds a new element at the begining of the list. The right side expressions should correspond to the attributes declared in the  \lstinline[language=galgas]!list! declaration:

\begin{lstlisting}[language=galgas]
@MyList aList := ... ;
@string aString := ... ;
@bool aBool := ... ;
[!?aList prependValue !aString !aBool];
\end{lstlisting}

\subsection{The concatenation operator}

The «~\lstinline[language=galgas]!.!~» operator can be used fot concatenating two lists of the same type:


\begin{lstlisting}[language=galgas]
@MyList firstList := ... ;
@MyList secondList := ... ;
@MyList thirdList := firstList . secondList ;
\end{lstlisting}

\section{Removing elements}

\subsection{The \lstinline[language=galgas]!popFirst! modifier}


The \lstinline[language=galgas]!popFirst! modifier removes and returns the first element of the list. The right side expressions should correspond to the attributes declared in the \lstinline[language=galgas]!list! declaration:\\

\begin{lstlisting}[language=galgas]
@MyList aList := ... ;
@string aString ;
@bool aBool ;
[!?aList popFirst ?aString ?aBool];
\end{lstlisting}

If the list is empty when \lstinline[language=galgas]!popFirst! modifier is invoked, a run-time error is raised and the input arguments are not valuated.

\subsection{The \lstinline[language=galgas]!popLast! modifier}


The \lstinline[language=galgas]!popLast! modifier removes and returns the last element of the list. The right side expressions should correspond to the attributes declared in the \lstinline[language=galgas]!list! declaration:

\begin{lstlisting}[language=galgas]
@MyList aList := ... ;
@string aString ;
@bool aBool ;
[!?aList popLast ?aString ?aBool];
\end{lstlisting}

If the list is empty when \lstinline[language=galgas]!popLast! is invoked, a run-time error is raised and the input arguments are not valuated.

\section{Methods}

\subsection{The \lstinline[language=galgas]!first! method}

The \lstinline[language=galgas]!first! method returns the first element of the list. The element is not removed. The right side expressions should correspond to the attributes declared in the \lstinline[language=galgas]!list! declaration:

\begin{lstlisting}[language=galgas]
@MyList aList := ... ;
@string aString ;
@bool aBool ;
[aList first ?aString ?aBool];
\end{lstlisting}

If the list is empty when \lstinline[language=galgas]!first! is invoked, a run-time error is raised and the input arguments are not valuated.

\subsection{The \lstinline[language=galgas]!last! method}

The \lstinline[language=galgas]!last! method returns the last element of the list. The element is not removed. The right side expressions should correspond to the attributes declared in the \lstinline[language=galgas]!list! declaration:\\

\begin{lstlisting}[language=galgas]
@MyList aList := ... ;
@string aString ;
@bool aBool ;
[aList last ?aString ?aBool];
\end{lstlisting}


If the list is empty when \lstinline[language=galgas]!last! is invoked, a run-time error is raised and the input arguments are not valuated.








\section{Readers}

\subsection{The \lstinline[language=galgas]!length! reader}

\begin{lstlisting}[language=galgas]
reader length -> @uint ;
\end{lstlisting}

The \lstinline[language=galgas]!length! reader returns the number of elements in the receiver's value.


\subsection{The \lstinline[language=galgas]!range! reader}

\begin{lstlisting}[language=galgas]
reader range -> @range ;
\end{lstlisting}

The \lstinline[language=galgas]!range! reader returns a range starting at $0$ of length equal to the number of elements in the receiver's value.




\subsection{The \lstinline[language=galgas]!subListFromIndex! reader}

\begin{lstlisting}[language=galgas]
reader subListFromIndex ?@uint inIndex -> @self
\end{lstlisting}

This reader returns a new list containing the elements of the receiver from the one at a given index to the end. The  \lstinline[language=galgas]!inIndex! value should be lower or equal to the length of the receiver's value. If \lstinline[language=galgas]!inIndex! is equal to the length of the receiver, the reader returns an empty list.


\subsection{The \lstinline[language=galgas]!subListWithRange! reader}

\begin{lstlisting}[language=galgas]
reader subListWithRange
  ?@range inRange
  -> @self
\end{lstlisting}

This reader returns a list containing the elements of the receiver that lie within a given range. The range must not exceed the length of the receiver's value, that is $range\_start + range\_length \leqslant list\_length$. If the range's length is equal to zero, this reader returns an empty list.





\section{Enumerating a list with a foreach instruction}

The \lstinline[language=galgas]!foreach! instruction can be used for enumerating list objects. By default, lists are enumerated in the insertion order; enumeration in the reverse order is performed using the «~\lstinline[language=galgas]!>!~» qualifier.

There are two ways for accessing element values:
\begin{itemize}
\item using the implicitly declared constants that receive the current attribute values;
\item declare explicitly constants that receive the current attribute values.
\end{itemize}

Given the list declaration:

\begin{lstlisting}[language=galgas]
list @MyList {
  @string mFirstAttribute ;
  @bool mSecondAttribute ;
}
\end{lstlisting}

\subsection{Enumeration using the implicitly declared constants}

For every attribute, a constant of the same name is available in the \lstinline[language=galgas]!do! instruction list. Theses constants receive the value of the corresponding attribute of the current element.

\begin{lstlisting}[language=galgas]
foreach aList do
  # the mFirstAttribute constant receives the value
  # of the mFirstAttribute attribute of the current element,
  # and the mSecondAttribute constant receives the value
  # of the mSecondAttribute attribute of the current element.
end foreach ;
\end{lstlisting}

\subsection{Enumeration using the explicitly declared constants}

The \lstinline[language=galgas]!foreach! header declares a sequence of constants, corresponding to the attribute list of the \lstinline[language=galgas]!do! declaration. Theses constants receive the value of the corresponding attribute of the current element.


\begin{lstlisting}[language=galgas]
foreach aList (@string kString @bool kBool) do
  # the kString constant receives the value
  # of the mFirstAttribute attribute of the current element,
  # and the kBool constant receives the value
  # of the mSecondAttribute attribute of the current element.
end foreach ;
\end{lstlisting}

\subsection{Enumeration in the reverse order}

In GALGAS 1.7.3 and later, you can enumerate a list in the reverse order using the «~\lstinline[language=galgas]!>!~» qualifier:

\begin{lstlisting}[language=galgas]
foreach > aList (@string kString @bool kBool) do
  ...
end foreach ;
\end{lstlisting}




\section{Direct Access of an element attribute}

In GALGAS 1.7.5 and later, lists can be used as an array. Each element of a list is associated with an \nomType{uint} index, spanning from 0 to element count (value returned by \lstinline[language=galgas]!length! reader) minus one.

The element retrieved with \lstinline[language=galgas]!first! method is at index 0.

The element retrieved with \lstinline[language=galgas]!last! method is at index equal to element count minus one.

\subsection{Read Access}

By default and for every attribute, a reader is provided to retrieve the value of this attribute for an element at a given index. For example, for an attribute named \emph{name}, the \emph{nameAtIndex} reader is provided. It accepts one \nomType{uint} argument, the value of the index.

You can disable the default reader generation, by using the «~\lstinline[language=galgas]!feature nogetter!~» qualifier.

For example:
\begin{lstlisting}[language=galgas]
list @MyList {
  @string mFirstAttribute ;
  @bool mSecondAttribute feature nogetter ;
}
...
@MyList aList := ... ;
@string s := [aList mFirstAttributeAtIndex !1] ;
\end{lstlisting}

One reader is available: \lstinline[language=galgas]!mFirstAttributeAtIndex!; the \lstinline[language=galgas]!mSecondAttributeAtIndex! reader is not available.


\subsection{Write Access}

By default, no modifier is provided for performing a direct write access to an attribute at a given index. You should use the «~\lstinline[language=galgas]!feature setter!~» qualifier for enabling setter generation for a given attribute.

The modifier name is the name of the attribute with the first letter capitalized, prefixed by \emph{set} and suffixed by \emph{AtIndex}: for an attribute named \emph{name}, the modifier is named \emph{setNameAtIndex}. It accepts two arguments, the first one is the new attribute's value, the second one an \nomType{uint} argument, the value of the index.

For example:

\begin{lstlisting}[language=galgas]
list @MyList {
  @string mFirstAttribute feature setter ;
  @bool mSecondAttribute ;
}
...
@string s := ... ;
[!?aList setMFirstAttributeAtIndex !s !1] ;
\end{lstlisting}

One modifier is available: \lstinline[language=galgas]!setMFirstAttributeAtIndex!; the \lstinline[language=galgas]!setMSecondAttributeAtIndex! modifier is not available.

\subsection{Example of read and write accesses}

\begin{lstlisting}[language=galgas]
list @myList {
  @string name ;
}
...
@myList strList [emptyList] ;
strList += !"a" ;
strList += !"b" ;
strList += !"c" ;
strList += !"d" ;
@string s := [strList nameAtIndex !0] ;
log s ; # displays LOGGING s: <@string:"a">
s := [strList nameAtIndex !1] ;
log s ; # displays LOGGING s: <@string:"b">
s := [strList nameAtIndex !2] ;
log s ; # displays LOGGING s: <@string:"c">
s := [strList nameAtIndex !3] ;
log s ; # displays LOGGING s: <@string:"d">
[!?strList setNameAtIndex !"x" !0] ;
[!?strList setNameAtIndex !"y" !1] ;
[!?strList setNameAtIndex !"z" !2] ;
[!?strList setNameAtIndex !"t" !3] ;
s := [strList nameAtIndex !0] ;
log s ; # displays LOGGING s: <@string:"x">
s := [strList nameAtIndex !1] ;
log s ; # displays LOGGING s: <@string:"y">
s := [strList nameAtIndex !2] ;
log s ; # displays LOGGING s: <@string:"z">
s := [strList nameAtIndex !3] ;
log s ; # displays LOGGING s: <@string:"t">
\end{lstlisting}

  %!TEX encoding = UTF-8 Unicode
%!TEX root = ../galgas-book.tex

%--------------------------------------------------------------
\chapter{Le type \texttt{sortedlist}}
%-------------------------------------------------------------

Le type \ggs+sortedlist+ permet de construire des listes ordonnées de valeurs.




\section{Déclaration}

La déclaration d'une \ggs+sortedlist+ nomme tous les attributs qui composent un élément de liste et la description du tri. Par Exemple :

\begin{galgascode}
sortedlist @MaListeOrdonnee {
  @char mCaractere ;
  @uint mEntier ;
}{
  mCaractere <, mEntier >
}
\end{galgascode}

La description du tri est exprimée par la liste ordonnée des attributs qui interviennent dans le tri, chacun d'eux étant suivi de l'ordre du tri (\ggs+<+ pour croissant, et \ggs+>+ pour décroissant). Ainsi, les élements des instances du type liste ordonnée ci-dessus sont triés par ordre croissant du champ caractère, puis par ordre décroissant du champ entier.

Déclarer une \ggs+sortedlist+ définit implicitement :
\begin{itemize}
  \item le constructeur \ggs+emptySortedList+ qui construit une liste vide (\refSubsectionPage{constructeurSortedlistEmptySortedList}) ;
  \item le constructeur \ggs+sortedListWithValue+ qui construit une liste contenant un élément (\refSubsectionPage{constructeurSortedlistSortedListWithValue}) ;
  \item l'opérateur \ggs*+=* pour ajouter un élément à une liste ordonnée (\refSubsectionPage{operateurSortedListPlusEgal}) ;
  \item l'opérateur \ggs*+=* pour ajouter tous les éléments d'une liste à une liste ordonnée (\refSubsectionPage{operateurSortedListPointEgal}) ;
  \item l'opérateur \ggs*+* pour construire une liste ordonnée à partir de deux listes ordonnées (\refSubsectionPage{operateurSortedListPoint}) ;
  \item le \emph{getter} \ggs+length+, qui retourne le nombre d'éléments d'une liste (\refSectionPage{readerSortedListLength}) ;
  \item le \emph{setter} \ggs+popGreatest+, qui retourne les champs du plus grand élément d'une liste, et retire cet élément de cette liste (\refSubsectionPage{modifierSortedListPopGreatest}) ;
  \item le \emph{setter} \ggs+popSmallest+, qui retourne les champs du plus grand élément d'une liste, et retire cet élément de cette liste (\refSubsectionPage{modifierSortedListPopSmallest}) ;
  \item la \emph{méthode} \ggs+greatest+, qui retourne les champs du plus grand élément d'une liste sans la modifier (\refSubsectionPage{methodeSortedListGreatest}) ;
  \item la \emph{méthode} \ggs+smallest+, qui retourne les champs du plus petit élément d'une liste sans la modifier (\refSubsectionPage{methodeSortedListSmallest}).
\end{itemize}








\section{Constructeurs}

\subsectionLabel{Constructeur \texttt{emptySortedList}}{constructeurSortedlistEmptySortedList}

Le constructeur \ggs+emptySortedList+ construit et retourne une liste vide. Par exemple :
\begin{galgascode}
@MaListeOrdonnee uneListe [emptySortedList] ;
\end{galgascode}


\subsectionLabel{Constructeur \texttt{sortedListWithValue}}{constructeurSortedlistSortedListWithValue}

Le constructeur \ggs+sortedListWithValue+ construit et retourne une liste comprenant un élément. Cet élément est spécifié par les arguments effectifs de l'appel : ce constructeur présente une séquence d'arguments en entrée correspondant aux champs de l'élément. Par exemple :

\begin{galgascode}
@MaListeOrdonnee uneListe [sortedListWithValue
  !'a' # Affecte au champ mCaractere
  !10  # Affecte au champ mEntier
] ;
\end{galgascode}






\section{Opérateurs}


\subsectionLabel{L'opérateur \texttt{+=}}{operateurSortedListPlusEgal}

L'opérateur \ggs*+=* ajoute un élément à la liste ordonnée, en maintenant la relation d'ordre. L'élément ajouté est spécifié par la séquences des valeurs à affecter à ses champs. Si il y a un ou plusieurs éléments égaux à l'élément ajouté, ce dernier est placé après les éléments existants. 


Cette opération est effectuée en $O(log (n))$ où $n$ est le nombre d'éléments de la liste.

Exemple :

\begin{galgascode}
@MaListeOrdonnee uneListe [emptySortedList] ;
uneListe += !'b' ! 1 ; # b1
uneListe += !'b' ! 2 ; # b2
uneListe += !'d' ! 1 ; # d1
uneListe += !'f' ! 1 ; # f1
uneListe += !'a' ! 1 ; # a1
uneListe += !'c' ! 1 ; # c1
uneListe += !'f' ! 2 ; # f2
\end{galgascode}

\subsectionLabel{L'opérateur \texttt{.=}}{operateurSortedListPointEgal}

L'opérateur \ggs*+=* ajoute tous les éléments de l'expression à la liste ordonnée, en maintenant la relation d'ordre. Si il y a un ou plusieurs éléments égaux à chaque élément ajouté, ce dernier est placé après les éléments existants. 

Exemple :
\begin{galgascode}
@MaListeOrdonnee uneListe = ... ;
@MaListeOrdonnee autreListe = ... ;
uneListe .= autreListe ;
\end{galgascode}

\subsectionLabel{L'opérateur \texttt{.}}{operateurSortedListPoint}

L'opérateur \ggs*+* combine deux listes ordonnées. Les éléments de la seconde liste égaux à ceux de la première liste sont placés après ceux de la première liste.

Exemple :
\begin{galgascode}
@MaListeOrdonnee uneListe = ... ;
@MaListeOrdonnee autreListe = ... ;
@MaListeOrdonnee troisiemeListe = uneListe . autreListe ;
\end{galgascode}







\sectionLabel{Getter \texttt{length}}{readerSortedListLength}

Le getter \ggs+length+ retourne un \ggs+@uint+ contenant le nombre d'éléments de la liste ordonnée.






\section{Setters}

\subsectionLabel{Setter \texttt{popGreatest}}{modifierSortedListPopGreatest}

Ce \emph{setter} retourne les champs du plus grand élément de la liste ordonnée, et le retire. Si la liste est vide, un message d'erreur est affiché, et les variables destinées à recevoir les valeurs des champs sont placées dans l'état \emph{invalide}. Par exemple :

\begin{galgascode}
@MaListeOrdonnee uneListe = ... ;
...
[!?uneListe popGreatest
  ?@char c
  ?@uint n
] ;
\end{galgascode}

Si \ggs+uneListe+ est vide, les variables \ggs+c+ et \ggs+n+ sont placées dans l'état \emph{invalide}.


\subsectionLabel{Setter \texttt{popSmallest}}{modifierSortedListPopSmallest}

Ce \emph{setter} retourne les champs du plus petit élément de la liste ordonnée, et le retire. Si la liste est vide, un message d'erreur est affiché, et les variables destinées à recevoir les valeurs des champs sont placées dans l'état \emph{invalide}. Par exemple :

\begin{galgascode}
@MaListeOrdonnee uneListe = ... ;
...
[!?uneListe popSmallest
  ?@char c
  ?@uint n
] ;
\end{galgascode}

Si \ggs+uneListe+ est vide, les variables \ggs+c+ et \ggs+n+ sont placées dans l'état \emph{invalide}.










\section{Méthodes}

\subsectionLabel{La méthode \texttt{greatest}}{methodeSortedListGreatest}

Cette méthode retourne les champs du plus grand élément de la liste ordonnée, sans le retirer. La liste n'est donc pas modifiée. Si elle est vide, un message d'erreur est affiché, et les variables destinées à recevoir les valeurs des champs sont placées dans l'état \emph{invalide}. Par exemple :

\begin{galgascode}
@MaListeOrdonnee uneListe = ... ;
...
[uneListe greatest
  ?@char c
  ?@uint n
] ;
\end{galgascode}

Si \ggs+uneListe+ est vide, les variables \ggs+c+ et \ggs+n+ sont placées dans l'état \emph{invalide}.


\subsectionLabel{La méthode \texttt{smallest}}{methodeSortedListSmallest}

Cette méthode retourne les champs du plus petit élément de la liste ordonnée, sans le retirer. La liste n'est donc pas modifiée. Si elle est vide, un message d'erreur est affiché, et les variables destinées à recevoir les valeurs des champs sont placées dans l'état \emph{invalide}. Par exemple :

\begin{galgascode}
@MaListeOrdonnee uneListe = ... ;
...
[uneListe smallest
  ?@char c
  ?@uint n
] ;
\end{galgascode}

Si \ggs+uneListe+ est vide, les variables \ggs+c+ et \ggs+n+ sont placées dans l'état \emph{invalide}.




\section{Énumération avec l'instruction \texttt{for}}

L'instruction \ggs+for+ (\refSectionPage{instructionFor}) permet d'énumérer les éléments d'une liste ordonnée, par ordre croissant ou décroissant.

Pour effectuer l'énumération par ordre croissant, écrire :
\begin{galgascode}
foreach uneListe do
  ...
end foreach ;
\end{galgascode}

Pour effectuer l'énumération par ordre décroissant, écrire :
\begin{galgascode}
foreach > uneListe do
  ...
end foreach ;
\end{galgascode}

À l'intérieur de la boucle, pour chaque champ des éléments de la liste, un constante dont le nom est celui du champ est définie et prend la valeur du champ correspondant de l'élément courant.

Par exemple :

\begin{galgascode}
@MaListeOrdonnee uneListe [emptySortedList] ;
uneListe += !'b' ! 1 ; # b1
uneListe += !'b' ! 2 ; # b2
uneListe += !'d' ! 1 ; # d1
uneListe += !'f' ! 1 ; # f1
uneListe += !'a' ! 1 ; # a1
uneListe += !'c' ! 1 ; # c1
uneListe += !'f' ! 2 ; # f2
@string s = "" ;
foreach uneListe do
  s .= [mCaractere string] . [mEntier string] . " " ;
end foreach ;
message s . "\n" ; # Affiche "a1 b2 b1 c1 d1 f2 f1"
s = "" ;
foreach > uneListe do
  s .= [mCaractere string] . [mEntier string] . " " ;
end foreach ;
message s . "\n" ; # Affiche "f1 f2 d1 c1 b1 b2 a1"
\end{galgascode}

  %!TEX encoding = UTF-8 Unicode
%!TEX root = ../galgas-book.tex

%--------------------------------------------------------------
\chapter{Le type \texttt{array}}
%-------------------------------------------------------------

Le type \emph{array} permet de réaliser des tableaux dont la dimension et le type de l'élément sont fixés à la compilation.

\section{Déclaration d'un type tableau}

La déclaration d'un type tableau contient les informations suivantes :
\begin{itemize}
  \item le type \galgas{@TypeElement} qui cite le type de l'élément de tableau ;
  \item la dimension du tableau, qui doit être un nombre entier strictement positif ;
  \item le type \galgas{@TypeTableau} qui est le nom donné au type de tableau.
\end{itemize}

La déclaration d'un type tableau a la syntaxe suivante :
\begin{lstlisting}[language=galgas]
array @TypeTableau : @TypeElement [dimension] ;
\end{lstlisting}

Par exemple :
\begin{lstlisting}[language=galgas]
array @monTableau : @string [3] ;
\end{lstlisting}


\section{Constructeur d'un type tableau}

Le seul constructeur d'un type tableau est le constructeur \galgas{new}. Il a pour but de fixer les dimensions initiales du tableau (il pourra ensuite être redimensionné). Il comporte \emph{dimension} arguments de type \galgas{@uint}, qui fixent la taille initiale de chaque axe.
Par exemple :
\begin{lstlisting}[language=galgas]
  @monTableau t [new !2 !3 !4] ;
\end{lstlisting}

Cette déclaration crée un tableau à $2*3*4$ éléments. Ces éléments sont par défaut \emph{invalides}, c'est à dire que leur lecture par le getter \galgas{valueAtIndex} déclenche une \emph{run-time error}. Pour être valide, un élément doit avoir été initialisé par un appel au setter \galgas{setValueAtIndex}.

Il est valide d'affecter la valeur $0$ à un ou plusieurs axes. Le tableau ne contient alors aucun élément.


\section{Accès à un élément}

L'accès à la valeur d'un élément s'effectue par le getter \galgas{valueAtIndex}. La modification de la valeur d'un élément est réalisée par le setter \galgas{setValueAtIndex} ou le setter \galgas{forceValueAtIndex}.

\subsection{Le getter \texttt{valueAtIndex}}

Ce getter comporte \emph{dimension} arguments de type \galgas{@uint}, qui précisent l'indice pour chaque axe. Les indices sont comptés à partir de zéro (comme en C).

Par exemple :
\begin{lstlisting}[language=galgas]
  @string s := [t valueAtIndex !1 !2 !2] ;
\end{lstlisting}


Une \emph{run-time error} est déclenchée si un indice dépasse sa borne correspondante, et la valeur retournée est \emph{invalide}. Si les indices ont des valeurs correctes, l'élément est retourné ; si cet élément est invalide, une \emph{run-time error} est déclenchée, et une valeur \emph{invalide} est retournée.






\subsection{Setter \texttt{setValueAtIndex}}

Ce setter comporte (\emph{dimension}+1) arguments :
\begin{itemize}
  \item le premier argument est type \galgas{@TypeElement}, et contient la valeur à écrire ;
  \item les \emph{dimension} suivants arguments sont de type \galgas{@uint} et précisent l'indice pour chaque axe.
\end{itemize} 
  
Les indices sont comptés à partir de zéro (comme en C). Une \emph{run-time error} est déclenchée si un indice dépasse sa borne correspondante, et le tableau est alors non modifié.

Par exemple :
\begin{lstlisting}[language=galgas]
  @string s := ... ;
  [!?t setValueAtIndex !s !1 !2 !2] ;
\end{lstlisting}





\subsection{Setter \texttt{forceValueAtIndex}}

Ce setter comporte (\emph{dimension}+1) arguments :
\begin{itemize}
  \item le premier argument est type \galgas{@TypeElement}, et contient la valeur à écrire ;
  \item les \emph{dimension} suivants arguments sont de type \galgas{@uint} et précisent l'indice pour chaque axe.
\end{itemize} 
  
Les indices sont comptés à partir de zéro (comme en C). Contrairement au setter \galgas{setValueAtIndex}, aucune \emph{run-time error} n'est déclenchée si un indice dépasse sa borne correspondante : le tableau est d'abord agrandi, ce qui ajoute des éléments invalides, puis l'élément désigné par les indices est affecté.

Par exemple :
\begin{lstlisting}[language=galgas]
  @string s := ... ;
  [}?t forceValueAtIndex !s !5 !4 !4] ;
\end{lstlisting}





\section{Validité d'un élément}

Le getter \galgas{isValueValidAtIndex} permet de savoir si un élément est valide ou non, c'est à dire si sa lecture déclenchera une \emph{run-time error}. Le setter \galgas{invalidateValueAtIndex} invalide un élément.

\subsection{Le getter \texttt{isValueValidAtIndex}}

Ce getter comporte \emph{dimension} arguments de type \galgas{@uint}, qui précisent l'indice pour chaque axe. Les indices sont comptés à partir de zéro (comme en C). Une \emph{run-time error} est déclenchée si un indice dépasse sa borne correspondante, et la valeur retournée est \emph{invalide}. Il renvoie une valeur de type \galgas{@bool}, suivant que l'élément est valide ou non.

Par exemple :
\begin{lstlisting}[language=galgas]
  @bool b := [t isValueValidAtIndex !1 !2 !2] ;
\end{lstlisting}


\subsection{Setter \texttt{invalidateValueAtIndex}}

Ce setter comporte \emph{dimension} arguments de type \galgas{@uint}, qui précisent l'indice pour chaque axe. Les indices sont comptés à partir de zéro (comme en C). Une \emph{run-time error} est déclenchée si un indice dépasse sa borne correspondante. Il invalide l'élément correspondant, c'est dire qu'un appel au getter \galgas{valueAtIndex} pour lire cet élément déclenchera une \emph{run-time error}.

Par exemple :
\begin{lstlisting}[language=galgas]
  [!?t invalidateValueAtIndex !1 !2 !2] ;
\end{lstlisting}





\section{Contrôle des tailles des axes}

Le getter \galgas{axisCount} renvoie la dimension d'un tableau, c'est à dire le nombre de ces axes, le getter \galgas{sizeForAxis} renvoie la taille allouée à un axe particulier. Les setters \galgas{setSizeForAxis} et \galgas{setSize} permettent de modifier la taille d'un tableau.



\subsection{Le getter \texttt{axisCount}}

Ce getter sans argument renvoie un \galgas{@uint} qui contient le nombre d'axes d'un tableau. Comme ce nombre est fixé statiquement par la déclaration de type, la valeur retournée est toujours la même, pour toutes les objets d'un même type tableau.


Par exemple, pour la déclaration :
\begin{lstlisting}[language=galgas]
array @monTableau : @string [3] ;
\end{lstlisting}
Pour tous les objets de type \galgas{@monTableau}, l'appel au getter \galgas{axisCount} renvoie la valeur $3$.


\subsection{Le getter \texttt{sizeForAxis}}

Ce getter présente un argument de type \galgas{@uint} qui est l'indice de l'axe interrogé. Les axes sont numérotés à partir de zéro, c'est à dire que le premier axe a l'indice $0$, le deuxième l'indice $1$, \dots~Une \emph{run-time error} est déclenchée si la valeur de l'argument est supérieure ou égale à la dimension du tableau, et la valeur renvoyée est invalide. Sinon, il renvoie un \galgas{@uint} qui contient la taille attribuée à l'axe correspondant.


\subsection{Le getter \texttt{rangeForAxis}}

Ce getter présente un argument de type \galgas{@uint} qui est l'indice de l'axe interrogé. Les axes sont numérotés à partir de zéro, c'est à dire que le premier axe a l'indice $0$, le deuxième l'indice $1$, \dots~Une \emph{run-time error} est déclenchée si la valeur de l'argument est supérieure ou égale à la dimension du tableau, et la valeur renvoyée est invalide. Sinon, il renvoie un \galgas{@range} qui commence à $0$ et qui a pour longueur la taille attribuée à l'axe correspondant.




\subsection{Setter \texttt{setSizeForAxis}}

Ce setter permet de changer la taille d'un axe sans changer les tailles attribuées aux autres axes. Il présente deux arguments de type \galgas{@uint} :
\begin{itemize}
  \item le premier est la nouvelle taille ;
  \item le second est l'indice de l'axe concerné.
\end{itemize}

Les axes sont numérotés à partir de zéro, c'est à dire que le premier axe a l'indice $0$, le deuxième l'indice $1$, \dots~Une \emph{run-time error} est déclenchée si la valeur de l'argument est supérieure ou égale à la dimension du tableau, et le tableau n'est pas modifié.
 
Diminuer la taille d'un axe fait disparaître des éléments, qui sont alors perdus. Si la nouvelle taille est zéro, le tableau est vidé de tous ses éléments.

Augmenter la taille fait apparaître de nouveaux éléments, qui sont invalides par défaut. Il faudra alors explicitement les initialiser individuellement par un appel au setter \galgas{setValueAtIndex}.




\subsection{Setter \texttt{setSize}}

Ce setter permet de changer les tailles de tous les axes. Il présente \galgas{@uint} arguments de type \galgas{@uint} qui contiennent les nouvelles tailles de chaque axe.

Diminuer la taille d'un axe fait disparaître des éléments, qui sont alors perdus. Si une des nouvelles tailles est zéro, le tableau est vidé de tous ses éléments.

Augmenter une taille fait apparaître de nouveaux éléments, qui sont invalides par défaut. Il faudra alors explicitement les initialiser individuellement par un appel au setter \galgas{setValueAtIndex}.


\section{Comparaison}

Un type tableau implémente les opérateurs \galgas{=} et \galgas{\!=}. L'égalité de deux tableaux est testé comme suit :
\begin{itemize}
  \item les tailles de chaque axe doivent être identiques ;
  \item les éléments doivent être identiques.
\end {itemize}

  %!TEX encoding = UTF-8 Unicode
%!TEX root = ../galgas-book.tex

%--------------------------------------------------------------
\chapter{Le type \texttt{class}}

\section{Déclaration d'une classe}

Voici différents exemples de déclaration de classes :

\begin{galgas}
abstract class @A {
  @uint mA ;
}
class @B extends @A {
  @string mB ;
}
class @C extends @B {
  @data mC ;
}
\end{galgas}

La classe \ggs+@A+ est abstraite (c'est-à-dire qu'elle ne peut pas être instanciée), la classe \ggs+@B+ hérite de \ggs+@A+. Une classe déclare zéro, un ou plusieurs attributs. L'héritage multiple n'est pas implémenté en GALGAS.

Une classe qui hérite d'une autre peut être abstraite :
\begin{galgas}
abstract class @D extends @C {
  ...
 }
\end{galgas}

Une classe non abstraite définit implicitement le constructeur \ggs+new+, et des \emph{getters} pour lire les attributs, et des \emph{setters} pour les écrire. On ne peut pas définir explicitement d'autres constructeurs, \emph{getters} ou \emph{setters} à l'intérieur de la classe. Cependant,  les extensions (\refChapterPage{extensions}) permettent de définir \emph{getters}, \emph{méthodes} et \emph{setters} associés à une classe.












\section{Le constructeur \texttt{new}}

Le constructeur \ggs+new+ est implicitement pour toute classe non abstraite (c'est à dire les classes \ggs+@B+ et \ggs+@C+). Ce constructeur présente un argument par attribut déclaré dans la classe instanciée et dans toutes les classes mère. L'ordre des arguments est celui obtenu en parcourant la hiérarchie de classes, en commençant par la classe racine. Par exemple on écrira :

\begin{galgas}
@B b [new
  !0 # Attribut mA de @A
  !"Hello" # Attribut mB de @B
] ;
@C c [new
  !0 # Attribut mA de @A
  !"Hello" # Attribut mB de @B
  ![@data emptyData] # Attribut mC de @C
] ;
\end{galgas}








\section{Lecture d'un attribut}

Par défaut, la lecture d'un attribut est activée par la définition implicite d'un \emph{getter}, dont le nom est le nom de l'attribut. Ainsi, pour une variable \ggs+b+ de type \ggs+@B+, on pourra écrire :

\begin{galgas}
@uint v = [b mA] ;
@string s = [b mB] ;
\end{galgas}

Il est possible d'inhiber la génération implicite d'un \emph{getter} de lecture d'un attribut en complétant sa déclaration par \ggs+feature %nogetter+, comme par exemple :

\begin{galgas}
abstract class @A {
  @uint mA feature nogetter ;
}
\end{galgas}

L'écriture \ggs+[b mA]+ sera alors rejetée par le compilateur.









\section{Écriture d'un attribut}

Par défaut, l'écriture d'un attribut n'est pas activée.

Pour activer la génération d'un \emph{setter} permettant décrire un attribut, compléter la déclaration de cet attribut par \ggs+feature %setter+. Un \emph{setter} est alors engendré, et porte le nom \texttt{set<Attribut>}, c'est à dire le nom de l'attribut avec sa première lettre en majuscule, précédé par \texttt{set}. Par exemple :

\begin{galgas}
abstract class @A {
  @uint mA feature setter ;
}
\end{galgas}


Pour modifier l'attribut \ggs+mA+, on écrira :

\begin{galgas}
[!?b setMA !12] ;
\end{galgas}

Si on veut à la fois inhiber la génération implicite d'un \emph{getter} de lecture d'un attribut et engendrer le \emph{setter} d'écriture, il suffit de déclarer l'attribut par :

\begin{galgas}
  @uint mA feature nogetter, setter ;
\end{galgas}

Ou encore :

\begin{galgas}
  @uint mA feature setter, nogetter ;
\end{galgas}












\section{Conversions entre objets de classes différentes}

Pour toute cette section, nous illustrons les constructions décrites en nous basant sur les trois variables suivantes :

\begin{galgas}
@A a ;
@B b = ... ;
@C c = ... ;
\end{galgas}

\subsection{Affectation polymorphique}

GALGAS accepte l'affectation polymorphique qui est par exemple \ggs+a = b+. Elle est autorisée aussi lors de l'affectation d'une expression effective à un paramètre formel dans une instruction d'appel (de routine, de fonction, de méthode, ...)

L'affectation polymorphique inverse (qui consisterait à écrire \ggs+b = a+) est logiquement refusée par le compilateur.

Il y a trois constructions qui permettent d'effectuer cette opération :
\begin{itemize}
  \item l'expression de conversion polymorphique inverse (\refSubsectionPage{expConversionPolymorphiqueInverse}) ;
  \item l'expression de test du type dynamique (\refSubsectionPage{testTypeDynamiqueExpression}) ;
  \item l'instruction \ggs+cast+ (\refSectionPage{instructionCast}).
\end{itemize}

Pour effectuer ponctuellement une affectation polymorphique inverse, on écrit (les parenthèses sont obligatoires) :

\begin{galgas}
@T resultat = (cast expression : @T) ;
\end{galgas}

Si le type dynamique de l'\ggs+expression+ est \ggs+@T+ ou une de ses classes héritières, l'expression de conversion polymorphique renvoie un objet de type \ggs+@T+ contenant la valeur de \ggs+expression+. Dans le cas contraire, un message d'erreur est affiché, et la variable \ggs+resultat+ est non construite.

L'exécution échoue donc avec émission de message d'erreur si la conversion n'est pas possible. 


Grâce à l'\emph{expression de test du type dynamique}, il est possible de tester si une conversion est possible. On peut donc écrire :

\begin{galgas}
if (expression is @B) then
  const @B variable = (cast expression : @B) ;
  ...
elsif (expression is @C) then
  const @C variable = (cast expression : @C) ;
  ...
else
  message "conversion impossible" ;
end if ;
\end{galgas}

L'instruction \ggs+cast+ permet simplement d'exprimer de manière plus élégante une série de test de conversion. La forme équivalent à l'instruction \ggs+if+ précédente est :

\begin{galgas}
cast expression
when >= @B variable :
  ...
when >= @C variable :
  ...
else
  message "conversion impossible" ;
end cast ;
\end{galgas}



  %!TEX encoding = UTF-8 Unicode
%!TEX root = ../galgas-book.tex

%--------------------------------------------------------------
\chapterLabel{Le type \texttt{enum}}{typeEnum}
%-------------------------------------------------------------

\tableDesMatieresLocaleDeProfondeurRelative{2}



\section{Déclaration}

La déclaration d'un type \ggst+enum+ nomme l'ensemble des constantes du type énuméré. Plusieurs types énumérés peuvent définir des constantes de même nom. Dans ce premier exemple, les constantes n'ont pas de valeur associée.

Par exemple :

\begin{galgas34}
enum @feuTricolore {
  case vert
  case orange
  case rouge
}
\end{galgas34}

Il est possible d'associer plusieurs valeurs à chaque constante d'un type énuméré. Par exemple~:

\begin{galgas3}
enum @monOption {
  case noOption
  case option1 (@string name)
  case option2 (@uint value @string name)
}
\end{galgas3}

\begin{galgas4}
enum @monOption {
  case noOption
  case option1 (@string name)
  case option2 (@uint value, @string name)
}
\end{galgas4}

Chaque valeur associée est spécifiée par son type et son nom. Noter la virgule de séparation en GALGAS4.






\section{Instanciation}

Chaque constante définit un initialisateur de même nom. Si la constante n'a pas de valeur associée, on écrit :

\begin{galgas34}
var @feuTricolore feu = @feuTricolore.vert ()
\end{galgas34}

Les parenthèses peuvent être omises :
\begin{galgas34}
var @feuTricolore feu = @feuTricolore.vert
\end{galgas34}

L'annotation de type peut être omise :

\begin{galgas34}
var @feuTricolore feu = .vert
\end{galgas34}

\begin{galgas34}
var feu = @feuTricolore.vert
\end{galgas34}


Si la constante présente des valeurs associées, l'initialisateur présente autant d'arguments, chaque ayant comme étiquette le nom donné dans la déclaration~:

\begin{galgas3}
var @monOption feu = @monOption.option2 {!name: "toto" !value: 0}
\end{galgas3}

\begin{galgas34}
var @monOption feu = @monOption.option2 (!name: "toto", !value: 0)
\end{galgas34}

Comme dans l'exemple précédent, l'annotation de type peut être omise.








\section{L'instruction \texttt{switch}}

L'instruction \ggst+switch+ (\refSectionPage{instructionSwitch}) est dédiée à tous les types énumérés. On écrit par exemple :

\begin{galgas34}
var @feuTricolore feu = ...
switch feu
case vert : ...
case orange : ...
case rouge : ...
end
\end{galgas34}

Chaque constante doit apparaître une et une seule fois~: l'instruction \ggst+switch+ est exhaustive.

Si une constante présente des valeurs associées, alors elles doivent toutes être mentionnées~:

\begin{galgas3}
var @monOption uneOption = ...
switch uneOption
case noOption : ...
case option1 (@string unNom) : ...
case option2 (@uint unEntier @string unNom) : ...
end
\end{galgas3}

L'annotation de type d'une valeur associée peut être omise~:

\begin{galgas3}
var @monOption uneOption = ...
switch uneOption
case noOption : ...
case option1 (unNom) : ...
case option2 (unEntier unNom) : ...
end
\end{galgas3}

Si une valeur associée n'est pas utile, on peut remplacer son nom par un joker~:
\begin{galgas3}
var @monOption uneOption = ...
switch uneOption
case noOption : ...
case option1 (unNom) : ...
case option2 (unEntier *) : ...
end
\end{galgas3}

Plusieurs \emph{jokers} consécutifs peuvent être regroupés~:

\begin{galgas3}
var @monOption uneOption = ...
switch uneOption
case noOption : ...
case option1 (unNom) : ...
case option2 (2*) : ...
end
\end{galgas3}

En GALGAS4, les valeurs associées consécutives doivent être séparées par des virgules~:

\begin{galgas4}
var @monOption uneOption = ...
switch uneOption
case noOption : ...
case option1 (unNom) : ...
case option2 (unEntier, unNom) : ...
end
\end{galgas4}

\begin{galgas4}
var @monOption uneOption = ...
switch uneOption
case noOption : ...
case option1 (unNom) : ...
case option2 (unEntier, *) : ...
end
\end{galgas4}






\sectionLabel{Tester une valeur}{testerValeurEnum}

Un type émuméré définit implicitement un \emph{getter} sans argument pour chaque constante déclarée~; ce getter a pour nom le nom de la constante.

\subsection{Getter \texttt{is\emph{Name}}}

Un type émuméré définit implicitement un \emph{getter} sans argument pour chaque constante déclarée~; ce getter a pour nom \texttt{is} suivi du nom de la constante avec la première lettre en majuscule. Pour le type \ggsq!@monOption! cité enb exemple au dessus, les trois getters définis sont~:
\begin{galgas4}
func @monOption.isNoOption () -> @bool
func @monOption.isOption1 () -> @bool
func @monOption.isOption2 () -> @bool
\end{galgas4}
\begin{galgas3}
func @monOption isNoOption -> @bool
func @monOption isOption1 -> @bool
func @monOption isOption2 -> @bool
\end{galgas3}

Ainsi~:

\begin{galgas4}
var v = @monOption.noOption
let @bool b = v.isNoOption // b est true
v = @monOption.option1 (!name: "xyz")
let @bool c = v.isNoOption // c est false
\end{galgas4}
\begin{galgas3}
var v = @monOption.noOption
let @bool b = [v isNoOption] // b est true
v = @monOption.option1 (!name: "xyz")
let @bool c = [v isNoOption] // c est false
\end{galgas3}



\subsection{Getter \texttt{get\emph{Name}}}

Uniquement pour les constantes ayant des valeurs associées, est défini un getter sans argument nommé \texttt{get} suivi du nom  de la constante avec la première lettre en majuscule. Ce getter renvoie une valeur optionnelle qui permet de savoir si le récepteur a pour valeur cette constante, et d'obtenir les valeurs associées.

Pour chaque constante ayant des valeurs associées, un type structure est implicitement défini, dont les propriétés sont les valeurs associées. Ainsi, sont définis~:
\begin{galgas4}
struct @monOption.@option1 {
  public let @string name
}
struct @monOption.@option2 {
  public let @uint value
  public let @string name
}
\end{galgas4}

Les \emph{getter} associés ont pour signature~:
\begin{galgas4}
func @monOption.getOption1 () -> @monOption.@option1?
func @monOption.getOption2 () -> @monOption.@option2?
\end{galgas4}
\begin{galgas3}
func @monOption getOption1 -> @monOption.@option1?
func @monOption getOption2 -> @monOption.@option2?
\end{galgas3}

Pour accéder aux aux valeurs associées, il faut extraire la valeur embarquée dans la valeur optionnelle renvoyée, ce qui peut être fait dans une instruction \ggsq!if!~:

\begin{galgas4}
var v = ...
if let x = v.getOption1 then
  // Ici, v a pour valeur option1
  // x a pour type @monOption.@option1
  // On accède à la valeur associée par x.name
else
  // Ici, v a pour une valeur autre que option1
  // et x n'est pas accessible
end
\end{galgas4}

\begin{galgas3}
var v = ...
if let x = [v getOption1] then
  // Ici, v a pour valeur option1
  // x a pour type @monOption.@option1
  // On accède à la valeur associée par x.name
else
  // Ici, v a pour une valeur autre que option1
  // et x n'est pas accessible
end
\end{galgas3}












\section{Méthodes d'extraction}

Uniquement pour les constantes ayant des valeurs associées, est défini un méthode nommée \texttt{extract} suivi du nom  de la constante avec la première lettre en majuscule. Cette méthode renvoie dans les arguments de sortie les valeurs associées.

{\bf Si le type émuméré n'a pas la valeur attendue, une erreur d'exécution est déclenchée, et les arguments de sortie ne sont pas construits~: il est plus propre d'utiliser un getter \texttt{get\emph{Name}} qui effectue à la fois le test et l'extraction.}


Les \emph{méthodes} associées ont pour signature~:
\begin{galgas4}
proc @monOption.extractOption1 (!unNom: @string outNom)
proc @monOption.extractOption2 (!unEntier: @uint outEntier, !unNom: @string outNom)
\end{galgas4}
\begin{galgas3}
method @monOption extractOption1 !unNom: @string outNom
method @monOption extractOption2 !unEntier: @uint outEntier !unNom: @string outNom
\end{galgas3}

Ces méthodes peuvent s'utiliser comme suit~:
\begin{galgas4}
var v = ...
if v.isOption1 then
  v.extractOption1 (?unNom: let nom) 
  // Ici, 'nom' a pour valeur la valeur associée
else
  // ...
end
\end{galgas4}

\begin{galgas3}
var v = ...
if v.isOption1 then
  [v extractOption1 ?unNom: let nom] 
  // Ici, 'nom' a pour valeur la valeur associée
else
  // ...
end
\end{galgas3}













\section{Comparaison}

Par défaut, un type enuméré n'accepte aucun des six opérateurs de comparaison (\ggsq+==+, \ggsq+!=+, \ggsq+<+, \ggsq+<=+, \ggsq+>+, \ggsq+>+).

\subsection{L'attribut \texttt{\%equatable}}

Si le type énuméré est déclaré avec l'attribut \ggsq!%equatable!, alors les opérateurs \ggst+==+, \ggst+!=+ sont définis. Ceci exige que tous les types des valeurs associées soient eux-mêmes \ggsq!%equatable! ou \ggsq!%comparable!.

Comme aucune relation d'ordre n'est définie, l'ordre de déclaration des constantes est sans importance.

\begin{galgas3}
enum @monOption %equatable {
  case noOption
  case option1 (@string name)
  case option2 (@uint value @string name)
}
\end{galgas3}

\begin{galgas4}
enum @monOption %equatable {
  case noOption
  case option1 (@string name)
  case option2 (@uint value, @string name)
}
\end{galgas4}


\subsection{L'attribut \texttt{\%comparable}}

Si le type énuméré est déclaré avec l'attribut \ggsq!%comparable!, alors les six opérateurs \ggst+==+, \ggst+!=+, \ggsq+<+, \ggsq+<=+, \ggsq+>+ et \ggsq+>+ sont définis. Ceci exige que tous les types des valeurs associées soient eux-mêmes \ggsq!%comparable!.

\begin{galgas3}
enum @monOption %comparable {
  case noOption
  case option1 (@string name)
  case option2 (@uint value @string name)
}
\end{galgas3}

\begin{galgas4}
enum @monOption %comparable {
  case noOption
  case option1 (@string name)
  case option2 (@uint value, @string name)
}
\end{galgas4}

L'ordre est lexicographique : c'est celui de la déclaration, c'est-à-dire que \ggsq!@monOption.noOption < @monOption.option1! et \ggsq!@monOption.option1 < @monOption.option2!.

Pour les constantes ayant des valeurs associées, l'ordre est obtenu en comparant ces valeurs les unes après les autres~: par exemple pour \ggsq!option2!, on compare d'abord \ggsq!value!, puis \ggsq!name!. 









\section{Exemple d'utilisation des valeurs associées}

Associer des valeurs à chaque constante permet d'alléger dans certains cas le code à écrire. Supposons par exemple que l'on ait dans un langage une construction optionnelle :

\begin{galgas34}
rule <regleProduction> {
  select
  or
    $option$
    $identifier$ (?let nomOption)
  end
}
\end{galgas34}

Comment construire l'arbre syntaxique abstrait ? Il y a en fait trois possibilités.

\textbf{Première solution.} La première consiste à considérer la chaîne vide comme significative de l'absence d'option :
\begin{galgas34}
rule <regleProduction> {
  let @lstring nomOption
  select
    nomOption = "".here
  or
    $option$
    $identifier$ (?nomOption)
  end
}
\end{galgas34}

Évidemment, cette solution est acceptable uniquement si l'information associée est simple, et si une valeur particulière peut être considérée comme l'absence d'option.

\textbf{Deuxième solution.} La deuxième solution fait appel à trois classes :
\begin{galgas4}
abstract class @abstractOption {}

class @noOption : @abstractOption {}

class @option : @abstractOption { public let @lstring name }
\end{galgas4}

La construction est réalisée par :
\begin{galgas4}
rule <regleProduction> {
  let @abstractOption optionAST
  select
    optionAST = @noOption ()
  or
    $option$
    $identifier$ (?let nom)
    optionAST = @option (nom)
  end
}
\end{galgas4}

Cette solution, plus générale, est plus lourde à mettre en œuvre : trois classes.

\textbf{Troisième solution.} La troisième et dernière solution consiste à écrire un type énuméré possédant des valeurs associées :

\begin{galgas4}
enum @option {
  case noOption
  case optionPresente (@lstring name)
}
\end{galgas4}

À la constante \ggst+optionPresente+ est associée une valeur de type \ggst+@lstring+, identifiée par le nom \ggst+name+. La construction est maintenant réalisée par :
\begin{galgas4}
rule <regleProduction> {
  let @option optionAST
  select
    optionAST = .noOption
  or
    $option$
    $identifier$ (?let nomOption)
    optionAST = .optionPresente (!name: nomOption)
  end
}
\end{galgas4}













  %!TEX encoding = UTF-8 Unicode
%!TEX root = ../galgas-book.tex

%--------------------------------------------------------------
\chapter{Le type \texttt{graph}}
%-------------------------------------------------------------

Le type \galgas{graph} permet de faire des opérations sur les graphes orientés.

Chaque nœud est identifié par un nom qui est une chaîne de caractères (de type \galgas{@string}), et est associé à une information utilisateur de type quelconque.

Un arc est identifié par un couple de nœuds.


Un type \galgas{graph} se déclare comme suit :
\lstset{emph={@nom_du_type_graph, @nom_liste_information}, emphstyle=\emph}
\begin{galgascode}
graph @nom_du_type_graph (@nom_liste_information) {
}
\end{galgascode}

Le nom \galgas{@nom_du_type_graph} est le nom donné au type. Le nom \galgas{@nom_liste_information} nomme un type qui spécifie l'information utilisateur associée à chaque nœud.

Attention, le type \galgas{@nom_liste_information} est un type \emph{liste}, et l'information utilisateur a pour type l'élement associé, c'est à dire \galgas{@nom_liste_information.element}. 

Par exemple, si l'on veut manipuler des graphes dont l'information associée est un entier \galgas{@uint}, on déclarera :
\begin{galgascode}
graph @monGraphe (@uintlist) {
}
\end{galgascode}

Si l'information associée est par exemple composée d'un entier et d'une chaîne de caractères, il faut déclarer un type liste :
\begin{galgascode}
list @maListe {
  @uint monInfo1 ;
  @string monInfo2 ;
}
graph @monGraphe (@maListe) {
}
\end{galgascode}






\section{Entrer les nœuds}



\section{Entrer les arcs}





  %!TEX encoding = UTF-8 Unicode
%!TEX root = ../galgas-book.tex

%--------------------------------------------------------------
\chapter{Le type \texttt{map}}
%-------------------------------------------------------------

Un objet de type \galgas{map} est une table de symboles, chaque symbole étant associé à des valeurs.

\section{Déclaration}

La déclaration d'un type \galgas{map} nomme :
\begin{itemize}
  \item les attributs qui sont associés à une clé ;
  \item les \emph{modifiers} d'insertion ;
  \item les \emph{méthodes} de recherche ;
  \item les \emph{modifiers} de retrait ;
\end{itemize}

Les clés sont déclarées implicitement et sont du \refTypePredefini{lstring}.

Par exemple :

\begin{galgascode}
map @MaTable {
  @string mPremier ;
  @bool mSecond ;
  insert insertKey error message "the '%K' key is already declared in %L";
  search searchKey error message "the '%K' key is not defined" ;
  remove removeKey error message "the '%K' key is not defined" ;
}
\end{galgascode}






\section{Modifiers d'insertion}

Une \galgas{map} peut déclarer zéro, un ou plusieurs \emph{modifiers} d'insertion. Un \emph{modifier} d'insertion permet d'insérer une nouvelle entrée à une table. Une erreur est déclenchée en cas de tentative d'une clé déjà existante.


Un \emph{modifier} d'insertion est déclaré par :

\lstset{emph={nom}, emphstyle=\emph}
\begin{galgascode}
insert nom error message "message_erreur" ;
\end{galgascode}

L'identificateur \galgas{nom} donne un nom au \emph{modifier} d'insertion ; ce nom doit être unique parmi les \emph{modifiers} d'insertion et de retrait. La chaîne de caractères \galgas{"message_erreur"} définit le message d'erreur qui est affiché en cas de tentative d'une clé déjà existante. Cette chaîne accepte deux séquences d'échappement :
\begin{itemize}
  \item \colorbox{\couleurCodeGALGAS}{\texttt{\%K}}, qui est remplacée par la chaîne de caractères de la clé existante ;
  \item \colorbox{\couleurCodeGALGAS}{\texttt{\%L}}, qui est remplacée par la chaîne décrivant la position de la clé existante dans les fichiers source.
\end{itemize}


Un \emph{modifier} d'insertion est appelé dans une \emph{instruction d'appel de modifier}, comprenant tous ses arguments en sortie :
\begin{itemize}
  \item le premier argument est une expression de type \galgas{@lstring} qui caractérise la clé à insérer ;
  \item ensuite, pour chaque attribut déclaré, une expression du type de cet attribut.
\end{itemize}

Par exemple :
\begin{galgascode}
@MaTable uneTable [emptyMap] ;
@lstring clef := ... ;
@string s := ... ;
@uint v := ... ;
[!?uneTable insertKey !clef !s !v] ;
\end{galgascode}











\section{Méthodes de recherche}

Une \galgas{map} peut déclarer zéro, une ou plusieurs \emph{méthodes} de recherche. Une \emph{méthode} de recherche permet de rechercher une entrée d'une table, et retourne la valeur de ses attributs associés. Une erreur est déclenchée si la clé n'existe pas.


Une \emph{méthode} de recherche est déclarée par :

\lstset{emph={nom}, emphstyle=\emph}
\begin{galgascode}
search nom error message "message_erreur" ;
\end{galgascode}

L'identificateur \galgas{nom} donne un nom à la \emph{méthode} de recherche ; ce nom doit être unique parmi ces \emph{méthodes}. La chaîne de caractères \galgas{"message_erreur"} définit le message d'erreur qui est affiché en cas de recherche d'une clé inexistante. Cette chaîne accepte une séquence d'échappement :
\begin{itemize}
  \item \colorbox{\couleurCodeGALGAS}{\texttt{\%K}}, qui est remplacée par la chaîne de caractères de la clé inexistante recherchée ;
\end{itemize}


Une \emph{méthode} de recherche est appelée dans une \emph{instruction d'appel de méthode} :
\begin{itemize}
  \item le premier argument (sortie) est une expression de type \galgas{@lstring} qui caractérise la clé à rechercher ;
  \item ensuite, pour chaque attribut déclaré, un argument en entrée nommant une variable destinée à recevoir la valeur de l'attribut correspondant.
\end{itemize}

Par exemple :
\begin{galgascode}
@MaTable uneTable [emptyMap] ;
...
@lstring clef := ... ;
[!?uneTable searchKey !clef ?@string s ?@uint v] ;
\end{galgascode}













\section{Modifiers de retrait}

Une \galgas{map} peut déclarer zéro, un ou plusieurs \emph{modifiers} de retrait. Un \emph{modifier} de recherche permet de retirer une entrée d'une table, et retourne la valeur des attributs de la clé retirée. Une erreur est déclenchée si la clé n'existe pas.


Un \emph{modifier} de retrait est déclaré par :

\lstset{emph={nom}, emphstyle=\emph}
\begin{galgascode}
remove nom error message "message_erreur" ;
\end{galgascode}

L'identificateur \galgas{nom} donne un nom au \emph{modifier} de retrait ; ce nom doit être unique parmi les \emph{modifiers} d'insertion et de retrait. La chaîne de caractères \galgas{"message_erreur"} définit le message d'erreur qui est affiché en cas de recherche d'une clé inexistante. Cette chaîne accepte une séquence d'échappement :
\begin{itemize}
  \item \galgas{\%K}, qui est remplacée par la chaîne de caractères de la clé inexistante à retirer ;
\end{itemize}


Un \emph{modifier} de retrait est appelé dans une \emph{instruction d'appel de modifier} :
\begin{itemize}
  \item le premier argument (sortie) est une expression de type \galgas{@lstring} qui caractérise la clé à retirer ;
  \item ensuite, pour chaque attribut déclaré, un argument en entrée nommant une variable destinée à recevoir la valeur de l'attribut correspondant de la clé retirée.
\end{itemize}

Par exemple :
\begin{galgascode}
@MaTable uneTable [emptyMap] ;
...
@lstring clef := ... ;
[!?uneTable removeKey !clef ?@string s ?@uint v] ;
\end{galgascode}






\section{Constructeurs}

\subsection{Constructeur \texttt{emptyMap}}

\begin{galgascode}
constructor @T emptyMap -> @T ;
\end{galgascode}

Ce constructeur permet d'instancier une table vide. Exemple :
\begin{galgascode}
@MaTable uneTable [emptyMap] ;
\end{galgascode}

 

\subsection{Constructeur \texttt{mapWithMapToOverride}}

\begin{galgascode}
constructor @T mapWithMapToOverride ?@T inMapToOverride -> @T ;
\end{galgascode}

Ce constructeur permet d'instancier une table vide, qui surcharge la table \galgas{inMapToOverride} citée en argument. Exemple :
\begin{galgascode}
@MaTable uneTable [emptyMap] ;
@MaTable autreTableTable [inMapToOverride !uneTable] ;
\end{galgascode}

\section{Readers}

%\subsection{Le reader \texttt{allKeyList}}
%
%\begin{galgascode}
%reader @T allKeyList -> @lstringlist ;
%\end{galgascode}
%
%Le \emph{reader} \galgas{allKeyList} retourne la liste construite avec toutes les clés du receveur, dans la table de premier niveau et dans les tables surchargées. L'ordre de la liste est :
%\begin{itemize}
%  \item d'abord les clés de la table de premier niveau, puis celles des tables surchargées, dans l'ordre de la surcharge ;
%  \itel pour chaque table, les clés apparaissent dans l'ordre alphabétique croissant.
%\end{itemize}

\subsection{Le reader \texttt{count}}

\begin{galgascode}
reader @T count -> @uint ;
\end{galgascode}


Le \emph{reader} \galgas{count} retourne un \galgas{@uint} qui contient le nombre d'entrées de la table de premier niveau du receveur.



\subsection{Le reader \texttt{hasKey}}

\begin{galgascode}
reader @T hasKey ??@string inKey -> @bool ;
\end{galgascode}


Le \emph{reader} \galgas{hasKey} retourne un \galgas{@bool} qui est \galgas{true} si la clé \galgas{inKey} est dans la table de premier niveau du receveur, \galgas{false} dans le cas contraire.



\subsection{Le reader \texttt{keyList}}

\begin{galgascode}
reader @T keyList -> @lstringlist ;
\end{galgascode}


Le \emph{reader} \galgas{keyList} retourne la liste construite avec toutes les clés de la table de premier niveau du receveur. L'ordre de la liste est l'ordre alphabétique croissant des clés.



\subsection{Le reader \texttt{keySet}}

\begin{galgascode}
reader @T keySet -> @stringset ;
\end{galgascode}


Le \emph{reader} \galgas{keySet} retourne l'ensemble de toutes les clés de la table de premier niveau du receveur.





\subsection{Le reader \texttt{locationForKey}}

\begin{galgascode}
reader @T locationForKey ??@string inKey -> @location ;
\end{galgascode}


Le \emph{reader} \galgas{locationForKey} retourne un \galgas{@location} qui contient l'information de position de la clé \galgas{inKey} dans la table de premier niveau du receveur. Une erreur d'exécution est déclenchée si cette clé n'existe pas.








\subsection{Le reader \texttt{overriddenMap}}

\begin{galgascode}
reader @T overriddenMap -> @T ;
\end{galgascode}


Le \emph{reader} \galgas{overriddenMap} retourne la table obtenue en amputant de la valeur du receveur la table de premier niveau. Si le receveur n'a pas de table surchargée, une erreur d'exécution est déclenchée.





\section{Énumération}

L'instruction \galgas{foreach} permet d'énumérer des objets de type \galgas{map}. Uniquement la table de premier niveau est énumérée. Par défaut, l'énumération s'effectue dans l'ordre croissant des clés. Pour énumérer dans l'ordre décroissant, utiliser le qualifier \galgas{>}.

À l'intérieur du coprs de la boucle, sont implicitement définies :
\begin{itemize}
  \item la constante \galgas{lkey}, de type \galgas{@lstring}, qui a pour valeur la clé de l'entrée courante ;
  \item pour chaque attribut, une constante du type de l'attribut, et portant le nom de cet attribut, qui a pour valeur la valeur de cet attribut de l'entrée courante.
\end{itemize}

Par exemple :
\begin{galgascode}
@MaTable uneTable [emptyMap] ;
[!?uneTable insertKey ![@lstring new !"z" !here] !"world" !5] ;
[!?uneTable insertKey ![@lstring new !"a" !here] !"hello" !10] ;
foreach aMap do
  message lkey->string . " " . mPremier . " " . mSecond . "\n" ;
end foreach ;
\end{galgascode}

L'affichage produit est :

\begin{galgascode}
a hello 10
z world 5
\end{galgascode}

%====== Setting an attribute of an entry ======
%
%^Available in GALGAS 1.8.4 and later. ^
%
%Given a key, you can directly set an attribute of a map entry.
%
%In order to enable this feature, you have to associate the setter feature to the given attribute.
%
%map @MaTable {
%  @string mPremier ;
%  @bool mSecond feature setter ;
%}
%
%For every attribute declared with this feature, a modifier is available; its name is build by the concatenation of three patterns:
%  - the string set;
%  - the attribute name, with the first letter capitalized;
%  - the string ForKey.
%
%So, for the mSecond attribute, the associated modifier name is setmSecondForKey.
%
%The modifier has two input arguments:
%  - the key, an @string expression;
%  - the value to set to the attribute.
%
%For example:
%
%@MaTable aMap := ... ;
%@string s := ... ;
%@bool v := ... ;
%[!?aMap setmSecondForKey !s !v] ;
%
%A run-time error is raised if the key value does not exist in the map.
%
%====== Using the with instruction on a map object ======
%
%^Available in GALGAS 1.8.4 and later. ^
%
%The with instruction enables a direct access on all attributes of an entry. Its syntax is:
%
%with //prefix// !?//map_object// //search_method// !//key_expression// do
%  ...
%else
%  ...
%end with ;
%
%The //prefix// and the else part are optional.
%
%There are two different behaviours, depending from the //search_method//:
%  * the //search_method// is one declared in the search declarations;
%  * the //search_method// is the predefined hasKey identifier.
%
%===== with instruction naming a declared search method =====
%
%The //key_expression// should be an @lstring expression.
%
%Given this map type declaration:
%
%map @MaTable {
%  @string mPremier ;
%  @bool mSecond ;
%  search searchKey error message "the %%'%K'%% key is not defined" ;
%  ...
%}
%
%You can write:
%
%@MaTable aMap := ... ;
%@lstring aKey := ... ;
%with !?aMap searchKey !aKey do
%  # ... 
%else
%  # ...
%end with ;
%
%The aMap object is accessed in read/write mode.
%
%If the aKey object value does not correspond to an existing entry, an error message is displayed and the else part is executed. The error message is based upon the search method name, here searchKey. The error location is given by aKey location.
%
%If the aKey object value corresponds to an existing entry, the do part is executed. The entry's attributes can be fully accessed in this part (you can read, write or modify them), using directly their names. The entry's key (an @lstring object) can be accessed in read mode (you cannot modify it), using the key identifier. For example:
%
%
%@MaTable aMap := ... ;
%@lstring aKey := ... ;
%with !?aMap searchKey !aKey do
%  mPremier .= %%"xyz"%% ; # mPremier is accessed in read/write mode
%  mSecond := true ; # mSecond is accessed in write mode
%  log key ; # key is can only be accessed in read mode
%end with ;
%
%The only constraint is that all attributes should be valuated at the end of the do part.
%
%===== with instruction naming the predefined hasKey method =====
%
%The //key_expression// should be an @string expression.
%
%Given this map type declaration:
%
%map @MaTable {
%  @string mPremier ;
%  @bool mSecond ;
%  ...
%}
%
%You can write:
%
%@MaTable aMap := ... ;
%@string aKey := ... ;
%with !?aMap hasKey !aKey do
%  # ... 
%else
%  # ...
%end with ;
%
%The aMap object is accessed in read/write mode.
%
%If the aKey object value does not correspond to an existing entry, no error message is displayed and the else part is executed.
%
%If the aKey object value corresponds to an existing entry, the do part is executed.  The entry's attributes can be fully accessed in this part (you can read, write or modify them), using directly their names, the entry's key (an @lstring object) can be accessed in read mode (you cannot modify it), using the key identifier, exactly as in the previously described behaviour.
%
%===== Using a prefix =====
%
%This feature enables to prepend with a prefix the names used for accessing the attributes and the key in the do part. It could be useful for avoiding name conflicts.
%
%It is available for both search methods.
%
%Given this map type declaration:
%
%map @MaTable {
%  @string mPremier ;
%  @bool mSecond ;
%  ...
%}
%
%You can write:
%
%@MaTable aMap := ... ;
%@string aKey := ... ;
%with xyz_ : !?aMap hasKey !aKey do
%  xyz_mPremier .= %%"xyz"%% ; # mPremier is accessed in read/write mode
%  xyz_mSecond := true ; # mSecond is accessed in write mode
%  log xyz_key ; # key is can only be accessed in read mode
%end with ;
%
%Using the xyz_ prefix, all attributes and the key should be accessed using this prefix. 
  %!TEX encoding = UTF-8 Unicode
%!TEX root = ../galgas-book.tex

%--------------------------------------------------------------
\chapterLabel{Le type structure}{typeStructure}
%-------------------------------------------------------------

\tableDesMatieresLocaleDeProfondeurRelative{1}



Le mot-clé \ggst!struct! permet de définir des types de structure. Un objet de type structure a une sémantique de valeur. Une déclaration de structure doit déclarer au moins une propriété. Par exemple :

\begin{galgas34}
struct @MaStructure {
  public var @uint propriété
}
\end{galgas34}

Il est possible d'associer une valeur initiale à la déclaration d'une propriété~:
\begin{galgas34}
struct @MaStructure2 {
  public var @uint propriété = 9
}
\end{galgas34}

En GALGAS 3, les propriétés non initialisées peuvent être déclarées avec \ggst!public! et \ggst!var! implicites (cette syntaxe est obsolète et n'existe pas en GALGAS 4)~:
\begin{galgas3}
struct @MaStructure {
  @uint propriété
}
\end{galgas3}

Une déclaration de structure peut aussi déclarer~: des initialisateurs (\refSectionPage{initStruct}), des \emph{getters} (\refSectionPage{getterStruct}), des \emph{methodes} (\refSectionPage{methodStruct}) et des \emph{setters} (\refSectionPage{setterStruct}).

Ces déclarations peuvent apparaître soit dans la déclaration de structure, soit comme une extension (\refChapterPage{chapitreExtensions}).












\sectionLabel{initialisateurs}{initStruct}

Lorque que l'on instancie un type structure, on appelle un \emph{initialisateur}. Celui-ci a pour rôle de fixer une valeur initiale à toutes les propriétés de la structure instancée.

En GALGAS 3, on peut instancier une structure avec le constructeur \ggst!new! (\refSubsectionPage{constructeurNewStruct}), qui est automatiquement engendré par toute structure~; cette construction est obsolète et est remplacée par l'appel d'un initialisateur.

Toute structure implémente un initialisateur. Si une structure ne déclare aucun initialisateur, un initialisateur par défaut est automatiquement engendré (\refSubsectionPage{initialisateurDefautStruct}). L'écriture d'un initialisateur est présenté à la \refSubsectionPage{initialisateurStruct}.

\subsectionLabel{Initialisateur synthétisé}{initialisateurDefautStruct}

L'appel de l'initialisateur synthétisé comprend une valeur par propriété non initialisée déclarée par le type structure.

Par exemple, pour la déclaration~:
\begin{galgas34}
struct @maStructure {
  public var @uint propriété1
  public var @bool propriété2
}
\end{galgas34}

La syntaxe la plus générale d'appel de l'initialisateur synthétisé est~:
\begin{galgas4}
var aVariable = @maStructure.init (!propriété1: 123, !propriété2: true)
\end{galgas4}
\begin{galgas3}
var aVariable = @maStructure.init {!propriété1: 123 !propriété2: true}
\end{galgas3}

On peut omettre \ggst!.init!~:
\begin{galgas4}
var aVariable = @maStructure (!propriété1: 123, !propriété2: true)
\end{galgas4}
\begin{galgas3}
var aVariable = @maStructure {!propriété1: 123 !propriété2: true}
\end{galgas3}


Si le contexte le permet, l'annotation de type peut être omis lors de l'appel de l'initialisateur~:
\begin{galgas4}
var @maStructure aVariable = .init (!propriété1: 123, !propriété2: true)
\end{galgas4}
\begin{galgas3}
var @maStructure aVariable = .init {!propriété1: 123 !propriété2: true}
\end{galgas3}


Il est possible d'ajouter l'attribut \ggst=%noArgumentLabel= à la déclaration d'une propriété non initialisée, pour supprimer dans l'appel de l'initialisateur synthétisé l'étiquette d'argument pour cette propriété. Par exemple, si on déclare~:
\begin{galgas3}
struct @maStructure {
  public var @uint propriété1 %noArgumentLabel
  public var @bool propriété2
}
\end{galgas3}

Alors l'appel de l'initialisateur synthétisé devient~:
\begin{galgas4}
var aVariable = @maStructure.init (!123, !propriété2: true)
// Ou bien, en éliminant l'argument d'étiquette vide :
var aVariable = @maStructure.init (123, !propriété2: true)
\end{galgas4}
\begin{galgas3}
var aVariable = @maStructure.init {!123 !propriété2: true}
\end{galgas3}

On peut omettre \ggst!.init! ou l'annotation de type si le contexte le permet.

Si la propriété est initialisée, alors \ggst=%noArgumentLabel= est invalide (déclenche une erreur de syntaxe).





\subsectionLabel{Initialisateur}{initialisateurStruct}



\subsectionLabel{GALGAS 3 : constructeur \texttt{new}}{constructeurNewStruct}

{\bf Cette construction est obsolète et est remplacée par l'appel d'un initialisateur.}

En GALGAS 3, tout type structure définit implicitement le constructeur \ggst!new!. Son appel comprend une valeur par propriété non initialisée déclarée par le type structure.

Par exemple, pour la déclaration~:
\begin{galgas3}
struct @maStructure {
  public var @uint propriété1
  @bool propriété2 // Syntaxe obsolète, autorisée en GALGAS 3
}
\end{galgas3}

L'appel du constructeur \ggst!new! est~:
\begin{galgas3}
var aVariable = @maStructure.new {!123 !true}
\end{galgas3}

Si le contexte le permet, l'annotation de type peut être omis lors de l'appel du constructeur~:
\begin{galgas3}
var @maStructure aVariable = .new {!123 !true}
\end{galgas3}

Il est possible d'ajouter l'attribut \ggst=%selector= à la déclaration d'une propriété de structure. Le faire impose d'utiliser le sélecteur portant le nom de la propriété dans l'appel du constructeur \ggst=new=. Par exemple, si on déclare~:
\begin{galgas3}
struct @maStructure {
  public var @uint propriété1 %selector
  @bool propriété2 // Syntaxe obsolète, autorisée en GALGAS 3
}
\end{galgas3}

Alors l'appel du constructeur \ggst!new! devient~:
\begin{galgas3}
var aVariable = @maStructure.new {!propriété1: 123 !true}
\end{galgas3}


%\subsection{Constructeur \texttt{default}}
%
%Si chacune des propriétés accepte le constructeur par défaut, alors le type structure accepte le constructeur pas défaut. C'est le cas de la structure \ggst!@maStructure! définie au dessus~: \ggst!@uint! accepte le constructeur par défaut (initialisation à \ggst!0!), ainsi que \ggst!@bool! (initialisation à \ggst!false!). Donc~:
%\begin{galgas3}
%var aVariable = @maStructure.default
%\end{galgas3}
%Initialise les propriétés de \ggst!aVariable! respectivement à \ggst!0! et \ggst!false!. On peut aussi écrire~:
%\begin{galgas3}
%@maStructure aVariable = .default
%\end{galgas3}


\section{Accès aux propriétés}

La notation pointée \ggst!variable.propriété! permet d'accéder à une propriété d'une structure, aussi bien en lecture, en écriture et en lecture/écriture.

Exemple d'accès en lecture~:
\begin{galgas3}
@uint v = aVariable.mProp1
\end{galgas3}

Exemple d'accès en écriture~:
\begin{galgas3}
aVariable.mProp1 = 10
\end{galgas3}


Exemple d'accès en lecture/écriture~:
\begin{galgas3}
aVariable.mProp1 ++
\end{galgas3}





\sectionLabel{Getters}{getterStruct}

Un type structure définit un \emph{getter} sans argument par propriété, qui permet d'accéder en lecture à cette propriété. Son nom est celui de la propriété. Par exemple, à la place de~:
\begin{galgas3}
@uint v = aVariable.mProp1
\end{galgas3}

On peut écrire~:
\begin{galgas3}
@uint v = [aVariable mProp1]
\end{galgas3}



\sectionLabel{Méthodes}{methodStruct}



\sectionLabel{Setters}{setterStruct}


\section{Types structure prédéfinis}

Plusieurs types préféfinis GALGAS sont des structures.

\subsectionTypePredefiniLabelIndex{lbigint}

\begin{galgas3}
struct @lbigint {
  @bigint bigint
  @location location
}
\end{galgas3}



\subsectionTypePredefiniLabelIndex{lbool}

\begin{galgas3}
struct @lbool {
  @bool bool
  @location location
}
\end{galgas3}



\subsectionTypePredefiniLabelIndex{lchar}

\begin{galgas3}
struct @lchar {
  @char char
  @location location
}
\end{galgas3}


\subsectionTypePredefiniLabelIndex{ldouble}

\begin{galgas3}
struct @ldouble {
  @double double
  @location location
}
\end{galgas3}







\subsectionTypePredefiniLabelIndex{lsint}

\begin{galgas3}
struct @lsint {
  @sint sint
  @location location
}
\end{galgas3}








\subsectionTypePredefiniLabelIndex{lsint64}

\begin{galgas3}
struct @lsint64 {
  @sint64 sint64
  @location location
}
\end{galgas3}







\subsectionTypePredefiniLabelIndex{lstring}

\begin{galgas3}
struct @lstring {
  @string string
  @location location
}
\end{galgas3}







\subsectionTypePredefiniLabelIndex{luint}

\begin{galgas3}
struct @luint {
  @uint uint
  @location location
}
\end{galgas3}





\subsectionTypePredefiniLabelIndex{luint64}


\begin{galgas3}
struct @luint64 {
  @uint64 uint64
  @location location
}
\end{galgas3}


\subsectionTypePredefiniLabelIndex{range}

Le type \ggst!@range! définit les intervalles d'entiers non signés 32 bits (\ggst!@uint!).

\begin{galgas3}
struct @range {
  @uint start
  @uint length
}
\end{galgas3}

La plupart des propriétés du type \ggst!@range! découle de cette définition (\refChapterPage{typeStructure}).

\ggst+@range.new {!a !b}+, où \ggst!a! et \ggst!b! sont des expressions de type \ggst!@uint!, représente~:
\begin{itemize}
  \item un intervalle vide si \ggst!b! est égal à zéro ;
  \item l'intervalle $[a, a+b-1]$ si \ggst!b! est strictement positif.
\end{itemize}



\subsubsectionLabel{Opérateurs \texttt{...} et \texttt{..<}}{operateurIntervalleRange}

Deux opérateurs permettent de construire plus facilement des objets de type \ggst!@range!.

L'opérateur \ggst!...! permet de définir un intervalle fermé à partir de sa borne inférieure et de sa borne supérieure~: si \ggst!a! et \ggst!b! sont des expressions de type \ggst!@uint!, l'expression \ggst!a ... b! est équivalente à la construction \ggst*@range.new {!a !b - a + 1}*. Une exception est levée si $b < a$.

L'opérateur \ggst!..<! permet de définir un intervalle ouvert à gauche à partir de sa borne inférieure et de sa borne supérieure~: si \ggst!a! et \ggst!b! sont des expressions de type \ggst!@uint!, l'expression \ggst!a ..< b! est équivalente à \ggst*@range.new {!a !b - a}*. Une exception est levée si $b < a$.

\subsubsection{Type \texttt{@range} et instruction \texttt{for}}

On peut utiliser une expression de type \ggst!@range! avec l'instruction \ggst!for!~:

\begin{galgas3}
for i in @range.new {!12 !5} do
  # i prend successivement les valeurs 12, 13, 14, 15, 16
end
\end{galgas3}

Et, avec l'opérateur \ggst!...!~:
\begin{galgas3}
for i in 12 ... 16 do
  # i prend successivement les valeurs 12, 13, 14, 15, 16
end
\end{galgas3}

Et l'opérateur \ggst!..<!~:
\begin{galgas3}
for i in 12 ..< 17 do
  # i prend successivement les valeurs 12, 13, 14, 15, 16
end
\end{galgas3}

Si l'on veut parcourir l'énumération à partir de la dernière valeur, on utilise le modificateur \ggst!>! après le mot-clé \ggst!for!~:
\begin{galgas3}
for > i in @range.new {!12 !5} do
  # i prend successivement les valeurs 16, 15, 14, 13, 12
end
\end{galgas3}

\begin{galgas3}
for > i in 12 ... 16 do
  # i prend successivement les valeurs 16, 15, 14, 13, 12
end
\end{galgas3}

\begin{galgas3}
for > i in 12 ..< 17 do
  # i prend successivement les valeurs 16, 15, 14, 13, 12
end
\end{galgas3}



  %!TEX encoding = UTF-8 Unicode
%!TEX root = ../galgas-book.tex

%--------------------------------------------------------------
\chapterLabel{Le type \texttt{extern}}{typeExtern}
%-------------------------------------------------------------

\tableDesMatieresLocaleDeProfondeurRelative{1}


Un type \ggst+extern+ est déclaré et spécifié en GALGAS, et implémenté par une classe C++. Ceci permet de définir des types qui seraient difficilement exprimables en GALGAS.

On va voir sur un exemple comment déclarer et implémenter :
\begin{itemize}
  \item un type externe minimum ;
  \item un constructeur ;
  \item un \emph{setter} ;
  \item une \emph{méthode} ;
  \item un \emph{getter} ;
  \item une \emph{méthode} de classe.
\end{itemize}

L'exemple consiste à implémenter le type \ggst+@complex+ qui représente les nombres complexes.

\section{Type externe minimum}

L'implémentation minimum ne sera pas opérationnelle, car elle ne comprendra pas de constructeur : on ne pourra donc pas instancier d'objet du type \ggst+@complex+. L'ajout de constructeur sera présenté à la section suivante. De même, cette implémentation minimum ne définira ni \emph{setter}, ni \emph{méthode}, ni \emph{getter}.


\subsection{Déclaration en GALGAS}

La description minimum est la suivante :
\begin{galgas3}
extern @complex {
  "// No Predeclaration\n"
}{
  "  private : bool mIsValid ;\n"
  "  private : double mReal ;\n"
  "  private : double mImaginary ;\n"
}{
}
\end{galgas3}

Cette description est divisée en trois parties, délimitées par les accolades \ggst+{+ et \ggst+}+.

\textbf{Première partie.} Elle cite une séquence de chaînes de caractères, qui seront écrites telles quelles dans le fichier d'en-tête C++ engendré, juste avant la déclaration de la classe C++ ; on peut y placer là des pré-déclarations de classe, des inclusions de fichier, … Pour le type \ggst+@complex+, aucune pré-déclaration n'est nécessaire, aussi on place un simple commentaire C++, de façon à le localiser dans le fichier d'en-tête C++ engendré.

\subsection{Implémentation en C++}






\section{Constructeur}


\section{Setter}


\section{Méthode}


\section{Getter}


\section{Méthode de classe}


  %!TEX encoding = UTF-8 Unicode
%!TEX root = ../galgas-book.tex

\chapter{Compléter le système de types}

\tableDesMatieresDuChapitre


\section{Ajouter une méthode , un \emph{getter}, un \emph{setter} ou un constructeur à un type prédéfini}

Ajouter une méthode, un \emph{getter}, un \emph{setter} ou un constructeur à un type prédéfini s'effectue en quatre temps :
\begin{enumerate}
\item ajouter la méthode, le \emph{getter}, le \emph{setter} ou le constructeur dans GALGAS ;
\item reconstruire le fichier d'en-tête des types prédéfinis ;
\item implémenter la méthode, le \emph{getter}, le \emph{setter} ou le constructeur en C++ ;
\item mettre à jour la documentation \LaTeX.
\end{enumerate}

À titre d'exemple, nous allons montrer comment la méthode \ggs!makeDirectoryAndWriteToExecutableFile! de la classe \ggs!@string! a été ajoutée.

\subsection{Ajouter la méthode dans GALGAS}

Éditez le fichier GALGAS, en fonction du \refTableau{ajoutMethodeTypePrédéfini}. Comme c'est une méthode que nous voulons ajouter, on édite le fichier \texttt{galgas-sources/semanticsInstanceMethods.galgas}.

Dans ce fichier, il y a une méthode pour chaque type prédéfini. Pour la classe \ggs!@string!, on a :
\begin{galgas}
override method @stringPredefinedTypeAST getInstanceMethodMap
  ?!@unifiedTypeMap ioUnifiedTypeMap
  !@instanceMethodMap outInstanceMethodMap
{
  outInstanceMethodMap = {}
  enterInstanceMethodWithInputArgument (
    !?outInstanceMethodMap
    !?ioUnifiedTypeMap
    !inputArgTypeName:"string"
    !inputArgName:"inFilePath"
    !methodName:"writeToFile"
    !true
  )
  ...
\end{galgas}

\begin{table}[t]
  \centering
  \begin{tabular}{lll}
    \textbf{Opération} & \textbf{Fichier} \\
    Ajouter un constructeur &  \tpp{galgas-sources/semanticsConstructors.galgas}\\
    Ajouter un \emph{getter} &  \tpp{galgas-sources/semanticsGetters.galgas}\\
    Ajouter un \emph{setter} &  \tpp{galgas-sources/semanticsSetters.galgas}\\
    Ajouter une méthode &  \tpp{galgas-sources/semanticsInstanceMethods.galgas}\\
    Ajouter une méthode de type &  \tpp{galgas-sources/semanticsTypeMethods.galgas}\\
  \end{tabular}
  \caption{Fichier GALGAS à éditer pour compléter un type prédéfini}
  \labelTableau{ajoutMethodeTypePrédéfini}
\end{table}

Pour ajouter la méthode \ggs!makeDirectoryAndWriteToExecutableFile! de la classe \ggs!@string!, on complète cette méthode par :
\begin{galgas}
override method @stringPredefinedTypeAST getInstanceMethodMap
  ?!@unifiedTypeMap ioUnifiedTypeMap
  !@instanceMethodMap outInstanceMethodMap
{
  outInstanceMethodMap = {}
  ...
  enterInstanceMethodWithInputArgument (
    !?outInstanceMethodMap
    !?ioUnifiedTypeMap
    !inputArgTypeName:"string"
    !inputArgName:"inFilePath"
    !methodName:"makeDirectoryAndWriteToExecutableFile"
    !true
  )
}
\end{galgas}

\subsection{Reconstruire le fichier d'en-tête des types prédéfinis}

Le fichier \tpp{libpm/galgas2/predefined-types.h} contient la déclaration C++ de tous les types prédéfinis. Le fichier \tpp{libpm/galgas2/predefined-types.cpp} contient l'implémentation des constructions génériques des types prédéfinis. {\bf Surtout n'éditez pas ces fichiers à la main !} On va utiliser GALGAS pour les reconstruire. Pour cela, appeler le script Shell \tpp{libpm/galgas2/-build-builtin-type-headers.command}. L'exécution de celui-ci recompile GALGAS, et engendre les nouvelles versions des fichiers \tpp{predefined-types.h} et \tpp{predefined-types.cpp}. Voici ce que l'on obtient :

\begin{mdframed}[hidealllines=true,backgroundcolor=gray!10] \tt\small
\textcolor{blue}{Native Compiling for Mac OS X (debug): all-declarations-26.cpp}\\
...\\
\textcolor{blue}{Native Compiling for Mac OS X (debug): check-gmp.cpp}\\
\textcolor{blue}{Native Linking for Mac OS X (debug): galgas-debug}\\
Done at +12s\\
\textcolor{OliveGreen}{Replaced '/Volumes/dev-svn/galgas/libpm/galgas2/predefined-types.h'.}\\
No warning, no error.\\
\verb![!Displayed from file 'all-declarations-19.cpp' at line 952\verb!]!\\
11800 memory blocks, 4158 arraies have been used.\\
7052 POD arraies have been used, 706 have been reallocated (509 with pointer change).
\end{mdframed}

Ici, seul le fichier \tpp{predefined-types.h} a été modifié, le fichier \tpp{predefined-types.cpp} ne nécessitait pas de modification.



\subsection{Implémenter la méthode en C++}

Maintenant, effectuer la compilation C++ du projet GALGAS, soit avec Xcode, soit avec le \emph{makefile} natif de votre choix. {\bf Il est normal 
que cette compilation échoue, la méthode n'a pas encore été implémentée.}

Éditez le fichier \tpp{libpm/galgas2/GALGAS\_string.cpp} et ajouter la méthode :

\begin{lstlisting}[language=C++]
void GALGAS_string::
method_makeDirectoryAndWriteToExecutableFile (GALGAS_string inFilePath,
                                              C_Compiler * inCompiler
                                              COMMA_LOCATION_ARGS) const {
  if (isValid () && inFilePath.isValid ()) {
  //--- Make directory
    const C_String directory = inFilePath.mString.stringByDeletingLastPathComponent () ;
    bool ok = C_FileManager::makeDirectoryIfDoesNotExist (directory) ;
    if (! ok) {
      C_String message ;
      message << "cannot create '" << directory << "' directory" ;
      inCompiler->onTheFlyRunTimeError (message COMMA_THERE) ;
    }else{
      method_writeToExecutableFile (inFilePath, inCompiler COMMA_THERE) ;
    }
  }
}
\end{lstlisting}

Maintenant la compilation C++ de GALGAS s'effectue correctement. Mais ce n'est pas terminé !

\subsection{Finaliser le nouveau compilateur GALGAS}

L'exécutable GALGAS embarque le source de la libraire \tpp{libpm}. Or, à ce stade, c'est l'ancienne version qui est empbarquée. Lorsque l'on compile le projet \tpp{+galgas.galgasProject}, la librairie \tpp{libpm} est intégrée dans les sources C++ engendrés. 

Il faut donc effectuer itérativement des cycles \emph{compilation GALGAS} -- \emph{compilation C++} tant que la compilation GALGAS apporte des modifications du code C++ engendré.








\part{Sous-programmes}
%!TEX encoding = UTF-8 Unicode
%!TEX root = ../galgas-book.tex

%--------------------------------------------------------------
\chapterLabel{Sous-programmes}{sousprgm}\index{Sous-programmes}
%-------------------------------------------------------------

GALGAS définit les sous-programmes suivants :
\begin{itemize}
  \item les \emph{fonctions} (dans ce chapitre, \refSectionPage{declarationFonction}) ;
  \item les \emph{procédures} (dans ce chapitre, \refSectionPage{declarationProcedure}) ;
  \item les \emph{méthodes} (\refSectionPage{categoryMethod}) ;
  \item les \emph{getters} (\refSectionPage{categoryReader}) ;
  \item les \emph{setters} (\refSectionPage{categoryModifier}).
\end{itemize}

En GALGAS, \emph{méthodes}, \emph{getters} et \emph{setters} s'appliquent sur un objet d'un type quelconque (qui n'est donc pas forcément un type \emph{classe}). Pour les types définis par l'utilisateur, \emph{méthodes}, \emph{getters} et \emph{setters} sont toujours déclarés en dehors de la déclaration du type auquel ils s'appliquent.


À chaque nature de sous-programme correspond une construction particulière pour l'appeler (\refTableau{appelSousProgramme}).

\begin{table}[t]
  \centering
    \begin{tabular}{lll}
      \textbf{Sous-programme} & \textbf{Construction}  & \textbf{Référence} \\
      \emph{routine} & Instruction d'appel de routine & \refSectionPage{appelRoutine} \\
      \emph{fonction} & Appel de fonction (dans une expression) & \refSubsectionPage{appelFonction} \\
      \emph{méthode} & Instruction d'appel de méthode & \refSectionPage{methodCallInstruction} \\
      \emph{getter} & Appel de getter (dans une expression) & \refSubsectionPage{appelReader} \\
      \emph{setter} & Instruction d'appel de setter & \refSectionPage{modifierCallInstruction} \\
    \end{tabular}
  \caption{Constructions d'appel de sous programme}
  \labelTableau{appelSousProgramme}
  \ligne
\end{table}








\sectionLabel{Arguments formels et paramètres effectifs}{correspondanceArgFormelsParametresEffectifs}

\subsection{Argument formel en entrée}

Le \refTableau{ArgumentFormelEntree} liste les différentes formes d'un argument formel en entrée. Le paramètre effectif correspondant est une expression précédée par \galgas{\!}.

\begin{table}[t]
  \centering
  \subfloat[Argument formel]{
    \begin{tabular}{ll}
      \textbf{Syntaxe} & \textbf{Remarque} \\
      \galgast{?}\galgas{@T variable} & \galgas{variable} est modifiable localement \\
      \galgast{?}\galgas{@T unused variable} & \galgas{variable} n'est pas utilisée \\
      \galgast{?}\galgas{let @T constante} & \galgas{constante} est une constante \\
      \galgast{?}\galgas{let @T unused constante} & \galgas{constante} est une constante inutilisée \\
    \end{tabular}
  }
  \subfloat[Paramètre effectif]{
    \begin{tabular}{ll}
      \textbf{Syntaxe} \\
      \galgas{\!expression}~~~~~~~~~~~\\
      \\
      \\
      \\
    \end{tabular}
  }
  \caption{Argument formel en entrée, paramètre effectif en sortie}
  \labelTableau{ArgumentFormelEntree}
  \ligne
\end{table}

\subsection{Argument formel en entrée/sortie}

Le \refTableau{ArgumentFormelEntreeSortie} liste les différentes formes d'un argument formel en entrée. Le paramètre effectif correspondant est une \emph{cible} précédée par \galgast{!?}. Une \emph{cible} est soit une variable, soit l'accès à un champ d'une variable de type \galgas{struct}.

\begin{table}[t]
  \centering
  \subfloat[Argument formel]{
    \begin{tabular}{ll}
      \textbf{Syntaxe} & \textbf{Remarque} \\
      \galgast{?!}\galgas{@T variable} & \galgas{variable} est modifiable \\
      \galgast{?!}\galgas{@T unused variable} & \galgas{variable} n'est pas utilisée \\
    \end{tabular}
  }
  \subfloat[Paramètre effectif]{
    \begin{tabular}{l}
      \textbf{Syntaxe} \\
      \galgast{!?}\galgas{cible}~~~~~~~~~~~\\
      \\
    \end{tabular}
  }
  \caption{Argument formel en entrée/sortie, paramètre effectif en sortie/entrée}
  \labelTableau{ArgumentFormelEntreeSortie}
  \ligne
\end{table}

\subsection{Argument formel en sortie}

\begin{table}[t]
  \centering
  \subfloat[Argument formel]{
    \begin{tabular}{ll}
      \textbf{Syntaxe} \\
      \galgas{\!@T var}~~~~~~~~~~~~\\
      \\
      \\
      \\
    \end{tabular}
  }
  \subfloat[Paramètre effectif]{
    \begin{tabular}{ll}
      \textbf{Syntaxe} & \textbf{Remarque} \\
      \galgast{?}\galgas{variable} & Affectation de variable \\
      \galgast{?}\galgas{@T variable} & Déclaration et affectation de variable \\
      \galgast{?}\galgas{*} & Variable anonyme \\
      \galgast{?}\galgas{let @T constante} & Déclaration et affectation de constante \\
    \end{tabular}
  }
  \caption{Argument formel en sortie, paramètre effectif en entrée}
  \labelTableau{ArgumentFormelSortie}
  \ligne
\end{table}






\sectionLabel{Liste d'arguments formels en entrée, en sortie, ou en entrée/sortie}{listeArgumentsFormels}






\sectionLabel{Liste de paramètres effectifs en entrée}{listeParametresEffectifsEntree}




%!TEX encoding = UTF-8 Unicode
%!TEX root = ../galgas-book.tex

%--------------------------------------------------------------
\chapterLabel{Fonctions et procédures}{fonctions}
%-------------------------------------------------------------

GALGAS définit les sous-programmes suivants :
\begin{itemize}
  \item les \emph{fonctions} (dans ce chapitre, \refSectionPage{declarationFonction}) ;
  \item les \emph{procédures} (dans ce chapitre, \refSectionPage{declarationProcedure}) ;
  \item les \emph{méthodes} (\refSectionPage{categoryMethod}) ;
  \item les \emph{getters} (\refSectionPage{categoryGetter}) ;
  \item les \emph{setters} (\refSectionPage{categorySetter}).
\end{itemize}

\sectionLabel{Fonction}{declarationFonction}

Une fonction GALGAS n'accepte que des arguments en entrée, et retourne une valeur. Elle est appelée dans une expression (\refSubsectionPage{appelFonction}).

\subsection{déclaration d'une fonction}\index{Fonction!Déclaration}

\begin{galgas}
private # Optionnel
func nom_fonction liste_arguments_entree -> @T var_resultat {
  liste_instructions
}
\end{galgas}

Une fonction est désignée par \ggs+nom_fonction+. Ce nom est unique dans un projet GALGAS. La liste des paramètres d'entrée peut être vide (\refSectionPage{listeParametresEffectifsEntree}). La valeur renvoyée par l'exécution de la fonction est la valeur de \ggs+var_resultat+ à l'issue de l'exécution de la \ggs+liste_instructions+. Aussi, l'exécution de la \ggs+liste_instructions+ doit valuer \ggs+var_resultat+.

Exemple :

\begin{galgas}
func produit ?@uint a ?@uint b -> @uint resultat {
  resultat = a * b
}
\end{galgas}



\subsection{Fonction interne à un fichier}

En préfixant la déclaration d'une fonction par \ggs+private+, on limite son appel aux expressions situées dans le même fichier que la déclaration.




\subsection{Fonction \texttt{\%once}}

Une fonction sans argument accepte le qualificatif \ggs+%once+ :

\begin{galgas}
func %once masque-> @uint resultat {
  resultat = 1 << 16
}
\end{galgas}

Le qualificatif \ggs+%once+ organise le cache du résultat : celui-ci est calculé lors du premier appel, est mémorisé internement, et est retourné directement lors des appels ultérieurs.

Une fonction \ggs+%once+ peut être déclarée interne en la préfixant par \ggs+private+.

\begin{galgas}
private func %once masque-> @uint resultat {
  resultat = 1 << 16
}
\end{galgas}





\sectionLabel{Procédure}{declarationProcedure}\index{Procédure!Déclaration}

Une procédure GALGAS accepte que des arguments en entrée, en sortie, en entrée/sortie. Elle est appelée dans une instruction (\refSectionPage{instructionAppelProcedure}).

\subsection{Déclaration d'une procédure}\index{Procédure!Déclaration}

\begin{galgas}
private # Optionnel
proc nom_procedure liste_arguments {
  liste_instructions
}
\end{galgas}

Une procédure est désignée par \ggs+nom_procedure+. Ce nom est unique dans un projet GALGAS. La liste des paramètres en entrée, en sortie ou en entrée/sortie est décrite à la \refSectionPage{listeArgumentsFormels}).

Exemple :

\begin{galgas}
proc produit ?@uint a ?@uint b !@uint resultat {
  resultat = a * b
}
\end{galgas}



\subsection{Procédure interne à un fichier}

En préfixant la déclaration d'une procédure par \ggs+private+, on limite son appel aux instructions situées dans le même fichier que la déclaration.




%!TEX encoding = UTF-8 Unicode
%!TEX root = ../galgas-book.tex

%--------------------------------------------------------------
\chapterLabel{Extensions}{extensions}\index{Extensions}
%-------------------------------------------------------------

\emph{Categories} are the way for adding \emph{getters}, \emph{methods} and \emph{setters} to any type. They are defined outside type declarations.

You can declare for any type:
\begin{itemize}
\item \emph{category getters};
\item \emph{category methods};
\item \emph{category setters}.
\end{itemize}

Additional features are available for classes and are described in \refSectionPage{categoriesAndClasses}.


A \emph{category getter} is called in an expression. As expressions have no side-effect, a category getter cannot change current object's value.

A \emph{category method} is called by the \emph{method call instruction} (\refSectionPage{methodCallInstruction}). A category method cannot modify current object's value.

A \emph{category setter} is called by the \emph{setter call instruction} (\refSectionPage{setterCallInstruction}). A category setter can modify current object's value.

Within the category getter, method and setter instruction list, the \lstinline[language=galgas]!self! key word is allowed in any expression. It represents a copy of the current object. Of course, the current is lazily copied only when required.

The \lstinline[language=galgas]!self! key word is just a syntactic tag for representing a write or a read/write access to the current object. Using \lstinline[language=galgas]!self! is not allowed in category methods and category getters since they cannot modify the current object. Using \lstinline[language=galgas]!self! in category setters is explained in \refSectionPage{categorySetter}. 


A category getter, method and setter can be declared in:
\begin{itemize}
\item a \emph{semantics} component;
\item a \emph{syntax} component;
\item a \emph{program} component.
\end{itemize}

A declared category getter, method and setter has a global scope, meaning it is available in the current component, and in any component that includes it directly or indirectly.

A type does not accept several category getters with the same name. During compilation of the project file, the project global checking mechanism detects such declarations and issues an error. Consequently, it is forbidden to declare a category getter with the same name than a predefined getter: the compiler issues an error on on a such declaration. The same rules apply on category methods and category setters.

However, it is safe to declare for a given type a category getter, a category method and a category setter with the same name. GALGAS compiler uses different naming spaces for them, and call syntax are different, so there is no ambiguity.










\sectionLabel{Category getter}{categoryGetter}

A category getter is declared like a function, but its header names a type and a getter name. As a function, it accepts zero, one or more input and constant input formal parameters.

For example, the following code add a getter to the \ggs+uint64+ that computes the square of its value:
\begin{lstlisting}[language=galgas]
getter @uint64 square -> @uint64 outResult {
  outResult = self * self
}
\end{lstlisting}

This getter is called like a predefined getter:
\begin{lstlisting}[language=galgas]
@uint64 v = 7L
log "Square of 7": [v square] # LOGGING Square of 7 : <@uint64:49>
\end{lstlisting}

You can add a category getter to a list :
\begin{lstlisting}[language=galgas]
getter @uintlist sum -> @uint outResult {
  outResult = 0
  for self do
    outResult = outResult + mValue
  }
}
\end{lstlisting}

For counting the number of element values greater than the value given in argument:
\begin{lstlisting}[language=galgas]
getter @uintlist countValuesGreaterThan
  ?let @uint inTestValue -> @uint outResult
{
  outResult = 0
  for self do
    if mValue > inTestValue then
      outResult ++
    end if
  }
}
\end{lstlisting}

When used with a struct or class type, current object attributes values can be read by naming the attribute in an expression. For example, the \refTypePredefini{lstring} has an attribute 
\lstinline[language=galgas]!string! whose type is \ggs+@string+. The following getter returns the value of this attribute, appended with the \lstinline[language=galgas]?" !"? string:
\begin{lstlisting}[language=galgas]
getter @lstring op -> @string outResult {
  outResult = string . " !"
}
\end{lstlisting}







\sectionLabel{Category method}{categoryMethod}

A category method is declared like a routine, but its header names a type and a method name. As a routine, it accepts zero, one or more input, output, input/output constant input formal parameters.

For example, the following code add a method to the \ggs+uint64+ that computes the square of its value:
\begin{lstlisting}[language=galgas]
method @uint64 square !@uint64 outResult {
  outResult = self * self
}
\end{lstlisting}

This getter is called like a predefined method:
\begin{lstlisting}[language=galgas]
@uint64 v
[7L square ?v]
log "Square of 7": v # LOGGING Square of 7 : <@uint64:49>
\end{lstlisting}

You can add a category method to a list :
\begin{lstlisting}[language=galgas]
method @uintlist sum !@uint outResult {
  outResult = 0
  for self do
    outResult = outResult + mValue
  }
}
\end{lstlisting}

For counting the number of element values greater than the value given in argument:
\begin{lstlisting}[language=galgas]
method @uintlist countValuesGreaterThan
  ?let @uint inTestValue
  !@uint outResult
{
  outResult = 0
  for self do
    if mValue > inTestValue then
      outResult ++
    end if
  }
}
\end{lstlisting}

When used with a struct or class type, current object attributes values can be read by naming the attribute in an expression. For example, the \refTypePredefini{lstring} has an attribute 
\lstinline[language=galgas]!string! whose type is \ggs+@string+. The following method returns the value of this attribute, appended with the \lstinline[language=galgas]?" !"? string:
\begin{lstlisting}[language=galgas]
method @lstring op !@string outResult {
  outResult = string . " !"
}
\end{lstlisting}











\sectionLabel{Category setter}{categorySetter}

A category method is declared like a routine, but its header names a type and a setter name. As a routine, it accepts zero, one or more output, input/output, input and constant input formal parameters. Unlike a category method, a category setter can change the value of the current object.

For structure and classes types, attributes can be read, written, read / written. For example :
\begin{lstlisting}[language=galgas]
setter @lstring appendInt ?let @uint inValue {
  string .= [inValue string]
}
\end{lstlisting}


The \lstinline[language=galgas]!self! key word is used as a syntactic tag for denoting a read/write or a write access on the current object. This key word is syntactically accepted in the following constructs:
\begin{enumerate}
\item the \emph{setter call instruction} (\refSectionPage{setterCallInstruction});
\item the \emph{concat instruction} (\refSectionPage{concatInstruction});
\item the \emph{increment instruction} (\refSectionPage{incrementInstruction});
\item the \emph{decrement instruction} (\refSectionPage{decrementInstruction});
\item the \emph{assignment instruction} (\refSectionPage{assignmentInstruction}).
\end{enumerate}

Example of using \lstinline[language=galgas]!self! in setter call instruction; this setter prepends the square of argument value to the \galgas{@uint64list} value:
\begin{lstlisting}[language=galgas]
setter @uint64list prependSquare ?let @uint64 inValue {
  [!?self prependValue !inValue * inValue]
}
\end{lstlisting}


Example of using \lstinline[language=galgas]!self! in the append instruction; this setter appends the square of argument value to the \galgas{@uintlist} value:
\begin{lstlisting}[language=galgas]
setter @uintlist appendSquare ?let @uint inValue {
  self += !inValue * inValue
}
\end{lstlisting}
This construct is valid only for types that handle the \lstinline[language=galgas]!+=! operator.


Example of using \lstinline[language=galgas]!self! in the concat instruction; this setter appends to the string all items of the \galgas{@stringlist} argument value:
\begin{lstlisting}[language=galgas]
setter @string concatList ?let @stringlist inList {
  for inList do
    self .= mValue
  }
}
\end{lstlisting}
This construct is valid only for types that handle the \lstinline[language=galgas]!.=! operator.




Example of using \lstinline[language=galgas]!self! in the increment instruction; this setter increments the receiver's value:
\begin{lstlisting}[language=galgas]
setter @uint increment {
  self ++
}
\end{lstlisting}
This construct is valid only for types that handle the \lstinline[language=galgas]!++! operator, such as \refTypePredefini{uint}, \ggs+uint64+, \refTypePredefini{sint}, \refTypePredefini{sint64}.





Example of using \lstinline[language=galgas]!self! in the assignment instruction; this setter removes all odd values of the receiver:
\begin{lstlisting}[language=galgas]
setter @uintlist removeOddValues {
  @uintlist listWithEvenValues [emptyList]
  for self do
    if (mValue & 1) == 0 then
      listWithEvenValues += !mValue
    end if
  }
  self = listWithEvenValues
}
\end{lstlisting}
This construct is valid only for types, but class types.










\sectionLabel{Categories and classes}{categoriesAndClasses}


Additional features are available only for classes; in addition to category getters, methods and setters described in the above sections, you can declare:
\begin{itemize}
\item \emph{abstract} category getters, methods, setters;
\item \emph{overriding} category getters, methods, setters;
\item \emph{overriding abstract} category getters, methods, setters.
\end{itemize}

\emph{Abstract} category getters, methods, setters and \emph{overriding abstract} category getters, methods, setters do not contain any instruction list: they act as \emph{prototypes}.

Examples of \emph{abstract} category getters, methods, setters declarations :
{\lstset{emph={getterName, methodName, setterName}, emphstyle=\galgasEmphStyle}
\begin{lstlisting}[language=galgas]
abstract getter @aType getterName
  ?anOtherType aParameter
  -> @resultType outResult
;

abstract method @aType methodName
  ?anOtherType aParameter
;

abstract setter @aType setterName
  ?anOtherType aParameter
;
\end{lstlisting}
}

Examples of \emph{overriding} category getters, methods, setters declarations :
{\lstset{emph={instructions, getterName, methodName, setterName}, emphstyle=\galgasEmphStyle}
\begin{lstlisting}[language=galgas]
override getter @aType getterName
  ?anOtherType aParameter
  -> @resultType outResult
{
  instructions
}

override method @aType methodName
  ?anOtherType aParameter
{
  instructions
}

override setter @aType setterName
  ?anOtherType aParameter
{
  instructions
}
\end{lstlisting}
}

Examples of \emph{overriding abstract} category getters, methods, setters declarations :
{\lstset{emph={getterName, methodName, setterName}, emphstyle=\galgasEmphStyle}
\begin{lstlisting}[language=galgas]
override abstract getter @aType getterName
  ?@anOtherType aParameter
  -> @resultType outResult
;

override abstract method @aType methodName
  ?anOtherType aParameter
;

override abstract setter @aType setterName
  ?anOtherType aParameter
;
\end{lstlisting}
}


Neither \emph{abstract} category getters, methods, setters, neither \emph{overriding abstract} category getters, methods, setters cannot be declared for concrete classes. Any kind of category getter, method, setter can be declared for abstract classes.

If an \emph{abstract} category getter, method, setter, or an \emph{overriding abstract} category getter, method, setter is declared for an abstract class, it should be also declared as \emph{overriding} with the same name for every first concrete successor class.

A category getter, method, setter that has the same name as a category getter, method, setter declared for one of its super classes should be declared as \emph{overriding}.

An abstract category getter, method, setter that has the same name as a category getter, method, setter declared for one of its super classes should be declared as \emph{overriding abstract}.

The following example illustrates how theses rules should be applied. In the \refFigure{}{figureCategoryExample}, four classes are shown. An arrow links a class to its super class. The \lstinline[language=galgas]!@A! and \lstinline[language=galgas]!@C! classes are abstract. \lstinline[language=galgas]!m1! is a name for any getter, method or setter.

\begin{figure}[t]
  \centering
   \begin{tikzpicture}[every state/.style={draw, circle}, node distance=1.5cm, <-, thick]
     \node at (0, 4.5) [state, accepting] (A) {\lstinline[language=galgas]!@A!};
     \node at (0, 3) [state]            (B) {\lstinline[language=galgas]!@B!};
     \node at (0, 1.5) [state, accepting] (C) {\lstinline[language=galgas]!@C!};
     \node at (0, 0) [state]            (D) {\lstinline[language=galgas]!@D!};
     \path (A) edge (B) ;
     \path (B) edge (C) ;
     \path (C) edge (D) ;
     \node at (1.5, 4.5) [right] {\lstinline[language=galgas]!abstract @A m1!} ;
     \node at (1.5, 3) [right] {\lstinline[language=galgas]!override @B m1!} ;
     \node at (1.5, 1.5) [right] {\lstinline[language=galgas]!override abstract @C m1!} ;
     \node at (1.5, 0) [right] {\lstinline[language=galgas]!override @D m1!} ;
   \end{tikzpicture}
  \caption{inheritance graph and categories}
  \labelFigure{figureCategoryExample}
  \ligne
\end{figure}

\lstinline[language=galgas]!m1! is declared as \lstinline[language=galgas]!abstract! for the \lstinline[language=galgas]!@A! class. It is allowed since \lstinline[language=galgas]!@A! is abstract. Consequently, the concrete \lstinline[language=galgas]!@B! class should override \lstinline[language=galgas]!m1!. The \lstinline[language=galgas]!@C! class is also abstract, and \lstinline[language=galgas]!m1! can be declared as \lstinline[language=galgas]!abstract! for this class. But as it has been also declared for one of theses super class, it should also declared as \lstinline[language=galgas]!override!. As \lstinline[language=galgas]!@D! is concrete, \lstinline[language=galgas]!m1! should be declared for this class with \lstinline[language=galgas]!override! tag.

















\part{Filewrappers et templates}
%!TEX encoding = UTF-8 Unicode
%!TEX root = ../galgas-book.tex

%--------------------------------------------------------------
\chapterLabel{Filewrappers}{filewrapper}
%-------------------------------------------------------------

Un \emph{filewrapper} permet d'embarquer dans le code engendré une arborescence de fichiers. Comme on va le voir dans la section suivante, la déclaration d'un \emph{filewrapper} désigne un répertoire, qui va être exploré au moment de la compilation GALGAS de façon à embarquer dans le code engendré la copie de certains fichiers. Ces  fichiers peuvent être de trois sortes :
\begin{itemize}
  \item des fichiers \emph{texte} ; ils sont sélectionnés par leur extension : la déclaration d'un \emph{filewrapper} liste toutes les extensions des fichiers texte embarqués ;
  \item des fichiers \emph{binaires} ; de même, ils sont sélectionnés par leur extension, et la déclaration d'un \emph{filewrapper} liste toutes les extensions des fichiers binaires embarqués ;
  \item des \emph{templates}, qui sont sélectionnés par leur nom ; ils sont analysés lors de leur lecture.
\end{itemize}


L'exploration des fichiers embarqués peut s'effectuer soit de manière statique, soit dynamique à l'aide d'un objet de \refTypePredefini{filewrapper}.









\section{Déclararation d'un \texttt{filewrapper}}

Un \emph{filewrapper} peut être déclaré dans un composant \emph{syntax}, \emph{semantics} ou \emph{program}. Sa déclaration est la suivante :

\begin{galgascode}
filewrapper nom in "chemin" {
 "extension_texte", ...
}{
 "extension_binaire", ...
}{
 declaration_de_templates
}
\end{galgascode}

Où :
\begin{itemize}
  \item \galgas{nom} est le nom, interne à GALGAS, donné au \emph{filewrapper} ; ce nom doit être unique à chaque \emph{filewrapper} ;
  \item \galgas{"chemin"} est le chemin du répertoire qui va être exploré récursivement au moment de la compilation ; c'est soit un chemin absolu (il commence par un \galgas{/}), soit un chemin relatif, par rapport au répertoire qui contient le fichier source qui déclare le \emph{filewrapper}.
\end{itemize}

La déclaration est divisée en trois parties délimitées par des accolades \galgas{\{ ... \}} :
\begin{itemize}
  \item la première partie (\galgas{"extension_texte", ...}) liste les extensions des fichiers texte qui sont embarqués ; à la compilation GALGAS, le répertoire désigné est exploré récursivement, et les fichiers dont l'extension est l'une des extensions citées sont embarqués, ainsi que leurs chemins relatifs ;
  \item la deuxième partie (\galgas{"extension_binaire", ...}) liste les extensions des fichiers binaires qui sont embarqués ; de même, à la compilation GALGAS, le répertoire désigné est exploré récursivement, et les fichiers dont l'extension est l'une des extensions citées sont embarqués, ainsi que leurs chemins relatifs ;
  \item la troisième et dernière partie (\galgas{declaration_de_templates}) contient les déclarations de \emph{templates}.
\end{itemize}

Chacune de ces parties peut être vide si on ne veut pas embarquer de ficher ou ne définir aucun template.










\part{Instructions et expressions}
%!TEX encoding = UTF-8 Unicode
%!TEX root = ../galgas-book.tex

%--------------------------------------------------------------
\chapter{Contrôle de l'accès aux variables et aux constantes}
%-------------------------------------------------------------

\tableDesMatieresLocaleDeProfondeurRelative{1}


%---------------------- PARAMÉTRAGE DE L'AFFICHAGE DES AUTOMATES ------------------------------
% https://en.wikibooks.org/wiki/LaTeX/Colors

\newcommand\FondAutomate{LightGray!15}

\newcommand\VertAutomate{PineGreen}
\newcommand\OrangeAutomate{Orange}
\newcommand\RougeAutomate{Red}

\newcommand\EtatVert[4]{
  \node[draw=\VertAutomate,thick,fill=\FondAutomate,chamfered rectangle] (#1) at (#3, #4) {\tt\small #2};
}

\newcommand\EtatOrange[4]{
  \node[draw=\OrangeAutomate,thick,fill=\FondAutomate,chamfered rectangle] (#1) at (#3, #4) {\tt\small #2};
}

\newcommand\EtatRouge[4]{
  \node[draw=\RougeAutomate,thick,fill=\FondAutomate,chamfered rectangle] (#1) at (#3, #4) {\tt\small #2};
}

\newcommand\FlecheVerte[4]{
  \draw [-stealth, thick, \VertAutomate]
    (#1) edge[#4] node[draw=\VertAutomate,rounded corners,fill=\FondAutomate] {\tt\small #2} (#3) ;
}

\newcommand\FlecheOrange[4]{
  \draw [-stealth, thick, \OrangeAutomate]
    (#1) edge[#4] node[draw=\OrangeAutomate,rounded corners,fill=\FondAutomate] {\tt\small#2} (#3) ;
}

\newcommand\FlecheRouge[4]{
  \draw [-stealth, thick, \RougeAutomate]
    (#1) edge[#4] node[draw=\RougeAutomate,rounded corners,fill=\FondAutomate] {\tt\small#2} (#3) ;
}

\newcommand\BoucleVerte[3]{
  \path (#1) edge [loop #2,-stealth, thick, \VertAutomate] node[draw=\VertAutomate,rounded corners,fill=\FondAutomate]
    {\tt\small#3} (#1) ;
}

\newcommand\BoucleOrange[3]{
  \path (#1) edge [loop #2,-stealth, thick, \OrangeAutomate] node[draw=\OrangeAutomate,rounded corners,fill=\FondAutomate]
    {\tt\small#3} (#1) ;
}

\newcommand\BoucleRouge[3]{
  \path (#1) edge [loop #2,-stealth, thick, \RougeAutomate] node[draw=\RougeAutomate,rounded corners,fill=\FondAutomate]
    {\tt\small#3} (#1) ;
}

%\newcommand\ActionInterdite[3]{
%  \node[#2 of #1,draw=\RougeAutomate,thick,fill=\FondAutomate] (Z) {~};
%  \draw[-stealth,thick,\RougeAutomate] (#1) edge node[draw=\RougeAutomate,rounded corners,fill=\FondAutomate] {\it #3} (Z);
%}

\newcommand\FlecheEtatInitial[2]{
  \node[#2 of #1,draw=black,circle,thick,fill=\FondAutomate] (Z) {};
  \draw[-stealth,thick,black] (Z) -- (#1);
}

%----------------------------------------------------------------------------------------------

Le compilateur GALGAS effectue une surveillance très stricte des accès aux objets -- constantes, variables et paramètres formels. Il signale ainsi par des \emph{alertes} et des \emph{erreurs} tout violation des règles d'accès.

On peut illustrer le résultat de cette surveillance par le fragment de code suivant :
\begin{galgas3}
var @uint x
if condition then
  x = 2
end
let y = x # Une erreur de compilation est déclenchée ici
\end{galgas3}

Quelle serait la valeur de la variable \ggst!x! à l'issue de l'exécution de ce code ? Si \ggst!condition! est vrai, \ggst!x! vaut $2$ ; sinon, \ggst!x! n'a pas de valeur.

Le compilateur GALGAS détecte cette situation et considère que la variable est initialisée si elle l'est par toutes les branches de l'instruction \ggst!if!. Dans le cas contraire, comme ci-dessus, elle est considérée comme non initialisée. Aussi sa lecture déclenche un message d'erreur. Pour que l'analyse sémantique ne détecte pas d'erreur, il faut donc que les deux branches affectent une valeur à \ggst!x! :

\begin{galgas3}
var @uint x
if condition then
  x = 2
else
  x = 4
end
let y = x # Ok
\end{galgas3}

Pour contrôler le bon usage des variables et des constantes locales, le compilateur GALGAS associe pendant la compilation un automate d'états finis à chaque variable locale.














\section{Variable locale}

L'automate d'états finis associé à une variable locale est présenté à la \refFigure{}{automateEtatsVariableLocale}. C'est la forme la plus générale, les autres automates s'en déduisent.


\begin{figure}[ht!]
  \centering
  \small
  \begin{tikzpicture}
    \EtatVert{INVALID_STATE}{INVALID}{0cm}{0cm}
    \EtatOrange{DECLARED_STATE}{DECLARED}{3cm}{0cm}
    \EtatOrange{INITIALIZED_STATE}{INITIALIZED}{6cm}{0cm}
    \EtatOrange{READ_STATE}{READ}{9cm}{0cm}
    \EtatVert{MUTATED_STATE}{MUTATED}{12cm}{0cm}

    \FlecheEtatInitial{DECLARED_STATE}{above = 1cm}

    \BoucleVerte{INVALID_STATE}{above}{read}
    \BoucleVerte{INVALID_STATE}{below}{write}

    \FlecheVerte{DECLARED_STATE}{write}{INITIALIZED_STATE}{bend left}
    \FlecheRouge{DECLARED_STATE}{read}{INVALID_STATE}{bend right}
%    \ActionInterdite{DECLARED_STATE}{left = 2cm}{read}

    \BoucleOrange{INITIALIZED_STATE}{below}{write}
    \FlecheVerte{INITIALIZED_STATE}{read}{READ_STATE}{bend left=35}

    \FlecheVerte{READ_STATE}{write}{MUTATED_STATE}{bend left=35}
    \BoucleVerte{READ_STATE}{above}{read}

    \BoucleVerte{MUTATED_STATE}{above}{read}
    \BoucleVerte{MUTATED_STATE}{below}{write}
  \end{tikzpicture}
  \caption{Automate des états d'une variable locale}
  \labelFigure{automateEtatsVariableLocale}
\end{figure}



Les états sont les suivants~:
\begin{itemize}
  \item \texttt{INVALID}, une erreur d'accès a été détectée ;
  \item \texttt{DECLARED}, état d'une variable déclarée non initialisée ;
  \item \texttt{INITIALIZED}, état d'une variable déclarée et initialisée, mais jamais lue ;
  \item \texttt{READ}, état d'une variable déclarée, initialisée, lue au moins une fois, mais jamais modifiée ;
  \item \texttt{MUTATED}, état d'une variable déclarée, initialisée, et modifiée au moins une fois.
\end{itemize}

Un état initial est désigné par une flèche noire. La déclaration d'une variable locale place son automate dans l'état \texttt{DECLARED}~:

\begin{galgas3}
var @uint x # État 'declared'
\end{galgas3}

Il y a deux actions possibles~:
\begin{itemize}
  \item l'action \texttt{write}, qui exprime l'affectation d'une valeur à la variable~;
  \item l'action \texttt{read}, qui exprime la lecture de la valeur de la variable.
\end{itemize}

L'automate est \emph{complet}, c'est-à-dire que les deux actions sont prises en compte dans tous les états.

Les transitions sont présentées en couleur, selon leur validité~:
\begin{itemize}
  \item \emph{vert}, la transition est correcte et ne donne lieu à aucune émission de message d'alerte ou d'erreur~;
  \item \emph{orange}, la transition est correcte et mais donne lieu à l'émission d'un message d'alerte~;
  \item \emph{rouge}, la transition est incorrecte et donne lieu à l'émission d'un message d'erreur.
\end{itemize}

Ainsi, la lecture d'une variable dans l'état \texttt{DECLARED} est une erreur~: en effet, la variable n'a pas de valeur. La transition correspondante est donc en rouge. Voici un exemple qui illustre cette situation~:
\begin{galgas3}
var @uint x # État 'declared'
var y = x # Erreur, x n'a pas de valeur
\end{galgas3}

L'écriture d'une variable qui est dans l'état \texttt{INITIALIZED} n'est pas une erreur, mais révèle une anomalie~: la valeur écrasée n'a jamais été lue~; un message d'alerte est donc émis. Par exemple :
\begin{galgas3}
var @uint x = 2 # État 'initialized'
var x = 3 # Alerte, l'initialisation de x à 2 est inutile
\end{galgas3}

Remarquons que les actions \texttt{read} et \texttt{write} sont acceptées silencieusement dans l'état \texttt{INVALID}~: en effet, on arrive dans cet état après l'occurrence d'une erreur qui a été signalée à l'utilisateur, en acceptant silencieusement on n'émet pas de message d'erreur à chaque accès à la variable.

Enfin, l'état final de l'automate. À la fin de la portée de la variable, son automate est supprimé. Au moment de sa suppression, son état courant est considéré comme son état final. Trois situations peuvent survenir, qui sont reflètées la couleur du cadre de l'état~:
\begin{itemize}
  \item \emph{verte}, l'état final est correct et ne donne lieu à aucune émission de message d'alerte ou d'erreur~;
  \item \emph{orange}, l'état final est correct et mais donne lieu à l'émission d'un message d'alerte~;
  \item \emph{rouge}, l'état final est incorrect et donne lieu à l'émission d'un message d'erreur.
\end{itemize}

Ainsi, l'état \texttt{READ} est un état final correct pour une variable locale. L'état \texttt{INVALIDE} est aussi silencieusement accepté, être dans cet état signifie qu'un message d'erreur a déjà été émis pour cette variable.

L'état \texttt{DECLARED} comme état final signifie que la variable a été déclarée, sans être initialisée~: la variable est inutile, et un message d'alerte est émis.

L'état \texttt{INITIALIZED} comme état final signifie que la variable a été déclarée, initialisée, mais jamais lue~: comme précédemment, la variable est inutile, et un message d'alerte est émis.

L'état \texttt{READ} comme état final signifie que la variable a été déclarée, initialisée, lue, mais jamais modifiée~: c'est en fait une constante et un message d'alerte est émis, qui suggère de transformer la variable locale en constante locale.



\section{Constante locale}

L'automate d'états finis associé à une constante locale est présenté à la \refFigure{}{automateEtatsConstanteLocale}. Une constante locale peut être écrite une seule fois, à partir de l'état \texttt{DECLARED}. En conséquence, une action \texttt{write} à partir des états \texttt{INITIALIZED} et \texttt{READ} est une erreur (elle apparaît en rouge) et redirige vers l'état \texttt{INVALID}. L'état \texttt{MUTATED} n'est plus accessible, et a été supprimé de la \refFigure{}{automateEtatsConstanteLocale}.




\begin{figure}[ht!]
  \centering
  \small
  \begin{tikzpicture}
    \EtatVert{INVALID_STATE}{INVALID}{0cm}{0cm}
    \EtatOrange{DECLARED_STATE}{DECLARED}{3cm}{0cm}
    \EtatOrange{INITIALIZED_STATE}{INITIALIZED}{6cm}{0cm}
    \EtatVert{READ_STATE}{READ}{9cm}{0cm}

    \FlecheEtatInitial{DECLARED_STATE}{above = 1cm}

    \BoucleVerte{INVALID_STATE}{above}{read}
    \BoucleVerte{INVALID_STATE}{below}{write}

    \FlecheVerte{DECLARED_STATE}{write}{INITIALIZED_STATE}{bend left}
    \FlecheRouge{DECLARED_STATE}{read}{INVALID_STATE}{bend right}

%    \ActionInterdite{INITIALIZED_STATE}{below = 1cm}{write}
    \FlecheRouge{INITIALIZED_STATE}{write}{INVALID_STATE}{bend left= 20}
    \FlecheVerte{INITIALIZED_STATE}{read}{READ_STATE}{bend left=35}

    \FlecheRouge{READ_STATE}{write}{INVALID_STATE}{bend left}
%    \ActionInterdite{READ_STATE}{below = 1cm}{write}
    \BoucleVerte{READ_STATE}{above}{read}
  \end{tikzpicture}
  \caption{Automate des états d'une constante locale}
  \labelFigure{automateEtatsConstanteLocale}
\end{figure}

Voici un exemple.
\begin{galgas3}
let @uint x = 2 # État 'initialized'
...
var y = x # Ok, l'état de 'x' est 'read'
\end{galgas3}

On n'est pas obligé de fournir une valeur à la déclaration d'une constante. On peut ainsi écrire~:
\begin{galgas3}
let @uint x # État 'declared'
...
x = 2 # État 'initialized'
...
var y = x # Ok, l'état de 'x' est 'read'
\end{galgas3}



%\section{Paramètre formel constant en entrée}
%
%\begin{figure}[ht!]
%  \centering
%  \small
%  \begin{tikzpicture}
%    \EtatVert{constantInputDeclaredAsUnused}{Constant input declared unused}{0cm}{3cm}
%    \EtatOrange{constantInput}{Constant input}{6cm}{3cm}
%    \EtatVert{readConstantInput}{Read constant input}{0cm}{0cm}
%    \EtatVert{droppedConstantInput}{Dropped constant input}{6cm}{0cm}
%
%    \FlecheOrange{constantInputDeclaredAsUnused}{drop}{droppedConstantInput}{bend left=20}
%    \FlecheVerte{constantInput}{read}{readConstantInput}{bend right=20}
%    \FlecheVerte{constantInput}{drop}{droppedConstantInput}{bend left}
%
%    \FlecheVerte{readConstantInput}{drop}{droppedConstantInput}{bend right}
%
%    \FlecheOrange{constantInputDeclaredAsUnused}{read}{readConstantInput}{bend right}
%
%    \BoucleRouge{droppedConstantInput}{right}{write, rw, read}
%    \BoucleOrange{droppedConstantInput}{below}{drop}
%    \BoucleRouge{readConstantInput}{below}{write, rw}
%    \BoucleRouge{constantInput}{right}{write, rw}
%    \BoucleVerte{readConstantInput}{left}{read}
%    \BoucleRouge{constantInputDeclaredAsUnused}{above}{rw, write}
%  \end{tikzpicture}
%  \caption{Automate des états d'un paramètre formel constant en entrée}
%  \labelFigure{automateEtatsParametreEntreeConstant}
%\end{figure}
%
%
%
%
%
%
%
%
%
%\section{Paramètre formel variable en entrée}
%
%\begin{figure}[ht!]
%  \centering
%  \small
%  \begin{tikzpicture}
%    \EtatVert{inputDeclaredAsUnused}{Mutable input parameter declared unused}{0cm}{3cm}
%    \EtatOrange{input}{Mutable unread input parameter}{6cm}{3cm}
%    \EtatVert{readInput}{Read mutable input parameter}{0cm}{0cm}
%    \EtatRouge{droppedInput}{Dropped mutable input parameter}{6cm}{0cm}
%
%    \FlecheOrange{inputDeclaredAsUnused}{drop}{droppedInput}{bend right=20}
%    \FlecheVerte{input}{read}{readInput}{bend right=20}
%    \FlecheVerte{readInput}{write, rw}{input}{bend right=10}
%    \FlecheVerte{input}{drop}{droppedInput}{bend left}
%    \FlecheVerte{droppedInput}{write}{input}{bend left}
%
%    \FlecheVerte{readInput}{drop}{droppedInput}{bend right}
%
%    \FlecheOrange{inputDeclaredAsUnused}{read}{readInput}{bend right}
%    \FlecheOrange{inputDeclaredAsUnused}{rw, write}{input}{bend left}
%
%    \BoucleOrange{input}{above}{write}
%    \BoucleVerte{input}{right}{rw}
%    \BoucleRouge{droppedInput}{right}{rw, read}
%    \BoucleOrange{droppedInput}{below}{drop}
%    \BoucleVerte{readInput}{left}{read}
%  \end{tikzpicture}
%  \caption{Automate des états d'un paramètre formel variable en entrée}
%  \labelFigure{automateEtatsParametreEntreeVariable}
%\end{figure}
%
%
%
%
%
%
%
%
%
%
%
%
%
%
%
%\section{Paramètre formel en entrée / sortie}
%
%\begin{figure}[ht!]
%  \centering
%  \small
%  \begin{tikzpicture}
%    \EtatVert{inputOutputDeclaredAsUnused}{Inout parameter declared unused}{0cm}{3cm}
%    \EtatOrange{unaccessedInputOutput}{Unaccessed inout parameter}{6cm}{3cm}
%    \EtatVert{accessedInputOutput}{Accessed inout parameter}{0cm}{0cm}
%    \EtatRouge{droppedInputOutput}{Dropped inout parameter}{6cm}{0cm}
%
%    \FlecheVerte{unaccessedInputOutput}{read, rw, write}{accessedInputOutput}{bend right=20}
%
%    \FlecheVerte{unaccessedInputOutput}{drop}{droppedInputOutput}{bend left}
%    \FlecheVerte{droppedInputOutput}{write}{accessedInputOutput}{bend right}
%
%    \FlecheVerte{accessedInputOutput}{drop}{droppedInputOutput}{bend right}
%
%    \FlecheOrange{inputOutputDeclaredAsUnused}{read, rw, write}{accessedInputOutput}{bend right}
%    \FlecheOrange{inputOutputDeclaredAsUnused}{drop}{droppedInputOutput}{bend left=20}
%
%    \BoucleRouge{droppedInputOutput}{right}{rw, read}
%    \BoucleOrange{droppedInputOutput}{below}{drop}
%    \BoucleVerte{accessedInputOutput}{left}{read, rw, write}
%  \end{tikzpicture}
%  \caption{Automate des états d'un paramètre formel en entrée / sortie}
%  \labelFigure{automateEtatsParametreEntreeSortie}
%\end{figure}
%
%
%
%
%
%
%
%
%
%
%
%
%
%
%
%\section{Paramètre formel en sortie}
%
%
%\begin{figure}[ht!]
%  \centering
%  \small
%  \begin{tikzpicture}
%    \EtatVert{definedOutputParameter}{Defined output parameter}{0cm}{3cm}
%    \EtatRouge{undefinedOutputParameter}{Undefined output parameter}{6cm}{3cm}
%
%    \FlecheVerte{undefinedOutputParameter}{write}{definedOutputParameter}{bend right}
%
%    \FlecheVerte{definedOutputParameter}{drop}{undefinedOutputParameter}{bend right}
%
%    \BoucleOrange{definedOutputParameter}{left}{write}
%    \BoucleRouge{undefinedOutputParameter}{right}{read, rw}
%    \BoucleVerte{definedOutputParameter}{above}{rw, read}
%    \BoucleOrange{undefinedOutputParameter}{below}{drop}
%  \end{tikzpicture}
%  \caption{Automate des états d'un paramètre formel en sortie}
%  \labelFigure{automateEtatsParametreSortie}
%\end{figure}
%

%!TEX encoding = UTF-8 Unicode
%!TEX root = ../galgas-book.tex

%--------------------------------------------------------------
\chapter{Semantic expressions}
%-------------------------------------------------------------





%!TEX encoding = UTF-8 Unicode
%!TEX root = ../galgas-book.tex

%--------------------------------------------------------------
\chapter{Semantic Instructions}
%-------------------------------------------------------------



\sectionLabel{Append Instruction}{appendInstruction}


\sectionLabel{Assignment Instruction}{assignmentInstruction}


\section{Cast Instruction}


\sectionLabel{Concat Instruction}{concatInstruction}


\sectionLabel{Decrement Instruction}{decrementInstruction}




\section{Drop Instruction}

{\lstset{emph={variable}, emphstyle=\emph}
\begin{galgascode}
drop variable, ... ;
\end{galgascode}
}

\section{Error Instruction}


\section{Extern Action Call Instruction}




\section{L'instruction \texttt{for}}





\sectionLabel{L'instruction \texttt{foreach}}{instructionForeach}




\sectionLabel{Increment Instruction}{incrementInstruction}










\section{If Instruction}


\subsection{Syntax}

The \emph{if} instruction has the following syntax:
{\lstset{emph={expression, instructions, expression2, instructions2, else_instructions}, emphstyle=\emph}
\begin{galgascode}
if expression then
  instructions
elsif expression2 then
  instructions2
...
else
  else_instructions
end if ;  
\end{galgascode}
}

More precisely, it contains :
\begin{itemize}
\item zero, one or more \emph{elsif} branches ;
\item zero or one \emph{else} branch.
\end{itemize}


\subsection{Static semantics}


No \emph{else} branch is equivalent to an \emph{else} branch without any instruction.


The \emph{elsif} branches are just syntactic sugar: it is semantically equivalent to use embedded \emph{if} instructions instead. For example:
{\lstset{emph={expression, instructions, expression2, instructions2, else_instructions}, emphstyle=\emph}
\begin{galgascode}
if expression then
  instructions
elsif expression2 then
  instructions2
else
  else_instructions
end if ;  
\end{galgascode}
}
is equivalent to:
{\lstset{emph={expression, instructions, expression2, instructions2, else_instructions}, emphstyle=\emph}
\begin{galgascode}
if expression then
  instructions
else
  if expression2 then
    instructions2
  else
    else_instructions
  end if ;  
end if ;  
\end{galgascode}
}

So, for describing \emph{if} instruction static and dynamic semantics, we only need to describe an \emph{if} instruction without any \emph{elsif} branch and with an \emph{else} branch:
{\lstset{emph={expression, instructions, else_instructions}, emphstyle=\emph}
\begin{galgascode}
if expression then
  instructions
else
  else_instructions
end if ;
\end{galgascode}
}

The static semantics evaluates the \emph{expression} type, and applies the following rules until success:
\begin{enumerate}
\item the \emph{expression} type is \refTypePredefini{bool};
\item the \emph{expression} type is an \emph{structure} type, it has a attribute named \emph{bool}, whose type is \refTypePredefini{bool};
\item the \emph{expression} type has a reader without any argument named \emph{bool} that returns a \refTypePredefini{bool} value.
\end{enumerate}

Most expressions you write fall in the first case.

Applying the second rule enables to use an \refTypePredefini{lbool} expression as an \emph{if} expression. For example:
{\lstset{emph={expression, instructions, else_instructions}, emphstyle=\emph}
\begin{galgascode}
@lbool var := ... ;
if var then
  instructions
else
  else_instructions
end if ;
\end{galgascode}
}

The \emph{var} object belongs to the \refTypePredefini{lbool} type: so first rule fails. But \refTypePredefini{lbool} is a \emph{structure} type, it has a \emph{bool} attribute with the \refTypePredefini{bool} type, so the second rule succeeds. It is semantically equivalent to write:
{\lstset{emph={expression, instructions, else_instructions}, emphstyle=\emph}
\begin{galgascode}
@lbool var := ... ;
if var->bool then
  instructions
else
  else_instructions
end if ;
\end{galgascode}
}

The third rule applies on a \emph{class} type that defines a category reader with argument named \emph{bool} that returns a \refTypePredefini{bool} type. For example, declaring:
\begin{galgascode}
class @myClass { ... }

reader @myClass bool -> @bool outResult : ... end reader ;
\end{galgascode}

enables to write:
{\lstset{emph={expression, instructions, else_instructions}, emphstyle=\emph}
\begin{galgascode}
@myClass myObject := ... ;
if myObject then
  instructions
else
  else_instructions
end if ;
\end{galgascode}
}

It is semantically equivalent to write:
{\lstset{emph={expression, instructions, else_instructions}, emphstyle=\emph}
\begin{galgascode}
@myClass myObject := ... ;
if [myObject bool] then
  instructions
else
  else_instructions
end if ;
\end{galgascode}
}


\subsection{Dynamic semantics}

According to the preceding section, we only need to describe the dynamic semantic of an \emph{if} instruction without any \emph{elsif} branch and with an \emph{else} branch:
{\lstset{emph={expression, instructions, else_instructions}, emphstyle=\emph}
\begin{galgascode}
if expression then
  instructions
else
  else_instructions
end if ;  
\end{galgascode}
}



The \emph{expression} is first computed :
\begin{itemize}
\item if the evaluation fails, neither the \emph{if} instructions, neither the \emph{else} instructions are executed;
\item if the evaluation result is \emph{true}, the \emph{if} instructions are executed ;
\item if the evaluation result is \emph{false}, the \emph{else} instructions are executed.
\end{itemize}


\section{Grammar Instruction}

\section{Local Variable Declaration Instruction}


{\lstset{emph={variable}, emphstyle=\emph}
\begin{galgascode}
@type variable ;
\end{galgascode}
}

{\lstset{emph={variable, expression}, emphstyle=\emph}
\begin{galgascode}
@type variable := expression ;
\end{galgascode}
}

{\lstset{emph={variable, constructor, arguments}, emphstyle=\emph}
\begin{galgascode}
@type variable [constructor arguments] ;
\end{galgascode}
}


\section{Local Constant Declaration Instruction}




\sectionLabel{L'instruction \texttt{log}}{instructionLog}




\section{Loop Instruction}


\subsection{Syntax}

The \emph{loop} instruction has the following syntax:
{\lstset{emph={expression, instructions_1, instructions_2, variant_expression}, emphstyle=\emph}
\begin{galgascode}
loop variant_expression
: instructions_1
while expression do
  instructions_2
end loop ;  
\end{galgascode}
}

The \emph{instructions\_1} and \emph{instructions\_2} are possibly empty instruction lists. If the \emph{instructions\_1} is empty, the preceeding « : » can be omitted :
{\lstset{emph={expression, instructions_1, instructions_2, variant_expression}, emphstyle=\emph}
\begin{galgascode}
loop variant_expression
while expression do
  instructions_2
end loop ;  
\end{galgascode}
}

\subsection{Semantics}

The \emph{variant\_expression} is an \galgas{@uint} expression that ensures the loop is not endless: it is computed at the beginning of the loop execution, and is decremented by one at the end of every iteration. If it reaches zero, a run-time error is raised.

The \emph{expression} is an \galgas{@bool} expression that expresses repetitive execution.

The \emph{loop} instruction execution is illustrated by the flowchart given in \refFigure{}{loopInstructionFlowchart}.

\begin{figure}[ht]
  \centering
  \small
  \begin{tikzpicture}[very thick]
    \node [rounded corners=5pt, shape=rectangle, draw] (start) {\textsc{begin}} ;
    \node [shape=rectangle, draw] (variant) [below=of start] {$variant := variant\_expression~value$} ;
    \node [shape=diamond, draw] (premierTest) [below=of variant] {$variant > 0$} ;
    \node [shape=rectangle, draw] (error1) [right=of premierTest] {$loop~variant~error$} ;
    \node [shape=rectangle, draw] (body0) [below=of premierTest] {$instructions\_1$} ;
    \node [shape=diamond, draw] (exp) [below=of body0] {$expression$} ;
    \node [shape=diamond, draw] (variantTest) [below=of exp] {$variant > 0$} ;
    \node [shape=rectangle, draw] (decTest) [left=of variantTest] {$variant {-}{-}$} ;
    \node [shape=rectangle, draw] (body1) [above=of decTest] {$instructions\_2$} ;
    \node [shape=rectangle, draw] (error) [right=of variantTest] {$loop~variant~error$} ;
    \node [rounded corners=5pt, shape=rectangle, draw] (end) [right=of error] {\textsc{end}} ;
    
    \draw [->] (start) -- (variant) ;
    \draw [->] (variant) -- (premierTest) ;
    \draw [->] (premierTest) to node[right] {$yes$} (body0) ;
    \draw [->] (premierTest) to node[above] {$no$} (error1) ;
    \draw [->] (body0) -- (exp) ;
    \draw [->] (exp) to node[right] {$true$} (variantTest) ;
    \draw [->] (variantTest) to node[above] {$yes$} (decTest) ;
    \draw [->] (variantTest) to node[above] {$no$} (error) ;
    \draw [->] (decTest) -- (body1) ;
    \draw [->, bend left] (exp.east) to node[above] {$false$} (end.north) ;
    \draw [->] (body1.north) .. controls +(north:2cm) and +(left:2cm) .. (body0.west) ;
    \draw [->] (error) -- (end) ;
    \draw [->] (error1.east) .. controls +(right:2cm) .. (end) ;
  \end{tikzpicture}
  \caption{\emph{loop} instruction flowchart}
  \labelFigure{loopInstructionFlowchart}
\end{figure}


















\sectionLabel{Method Call Instruction}{methodCallInstruction}




\sectionLabel{Modifier Call Instruction}{modifierCallInstruction}




\section{Switch Instruction}




\section{Send Instruction}




\section{Warning Instruction}



%!TEX encoding = UTF-8 Unicode
%!TEX root = ../galgas-book.tex

%--------------------------------------------------------------
\chapter{Instructions syntaxiques}
%-------------------------------------------------------------

Les six instruction décrites dans ce chapitre ne peuvent être utilisées qu'à l'intérieur des règles de production.


\sectionLabel{Instruction d'appel de terminal}{instruction-appel-terminal}






\section{Instruction d'appel de non terminal}





\section{Instruction \texttt{select}}








\section{Instruction \texttt{repeat}}







\section{Instruction \texttt{parse}}







\sectionLabel{Instruction \texttt{send}}{instruction-send-syntaxique}








%-----------------------------------------------------------------------------------------------------------------------*
%                                                                                                                       *
%   I N D E X                                                                                                           *
%                                                                                                                       *
%-----------------------------------------------------------------------------------------------------------------------*

\part{Index}
%\cleardoublepage % Pour commencer a une page impaire
\phantomsection  % Pour faire correctement pointer l'hyperlien dans la table des matières

%--- Ecrire l'index
{\small
\printindex
}

%-----------------------------------------------------------------------------------------------------------------------*
%                                                                                                                       *
%   F I N    D U    D O C U M E N T                                                                                     *
%                                                                                                                       *
%-----------------------------------------------------------------------------------------------------------------------*

\end{document}

%-----------------------------------------------------------------------------------------------------------------------*

