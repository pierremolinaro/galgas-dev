%!TEX encoding = UTF-8 Unicode

\documentclass [a4paper, 11pt] {book}

%-----------------------------------------------------------------------------------------------------------------------*
%                                                                                                                       *
%   « N O I R    E T    B L A N C »    O U    « C O U L E U R »                                                         *
%                                                                                                                       *
%-----------------------------------------------------------------------------------------------------------------------*

%--- Par défaut, l'impression se fait en couleur
\providecommand{\sortieEnCouleur}{true}

%-----------------------------------------------------------------------------------------------------------------------*
%                                                                                                                       *
%   R É G L A G E S    « F R A N Ç A I S »                                                                              *
%                                                                                                                       *
%-----------------------------------------------------------------------------------------------------------------------*

%--- Paquetage pour imposer les réglages français
\usepackage[francais]{babel}

%-----------------------------------------------------------------------------------------------------------------------*
%                                                                                                                       *
%   E N C O D A G E    D E S    S O U R C E S     :     U T F 8                                                         *
%                                                                                                                       *
%-----------------------------------------------------------------------------------------------------------------------*

%--- Paquetage pour le codage des sources en UTF-8
\usepackage[utf8]{inputenc}

%--- Latex demande ce paquetage pour mieux afficher le caractère "°" et \textquotesingle "'"
\usepackage{textcomp}

%--- Ce paquetage permet d'effectuer certaines césures, et ainsi d'éviter les messages "Overfull \hbox"
\usepackage[T1]{fontenc}

%-----------------------------------------------------------------------------------------------------------------------*
%                                                                                                                       *
%   M I S E    E N    P A G E S                                                                                         *
%                                                                                                                       *
%-----------------------------------------------------------------------------------------------------------------------*

% Voir "Une courte introduction à Latex2e", § 6.4

%--- Marge gauche : 2,8 cm ; le paramètre \hoffset contient cette valeur, moins 1 pouce
%    \hoffset = 2,8 cm - 2,54 cm = 0,26 cm
\setlength{\hoffset}{0.26 cm}

%--- Marges supplémentaires, différenciées pour les pages gauches et droites ; ici, aucune.
\setlength{\oddsidemargin }{0 cm}
\setlength{\evensidemargin}{0 cm}

%--- Largeur du texte
%    \textwidth = 210 mm - 28 mm - 28 mm = 15,4 cm
\setlength{\textwidth}{15.4 cm}

%--- Marge haute : 2,8 cm ; le paramètre \voffset contient cette valeur, moins 1 pouce
%    \voffset = 2,8 cm - 2,54 cm = 0,26 cm
\setlength{\voffset}{0.26 cm}

%--- Distance entre la marge haute et l'en-tête : 0 cm
\setlength{\topmargin}{0 cm}

%--- Hauteur de l'en-tête de chaque page : 1 cm
\setlength{\headheight}{1 cm}

%--- Distance entre l'en-tête de chaque page et le corps : 0,5 cm
\setlength{\headsep}{0.5 cm}

%--- Hauteur du corps
%    \textheight = 29,7 cm - 2,8 cm - 2,8 cm - 1,5 cm = 22,6 cm
\setlength{\textheight}{22.6 cm}

%-----------------------------------------------------------------------------------------------------------------------*
%                                                                                                                       *
%   C H O I X    D E    L A    P O L I C E                                                                              *
%                                                                                                                       *
%-----------------------------------------------------------------------------------------------------------------------*

% http://www.cuk.ch/articles/4237
% Un seul des choix suivants doit être validé ; si aucun, c'est la police par défaut qui est utilisée

%---------------------------------------------------- Pour utiliser la police "Times"
%\renewcommand{\rmdefault}{ptm}
%\usepackage{mathptmx}

%---------------------------------------------------- Pour utiliser la police "Palatino"
%\renewcommand{\rmdefault}{ppl}
%\usepackage{mathpazo}

%---------------------------------------------------- Pour utiliser la police "Bookman"
%\usepackage{bookman}

%---------------------------------------------------- Pour utiliser la police "Fourier"
\usepackage{fouriernc}
\usepackage[scaled=0.875]{helvet}
%\usepackage{courier}

%---------------------------------------------------- Pour utiliser la police "Beton, euler"
%\usepackage{beton, euler}

%---------------------------------------------------- Pour utiliser la police "Utopia - MathDesign"
%\usepackage[utopia]{mathdesign}

%---------------------------------------------------- Pour utiliser la police "Charter - MathDesign"
%\usepackage[charter]{mathdesign}

%-----------------------------------------------------------------------------------------------------------------------*
%                                                                                                                       *
%   G E S T I O N   D E    L A     C O U L E U R                                                                        *
%                                                                                                                       *
%-----------------------------------------------------------------------------------------------------------------------*

%--- Ce paquetage permet de définir des couleurs de la forme yellow!50 (jaune à 50 %)
%\usepackage[svgnames]{xcolor}


%-----------------------------------------------------------------------------------------------------------------------*
%                                                                                                                       *
%   E X T E N S I O N S    P O U R    L ' É C R I T U R E    D E S     F O R M U L E S    M A T H É M A T I Q U E S     *
%                                                                                                                       *
%-----------------------------------------------------------------------------------------------------------------------*

%--- Extensions pour l'écriture des formules mathématiques
\usepackage{amsmath}
\usepackage{amssymb}
\usepackage{amsfonts}

%--- Paquetage "IEEEtrantools"
% Pour créer des tableaux d'équations, bien alignées
% Voir courte-intro-latex.pdf, page §3.5.2 page 83
\usepackage[retainorgcmds]{IEEEtrantools}


%-----------------------------------------------------------------------------------------------------------------------*
%                                                                                                                       *
%   P A Q U E T A G E    « I F T H E N »                                                                                *
%                                                                                                                       *
%-----------------------------------------------------------------------------------------------------------------------*

%--- Ce paquetage permet d'effectuer des tests : \ifthenelse{test}{bloc then}{bloc else}
\usepackage{ifthen}

%-----------------------------------------------------------------------------------------------------------------------*
%                                                                                                                       *
%   E X T E N S I O N S    P O U R    P R É S E N T E R    L E S    T A B L E A U X                                     *
%                                                                                                                       *
%-----------------------------------------------------------------------------------------------------------------------*

\usepackage{array}
\usepackage[table]{xcolor} % À placer avant \usepackage{listings}

%\renewcommand{\arraystretch}{1.2}

%--- Couleur de fond alternée des tableaux
\ifthenelse{\equal{\sortieEnCouleur}{true}}{
  \newcommand\fondTableau{yellow!25}
}{
  \newcommand\fondTableau{gray!25}
}


%--- Ce paquetage permet de changer le style des légendes des tableaux et des figures (voir caption-eng.pdf) :
%  - l'étiquette est en italique gras
%  - le titre est en italique.
\usepackage[font=it, labelfont=bf]{caption}

%--- Par défaut, le paquetage nomme "Table" les tableaux. La commande
%   suivante impose le nom "Tableau"
% Voir http://fr.wikibooks.org/wiki/LaTeX/Éléments_flottants_et_figures
\addto\captionsfrench{\def\tablename{Tableau}}

%------------------------------------------------------------------------------------------ RÉFÉRENCES À UN TABLEAU
% La référence au tableau "nom-du-tableau" est définie par \labelTableau{nom-du-tableau}
\newcommand\labelTableau[1]{\label{tab:#1}}
% Latex autorise deux types d'appel à une référence \ref{tab:nom-du-tableau} et \pageref{tab:nom-du-tableau}

% \refTableau{}{nom-du-tableau} ---> "tableau x.y"   où x.y est le n° du tableau
\newcommand\refTableau[1]{\hyperref[tab:#1]{tableau \ref*{tab:#1}}}

% \refTableauSansPrefixe{}{nom-du-tableau} ---> "x.y"   où x.y est le n° du tableau
\newcommand\refTableauSansPrefixe[1]{\hyperref[tab:#1]{\ref*{tab:#1}}}

% \refTableauPage{}{nom-du-tableau} ---> "tableau x.y page n"   où x.y est le n° du tableau
\newcommand\refTableauPage[1]{\hyperref[tab:#1]{tableau \ref*{tab:#1} page \pageref{tab:#1}}}

% \refTableauPageSansPrefixe{}{nom-du-tableau} ---> "x.y page n"   où x.y est le n° du tableau
\newcommand\refTableauPageSansPrefixe[1]{\hyperref[tab:#1]{\ref*{tab:#1} page \pageref{tab:#1}}}

%-----------------------------------------------------------------------------------------------------------------------*
%                                                                                                                       *
%   P A Q U E T A G E    « L I S T I N G S »                                                                            *
%                                                                                                                       *
%-----------------------------------------------------------------------------------------------------------------------*

\usepackage{listings} % À placer après \usepackage[table]{xcolor}

%-----------------------------------------------------------------------------------------------------------------------*
%                                                                                                                       *
%   T I K Z    -    P G F                                                                                               *
%                                                                                                                       *
%-----------------------------------------------------------------------------------------------------------------------*

\usepackage{tikz}
\usetikzlibrary{calc}
\usepackage{pgfplots}
\usetikzlibrary{arrows}
\usetikzlibrary{decorations}
\usetikzlibrary{decorations.pathmorphing}
\usetikzlibrary{shapes.callouts}
\usetikzlibrary{automata}
\usetikzlibrary{positioning}
\usepgflibrary{shapes.geometric}

%-----------------------------------------------------------------------------------------------------------------------*
%                                                                                                                       *
%   T E X T E    E N C E R C L É                                                                                        *
%                                                                                                                       *
%-----------------------------------------------------------------------------------------------------------------------*

%\newcommand\texteEncercle[1]{\tikz \node[draw,circle, inner sep=0.5, text height=2mm]{#1}}
\newcommand\texteEncercle[1]{#1}

%-----------------------------------------------------------------------------------------------------------------------*
%                                                                                                                       *
%   I N C L U S I O N    D E S   M A C R O S   D E S T I N É E S   À   L ' É C R I T U R E   
%                                                                                                                       *
%                                        D E   C O D E   G A L G A S                                                    *
%                                                                                                                       *
%-----------------------------------------------------------------------------------------------------------------------*

\usepackage{relsize}

%!TEX encoding = UTF-8 Unicode
%!TEX root = ../galgas-book.tex

%-----------------------------------------------------------------------------------------------------------------------*
%   A F F I C H A G E    D U    C O D E    G A L G A S                                                                  *
%-----------------------------------------------------------------------------------------------------------------------*

\newcommand\tpp[1]{\colorbox{gray!12}{\ttfamily #1}}

%-----------------------------------------------------------------------------------------------------------------------*
%   A F F I C H A G E    D U    C O D E    G A L G A S                                                                  *
%-----------------------------------------------------------------------------------------------------------------------*

\newcommand\keywordsStyleGalgas[1]{\textcolor{blue}{\textbf{#1}}}
\newcommand\delimitersStyleGalgas[1]{\textcolor{brown}{\textbf{#1}}}
\newcommand\selectorStyleGalgas[1]{\textcolor{orange}{#1}}
\newcommand\terminalStyleGalgas[1]{\textcolor{orange}{#1}}
\newcommand\nonTerminalStyleGalgas[1]{\textcolor{orange}{#1}}
\newcommand\integerStyleGalgas[1]{\textcolor{brown}{#1}}
\newcommand\floatStyleGalgas[1]{\textcolor{magenta}{#1}}
\newcommand\characterStyleGalgas[1]{\textcolor{cyan}{#1}}
\newcommand\stringStyleGalgas[1]{\textcolor{gray}{#1}}
\newcommand\typeNameStyleGalgas[1]{\textcolor{gray}{#1}}
\newcommand\attributeStyleGalgas[1]{\textcolor{brown}{#1}}
\newcommand\commentStyleGalgas[1]{\textcolor{red}{#1}}

%\newcommand\lexicalErrorGalgas{\textcolor{red}{\textbullet ERRLEX\textbullet}}

\newcommand\keywordsStylegalgas[1]{\textcolor{blue}{\textbf{#1}}}
\newcommand\delimitersStylegalgas[1]{\textcolor{brown}{\textbf{#1}}}
\newcommand\selectorStylegalgas[1]{\textcolor{orange}{#1}}
\newcommand\terminalStylegalgas[1]{\textcolor{orange}{#1}}
\newcommand\nonTerminalStylegalgas[1]{\textcolor{orange}{#1}}
\newcommand\integerStylegalgas[1]{\textcolor{brown}{#1}}
\newcommand\floatStylegalgas[1]{\textcolor{magenta}{#1}}
\newcommand\characterStylegalgas[1]{\textcolor{cyan}{#1}}
\newcommand\stringStylegalgas[1]{\textcolor{gray}{#1}}
\newcommand\typeNameStylegalgas[1]{\textcolor{gray}{#1}}
\newcommand\attributeStylegalgas[1]{\textcolor{brown}{#1}}
\newcommand\commentStylegalgas[1]{\textcolor{red}{#1}}

%\newcommand\lexicalErrorgalgas{\textcolor{red}{\textbullet ERRLEX\textbullet}}

\newmdenv[
  topline=false,
  bottomline=false,
  rightline=false,
%  skipabove=\topsep,
%  skipbelow=\topsep,
  linecolor=blue!25,
  linewidth=2pt
]{siderules}

\newwrite\tempfile

\makeatletter
\newenvironment{galgas}{%
  \begingroup
  \@bsphack
  \immediate\openout\tempfile=temp.galgas%
  \let\do\@makeother\dospecials
  \catcode`\^^M\active
  \verbatim@startline
  \verbatim@addtoline
  \verbatim@finish
  \def\verbatim@processline{\immediate\write\tempfile{\the\verbatim@line}}%
  \verbatim@start
}{
  \immediate\closeout\tempfile
  \@esphack
  \endgroup
  \immediate\write18{galgas --mode=latex:Galgas temp.galgas}
  {\singlespacing\begin{siderules}\ttfamily\input{temp.galgas.tex}\end{siderules}}
}
\makeatother

%-----------------------------------------------------------------------------------------------------------------------*
% COMMANDE \ggs : affichage de code en ligne galgas                                                                     *
%-----------------------------------------------------------------------------------------------------------------------*

\makeatletter
\newcommand*\ggs{%
  \@bsphack%
  \begingroup%
  \let\do\@makeother\dospecials%
  \let\do\do@noligs\verbatim@nolig@list%
  \catcode`\^^M=15\relax%
  \@vobeyspaces%
  \@ggs{\temporary}%
}%
\newcommand\@ggs[2]{%
  \catcode`-=12\relax%
  \catcode`<=12\relax%
  \catcode`>=12\relax%
  \catcode`,=12\relax%
  \catcode`'=12\relax%
  \catcode``=12\relax%
  \catcode`#2\active%
  \catcode`~\active%
  \lccode`\~`#2\relax%
  \begingroup%
  \lowercase{%
    \def\@tempa##1~{%
      \expandafter\endgroup%
      \expandafter\DeclareRobustCommand%
      \expandafter*%
      \expandafter#1%
      \expandafter{\@tempa}%
      \@esphack%
      \immediate\openout\tempfile=temp.galgas%
      \immediate\write\tempfile{##1}%
      \immediate\closeout\tempfile%
      \immediate\write18{galgas --mode=latex:Galgas temp.galgas}%
      \colorbox{gray!6}{\ttfamily\input{temp.galgas.tex}\unskip}%
    }%
  }%
  \ifnum`#2=`\~\else\@makeother\~\fi%
  \expandafter\endgroup%
  \@tempa%
}%
\makeatother

%-----------------------------------------------------------------------------------------------------------------------*
%                                                                                                                       *
% A F F I C H A G E    E T    C R O S S   R É F É R E N C E    D E S    T Y P E S    P R É D É F I N I S   G A L G A S  *
%                                                                                                                       *
%-----------------------------------------------------------------------------------------------------------------------*

%--- Les deux macros suivantes définissent une section et une sous-section :
%      - en formattant le titre
%      - en définissant un label pour cross référence ;
%      - en définissant une entrée dans l'index

% Exemple d'appel : \sectionTypePredefiniLabelIndex{bool}

\newcommand \chapitreTypePredefiniLabelIndex[1] {\chapter{Le type \texttt{@#1}}\label{type:#1}\index{Type!"@#1}}

\newcommand \sectionTypePredefiniLabelIndex[1] {\section{Le type \texttt{@#1}}\label{type:#1}\index{Type!"@#1}}

\newcommand \subsectionTypePredefiniLabelIndex[1] {\subsection{Le type \texttt{@#1}}\label{type:#1}\index{Type!"@#1}}

%--- Cette macro établit un hyperlien vers un type prédéfini
% Exemple d'appel : \refTypePredefini{bool} -- affiche --> @bool type (page xx)

\newcommand \refTypePredefini[1] {\hyperref[type:#1]{\texttt{@#1} (page \pageref{type:#1})}}

%-----------------------------------------------------------------------------------------------------------------------*
%   G E T T E R   C R O S S    R E F E R E N C I N G                                                                    *
%-----------------------------------------------------------------------------------------------------------------------*

\newcommand\subsectionGetter[2]{\subsection{Getter \texttt{#1}}\label{getter:#2:#1}\index{#1!"@#2 getter}}

%-----------------------------------------------------------------------------------------------------------------------*

% Exemple d'appel : \refGetterPage{bool}{string} -- affiche --> @bool string getter (page xx)
\newcommand\refGetterPage[2] {\hyperref[getter:#1:#2]{getter \texttt{@#1 #2} à la page \pageref{getter:#1:#2}}}

%-----------------------------------------------------------------------------------------------------------------------*
%   S E T T E R   C R O S S    R E F E R E N C I N G                                                                    *
%-----------------------------------------------------------------------------------------------------------------------*

\newcommand\subsectionSetter[2]{\subsection{Setter \texttt{#1}}\label{setter:#2:#1}\index{#1!"@#2 getter}}

%-----------------------------------------------------------------------------------------------------------------------*

% Exemple d'appel : \refSetterPage{bool}{string} -- affiche --> @bool string setter (page xx)
\newcommand\refSetterPage[2] {\hyperref[setter:#1:#2]{setter \texttt{@#1 #2} à la page \pageref{setter:#1:#2}}}


%-----------------------------------------------------------------------------------------------------------------------*
%   C O N S T R U C T O R   C R O S S    R E F E R E N C I N G                                                          *
%-----------------------------------------------------------------------------------------------------------------------*

\newcommand\subsectionConstructor[2]{\subsection{Constructeur \texttt{#1}}\label{constructor:#2:#1}\index{#1!"@#2 constructor}}

%-----------------------------------------------------------------------------------------------------------------------*

\newcommand \refConstructorPage[2] {\hyperref[constructor:#1:#2]{#2 constructor (page \pageref{constructor:#1:#2})}}

%-----------------------------------------------------------------------------------------------------------------------*


%-----------------------------------------------------------------------------------------------------------------------*
%                                                                                                                       *
%   P A Q U E T A G E    « L O N G T A B L E »                                                                          *
%                                                                                                                       *
%-----------------------------------------------------------------------------------------------------------------------*

%--- Pour afficher correctement des tables sur plusieurs pages
%\usepackage{longtable}

%-----------------------------------------------------------------------------------------------------------------------*
%                                                                                                                       *
%   E N - T Ê T E S    E T    P I E D S    D E    P A G E S                                                             *
%                                                                                                                       *
%-----------------------------------------------------------------------------------------------------------------------*

% Grâce au package "fancyhdr"
% voir http://www.exomatik.net/U-Latex/Personnaliser#toc2
%      http://www.trustonme.net/didactels/250.html
\usepackage{fancyhdr}
\pagestyle{fancy}
%--- Numéro de page : à gauche pages paires, à droite pages impaires
\fancyhead[EL,OR]{\thepage}
%--- Nom de chapitre : à droite page paires
\fancyhead[ER]{\leftmark}
%--- Nom de section : à gauche page impaires
\fancyhead[OL]{\rightmark}
%--- Version GALGAS : au milieu du pied de chaque page
\fancyfoot[C]{GALGAS, version GALGASCURRENTVERSION}
%--- filet en haut et en bas de chaque page
\renewcommand{\headrulewidth}{0.5 pt}
\renewcommand{\footrulewidth}{0.5 pt}

\renewcommand{\chaptermark}[1]{\markboth{\bsc{\chaptername~\thechapter{}.} #1}{}}
\renewcommand{\sectionmark}[1]{\markright{\bsc{\thesection{}.} #1}{}}

%-----------------------------------------------------------------------------------------------------------------------*
%                                                                                                                       *
%   G E S T I O N    D E    L ' I N D E X                                                                               *
%                                                                                                                       *
%-----------------------------------------------------------------------------------------------------------------------*

% http://www.cuk.ch/articles/4097
% http://www.tuteurs.ens.fr/logiciels/latex/makeindex.html
% http://linux.die.net/man/1/makeindex
%
% Attention ! Les deux commandes suivantes, ainsi que le \printindex placé plus bas ne
% sont pas suffisants pour construire l'index : il faut utiliser l'utilitaire "makeIndex"
% Voir le fichier de commande "build.command"
\usepackage{makeidx}
\makeindex

%-----------------------------------------------------------------------------------------------------------------------*
%                                                                                                                       *
%   T O C B I D I N D                                                                                                   *
%                                                                                                                       *
%-----------------------------------------------------------------------------------------------------------------------*

%    Pour faire figurer la liste des tableaux (et la table des matières)
%    dans la table des matières
\usepackage{tocbibind}

\setcounter{tocdepth}{3}

%-----------------------------------------------------------------------------------------------------------------------*
%                                                                                                                       *
%   H Y P E R R E F                                                                                                     *
%                                                                                                                       *
%-----------------------------------------------------------------------------------------------------------------------*

%--- Pour les hyperliens, et le contrôle de la génération PDF 
\usepackage{hyperref}
\hypersetup{colorlinks=true}
\ifthenelse{\equal{\sortieEnCouleur}{true}}{
  \hypersetup{linkcolor=blue}
}{
  \hypersetup{linkcolor=black}
}
\hypersetup{breaklinks=true}

%-----------------------------------------------------------------------------------------------------------------------*
%                                                                                                                       *
%   R É F É R N C E S                                                                                                   *
%                                                                                                                       *
%-----------------------------------------------------------------------------------------------------------------------*

% Au lieu d'écrire \chapter{titre-chapitre}, on écrit \chapterLabel{titre-chapitre}{label-chapitre}
\newcommand \chapterLabel[2]{\chapter{#1}\label{chapter:#2}}

% \refChapter{label-chapter} ---> "chapitre n"
\newcommand \refChapter[1]{\hyperref[chapter:#1]{chapitre \ref*{chapter:#1}}}

\newcommand \refChapterPage[1]{\hyperref[chapter:#1]{chapitre \ref*{chapter:#1} page \pageref{chapter:#1}}}

% Au lieu d'écrire \section{titre-section}, on écrit \sectionLabel{titre-section}{label-section}
\newcommand \sectionLabel[2]{\section{#1}\label{sec:#2}}


% \refSectionPage{label-section} ---> "section x.y page n"   où x.y est le n° de la section
\newcommand\refSectionPage[1]{\hyperref[sec:#1]{section \ref*{sec:#1} page \pageref{sec:#1}}}

%------------------------------------------------------------------------------------------ RÉFÉRENCES À UNE SUB-SECTION
% Au lieu d'écrire \subsection{titre-section}, on écrit \subsectionLabel{titre-section}{label-section}
\newcommand \subsectionLabel[2]{\subsection{#1}\label{subsec:#2}}


% \refSubsectionPage{label-section} ---> "section x.y page n"   où x.y est le n° de la sub-section
\newcommand\refSubsectionPage[1]{\hyperref[subsec:#1]{section \ref*{subsec:#1} page \pageref{subsec:#1}}}

%------------------------------------------------------------------------------------------ RÉFÉRENCES À UNE FIGURE
% La référence au tableau "nom-de-la-figure" est définie par \labelFigure{nom-de-la-figure}
\newcommand\labelFigure[1]{\label{fig:#1}}
% Latex autorise deux types d'appel à une référence \ref{fig:nom-de-la-figure} et \pageref{fig:nom-de-la-figure}

% \refFigure{}{nom-de-la-figure}   ---> "figure x.y"   où x.y est le n° de la figure
% \refFigure{z}{nom-de-la-figure}  ---> "figure x.y.z" où x.y est le n° de la figure
\newcommand\refFigure[2]{\hyperref[fig:#2]{figure \ref*{fig:#2}{\ifthenelse{\equal{#1}{}}{}{.#1}}}}

% \refFigureSansPrefixe{}{nom-de-la-figure}   ---> "x.y"   où x.y est le n° de la figure
% \refFigureSansPrefixe{z}{nom-de-la-figure}  ---> "x.y.z" où x.y est le n° de la figure
\newcommand\refFigureSansPrefixe[2]{\hyperref[fig:#2]{\ref*{fig:#2}{\ifthenelse{\equal{#1}{}}{}{.#1}}}}

% \refFigurePage{}{nom-de-la-figure}   ---> "figure x.y page n"   où x.y est le n° de la figure
% \refFigurePage{z}{nom-de-la-figure}  ---> "figure x.y.z page n" où x.y est le n° de la figure
\newcommand\refFigurePage[2]{\hyperref[fig:#2]{figure \ref*{fig:#2}{\ifthenelse{\equal{#1}{}}{}{.#1}} page \pageref{fig:#2}}}

% \refFigurePageSansPrefixe{}{nom-de-la-figure}   ---> "x.y page n"   où x.y est le n° de la figure
% \refFigurePageSansPrefixe{z}{nom-de-la-figure}  ---> "x.y.z page n" où x.y est le n° de la figure
\newcommand\refFigurePageSansPrefixe[2]{\hyperref[fig:#2]{\ref*{fig:#2}{\ifthenelse{\equal{#1}{}}{}{.#1}} page \pageref{fig:#2}}}

%-----------------------------------------------------------------------------------------------------------------------*
%                                                                                                                       *
%   S H O W K E Y S    ( P O U R    D É B O G U E R )                                                                   *
%                                                                                                                       *
%-----------------------------------------------------------------------------------------------------------------------*

%\usepackage{showkeys}

%-----------------------------------------------------------------------------------------------------------------------*
%                                                                                                                       *
%   T I T L E T O C                                                                                                     *
%                                                                                                                       *
%-----------------------------------------------------------------------------------------------------------------------*

%--- Description dans le paquetage titlesec
% http://forum.mathematex.net/latex-f6/formatage-avance-de-la-table-des-matieres-t11559.html

%\usepackage{titletoc}


%--- Par défaut dans la tables des matières, le numéro de sous-section est trop long et mange le début du titre
%\titlecontents{subsection}[2.5cm]{}{\hspace*{-5.0em}\hyperref[subsection \thecontentslabel]{\thecontentslabel}\hspace*{0.5em}}{}{\titlerule*[0.66pc]{.}\contentspage}{}

%\titlecontents{subsection}[2.5cm]{}{\hspace*{-5.0em}\thecontentslabel\hspace*{0.5em}}{}{\titlerule*[0.66pc]{.}\contentspage}{}

%\titlecontents{subsection}
%  [2cm]% retrait gauche
%  {}% pour les entrées numérotées et non numérotées
%  {\hspace*{-3.0em}\makebox[2.5em]{\hspace*{0pt plus 1 fill minus 1fill}\thecontentslabel.}\hspace*{0.5em}}% pour les entrées numérotées uniquement
%  {}% pour les entrées non numérotées uniquement
%  {\titlerule*[0.66em]{.}\contentspage}% numéro de page

%-----------------------------------------------------------------------------------------------------------------------*
%                                                                                                                       *
%   D P R O G R E S S                                                                                                   *
%                                                                                                                       *
%-----------------------------------------------------------------------------------------------------------------------*

%--- Affiche les sections dans le log
\usepackage{dprogress}

%-----------------------------------------------------------------------------------------------------------------------*
%                                                                                                                       *
%   D É B U T    D U    D O C U M E N T                                                                                 *
%                                                                                                                       *
%-----------------------------------------------------------------------------------------------------------------------*


\begin{document} 

%-----------------------------------------------------------------------------------------------------------------------*
%                                                                                                                       *
%   P A G E    D E    T I T R E                                                                                         *
%                                                                                                                       *
%-----------------------------------------------------------------------------------------------------------------------*

\title{\Huge{\textbf{GALGAS}}\\~\\ \normalsize{Version GALGASCURRENTVERSION}}
\author{Jean-Luc Béchennec\\Mikaël Briday\\Pierre Molinaro}
\date \today 

\maketitle

%-----------------------------------------------------------------------------------------------------------------------*
%                                                                                                                       *
%   T A B L E    D E S    M A T I È R E S                                                                               *
%                                                                                                                       *
%-----------------------------------------------------------------------------------------------------------------------*

\tableofcontents
 
%-----------------------------------------------------------------------------------------------------------------------*
%                                                                                                                       *
%   L I S T E    D E S    T A B L E A U X                                                                               *
%                                                                                                                       *
%-----------------------------------------------------------------------------------------------------------------------*

\listoftables
\addtocontents{lot}{\protect\thispagestyle{empty}\protect\pagestyle{empty}}

%-----------------------------------------------------------------------------------------------------------------------*
%                                                                                                                       *
%   L I S T E    D E S    F I G U R E S                                                                                 *
%                                                                                                                       *
%-----------------------------------------------------------------------------------------------------------------------*

\listoffigures
\addtocontents{lof}{\protect\thispagestyle{empty}\protect\pagestyle{empty}}

%-----------------------------------------------------------------------------------------------------------------------*
%                                                                                                                       *
%   L E S    C H A P I T R E S                                                                                          *
%                                                                                                                       *
%-----------------------------------------------------------------------------------------------------------------------*

\input{chapter-installation/installation.tex}

%!TEX encoding = UTF-8 Unicode
%!TEX root = ../galgas-book.tex

%--------------------------------------------------------------
\chapter{Using GALGAS}
%-------------------------------------------------------------


\section{Command Line Options}


\section{Creating a New Project}


%!TEX encoding = UTF-8 Unicode
%!TEX root = ../galgas-book.tex

%--------------------------------------------------------------
\chapter{Lexical Elements}
%-------------------------------------------------------------

%Avant
%[\input{|"/bin/echo -n ab"}]
%Après
%
%
%(\immediate\write18{"/bin/echo -n ab"})
%
%\begin{filecontents}{myfile.tex}
%     This text gets written to \texttt{myfile.tex}.\\
%     Zis text gets written to \texttt{myfile.tex}.
%\end{filecontents}
%
%\input{myfile.tex}


\part{Le système de types}
  %!TEX encoding = UTF-8 Unicode
%!TEX root = ../galgas-book.tex

\chapter{Présentation du système de types}






\section{Opérations définies pour tous les types}

Tout type implémente implicitement :
\begin{itemize}
  \item l'opérateur \galgas{==} ;
  \item l'opérateur \galgas{\!=} ;
  \item le \emph{reader} \galgas{description} ;
  \item le \emph{reader} \galgas{dynamicType} ;
  \item le \emph{reader} \galgas{object}.
\end{itemize}

La plupart des types implémentent le constructeur par défaut \galgas{default} (voir \refSectionPage{constructeurParDefaut}). 


\subsection{L'opérateur \texttt{==}}

\begin{galgascode}
operator @T == -> @bool ;
\end{galgascode}

Cet opérateur permet de tester l'identité entre de deux objets de même type. 

\subsection{L'opérateur \texttt{!=}}

\begin{galgascode}
operator @T != -> @bool ;
\end{galgascode}

Cet opérateur permet de tester la non identité entre de deux objets de même type. Il renvoie le complément logique du résultat de l'application de l'opérateur \galgas{==}.





\subsection{Le reader \texttt{description}}

\begin{galgascode}
reader @T description -> @string ;
\end{galgascode}

Le \emph{reader} \galgas{description} retourne une description textuelle du receveur, la même que celle affichée par l'instruction \galgas{log} (\refSectionPage{instructionLog}).



\subsection{Le reader \texttt{dynamicType}}

\begin{galgascode}
reader @T dynamicType -> @type ;
\end{galgascode}

Le \emph{reader} \galgas{dynamicType} retourne un objet de type \galgas{@type}, dont la valeur représente le type dynamique du receveur (voir aussi la définition du \refTypePredefini{type}).

Pour tous les types sauf les classes, leurs instances sont du même type que le type statique :

\begin{galgascode}
@uint n := 2 ;
@type t := [n dynamicType] ;
log t ; # Affiche @uint
\end{galgascode}

Pour les instances de classes, le jeu des affectations polymorphiques peut entraîner que le type dynamique soit une classe héritière du type statique.

Par exemple, en déclarant :
\begin{galgascode}
class @A { }
class @B extends @A { }
\end{galgascode}

Et avec la séquence d'instructions suivante :
\begin{galgascode}
@B b [new] ;
@type t := [b dynamicType] ;
log t ; # Affiche @B, type statique de b : @B
@A a := b ; # Affectation polymorphique
t := [a dynamicType] ;
log t ; # Affiche @B, type statique de a : @A
\end{galgascode}





\subsection{Le reader \texttt{object}}

\begin{galgascode}
reader @T object -> @object ;
\end{galgascode}


Le \emph{reader} \galgas{object} retourne un objet de type \galgas{@object}. Une variable de \refTypePredefini{object} peut encapsuler tout type de valeur.

%====== Readers ======
%===== description =====
%
%''**reader** description %%->%% @string ;''\\
%
%This reader returns a string representation of the receiver's value.
%
%===== dynamicType =====
%
%|Available on GALGAS 1.9.5 and later|
%
%''**reader** dynamicType %%->%% @type ;''\\
%
%This reader returns the dynamic type of the receiver's value.
%===== object =====
%
%|Available on GALGAS 1.9.5 and later|
%
%''**reader** object %%->%% @object ;''\\
%
%This reader returns an ''@object'' instance that embeds the receiver's value.











\sectionLabel{Constructeur par défaut}{constructeurParDefaut}

Pour la plupart des types, un constructeur par défaut est implicitement défini (voir la définition précise \refSubsectionPage{constructeurParDefautPourChaqueType}). Celui-ci est invoqué par le mot réservé \galgas{default}.

Le constructeur par défaut peut être utilisé dans deux constructions :
\begin{itemize}
  \item la déclaration d'une variable ou d'une constante ;
  \item dans une expression.
\end{itemize}

\subsection{Intérêt du constructeur par défaut}


L'intérêt du constructeur par défaut est qu'il allège l'écriture de l'initialisation des variables de certains types. Ce n'est pas une construction qui apporte de l'expressivité au langage (on peut très bien se passer d'appeler des constructeurs par défaut).

Pour un type comme \galgas{@uint}, écrire \galgas{@uint v [default] ;} est sémantiquement équivalent à écrire \galgas{@uint v := 0 ;}. On voit que le constructeur par défaut présente peu d'utilité ici.

Par contre, si l'on a un type structure :

\begin{galgascode}
struct @T {
  @uneMap mMap ;
  @uneListe mList ;
  @stringlist mStringList ;
  @stringset mStringSet ;
}
\end{galgascode}

Déclarer et initialiser une variable de ce type s'écrit :

\begin{galgascode}
@T variable [new
  ![@uneMap emptyMap]
  ![@uneListe emptyList]
  ![@stringlist emptyList]
  ![@stringset emptySet]
] ;
\end{galgascode}

Avec le constructeur par défaut, cette instruction s'écrit simplement :

\begin{galgascode}
@T variable [default] ;
\end{galgascode}

Pour une structure, comme on va le voir plus bas, le constructeur par défaut appelle le constructeur par défaut pour chaque champ ; le constructeur par défaut d'une \galgas{map} est équivalent à \galgas{emptyMap}, celui d'une \galgas{list}  équivalent à \galgas{emptyList}, et celui d'un \galgas{@stringset}  équivalent à \galgas{emptySet}.


\subsection{Appel dans la déclaration d'une variable ou d'une constante}

\begin{galgascode}
@T variable [default] ;
\end{galgascode}

Ceci déclare une variable de type \galgas{@T} et l'initialise avec le constructeur par défaut. Pour une constante, la syntaxe est :

\begin{galgascode}
const @T constante [default] ;
\end{galgascode}


\subsection{Appel dans une expression}

L'expression \galgas{[@T default]} invoque le constructeur par défaut du type \galgas{@T} et renvoie un objet initialisé du type \galgas{@T}.

\subsectionLabel{Les constructeurs par défaut pour chaque type}{constructeurParDefautPourChaqueType}

Le \refTableau{constructeurParDefaut} précise par chaque type l'existence du constructeur par défaut.


\begin{table}[t]
  \centering
%  \rowcolors{2}{\fondTableau}{}
  \begin{tabular}{@{}lllllll@{}}
  \textbf{Type} & \textbf{Constructeur par défaut} \\
  \galgas{abstract class @T} & \emph{Pas de constructeur par défaut} \\
  \galgas{@bool} & Initialisation à \galgas{false} \\
  \galgas{@application} & \emph{Pas de constructeur par défaut} \\
  \galgas{array @T} & \emph{Pas de constructeur par défaut} \\
  \galgas{@char} & Initialisation au caractère \texttt{NULL} \\
  \galgas{class @T} & Oui, si tous les attributs possèdent un constructeur par défaut \\
  \galgas{@data} & Équivalent au constructeur \galgas{emptyData} \\
  \galgas{@double} & Initialisation à \texttt{0.0} \\
  \galgas{@filewrapper} & \emph{Pas de constructeur par défaut} \\
  \galgas{@function} & \emph{Pas de constructeur par défaut} \\
  \galgas{graph @T} & Équivalent au constructeur \galgas{emptyGraph} \\
  \galgas{list @T} & Équivalent au constructeur \galgas{emptyList} \\
  \galgas{map @T} & Équivalent au constructeur \galgas{emptyMap} \\
  \galgas{listmap @T} & Équivalent au constructeur \galgas{emptyMap} \\
  \galgas{@object} & \emph{Pas de constructeur par défaut} \\
  \galgas{@sint} & Initialisation à \galgas{0S} \\
  \galgas{@sint64} & Initialisation à \galgas{0LS} \\
  \galgas{sortedlist @T} & Équivalent au constructeur \galgas{emptySortedList} \\
  \galgas{@string} & Initialisation à chaîne vide \galgas{""} \\
  \galgas{@stringset} & Équivalent au constructeur \galgas{emptySet} \\
  \galgas{struct @T} & Oui, si tous les attributs possèdent un constructeur par défaut \\
  \galgas{@type} & \emph{Pas de constructeur par défaut} \\
  \galgas{@uint} & Initialisation à \galgas{0} \\
  \galgas{@uint64} & Initialisation à \galgas{0L} \\
  \end{tabular}
  \caption{Constructeur par défaut pour chaque type}
  \labelTableau{constructeurParDefaut}
  \ligne
\end{table}

Remarques :
\begin{itemize}
  \item une classe abstraite ne possède pas de constructeur par défaut ;
  \item une classe concrète possède un constructeur par défaut si tous les attributs (ceux déclarés dans la classe et les super classes) en possèdent un ; la valeur par défaut est celle définie par l'appel du constructeur par défaut sur tous ces attributs ;
  \item une structure possède un constructeur par défaut si tous ces champs en possèdent un ; la valeur par défaut est celle définie par l'appel du constructeur par défaut sur tous les champs.
\end{itemize}


  %--------------------------------------------------------------
\chapter{Predefined Types} \label{predefinedTypes}
%-------------------------------------------------------------

GALGAS predefines several types. This chapter presents all their features, including their constructors, readers, modifiers, methods, ...


\begin{description}
\item The predefined types are:
\begin{itemize}
\item \lienSectionType{@location}, whose value points out a location in a source file;
\item \lienSectionType{@uint}, the 32-bit unsigned integers.
\end{itemize}
\end{description}

%!TEX encoding = UTF-8 Unicode
%!TEX root = ../galgas-book.tex

\chapitreTypePredefiniLabelIndex{location}

An \galgas{@location} object value is a location in a source file. Objects of this type are useful for pointing out an error or a warning location.

\section{The \texttt{here} Keyword}

The \galgas{here} keyword indicates the current parsing location is the current source file. Assigning an \galgas{@location} object from the \galgas{here} keyword is a way for initializing an \galgas{@location} object:\newline

\texttt{@location currentLocation := here ;}

\section{Constructor}

\constructeurSansArgument{nowhere}
{location}
{2.1.2}
{location}
{Returns an \galgas{@location} that does not points out any location.}
{The returned object responds \galgas{true} to the \refReaderPage{location}{isNowhere}.}

\section{Readers}

\readerSansArgument{column}
{location}
{1.8.2}
{uint}
{Returns an \galgas{@uint} value containing the column of the receiver's value.}
{this reader raises a run-time error if the receiver's value responds \galgas{true} to the \refReaderPage{location}{isNowhere}.}


\readerSansArgument{isNowhere}
{location}
{2.1.2}
{bool}
{Returns an \galgas{@bool} value indicating whether the receiver'value points out a source location or does not.}
{this reader returns \galgas{true} if the receiver's value does not point out an actual location in a text source (i.e. it has been constructed using the nowhere constructor), and \galgas{false} if the receiver's value points out an actual location in a text source (i.e. it has been constructed using the \galgas{here} keyword.}


\readerSansArgument{line}
{location}
{1.8.2}
{uint}
{Returns an \galgas{@uint} value containing the line of the receiver's value.}
{this reader raises a run-time error if the receiver's value responds \galgas{true} to the \refReaderPage{location}{isNowhere}.}


\readerSansArgument{locationIndex}
{location}
{1.8.2}
{uint}
{Returns an \galgas{@uint} value containing the the offset from the the beginning of the source of the location defined by receiver's value.}
{this reader raises a run-time error if the receiver's value responds \galgas{true} to the \refReaderPage{location}{isNowhere}.}


\readerSansArgument{locationString}
{location}
{1.8.2}
{string}
{returns an \galgas{@string} object that contains a string representation of the location defined by receiver's value.}
{this reader raises a run-time error if the receiver's value responds \galgas{true} to the \refReaderPage{location}{isNowhere}.}

%!TEX encoding = UTF-8 Unicode
%!TEX root = ../galgas-book.tex

\chapitreTypePredefiniLabelIndex{uint}

\tableDesMatieresLocaleDeProfondeurRelative{1}


An \ggst+@uint+ object value is a 32-bit unsigned integer value. You can initialize an \ggst+@uint+ object from an unsigned integer constant:\\

\begin{galgas3}
@uint myUnsignedInteger = 123_456 ;
\end{galgas3}

Note that a 32-bit unsigned integer constant is characterized by no suffix.

\section{Constructors}

\subsectionConstructor{errorCount}{uint}

\begin{galgas3}
constructor errorCount -> @uint
\end{galgas3}


Returns an \ggst+@uint+ object that contains the number of errors. The returned value is the cumulative count of errors from the beginning of execution.

\textbf{Exemple :}
\begin{galgas3}
@uint x = [@uint errorCount] ;
\end{galgas3}




\subsectionConstructor{max}{uint}

\begin{galgas3}
constructor max -> @uint
\end{galgas3}

Returns an \ggst+@uint+ object that the maximum value of the 32-bit unsigned range ($2^{32}-1$).






\subsectionConstructor{random}{uint}

\begin{galgas3}
constructor random -> @uint
\end{galgas3}

Retourne une valeur aléatoire de type \ggst+@uint+. La procédure de type \refStaticProcPage{uint}{setRandomSeed} permet d'en fixer la valeur initiale.

\begin{galgas3}
  let v = @uint.random
\end{galgas3}


{\bf Note. } Sur Unix, la valeur renvoyée est la valeur renvoyée par l'appel de la fonction \texttt{random} de la librairie \texttt{libc}. Sur Windows, c'est la fonction \texttt{rand} qui est appelée.


\subsectionConstructor{valueWithMask}{uint}

\begin{galgas3}
constructor valueWithMask ?@uint inLowerIndex ?@uint inUpperIndex -> @uint
\end{galgas3}


Returns an \ggst+@uint+ object with bits from \emph{inLowerIndex} to \emph{inUpperIndex} equal to 1.

A run-time error is raised if \emph{inLowerIndex $>$ inUpperIndex} or if \emph{inUpperIndex $>$ 31}.



\textbf{Exemple :}
\begin{galgas3}
@uint x = [@uint valueWithMask !2 !4] ; # x is equal to 28 (0b1_1100)
\end{galgas3}




\subsectionConstructor{warningCount}{uint}

\begin{galgas3}
constructor warningCount -> @uint
\end{galgas3}


Returns an \ggst+@uint+ object that contains the number of warnings. The returned value is the cumulative count of warnings from the beginning of execution.





\section{Procédure de type}


\subsectionStaticProc{setRandomSeed}{uint}


\begin{galgas3box}
proc @uint setRandomSeed ?@uint inSeed
\end{galgas3box}

Affecte la valeur initiale utilisée par le générateur de nombres aléatoires (voir le \refConstructorPage{uint}{random}) Par exemple~:

\begin{galgas3}
  [@uint setRandomSeed !0]
\end{galgas3}






\section{Getters}

\subsectionGetter{alphaString}{uint}

Ce \emph{getter} permet de convertir un \ggst!@uint! en une chaîne de caractères, telle que l'ordre des entiers est conservé sur la chaîne obtenue.

La chaîne obtenue comporte exactement 7 lettres minuscules. C'est en fait une conversion en base 26, la lettre \ggst=a= ayant la valeur $0$, et la lettre \ggst=z= la valeur $25$.


\begin{galgas3}
  message [0 alphaString] + "\n"         # aaaaaaa
  message [12_345 alphaString] + "\n"    # aaaasgv
  message [@uint.max alphaString] + "\n" # nxmrlxv
\end{galgas3}



\subsectionGetter{bigint}{uint}

Ce \emph{getter} permet de convertir un \ggst!@uint! en \ggst!@bigint!. Comme la plage des valeurs des \ggst!bigint! n'est limitée que par la mémoire disponible, il n'échoue jamais.

\begin{galgas3}
  message [[1234 bigint] string] + "\n" # 1234
\end{galgas3}


\subsectionGetter{double}{uint}

\begin{galgas3}
getter double -> @double
\end{galgas3}

Returns the receiver's value converted in a \ggst+@double+ object. As a 32-bit integer value can always be converted in a \ggst+@double+ value, this getter never fails.



\subsectionGetter{hexString}{uint}

\begin{galgas3}
getter hexString -> @string
\end{galgas3}

Returns the an hexadecimal string representation of the receiver value, prefixed by the string \texttt{0x}. For getting an hexadecimal representation string without any prefix, see \refGetterPage{uint}{xString}.



\subsectionGetter{hexStringSeparatedBy}{uint}

\begin{galgas3}
getter hexStringSeparatedBy ?@char inSeparator ?@uint inGroup -> @string
\end{galgas3}

Returns the an hexadecimal string representation of the receiver value, prefixed by the string \texttt{0x}. Groups of \ggst=inGroup= digits are separated by the \ggst=inSeparator= character.

If \ggst=inGroup= is equal to zero, a run-time error is raised.

For example:
\begin{galgas3}
let s = [0x12345678 hexStringSeparatedBy !'_' !2] # 0x12_34_56_78
\end{galgas3}



\subsectionGetter{isInRange}{uint}

\begin{galgas3}
getter isInRange ?@range inRange -> @bool
\end{galgas3}

{Returns an \ggst+@bool+ value indicating whether the receiver'value belongs to \ggst+inRange+ range : for a receiver's value equal to $v$ and a range of length $length$ starting at $start$, it returns \ggst+true+ if $((v \geqslant start)~and~(v<(start+length)))$, and \ggst+false+ otherwise.



\subsectionGetter{isUnicodeValueAssigned}{uint}

\begin{galgas3}
getter isUnicodeValueAssigned -> @bool
\end{galgas3}

Returns an \ggst+@bool+ value indicating whether the receiver'value represents an assigned Unicode character. It returns \ggst+true+ if the receiver value represents an assigned Unicode character, \ggst+false+ and otherwise.

\textbf{Exemple :}
\begin{galgas3}
[0xFFFF isUnicodeValueAssigned] # is false, as \uFFFF is not assigned.
[0x41 isUnicodeValueAssigned] # is true, as \u0041 is assigned (LATIN CAPITAL LETTER A).
\end{galgas3}



\subsectionGetter{lsbIndex}{uint}

\begin{galgas3}
getter lsbIndex -> @uint
\end{galgas3}

Returns an \ggst+@uint+ value of the index of the most significant bit of the receiver value. It raises a run-time error if the receiver value is zero.

\textbf{Exemple :}
\begin{galgas3}
@uint value = 192 ; # 192 is ...011000000 in binary
@uint x = [value lsbIndex] ; # x is equal to 7
\end{galgas3}

The most significant bit of 192 is the 7th bit.




\subsectionGetter{significantBitCount}{uint}

\begin{galgas3}
getter significantBitCount -> @uint
\end{galgas3}

Returns the number of bits needed to express the receiver value. If the receiver value is zero, it returns 0 ; otherwise, it returns the most significant bit index plus one.

\textbf{Exemple :}
\begin{galgas3}
@uint value = 145 ; # 145 is 10010001 in binary
@uint x = [value significantBitCount] ; # x is equal to 8
\end{galgas3}






\subsectionGetter{sint}{uint}

\begin{galgas3}
getter sint -> @sint
\end{galgas3}

Returns the receiver's value in an \refTypePredefini{sint} (32-bit signed integer) object. An error is raised is receiver's value is greater than $2^{31}-1$.

This getter is the only way to convert an \refTypePredefini{uint} object into an \refTypePredefini{sint} object.




\subsectionGetter{sint64}{uint}

\begin{galgas3}
getter sint64 -> @sint64
\end{galgas3}

Returns the receiver's value in an \refTypePredefini{sint64} (64-bit signed integer) object. As a 32-bit unsigned value can always be converted in a 64-bit signed value, this getter never fails.

This getter is the only way to convert an \refTypePredefini{uint} object into an \refTypePredefini{sint64} object.


\subsectionGetter{string}{uint}

\begin{galgas3}
getter string -> @string
\end{galgas3}

Returns a decimal string representation of the receiver's value. For an hexadecimal string representation of the receiver's value, see \refGetterPage{uint}{hexString} and \refGetterPage{uint}{xString}.




\subsectionGetter{uint64}{uint}

\begin{galgas3}
getter uint64 -> @uint64
\end{galgas3}

Returns the receiver's value in an \refTypePredefini{uint64} (64-bit unsigned integer) object. As a 32-bit unsigned value can always be converted in a 64-bit unsigned value, this getter never fails.

This getter is the only way to convert an \refTypePredefini{uint} object into an \refTypePredefini{uint64} object.




\subsectionGetter{xString}{uint}

\begin{galgas3}
getter xString -> @string
\end{galgas3}

Returns an hexadecimal string representation of the receiver's value (without any prefix). For an decimal string representation of the receiver's value, see the \refGetterPage{uint}{hexString}; for a decimal string representation of the receiver's value, see the \refGetterPage{uint}{string}.







\section{Arithmétique}

\subsection{Opérateurs infixés}

Le type \ggst+@uint+ accepte les opérateurs arithmétiques infixés suivants :
\begin{itemize}
  \item \ggst!+!, addition, une erreur d'exécution est déclenchée en cas de débordement ;
  \item \ggst!-!, soustraction, une erreur d'exécution est déclenchée en cas de débordement ;
  \item \ggst!*!, multiplication, une erreur d'exécution est déclenchée en cas de débordement ;
  \item \ggst!/!, division, une erreur d'exécution est déclenchée si le diviseur est nul ;
  \item \ggst!mod!, calcul du reste, une erreur d'exécution est déclenchée si le diviseur est nul ;
  \item \ggst!&+!, addition, le résultat étant silencieusement tronqué sur 32 bits ;
  \item \ggst!&-!, soustraction, le résultat étant silencieusement tronqué sur 32 bits ;
  \item \ggst!&*!, multiplication, le résultat étant silencieusement tronqué sur 32 bits ;
  \item \ggst!&/!, division, qui retourne zéro si le diviseur est nul.
\end{itemize}

Ces opérateurs exigent que les deux opérandes soient des objets du même type \ggst+@uint+.

\subsection{Opérateur préfixé}
Le type \ggst+@uint+ accepte un opérateur arithmétique préfixé :
\begin{itemize}
  \item \ggst!+!, qui retourne simplement la valeur de l'opérande.
\end{itemize}

\subsectionLabel{Instructions}{instructionsUINT}

Le type \ggst+@uint+ accepte les deux instructions arithmétiques suivantes :
\begin{itemize}
  \item \ggst!+=!, addition, une erreur d'exécution est déclenchée en cas de débordement ;
  \item \ggst!-=!, soustraction, une erreur d'exécution est déclenchée en cas de débordement ;
  \item \ggst!*=!, multiplication, une erreur d'exécution est déclenchée en cas de débordement ;
  \item \ggst!/=!, division, une erreur d'exécution est déclenchée en cas division par zéro ;
  \item \ggst!++!, incrémentation, une erreur d'exécution est déclenchée en cas de débordement ;
  \item \ggst!--!, décrémentation, une erreur d'exécution est déclenchée en cas de débordement ;
  \item \ggst!&++!, incrémentation, le résultat étant silencieusement tronqué sur 32 bits ;
  \item \ggst!&--!, décrémentation, le résultat étant silencieusement tronqué sur 32 bits.
\end{itemize}

\ggst!x+=y! est équivalent à \ggst!x=x+y! ; \ggst!x-=y! est équivalent à \ggst!x=x-y!.
La variable cible \ggst!x!, comme l'expression source \ggst!y! doivent être du même type \ggst+@uint+.

Incrémentation et décrémentation sont des instructions, et ne peuvent pas apparaître des expressions.
\begin{galgas3}
@uint n = ... ; n ++ # Incrémentation
\end{galgas3}

\begin{galgas3}
@uint n = ... ; n -- # Décrémentation
\end{galgas3}




\section{Shift Operators}


The \ggst+@uint+ type supports right and left shift operators:\newline

\begin{tabular}{|c|c|}
\hline
$<<$ & Left shift \\
\hline
$>>$ & Right shift \\
\hline
\end{tabular}

Theses operators require both arguments to be \ggst+@uint+ objects.\newline

Note the right shift inserts always a zero bit in the most significant bit location (it is a logical right shift).\newline

The actual amount of the shift is the value of the right-hand operand masked by 31, i.e. the shift distance is always between 0 and 31.




\section{Logical Operators}

The \ggst+@uint+ type supports the three bit-wise logical operators:\newline

\begin{tabular}{|c|c|}
\hline
$\&$ & Bit-wise and \\
\hline
\textbar & Bit-wise or \\
\hline
\^\  & Bit-wise exclusive or \\
\hline
\end{tabular}

Theses operators require both arguments to be \ggst+@uint+ objects.\newline


The \ggst+@uint+ type supports the bit-wise logical unary operator:\newline

\begin{tabular}{|c|c|}
\hline
$\sim$ & Bit-wise complementation \\
\hline
\end{tabular}

This operator returns an \ggst+@uint+ object.







\section{Comparison Operators}

The \ggst+@uint+ type supports the six comparison operators:\newline

\begin{tabular}{|c|c|}
\hline
$=$ & Equality \\
\hline
$!=$ & Non Equality \\
\hline
$<$  & Strict Lower Than \\
\hline
$<=$  & Lower or Equal \\
\hline
$>$  & Strict Greater Than \\
\hline
$>=$  & Greater or Equal \\
\hline
\end{tabular}

\vspace{2mm}
Theses operators require both arguments to be \ggst+@uint+ objects, and return a \ggst+@bool+ object.





  %!TEX encoding = UTF-8 Unicode
%!TEX root = ../galgas-book.tex

\chapitreTypePredefiniLabelIndex{binaryset}

\tableDesMatieresLocaleDeProfondeurRelative{1}


Le type \ggst+@binaryset+ encode des ensembles, des relations binaires, des expressions booléennes. Il est implémenté par des BDD (Binary Decision Diagrams).


\section{Constructeurs}

\subsectionConstructor{binarySetWithBit}{binaryset}

\begin{galgas3box}
constructor binarySetWithBit ?@uint inBitIndex -> @binaryset
\end{galgas3box}


Retourne un \ggst+@binaryset+ dont le bit \ggst+inBitIndex+ est égal à 1.


\textbf{Exemple :}
\begin{galgas3}
@binaryset s = .binarySetWithBit {!2}
log s # Affiche <@binaryset: 1XX>
\end{galgas3}


\subsectionConstructor{binarySetWithEqualComparison}{binaryset}

\begin{galgas3box}
constructor binarySetWithEqualComparison
  ?@uint inLeftFirstIndex
  ?@uint inBitCount
  ?@uint inRightFirstIndex
  -> @binaryset
\end{galgas3box}




Retourne un \ggst+@binaryset+ qui encode la relation d'égalité entre deux variables.

Ce constructeur retourne un binary set qui encode la relation \emph{a~==~b}, où \emph{a} est encodé à partir du bit d'indice \emph{inLeftFirstIndex} jusqu'au bit d'indice \emph{inLeftFirstIndex  + inBitCount - 1}, et \emph{b} est encodé à partir du bit d'indice bit \emph{inRightFirstIndex} jusqu'au bit d'indice \emph{inRightFirstIndex + inBitCount - 1}.

\textbf{Exemple :}
\begin{galgas3}
@binaryset s = .binarySetWithEqualComparison {!0 !2 !3}
log s # Affiche <@binaryset: 00x00, 01X01, 10X10, 11X11>
\end{galgas3}




\subsectionConstructor{binarySetWithEqualToConstant}{binaryset}

\begin{galgas3box}
constructor binarySetWithEqualToConstant
  ?@uint inLeftFirstIndex
  ?@uint inBitCount
  ?@uint64 inConstant
  -> @binaryset
\end{galgas3box}


Retourne un \ggst+@binaryset+ object that encodes a equality relation between a variable and a constant.

Ce constructeur retourne un objet qui encode la relation the \emph{a~==~cst}, où \emph {a} est encodé à partir du bit d'indice \emph{inBitIndex} jusqu'au bit d'indice \emph{inBitIndex  + inBitCount - 1}, et \emph{cst} est défini par l'argument \emph{inConstant}.

\textbf{Exemple :}
\begin{galgas3}
@binaryset s = .binarySetWithEqualToConstant {!0 !6 !23L}
log s # Affiche <@binaryset: 10111>
\end{galgas3}




\subsectionConstructor{binarySetWithGreaterOrEqualComparison}{binaryset}

\begin{galgas3box}
constructor binarySetWithGreaterOrEqualComparison
  ?@uint inLeftFirstIndex
  ?@uint inBitCount
  ?@uint inRightFirstIndex
  -> @binaryset
\end{galgas3box}


Retourne un \ggst+@binaryset+ object qui encode la relation \emph{supérieur ou égal} entre deux variables.

Ce constructeur retourne un binary set qui encode la relation \emph{a~>=~b}, où \emph{a} est encodé à partir du bit d'indice \emph{inLeftFirstIndex} jusqu'au bit d'indice \emph{inLeftFirstIndex  + inBitCount - 1}, et \emph{b} est encodé à partir du bit d'indice bit \emph{inRightFirstIndex} jusqu'au bit d'indice \emph{inRightFirstIndex + inBitCount - 1}.

\textbf{Exemple :}
\begin{galgas3}
@binaryset s = .binarySetWithGreaterOrEqualComparison {!0 !2 !3}
log s # Affiche <@binaryset: 00XXX, 01X01, 01X1X, 10X1X, 11X11>
\end{galgas3}



\subsectionConstructor{binarySetWithGreaterOrEqualToConstant}{binaryset}

\begin{galgas3box}
constructor binarySetWithGreaterOrEqualToConstant
  ?@uint inLeftFirstIndex
  ?@uint inBitCount
  ?@uint64 inConstant
  -> @binaryset
\end{galgas3box}



Retourne un \ggst+@binaryset+ object that encodes a greater or equal relation between a variable and a constant.

The constructor returns a binary set that encodes the \emph{a~>=~cst} relation, where \emph {a} est encodé à partir du bit d'indice \emph{inBitIndex} jusqu'au bit d'indice \emph{inBitIndex  + inBitCount - 1}, and \emph{cst} is defined by the \emph{inConstant} argument.




\subsectionConstructor{binarySetWithLowerOrEqualComparison}{binaryset}

\begin{galgas3box}
constructor binarySetWithLowerOrEqualComparison
  ?@uint inLeftFirstIndex
  ?@uint inBitCount
  ?@uint inRightFirstIndex
  -> @binaryset
\end{galgas3box}


Retourne un \ggst+@binaryset+ object that encodes a lower or equal relation between two variables.

The constructor returns a binary set that encodes the \emph{a~<=~b} relation, where \emph{a} est encodé à partir du bit d'indice \emph{inLeftFirstIndex} jusqu'au bit d'indice \emph{inLeftFirstIndex  + inBitCount - 1}, and \emph{b} est encodé à partir du bit d'indice \emph{inRightFirstIndex} to \emph{inRightFirstIndex + inBitCount - 1}.

\textbf{Exemple :}
\begin{galgas3}
@binaryset s = .binarySetWithLowerOrEqualComparison !0 !2 !3]
log s # Affiche <@binaryset: 00X00, 01X0X, 10X0X, 10X10, 11XXX>
\end{galgas3}




\subsectionConstructor{binarySetWithLowerOrEqualToConstant}{binaryset}

\begin{galgas3box}
constructor binarySetWithLowerOrEqualToConstant
  ?@uint inLeftFirstIndex
  ?@uint inBitCount
  ?@uint64 inConstant
  -> @binaryset
\end{galgas3box}


Retourne un \ggst+@binaryset+ object that encodes a lower or equal relation between a variable and a constant.

The constructor returns a binary set that encodes the \emph{a~<=~cst} relation, where \emph {a} est encodé à partir du bit d'indice \emph{inBitIndex} jusqu'au bit d'indice \emph{inBitIndex  + inBitCount - 1}, and \emph{cst} is defined by the \emph{inConstant} argument.




\subsectionConstructor{binarySetWithNotEqualComparison}{binaryset}

\begin{galgas3box}
constructor binarySetWithNotEqualComparison
  ?@uint inLeftFirstIndex
  ?@uint inBitCount
  ?@uint inRightFirstIndex
  -> @binaryset
\end{galgas3box}



Retourne un \ggst+@binaryset+ object that encodes an inequality relation between two variables.

The constructor returns a binary set that encodes the \emph{a~!=~b} relation, where \emph{a} est encodé à partir du bit d'indice \emph{inLeftFirstIndex} jusqu'au bit d'indice \emph{inLeftFirstIndex  + inBitCount - 1}, and \emph{b} est encodé à partir du bit d'indice \emph{inRightFirstIndex} to \emph{inRightFirstIndex + inBitCount - 1}.

\textbf{Exemple :}
\begin{galgas3}
@binaryset s = .binarySetWithNotEqualComparison !0 !2 !3]
log s # Affiche <@binaryset: 00X01, 00X1X, 01X00, 01X1X, 10X0X, 10X11, 11X0X, 11X10>
\end{galgas3}




\subsectionConstructor{binarySetWithNotEqualToConstant}{binaryset}

\begin{galgas3box}
constructor binarySetWithNotEqualToConstant
  ?@uint inLeftFirstIndex
  ?@uint inBitCount
  ?@uint64 inConstant
  -> @binaryset
\end{galgas3box}


Retourne un \ggst+@binaryset+ object that encodes an inequality relation between a variable and a constant.

The constructor returns a binary set that encodes the \emph{a~!=~cst} relation, where \emph {a} est encodé à partir du bit d'indice \emph{inBitIndex} jusqu'au bit d'indice \emph{inBitIndex  + inBitCount - 1}, and \emph{cst} is defined by the \emph{inConstant} argument.







\subsectionConstructor{binarySetWithPredicateString}{binaryset}

\begin{galgas3box}
constructor binarySetWithPredicateString ?@string inPredicateString -> @binaryset
\end{galgas3box}

Returns the \ggst+@binaryset+ object described by the \emph{inPredicateString} argument.

The \emph{inBitString} argument string encodes a predicate string, such as those returned by \refGetterPage{binaryset}{predicateStringValue}.

\begin{description}
\item The \emph{inBitString} argument string characters should have one of the five following values:
\begin{itemize}
\item \texttt{\textquotesingle 0\textquotesingle}: a bit set to zero;
\item \texttt{\textquotesingle 1\textquotesingle}: a bit set to one;
\item \texttt{\textquotesingle X\textquotesingle}: a don't care bit;
\item \texttt{\textquotesingle~\textquotesingle}: a separator (non significant character);
\item \texttt{\textquotesingle\textbar\textquotesingle}: the boolean \emph{or} operation (in infix notation).
\end{itemize}
\end{description}


\textbf{Exemple :}
An empty predicate string (or a string containing only spaces) provides an empty binary set:
\begin{galgas3}
@binaryset s = .binarySetWithPredicateString !" "]
@bool b = = .s isEmptySet]; # b is true
\end{galgas3}


A predicate string containing only 'X' characters (at least one) provides an full binary set:
\begin{galgas3}
@binaryset s = .binarySetWithPredicateString !" X X"] # Spaces are non significant
@bool b = [s isFullSet]; # b is true
\end{galgas3}


A predicate string can encode a binary value (MSB first):
\begin{galgas3}
@binaryset s [binarySetWithPredicateString !"1100"] # 12 in decimal
log s # Affiche <@binaryset: 1100>
\end{galgas3}


You can use the boolean '|' operator for providing an or'ed values:
\begin{galgas3}
@binaryset s [binarySetWithPredicateString !" 1100 | 1101"]
log s # Affiche <@binaryset: 110X>
\end{galgas3}



You can use you can use don't care bits and '|' operator together:
\begin{galgas3}
@binaryset s [binarySetWithPredicateString !"11X00X0 | 111XXX"]
log s # Affiche <@binaryset: 1100X0, 111XXX>
\end{galgas3}




\subsectionConstructor{binarySetWithStrictGreaterComparison}{binaryset}

\begin{galgas3box}
constructor binarySetWithStrictGreaterComparison
  ?@uint inLeftFirstIndex
  ?@uint inBitCount
  ?@uint inRightFirstIndex
  -> @binaryset
\end{galgas3box}


Retourne un \ggst+@binaryset+ object that encodes a strict greater than relation between two variables.

The constructor returns a binary set that encodes the \emph{a~>~b} relation, where \emph{a} est encodé à partir du bit d'indice \emph{inLeftFirstIndex} jusqu'au bit d'indice \emph{inLeftFirstIndex  + inBitCount - 1}, and \emph{b} est encodé à partir du bit d'indice \emph{inRightFirstIndex} to \emph{inRightFirstIndex + inBitCount - 1}.

\textbf{Exemple :}
\begin{galgas3}
@binaryset s [binarySetWithStrictGreaterComparison !0 !2 !3]
log s # Affiche <@binaryset: 00X01, 00X1X, 01X1X, 10X11>
\end{galgas3}




\subsectionConstructor{binarySetWithStrictGreaterThanConstant}{binaryset}

\begin{galgas3box}
constructor binarySetWithStrictGreaterThanConstant
  ?@uint inLeftFirstIndex
  ?@uint inBitCount
  ?@uint64 inConstant
  -> @binaryset
\end{galgas3box}


Retourne un \ggst+@binaryset+ object that encodes a strict greater than relation between a variable and a constant.

The constructor returns a binary set that encodes the \emph{a~>~cst} relation, where \emph {a} est encodé à partir du bit d'indice \emph{inBitIndex} jusqu'au bit d'indice \emph{inBitIndex  + inBitCount - 1}, and \emph{cst} is defined by the \emph{inConstant} argument.




\subsectionConstructor{binarySetWithStrictLowerComparison}{binaryset}

\begin{galgas3box}
constructor binarySetWithStrictLowerComparison
  ?@uint inLeftFirstIndex
  ?@uint inBitCount
  ?@uint inRightFirstIndex
  -> @binaryset
\end{galgas3box}


Retourne un \ggst+@binaryset+ object that encodes a strict lower than relation between two variables.

The constructor returns a binary set that encodes the \emph{a~<~b} relation, where \emph{a} est encodé à partir du bit d'indice \emph{inLeftFirstIndex} jusqu'au bit d'indice \emph{inLeftFirstIndex  + inBitCount - 1}, and \emph{b} est encodé à partir du bit d'indice \emph{inRightFirstIndex} to \emph{inRightFirstIndex + inBitCount - 1}.

\textbf{Exemple :}
\begin{galgas3}
@binaryset s [binarySetWithStrictLowerComparison !0 !2 !3]
log s # Affiche <@binaryset: 01X00, 10X0X, 11X0X, 11X10>
\end{galgas3}




\subsectionConstructor{binarySetWithStrictLowerThanConstant}{binaryset}

\begin{galgas3box}
constructor binarySetWithStrictLowerThanConstant
  ?@uint inLeftFirstIndex
  ?@uint inBitCount
  ?@uint64 inConstant
  -> @binaryset
\end{galgas3box}


Retourne un \ggst+@binaryset+ object that encodes a strict lower than relation between a variable and a constant.

The constructor returns a binary set that encodes the \emph{a~<~cst} relation, where \emph {a} est encodé à partir du bit d'indice \emph{inBitIndex} jusqu'au bit d'indice \emph{inBitIndex  + inBitCount - 1}, and \emph{cst} is defined by the \emph{inConstant} argument.




\subsectionConstructor{emptyBinarySet}{binaryset}

\begin{galgas3box}
constructor emptyBinarySet -> @binaryset
\end{galgas3box}


Retourne un empty \ggst+@binaryset+ object.





\subsectionConstructor{fullBinarySet}{binaryset}

\begin{galgas3box}
constructor fullBinarySet -> @binaryset
\end{galgas3box}

Returns a full \ggst+@binaryset+ object.


\section{Getters}



\subsectionGetter{accessibleStates}{binaryset}

\begin{galgas3box}
getter accessibleStates -> @binaryset
\end{galgas3box}

Returns the set of accessible states from an initial state set. It computes the set of accessible states from the \emph{inInitialStateSet} state set using the accessibility relation encoded by the receiver.

\textbf{Exemple :}
\begin{galgas3}
@binaryset gr [binarySetWithPredicateString !"0001 0000"] # Edge 0 -> 1
gr = gr | [@binaryset binarySetWithPredicateString !"0010 0001"] # Edge 1 -> 2
gr = gr | [@binaryset binarySetWithPredicateString !"0011 0010"] # Edge 2 -> 3
gr = gr | [@binaryset binarySetWithPredicateString !"0100 0011"] # Edge 3 -> 4
gr = gr | [@binaryset binarySetWithPredicateString !"0101 0100"] # Edge 4 -> 5
@binaryset initialState [binarySetWithPredicateString !"0000"] # 0 is the initial state
@binaryset accessibleStates = [gr accessibleStates !initialState !4]
message " Accessible:"
@uint64list valueList = [accessibleStates uint64ValueList !4]
foreach valueList do
  message " " . [mValue string]
end foreach
message "\n"
\end{galgas3}


This program Affiche: \texttt{Accessible: 0 1 2 3 4 5}.



\subsectionGetter{binarySetByTranslatingFromIndex}{binaryset}

\begin{galgas3box}
getter binarySetByTranslatingFromIndex ?@uint inFirstIndex ?@uint inTranslation -> @string
\end{galgas3box}


Returns a \ggst+@binaryset+ value computed by translating the receiver's value by \emph{inTranslation} bits from index \emph{inFirstIndex}.



\subsectionGetter{compressedValueCount}{binaryset}

\begin{galgas3box}
getter compressedValueCount -> @uint64
\end{galgas3box}

Returns in an \ggst+@uint64+ value the number of different compressed string values encoded by receiver's value.



\subsectionGetter{compressedStringValueList}{binaryset}

\begin{galgas3box}
getter compressedStringValueList ?@uint inBitCount -> @stringlist
\end{galgas3box}

Returns the list of compressed string values corresponding to receiver's value, considering it uses \emph{inBitCount} bits.










\subsectionGetter{containsValue}{binaryset}

\begin{galgas3box}
getter containsValue ?@uint inFirstBit ?@uint inBitCount -> @bool
\end{galgas3box}


Retourne un \ggst+@bool+ value indicating whether the receiver'value contains a given value: \ggst+true+ if the receiver's contains a value, and \ggst+false+ otherwise; this value is computed from the \emph{inBitCount} first bits of \emph{inValue} value, shifted left by \emph{inFirstBit}.


\textbf{Exemple :}
\begin{galgas3}
var s = @binaryset.binarySetWithPredicateString {!"0 00XX X111| 1 1111 1111"}
log s # Affiche <@binaryset: 000XXX111, 111111111>
@bool b = [s containsValue !127L !0 !7]
log b # Affiche <@bool:true>
b = [s containsValue !31L !1 !7]
log b # Affiche <@bool:true>
b = [s containsValue !63L !1 !8]
log b # Affiche <@bool:false>
b = [s containsValue !7L !0 !9]
log b # Affiche <@bool:true>
b = [s containsValue !7L !0 !10]
log b # Affiche <@bool:true>
b = [s containsValue !32767L !1 !12]
log b # Affiche <@bool:true>
\end{galgas3}








\subsectionGetter{equalTo}{binaryset}

\begin{galgas3box}
getter equalTo ?@binaryset inOperand -> @binaryset
\end{galgas3box}

Returns the complement of the exclusive or between the receiver's value and the operand's value.

Note that \ggst+[a equalTo !b]+ is equivalent to \texttt{$\sim$ (a $\wedge$ b)}.

This operation returns un \ggst+@binaryset+ value; do not confuse with \ggst+==+ operator that Retourne un \ggst+@bool+ value.







\subsectionGetter{existOnBitIndex}{binaryset}

\begin{galgas3box}
getter existOnBitIndex ?@uint inBitIndex -> @binaryset
\end{galgas3box}

Returns the binary computed by applying the \emph{exist} operator on the \emph{inBitIndex} bit of the receiver's value.






\subsectionGetter{existsOnBitRange}{binaryset}

\begin{galgas3box}
getter existsOnBitRange ?@uint inFirstBitIndex ?@uint inBitCount -> @bool
\end{galgas3box}


Returns the binary computed by applying the \emph{exist} operator on the receiver's value, from \emph{inFirstBitIndex} bit index until the \emph{inFirstBitIndex + inBitCount - 1} bit index.


\textbf{Exemple :}
\begin{galgas3}
@binaryset s [binarySetWithPredicateString !"01110010"]
log s # Affiche <@binaryset: 01110010>
@binaryset ss = [s existsOnBitRange !2 !3]
log s # Affiche <@binaryset: 011XXX10>
\end{galgas3}







\subsectionGetter{existOnBitIndexAndBeyond}{binaryset}

\begin{galgas3box}
getter existOnBitIndexAndBeyond ?@uint inBitIndex -> @binaryset
\end{galgas3box}

Returns the binary set computed by applying the \emph{exist} operator on all bits from \emph{inFirstBitIndex} bit index of the receiver's value.







\subsectionGetter{forAllOnBitIndex}{binaryset}

\begin{galgas3box}
getter forAllOnBitIndex ?@uint inBitIndex -> @binaryset
\end{galgas3box}

Returns the binary set computed by applying the \emph{for all} operator on the \emph{inFirstBitIndex} bit index of the receiver's value.







\subsectionGetter{forAllOnBitIndexAndBeyond}{binaryset}

\begin{galgas3box}
getter forAllOnBitIndexAndBeyond ?@uint inBitIndex -> @binaryset
\end{galgas3box}


Returns the binary computed by applying the \emph{for all} operator on all bits from \emph{inFirstBitIndex} bit index of the receiver's value.








\subsectionGetter{greaterOrEqualTo}{binaryset}

\begin{galgas3box}
getter greaterOrEqualTo ?@binaryset inOperand -> @binaryset
\end{galgas3box}


Returns the complement of the exclusive or between the receiver's value and the operand's value.

Note that \ggst+[a greaterOrEqualTo !b]+ is equivalent to \texttt{(a \textbar ~$\sim$b)}.








\subsectionGetter{isEmpty}{binaryset}

\begin{galgas3box}
getter isEmpty -> @bool
\end{galgas3box}

Returns a \ggst+@bool+ value that indicates whether the receiver's value is the empty set :  \ggst+true+ if receiver's value is the empty set, and \ggst+false+ otherwise.







\subsectionGetter{isFull}{binaryset}

\begin{galgas3box}
getter isFull -> @bool
\end{galgas3box}

Returns a \ggst+@bool+ value that indicates whether the receiver's value is the full set : \ggst+true+ if receiver's value is the full set, and \ggst+false+ otherwise.







\subsectionGetter{ITE}{binaryset}

\begin{galgas3box}
getter ITE ?@binaryset inThenOperand ?@binaryset inElseOperand -> @binaryset
\end{galgas3box}


Returns the binary set computed by applying the \emph{ite} operator on the receiver's value, the \emph{inThenOperand} argument, and the  \emph{inElseOperand} argument.

{\texttt{ite (x, y, z)} is \texttt{(x \& y) \textbar ($\sim$x \& z)}.}







\subsectionGetter{lowerOrEqualTo}{binaryset}

\begin{galgas3box}
getter lowerOrEqualTo ?@binaryset inOperand -> @binaryset
\end{galgas3box}


Returns the binary set computed by applying the \emph{lower or equal} operator on the receiver's value and the \emph{inOperand} argument.
{\texttt{[a lowerOrEqualTo !b]} is \texttt{(($\sim$x) \textbar y)}.}







\subsectionGetter{notEqualTo}{binaryset}

\begin{galgas3box}
getter notEqualTo ?@binaryset inOperand -> @binaryset
\end{galgas3box}


Returns the binary set computed by applying the \emph{not equal} operator on the receiver's value and the \emph{inOperand} argument.
{\texttt{[a notEqualTo !b]} is \texttt{(x $\wedge$ y)}.}







\subsectionGetter{predicateStringValue}{binaryset}

\begin{galgas3box}
getter predicateStringValue -> @string
\end{galgas3box}

Returns a string representation of the receiver's value. The returned string is compatible with the \refConstructorPage{binaryset}{binarySetWithPredicateString}.







\subsectionGetter{strictGreaterThan}{binaryset}

\begin{galgas3box}
getter strictGreaterThan ?@binaryset inOperand -> @binaryset
\end{galgas3box}

Returns the binary set computed by applying the \emph{strict greater} operator on the receiver's value and the \emph{inOperand} argument.
{\texttt{[a strictGreaterThan !b]} is \texttt{(x \& $\sim$y)}.}







\subsectionGetter{strictLowerThan}{binaryset}

\begin{galgas3box}
getter strictLowerThan ?@binaryset inOperand -> @binaryset
\end{galgas3box}

Returns the binary set computed by applying the \emph{strict lower} operator on the receiver's value and the \emph{inOperand} argument.
{\texttt{[a strictLowerThan !b]} is \texttt{($\sim$x \& y)}.}







\subsectionGetter{stringValueList}{binaryset}

\begin{galgas3box}
getter stringValueList ?@uint inBitCount -> @stringlist
\end{galgas3box}

Returns the list of string values corresponding to receiver's value, considering it uses \emph{inBitCount} bits.







\subsectionGetter{stringValueListWithNameList}{binaryset}

\begin{galgas3box}
getter stringValueListWithNameList
  ?@uint inBitCount
  ?@stringlist inNameList
  -> @stringlist
\end{galgas3box}


Returns the list of named values corresponding to receiver's value, considering it uses \emph{inBitCount} bits. First, the receiver is enumerated, considering it uses \emph{inBitCount} bits. Each enumerated value is used as an index of \emph{inNameList}, and the string value at this index is appended at the end of the returned value.







\subsectionGetter{swap021}{binaryset}

\begin{galgas3box}
getter swap021
  ?@uint inBitCount1
  ?@uint inBitCount2
  ?@uint inBitCount3
  -> @binaryset
\end{galgas3box}



Returns the transposed \emph{(x, z, y)} relation.

This getter considers that the receiver encodes an \emph{(x, y, z)} relation, where \emph{x} is defined by bits index \emph{0} to \emph{inBitCount1  - 1}, \emph{y} is defined by bits index \emph{inBitCount1} to \emph{inBitCount1 + inBitCount2 - 1} and  \emph{z} is defined by bits index \emph{inBitCount1 + inBitCount2} to \emph{inBitCount1 + inBitCount2 + inBitCount3 - 1}.







\subsectionGetter{swap01}{binaryset}

\begin{galgas3box}
getter swap01 ?@uint inBitCount1 ?@uint inBitCount2 -> @binaryset
\end{galgas3box}


Returns the transposed \emph{(y, x)} relation.

This getter considers that the receiver encodes an \emph{(x, y)} relation, where \emph{x} is defined by bits index \emph{0} to \emph{inBitCount1  - 1}, \emph{y} is defined by bits index \emph{inBitCount1} to \emph{inBitCount1 + inBitCount2 - 1}.





\subsectionGetter{swap102}{binaryset}

\begin{galgas3box}
getter swap102
  ?@uint inBitCount1
  ?@uint inBitCount2
  ?@uint inBitCount3
  -> @binaryset
\end{galgas3box}

Returns the transposed \emph{(y, x, z)} relation.

This getter considers that the receiver encodes an \emph{(x, y, z)} relation, where \emph{x} is defined by bits index \emph{0} to \emph{inBitCount1  - 1}, \emph{y} is defined by bits index \emph{inBitCount1} to \emph{inBitCount1 + inBitCount2 - 1} and  \emph{z} is defined by bits index \emph{inBitCount1 + inBitCount2} to \emph{inBitCount1 + inBitCount2 + inBitCount3 - 1}.






\subsectionGetter{swap120}{binaryset}

\begin{galgas3box}
getter swap120
  ?@uint inBitCount1
  ?@uint inBitCount2
  ?@uint inBitCount3
  -> @binaryset
\end{galgas3box}

Returns the transposed \emph{(y, z, x)} relation.

This getter considers that the receiver encodes an \emph{(x, y, z)} relation, where \emph{x} is defined by bits index \emph{0} to \emph{inBitCount1  - 1}, \emph{y} is defined by bits index \emph{inBitCount1} to \emph{inBitCount1 + inBitCount2 - 1} and  \emph{z} is defined by bits index \emph{inBitCount1 + inBitCount2} to \emph{inBitCount1 + inBitCount2 + inBitCount3 - 1}.






\subsectionGetter{swap201}{binaryset}

\begin{galgas3box}
getter swap201
  ?@uint inBitCount1
  ?@uint inBitCount2
  ?@uint inBitCount3
  -> @binaryset
\end{galgas3box}

Returns the transposed \emph{(z, x, y)} relation.

This getter considers that the receiver encodes an \emph{(x, y, z)} relation, where \emph{x} is defined by bits index \emph{0} to \emph{inBitCount1  - 1}, \emph{y} is defined by bits index \emph{inBitCount1} to \emph{inBitCount1 + inBitCount2 - 1} and  \emph{z} is defined by bits index \emph{inBitCount1 + inBitCount2} to \emph{inBitCount1 + inBitCount2 + inBitCount3 - 1}.






\subsectionGetter{swap210}{binaryset}

\begin{galgas3box}
getter swap210
  ?@uint inBitCount1
  ?@uint inBitCount2
  ?@uint inBitCount3
  -> @binaryset
\end{galgas3box}

Returns the transposed \emph{(z, y, x)} relation.

This getter considers that the receiver encodes an \emph{(x, y, z)} relation, where \emph{x} is defined by bits index \emph{0} to \emph{inBitCount1  - 1}, \emph{y} is defined by bits index \emph{inBitCount1} to \emph{inBitCount1 + inBitCount2 - 1} and  \emph{z} is defined by bits index \emph{inBitCount1 + inBitCount2} to \emph{inBitCount1 + inBitCount2 + inBitCount3 - 1}.








\subsectionGetter{transitiveClosure}{binaryset}

\begin{galgas3box}
getter transitiveClosure ?@uint inBitCount -> @binaryset
\end{galgas3box}


Returns the transitive closure of the relation encoded by the receiver.

This getter considers that the receiver encodes an \emph{(x, y)} relation, where \emph{x} is defined by bits index \emph{0} to \emph{inBitCount  - 1}, \emph{y} is defined by bits index \emph{inBitCount} to \emph{2 * inBitCount - 1}.






\subsectionGetter{transposedBy}{binaryset}

\begin{galgas3box}
getter transposedBy ?@uintlist inVector -> @binaryset
\end{galgas3box}

Retourne la valeur transposée du récepteur. L'argument \ggst+inVector+ spécifie comment la transposition s'opère : la valeur à l'indice $i$ est l'indice de destination du bit $i$ dans le \emph{binaryset} renvoyé.

{\bf 1\textsuperscript{er} exemple.} Si on veut échanger les bits $0$ et $1$, on écrit :
\begin{galgas3}
let vector = @uintlist {!1, !0}
let result = [myBinarySet transposedBy !vector]
\end{galgas3}

{\bf 2\textsuperscript{e} exemple.}
\begin{galgas3}
  let b = @binaryset.binarySetWithStrictGreaterComparison {!0 !2 !4}
    & @binaryset.binarySetWithEqualToConstant {!2 !2 !0}
  log b # <@binaryset: 000001, 00001X, 01001X, 100011>
  let vr = @uintlist {!0, !1, !4, !5, !2, !3}
  let r = [b transposedBy !vr]
  log r # <@binaryset: 000001, 00001X, 00011X, 001011>
  let vs = @uintlist {!4, !5, !0, !1, !2, !3}
  let s = [b transposedBy !vs]
  log s # <@binaryset: 010000, 100X00, 110X00, 111000>
\end{galgas3}

La constante \ggst=b= encode la relation $A > B$, où $A$ est encodé par les bits 0 et 1, et $B$ par les bits 4 et 5. Les bits 2 et 3 sont fixés à 0. Dans le résultat \ggst=r=, $A$ est encodé par les bits 0 et 1 (inchangés), $B$ par les bits 2 et 3, et maintenant les bits 4 et 5 sont fixés à 0. Dans le résultat \ggst=s=, $A$ est encodé par les bits 4 et 5, $B$ par les bits 2 et 3, et les bits 0 et 1 sont fixés à 0.












\subsectionGetter{uint64ValueList}{binaryset}

\begin{galgas3box}
getter uint64ValueList ?@uint inBitCount -> @uint64list
\end{galgas3box}


Returns the list of \ggst+@uint64+ values corresponding to receiver's value, considering it uses \emph{inBitCount} bits.








\subsectionGetter{valueCount}{binaryset}

\begin{galgas3box}
getter valueCount ?@uint inBitCount -> @uint64
\end{galgas3box}


Returns in an \ggst+@uint64+ object the number of different values encoded by receiver, considering it uses \emph{inBitCount} bits. No overflow test in performed.







%-------------------------------

\section{Logical Operators}

The \ggst+@binaryset+ type supports the three logical operators:\newline

\begin{tabular}{|c|c|}
\hline
\texttt{$\&$} & Logical And, intersection \\
\hline
\texttt{\textbar} & Logical Or, union \\
\hline
\texttt{$\wedge$}  & Exclusive or \\
\hline
\end{tabular}

Theses operators require both arguments to be \ggst+@binaryset+ objects and return an \ggst+@binaryset+ object.\newline


The \ggst+@binaryset+ type supports the logical unary operator:\newline

\begin{tabular}{|c|c|}
\hline
$\sim$ & Negation, Complementation \\
\hline
\end{tabular}

This operator Retourne un \ggst+@binaryset+ object.







\section{Comparison Operators}

The \ggst+@binaryset+ type supports the two comparison operators:\newline

\begin{tabular}{|c|c|}
\hline
$=$ & Equality \\
\hline
$!=$ & Non Equality \\
\hline
\end{tabular}

Theses operators require both arguments to be \ggst+@binaryset+ objects, and return a \ggst+@bool+ object. These operations are very fast and are performed in a constant time (integer equality comparison).

Do not confuse with \refGetterPage{binaryset}{equalTo} and \refGetterPage{binaryset}{notEqualTo} that return a \ggst+@binaryset+ object.







\section{Shift Operators}

The \ggst+@binaryset+ type supports the two shift operators:\newline

\begin{tabular}{|c|c|}
\hline
$<<$ & Left Shift \\
\hline
$>>$ & Right Shift \\
\hline
\end{tabular}

\textbf{Exemple :}
\begin{galgas3}
@binaryset b [binarySetWithPredicateString !"1010"]
log b # Affiche: <@binaryset: 1010>
@binaryset bb = b << 3
log bb # Affiche: <@binaryset: 1010XXX>
\end{galgas3}


  %!TEX encoding = UTF-8 Unicode
%!TEX root = ../galgas-book.tex

\chapitreTypePredefiniLabelIndex{bool}

\tableDesMatieresLocaleDeProfondeurRelative{1}


Le type \ggst+@bool+ est le type booléen. Les deux mots réservés \ggst+true+ et \ggst+false+ sont du type \ggst+@bool+ type, et dénote les valeurs \emph{vari} et \emph{faux}. Le seul constructeur du \ggst+@bool+ type est le constructeur \ggst!default!, qui initialise un booléen à \ggst+false+.


\section{Conversion en chaîne de caractères}

\subsectionGetter{cString}{bool}

\begin{galgas3box}
getter cString -> @string
\end{galgas3box}

Retourne la chaîne \ggst!"true"! si le booléen est vrai, et la chaîne \ggst!"false"! dans le cas contraire.







\subsectionGetter{ocString}{bool}

\begin{galgas3box}
getter ocString -> @string
\end{galgas3box}

Retourne la chaîne \ggst!"YES"! si le booléen est vrai, et la chaîne \ggst!"NO"! dans le cas contraire.




\section{Conversion en entier}


\subsectionGetter{bigint}{bool}

\begin{galgas3box}
getter bigint -> @bigint
\end{galgas3box}

Retourne l'entier \ggst!1G! si le booléen est vrai, et l'entier \ggst!0G! dans le cas contraire.

\begin{galgas3}
  message [[false bigint] string] + "\n" # 0
  message [[true bigint] string] + "\n" # 1
\end{galgas3}


\subsectionGetter{sint}{bool}

\begin{galgas3box}
getter sint -> @sint
\end{galgas3box}

Retourne l'entier \ggst!1S! si le booléen est vrai, et l'entier \ggst!0S! dans le cas contraire.




\subsectionGetter{sint64}{bool}

\begin{galgas3box}
getter sint64 -> @sint64
\end{galgas3box}

Retourne l'entier \ggst!1LS! si le booléen est vrai, et l'entier \ggst!0LS! dans le cas contraire.




\subsectionGetter{uint}{bool}

\begin{galgas3box}
getter uint -> @uint
\end{galgas3box}

Retourne l'entier \ggst!1! si le booléen est vrai, et l'entier \ggst!0! dans le cas contraire.




\subsectionGetter{uint64}{bool}

\begin{galgas3box}
getter uint64 -> @uint64
\end{galgas3box}

Retourne l'entier \ggst!1L! si le booléen est vrai, et l'entier \ggst!0L! dans le cas contraire.




\section{Opérateurs logiques}

\begin{galgas3box}
operator @bool & @bool -> @bool
operator @bool | @bool -> @bool
operator @bool ^ @bool -> @bool
operator not @bool -> @bool
\end{galgas3box}

Le type \ggst+@bool+ accepte les trois opérateurs suivants
\begin{itemize}
\item l'opérateur \ggst!&! infixé qui effectue un \emph{et logique} ;
\item l'opérateur \ggst!|! infixé qui effectue un \emph{ou logique} ;
\item l'opérateur \ggst!^! infixé qui effectue un \emph{ou exclusif logique} ;
\item l'opérateur \ggst!not! infixe qui effectue la\emph{négation logique}.
\end{itemize}








\section{Comparaison}

Le type \ggst!@bool! implémente les six opérateurs de comparaison \ggst!==!, \ggst+!=+, \ggst!<!, \ggst!<=!, \ggst!>! et \ggst!>=!, avec \ggst!false < true!.

  %!TEX encoding = UTF-8 Unicode
%!TEX root = ../galgas-book.tex

\chapitreTypePredefiniLabelIndex{char}

An \galgas{@char} object value is an Unicode character. You can initialize an \galgas{@char} object from a character constant:

\begin{galgascode}
@char myCharacter = 'A' ;
\end{galgascode}


You have several ways for writing a literal character constant. In any case, it should define an assigned Unicode character. A compile-time error is raised if it does not.


A literal character constant is a single character or an escape sequence enclosed by single quotes (\galgas{'}).

For an ASCII printable character:

\begin{galgascode}
@char myCharacter = 'a' ;
\end{galgascode}


If you want to get ASCII source text file, any character that does not correspond to an ASCII printable character should be expressed with an escape sequence.

Otherwise, for any printable Unicode character, you can write it directly, without escape sequence, provided your text file encoding supports this character:\\

\texttt{@char myCharacter = \textquotesingle\ae\textquotesingle ;}\\

The following escape sequences are defined (they begin with a '\textquotesingle').

\begin{tabular}{|c|c|}
\hline
Character Constant & Meaning \\
\hline
\texttt{\textquotesingle\textbackslash f\textquotesingle} & A Form Feed Character \\
\hline
\texttt{\textquotesingle\textbackslash n\textquotesingle} & A New Line Character \\
\hline
\texttt{\textquotesingle\textbackslash r\textquotesingle} & A Carriage Return Character \\
\hline
\texttt{\textquotesingle\textbackslash v\textquotesingle} & A Vertical Tabulation Character \\
\hline
\texttt{\textquotesingle\textbackslash\textbackslash\textquotesingle} & A back slash Character \\
\hline
\texttt{\textquotesingle\textbackslash 0\textquotesingle} & A Nul Character \\
\hline
\texttt{\textquotesingle\textbackslash\textquotesingle\textquotesingle} & A Single Quote Character \\
\hline
\end{tabular}


\begin{tabular}{|c|c|}
\hline
Character Constant & Meaning \\
\hline
\texttt{\textquotesingle\textbackslash uABCD\textquotesingle} & An Unicode Character \\
\hline
\end{tabular}

Where \emph{ABCD} is a four digit hexadecimal number that represents an assigned Unicode point code. For example:

\texttt{@char myCharacter = \textquotesingle\textbackslash u03A0\textquotesingle ; \# The '$\Sigma$' character}\\

Note: an unassigned point code raises a compile-time error:\\
\texttt{@char myCharacter = \textquotesingle\textbackslash uFFFF\textquotesingle ; \# The \textbackslash uFFFF point code is not assigned}\\


\begin{tabular}{|c|c|}
\hline
Character Constant & Meaning \\
\hline
\texttt{\textquotesingle\textbackslash Uabcdxyzt\textquotesingle} & An Unicode Character \\
\hline
\end{tabular}

Where \emph{abcdxyzt} is a eight digit hexadecimal number that represents an assigned Unicode point code. For example:

\texttt{@char myCharacter = \textquotesingle\textbackslash U0010170\textquotesingle ; \# The 'GREEK ACROPHONIC NAXIAN FIVE HUNDRED' character}\\

Note: an unassigned point code raises a compile-time error:\\

\texttt{@char myCharacter = \textquotesingle\textbackslash U0000FFFF\textquotesingle ; \# Raises a compile-time error: \textbackslash U0000FFFF is not assigned.}\\

Any point code beyond \textbackslash U0010FFFF is invalid and not assigned.




\section{Constructors}



\constructeurSansArgument{replacementCharacter}
{char}
{1.8.3}
{char}
{Returns an \galgas{@char} object corresponding to Unicode replacement character (\texttt{\textquotesingle\textbackslash uFFFD}.}
{}







\constructeurUnArgument{unicodeCharacterWithUnsigned}
{char}
{1.8.3}
{char}
{@uint inValue}
{Returns an \galgas{@char} object from an Unicode code point.}
{A run-time error is raised if the \emph{inValue} value does not represent an assigned Unicode value. You can check if an \galgas{@uint} value represents an assigned Unicode value with the \refGetterPage{uint}{isUnicodeValueAssigned}.}


\section{Getters}


\subsectionGetter{isalnum}{char}

\begin{galgascode}
getter isalnum -> @bool
\end{galgascode}

Returns an \galgas{@bool} value indicating whether the receiver'value represents an ASCII letter or an ASCII digit: \galgas{true} if the receiver'value represents an ASCII letter or an ASCII digit (between \ggs+'A'+ and \ggs+'Z'+, or between \ggs'a'+ and \ggs+'z'+, or between \ggs+'0'+ and \ggs+'9'+, and \galgas{false} otherwise.




\subsectionGetter{isalpha}{char}

\begin{galgascode}
getter isalpha -> @bool
\end{galgascode}

Returns an \galgas{@bool} value indicating whether the receiver'value represents an ASCII letter: \galgas{true} if the receiver'value represents an ASCII letter (between \ggs+'A'+ and \ggs+'Z'+, or between \ggs+'a'+ and \ggs+'z'+), and \galgas{false} otherwise.




\subsectionGetter{iscntrl}{char}

\begin{galgascode}
getter iscntrl -> @bool
\end{galgascode}

Returns an \galgas{@bool} value indicating whether the receiver'value represents an ASCII control character: \galgas{true} if the receiver'value represents an ASCII control character (strictly before the \emph{SPACE} character), and \galgas{false} otherwise.





\subsectionGetter{isdigit}{char}

\begin{galgascode}
getter isdigit -> @bool
\end{galgascode}

Returns an \galgas{@bool} value indicating whether the receiver'value represents an ASCII digit: \galgas{true} if the receiver'value represents an ASCII digit (between \ggs+'0'+ and \ggs+'9'+), and \galgas{false} otherwise.





\subsectionGetter{islower}{char}

\begin{galgascode}
getter islower -> @bool
\end{galgascode}

Returns an \galgas{@bool} value indicating whether the receiver'value represents an ASCII lowercase ASCII letter: \galgas{true} if the receiver'value represents an ASCII lowercase letter (between \ggs+'a'+ and \ggs+'z'+), and \galgas{false} otherwise.






\subsectionGetter{isUnicodeCommand}{char}

\begin{galgascode}
getter isUnicodeCommand -> @bool
\end{galgascode}

Returns an \galgas{@bool} value indicating whether the receiver'value represents an Unicode command: \galgas{true} if the receiver'value represents an Unicode command, and \galgas{false} otherwise.






\subsectionGetter{isUnicodeLetter}{char}

\begin{galgascode}
getter isUnicodeLetter -> @bool
\end{galgascode}

Returns an \galgas{@bool} value indicating whether the receiver'value represents an Unicode letter: \galgas{true} if the receiver'value represents an Unicode letter, and \galgas{false} otherwise.






\subsectionGetter{isUnicodeMark}{char}

\begin{galgascode}
getter isUnicodeMark -> @bool
\end{galgascode}

Returns an \galgas{@bool} value indicating whether the receiver'value represents an Unicode mark character: \galgas{true} if the receiver'value represents an Unicode mark character, and \galgas{false} otherwise.






\subsectionGetter{isUnicodePunctuation}{char}

\begin{galgascode}
getter isUnicodePunctuation -> @bool
\end{galgascode}

Returns an \galgas{@bool} value indicating whether the receiver'value represents an Unicode punctuation character: \galgas{true} if the receiver'value represents an Unicode punctuation character, and \galgas{false} otherwise.






\subsectionGetter{isUnicodeSeparator}{char}

\begin{galgascode}
getter isUnicodeSeparator -> @bool
\end{galgascode}

Returns an \galgas{@bool} value indicating whether the receiver'value represents an Unicode separator character: \galgas{true} if the receiver'value represents an Unicode separator character, and \galgas{false} otherwise.






\subsectionGetter{isUnicodeSymbol}{char}

\begin{galgascode}
getter isUnicodeSymbol -> @bool
\end{galgascode}

Returns an \galgas{@bool} value indicating whether the receiver'value represents an Unicode symbol character: \galgas{true} if the receiver'value represents an Unicode symbol character, and \galgas{false} otherwise.









\subsectionGetter{isupper}{char}

\begin{galgascode}
getter isupper -> @bool
\end{galgascode}

Returns an \galgas{@bool} value indicating whether the receiver'value represents an ASCII uppercase ASCII letter: \galgas{true} if the receiver'value represents an ASCII uppercase letter (between \ggs+'A'+ and \ggs+'Z'+, and \galgas{false} otherwise.





\subsectionGetter{string}{char}

\begin{galgascode}
getter string -> @string
\end{galgascode}

Returns returns a string representation of the receiver's value: a one character \galgas{@string} object, containing the receiver's value.




\subsectionGetter{uint}{char}

\begin{galgascode}
getter uint -> @uint
\end{galgascode}

Returns an \galgas{@uint} object representing the Unicode code point of the receiver's value.




\subsectionGetter{unicodeName}{char}

\begin{galgascode}
getter unicodeName -> @string
\end{galgascode}

Returns the unicode name of the receiver's value: for an decimal string representation of the receiver's value, see the \refGetterPage{uint}{hexString}; for a decimal string representation of the receiver's value, see the \refGetterPage{uint}{string}.

\textbf{Exemple :}
\begin{galgascode}
['\AE' unicodeName] # returns "LATIN CAPITAL LETTER AE"
\end{galgascode}




\subsectionGetter{unicodeToLower}{char}

\begin{galgascode}
getter unicodeToLower -> @char
\end{galgascode}

Returns the lowercase character corresponding to the receiver's value: if the receiver's value is an Unicode uppercase character, this getter returns the corresponding lowercase character. Otherwise, it returns the receiver's value.

\textbf{Exemple :}
\begin{galgascode}
['Æ' unicodeToLower] # returns 'æ'
['æ' unicodeToLower] # returns 'æ'
\end{galgascode}




\subsectionGetter{unicodeToUpper}{char}

\begin{galgascode}
getter unicodeToUpper -> @char
\end{galgascode}

Returns the uppercase character corresponding to the receiver's value: if the receiver's value is an Unicode lowercase character, this getter returns the corresponding uppercase character. Otherwise, it returns the receiver's value.

\textbf{Exemple :}
\begin{galgascode}
['Æ' unicodeToUpper] # returns 'Æ'
['æ' unicodeToUpper] # returns 'Æ'
\end{galgascode}





\section{Comparison Operators}

The \galgas{@char} type supports the six comparison operators:\newline

\begin{tabular}{|c|c|}
\hline
$=$ & Equality \\
\hline
$!=$ & Non Equality \\
\hline
$<$  & Strict Lower Than \\
\hline
$<=$  & Lower or Equal \\
\hline
$>$  & Strict Greater Than \\
\hline
$>=$  & Greater or Equal \\
\hline
\end{tabular}

Theses operators require both arguments to be \galgas{@char} objects, and return a \ggs+@bool+ object. Comparison is done by comparing of the Unicode code point's value.



  %!TEX encoding = UTF-8 Unicode
%!TEX root = ../galgas-book.tex

\chapitreTypePredefiniLabelIndex{double}

The \ggs+@double+ object values correspond to the C type \ggs+@double+ values. You can initialize an \ggs+@double+ object from a float constant:

\begin{galgascode}
@double myDouble = 123.456
\end{galgascode}

Note that a \ggs+@double+ constant is characterized by the occurrence of the decimal point (.)

\section{Constructor}

\subsectionConstructor{doubleWithBinaryImage}{double}

\begin{galgascode}
constructor doubleWithBinaryImage ?@uint inValue -> @double
\end{galgascode}


Returns a double object from the binary image of the argument.



\subsectionConstructor{pi}{double}

\begin{galgascode}
constructor pi -> @double
\end{galgascode}



Returns an approximation of the $\pi$ constant value (\ggs+3.14159265358979323846264338327950288+).

\section{Getters}

\subsectionGetter{binaryImage}{double}

\begin{galgascode}
getter binaryImage -> @uint64
\end{galgascode}

Returns the binary image of the value of receiver's value.




\subsectionGetter{cos}{double}

\begin{galgascode}
getter cos -> @double
\end{galgascode}

Returns the \emph{cosine} value of receiver's value, expressed in radian.




\subsectionGetter{sin}{double}

\begin{galgascode}
getter sint -> @double
\end{galgascode}

Returns the \emph{sine} value of receiver's value, expressed in radian.




\subsectionGetter{sint}{double}

\begin{galgascode}
getter sint -> @sint
\end{galgascode}

Returns the receiver's value in an \refTypePredefini{sint} (32-bit signed integer) object: if receiver's value is outside \ggs+@sint+ bounds, a runtime error is raised.



\subsectionGetter{sint64}{double}

\begin{galgascode}
getter sint64 -> @sint64
\end{galgascode}

Returns the receiver's value in an \refTypePredefini{sint64} (64-bit signed integer) object: if receiver's value is outside \ggs+@sint64+ bounds, a runtime error is raised.




\subsectionGetter{string}{double}

\begin{galgascode}
getter string -> @string
\end{galgascode}

Returns a decimal string representation of the receiver's value (this getter never fails).




\subsectionGetter{tan}{double}

\begin{galgascode}
getter tan -> @double
\end{galgascode}

Returns the \emph{tangent} value of receiver's value, expressed in radian.







\subsectionGetter{uint}{double}

\begin{galgascode}
getter uint -> @uint
\end{galgascode}

Returns the receiver's value in an \refTypePredefini{uint} (32-bit unsigned integer) object: if receiver's value is outside \ggs+@uint+ bounds, a runtime error is raised.





\subsectionGetter{uint64}{double}

\begin{galgascode}
getter uint64 -> @uint64
\end{galgascode}

Returns the receiver's value in an \refTypePredefini{uint64} (64-bit unsigned integer) object: if receiver's value is outside \ggs+@uint64+ bounds, a runtime error is raised.




\section{Arithmetic Operators}

The \ggs+@double+ type supports the five arithmetic diadic operators:\newline

\begin{tabular}{|c|c|}
\hline
$+$ & Addition \\
\hline
$-$ & Substraction \\
\hline
$*$ & Multiplication \\
\hline
$/$ & Division \\
\hline
\ggs+mod+ & Modulo \\
\hline
\end{tabular}

Theses operators require both arguments to be \ggs+@double+ objects.\newline

A run-time error is raised if the operation leads to an overflow.

The \ggs+@double+ type supports the following arithmetic unary operators:\newline

\begin{tabular}{|c|c|}
\hline
$+$ & No operation \\
$-$ & Negate \\
\hline
\end{tabular}

This operator returns the receiver's value (an \ggs+@double+ object).






\section{Comparison Operators}

The \ggs+@double+ type supports the six comparison operators:\newline

\begin{tabular}{|c|c|}
\hline
$=$ & Equality \\
\hline
$!=$ & Non Equality \\
\hline
$<$  & Strict Lower Than \\
\hline
$<=$  & Lower or Equal \\
\hline
$>$  & Strict Greater Than \\
\hline
$>=$  & Greater or Equal \\
\hline
\end{tabular}

Theses operators require both arguments to be \ggs+@double+ objects, and return a \ggs+@bool+ object.



  %!TEX encoding = UTF-8 Unicode
%!TEX root = ../galgas-book.tex

\chapitreTypePredefiniLabelIndex{location}

An \galgas{@location} object value is a location in a source file. Objects of this type are useful for pointing out an error or a warning location.

\section{The \texttt{here} Keyword}

The \galgas{here} keyword indicates the current parsing location is the current source file. Assigning an \galgas{@location} object from the \galgas{here} keyword is a way for initializing an \galgas{@location} object:\newline

\texttt{@location currentLocation := here ;}

\section{Constructor}

\constructeurSansArgument{nowhere}
{location}
{2.1.2}
{location}
{Returns an \galgas{@location} that does not points out any location.}
{The returned object responds \galgas{true} to the \refReaderPage{location}{isNowhere}.}

\section{Readers}

\readerSansArgument{column}
{location}
{1.8.2}
{uint}
{Returns an \galgas{@uint} value containing the column of the receiver's value.}
{this reader raises a run-time error if the receiver's value responds \galgas{true} to the \refReaderPage{location}{isNowhere}.}


\readerSansArgument{isNowhere}
{location}
{2.1.2}
{bool}
{Returns an \galgas{@bool} value indicating whether the receiver'value points out a source location or does not.}
{this reader returns \galgas{true} if the receiver's value does not point out an actual location in a text source (i.e. it has been constructed using the nowhere constructor), and \galgas{false} if the receiver's value points out an actual location in a text source (i.e. it has been constructed using the \galgas{here} keyword.}


\readerSansArgument{line}
{location}
{1.8.2}
{uint}
{Returns an \galgas{@uint} value containing the line of the receiver's value.}
{this reader raises a run-time error if the receiver's value responds \galgas{true} to the \refReaderPage{location}{isNowhere}.}


\readerSansArgument{locationIndex}
{location}
{1.8.2}
{uint}
{Returns an \galgas{@uint} value containing the the offset from the the beginning of the source of the location defined by receiver's value.}
{this reader raises a run-time error if the receiver's value responds \galgas{true} to the \refReaderPage{location}{isNowhere}.}


\readerSansArgument{locationString}
{location}
{1.8.2}
{string}
{returns an \galgas{@string} object that contains a string representation of the location defined by receiver's value.}
{this reader raises a run-time error if the receiver's value responds \galgas{true} to the \refReaderPage{location}{isNowhere}.}

  %!TEX encoding = UTF-8 Unicode
%!TEX root = ../galgas-book.tex

\sectionTypePredefiniLabelIndex{sint}

An \nomType{sint} object value is a 32-bit signed integer value. You can initialize an \nomType{sint} object from an 32-bit signed integer constant:\\

\texttt{@sint mySignedInteger := 123\_456S ;}

Note that a 32-bit signed integer constant is characterized by the 'S' suffix.




\constructeurSansArgument{min}
{@sint}
{1.3.0}
{@sint}
{Returns an \nomType{sint} object that the minimum value of the 32-bit signed range.}
{the returned value is $-2^{31}$.}





\constructeurSansArgument{max}
{@sint}
{1.3.0}
{@sint}
{Returns an \nomType{sint} object that the maximum value of the 32-bit signed range.}
{the returned value is $2^{31}-1$.}





\readerSansArgument{double}
{@sint}
{1.9.8}
{@double}
{Returns the receiver's value converted in a \nomType{double} object.}
{as a 32-bit integer value can always be converted in a \nomType{double} value, this reader never fails.}





\readerSansArgument{sint64}
{@sint}
{1.6.12}
{@sint64}
{Returns the receiver's value in an \refTypePredefini{sint64} (64-bit signed integer) object.}
{as a 32-bit signed value can always be converted in a 64-bit signed value, this reader never fails.}

This reader is the only way to convert an \refTypePredefini{sint} object into an \refTypePredefini{sint64} object.





\readerSansArgument{string}
{@sint}
{1.6.12}
{@string}
{Returns a decimal string representation of the receiver's value.}
{for an hexadecimal string representation of the receiver's value, see \refReaderPage{uint}{hexString} and \refReaderPage{uint}{xString}.}







\readerSansArgument{uint}
{@sint}
{1.3.0}
{@uint}
{Returns the receiver's value in an \refTypePredefini{uint} (32-bit unsigned integer) object.}
{an error is raised is receiver's value is negative.}

This reader is the only way to convert an \refTypePredefini{sint} object into an \refTypePredefini{uint} object.




\readerSansArgument{uint64}
{@sint}
{1.3.0}
{@uint64}
{Returns the receiver's value in an \refTypePredefini{uint64} (64-bit unsigned integer) object.}
{an error is raised is receiver's value is negative.}

This reader is the only way to convert an \refTypePredefini{sint} object into an \refTypePredefini{uint64} object.





\subsection{Incrementation and decrementation}

The \refTypePredefini{sint} supports incrementation and decrementation instructions.

\texttt{@sint n := ... ; n ++ ; \# Incrementation}

\texttt{@sint p := ... ; p -- ; \# Decrementation}\newline

The incrementation instruction raises an error if receiver's value is equal to $2^{31}-1$.\newline

The decrementation instruction raises an error if receiver's value is equal to $-2^{31}$.\newline

Note that incrementation and decrementation are not available within an expression.




\subsection{Arithmetic Operators}

The \nomType{sint} type supports the five arithmetic diadic operators:\newline

\begin{tabular}{|c|c|}
\hline
$+$ & Addition \\
\hline
$-$ & Substraction \\
\hline
$*$ & Multiplication \\
\hline
$/$ & Division \\
\hline
$\%$ & Modulo \\
\hline
\end{tabular}

Theses operators require both arguments to be \nomType{sint} objects.\newline

A run-time error is raised if the operation leads to a 32-bit signed overflow.

The \nomType{sint} type supports the following arithmetic unary operators:\newline

\begin{tabular}{|c|c|}
\hline
$+$ & No operation \\
\hline
$-$ & Negate \\
\hline
\end{tabular}

This operator returns the receiver's value (an \nomType{sint} object). A run-time error is raised if "-" operator is invoked on an object whose value is $-2^{31}$.






\subsection{Shift Operators}


The \nomType{sint} type supports right and left shift operators:\newline

\begin{tabular}{|c|c|}
\hline
$<<$ & Left shift \\
\hline
$>>$ & Right shift \\
\hline
\end{tabular}

Theses operators require the right argument to be \nomType{sint} object, and the left argument to be \nomType{uint} object.\newline

Note the right shift inserts a zero bit in the most significant bit location if the receiver's value is negative, and a one bit otherwise (it is a arithmetic right shift).\newline

The actual amount of the shift is the value of the right-hand operand masked by 31, i.e. the shift distance is always between 0 and 31.




\subsection{Logical Operators}

The \nomType{sint} type supports the three bit-wise logical operators:\newline

\begin{tabular}{|c|c|}
\hline
$\&$ & Bit-wise and \\
\hline
\textbar & Bit-wise or \\
\hline
\^\  & Bit-wise exclusive or \\
\hline
\end{tabular}

Theses operators require both arguments to be \nomType{sint} objects.\newline


The \nomType{sint} type supports the bit-wise logical unary operator:\newline

\begin{tabular}{|c|c|}
\hline
$\sim$ & Bit-wise complementation \\
\hline
\end{tabular}

This operator returns an \nomType{sint} object.







\subsection{Comparison Operators}

The \nomType{sint} type supports the six comparison operators:\newline

\begin{tabular}{|c|c|}
\hline
$=$ & Equality \\
\hline
$!=$ & Non Equality \\
\hline
$<$  & Strict Lower Than \\
\hline
$<=$  & Lower or Equal \\
\hline
$>$  & Strict Greater Than \\
\hline
$>=$  & Greater or Equal \\
\hline
\end{tabular}

Theses operators require both arguments to be \nomType{sint} objects, and return a \nomType{bool} object.



  %!TEX encoding = UTF-8 Unicode
%!TEX root = ../galgas-book.tex

\chapitreTypePredefiniLabelIndex{sint64}

\tableDesMatieresLocaleDeProfondeurRelative{1}


An \ggst+@sint64+ object value is a 64-bit signed integer value. You can initialize an \ggst+@sint64+ object from an 64-bit signed integer constant:\\

\texttt{@sint64 mySignedInteger = 123\_456LS ;}

Note that a 64-bit signed integer constant is characterized by the 'LS' suffix.

\section{Constructors}


\subsectionConstructor{min}{sint64}

\begin{galgas3}
constructor min -> @sint64
\end{galgas3}

Returns an \ggst+@sint64+ object that the minimum value of the 64-bit signed range ($-2^{63}$).





\subsectionConstructor{max}{sint64}

\begin{galgas3}
constructor max -> @sint64
\end{galgas3}

Returns an \ggst+@sint64+ object that the maximum value of the 64-bit signed range ($2^{63}-1$).


\section{Getters}


\subsectionGetter{bigint}{sint64}

Ce \emph{getter} permet de convertir un \ggst!@sint64! en \ggst!@bigint!. Comme la plage des valeurs des \ggst!bigint! n'est limitée que par la mémoire disponible, il n'échoue jamais.

\begin{galgas3}
  message [[-1234LS bigint] string] + "\n" # -1234
\end{galgas3}



\subsectionGetter{double}{sint64}

\begin{galgas3}
getter double -> @double
\end{galgas3}

Returns the receiver's value converted in a \ggst+@double+ object. As a 64-bit integer value can always be converted in a \ggst+@double+ value, this getter never fails.



\subsectionGetter{hexStringSeparatedBy}{sint64}

\begin{galgas3}
getter hexStringSeparatedBy ?@char inSeparator ?@uint inGroup -> @string
\end{galgas3}

Returns the an hexadecimal string representation of the receiver value, prefixed by the string \texttt{0x}. Groups of \ggst=inGroup= digits are separated by the \ggst=inSeparator= character.

If \ggst=inGroup= is equal to zero, a run-time error is raised.

For example:
\begin{galgas3}
let s = [0x123456789ABCDEF0LS hexStringSeparatedBy !'_' !3] # 0x1_234_567_89A_BCD_EF0
\end{galgas3}




\subsectionGetter{sint}{sint64}

\begin{galgas3}
getter sint -> @sint
\end{galgas3}

Returns the receiver's value in an \refTypePredefini{sint} (32-bit signed integer) object. An error is raised is receiver's value is lower than $-2^{31}$ or greater than $2^{31}-1$.

This getter is the only way to convert an \refTypePredefini{sint64} object into an \refTypePredefini{sint} object.





\subsectionGetter{string}{sint64}

\begin{galgas3}
getter string -> @string
\end{galgas3}

Returns a decimal string representation of the receiver's value. This getter never fails.








\subsectionGetter{uint}{sint64}

\begin{galgas3}
getter uint -> @uint
\end{galgas3}

Returns the receiver's value in an \refTypePredefini{uint} (32-bit unsigned integer) object. An error is raised is receiver's value is negative or greater than $2^{32}-1$.

This getter is the only way to convert an \refTypePredefini{sint64} object into an \refTypePredefini{uint} object.





\subsectionGetter{uint64}{sint64}

\begin{galgas3}
getter uint64 -> @uint64
\end{galgas3}

Returns the receiver's value in an \refTypePredefini{uint64} (64-bit unsigned integer) object. This getter raises a run-time error if the receiver's value is negative.

This getter is the only way to convert an \refTypePredefini{sint64} object into an \refTypePredefini{uint64} object.








\section{Arithmétique}

\subsection{Opérateurs infixés}

Le type \ggst+@sint64+ accepte les opérateurs arithmétiques infixés suivants :
\begin{itemize}
  \item \ggst!+!, addition, une erreur d'exécution est déclenchée en cas de débordement ;
  \item \ggst!-!, soustraction, une erreur d'exécution est déclenchée en cas de débordement ;
  \item \ggst!*!, multiplication, une erreur d'exécution est déclenchée en cas de débordement ;
  \item \ggst!/!, division, une erreur d'exécution est déclenchée si le diviseur est nul ;
  \item \ggst!mod!, calcul du reste, une erreur d'exécution est déclenchée si le diviseur est nul ;
  \item \ggst!&+!, addition, le résultat étant silencieusement tronqué sur 64 bits ;
  \item \ggst!&-!, soustraction, le résultat étant silencieusement tronqué sur 64 bits ;
  \item \ggst!&*!, multiplication, le résultat étant silencieusement tronqué sur 64 bits ;
  \item \ggst!&/!, division, qui retourne zéro si le diviseur est nul.
\end{itemize}

Ces opérateurs exigent que les deux opérandes soient des objets du même type \ggst+@sint64+.

\subsection{Opérateurs préfixés}
Le type \ggst+@sint64+ accepte les opérateurs arithmétiques préfixés suivants :
\begin{itemize}
  \item \ggst!+!, qui retourne simplement la valeur de l'opérande ;
  \item \ggst!-!, négation arithmétique, une erreur d'exécution est déclenchée si l'opérande est égal à $-2^{63}$ ;
  \item \ggst!&-!, négation arithmétique, sans détection de débordement : la négation de $-2^{63}$ est $-2^{63}$.
\end{itemize}

La valeur renvoyée est du même type  \ggst+@sint64+.


\subsectionLabel{Instructions}{instructionsSINT64}

Le type \ggst+@sint64+ accepte les instructions arithmétiques suivantes :
\begin{itemize}
  \item \ggst!+=!, addition, une erreur d'exécution est déclenchée en cas de débordement ;
  \item \ggst!-=!, soustraction, une erreur d'exécution est déclenchée en cas de débordement ;
  \item \ggst!*=!, multiplication, une erreur d'exécution est déclenchée en cas de débordement ;
  \item \ggst!/=!, division, une erreur d'exécution est déclenchée en cas division par zéro ;
  \item \ggst!++!, incrémentation, une erreur d'exécution est déclenchée en cas de débordement ;
  \item \ggst!--!, décrémentation, une erreur d'exécution est déclenchée en cas de débordement ;
  \item \ggst!&++!, incrémentation, le résultat étant silencieusement tronqué sur 64 bits ;
  \item \ggst!&--!, décrémentation, le résultat étant silencieusement tronqué sur 64 bits.
\end{itemize}

\ggst!x+=y! est équivalent à \ggst!x=x+y! ; \ggst!x-=y! est équivalent à \ggst!x=x-y!.
La variable cible \ggst!x!, comme l'expression source \ggst!y! doivent être du même type \ggst+@sint64+.

Incrémentation et décrémentation sont des instructions, et ne peuvent pas apparaître des expressions.
\begin{galgas3}
@sint64 n = ... ; n ++ # Incrémentation
\end{galgas3}

\begin{galgas3}
@sint64 n = ... ; n -- # Décrémentation
\end{galgas3}







\section{Shift Operators}


The \ggst+@sint64+ type supports right and left shift operators:\newline

\begin{tabular}{|c|c|}
\hline
$<<$ & Left shift \\
\hline
$>>$ & Right shift \\
\hline
\end{tabular}

Theses operators require the right argument to be \ggst+@sint64+ object, and the left argument to be \ggst+@uint+ object.\newline

Note the right shift inserts a zero bit in the most significant bit location if the receiver's value is negative, and a one bit otherwise (it is a arithmetic right shift).\newline

The actual amount of the shift is the value of the right-hand operand masked by 63, i.e. the shift distance is always between 0 and 63.




\section{Logical Operators}

The \ggst+@sint64+ type supports the three bit-wise logical operators:\newline

\begin{tabular}{|c|c|}
\hline
$\&$ & Bit-wise and \\
\hline
\textbar & Bit-wise or \\
\hline
\^\  & Bit-wise exclusive or \\
\hline
\end{tabular}

Theses operators require both arguments to be \ggst+@sint64+ objects.\newline


The \ggst+@sint64+ type supports the bit-wise logical unary operator:\newline

\begin{tabular}{|c|c|}
\hline
$\sim$ & Bit-wise complementation \\
\hline
\end{tabular}

This operator returns an \ggst+@sint64+ object.







\section{Comparison Operators}

The \ggst+@sint64+ type supports the six comparison operators:\newline

\begin{tabular}{|c|c|}
\hline
$=$ & Equality \\
\hline
$!=$ & Non Equality \\
\hline
$<$  & Strict Lower Than \\
\hline
$<=$  & Lower or Equal \\
\hline
$>$  & Strict Greater Than \\
\hline
$>=$  & Greater or Equal \\
\hline
\end{tabular}

Theses operators require both arguments to be \ggst+@sint64+ objects, and return a \ggst+@bool+ object.



  %!TEX encoding = UTF-8 Unicode
%!TEX root = ../galgas-book.tex

\chapitreTypePredefiniLabelIndex{string}

A \galgas{@string} object value is an Unicode character string value. The @string type defines several constructors, getters constant methods and setters, described below.

\paragraph{Literal String Constants.}

Characters strings are written enclosed within quotation marks (") characters, as in many languages. For example: "a string". Note that a literal string constant is an actual @string object, so a getter can be used on it. For example: \lstinline[language=galgas]{["ae" uppercaseString]} returns the "AE" string.

\section{Getters}

\readerUnArgument{containsCharacter}
{string}
{2.5.0}
{bool}
{@char inCharacter}
{Returns true if the receiver contains the given charactezr, and false oteherwise.}
{}

\begin{galgascode}
@string s := "abcdef";
@string s2 := [s rightSubString!3]; # The value of s2 is "def"
\end{galgascode}

\readerDeuxArguments{subString}
{string}
{1.7.8}
{string}
{@uint inStart}
{@uint inLength}
{Creates and returns the string built with the \emph{inLength} last characters of the receiver. If the receiver contains less than inLength characters, the receiver’s value is returned.}
{}


%\constructeurSansArgument{emptySet}
%{@stringset}
%{1.3.0}
%{@stringset}
%{Creates and returns an empty \galgas{@stringset} object.}
%{}
%
%\constructeurUnArgument{setWithString}
%{@stringset}
%{1.3.0}
%{@stringset}
%{@string inString}
%{Creates and returns an \galgas{@stringset} object that contains the value of the \emph{inString} argument object.}
%{}
%
%
%\readerSansArgument{count}
%{@stringset}
%{1.3.0}
%{@uint}
%{Returns the number of strings in the set.}
%{}
%
%
%
%\readerUnArgument{hasKey}
%{@stringset}
%{1.3.0}
%{@bool}
%{@string inString}
%{Returns a boolean value that indicates whether the value of \emph{inString} argument is present in the set.}
%{returns \motCle{true} if the value of \emph{inString} argument is present in the set, \motCle{false} otherwise.}
%
%
%
%
%\modifierUnArgument{removeKey}
%{@stringset}
%{1.3.0}
%{@string inString}
%{Removes the value of \emph{inString} argument from the receiver's value.}
%{if the receiver's value does not contain the value of \emph{inString} argument, this setter leaves the receiver's value unchanged.}
%
%
%
%
%
%
%\subsection{the \emph{+=} Operator}
%
%The \emph{+=} operator adds a string value to the receiver. If the receiver's value already contains the added value, this operator has no effect.
%
%\exempleTroisLignes
%{}
%{@string aString := ... ;}
%{@stringset aStringSet := ... ;}
%{aStringSet += !aString ;}
%
%
%
%
%\subsection{the \emph{$\&$} Operator}
%
%The \emph{$\&$} operator returns the intersection of its operand values.
%
%\exempleTroisLignes
%{}
%{@stringset s1 := ... ;}
%{@stringset s2 := ... ;}
%{@stringset s := s1 \& s2 ; \# s is the intersection of s1 and s2}
%
%
%
%
%
%
%\subsection{the \emph{$\textbar$} Operator}
%
%The \emph{$\textbar$} operator returns the union of its operand values.
%
%\exempleTroisLignes
%{}
%{@stringset s1 := ... ;}
%{@stringset s2 := ... ;}
%{@stringset s := s1 \textbar s2 ; \# s is the union of s1 and s2}
%
%
%
%
%
%
%\subsection{the \emph{$-$} Operator}
%
%The \emph{$-$} operator returns the difference of its operand values.
%
%\exempleTroisLignes
%{}
%{@stringset s1 := ... ;}
%{@stringset s2 := ... ;}
%{@stringset s := s1 - s2 ; \# s is the difference of s1 and s2}
%
%
%
%
%
%
%
%
%\subsection{Enumerating \galgas{@stringset} objects}
%
%
%The \motCle{foreach} instruction can be used for enumerating \galgas{@stringset} values; enumeration is performed in the ascending order, or in the reverse alphabetical order using the '>' qualifier.
%
%\texttt{@stringset s := ... ;}\newline
%\textbf{foreach} \texttt {s} \textbf {do}\newline
%\texttt{\# the \emph{key} constant has the value of current entry of \emph{s} stringset}\newline
%\textbf{end foreach} \texttt{;}
%
%
%
%
%
%
%
%\subsection{Comparison Operators}
%
%The \galgas{@stringset} type supports the six comparison operators:\newline
%
%\begin{tabular}{|c|c|}
%\hline
%$=$ & Equality \\
%\hline
%$!=$ & Non Equality \\
%\hline
%$<$  & Strict Inclusion \\
%\hline
%$<=$  & Inclusion or Equality \\
%\hline
%$>$  & Strict Greater \\
%\hline
%$>=$  & Greater or Equality \\
%\hline
%\end{tabular}
%
%Theses operators require both arguments to be \galgas{@stringset} objects, and return a \galgas{@stringset} object.
%
%

  %!TEX encoding = UTF-8 Unicode
%!TEX root = ../galgas-book.tex

\chapitreTypePredefiniLabelIndex{stringset}

An \ggs+@stringset+ object value is a set of \ggs+@string+ values.\\

\section{Constructors}

\subsectionConstructor{emptySet}{stringset}

\begin{galgascode}
constructor emptySet -> @stringset
\end{galgascode}


Creates and returns an empty \ggs+@stringset+ object.

\subsectionConstructor{setWithString}{stringset}

\begin{galgascode}
constructor setWithString ?@string inString -> @stringset
\end{galgascode}


Creates and returns an \ggs+@stringset+ object that contains the value of the \emph{inString} argument object.

\section{Getters}

\subsectionGetter{count}{stringset}

\begin{galgascode}
getter count -> @uint
\end{galgascode}

Returns the number of strings in the set.



\subsectionGetter{hasKey}{stringset}

\begin{galgascode}
getter hasKey ?@string inString -> @bool
\end{galgascode}

Returns a boolean value that indicates whether the value of \emph{inString} argument is present in the set: \ggs+true+ if the value of \emph{inString} argument is present in the set, \ggs+false+ otherwise.


\subsectionGetter{anyString}{stringset}

\begin{galgascode}
getter anyString -> @string
\end{galgascode}

Retourne une des chaînes de caractères contenue dans le récepteur. Si le récepteur est vide, une erreur d'exécution est déclenchée.




\section{Setter}

\subsectionSetter{removeKey}{stringset}

\begin{galgascode}
setter removeKey ?@string inString
\end{galgascode}


Removes the value of \emph{inString} argument from the receiver's value. If the receiver's value does not contain the value of \emph{inString} argument, this setter leaves the receiver's value unchanged.






\section{the \texttt{+=} Operator}

The \emph{+=} operator adds a string value to the receiver. If the receiver's value already contains the added value, this operator has no effect.

\textbf{exemple :}
\begin{galgascode}
@string aString = ... ;
@stringset aStringSet = ... ;
aStringSet += !aString ;
\end{galgascode}




\section{the \emph{$\&$} Operator}

The \emph{$\&$} operator returns the intersection of its operand values.

\textbf{exemple :}
\begin{galgascode}
@stringset s1 = ... ;
@stringset s2 = ... ;
@stringset s = s1 & s2 ; # s is the intersection of s1 and s2
\end{galgascode}






\section{the \emph{$\textbar$} Operator}

The \emph{$\textbar$} operator returns the union of its operand values.

\textbf{exemple :}
\begin{galgascode}
@stringset s1 = ... ;
@stringset s2 = ... ;
@stringset s = s1 | s2 ; # s is the union of s1 and s2
\end{galgascode}






\section{the \emph{$-$} Operator}

The \emph{$-$} operator returns the difference of its operand values.

\textbf{exemple :}
\begin{galgascode}
@stringset s1 = ... ;
@stringset s2 = ... ;
@stringset s = s1 - s2 ; \# s is the difference of s1 and s2
\end{galgascode}








\section{Enumerating \texttt{@stringset} objects}


The \ggs+for+ instruction can be used for enumerating \ggs+@stringset+ values; enumeration is performed in the ascending order, or in the reverse alphabetical order using the '>' qualifier.

\texttt{@stringset s = ... ;}\newline
\textbf{foreach} \texttt {s} \textbf {do}\newline
\texttt{\# the \emph{key} constant has the value of current entry of \emph{s} stringset}\newline
\textbf{end foreach} \texttt{;}







\section{Comparison Operators}

The \ggs+@stringset+ type supports the six comparison operators:\newline

\begin{tabular}{|c|c|}
\hline
$=$ & Equality \\
\hline
$!=$ & Non Equality \\
\hline
$<$  & Strict Inclusion \\
\hline
$<=$  & Inclusion or Equality \\
\hline
$>$  & Strict Greater \\
\hline
$>=$  & Greater or Equality \\
\hline
\end{tabular}

Theses operators require both arguments to be \ggs+@stringset+ objects, and return a \ggs+@stringset+ object.



  %!TEX encoding = UTF-8 Unicode
%!TEX root = ../galgas-book.tex

\chapitreTypePredefiniLabelIndex{uint}

\tableDesMatieresLocaleDeProfondeurRelative{1}


An \ggst+@uint+ object value is a 32-bit unsigned integer value. You can initialize an \ggst+@uint+ object from an unsigned integer constant:\\

\begin{galgas3}
@uint myUnsignedInteger = 123_456 ;
\end{galgas3}

Note that a 32-bit unsigned integer constant is characterized by no suffix.

\section{Constructors}

\subsectionConstructor{errorCount}{uint}

\begin{galgas3}
constructor errorCount -> @uint
\end{galgas3}


Returns an \ggst+@uint+ object that contains the number of errors. The returned value is the cumulative count of errors from the beginning of execution.

\textbf{Exemple :}
\begin{galgas3}
@uint x = [@uint errorCount] ;
\end{galgas3}




\subsectionConstructor{max}{uint}

\begin{galgas3}
constructor max -> @uint
\end{galgas3}

Returns an \ggst+@uint+ object that the maximum value of the 32-bit unsigned range ($2^{32}-1$).






\subsectionConstructor{random}{uint}

\begin{galgas3}
constructor random -> @uint
\end{galgas3}

Retourne une valeur aléatoire de type \ggst+@uint+. La procédure de type \refStaticProcPage{uint}{setRandomSeed} permet d'en fixer la valeur initiale.

\begin{galgas3}
  let v = @uint.random
\end{galgas3}


{\bf Note. } Sur Unix, la valeur renvoyée est la valeur renvoyée par l'appel de la fonction \texttt{random} de la librairie \texttt{libc}. Sur Windows, c'est la fonction \texttt{rand} qui est appelée.


\subsectionConstructor{valueWithMask}{uint}

\begin{galgas3}
constructor valueWithMask ?@uint inLowerIndex ?@uint inUpperIndex -> @uint
\end{galgas3}


Returns an \ggst+@uint+ object with bits from \emph{inLowerIndex} to \emph{inUpperIndex} equal to 1.

A run-time error is raised if \emph{inLowerIndex $>$ inUpperIndex} or if \emph{inUpperIndex $>$ 31}.



\textbf{Exemple :}
\begin{galgas3}
@uint x = [@uint valueWithMask !2 !4] ; # x is equal to 28 (0b1_1100)
\end{galgas3}




\subsectionConstructor{warningCount}{uint}

\begin{galgas3}
constructor warningCount -> @uint
\end{galgas3}


Returns an \ggst+@uint+ object that contains the number of warnings. The returned value is the cumulative count of warnings from the beginning of execution.





\section{Procédure de type}


\subsectionStaticProc{setRandomSeed}{uint}


\begin{galgas3box}
proc @uint setRandomSeed ?@uint inSeed
\end{galgas3box}

Affecte la valeur initiale utilisée par le générateur de nombres aléatoires (voir le \refConstructorPage{uint}{random}) Par exemple~:

\begin{galgas3}
  [@uint setRandomSeed !0]
\end{galgas3}






\section{Getters}

\subsectionGetter{alphaString}{uint}

Ce \emph{getter} permet de convertir un \ggst!@uint! en une chaîne de caractères, telle que l'ordre des entiers est conservé sur la chaîne obtenue.

La chaîne obtenue comporte exactement 7 lettres minuscules. C'est en fait une conversion en base 26, la lettre \ggst=a= ayant la valeur $0$, et la lettre \ggst=z= la valeur $25$.


\begin{galgas3}
  message [0 alphaString] + "\n"         # aaaaaaa
  message [12_345 alphaString] + "\n"    # aaaasgv
  message [@uint.max alphaString] + "\n" # nxmrlxv
\end{galgas3}



\subsectionGetter{bigint}{uint}

Ce \emph{getter} permet de convertir un \ggst!@uint! en \ggst!@bigint!. Comme la plage des valeurs des \ggst!bigint! n'est limitée que par la mémoire disponible, il n'échoue jamais.

\begin{galgas3}
  message [[1234 bigint] string] + "\n" # 1234
\end{galgas3}


\subsectionGetter{double}{uint}

\begin{galgas3}
getter double -> @double
\end{galgas3}

Returns the receiver's value converted in a \ggst+@double+ object. As a 32-bit integer value can always be converted in a \ggst+@double+ value, this getter never fails.



\subsectionGetter{hexString}{uint}

\begin{galgas3}
getter hexString -> @string
\end{galgas3}

Returns the an hexadecimal string representation of the receiver value, prefixed by the string \texttt{0x}. For getting an hexadecimal representation string without any prefix, see \refGetterPage{uint}{xString}.



\subsectionGetter{hexStringSeparatedBy}{uint}

\begin{galgas3}
getter hexStringSeparatedBy ?@char inSeparator ?@uint inGroup -> @string
\end{galgas3}

Returns the an hexadecimal string representation of the receiver value, prefixed by the string \texttt{0x}. Groups of \ggst=inGroup= digits are separated by the \ggst=inSeparator= character.

If \ggst=inGroup= is equal to zero, a run-time error is raised.

For example:
\begin{galgas3}
let s = [0x12345678 hexStringSeparatedBy !'_' !2] # 0x12_34_56_78
\end{galgas3}



\subsectionGetter{isInRange}{uint}

\begin{galgas3}
getter isInRange ?@range inRange -> @bool
\end{galgas3}

{Returns an \ggst+@bool+ value indicating whether the receiver'value belongs to \ggst+inRange+ range : for a receiver's value equal to $v$ and a range of length $length$ starting at $start$, it returns \ggst+true+ if $((v \geqslant start)~and~(v<(start+length)))$, and \ggst+false+ otherwise.



\subsectionGetter{isUnicodeValueAssigned}{uint}

\begin{galgas3}
getter isUnicodeValueAssigned -> @bool
\end{galgas3}

Returns an \ggst+@bool+ value indicating whether the receiver'value represents an assigned Unicode character. It returns \ggst+true+ if the receiver value represents an assigned Unicode character, \ggst+false+ and otherwise.

\textbf{Exemple :}
\begin{galgas3}
[0xFFFF isUnicodeValueAssigned] # is false, as \uFFFF is not assigned.
[0x41 isUnicodeValueAssigned] # is true, as \u0041 is assigned (LATIN CAPITAL LETTER A).
\end{galgas3}



\subsectionGetter{lsbIndex}{uint}

\begin{galgas3}
getter lsbIndex -> @uint
\end{galgas3}

Returns an \ggst+@uint+ value of the index of the most significant bit of the receiver value. It raises a run-time error if the receiver value is zero.

\textbf{Exemple :}
\begin{galgas3}
@uint value = 192 ; # 192 is ...011000000 in binary
@uint x = [value lsbIndex] ; # x is equal to 7
\end{galgas3}

The most significant bit of 192 is the 7th bit.




\subsectionGetter{significantBitCount}{uint}

\begin{galgas3}
getter significantBitCount -> @uint
\end{galgas3}

Returns the number of bits needed to express the receiver value. If the receiver value is zero, it returns 0 ; otherwise, it returns the most significant bit index plus one.

\textbf{Exemple :}
\begin{galgas3}
@uint value = 145 ; # 145 is 10010001 in binary
@uint x = [value significantBitCount] ; # x is equal to 8
\end{galgas3}






\subsectionGetter{sint}{uint}

\begin{galgas3}
getter sint -> @sint
\end{galgas3}

Returns the receiver's value in an \refTypePredefini{sint} (32-bit signed integer) object. An error is raised is receiver's value is greater than $2^{31}-1$.

This getter is the only way to convert an \refTypePredefini{uint} object into an \refTypePredefini{sint} object.




\subsectionGetter{sint64}{uint}

\begin{galgas3}
getter sint64 -> @sint64
\end{galgas3}

Returns the receiver's value in an \refTypePredefini{sint64} (64-bit signed integer) object. As a 32-bit unsigned value can always be converted in a 64-bit signed value, this getter never fails.

This getter is the only way to convert an \refTypePredefini{uint} object into an \refTypePredefini{sint64} object.


\subsectionGetter{string}{uint}

\begin{galgas3}
getter string -> @string
\end{galgas3}

Returns a decimal string representation of the receiver's value. For an hexadecimal string representation of the receiver's value, see \refGetterPage{uint}{hexString} and \refGetterPage{uint}{xString}.




\subsectionGetter{uint64}{uint}

\begin{galgas3}
getter uint64 -> @uint64
\end{galgas3}

Returns the receiver's value in an \refTypePredefini{uint64} (64-bit unsigned integer) object. As a 32-bit unsigned value can always be converted in a 64-bit unsigned value, this getter never fails.

This getter is the only way to convert an \refTypePredefini{uint} object into an \refTypePredefini{uint64} object.




\subsectionGetter{xString}{uint}

\begin{galgas3}
getter xString -> @string
\end{galgas3}

Returns an hexadecimal string representation of the receiver's value (without any prefix). For an decimal string representation of the receiver's value, see the \refGetterPage{uint}{hexString}; for a decimal string representation of the receiver's value, see the \refGetterPage{uint}{string}.







\section{Arithmétique}

\subsection{Opérateurs infixés}

Le type \ggst+@uint+ accepte les opérateurs arithmétiques infixés suivants :
\begin{itemize}
  \item \ggst!+!, addition, une erreur d'exécution est déclenchée en cas de débordement ;
  \item \ggst!-!, soustraction, une erreur d'exécution est déclenchée en cas de débordement ;
  \item \ggst!*!, multiplication, une erreur d'exécution est déclenchée en cas de débordement ;
  \item \ggst!/!, division, une erreur d'exécution est déclenchée si le diviseur est nul ;
  \item \ggst!mod!, calcul du reste, une erreur d'exécution est déclenchée si le diviseur est nul ;
  \item \ggst!&+!, addition, le résultat étant silencieusement tronqué sur 32 bits ;
  \item \ggst!&-!, soustraction, le résultat étant silencieusement tronqué sur 32 bits ;
  \item \ggst!&*!, multiplication, le résultat étant silencieusement tronqué sur 32 bits ;
  \item \ggst!&/!, division, qui retourne zéro si le diviseur est nul.
\end{itemize}

Ces opérateurs exigent que les deux opérandes soient des objets du même type \ggst+@uint+.

\subsection{Opérateur préfixé}
Le type \ggst+@uint+ accepte un opérateur arithmétique préfixé :
\begin{itemize}
  \item \ggst!+!, qui retourne simplement la valeur de l'opérande.
\end{itemize}

\subsectionLabel{Instructions}{instructionsUINT}

Le type \ggst+@uint+ accepte les deux instructions arithmétiques suivantes :
\begin{itemize}
  \item \ggst!+=!, addition, une erreur d'exécution est déclenchée en cas de débordement ;
  \item \ggst!-=!, soustraction, une erreur d'exécution est déclenchée en cas de débordement ;
  \item \ggst!*=!, multiplication, une erreur d'exécution est déclenchée en cas de débordement ;
  \item \ggst!/=!, division, une erreur d'exécution est déclenchée en cas division par zéro ;
  \item \ggst!++!, incrémentation, une erreur d'exécution est déclenchée en cas de débordement ;
  \item \ggst!--!, décrémentation, une erreur d'exécution est déclenchée en cas de débordement ;
  \item \ggst!&++!, incrémentation, le résultat étant silencieusement tronqué sur 32 bits ;
  \item \ggst!&--!, décrémentation, le résultat étant silencieusement tronqué sur 32 bits.
\end{itemize}

\ggst!x+=y! est équivalent à \ggst!x=x+y! ; \ggst!x-=y! est équivalent à \ggst!x=x-y!.
La variable cible \ggst!x!, comme l'expression source \ggst!y! doivent être du même type \ggst+@uint+.

Incrémentation et décrémentation sont des instructions, et ne peuvent pas apparaître des expressions.
\begin{galgas3}
@uint n = ... ; n ++ # Incrémentation
\end{galgas3}

\begin{galgas3}
@uint n = ... ; n -- # Décrémentation
\end{galgas3}




\section{Shift Operators}


The \ggst+@uint+ type supports right and left shift operators:\newline

\begin{tabular}{|c|c|}
\hline
$<<$ & Left shift \\
\hline
$>>$ & Right shift \\
\hline
\end{tabular}

Theses operators require both arguments to be \ggst+@uint+ objects.\newline

Note the right shift inserts always a zero bit in the most significant bit location (it is a logical right shift).\newline

The actual amount of the shift is the value of the right-hand operand masked by 31, i.e. the shift distance is always between 0 and 31.




\section{Logical Operators}

The \ggst+@uint+ type supports the three bit-wise logical operators:\newline

\begin{tabular}{|c|c|}
\hline
$\&$ & Bit-wise and \\
\hline
\textbar & Bit-wise or \\
\hline
\^\  & Bit-wise exclusive or \\
\hline
\end{tabular}

Theses operators require both arguments to be \ggst+@uint+ objects.\newline


The \ggst+@uint+ type supports the bit-wise logical unary operator:\newline

\begin{tabular}{|c|c|}
\hline
$\sim$ & Bit-wise complementation \\
\hline
\end{tabular}

This operator returns an \ggst+@uint+ object.







\section{Comparison Operators}

The \ggst+@uint+ type supports the six comparison operators:\newline

\begin{tabular}{|c|c|}
\hline
$=$ & Equality \\
\hline
$!=$ & Non Equality \\
\hline
$<$  & Strict Lower Than \\
\hline
$<=$  & Lower or Equal \\
\hline
$>$  & Strict Greater Than \\
\hline
$>=$  & Greater or Equal \\
\hline
\end{tabular}

\vspace{2mm}
Theses operators require both arguments to be \ggst+@uint+ objects, and return a \ggst+@bool+ object.



  %!TEX encoding = UTF-8 Unicode
%!TEX root = ../galgas-book.tex

\chapitreTypePredefiniLabelIndex{uint64}

An \galgas{@uint64} object value is a 64-bit unsigned integer value. You can initialize an \galgas{@uint64} object from a 64-bit unsigned integer constant:\\

\texttt{@uint64 myUnsignedInteger := 123\_456L ;}\newline

Note the 'L' suffix is required for a 64-bit unsigned integer constant.

\section{Constructeurs}

\constructeurSansArgument{max}
{uint64}
{1.3.0}
{uint64}
{Returns an \galgas{@uint64} object that the maximum value of the 64-bit unsigned range.}
{The returned value is $2^{64}-1$.}


\constructeurUnArgument{uint64BaseValueWithCompressedBitString}
{uint64}
{1.6.4}
{uint64}
{@string inBitString}
{Returns an \galgas{@uint64} object computed from a string containing '0', '1' or 'X' characters, replacing all occurrences of 'X' by '0'.}
{the inBitString argument should contain only '0', '1' or 'X' characters. A run time exception is raised if an other character appears.

This constructor considers the \emph{inBitString} argument value as a binary encoding of an integer value. First, it internally replaces all 'X's by '0's, and then converts the resulting string into an integer value that is the one returned by this constructor.

Note that the first character of the \emph{inBitString} argument value corresponds to the most significant bit of the converted value.}


\textbf{Exemple :}
\begin{galgascode}
@uint64 v [uint64BaseValueWithCompressedBitString !"01XX10"] ;
log v ; # Displays <@uint64:18> ;
\end{galgascode}





\constructeurUnArgument{uint64MaskWithCompressedBitString}
{uint64}
{1.6.4}
{uint64}
{@string inBitString}
{Returns an \galgas{@uint64} object computed from a string containing '0', '1' or 'X' characters, replacing all occurrences of '0' by '1' and all occurrences of 'X' by '0'.}
{the \emph{inBitString} argument should contain only '0', '1' and 'X' characters. A run time exception is raised if an other character appears.

This constructor considers the \emph{inBitString} argument value as a binary encoding of an integer value. First, it internally replaces all '0's by '1's and all 'X's by '0's, and then converts the resulting string into an integer value that is the one returned by this constructor.

Note that the first '0' or '1' character of the \emph{inBitString} argument value corresponds to the most significant Bit of the converted value.}

\textbf{Exemple :}
\begin{galgascode}
@uint64 v [uint64MaskWithCompressedBitString !"01XX10"] ;
log v ; \# Displays <@uint64:51> ;
\end{galgascode}



\constructeurUnArgument{uint64WithBitString}
{uint64}
{1.6.4}
{uint64}
{@string inBitString}
{Returns an \galgas{@uint64} object computed from a string containing '0' or '1' characters.}
{the \emph{inBitString} argument should contain only '0' and '1' characters. A run time exception is raised if an other character appears.

This constructor considers the \emph{inBitString} argument value as a binary encoding of an integer value. It returns an \galgas{@uint64} object containing the converted value.

Note that the first '1' character of the \emph{inBitString} argument value corresponds to the most significant bit of the converted value.}

\textbf{Exemple :}
\begin{galgascode}
@uint64 v [uint64WithBitString !"0101"]] ;
log v ; # Displays <@uint64:5> ;
\end{galgascode}


\section{Readers}

\readerSansArgument{double}
{uint64}
{1.9.8}
{double}
{Returns the receiver's value converted in a \galgas{@double} object.}
{as a 64-bit integer value can always be converted in a \galgas{@double} value, this reader never fails.}



\readerSansArgument{hexString}
{uint64}
{1.5.2}
{string}
{Returns the an hexadecimal string representation of the receiver value, prefixed by the string "0x".}
{for getting an hexadecimal representation string without "0x" prefix, see \refReaderPage{uint64}{xString}.}





\readerSansArgument{sint}
{uint64}
{1.6.12}
{sint}
{Returns the receiver's value in an \refTypePredefini{sint} (32-bit signed integer) object.}
{an error is raised is receiver's value is greater than $2^{31}-1$.}

This reader is the only way to convert an \refTypePredefini{uint64} object into an \refTypePredefini{sint} object.




\readerSansArgument{sint64}
{uint64}
{1.6.12}
{sint64}
{Returns the receiver's value in an \refTypePredefini{sint64} (64-bit signed integer) object.}
{an error is raised is receiver's value is greater than $2^{63}-1$.}

This reader is the only way to convert an \refTypePredefini{uint64} object into an \refTypePredefini{sint64} object.


\readerSansArgument{string}
{uint64}
{1.6.12}
{string}
{Returns a decimal string representation of the receiver's value.}
{for an hexadecimal string representation of the receiver's value, see \refReaderPage{uint64}{hexString} and \refReaderPage{uint64}{xString}.}



\readerSansArgument{uint}
{uint64}
{1.6.12}
{uint}
{Returns the receiver's value in an \refTypePredefini{uint} (32-bit unsigned integer) object.}
{an error is raised is receiver's value is greater than $2^{32}-1$.}

This reader is the only way to convert an \refTypePredefini{uint64} object into an \refTypePredefini{uint} object.


\readerDeuxArguments{uintSlice}
{uint64}
{1.6.0}
{uint}
{@uint inStartBit}
{@uint inBitCount}
{Returns an \refTypePredefini{uint} value, extracted from a bit slice of the receiver's value.}
{the receiver's value is right shifted by \emph{inStartBit}, and the resulted value is and'ed with a mask equal to $2^{inBitCount}-1$.}


\textbf{Exemple :}
\begin{galgascode}
@uint64 v := 0x1234_5678_9ABC_DEF0L ;
@uint result := [v uintSlice !4 !5] ; # The result value is 0x8_9ABC
\end{galgascode}




%\defReaderSansArgument{xString}{uint64}
%{1.9.10}
%{string}
%{Returns an hexadecimal string representation of the receiver's value (without any prefix).}
%{for an decimal string representation of the receiver's value, see the \refReaderPage{uint64}{hexString}; for a decimal string representation of the receiver's value, see the \refReaderPage{uint64}{string}.}
%{}{}

\readerSansArgument{xString}
{uint64}
{1.9.10}
{string}
{Returns an hexadecimal string representation of the receiver's value (without any prefix).}
{for an decimal string representation of the receiver's value, see the \refReaderPage{uint64}{hexString}; for a decimal string representation of the receiver's value, see the \refReaderPage{uint64}{string}.}






\section{Incrementation and decrementation}

The \refTypePredefini{uint64} supports incrementation and decrementation instructions.

\texttt{@uint64 n := ... ; n ++ ; \# Incrementation}

\texttt{@uint64 p := ... ; p -- ; \# Decrementation}\newline

The incrementation instruction raises an error if receiver's value is equal to $2^{64}-1$.\newline

The incrementation instruction raises an error if receiver's value is equal to 0.\newline

Note that incrementation and decrementation are not available within an expression.




\section{Arithmetic Operators}

The \galgas{@uint64} type supports the five arithmetic diadic operators:\newline

\begin{tabular}{|c|c|}
\hline
$+$ & Addition \\
\hline
$-$ & Substraction \\
\hline
$*$ & Multiplication \\
\hline
$/$ & Division \\
\hline
\galgas{mod} & Modulo \\
\hline
\end{tabular}

Theses operators require both arguments to be \galgas{@uint64} objects.\newline

A run-time error is raised if the operation leads to a 64-bit unsigned overflow.

The \galgas{@uint64} type supports the following arithmetic unary operator:\newline

\begin{tabular}{|c|c|}
\hline
$+$ & No operation \\
\hline
\end{tabular}

This operator returns the receiver's value (an  \galgas{@uint64} object).




\section{Shift Operators}


The \galgas{@uint} type supports right and left shift operators:\newline

\begin{tabular}{|c|c|}
\hline
$<<$ & Left shift \\
\hline
$>>$ & Right shift \\
\hline
\end{tabular}

Theses operators require the left argument to be \galgas{@uint64} object, and  the right argument to be \galgas{@uint} object.\newline

Note the right shift inserts always a zero bit in the most significant bit location (it is a logical right shift).\newline

The actual amount of the shift is the value of the right-hand operand masked by 63, i.e. the shift distance is always between 0 and 63.




\section{Logical Operators}

The \galgas{@uint64} type supports the three bit-wise logical diadic operators:

\begin{tabular}{|c|c|}
\hline
$\&$ & Bit-wise and \\
\hline
\textbar & Bit-wise or \\
\hline
\^\  & Bit-wise exclusive or \\
\hline
\end{tabular}

Theses operators require both arguments to be \galgas{@uint64} objects.\newline


The \galgas{@uint64} type supports the bit-wise logical unary operator:

\begin{tabular}{|c|c|}
  \hline
  $\sim$ & Bit-wise complementation \\
  \hline
\end{tabular}

This operator returns an \galgas{@uint64} object.




\section{Comparison Operators}

The \galgas{@uint64} type supports the six comparison operators:

\begin{tabular}{|c|c|}
\hline
$=$ & Equality \\
\hline
$!=$ & Non Equality \\
\hline
$<$  & Strict Lower Than \\
\hline
$<=$  & Lower or Equal \\
\hline
$>$  & Strict Greater Than \\
\hline
$>=$  & Greater or Equal \\
\hline
\end{tabular}

Theses operators require both arguments to be \galgas{@uint64} objects, and return a \galgas{@bool} object.

  %!TEX encoding = UTF-8 Unicode
%!TEX root = ../galgas-book.tex

%--------------------------------------------------------------
\chapter{List Type}
%-------------------------------------------------------------

\section{List Type Declaration}

A \lstinline[language=galgas]!list! type declaration names all attributes of the list elements:

\begin{lstlisting}[language=galgas]
list @MyList {
  @string mFirstAttribute ;
  @bool mSecondAttribute ;
}
\end{lstlisting}

\section{Constructors}

\subsection{The \lstinline[language=galgas]!emptyList! constructor}

For every list, an \lstinline[language=galgas]!emptyList! constructor is implicitly declared. It returns an empty list:

\begin{lstlisting}[language=galgas]
@MyList aList := [@MyList emptyList] ;
\end{lstlisting}


\subsection{The \lstinline[language=galgas]!listWithValue! constructor}

A list can be constructed directly with one value:

\begin{lstlisting}[language=galgas]
@MyList aList := [@myList listWithValue !"c" !3] ;
\end{lstlisting}


Using this constructor is equivalent to:

\begin{lstlisting}[language=galgas]
@MyList aList := [@MyList emptyList] ;
aList += !"c" !3 ;
\end{lstlisting}

\section{Adding elements}

\subsection{The \lstinline[language=galgas]!+=! operator}

The  \lstinline[language=galgas]!+=! operator adds a new element at the end of the list. The right side expressions should correspond to the attributes declared in the \lstinline[language=galgas]!list! declaration:\\

\begin{lstlisting}[language=galgas]
@MyList aList := ... ;
@string aString := ... ;
@bool aBool := ... ;
aList += !aString !aBool ;''
\end{lstlisting}


\subsection{The \lstinline[language=galgas]!.=! operator}

The \lstinline[language=galgas]!.=! operator concats a list at the end of the target list:

\begin{lstlisting}[language=galgas]
@MyList aList := ... ;
@MyList secondList := ... ;
aList .= secondList ;''
\end{lstlisting}



\subsection{The \lstinline[language=galgas]!prependValue! modifier}

The \lstinline[language=galgas]!prependValue! modifier adds a new element at the begining of the list. The right side expressions should correspond to the attributes declared in the  \lstinline[language=galgas]!list! declaration:

\begin{lstlisting}[language=galgas]
@MyList aList := ... ;
@string aString := ... ;
@bool aBool := ... ;
[!?aList prependValue !aString !aBool];
\end{lstlisting}

\subsection{The concatenation operator}

The «~\lstinline[language=galgas]!.!~» operator can be used fot concatenating two lists of the same type:


\begin{lstlisting}[language=galgas]
@MyList firstList := ... ;
@MyList secondList := ... ;
@MyList thirdList := firstList . secondList ;
\end{lstlisting}

\section{Removing elements}

\subsection{The \lstinline[language=galgas]!popFirst! modifier}


The \lstinline[language=galgas]!popFirst! modifier removes and returns the first element of the list. The right side expressions should correspond to the attributes declared in the \lstinline[language=galgas]!list! declaration:\\

\begin{lstlisting}[language=galgas]
@MyList aList := ... ;
@string aString ;
@bool aBool ;
[!?aList popFirst ?aString ?aBool];
\end{lstlisting}

If the list is empty when \lstinline[language=galgas]!popFirst! modifier is invoked, a run-time error is raised and the input arguments are not valuated.

\subsection{The \lstinline[language=galgas]!popLast! modifier}


The \lstinline[language=galgas]!popLast! modifier removes and returns the last element of the list. The right side expressions should correspond to the attributes declared in the \lstinline[language=galgas]!list! declaration:

\begin{lstlisting}[language=galgas]
@MyList aList := ... ;
@string aString ;
@bool aBool ;
[!?aList popLast ?aString ?aBool];
\end{lstlisting}

If the list is empty when \lstinline[language=galgas]!popLast! is invoked, a run-time error is raised and the input arguments are not valuated.

\section{Methods}

\subsection{The \lstinline[language=galgas]!first! method}

The \lstinline[language=galgas]!first! method returns the first element of the list. The element is not removed. The right side expressions should correspond to the attributes declared in the \lstinline[language=galgas]!list! declaration:

\begin{lstlisting}[language=galgas]
@MyList aList := ... ;
@string aString ;
@bool aBool ;
[aList first ?aString ?aBool];
\end{lstlisting}

If the list is empty when \lstinline[language=galgas]!first! is invoked, a run-time error is raised and the input arguments are not valuated.

\subsection{The \lstinline[language=galgas]!last! method}

The \lstinline[language=galgas]!last! method returns the last element of the list. The element is not removed. The right side expressions should correspond to the attributes declared in the \lstinline[language=galgas]!list! declaration:\\

\begin{lstlisting}[language=galgas]
@MyList aList := ... ;
@string aString ;
@bool aBool ;
[aList last ?aString ?aBool];
\end{lstlisting}


If the list is empty when \lstinline[language=galgas]!last! is invoked, a run-time error is raised and the input arguments are not valuated.








\section{Readers}

\subsection{The \lstinline[language=galgas]!length! reader}

\begin{lstlisting}[language=galgas]
reader length -> @uint ;
\end{lstlisting}

The \lstinline[language=galgas]!length! reader returns the number of elements in the receiver's value.


\subsection{The \lstinline[language=galgas]!range! reader}

\begin{lstlisting}[language=galgas]
reader range -> @range ;
\end{lstlisting}

The \lstinline[language=galgas]!range! reader returns a range starting at $0$ of length equal to the number of elements in the receiver's value.




\subsection{The \lstinline[language=galgas]!subListFromIndex! reader}

\begin{lstlisting}[language=galgas]
reader subListFromIndex ?@uint inIndex -> @self
\end{lstlisting}

This reader returns a new list containing the elements of the receiver from the one at a given index to the end. The  \lstinline[language=galgas]!inIndex! value should be lower or equal to the length of the receiver's value. If \lstinline[language=galgas]!inIndex! is equal to the length of the receiver, the reader returns an empty list.


\subsection{The \lstinline[language=galgas]!subListWithRange! reader}

\begin{lstlisting}[language=galgas]
reader subListWithRange
  ?@range inRange
  -> @self
\end{lstlisting}

This reader returns a list containing the elements of the receiver that lie within a given range. The range must not exceed the length of the receiver's value, that is $range\_start + range\_length \leqslant list\_length$. If the range's length is equal to zero, this reader returns an empty list.





\section{Enumerating a list with a foreach instruction}

The \lstinline[language=galgas]!foreach! instruction can be used for enumerating list objects. By default, lists are enumerated in the insertion order; enumeration in the reverse order is performed using the «~\lstinline[language=galgas]!>!~» qualifier.

There are two ways for accessing element values:
\begin{itemize}
\item using the implicitly declared constants that receive the current attribute values;
\item declare explicitly constants that receive the current attribute values.
\end{itemize}

Given the list declaration:

\begin{lstlisting}[language=galgas]
list @MyList {
  @string mFirstAttribute ;
  @bool mSecondAttribute ;
}
\end{lstlisting}

\subsection{Enumeration using the implicitly declared constants}

For every attribute, a constant of the same name is available in the \lstinline[language=galgas]!do! instruction list. Theses constants receive the value of the corresponding attribute of the current element.

\begin{lstlisting}[language=galgas]
foreach aList do
  # the mFirstAttribute constant receives the value
  # of the mFirstAttribute attribute of the current element,
  # and the mSecondAttribute constant receives the value
  # of the mSecondAttribute attribute of the current element.
end foreach ;
\end{lstlisting}

\subsection{Enumeration using the explicitly declared constants}

The \lstinline[language=galgas]!foreach! header declares a sequence of constants, corresponding to the attribute list of the \lstinline[language=galgas]!do! declaration. Theses constants receive the value of the corresponding attribute of the current element.


\begin{lstlisting}[language=galgas]
foreach aList (@string kString @bool kBool) do
  # the kString constant receives the value
  # of the mFirstAttribute attribute of the current element,
  # and the kBool constant receives the value
  # of the mSecondAttribute attribute of the current element.
end foreach ;
\end{lstlisting}

\subsection{Enumeration in the reverse order}

In GALGAS 1.7.3 and later, you can enumerate a list in the reverse order using the «~\lstinline[language=galgas]!>!~» qualifier:

\begin{lstlisting}[language=galgas]
foreach > aList (@string kString @bool kBool) do
  ...
end foreach ;
\end{lstlisting}




\section{Direct Access of an element attribute}

In GALGAS 1.7.5 and later, lists can be used as an array. Each element of a list is associated with an \nomType{uint} index, spanning from 0 to element count (value returned by \lstinline[language=galgas]!length! reader) minus one.

The element retrieved with \lstinline[language=galgas]!first! method is at index 0.

The element retrieved with \lstinline[language=galgas]!last! method is at index equal to element count minus one.

\subsection{Read Access}

By default and for every attribute, a reader is provided to retrieve the value of this attribute for an element at a given index. For example, for an attribute named \emph{name}, the \emph{nameAtIndex} reader is provided. It accepts one \nomType{uint} argument, the value of the index.

You can disable the default reader generation, by using the «~\lstinline[language=galgas]!feature nogetter!~» qualifier.

For example:
\begin{lstlisting}[language=galgas]
list @MyList {
  @string mFirstAttribute ;
  @bool mSecondAttribute feature nogetter ;
}
...
@MyList aList := ... ;
@string s := [aList mFirstAttributeAtIndex !1] ;
\end{lstlisting}

One reader is available: \lstinline[language=galgas]!mFirstAttributeAtIndex!; the \lstinline[language=galgas]!mSecondAttributeAtIndex! reader is not available.


\subsection{Write Access}

By default, no modifier is provided for performing a direct write access to an attribute at a given index. You should use the «~\lstinline[language=galgas]!feature setter!~» qualifier for enabling setter generation for a given attribute.

The modifier name is the name of the attribute with the first letter capitalized, prefixed by \emph{set} and suffixed by \emph{AtIndex}: for an attribute named \emph{name}, the modifier is named \emph{setNameAtIndex}. It accepts two arguments, the first one is the new attribute's value, the second one an \nomType{uint} argument, the value of the index.

For example:

\begin{lstlisting}[language=galgas]
list @MyList {
  @string mFirstAttribute feature setter ;
  @bool mSecondAttribute ;
}
...
@string s := ... ;
[!?aList setMFirstAttributeAtIndex !s !1] ;
\end{lstlisting}

One modifier is available: \lstinline[language=galgas]!setMFirstAttributeAtIndex!; the \lstinline[language=galgas]!setMSecondAttributeAtIndex! modifier is not available.

\subsection{Example of read and write accesses}

\begin{lstlisting}[language=galgas]
list @myList {
  @string name ;
}
...
@myList strList [emptyList] ;
strList += !"a" ;
strList += !"b" ;
strList += !"c" ;
strList += !"d" ;
@string s := [strList nameAtIndex !0] ;
log s ; # displays LOGGING s: <@string:"a">
s := [strList nameAtIndex !1] ;
log s ; # displays LOGGING s: <@string:"b">
s := [strList nameAtIndex !2] ;
log s ; # displays LOGGING s: <@string:"c">
s := [strList nameAtIndex !3] ;
log s ; # displays LOGGING s: <@string:"d">
[!?strList setNameAtIndex !"x" !0] ;
[!?strList setNameAtIndex !"y" !1] ;
[!?strList setNameAtIndex !"z" !2] ;
[!?strList setNameAtIndex !"t" !3] ;
s := [strList nameAtIndex !0] ;
log s ; # displays LOGGING s: <@string:"x">
s := [strList nameAtIndex !1] ;
log s ; # displays LOGGING s: <@string:"y">
s := [strList nameAtIndex !2] ;
log s ; # displays LOGGING s: <@string:"z">
s := [strList nameAtIndex !3] ;
log s ; # displays LOGGING s: <@string:"t">
\end{lstlisting}

  %!TEX encoding = UTF-8 Unicode
%!TEX root = ../galgas-book.tex

%--------------------------------------------------------------
\chapter{Le type \texttt{sortedlist}}
%-------------------------------------------------------------

Le type \ggs+sortedlist+ permet de construire des listes ordonnées de valeurs.




\section{Déclaration}

La déclaration d'une \ggs+sortedlist+ nomme tous les attributs qui composent un élément de liste et la description du tri. Par Exemple :

\begin{galgascode}
sortedlist @MaListeOrdonnee {
  @char mCaractere ;
  @uint mEntier ;
}{
  mCaractere <, mEntier >
}
\end{galgascode}

La description du tri est exprimée par la liste ordonnée des attributs qui interviennent dans le tri, chacun d'eux étant suivi de l'ordre du tri (\ggs+<+ pour croissant, et \ggs+>+ pour décroissant). Ainsi, les élements des instances du type liste ordonnée ci-dessus sont triés par ordre croissant du champ caractère, puis par ordre décroissant du champ entier.

Déclarer une \ggs+sortedlist+ définit implicitement :
\begin{itemize}
  \item le constructeur \ggs+emptySortedList+ qui construit une liste vide (\refSubsectionPage{constructeurSortedlistEmptySortedList}) ;
  \item le constructeur \ggs+sortedListWithValue+ qui construit une liste contenant un élément (\refSubsectionPage{constructeurSortedlistSortedListWithValue}) ;
  \item l'opérateur \ggs*+=* pour ajouter un élément à une liste ordonnée (\refSubsectionPage{operateurSortedListPlusEgal}) ;
  \item l'opérateur \ggs*+=* pour ajouter tous les éléments d'une liste à une liste ordonnée (\refSubsectionPage{operateurSortedListPointEgal}) ;
  \item l'opérateur \ggs*+* pour construire une liste ordonnée à partir de deux listes ordonnées (\refSubsectionPage{operateurSortedListPoint}) ;
  \item le \emph{getter} \ggs+length+, qui retourne le nombre d'éléments d'une liste (\refSectionPage{readerSortedListLength}) ;
  \item le \emph{setter} \ggs+popGreatest+, qui retourne les champs du plus grand élément d'une liste, et retire cet élément de cette liste (\refSubsectionPage{modifierSortedListPopGreatest}) ;
  \item le \emph{setter} \ggs+popSmallest+, qui retourne les champs du plus grand élément d'une liste, et retire cet élément de cette liste (\refSubsectionPage{modifierSortedListPopSmallest}) ;
  \item la \emph{méthode} \ggs+greatest+, qui retourne les champs du plus grand élément d'une liste sans la modifier (\refSubsectionPage{methodeSortedListGreatest}) ;
  \item la \emph{méthode} \ggs+smallest+, qui retourne les champs du plus petit élément d'une liste sans la modifier (\refSubsectionPage{methodeSortedListSmallest}).
\end{itemize}








\section{Constructeurs}

\subsectionLabel{Constructeur \texttt{emptySortedList}}{constructeurSortedlistEmptySortedList}

Le constructeur \ggs+emptySortedList+ construit et retourne une liste vide. Par exemple :
\begin{galgascode}
@MaListeOrdonnee uneListe [emptySortedList] ;
\end{galgascode}


\subsectionLabel{Constructeur \texttt{sortedListWithValue}}{constructeurSortedlistSortedListWithValue}

Le constructeur \ggs+sortedListWithValue+ construit et retourne une liste comprenant un élément. Cet élément est spécifié par les arguments effectifs de l'appel : ce constructeur présente une séquence d'arguments en entrée correspondant aux champs de l'élément. Par exemple :

\begin{galgascode}
@MaListeOrdonnee uneListe [sortedListWithValue
  !'a' # Affecte au champ mCaractere
  !10  # Affecte au champ mEntier
] ;
\end{galgascode}






\section{Opérateurs}


\subsectionLabel{L'opérateur \texttt{+=}}{operateurSortedListPlusEgal}

L'opérateur \ggs*+=* ajoute un élément à la liste ordonnée, en maintenant la relation d'ordre. L'élément ajouté est spécifié par la séquences des valeurs à affecter à ses champs. Si il y a un ou plusieurs éléments égaux à l'élément ajouté, ce dernier est placé après les éléments existants. 


Cette opération est effectuée en $O(log (n))$ où $n$ est le nombre d'éléments de la liste.

Exemple :

\begin{galgascode}
@MaListeOrdonnee uneListe [emptySortedList] ;
uneListe += !'b' ! 1 ; # b1
uneListe += !'b' ! 2 ; # b2
uneListe += !'d' ! 1 ; # d1
uneListe += !'f' ! 1 ; # f1
uneListe += !'a' ! 1 ; # a1
uneListe += !'c' ! 1 ; # c1
uneListe += !'f' ! 2 ; # f2
\end{galgascode}

\subsectionLabel{L'opérateur \texttt{.=}}{operateurSortedListPointEgal}

L'opérateur \ggs*+=* ajoute tous les éléments de l'expression à la liste ordonnée, en maintenant la relation d'ordre. Si il y a un ou plusieurs éléments égaux à chaque élément ajouté, ce dernier est placé après les éléments existants. 

Exemple :
\begin{galgascode}
@MaListeOrdonnee uneListe = ... ;
@MaListeOrdonnee autreListe = ... ;
uneListe .= autreListe ;
\end{galgascode}

\subsectionLabel{L'opérateur \texttt{.}}{operateurSortedListPoint}

L'opérateur \ggs*+* combine deux listes ordonnées. Les éléments de la seconde liste égaux à ceux de la première liste sont placés après ceux de la première liste.

Exemple :
\begin{galgascode}
@MaListeOrdonnee uneListe = ... ;
@MaListeOrdonnee autreListe = ... ;
@MaListeOrdonnee troisiemeListe = uneListe . autreListe ;
\end{galgascode}







\sectionLabel{Getter \texttt{length}}{readerSortedListLength}

Le getter \ggs+length+ retourne un \ggs+@uint+ contenant le nombre d'éléments de la liste ordonnée.






\section{Setters}

\subsectionLabel{Setter \texttt{popGreatest}}{modifierSortedListPopGreatest}

Ce \emph{setter} retourne les champs du plus grand élément de la liste ordonnée, et le retire. Si la liste est vide, un message d'erreur est affiché, et les variables destinées à recevoir les valeurs des champs sont placées dans l'état \emph{invalide}. Par exemple :

\begin{galgascode}
@MaListeOrdonnee uneListe = ... ;
...
[!?uneListe popGreatest
  ?@char c
  ?@uint n
] ;
\end{galgascode}

Si \ggs+uneListe+ est vide, les variables \ggs+c+ et \ggs+n+ sont placées dans l'état \emph{invalide}.


\subsectionLabel{Setter \texttt{popSmallest}}{modifierSortedListPopSmallest}

Ce \emph{setter} retourne les champs du plus petit élément de la liste ordonnée, et le retire. Si la liste est vide, un message d'erreur est affiché, et les variables destinées à recevoir les valeurs des champs sont placées dans l'état \emph{invalide}. Par exemple :

\begin{galgascode}
@MaListeOrdonnee uneListe = ... ;
...
[!?uneListe popSmallest
  ?@char c
  ?@uint n
] ;
\end{galgascode}

Si \ggs+uneListe+ est vide, les variables \ggs+c+ et \ggs+n+ sont placées dans l'état \emph{invalide}.










\section{Méthodes}

\subsectionLabel{La méthode \texttt{greatest}}{methodeSortedListGreatest}

Cette méthode retourne les champs du plus grand élément de la liste ordonnée, sans le retirer. La liste n'est donc pas modifiée. Si elle est vide, un message d'erreur est affiché, et les variables destinées à recevoir les valeurs des champs sont placées dans l'état \emph{invalide}. Par exemple :

\begin{galgascode}
@MaListeOrdonnee uneListe = ... ;
...
[uneListe greatest
  ?@char c
  ?@uint n
] ;
\end{galgascode}

Si \ggs+uneListe+ est vide, les variables \ggs+c+ et \ggs+n+ sont placées dans l'état \emph{invalide}.


\subsectionLabel{La méthode \texttt{smallest}}{methodeSortedListSmallest}

Cette méthode retourne les champs du plus petit élément de la liste ordonnée, sans le retirer. La liste n'est donc pas modifiée. Si elle est vide, un message d'erreur est affiché, et les variables destinées à recevoir les valeurs des champs sont placées dans l'état \emph{invalide}. Par exemple :

\begin{galgascode}
@MaListeOrdonnee uneListe = ... ;
...
[uneListe smallest
  ?@char c
  ?@uint n
] ;
\end{galgascode}

Si \ggs+uneListe+ est vide, les variables \ggs+c+ et \ggs+n+ sont placées dans l'état \emph{invalide}.




\section{Énumération avec l'instruction \texttt{for}}

L'instruction \ggs+for+ (\refSectionPage{instructionFor}) permet d'énumérer les éléments d'une liste ordonnée, par ordre croissant ou décroissant.

Pour effectuer l'énumération par ordre croissant, écrire :
\begin{galgascode}
foreach uneListe do
  ...
end foreach ;
\end{galgascode}

Pour effectuer l'énumération par ordre décroissant, écrire :
\begin{galgascode}
foreach > uneListe do
  ...
end foreach ;
\end{galgascode}

À l'intérieur de la boucle, pour chaque champ des éléments de la liste, un constante dont le nom est celui du champ est définie et prend la valeur du champ correspondant de l'élément courant.

Par exemple :

\begin{galgascode}
@MaListeOrdonnee uneListe [emptySortedList] ;
uneListe += !'b' ! 1 ; # b1
uneListe += !'b' ! 2 ; # b2
uneListe += !'d' ! 1 ; # d1
uneListe += !'f' ! 1 ; # f1
uneListe += !'a' ! 1 ; # a1
uneListe += !'c' ! 1 ; # c1
uneListe += !'f' ! 2 ; # f2
@string s = "" ;
foreach uneListe do
  s .= [mCaractere string] . [mEntier string] . " " ;
end foreach ;
message s . "\n" ; # Affiche "a1 b2 b1 c1 d1 f2 f1"
s = "" ;
foreach > uneListe do
  s .= [mCaractere string] . [mEntier string] . " " ;
end foreach ;
message s . "\n" ; # Affiche "f1 f2 d1 c1 b1 b2 a1"
\end{galgascode}

  %!TEX encoding = UTF-8 Unicode
%!TEX root = ../galgas-book.tex

%--------------------------------------------------------------
\chapter{Le type \texttt{array}}
%-------------------------------------------------------------

Le type \emph{array} permet de réaliser des tableaux dont la dimension et le type de l'élément sont fixés à la compilation.

\section{Déclaration d'un type tableau}

La déclaration d'un type tableau contient les informations suivantes :
\begin{itemize}
  \item le type \galgas{@TypeElement} qui cite le type de l'élément de tableau ;
  \item la dimension du tableau, qui doit être un nombre entier strictement positif ;
  \item le type \galgas{@TypeTableau} qui est le nom donné au type de tableau.
\end{itemize}

La déclaration d'un type tableau a la syntaxe suivante :
\begin{lstlisting}[language=galgas]
array @TypeTableau : @TypeElement [dimension] ;
\end{lstlisting}

Par exemple :
\begin{lstlisting}[language=galgas]
array @monTableau : @string [3] ;
\end{lstlisting}


\section{Constructeur d'un type tableau}

Le seul constructeur d'un type tableau est le constructeur \galgas{new}. Il a pour but de fixer les dimensions initiales du tableau (il pourra ensuite être redimensionné). Il comporte \emph{dimension} arguments de type \galgas{@uint}, qui fixent la taille initiale de chaque axe.
Par exemple :
\begin{lstlisting}[language=galgas]
  @monTableau t [new !2 !3 !4] ;
\end{lstlisting}

Cette déclaration crée un tableau à $2*3*4$ éléments. Ces éléments sont par défaut \emph{invalides}, c'est à dire que leur lecture par le getter \galgas{valueAtIndex} déclenche une \emph{run-time error}. Pour être valide, un élément doit avoir été initialisé par un appel au setter \galgas{setValueAtIndex}.

Il est valide d'affecter la valeur $0$ à un ou plusieurs axes. Le tableau ne contient alors aucun élément.


\section{Accès à un élément}

L'accès à la valeur d'un élément s'effectue par le getter \galgas{valueAtIndex}. La modification de la valeur d'un élément est réalisée par le setter \galgas{setValueAtIndex} ou le setter \galgas{forceValueAtIndex}.

\subsection{Le getter \texttt{valueAtIndex}}

Ce getter comporte \emph{dimension} arguments de type \galgas{@uint}, qui précisent l'indice pour chaque axe. Les indices sont comptés à partir de zéro (comme en C).

Par exemple :
\begin{lstlisting}[language=galgas]
  @string s := [t valueAtIndex !1 !2 !2] ;
\end{lstlisting}


Une \emph{run-time error} est déclenchée si un indice dépasse sa borne correspondante, et la valeur retournée est \emph{invalide}. Si les indices ont des valeurs correctes, l'élément est retourné ; si cet élément est invalide, une \emph{run-time error} est déclenchée, et une valeur \emph{invalide} est retournée.






\subsection{Setter \texttt{setValueAtIndex}}

Ce setter comporte (\emph{dimension}+1) arguments :
\begin{itemize}
  \item le premier argument est type \galgas{@TypeElement}, et contient la valeur à écrire ;
  \item les \emph{dimension} suivants arguments sont de type \galgas{@uint} et précisent l'indice pour chaque axe.
\end{itemize} 
  
Les indices sont comptés à partir de zéro (comme en C). Une \emph{run-time error} est déclenchée si un indice dépasse sa borne correspondante, et le tableau est alors non modifié.

Par exemple :
\begin{lstlisting}[language=galgas]
  @string s := ... ;
  [!?t setValueAtIndex !s !1 !2 !2] ;
\end{lstlisting}





\subsection{Setter \texttt{forceValueAtIndex}}

Ce setter comporte (\emph{dimension}+1) arguments :
\begin{itemize}
  \item le premier argument est type \galgas{@TypeElement}, et contient la valeur à écrire ;
  \item les \emph{dimension} suivants arguments sont de type \galgas{@uint} et précisent l'indice pour chaque axe.
\end{itemize} 
  
Les indices sont comptés à partir de zéro (comme en C). Contrairement au setter \galgas{setValueAtIndex}, aucune \emph{run-time error} n'est déclenchée si un indice dépasse sa borne correspondante : le tableau est d'abord agrandi, ce qui ajoute des éléments invalides, puis l'élément désigné par les indices est affecté.

Par exemple :
\begin{lstlisting}[language=galgas]
  @string s := ... ;
  [}?t forceValueAtIndex !s !5 !4 !4] ;
\end{lstlisting}





\section{Validité d'un élément}

Le getter \galgas{isValueValidAtIndex} permet de savoir si un élément est valide ou non, c'est à dire si sa lecture déclenchera une \emph{run-time error}. Le setter \galgas{invalidateValueAtIndex} invalide un élément.

\subsection{Le getter \texttt{isValueValidAtIndex}}

Ce getter comporte \emph{dimension} arguments de type \galgas{@uint}, qui précisent l'indice pour chaque axe. Les indices sont comptés à partir de zéro (comme en C). Une \emph{run-time error} est déclenchée si un indice dépasse sa borne correspondante, et la valeur retournée est \emph{invalide}. Il renvoie une valeur de type \galgas{@bool}, suivant que l'élément est valide ou non.

Par exemple :
\begin{lstlisting}[language=galgas]
  @bool b := [t isValueValidAtIndex !1 !2 !2] ;
\end{lstlisting}


\subsection{Setter \texttt{invalidateValueAtIndex}}

Ce setter comporte \emph{dimension} arguments de type \galgas{@uint}, qui précisent l'indice pour chaque axe. Les indices sont comptés à partir de zéro (comme en C). Une \emph{run-time error} est déclenchée si un indice dépasse sa borne correspondante. Il invalide l'élément correspondant, c'est dire qu'un appel au getter \galgas{valueAtIndex} pour lire cet élément déclenchera une \emph{run-time error}.

Par exemple :
\begin{lstlisting}[language=galgas]
  [!?t invalidateValueAtIndex !1 !2 !2] ;
\end{lstlisting}





\section{Contrôle des tailles des axes}

Le getter \galgas{axisCount} renvoie la dimension d'un tableau, c'est à dire le nombre de ces axes, le getter \galgas{sizeForAxis} renvoie la taille allouée à un axe particulier. Les setters \galgas{setSizeForAxis} et \galgas{setSize} permettent de modifier la taille d'un tableau.



\subsection{Le getter \texttt{axisCount}}

Ce getter sans argument renvoie un \galgas{@uint} qui contient le nombre d'axes d'un tableau. Comme ce nombre est fixé statiquement par la déclaration de type, la valeur retournée est toujours la même, pour toutes les objets d'un même type tableau.


Par exemple, pour la déclaration :
\begin{lstlisting}[language=galgas]
array @monTableau : @string [3] ;
\end{lstlisting}
Pour tous les objets de type \galgas{@monTableau}, l'appel au getter \galgas{axisCount} renvoie la valeur $3$.


\subsection{Le getter \texttt{sizeForAxis}}

Ce getter présente un argument de type \galgas{@uint} qui est l'indice de l'axe interrogé. Les axes sont numérotés à partir de zéro, c'est à dire que le premier axe a l'indice $0$, le deuxième l'indice $1$, \dots~Une \emph{run-time error} est déclenchée si la valeur de l'argument est supérieure ou égale à la dimension du tableau, et la valeur renvoyée est invalide. Sinon, il renvoie un \galgas{@uint} qui contient la taille attribuée à l'axe correspondant.


\subsection{Le getter \texttt{rangeForAxis}}

Ce getter présente un argument de type \galgas{@uint} qui est l'indice de l'axe interrogé. Les axes sont numérotés à partir de zéro, c'est à dire que le premier axe a l'indice $0$, le deuxième l'indice $1$, \dots~Une \emph{run-time error} est déclenchée si la valeur de l'argument est supérieure ou égale à la dimension du tableau, et la valeur renvoyée est invalide. Sinon, il renvoie un \galgas{@range} qui commence à $0$ et qui a pour longueur la taille attribuée à l'axe correspondant.




\subsection{Setter \texttt{setSizeForAxis}}

Ce setter permet de changer la taille d'un axe sans changer les tailles attribuées aux autres axes. Il présente deux arguments de type \galgas{@uint} :
\begin{itemize}
  \item le premier est la nouvelle taille ;
  \item le second est l'indice de l'axe concerné.
\end{itemize}

Les axes sont numérotés à partir de zéro, c'est à dire que le premier axe a l'indice $0$, le deuxième l'indice $1$, \dots~Une \emph{run-time error} est déclenchée si la valeur de l'argument est supérieure ou égale à la dimension du tableau, et le tableau n'est pas modifié.
 
Diminuer la taille d'un axe fait disparaître des éléments, qui sont alors perdus. Si la nouvelle taille est zéro, le tableau est vidé de tous ses éléments.

Augmenter la taille fait apparaître de nouveaux éléments, qui sont invalides par défaut. Il faudra alors explicitement les initialiser individuellement par un appel au setter \galgas{setValueAtIndex}.




\subsection{Setter \texttt{setSize}}

Ce setter permet de changer les tailles de tous les axes. Il présente \galgas{@uint} arguments de type \galgas{@uint} qui contiennent les nouvelles tailles de chaque axe.

Diminuer la taille d'un axe fait disparaître des éléments, qui sont alors perdus. Si une des nouvelles tailles est zéro, le tableau est vidé de tous ses éléments.

Augmenter une taille fait apparaître de nouveaux éléments, qui sont invalides par défaut. Il faudra alors explicitement les initialiser individuellement par un appel au setter \galgas{setValueAtIndex}.


\section{Comparaison}

Un type tableau implémente les opérateurs \galgas{=} et \galgas{\!=}. L'égalité de deux tableaux est testé comme suit :
\begin{itemize}
  \item les tailles de chaque axe doivent être identiques ;
  \item les éléments doivent être identiques.
\end {itemize}

  %!TEX encoding = UTF-8 Unicode
%!TEX root = ../galgas-book.tex

%--------------------------------------------------------------
\chapter{Le type \texttt{class}}

\section{Déclaration d'une classe}

Voici différents exemples de déclaration de classes :

\begin{galgas}
abstract class @A {
  @uint mA ;
}
class @B extends @A {
  @string mB ;
}
class @C extends @B {
  @data mC ;
}
\end{galgas}

La classe \ggs+@A+ est abstraite (c'est-à-dire qu'elle ne peut pas être instanciée), la classe \ggs+@B+ hérite de \ggs+@A+. Une classe déclare zéro, un ou plusieurs attributs. L'héritage multiple n'est pas implémenté en GALGAS.

Une classe qui hérite d'une autre peut être abstraite :
\begin{galgas}
abstract class @D extends @C {
  ...
 }
\end{galgas}

Une classe non abstraite définit implicitement le constructeur \ggs+new+, et des \emph{getters} pour lire les attributs, et des \emph{setters} pour les écrire. On ne peut pas définir explicitement d'autres constructeurs, \emph{getters} ou \emph{setters} à l'intérieur de la classe. Cependant,  les extensions (\refChapterPage{extensions}) permettent de définir \emph{getters}, \emph{méthodes} et \emph{setters} associés à une classe.












\section{Le constructeur \texttt{new}}

Le constructeur \ggs+new+ est implicitement pour toute classe non abstraite (c'est à dire les classes \ggs+@B+ et \ggs+@C+). Ce constructeur présente un argument par attribut déclaré dans la classe instanciée et dans toutes les classes mère. L'ordre des arguments est celui obtenu en parcourant la hiérarchie de classes, en commençant par la classe racine. Par exemple on écrira :

\begin{galgas}
@B b [new
  !0 # Attribut mA de @A
  !"Hello" # Attribut mB de @B
] ;
@C c [new
  !0 # Attribut mA de @A
  !"Hello" # Attribut mB de @B
  ![@data emptyData] # Attribut mC de @C
] ;
\end{galgas}








\section{Lecture d'un attribut}

Par défaut, la lecture d'un attribut est activée par la définition implicite d'un \emph{getter}, dont le nom est le nom de l'attribut. Ainsi, pour une variable \ggs+b+ de type \ggs+@B+, on pourra écrire :

\begin{galgas}
@uint v = [b mA] ;
@string s = [b mB] ;
\end{galgas}

Il est possible d'inhiber la génération implicite d'un \emph{getter} de lecture d'un attribut en complétant sa déclaration par \ggs+feature %nogetter+, comme par exemple :

\begin{galgas}
abstract class @A {
  @uint mA feature nogetter ;
}
\end{galgas}

L'écriture \ggs+[b mA]+ sera alors rejetée par le compilateur.









\section{Écriture d'un attribut}

Par défaut, l'écriture d'un attribut n'est pas activée.

Pour activer la génération d'un \emph{setter} permettant décrire un attribut, compléter la déclaration de cet attribut par \ggs+feature %setter+. Un \emph{setter} est alors engendré, et porte le nom \texttt{set<Attribut>}, c'est à dire le nom de l'attribut avec sa première lettre en majuscule, précédé par \texttt{set}. Par exemple :

\begin{galgas}
abstract class @A {
  @uint mA feature setter ;
}
\end{galgas}


Pour modifier l'attribut \ggs+mA+, on écrira :

\begin{galgas}
[!?b setMA !12] ;
\end{galgas}

Si on veut à la fois inhiber la génération implicite d'un \emph{getter} de lecture d'un attribut et engendrer le \emph{setter} d'écriture, il suffit de déclarer l'attribut par :

\begin{galgas}
  @uint mA feature nogetter, setter ;
\end{galgas}

Ou encore :

\begin{galgas}
  @uint mA feature setter, nogetter ;
\end{galgas}












\section{Conversions entre objets de classes différentes}

Pour toute cette section, nous illustrons les constructions décrites en nous basant sur les trois variables suivantes :

\begin{galgas}
@A a ;
@B b = ... ;
@C c = ... ;
\end{galgas}

\subsection{Affectation polymorphique}

GALGAS accepte l'affectation polymorphique qui est par exemple \ggs+a = b+. Elle est autorisée aussi lors de l'affectation d'une expression effective à un paramètre formel dans une instruction d'appel (de routine, de fonction, de méthode, ...)

L'affectation polymorphique inverse (qui consisterait à écrire \ggs+b = a+) est logiquement refusée par le compilateur.

Il y a trois constructions qui permettent d'effectuer cette opération :
\begin{itemize}
  \item l'expression de conversion polymorphique inverse (\refSubsectionPage{expConversionPolymorphiqueInverse}) ;
  \item l'expression de test du type dynamique (\refSubsectionPage{testTypeDynamiqueExpression}) ;
  \item l'instruction \ggs+cast+ (\refSectionPage{instructionCast}).
\end{itemize}

Pour effectuer ponctuellement une affectation polymorphique inverse, on écrit (les parenthèses sont obligatoires) :

\begin{galgas}
@T resultat = (cast expression : @T) ;
\end{galgas}

Si le type dynamique de l'\ggs+expression+ est \ggs+@T+ ou une de ses classes héritières, l'expression de conversion polymorphique renvoie un objet de type \ggs+@T+ contenant la valeur de \ggs+expression+. Dans le cas contraire, un message d'erreur est affiché, et la variable \ggs+resultat+ est non construite.

L'exécution échoue donc avec émission de message d'erreur si la conversion n'est pas possible. 


Grâce à l'\emph{expression de test du type dynamique}, il est possible de tester si une conversion est possible. On peut donc écrire :

\begin{galgas}
if (expression is @B) then
  const @B variable = (cast expression : @B) ;
  ...
elsif (expression is @C) then
  const @C variable = (cast expression : @C) ;
  ...
else
  message "conversion impossible" ;
end if ;
\end{galgas}

L'instruction \ggs+cast+ permet simplement d'exprimer de manière plus élégante une série de test de conversion. La forme équivalent à l'instruction \ggs+if+ précédente est :

\begin{galgas}
cast expression
when >= @B variable :
  ...
when >= @C variable :
  ...
else
  message "conversion impossible" ;
end cast ;
\end{galgas}



  %!TEX encoding = UTF-8 Unicode
%!TEX root = ../galgas-book.tex

%--------------------------------------------------------------
\chapter{Le type \texttt{graph}}
%-------------------------------------------------------------

Le type \galgas{graph} permet de faire des opérations sur les graphes orientés.

Chaque nœud est identifié par un nom qui est une chaîne de caractères (de type \galgas{@string}), et est associé à une information utilisateur de type quelconque.

Un arc est identifié par un couple de nœuds.


Un type \galgas{graph} se déclare comme suit :
\lstset{emph={@nom_du_type_graph, @nom_liste_information}, emphstyle=\emph}
\begin{galgascode}
graph @nom_du_type_graph (@nom_liste_information) {
}
\end{galgascode}

Le nom \galgas{@nom_du_type_graph} est le nom donné au type. Le nom \galgas{@nom_liste_information} nomme un type qui spécifie l'information utilisateur associée à chaque nœud.

Attention, le type \galgas{@nom_liste_information} est un type \emph{liste}, et l'information utilisateur a pour type l'élement associé, c'est à dire \galgas{@nom_liste_information.element}. 

Par exemple, si l'on veut manipuler des graphes dont l'information associée est un entier \galgas{@uint}, on déclarera :
\begin{galgascode}
graph @monGraphe (@uintlist) {
}
\end{galgascode}

Si l'information associée est par exemple composée d'un entier et d'une chaîne de caractères, il faut déclarer un type liste :
\begin{galgascode}
list @maListe {
  @uint monInfo1 ;
  @string monInfo2 ;
}
graph @monGraphe (@maListe) {
}
\end{galgascode}






\section{Entrer les nœuds}



\section{Entrer les arcs}





  %!TEX encoding = UTF-8 Unicode
%!TEX root = ../galgas-book.tex

%--------------------------------------------------------------
\chapter{Le type \texttt{map}}
%-------------------------------------------------------------

Un objet de type \galgas{map} est une table de symboles, chaque symbole étant associé à des valeurs.

\section{Déclaration}

La déclaration d'un type \galgas{map} nomme :
\begin{itemize}
  \item les attributs qui sont associés à une clé ;
  \item les \emph{modifiers} d'insertion ;
  \item les \emph{méthodes} de recherche ;
  \item les \emph{modifiers} de retrait ;
\end{itemize}

Les clés sont déclarées implicitement et sont du \refTypePredefini{lstring}.

Par exemple :

\begin{galgascode}
map @MaTable {
  @string mPremier ;
  @bool mSecond ;
  insert insertKey error message "the '%K' key is already declared in %L";
  search searchKey error message "the '%K' key is not defined" ;
  remove removeKey error message "the '%K' key is not defined" ;
}
\end{galgascode}






\section{Modifiers d'insertion}

Une \galgas{map} peut déclarer zéro, un ou plusieurs \emph{modifiers} d'insertion. Un \emph{modifier} d'insertion permet d'insérer une nouvelle entrée à une table. Une erreur est déclenchée en cas de tentative d'une clé déjà existante.


Un \emph{modifier} d'insertion est déclaré par :

\lstset{emph={nom}, emphstyle=\emph}
\begin{galgascode}
insert nom error message "message_erreur" ;
\end{galgascode}

L'identificateur \galgas{nom} donne un nom au \emph{modifier} d'insertion ; ce nom doit être unique parmi les \emph{modifiers} d'insertion et de retrait. La chaîne de caractères \galgas{"message_erreur"} définit le message d'erreur qui est affiché en cas de tentative d'une clé déjà existante. Cette chaîne accepte deux séquences d'échappement :
\begin{itemize}
  \item \colorbox{\couleurCodeGALGAS}{\texttt{\%K}}, qui est remplacée par la chaîne de caractères de la clé existante ;
  \item \colorbox{\couleurCodeGALGAS}{\texttt{\%L}}, qui est remplacée par la chaîne décrivant la position de la clé existante dans les fichiers source.
\end{itemize}


Un \emph{modifier} d'insertion est appelé dans une \emph{instruction d'appel de modifier}, comprenant tous ses arguments en sortie :
\begin{itemize}
  \item le premier argument est une expression de type \galgas{@lstring} qui caractérise la clé à insérer ;
  \item ensuite, pour chaque attribut déclaré, une expression du type de cet attribut.
\end{itemize}

Par exemple :
\begin{galgascode}
@MaTable uneTable [emptyMap] ;
@lstring clef := ... ;
@string s := ... ;
@uint v := ... ;
[!?uneTable insertKey !clef !s !v] ;
\end{galgascode}











\section{Méthodes de recherche}

Une \galgas{map} peut déclarer zéro, une ou plusieurs \emph{méthodes} de recherche. Une \emph{méthode} de recherche permet de rechercher une entrée d'une table, et retourne la valeur de ses attributs associés. Une erreur est déclenchée si la clé n'existe pas.


Une \emph{méthode} de recherche est déclarée par :

\lstset{emph={nom}, emphstyle=\emph}
\begin{galgascode}
search nom error message "message_erreur" ;
\end{galgascode}

L'identificateur \galgas{nom} donne un nom à la \emph{méthode} de recherche ; ce nom doit être unique parmi ces \emph{méthodes}. La chaîne de caractères \galgas{"message_erreur"} définit le message d'erreur qui est affiché en cas de recherche d'une clé inexistante. Cette chaîne accepte une séquence d'échappement :
\begin{itemize}
  \item \colorbox{\couleurCodeGALGAS}{\texttt{\%K}}, qui est remplacée par la chaîne de caractères de la clé inexistante recherchée ;
\end{itemize}


Une \emph{méthode} de recherche est appelée dans une \emph{instruction d'appel de méthode} :
\begin{itemize}
  \item le premier argument (sortie) est une expression de type \galgas{@lstring} qui caractérise la clé à rechercher ;
  \item ensuite, pour chaque attribut déclaré, un argument en entrée nommant une variable destinée à recevoir la valeur de l'attribut correspondant.
\end{itemize}

Par exemple :
\begin{galgascode}
@MaTable uneTable [emptyMap] ;
...
@lstring clef := ... ;
[!?uneTable searchKey !clef ?@string s ?@uint v] ;
\end{galgascode}













\section{Modifiers de retrait}

Une \galgas{map} peut déclarer zéro, un ou plusieurs \emph{modifiers} de retrait. Un \emph{modifier} de recherche permet de retirer une entrée d'une table, et retourne la valeur des attributs de la clé retirée. Une erreur est déclenchée si la clé n'existe pas.


Un \emph{modifier} de retrait est déclaré par :

\lstset{emph={nom}, emphstyle=\emph}
\begin{galgascode}
remove nom error message "message_erreur" ;
\end{galgascode}

L'identificateur \galgas{nom} donne un nom au \emph{modifier} de retrait ; ce nom doit être unique parmi les \emph{modifiers} d'insertion et de retrait. La chaîne de caractères \galgas{"message_erreur"} définit le message d'erreur qui est affiché en cas de recherche d'une clé inexistante. Cette chaîne accepte une séquence d'échappement :
\begin{itemize}
  \item \galgas{\%K}, qui est remplacée par la chaîne de caractères de la clé inexistante à retirer ;
\end{itemize}


Un \emph{modifier} de retrait est appelé dans une \emph{instruction d'appel de modifier} :
\begin{itemize}
  \item le premier argument (sortie) est une expression de type \galgas{@lstring} qui caractérise la clé à retirer ;
  \item ensuite, pour chaque attribut déclaré, un argument en entrée nommant une variable destinée à recevoir la valeur de l'attribut correspondant de la clé retirée.
\end{itemize}

Par exemple :
\begin{galgascode}
@MaTable uneTable [emptyMap] ;
...
@lstring clef := ... ;
[!?uneTable removeKey !clef ?@string s ?@uint v] ;
\end{galgascode}






\section{Constructeurs}

\subsection{Constructeur \texttt{emptyMap}}

\begin{galgascode}
constructor @T emptyMap -> @T ;
\end{galgascode}

Ce constructeur permet d'instancier une table vide. Exemple :
\begin{galgascode}
@MaTable uneTable [emptyMap] ;
\end{galgascode}

 

\subsection{Constructeur \texttt{mapWithMapToOverride}}

\begin{galgascode}
constructor @T mapWithMapToOverride ?@T inMapToOverride -> @T ;
\end{galgascode}

Ce constructeur permet d'instancier une table vide, qui surcharge la table \galgas{inMapToOverride} citée en argument. Exemple :
\begin{galgascode}
@MaTable uneTable [emptyMap] ;
@MaTable autreTableTable [inMapToOverride !uneTable] ;
\end{galgascode}

\section{Readers}

%\subsection{Le reader \texttt{allKeyList}}
%
%\begin{galgascode}
%reader @T allKeyList -> @lstringlist ;
%\end{galgascode}
%
%Le \emph{reader} \galgas{allKeyList} retourne la liste construite avec toutes les clés du receveur, dans la table de premier niveau et dans les tables surchargées. L'ordre de la liste est :
%\begin{itemize}
%  \item d'abord les clés de la table de premier niveau, puis celles des tables surchargées, dans l'ordre de la surcharge ;
%  \itel pour chaque table, les clés apparaissent dans l'ordre alphabétique croissant.
%\end{itemize}

\subsection{Le reader \texttt{count}}

\begin{galgascode}
reader @T count -> @uint ;
\end{galgascode}


Le \emph{reader} \galgas{count} retourne un \galgas{@uint} qui contient le nombre d'entrées de la table de premier niveau du receveur.



\subsection{Le reader \texttt{hasKey}}

\begin{galgascode}
reader @T hasKey ??@string inKey -> @bool ;
\end{galgascode}


Le \emph{reader} \galgas{hasKey} retourne un \galgas{@bool} qui est \galgas{true} si la clé \galgas{inKey} est dans la table de premier niveau du receveur, \galgas{false} dans le cas contraire.



\subsection{Le reader \texttt{keyList}}

\begin{galgascode}
reader @T keyList -> @lstringlist ;
\end{galgascode}


Le \emph{reader} \galgas{keyList} retourne la liste construite avec toutes les clés de la table de premier niveau du receveur. L'ordre de la liste est l'ordre alphabétique croissant des clés.



\subsection{Le reader \texttt{keySet}}

\begin{galgascode}
reader @T keySet -> @stringset ;
\end{galgascode}


Le \emph{reader} \galgas{keySet} retourne l'ensemble de toutes les clés de la table de premier niveau du receveur.





\subsection{Le reader \texttt{locationForKey}}

\begin{galgascode}
reader @T locationForKey ??@string inKey -> @location ;
\end{galgascode}


Le \emph{reader} \galgas{locationForKey} retourne un \galgas{@location} qui contient l'information de position de la clé \galgas{inKey} dans la table de premier niveau du receveur. Une erreur d'exécution est déclenchée si cette clé n'existe pas.








\subsection{Le reader \texttt{overriddenMap}}

\begin{galgascode}
reader @T overriddenMap -> @T ;
\end{galgascode}


Le \emph{reader} \galgas{overriddenMap} retourne la table obtenue en amputant de la valeur du receveur la table de premier niveau. Si le receveur n'a pas de table surchargée, une erreur d'exécution est déclenchée.





\section{Énumération}

L'instruction \galgas{foreach} permet d'énumérer des objets de type \galgas{map}. Uniquement la table de premier niveau est énumérée. Par défaut, l'énumération s'effectue dans l'ordre croissant des clés. Pour énumérer dans l'ordre décroissant, utiliser le qualifier \galgas{>}.

À l'intérieur du coprs de la boucle, sont implicitement définies :
\begin{itemize}
  \item la constante \galgas{lkey}, de type \galgas{@lstring}, qui a pour valeur la clé de l'entrée courante ;
  \item pour chaque attribut, une constante du type de l'attribut, et portant le nom de cet attribut, qui a pour valeur la valeur de cet attribut de l'entrée courante.
\end{itemize}

Par exemple :
\begin{galgascode}
@MaTable uneTable [emptyMap] ;
[!?uneTable insertKey ![@lstring new !"z" !here] !"world" !5] ;
[!?uneTable insertKey ![@lstring new !"a" !here] !"hello" !10] ;
foreach aMap do
  message lkey->string . " " . mPremier . " " . mSecond . "\n" ;
end foreach ;
\end{galgascode}

L'affichage produit est :

\begin{galgascode}
a hello 10
z world 5
\end{galgascode}

%====== Setting an attribute of an entry ======
%
%^Available in GALGAS 1.8.4 and later. ^
%
%Given a key, you can directly set an attribute of a map entry.
%
%In order to enable this feature, you have to associate the setter feature to the given attribute.
%
%map @MaTable {
%  @string mPremier ;
%  @bool mSecond feature setter ;
%}
%
%For every attribute declared with this feature, a modifier is available; its name is build by the concatenation of three patterns:
%  - the string set;
%  - the attribute name, with the first letter capitalized;
%  - the string ForKey.
%
%So, for the mSecond attribute, the associated modifier name is setmSecondForKey.
%
%The modifier has two input arguments:
%  - the key, an @string expression;
%  - the value to set to the attribute.
%
%For example:
%
%@MaTable aMap := ... ;
%@string s := ... ;
%@bool v := ... ;
%[!?aMap setmSecondForKey !s !v] ;
%
%A run-time error is raised if the key value does not exist in the map.
%
%====== Using the with instruction on a map object ======
%
%^Available in GALGAS 1.8.4 and later. ^
%
%The with instruction enables a direct access on all attributes of an entry. Its syntax is:
%
%with //prefix// !?//map_object// //search_method// !//key_expression// do
%  ...
%else
%  ...
%end with ;
%
%The //prefix// and the else part are optional.
%
%There are two different behaviours, depending from the //search_method//:
%  * the //search_method// is one declared in the search declarations;
%  * the //search_method// is the predefined hasKey identifier.
%
%===== with instruction naming a declared search method =====
%
%The //key_expression// should be an @lstring expression.
%
%Given this map type declaration:
%
%map @MaTable {
%  @string mPremier ;
%  @bool mSecond ;
%  search searchKey error message "the %%'%K'%% key is not defined" ;
%  ...
%}
%
%You can write:
%
%@MaTable aMap := ... ;
%@lstring aKey := ... ;
%with !?aMap searchKey !aKey do
%  # ... 
%else
%  # ...
%end with ;
%
%The aMap object is accessed in read/write mode.
%
%If the aKey object value does not correspond to an existing entry, an error message is displayed and the else part is executed. The error message is based upon the search method name, here searchKey. The error location is given by aKey location.
%
%If the aKey object value corresponds to an existing entry, the do part is executed. The entry's attributes can be fully accessed in this part (you can read, write or modify them), using directly their names. The entry's key (an @lstring object) can be accessed in read mode (you cannot modify it), using the key identifier. For example:
%
%
%@MaTable aMap := ... ;
%@lstring aKey := ... ;
%with !?aMap searchKey !aKey do
%  mPremier .= %%"xyz"%% ; # mPremier is accessed in read/write mode
%  mSecond := true ; # mSecond is accessed in write mode
%  log key ; # key is can only be accessed in read mode
%end with ;
%
%The only constraint is that all attributes should be valuated at the end of the do part.
%
%===== with instruction naming the predefined hasKey method =====
%
%The //key_expression// should be an @string expression.
%
%Given this map type declaration:
%
%map @MaTable {
%  @string mPremier ;
%  @bool mSecond ;
%  ...
%}
%
%You can write:
%
%@MaTable aMap := ... ;
%@string aKey := ... ;
%with !?aMap hasKey !aKey do
%  # ... 
%else
%  # ...
%end with ;
%
%The aMap object is accessed in read/write mode.
%
%If the aKey object value does not correspond to an existing entry, no error message is displayed and the else part is executed.
%
%If the aKey object value corresponds to an existing entry, the do part is executed.  The entry's attributes can be fully accessed in this part (you can read, write or modify them), using directly their names, the entry's key (an @lstring object) can be accessed in read mode (you cannot modify it), using the key identifier, exactly as in the previously described behaviour.
%
%===== Using a prefix =====
%
%This feature enables to prepend with a prefix the names used for accessing the attributes and the key in the do part. It could be useful for avoiding name conflicts.
%
%It is available for both search methods.
%
%Given this map type declaration:
%
%map @MaTable {
%  @string mPremier ;
%  @bool mSecond ;
%  ...
%}
%
%You can write:
%
%@MaTable aMap := ... ;
%@string aKey := ... ;
%with xyz_ : !?aMap hasKey !aKey do
%  xyz_mPremier .= %%"xyz"%% ; # mPremier is accessed in read/write mode
%  xyz_mSecond := true ; # mSecond is accessed in write mode
%  log xyz_key ; # key is can only be accessed in read mode
%end with ;
%
%Using the xyz_ prefix, all attributes and the key should be accessed using this prefix. 
  %!TEX encoding = UTF-8 Unicode
%!TEX root = ../galgas-book.tex

%--------------------------------------------------------------
\chapter{Map Proxy Type}
%-------------------------------------------------------------

  %!TEX encoding = UTF-8 Unicode
%!TEX root = ../galgas-book.tex

%--------------------------------------------------------------
\chapterLabel{Le type structure}{typeStructure}
%-------------------------------------------------------------

\tableDesMatieresLocaleDeProfondeurRelative{1}



Le mot-clé \ggst!struct! permet de définir des types de structure. Un objet de type structure a une sémantique de valeur. Une déclaration de structure doit déclarer au moins une propriété. Par exemple :

\begin{galgas34}
struct @MaStructure {
  public var @uint propriété
}
\end{galgas34}

Il est possible d'associer une valeur initiale à la déclaration d'une propriété~:
\begin{galgas34}
struct @MaStructure2 {
  public var @uint propriété = 9
}
\end{galgas34}

En GALGAS 3, les propriétés non initialisées peuvent être déclarées avec \ggst!public! et \ggst!var! implicites (cette syntaxe est obsolète et n'existe pas en GALGAS 4)~:
\begin{galgas3}
struct @MaStructure {
  @uint propriété
}
\end{galgas3}

Une déclaration de structure peut aussi déclarer~: des initialisateurs (\refSectionPage{initStruct}), des \emph{getters} (\refSectionPage{getterStruct}), des \emph{methodes} (\refSectionPage{methodStruct}) et des \emph{setters} (\refSectionPage{setterStruct}).

Ces déclarations peuvent apparaître soit dans la déclaration de structure, soit comme une extension (\refChapterPage{chapitreExtensions}).












\sectionLabel{initialisateurs}{initStruct}

Lorque que l'on instancie un type structure, on appelle un \emph{initialisateur}. Celui-ci a pour rôle de fixer une valeur initiale à toutes les propriétés de la structure instancée.

En GALGAS 3, on peut instancier une structure avec le constructeur \ggst!new! (\refSubsectionPage{constructeurNewStruct}), qui est automatiquement engendré par toute structure~; cette construction est obsolète et est remplacée par l'appel d'un initialisateur.

Toute structure implémente un initialisateur. Si une structure ne déclare aucun initialisateur, un initialisateur par défaut est automatiquement engendré (\refSubsectionPage{initialisateurDefautStruct}). L'écriture d'un initialisateur est présenté à la \refSubsectionPage{initialisateurStruct}.

\subsectionLabel{Initialisateur synthétisé}{initialisateurDefautStruct}

L'appel de l'initialisateur synthétisé comprend une valeur par propriété non initialisée déclarée par le type structure.

Par exemple, pour la déclaration~:
\begin{galgas34}
struct @maStructure {
  public var @uint propriété1
  public var @bool propriété2
}
\end{galgas34}

La syntaxe la plus générale d'appel de l'initialisateur synthétisé est~:
\begin{galgas4}
var aVariable = @maStructure.init (!propriété1: 123, !propriété2: true)
\end{galgas4}
\begin{galgas3}
var aVariable = @maStructure.init {!propriété1: 123 !propriété2: true}
\end{galgas3}

On peut omettre \ggst!.init!~:
\begin{galgas4}
var aVariable = @maStructure (!propriété1: 123, !propriété2: true)
\end{galgas4}
\begin{galgas3}
var aVariable = @maStructure {!propriété1: 123 !propriété2: true}
\end{galgas3}


Si le contexte le permet, l'annotation de type peut être omis lors de l'appel de l'initialisateur~:
\begin{galgas4}
var @maStructure aVariable = .init (!propriété1: 123, !propriété2: true)
\end{galgas4}
\begin{galgas3}
var @maStructure aVariable = .init {!propriété1: 123 !propriété2: true}
\end{galgas3}


Il est possible d'ajouter l'attribut \ggst=%noArgumentLabel= à la déclaration d'une propriété non initialisée, pour supprimer dans l'appel de l'initialisateur synthétisé l'étiquette d'argument pour cette propriété. Par exemple, si on déclare~:
\begin{galgas3}
struct @maStructure {
  public var @uint propriété1 %noArgumentLabel
  public var @bool propriété2
}
\end{galgas3}

Alors l'appel de l'initialisateur synthétisé devient~:
\begin{galgas4}
var aVariable = @maStructure.init (!123, !propriété2: true)
// Ou bien, en éliminant l'argument d'étiquette vide :
var aVariable = @maStructure.init (123, !propriété2: true)
\end{galgas4}
\begin{galgas3}
var aVariable = @maStructure.init {!123 !propriété2: true}
\end{galgas3}

On peut omettre \ggst!.init! ou l'annotation de type si le contexte le permet.

Si la propriété est initialisée, alors \ggst=%noArgumentLabel= est invalide (déclenche une erreur de syntaxe).





\subsectionLabel{Initialisateur}{initialisateurStruct}



\subsectionLabel{GALGAS 3 : constructeur \texttt{new}}{constructeurNewStruct}

{\bf Cette construction est obsolète et est remplacée par l'appel d'un initialisateur.}

En GALGAS 3, tout type structure définit implicitement le constructeur \ggst!new!. Son appel comprend une valeur par propriété non initialisée déclarée par le type structure.

Par exemple, pour la déclaration~:
\begin{galgas3}
struct @maStructure {
  public var @uint propriété1
  @bool propriété2 // Syntaxe obsolète, autorisée en GALGAS 3
}
\end{galgas3}

L'appel du constructeur \ggst!new! est~:
\begin{galgas3}
var aVariable = @maStructure.new {!123 !true}
\end{galgas3}

Si le contexte le permet, l'annotation de type peut être omis lors de l'appel du constructeur~:
\begin{galgas3}
var @maStructure aVariable = .new {!123 !true}
\end{galgas3}

Il est possible d'ajouter l'attribut \ggst=%selector= à la déclaration d'une propriété de structure. Le faire impose d'utiliser le sélecteur portant le nom de la propriété dans l'appel du constructeur \ggst=new=. Par exemple, si on déclare~:
\begin{galgas3}
struct @maStructure {
  public var @uint propriété1 %selector
  @bool propriété2 // Syntaxe obsolète, autorisée en GALGAS 3
}
\end{galgas3}

Alors l'appel du constructeur \ggst!new! devient~:
\begin{galgas3}
var aVariable = @maStructure.new {!propriété1: 123 !true}
\end{galgas3}


%\subsection{Constructeur \texttt{default}}
%
%Si chacune des propriétés accepte le constructeur par défaut, alors le type structure accepte le constructeur pas défaut. C'est le cas de la structure \ggst!@maStructure! définie au dessus~: \ggst!@uint! accepte le constructeur par défaut (initialisation à \ggst!0!), ainsi que \ggst!@bool! (initialisation à \ggst!false!). Donc~:
%\begin{galgas3}
%var aVariable = @maStructure.default
%\end{galgas3}
%Initialise les propriétés de \ggst!aVariable! respectivement à \ggst!0! et \ggst!false!. On peut aussi écrire~:
%\begin{galgas3}
%@maStructure aVariable = .default
%\end{galgas3}


\section{Accès aux propriétés}

La notation pointée \ggst!variable.propriété! permet d'accéder à une propriété d'une structure, aussi bien en lecture, en écriture et en lecture/écriture.

Exemple d'accès en lecture~:
\begin{galgas3}
@uint v = aVariable.mProp1
\end{galgas3}

Exemple d'accès en écriture~:
\begin{galgas3}
aVariable.mProp1 = 10
\end{galgas3}


Exemple d'accès en lecture/écriture~:
\begin{galgas3}
aVariable.mProp1 ++
\end{galgas3}





\sectionLabel{Getters}{getterStruct}

Un type structure définit un \emph{getter} sans argument par propriété, qui permet d'accéder en lecture à cette propriété. Son nom est celui de la propriété. Par exemple, à la place de~:
\begin{galgas3}
@uint v = aVariable.mProp1
\end{galgas3}

On peut écrire~:
\begin{galgas3}
@uint v = [aVariable mProp1]
\end{galgas3}



\sectionLabel{Méthodes}{methodStruct}



\sectionLabel{Setters}{setterStruct}


\section{Types structure prédéfinis}

Plusieurs types préféfinis GALGAS sont des structures.

\subsectionTypePredefiniLabelIndex{lbigint}

\begin{galgas3}
struct @lbigint {
  @bigint bigint
  @location location
}
\end{galgas3}



\subsectionTypePredefiniLabelIndex{lbool}

\begin{galgas3}
struct @lbool {
  @bool bool
  @location location
}
\end{galgas3}



\subsectionTypePredefiniLabelIndex{lchar}

\begin{galgas3}
struct @lchar {
  @char char
  @location location
}
\end{galgas3}


\subsectionTypePredefiniLabelIndex{ldouble}

\begin{galgas3}
struct @ldouble {
  @double double
  @location location
}
\end{galgas3}







\subsectionTypePredefiniLabelIndex{lsint}

\begin{galgas3}
struct @lsint {
  @sint sint
  @location location
}
\end{galgas3}








\subsectionTypePredefiniLabelIndex{lsint64}

\begin{galgas3}
struct @lsint64 {
  @sint64 sint64
  @location location
}
\end{galgas3}







\subsectionTypePredefiniLabelIndex{lstring}

\begin{galgas3}
struct @lstring {
  @string string
  @location location
}
\end{galgas3}







\subsectionTypePredefiniLabelIndex{luint}

\begin{galgas3}
struct @luint {
  @uint uint
  @location location
}
\end{galgas3}





\subsectionTypePredefiniLabelIndex{luint64}


\begin{galgas3}
struct @luint64 {
  @uint64 uint64
  @location location
}
\end{galgas3}


\subsectionTypePredefiniLabelIndex{range}

Le type \ggst!@range! définit les intervalles d'entiers non signés 32 bits (\ggst!@uint!).

\begin{galgas3}
struct @range {
  @uint start
  @uint length
}
\end{galgas3}

La plupart des propriétés du type \ggst!@range! découle de cette définition (\refChapterPage{typeStructure}).

\ggst+@range.new {!a !b}+, où \ggst!a! et \ggst!b! sont des expressions de type \ggst!@uint!, représente~:
\begin{itemize}
  \item un intervalle vide si \ggst!b! est égal à zéro ;
  \item l'intervalle $[a, a+b-1]$ si \ggst!b! est strictement positif.
\end{itemize}



\subsubsectionLabel{Opérateurs \texttt{...} et \texttt{..<}}{operateurIntervalleRange}

Deux opérateurs permettent de construire plus facilement des objets de type \ggst!@range!.

L'opérateur \ggst!...! permet de définir un intervalle fermé à partir de sa borne inférieure et de sa borne supérieure~: si \ggst!a! et \ggst!b! sont des expressions de type \ggst!@uint!, l'expression \ggst!a ... b! est équivalente à la construction \ggst*@range.new {!a !b - a + 1}*. Une exception est levée si $b < a$.

L'opérateur \ggst!..<! permet de définir un intervalle ouvert à gauche à partir de sa borne inférieure et de sa borne supérieure~: si \ggst!a! et \ggst!b! sont des expressions de type \ggst!@uint!, l'expression \ggst!a ..< b! est équivalente à \ggst*@range.new {!a !b - a}*. Une exception est levée si $b < a$.

\subsubsection{Type \texttt{@range} et instruction \texttt{for}}

On peut utiliser une expression de type \ggst!@range! avec l'instruction \ggst!for!~:

\begin{galgas3}
for i in @range.new {!12 !5} do
  # i prend successivement les valeurs 12, 13, 14, 15, 16
end
\end{galgas3}

Et, avec l'opérateur \ggst!...!~:
\begin{galgas3}
for i in 12 ... 16 do
  # i prend successivement les valeurs 12, 13, 14, 15, 16
end
\end{galgas3}

Et l'opérateur \ggst!..<!~:
\begin{galgas3}
for i in 12 ..< 17 do
  # i prend successivement les valeurs 12, 13, 14, 15, 16
end
\end{galgas3}

Si l'on veut parcourir l'énumération à partir de la dernière valeur, on utilise le modificateur \ggst!>! après le mot-clé \ggst!for!~:
\begin{galgas3}
for > i in @range.new {!12 !5} do
  # i prend successivement les valeurs 16, 15, 14, 13, 12
end
\end{galgas3}

\begin{galgas3}
for > i in 12 ... 16 do
  # i prend successivement les valeurs 16, 15, 14, 13, 12
end
\end{galgas3}

\begin{galgas3}
for > i in 12 ..< 17 do
  # i prend successivement les valeurs 16, 15, 14, 13, 12
end
\end{galgas3}



  %!TEX encoding = UTF-8 Unicode
%!TEX root = ../galgas-book.tex

%--------------------------------------------------------------
\chapter{Predefined structure types}
%-------------------------------------------------------------

The following types are predefined, as particular structure types.

\sectionTypePredefiniLabelIndex{lbool}

The \galgas{@lbool} is predefined as:
\begin{lstlisting}[language=galgas]
struct @lbool {
  @bool bool ;
  @location location ;
}
\end{lstlisting}




\sectionTypePredefiniLabelIndex{lchar}

The \galgas{@lchar} is predefined as:
\begin{lstlisting}[language=galgas]
struct @lchar {
  @char char ;
  @location location ;
}
\end{lstlisting}







\sectionTypePredefiniLabelIndex{ldouble}

The \galgas{@ldouble} is predefined as:
\begin{lstlisting}[language=galgas]
struct @ldouble {
  @double double ;
  @location location ;
}
\end{lstlisting}







\sectionTypePredefiniLabelIndex{lsint}

The \galgas{@lsint} is predefined as:
\begin{lstlisting}[language=galgas]
struct @lsint {
  @sint sint ;
  @location location ;
}
\end{lstlisting}








\sectionTypePredefiniLabelIndex{lsint64}

The \galgas{@lsint64} is predefined as:
\begin{lstlisting}[language=galgas]
struct @lsint64 {
  @sint64 sint64 ;
  @location location ;
}
\end{lstlisting}







\sectionTypePredefiniLabelIndex{lstring}

The \galgas{@lstring} is predefined as:
\begin{lstlisting}[language=galgas]
struct @lstring {
  @string string ;
  @location location ;
}
\end{lstlisting}







\sectionTypePredefiniLabelIndex{luint}

The \galgas{@luint} is predefined as:
\begin{lstlisting}[language=galgas]
struct @luint {
  @uint uint ;
  @location location ;
}
\end{lstlisting}





\sectionTypePredefiniLabelIndex{luint64}

The \galgas{@luint64} is predefined as:
\begin{lstlisting}[language=galgas]
struct @luint64 {
  @uint64 uint64 ;
  @location location ;
}
\end{lstlisting}


\sectionTypePredefiniLabelIndex{range}

The \galgas{@range} is equivalent to the declaration:
\begin{lstlisting}[language=galgas]
struct @range {
  @uint start ;
  @uint length ;
}
\end{lstlisting}




\part{Sous-programmes}

%!TEX encoding = UTF-8 Unicode
%!TEX root = ../galgas-book.tex

%--------------------------------------------------------------
\chapter{Categories}\index{Categories}
%-------------------------------------------------------------

\emph{Categories} are the way for adding \emph{readers}, \emph{methods} and \emph{modifiers} to classes. They are defined outside class definitions.

\part{Instructions et expressions}

%!TEX encoding = UTF-8 Unicode
%!TEX root = ../galgas-book.tex

%--------------------------------------------------------------
\chapter{Semantic Instructions}
%-------------------------------------------------------------



\sectionLabel{Append Instruction}{appendInstruction}


\sectionLabel{Assignment Instruction}{assignmentInstruction}


\section{Cast Instruction}


\sectionLabel{Concat Instruction}{concatInstruction}


\sectionLabel{Decrement Instruction}{decrementInstruction}




\section{Drop Instruction}

{\lstset{emph={variable}, emphstyle=\emph}
\begin{galgascode}
drop variable, ... ;
\end{galgascode}
}

\section{Error Instruction}


\section{Extern Action Call Instruction}




\section{L'instruction \texttt{for}}





\sectionLabel{L'instruction \texttt{foreach}}{instructionForeach}




\sectionLabel{Increment Instruction}{incrementInstruction}










\section{If Instruction}


\subsection{Syntax}

The \emph{if} instruction has the following syntax:
{\lstset{emph={expression, instructions, expression2, instructions2, else_instructions}, emphstyle=\emph}
\begin{galgascode}
if expression then
  instructions
elsif expression2 then
  instructions2
...
else
  else_instructions
end if ;  
\end{galgascode}
}

More precisely, it contains :
\begin{itemize}
\item zero, one or more \emph{elsif} branches ;
\item zero or one \emph{else} branch.
\end{itemize}


\subsection{Static semantics}


No \emph{else} branch is equivalent to an \emph{else} branch without any instruction.


The \emph{elsif} branches are just syntactic sugar: it is semantically equivalent to use embedded \emph{if} instructions instead. For example:
{\lstset{emph={expression, instructions, expression2, instructions2, else_instructions}, emphstyle=\emph}
\begin{galgascode}
if expression then
  instructions
elsif expression2 then
  instructions2
else
  else_instructions
end if ;  
\end{galgascode}
}
is equivalent to:
{\lstset{emph={expression, instructions, expression2, instructions2, else_instructions}, emphstyle=\emph}
\begin{galgascode}
if expression then
  instructions
else
  if expression2 then
    instructions2
  else
    else_instructions
  end if ;  
end if ;  
\end{galgascode}
}

So, for describing \emph{if} instruction static and dynamic semantics, we only need to describe an \emph{if} instruction without any \emph{elsif} branch and with an \emph{else} branch:
{\lstset{emph={expression, instructions, else_instructions}, emphstyle=\emph}
\begin{galgascode}
if expression then
  instructions
else
  else_instructions
end if ;
\end{galgascode}
}

The static semantics evaluates the \emph{expression} type, and applies the following rules until success:
\begin{enumerate}
\item the \emph{expression} type is \refTypePredefini{bool};
\item the \emph{expression} type is an \emph{structure} type, it has a attribute named \emph{bool}, whose type is \refTypePredefini{bool};
\item the \emph{expression} type has a reader without any argument named \emph{bool} that returns a \refTypePredefini{bool} value.
\end{enumerate}

Most expressions you write fall in the first case.

Applying the second rule enables to use an \refTypePredefini{lbool} expression as an \emph{if} expression. For example:
{\lstset{emph={expression, instructions, else_instructions}, emphstyle=\emph}
\begin{galgascode}
@lbool var := ... ;
if var then
  instructions
else
  else_instructions
end if ;
\end{galgascode}
}

The \emph{var} object belongs to the \refTypePredefini{lbool} type: so first rule fails. But \refTypePredefini{lbool} is a \emph{structure} type, it has a \emph{bool} attribute with the \refTypePredefini{bool} type, so the second rule succeeds. It is semantically equivalent to write:
{\lstset{emph={expression, instructions, else_instructions}, emphstyle=\emph}
\begin{galgascode}
@lbool var := ... ;
if var->bool then
  instructions
else
  else_instructions
end if ;
\end{galgascode}
}

The third rule applies on a \emph{class} type that defines a category reader with argument named \emph{bool} that returns a \refTypePredefini{bool} type. For example, declaring:
\begin{galgascode}
class @myClass { ... }

reader @myClass bool -> @bool outResult : ... end reader ;
\end{galgascode}

enables to write:
{\lstset{emph={expression, instructions, else_instructions}, emphstyle=\emph}
\begin{galgascode}
@myClass myObject := ... ;
if myObject then
  instructions
else
  else_instructions
end if ;
\end{galgascode}
}

It is semantically equivalent to write:
{\lstset{emph={expression, instructions, else_instructions}, emphstyle=\emph}
\begin{galgascode}
@myClass myObject := ... ;
if [myObject bool] then
  instructions
else
  else_instructions
end if ;
\end{galgascode}
}


\subsection{Dynamic semantics}

According to the preceding section, we only need to describe the dynamic semantic of an \emph{if} instruction without any \emph{elsif} branch and with an \emph{else} branch:
{\lstset{emph={expression, instructions, else_instructions}, emphstyle=\emph}
\begin{galgascode}
if expression then
  instructions
else
  else_instructions
end if ;  
\end{galgascode}
}



The \emph{expression} is first computed :
\begin{itemize}
\item if the evaluation fails, neither the \emph{if} instructions, neither the \emph{else} instructions are executed;
\item if the evaluation result is \emph{true}, the \emph{if} instructions are executed ;
\item if the evaluation result is \emph{false}, the \emph{else} instructions are executed.
\end{itemize}


\section{Grammar Instruction}

\section{Local Variable Declaration Instruction}


{\lstset{emph={variable}, emphstyle=\emph}
\begin{galgascode}
@type variable ;
\end{galgascode}
}

{\lstset{emph={variable, expression}, emphstyle=\emph}
\begin{galgascode}
@type variable := expression ;
\end{galgascode}
}

{\lstset{emph={variable, constructor, arguments}, emphstyle=\emph}
\begin{galgascode}
@type variable [constructor arguments] ;
\end{galgascode}
}


\section{Local Constant Declaration Instruction}




\sectionLabel{L'instruction \texttt{log}}{instructionLog}




\section{Loop Instruction}


\subsection{Syntax}

The \emph{loop} instruction has the following syntax:
{\lstset{emph={expression, instructions_1, instructions_2, variant_expression}, emphstyle=\emph}
\begin{galgascode}
loop variant_expression
: instructions_1
while expression do
  instructions_2
end loop ;  
\end{galgascode}
}

The \emph{instructions\_1} and \emph{instructions\_2} are possibly empty instruction lists. If the \emph{instructions\_1} is empty, the preceeding « : » can be omitted :
{\lstset{emph={expression, instructions_1, instructions_2, variant_expression}, emphstyle=\emph}
\begin{galgascode}
loop variant_expression
while expression do
  instructions_2
end loop ;  
\end{galgascode}
}

\subsection{Semantics}

The \emph{variant\_expression} is an \galgas{@uint} expression that ensures the loop is not endless: it is computed at the beginning of the loop execution, and is decremented by one at the end of every iteration. If it reaches zero, a run-time error is raised.

The \emph{expression} is an \galgas{@bool} expression that expresses repetitive execution.

The \emph{loop} instruction execution is illustrated by the flowchart given in \refFigure{}{loopInstructionFlowchart}.

\begin{figure}[ht]
  \centering
  \small
  \begin{tikzpicture}[very thick]
    \node [rounded corners=5pt, shape=rectangle, draw] (start) {\textsc{begin}} ;
    \node [shape=rectangle, draw] (variant) [below=of start] {$variant := variant\_expression~value$} ;
    \node [shape=diamond, draw] (premierTest) [below=of variant] {$variant > 0$} ;
    \node [shape=rectangle, draw] (error1) [right=of premierTest] {$loop~variant~error$} ;
    \node [shape=rectangle, draw] (body0) [below=of premierTest] {$instructions\_1$} ;
    \node [shape=diamond, draw] (exp) [below=of body0] {$expression$} ;
    \node [shape=diamond, draw] (variantTest) [below=of exp] {$variant > 0$} ;
    \node [shape=rectangle, draw] (decTest) [left=of variantTest] {$variant {-}{-}$} ;
    \node [shape=rectangle, draw] (body1) [above=of decTest] {$instructions\_2$} ;
    \node [shape=rectangle, draw] (error) [right=of variantTest] {$loop~variant~error$} ;
    \node [rounded corners=5pt, shape=rectangle, draw] (end) [right=of error] {\textsc{end}} ;
    
    \draw [->] (start) -- (variant) ;
    \draw [->] (variant) -- (premierTest) ;
    \draw [->] (premierTest) to node[right] {$yes$} (body0) ;
    \draw [->] (premierTest) to node[above] {$no$} (error1) ;
    \draw [->] (body0) -- (exp) ;
    \draw [->] (exp) to node[right] {$true$} (variantTest) ;
    \draw [->] (variantTest) to node[above] {$yes$} (decTest) ;
    \draw [->] (variantTest) to node[above] {$no$} (error) ;
    \draw [->] (decTest) -- (body1) ;
    \draw [->, bend left] (exp.east) to node[above] {$false$} (end.north) ;
    \draw [->] (body1.north) .. controls +(north:2cm) and +(left:2cm) .. (body0.west) ;
    \draw [->] (error) -- (end) ;
    \draw [->] (error1.east) .. controls +(right:2cm) .. (end) ;
  \end{tikzpicture}
  \caption{\emph{loop} instruction flowchart}
  \labelFigure{loopInstructionFlowchart}
\end{figure}


















\sectionLabel{Method Call Instruction}{methodCallInstruction}




\sectionLabel{Modifier Call Instruction}{modifierCallInstruction}




\section{Switch Instruction}




\section{Send Instruction}




\section{Warning Instruction}




%!TEX encoding = UTF-8 Unicode
%!TEX root = ../galgas-book.tex

%--------------------------------------------------------------
\chapter{Semantic expressions}
%-------------------------------------------------------------






\part{Composants}

%!TEX encoding = UTF-8 Unicode
%!TEX root = ../galgas-book.tex

%--------------------------------------------------------------
\chapter{Le composant \texttt{lexique}}
%-------------------------------------------------------------

Le rôle d'un analyseur lexical est de grouper les caractères de la chaîne d'entrée en \emph{symboles terminaux}, ou encore \emph{terminaux}, en écartant les séparateurs comment les espaces ou les commentaires. 

En GALGAS, un analyseur lexical est défini par un composant \ggs+lexique+. Les composants \ggs+syntax+, qui définissent un ensemble de règles de production, font référence à un composant \ggs+lexique+.









\section{Définition d'un composant \texttt{lexique}}


En GALGAS, un composant \ggs+lexique+ a la structure suivante :

\begin{galgas}
lexique nom {
  declarations
}
\end{galgas}

Le \ggs+nom+ est le nom donné au composant ; il est utilisé pour référencer le composant \ggs+lexique+ dans un composant \ggs+syntax+.


Dans un composant \ggs+lexique+, cinq types de déclarations sont définis :
\begin{itemize}
  \item déclaration d'attribut lexical ;
  \item déclaration d'un symbole terminal ;
  \item déclaration d'une liste de symboles terminaux ;
  \item déclaration d'un message d'erreur lexical ;
  \item déclaration d'un style ;
  \item déclaration de règles d'analyse.
\end{itemize}

A //lexical attribute// carries the value associated with a terminal symbol: for example, the integer value of a literal integer constant, the string value of a character string constant, ...

In GALGAS, all terminal symbols must be declared either by a //single terminal symbol declaration//, either by a //list of terminal symbols declaration//. This defines the set of defined terminal symbols of your grammar.

Lexical error messages need also to be explicitly declared by //lexical error message declaration//. 

A //style declaration// declares a style identifier, for defining automatic coloring in a text editor. Currently, coloring is only available for Mac OS X Cocoa applications.

The order of declarations is not significant, but any entity must be declared before being used.

==== Lexical Rules Overview ====
The //lexical rules// define the executable part of a lexical component. Every lexical rule define //matching strings// that are are tested against substring from current location in input string. A matching string has a one character or more.

%\section{Fichiers engendrés}
%
%A lexical component description is translated in C++ code; for every lexical component, GALGAS generates a specific C++ class:
%  * the name of the class is the name of the \ggs+lexique+ component;
%  * this class is declared in a header file that is named the name of the \ggs+lexique+ component with the ''\textquotesingle.h\textquotesingle'' extension;
%  * this class is implemented in a file that is named the name of the \ggs+lexique+ component with the ''\textquotesingle.cpp\textquotesingle'' extension;
%  * this class inherits from ''C\_Lexique'' class (declared in ''libpm/galgas/C\_Lexique.h'' and implemented in ''libpm/galgas/C\_Lexique.cpp'').
%
%The two generated files are generated according the [[generated\_files|GALGAS file generation process]].


\section{Comment opère un analyseur lexical}

You can consider the lexical analyzer as an autonomous thread which analyzes the input string and which sends the sequence of the terminal symbols to the parser. Of course, for efficiency, the lexical analyzer is actually a parser subroutine.

The flowchart of a GALGAS lexical analyzer execution is:

{{ how\_works\_a\_lexical\_analyzer.png }}

When the input string is loaded from source file, a ''NUL'' character is appended as End Of String (eos) mark.

During execution, the lexical analyzer maintains a //current location// that designates the next character of the input string to be analyzed. Initially, current location points out the first character of the input string.

The lexical analyzer loops until the end of input string is reached. At the beginning of every loop, lexical attributes are reset to their default value.

Then, the first lexical rule matching expressions are tested against substring at current location in input string:
  * on match success, the first lexical rule is executed; usually, this execution sends a terminal symbol to the parser; however, in some cases as parsing a delimitor or a comment, no terminal symbol is sent;
  * on match failure, the lexical analyzer tries to find a match with the second lexical rule, and so on.

If no lexical rule matches, the character at current location is tested against eos character. On match success, the lexical analyzer sends once a predefined terminal symbol (denoted by ''\\$\\$'') to the parser, for telling it the end of input string is reached. On match failure, the //unknow character// lexical error is raised. The character at current location is discarded, that is the current location points out the next character of the input string.

\section{Ambiguïtés lexicales}

**GALGAS does not currently check that the set of lexical rules is unambiguous.** So, if the set is unambiguous, the rule order is not significant; if two or more rules introduce an ambiguity, the first defined one is used. 

\section{Un exemple}

This is very simple scanner, from ''galgas/samples/notSLRgrammar.ggs'':

|''**lexique** my\_scanner\_for\_not\_SLR\_grammar:\\ 
\#--- Identifiers\\ 
\\$id\\$ error **message** %%"%%an identifier%%"%% ;\\ 
**rule** \textquotesingle{a}\textquotesingle -> \textquotesingle{z}\textquotesingle | \textquotesingle{A}\textquotesingle -> \textquotesingle{Z}\textquotesingle :\\ 
 send \\$id\\$ ;\\ **end** **rule** ;\\ 
\#--- Delimitors\\ 
**list** delimitorsList error **message** %%"%%the %%'"%% . * . %%"'%% delimitor%%"%%: %%"%%=%%"%% , %%"%%*%%"%% ;\\ 
**rule** **list** delimitorsList ;\\ 
\#--- Separators\\ 
**rule** \textquotesingle\textbackslash{1}\textquotesingle -> %%' '%%:\\ 
**end** **rule** ;\\ 
**end** **lexique** ;''|

This \ggs+lexique+ component defines the following set of terminal symbols: ''\\$id\\$'' (explicitly declared), ''\\$=\\$'' and ''\\$*\\$'' (declared  by ''delimitorsList'' list.

The first rule sends the ''\\$id\\$'' terminal symbol each time a lower case or upper case character is found. The second rule names the ''delimitorsList'' list and sends the ''\\$=\\$'' or ''\\$*\\$'' terminal symbol each time the corresponding character is found. The last rule discards silently the space character and any control character.

Note that this scanner considers identifiers of only one character: ''ab'' is scanned as two consecutive identifiers.

===== Finding Sample Code =====

You can find examples of \ggs+lexique+ components in:
  * ''galgas/sample/alt\_sample.ggs'' file; this is a very basic scanner that handles one-letter identifier and four delimitors;
  * ''galgas/sample/arith\_expression.ggs'' file (for scanning literal integers); 
  * ''galgas/sample/test\_LR1\_grammar.ggs'' file gives an example of a small scanner for "toy" parser;
  * ''galgas/galgas/galgas\_sources/galgas\_scanner.ggs'' file: this is the actual scanner of the GALGAS language, and scans identifiers, keywords, delimiters, literal integers, literal characters, literal character strings, galgas type names (the '@' character followed by a sequence of letters), comments, ...   

\section{Déclarations lexicales}

\subsection{Déclaration d'un symbole terminal}

The //single terminal symbol declaration// declares a name used for naming a terminal symbol. This declaration just performs declaration, not scanning. For sending this terminal symbol to the parser, it must be named in a ''send'' lexical instruction within a lexical rule.

The declaration associates to the terminal symbol a possibly empty list of lexical attributes and a syntax error message (not a //lexical// error message), defined by a character string.

First example:

|''\$literal\_integer\$ error **message** %%"%%a decimal number%%"%%;''|

This declaration names no lexical attribute. Consequently, when the lexical send instruction ''send \$literal\_integer\$;'' will be called from a lexical rule, only the terminal symbol will be sent to the parser, but not the literal integer value. The parser has no way to get the actual value: all integer values share the same terminal symbol. It is sufficient for a pure parser, however a real compiler needs the actual value.

Second example:

|''@uint unsignedValueAttribute;\\ 
\$literal\_integer\$ !unsignedValueAttribute error **message** %%"%%a decimal number%%"%%;''|

In this declaration, the ''unsignedValueAttribute'' attribute is named in the terminal symbol declaration. So, when the lexical send instruction ''send \$literal\_integer\$;'' will be called from a lexical rule, the terminal symbol will be sent to the parser together with the unsigned value of the ''unsignedValueAttribute'' attribute, enabling the semantic instructions to catch it.

\subsection{Déclaration d'une liste de symboles terminaux}

The //list of terminal symbol declaration// associates to a name a list of terminal symbols with a generic syntax error message. It is typically used for declaring the keywords and the delimiters.

An example of key words declaration:

| ''**list** keywordList error **message** %%"%%the '%K' key word%%"%%: %%"%%if%%"%%, %%"%%then%%"%%, %%"%%else%%"%% ;'' |

The declared terminal symbols are: ''\$if\$'', ''\$then\$'', ''\$else\$''. The actual syntax error message is built from generic error message by replacing ''%K'' with terminal symbol string (for outputing a single ''%'', write ''%''''%''). So the syntax error message associated to the ''\$if\$'' terminal symbol is: "''the 'if' key word''".

An other example is a delimitor list declaration:

|''**list** delimitorList error **message** %%"%%the '%K' delimitor%%"%%: %%"%%.%%"%%, %%"%%;%%"%%, %%"%%(%%"%%, %%"%%)%%"%% ;''|

Actual scanning of a delimitor is done by a ''**rule** **list**'' lexical instruction.

\subsection{Déclaration d'un attribut terminal}

Lexical attributes carry values associated with terminal symbol. GALGAS handles string, unsigned, character, float lexical attributes. Every lexical attribute needs to be declared and its declaration names a GALGAS type name.


 The following table summerizes the attributes features and type notation:

%\^ Attribute Type \^ Type Name \^ Default Value \^ Corresponding C++ type \^
| ASCII String | ''@string'' | ''%%""%%'' (the empty string) | ''C\_String'' |
| ASCII Character | ''@char'' | ''%%'\0'%%'' | ''char'' |
| 32-bit Unsigned Integer | ''@uint'' | ''0'' | ''uint32'' |
| 32-bit Signed Integer | ''@sint'' | ''0'' | ''sint32'' |
| Float | ''@double'' | ''0.0'' | ''double'' |

In GALGAS, type names are identifiers prefixed by a ''@'' character.

An ''@string'', ''@char'', ''@uint'', ''@sint'', ''@double'' lexical attribute carry a string, character, unsigned, signed, double value.

In a ''**syntax**'' component, information that defines the location of the scanned terminal symbol in the input string is added to attribute value: so an ''@string'' object in the lexique component corresponds to an ''@lstring'' object in the syntax component. Location information is used by the parser and the semantic instructions for building syntax and semantic error messages that indicates //where// the error is located.

The //default value// is the one used at the beginning of every scanning loop for resetting lexical attribute.

The //corresponding C type// is useful if you want to write your own lexical actions (in C++). Please note that this correspondance is **only** available for lexical actions, and not for semantic action. The ''C\_String'' type is a C++ class that handles mutable character strings, without being worried about memory management. It is declared in the ''libpm/strings/C\_string.h'' file. The ''uint32'' type is the 32-bit unsigned integer type, and the ''sint32'' type is the 32-bit signed integer type. 
 

\subsection{Déclaration d'un message d'erreur lexicale}

The //lexical error message declaration// associates a name to a string. These error messages are used in lexical actions, and define the message that are displayed when a lexical error occurs.

|  ''**message** decimalNumberTooLarge: %%"%%decimal number too large%%"%%;'' |

 

\section{Règles lexicales}

There are two kinds of //lexical rules//:
  - the //list lexical rule//;
  - the //single lexical rule//.

\subsection{Règle s'appuyant sur une liste}

This is the simpliest form: it just names a previously defined list of terminal symbols; for example:

|''**rule** **list** delimitorList;''|

//Matching expressions// are the set of strings defined by the list. This rule tries to find a substring from input string at current location that matches a terminal symbol string defined in the list, sorted by decreasing length (so longest strings are tested first). On match success, //executing the rule// consists of sending the corresponding terminal symbol.

This kind of rule is typically used for scanning for a delimitor.

\subsection{Simple règle}

A //single lexical rule// has the following form:

|''**rule** //matching\_expression//:\\  //lexical\_instructions//\\ **end** **rule**;''|

The //matching expression// defines a set of matching strings, that are tested against the substring from input string at current location. On match, the //lexical instructions// are executed.

A matching expression can be:
  - a one-character string (for example, ''\textquotesingle{a}\textquotesingle'' matches the ''a'' character);
  - an union of one-character strings, defined by a character subrange (for example, ''\textquotesingle{a}\textquotesingle -> \textquotesingle{z}\textquotesingle'' matches a lower case letter);
  - a one or more characters string (for example, ''%%"%%=%%"%%'' matches the corresponding string);
  - an union of above (for example: ''\textquotesingle{A}\textquotesingle -> \textquotesingle{Z}\textquotesingle | \textquotesingle{a}\textquotesingle -> \textquotesingle{z}\textquotesingle'' matches a lower or upper case letter).

On match success, the current location is moved to designate the character after the matching string.

\section{Instructions lexicales}


\subsectionLabel{Instruction lexicale \texttt{select}}{instructionSelectLexical}

The //lexical select instruction// is the following:

|''**select**\\ **when** //matching\_expression\_1\_in\_select//: //lexical\_instructions\_1//\\ **when** //matching\_expression\_2\_in\_select//: //lexical\_instructions\_2//\\ ...\\ default //default\_lexical\_instructions//\\ **end** **select**;''|

A //lexical select instruction// has one or more ''**when**'' branches.

//matching expression\_1\_in\_select//, //matching expression\_2\_in\_select// conform to the defined above //matching\_expression//.

This instruction tries to match the different //matching expressions// until a matching success is found. In such case, the corresponding //lexical instructions// are executed. If all matching fail, the //default lexical instructions// are executed.

\subsectionLabel{Instruction lexicale \texttt{repeat}}{instructionRepeatLexical}

The //lexical repeat instruction// is the following:

|''**repeat**\\  //lexical\_instructions\_0//\\ **while** //matching\_expression\_1\_in\_repeat//: //lexical\_instructions\_1//\\ **while** //matching\_expression\_2\_in\_repeat//: //lexical\_instructions\_2//\\ ...\\ **end** **repeat**;''|

A //lexical while instruction// has one or more ''**while**'' branches.

//matching expression\_1\_in\_repeat//, //matching expression\_2\_in\_repeat// can be:
  - an expression conform to the defined above //matching\_expression//;
  - the ''~ //string//'' construct: the match succeeds when the //string// **is not** the current string;
  - the ''~ //string1//, //string2//, ...'' construct: the match succeeds when neither of //string1//, //string2//, ... are the current string.

This instruction first executes the //lexical instructions 0//. Then, it tries to match the different //matching expressions// until a matching success is found. In such case, the corresponding //lexical instructions// are executed, then the instruction is executed again (from //lexical instructions 0//). If all matching fail, execution of this instruction is complete (excution goes on the next instruction).

\subsection{Appel d'une action lexicale}

The //lexical action call instruction// calls a C++ defined method for performing computation and checking on lexical attributes. Its syntax is the following:

|''lexical\_action\_name (parameter, ...) ;''|

or

|''lexical\_action\_name (parameter, ...) error message\_name, ... ;''|

A lexical action is designated by its name. It accepts one or more parameters, and zero, one or more messages names.

A parameter is:
  - either a lexical attribute,
  - either a lexical function call;
  - either the joker character ''\textquotesingle*\textquotesingle'' that represents the character at current location.

A lexical action can be predefined or defined by the user. Predefined lexical actions are actually methods of ''C\_Lexique'' class (the generated scanner is a class that inherits from this class). User defined lexical actions must be implemented as methods of the generated scanner class.

**Note that no parameter type checking, no error message count checking is performed by GALGAS. ** A parameter type error or a message count error is detected at C++ compilation stage.
 
\subsection{Appel d'une fonction lexicale}

The //lexical function call// calls a C++ defined method for performing computation on lexical attributes. It can only appear as parameter of a lexical action call or a parameter of an other lexical function call. Its syntax is the following:

|''lexical\_function\_name (parameter, ...) ;''|

A lexical function is designated by its name. It accepts one or more parameters.

A lexical function parameter is:
  - either a lexical attribute,
  - either a lexical function call;
  - either the joker character ''\textquotesingle*\textquotesingle'' that represents the character at current location.

A lexical function can be predefined or defined by the user. Predefined lexical actions are actually methods of ''C\_Lexique'' class (the generated scanner is a class that inherits from this class). User defined lexical functions must be implemented as methods of the generated scanner class.

**Note that no parameter type checking is performed by GALGAS. ** A parameter type error is detected at C++ compilation stage.
 
\subsection{Instruction lexicale \texttt{error}}

The //lexical error instruction// raises a lexical error. Its syntax is:

|''error message\_name ;''|

The //message name// is the name of a previously declared lexical error message.

\subsection{Instruction lexicale \texttt{send}}

The //lexical send instruction// sends a terminal symbol to the parser. It has several forms:

=== First Form ===

|''send terminal\_symbol ;''|

This instruction sends inconditionnaly the //terminal symbol// to the parser.

=== Second Form ===

|''send search //attribute\_name// in //lexical\_list// default terminal\_symbol ;''|

This instruction first search for //attribute name// value in the //lexical list//. If found, the corresponding terminal symbol is sent to the parser. If not found, the default //terminal symbol// is sent.

Several consecutive ''search'' are accepted, allowing sequential searching in different lists:

|''send search //attribute\_name\_1// in //lexical\_list\_1// default search //attribute\_name\_2// in //lexical\_list\_2// default terminal\_symbol ;''|

\subsectionLabel{Instruction lexicale \texttt{drop}}{instructionLexicaleDrop}

|Available in GALGAS 1.5.6 and later.|


The //lexical drop instruction// does not send any terminal symbol to the parser. It is only significant for lexical coloring (see [[\#coloring\_comments|coloring comments]]).

This instruction names a terminal symbol:
|''**drop** //terminal\_symbol// ;''|


\subsection{Instruction lexicale \texttt{tag}}

|Available in GALGAS 1.5.6 and later.|

This instruction declares a new //tag identifier//.

|''**tag** //tag\_identifier// ;''|

A ''**tag**'' instruction records a location in the scanned file. The only way to use the declared tag identifier is the [[\#lexical\_rewind\_instruction|lexical rewind instruction]].

\subsection{Instruction lexicale \texttt{rewind}}

|Available in GALGAS 1.5.6 and later.|

|''**rewind** //tag\_identifier// send //terminal\_symbol//;''|

This instruction rewinds the scanned location from the tag identifier value, and sends the terminal symbol to the parser.








\section{Routines lexicales prédéfinies}

Lexical routine calls are instructions. Lexical function calls can appear as actual output parameters of routine calls and function calls. GALGAS predefines several lexical routines and several lexical functions (listed below).

A lexical routine accepts:
  * zero, one or more input/output or input formal arguments;
  * zero, one or more error messages.

Running the \texttt{-{}-print-predefined-lexical-actions} command line option lists all predefined routines and functions prototype.

\subsection{Routine \texttt{codePointToUnicode}}

\begin{galgas}
codePointToUnicode !@string inCodePointString
                   ?!@string ioString
\end{galgas}

\subsection{Routine \texttt{convertDecimalStringIntoSInt}}

\begin{galgas}
convertDecimalStringIntoSInt !@string inString
                             ?!@sint ioSignedNumber
                             error inNumberTooLargeError,
                                   inCharacterIsNotDecimalDigitError
\end{galgas}

\subsection{Routine \texttt{convertDecimalStringIntoSInt64}}

\begin{galgas}
convertDecimalStringIntoSInt64 !@string inString
                               ?!@sint64 ioSignedNumber
                               error inNumberTooLargeError,
                                     inCharacterIsNotDecimalDigitError
\end{galgas}

\subsection{Routine \texttt{convertDecimalStringIntoUInt}}

\begin{galgas}
convertDecimalStringIntoUInt !@string inString
                             ?!@uint ioUnsignedNumber
                             error inNumberTooLargeError,
                                   inCharacterIsNotDecimalDigitError
\end{galgas}

\subsection{Routine \texttt{convertDecimalStringIntoUInt64}}

\begin{galgas}
convertDecimalStringIntoUInt64 !@string inString
                               ?!@uint64 ioUnsignedNumber
                               error inNumberTooLargeError,
                                     inCharacterIsNotDecimalDigitError
\end{galgas}

\subsection{Routine \texttt{convertHTMLSequenceToUnicodeCharacter}}

\begin{galgas}
convertHTMLSequenceToUnicodeCharacter ?!@string inString
                                      ?!@char ioUnicodeCharacter
                                      error inUnassignedHTMLSequenceError
\end{galgas}

\subsection{Routine \texttt{convertHexStringIntoSInt}}

\begin{galgas}
convertHexStringIntoSInt !@string inString
                         ?!@sint ioSignedNumber
                         error inNumberTooLargeError,
                               inCharacterIsNotHexDigitError
\end{galgas}

\subsection{Routine \texttt{convertHexStringIntoSInt64}}

\begin{galgas}
convertHexStringIntoSInt64 !@string inString
                           ?!@sint64 ioSignedNumber
                           error inNumberTooLargeError,
                                 inCharacterIsNotHexDigitError
\end{galgas}

\subsection{Routine \texttt{convertHexStringIntoUInt}}

\begin{galgas}
convertHexStringIntoUInt !@string inString
                         ?!@uint ioUnsignedNumber
                         error inNumberTooLargeError,
                               inCharacterIsNotHexDigitError
\end{galgas}

\subsection{Routine \texttt{convertHexStringIntoUInt64}}

\begin{galgas}
convertHexStringIntoUInt64 !@string inString
                           ?!@uint64 ioUnsignedNumber
                           error inNumberTooLargeError,
                                 inCharacterIsNotHexDigitError
\end{galgas}

\subsection{Routine \texttt{convertStringToDouble}}

\begin{galgas}
convertStringToDouble !@string inString
                      ?!@double ioDouble
                      error inConversionError
\end{galgas}

This action tries to convert the string value of the first argument into a double value. On success, the resulting double is set to the second argument. The conversion error message is displayed on conversion error.

\subsection{Routine \texttt{convertUInt64ToSInt64}}

\begin{galgas}
convertUInt64ToSInt64 !@uint64 inUnsignedNumber
                      ?!@sint64 ioSignedNumber
                      error inNumberTooLargeError
\end{galgas}

If the unsigned value of the ''inUnsignedNumber'' argument is greater than ''2<sup>63</sup>-1'', the error is raised. Otherwise, the value is assigned to the ''ioSignedNumber'' argument.

\subsection{Routine \texttt{convertUIntToSInt}}

\begin{galgas}
convertUIntToSInt !@uint inUnsignedNumber
                  ?!@sint ioSignedNumber
                  error inNumberTooLargeError
\end{galgas}

If the unsigned value of the ''inUnsignedNumber'' argument is greater than ''2<sup>31</sup>-1'', the error is raised. Otherwise, the value is assigned to the ''ioSignedNumber'' argument.

\subsection{Routine \texttt{convertUnsignedNumberToUnicodeChar}}

\begin{galgas}
convertUnsignedNumberToUnicodeChar ?!@uint inUnsignedNumber
                                   ?!@char ioUnicodeCharacter
                                   error inUnassignedUnicodeValueError
\end{galgas}

\subsection{Routine \texttt{enterBinDigitIntoUInt}}

\begin{galgas}
enterBinDigitIntoUInt !@char inCharacter
                      ?!@uint ioUnsignedNumber
                      error inNumberTooLargeError,
                            inCharacterIsNotBinDigitError
\end{galgas}

\subsection{Routine \texttt{enterBinDigitIntoUInt64}}

\begin{galgas}
enterBinDigitIntoUInt64 !@char inCharacter
                        ?!@uint64 ioUnsignedNumber
                        error inNumberTooLargeError,
                              inCharacterIsNotBinDigitError
\end{galgas}

\subsection{Routine \texttt{enterCharacterIntoCharacter}}

\begin{galgas}
enterCharacterIntoCharacter ?!@char ioCharacter
                            !@char inCharacter
\end{galgas}

This routine performs ''ioCharacter = inCharacter'' assignment.

\subsection{Routine \texttt{enterCharacterIntoString}}

\begin{galgas}
enterCharacterIntoString ?!@string ioString
                         !@char inCharacter
\end{galgas}

Appends the character value of the second argument to the string value of the first argument. The resulting string is set to the first argument.

\subsection{Routine \texttt{enterDigitIntoASCIIcharacter}}

\begin{galgas}
enterDigitIntoASCIIcharacter ?!@char ioASCIICharacter
                             !@char inDecimalDigitCharacter
                             error inErrorCodeGreaterThan255,
                                   inErrorNotDecimalDigitCharacter
\end{galgas}

Build an ASCII character from its decimal definition.

First, the character value of the ''inDecimalDigitCharacter'' argument is tested to be a valid decimal digit, that is in one range ''[\textquotesingle0\textquotesingle, \textquotesingle9\textquotesingle]''. On failure, the ''inErrorNotDecimalDigitCharacter'' error message is displayed. On success, the unsigned value of the ''ioASCIICharacter'' argument is multiplied by ten, and is added the decimal value corresponding to second argument. If the result is lower or equal to ''2<sup>8</sup>-1'', it is set to the ''ioASCIICharacter'' argument. Otherwise, the ''inErrorCodeGreaterThan255'' error is raised.

Note: this lexical action treats characters as unsigned values.

\subsection{Routine \texttt{enterDigitIntoUInt}}

\begin{galgas}
enterDigitIntoUInt !@char inDecimalDigitCharacter
                   ?!@uint ioUnsignedNumber
                   error inNumberTooLargeError,
                         inCharacterIsNotDecimalDigitError
\end{galgas}

First, the value of ''inDecimalDigitCharacter'' argument is tested to be in the range ''[\textquotesingle0\textquotesingle, \textquotesingle9\textquotesingle]''. On failure, the ''inCharacterIsNotDecimalDigitError'' error message is displayed. On success, the unsigned value of the first argument is multiplied by ten, and is added the decimal value corresponding to the ''ioUnsignedNumber'' argument. If the result is lower or equal to ''2<sup>32</sup>-1'', it is set to the ''ioUnsignedNumber'' argument. Otherwise, the ''inNumberTooLargeError'' error is raised.

\subsection{Routine \texttt{enterDigitIntoUInt64}}

\begin{galgas}
enterDigitIntoUInt64 !@char inDecimalDigitCharacter
                     ?!@uint64 ioUnsignedNumber
                     error inNumberTooLargeError,
                           inCharacterIsNotDecimalDigitError
\end{galgas}

First, the value of ''inDecimalDigitCharacter'' argument is tested to be in the range ''[\textquotesingle0\textquotesingle, \textquotesingle9\textquotesingle]''. On failure, the ''inCharacterIsNotDecimalDigitError'' error message is displayed. On success, the unsigned value of the first argument is multiplied by ten, and is added the decimal value corresponding to the ''ioUnsignedNumber'' argument. If the result is lower or equal to ''2<sup>64</sup>-1'', it is set to the ''ioUnsignedNumber'' argument. Otherwise, the ''inNumberTooLargeError'' error is raised.

\subsection{Routine \texttt{enterHexDigitIntoASCIIcharacter}}

\begin{galgas}
enterHexDigitIntoASCIIcharacter ?!@char ioASCIICharacter
                                !@char inHexDigitCharacter
                                error inErrorCodeGreaterThan255,
                                      inErrorNotHexDigitCharacter
\end{galgas}

Build an ASCII character from its hexadecimal definition.

First, the character value of the ''inHexDigitCharacter'' argument is tested to be a valid hexadecimal digit, that is in one of the ranges ''[\textquotesingle0\textquotesingle, \textquotesingle9\textquotesingle]'', ''[\textquotesingle{a}\textquotesingle, \textquotesingle{f}\textquotesingle]'', ''[\textquotesingle{A}\textquotesingle, \textquotesingle{F}\textquotesingle]''. On failure, the ''inErrorNotHexDigitCharacter'' error message is displayed. On success, the unsigned value of the first argument is multiplied by sixteen, and is added the hexadecimal value corresponding to ''ioASCIICharacter'' argument. If the result is lower or equal to ''2<sup>8</sup>-1'', it is set to the ''ioASCIICharacter'' argument. Otherwise, the ''inErrorCodeGreaterThan255'' error is raised.

Note: this lexical action treats characters as unsigned values.

\subsection{Routine \texttt{enterHexDigitIntoUInt}}

\begin{galgas}
enterHexDigitIntoUInt !@char inHexDigitCharacter
                      ?!@uint ioUnsignedNumber
                      error inNumberTooLargeError,
                            inCharacterIsNotHexDigitError
\end{galgas}

First, the character value of the ''inHexDigitCharacter'' argument is tested to be a valid hexadecimal digit, that in one of the the ranges ''[\textquotesingle0\textquotesingle, \textquotesingle9\textquotesingle]'', ''[\textquotesingle{a}\textquotesingle, \textquotesingle{f}\textquotesingle]'', ''[\textquotesingle{A}\textquotesingle, \textquotesingle{F}\textquotesingle]''. On failure, the ''inCharacterIsNotHexDigitError'' error message is displayed. On success, the unsigned value of the ''ioUnsignedNumber'' argument is multiplied by sixteen, and is added the hexadecimal value corresponding to second argument. If the result is lower or equal to ''2<sup>32</sup>-1'', it is set to the ''ioUnsignedNumber'' argument. Otherwise, the first error is raised.

\subsection{Routine \texttt{enterHexDigitIntoUInt64}}

\begin{galgas}
enterHexDigitIntoUInt64 !@char inHexDigitCharacter
                        ?!@uint64 ioUnsignedNumber
                        error inNumberTooLargeError,
                              inCharacterIsNotHexDigitError
\end{galgas}

First, the character value of the ''inHexDigitCharacter'' argument is tested to be a valid hexadecimal digit, that in one of the the ranges ''[\textquotesingle0\textquotesingle, \textquotesingle9\textquotesingle]'', ''[\textquotesingle{a}\textquotesingle, \textquotesingle{f}\textquotesingle]'', ''[\textquotesingle{A}\textquotesingle, \textquotesingle{F}\textquotesingle]''. On failure, the ''inCharacterIsNotHexDigitError'' error message is displayed. On success, the unsigned value of the ''ioUnsignedNumber'' argument is multiplied by sixteen, and is added the hexadecimal value corresponding to second argument. If the result is lower or equal to ''2<sup>64</sup>-1'', it is set to the ''ioUnsignedNumber'' argument. Otherwise, the first error is raised.

\subsection{Routine \texttt{enterOctDigitIntoUInt}}

\begin{galgas}
enterOctDigitIntoUInt !@char inString
                      ?!@uint ioUnsignedNumber
                      error inNumberTooLargeError,
                            inCharacterIsNotOctDigitError
\end{galgas}

\subsection{Routine \texttt{enterOctDigitIntoUInt64}}

\begin{galgas}
enterOctDigitIntoUInt64 !@char inString
                        ?!@uint64 ioUnsignedNumber
                        error inNumberTooLargeError,
                              inCharacterIsNotOctDigitError
\end{galgas}

\subsection{Routine \texttt{multiplyUInt}}

\begin{galgas}
multiplyUInt !@uint inUnsignedNumber
             ?!@uint ioUnsignedNumber
             error inResultTooLargeError
\end{galgas}

Multiply the ''ioUnsignedNumber'' value by ''inUnsignedNumber'' value. Detection of overflow is performed.

\subsection{Routine \texttt{multiplyUInt64}}

\begin{galgas}
multiplyUInt64 !@uint inUnsignedNumber
               ?!@uint64 ioUnsignedNumber
               error inResultTooLargeError
\end{galgas}

Multiply the ''ioUnsignedNumber'' value by ''inUnsignedNumber'' value. Detection of overflow is performed.

\subsection{Routine \texttt{negateSInt}}

\begin{galgas}
negateSInt ?!@sint ioNumber
           error inNumberTooLargeError
\end{galgas}

\subsection{Routine \texttt{negateSInt64}}

\begin{galgas}
negateSInt64 ?!@sint64 ioNumber
             error inNumberTooLargeError
\end{galgas}


\subsection{Routine \texttt{resetString}}

\begin{galgas}
resetString ?!@string ioString
\end{galgas}








\section{Fonctions lexicales prédéfinies}


A lexical function accepts:
  * zero, one or more input formal arguments.

Running the \texttt{-{}-print-predefined-lexical-actions} command line option lists all predefined routines and functions prototype.

\subsection{Fonction \texttt{toLower}}

\begin{galgas}
toLower ?@char inCharacter -> @char
\end{galgas}

If the character value of the argument is an upper case letter, this function returns the corresponding lower case letter. Otherwise, it returns the unchanged character value of the argument.

\subsection{Fonction \texttt{toUpper}}

\begin{galgas}
toUpper ?@char inCharacter -> @char
\end{galgas}


If the character value of the argument is an lower case letter, this function returns the corresponding upper case letter. Otherwise, it returns the unchanged character value of the argument.



\section{Définir vos propres actions et fonctions lexicales}

You can define your own lexical actions and functions in C++ and make them available to called by lexical action call instructions.

\subsection{Où ?}

You must define your lexical actions and functions as a method of the C++ class generated by compilation of the \ggs+lexique+ component. You need to modify the generated code, adding method prototype declaration in class declaration.

**So that the method declaration that you added is not deleted at the time of a future compilation, define it in user zone 2 of the generated header file.** For more details, see [[generated\_files |file generation process page]].

For implementing your method, you can insert it in user zone 2 of the generated implementation file (for more details, see [[generated\_files |file generation process page]]). Alternatively, you can implement it in any other file, provided you include the needed header files.

\subsection{Correspondance entre les appels d'actions GALGAS et C++}

This table gives the correspondance between lexical argument types and C++ types. **Note this correspondance is only available for lexical arguments**.

%\^Lexical Formal Argument Type  \^C++ Type  \^
|''? @string''  |''**const** C\_String \&''|
|''?! @string''  |''C\_String \&''|
|''? @char''  |''**const** **char**''|
|''?! @char''  |''**char** \&''|
|''? @uint''  |''**const** uint32''|
|''?! @uint''  |''uint32 \&''|
|''? @sint''  |''**const** sint32''|
|''?! @sint''  |''sint32 \&''|
|''? @double''  |''**const** **double**''|
|''?! @double''  |''**double** \&''|

''?'' means the formal argument has input passing mode: it cannot be modified by the lexical action. ''?!'' means the formal argument has in/out passing mode: its value is got from the caller, can modified by the lexical action and is returned to the caller.

An error message argument corresponds to the C++ type ''**const** **char** *''.

In C++ generated code, the method call instruction generated by lexical action call names the lexical action name, prefixed by ''scanner\_routine\_''.

For example, consider the ''convertStringToDouble'' lexical action described below. This corresponds to the following method prototype:

''**void** scanner\_routine\_convertStringToDouble (**const** C\_String \&, **double** \&, **const char** *) ;''
==== Defining Action and Function Prototype ====

The prototype must conform to the rules presented in the [[\#Correspondance between Lexical Action Calls and C++ Called Methods|above]] section.

%\^Remember that GALGAS does not perform any checking on lexical action calls. Errors are detected at C++ compilation stage.\^

\section{Exemples d'analyseurs lexicaux}

\subsection{Analyser des identificateurs}

|''@string identifierString;\\ 
\$identifier\$ !identifierString error **message** %%"%%an identifier%%"%%;\\ 
**rule** %%'a'->'z' | 'A'->'Z'%%:\\ 
 **repeat**\\ 
  enterCharacterIntoString !?identifierString !* ;\\ 
 **while** %%'a'->'z' | 'A'->'Z' | '\_' | '0'->'9'%%:\\ 
 **end** **repeat** ;\\ 
 send \$identifier\$ ;\\
**end** **rule** ;''|

|''@string identifierString;\\ 
\$identifier\$ !identifierString error **message** %%"%%an identifier%%"%%;\\ 
**rule** %%'a'->'z' | 'A'->'Z'%%:\\ 
 **repeat**\\ 
  enterCharacterIntoString !?identifierString !toLower (!*) ;\\ 
 **while** %%'a'->'z' | 'A'->'Z' | '\_' | '0'->'9'%%:\\ 
 **end** **repeat** ;\\ 
 send \$identifier\$ ;\\
**end** **rule** ;''|

\subsection{Analyser des identificateurs et des mots-clés}

|''@string identifierString;\\ 
\\ 
\$identifier\$ !identifierString error **message** %%"%%an identifier%%"%%;\\ 
\\ 
**list** keywordList error **message** %%"the '%K' key word": "begin", "else", "end"%%;\\
\\ 
**rule** %%'a'->'z' | 'A'->'Z'%%:\\ 
 **repeat**\\ 
  enterCharacterIntoString !?identifierString !* ;\\ 
 **while** %%'a'->'z' | 'A'->'Z' | '\_' | '0'->'9'%%:\\ 
 **end** **repeat** ;\\ 
 send search identifierString in keywordList  default \$identifier\$ ;\\
**end** **rule** ;''|

\subsection{Analyser des délimiteurs}

|''**list** galgasDelimitorsList **error message** %%"the '%K' delimitor"%%:\\ 
 %%"*",  "|", ",",  ".",  "<>", "::", ">",  "<",  ";",  ":",%%\\ 
 %%"-",  "(", ")",  "->", "?", "==", "?", "!",  "=", "...",%%\\ 
 %%"[",  "]", "+=", "?!", "!?", "/",  "!=", "<=", ">=", "\&",%%\\ 
 %%"++", "{", "}"%% ;\\ 
\\ 
**rule list** galgasDelimitorsList ;''|

\subsection{Analyser des séparateurs}

|''**rule** %%'\u0001' -> ' '%% :\\ 
**end rule** ;''|

\subsection{Analyser des commentaires}

|''**rule** '\#' :\\ 
 **repeat**\\ 
 **while** %%'\u0001' -> '\u0009' | '\u000B' -> '\uFFFD'%% :\\ 
 **end repeat** ;\\ 
**end rule** ;''|

\subsection{Analyser des entiers décimaux non signés}

|''\$unsigned\_literal\_integer\$ !ulongValue **error message** %%"a decimal number"%% ;\\ 
\$signed\_literal\_integer\$ !longValue error **message** %%"a signed decimal number"%% ;\\ 
\\ 
**message** decimalNumberTooLarge : %%"decimal number too large"%% ;\\ 
\\ 
**message** internalError : %%"internal error"%% ;\\ 
\\ 
**rule** %%'0'->'9'%% :\\ 
 enterDigitIntoUlong !?ulongValue !* error decimalNumberTooLarge, internalError ;\\ 
 **repeat**\\ 
 **while** %%'0'->'9'%% :\\ 
  enterDigitIntoUlong !?ulongValue !* error decimalNumberTooLarge, internalError ;\\ 
 **while** %%'\_'%% :\\ 
 **end repeat** ;\\ 
 **select**\\ 
 **when** %%'S' | 's'%% :\\ 
  convertUlongToLong !?longValue !ulongValue %%error%% decimalNumberTooLarge ;\\ 
  send \$signed\_literal\_integer\$ ;\\ 
 default\\ 
  send \$unsigned\_literal\_integer\$ ;\\ 
 **end select** ;\\ 
**end rule** ;''|

\subsection{Analyser des entiers hexadécimaux non signés}

\subsection{Analyser des constantes caractère}

|''\$literal\_char\$ ! charValue **error message** %%"a character constant"%% ;\\ 
\\ 
**message** incorrectCharConstant : %%"incorrect literal character"%% ;\\ 
\\ 
**message** ASCIIcodeTooLargeError : %%"ASCII code > 255"%% ;\\ 
\\ 
**rule** %%'\''%% :\\ 
 **select**\\ 
 **when** %%'\\'%% :\\ 
  **select**\\ 
  **when** %%'f'%% :\\ 
   enterCharacterIntoCharacter !?charValue !%%'\f'%% ;\\ 
  **when** %%'n'%% :\\ 
   enterCharacterIntoCharacter !?charValue !%%'\n'%% ;\\ 
  **when** %%'r'%% :\\ 
   enterCharacterIntoCharacter !?charValue !%%'\r'%% ;\\ 
  **when** %%'t'%% :\\ 
   enterCharacterIntoCharacter !?charValue !%%'\t'%% ;\\ 
  **when** %%'v'%% :\\ 
   enterCharacterIntoCharacter !?charValue !%%'\v'%% ;\\ 
  **when** %%'\\'%% :\\ 
   enterCharacterIntoCharacter !?charValue !%%'\\'%% ;\\ 
  **when** %%'0'%% :\\ 
   enterCharacterIntoCharacter !?charValue !%%'\0'%% ;\\ 
  **when** %%'\''%% :\\ 
   enterCharacterIntoCharacter !?charValue !%%'\''%% ;\\ 
  **when** %%'0' -> '9'%% :\\ 
   **repeat**\\ 
    enterHexDigitIntoASCIIcharacter !?charValue !* error ASCIIcodeTooLargeError, internalError ;\\ 
   **while** %%'0' -> '9'%% :\\ 
   **end repeat** ;\\ 
  default\\ 
   error incorrectCharConstant ;\\ 
  **end select** ;\\ 
 **when** %%' ' -> '\uFFFD'%% :\\ 
  enterCharacterIntoCharacter !?charValue !* ;\\ 
 default\\ 
  error incorrectCharConstant ;\\ 
 **end select** ;\\ 
 **select**\\ 
 **when** %%'\''%% :\\ 
  send \$literal\_char\$ ;\\ 
 default\\ 
  error incorrectCharConstant ;\\ 
 **end select** ;\\ 
**end rule** ;''|

\subsection{Analyser des constantes chaîne de caractères}

\subsection{Analyser des constantes flottantes}

|''\$literal\_double\$ !floatValue !tokenString **error message** %%"a float number"%%;\\ 
\\ 
\$.\$ **error message** %%"the '.' delimitor"%%;\\ 
\\ 
**message** floatNumberConversionError : %%"invalid float number"%% ;\\ 
\\ 
**rule** %%'.'%% :\\ 
 **select**\\ 
 **when** %%'0'->'9'%% :\\ 
  enterCharacterIntoString !?tokenString !%%'0'%% ;\\ 
  enterCharacterIntoString !?tokenString !%%'.'%% ;\\ 
  enterCharacterIntoString !?tokenString !* ;\\ 
  **repeat**\\ 
  **while** %%'0'->'9'%% :\\ 
   enterCharacterIntoString !?tokenString !* ;\\ 
  **while** %%'\_'%% :\\ 
  **end repeat** ;\\ 
  convertStringToDouble !tokenString !?floatValue error floatNumberConversionError ;\\ 
  send \$literal\_double\$ ;\\ 
 default\\ 
  send \$.\$ ;\\ 
 **end select** ;\\
**end rule** ;''|

\section{\emph{Back tracking} avec les instructions \texttt{tag} et \texttt{rewind}}

|Available in GALGAS 1.5.6 and later.|

The ''**tag**'' and ''**rewind**'' instructions can be used for performing back tracking.

The first example is the way the non terminal symbols are scanned in GALGAS 1.5.6 (and later).

A non terminal is composed of a single '<' character, followed by a letter, zero, one or more letters, digits or underscore characters, is ended by a single '>' character. For example ''<abcdef>'' is a valid non terminal. However, ''<abcdef >'' is //not// a valid non terminal (because of the space before the final '>' character): it is considered as a '<' delimitor, followed by the ''abcdef'' identifier and by the '>' delimitor.

In the file ''galgas/galgas\_sources/galgas\_scanner.ggs'', the three delimitors befgging with a '<' character and the non terminal symbols are scanned by the following code:

''\$<\$ **error message** "the '<' delimitor" **style** delimitersStyle ;''\\
''%%\$<=\$%% **error message** "the '<=' delimitor" **style** delimitersStyle ;''\\
''%%\$<<\$%% **error message** "the '<<' delimitor" **style** delimitersStyle ;''\\
''\$non\_terminal\_symbol\$ ! tokenString **error message** "a non terminal symbol <...>" **style** nonTerminalStyle ;''\\

''**rule** '<' :''\\
'' **tag** onlyInfDelimiter ;''\\
'' **select**''\\
'' **when** '=' :''\\
'' send %%\$<=\$%% ;''\\
'' **when** '<' :''\\
''  send %%\$<<\$%% ;''\\
'' **when** %%'a' -> 'z' | 'A' ->'Z'%% :''\\
''  **repeat**''\\
''   enterCharacterIntoString !?tokenString !* ;''\\
''  **while** %%'a' -> 'z' | 'A' ->'Z' | '0' -> '9' | '\_'%% :''\\
''  **end repeat** ;''\\
''  **select**''\\
''  **when** '>' :''\\
''   send \$non\_terminal\_symbol\$ ;''\\
''  default''\\
''   **rewind** onlyInfDelimiter send \$<\$ ;''\\
''  **end select** ;''\\
'' default''\\
''  send \$<\$ ;''\\
'' **end select** ;''\\
''**end rule** ;''\\

The ''**tag**'' instruction records a scanning location. When the final '>' character is not found, the scanner is rewinded at the character following the '<' character, and the ''\$<\$'' terminal is sent. On next scanning, an identifier (or a key word) will be found.

The second examples shows how to scan for integer constants, float constants, and array bounds in Pascal :
  * an integer constant is a (non empty) sequence of digits ;
  * a float constant is a (non empty) sequence of digits, following by a dot and a (possibly empty) sequence of digits;
  * an array bound is an integer constant, followed by the '..' delimitor (two dots) and an integer constant.

The problem is that ''1..2'' should not be scanned as a float constant, a single dot delimitor, and an integer constant.

This can be achieved by the following code:

''**rule** %%'0' -> '9'%% :''\\
'' **repeat**''\\
'' **while** %%'0' -> '9'%% :''\\
'' **end repeat** ;''\\
'' **tag** endOfIntegerConstant ;''\\
'' **select**''\\
'' **when** %%'.'%% :''\\
''  **select**''\\
''  **when** %%'.'%% :''\\
''   **rewind** endOfIntegerConstant send \$integer\_constant\$ ;''\\
''  **when** %%'0' -> '9'%% :''\\
''   **repeat**''\\
''   **while** %%'0' -> '9'%% :''\\
''   **end repeat** ;''\\
''   send \$float\_constant\$ ;''\\
''  default''\\
''   send \$float\_constant\$ ;''\\
''  **end select** ;''\\
'' default''\\
''  send \$integer\_constant\$ ;''\\
'' **end select** ;''\\
''**end rule** ;''\\


\section{Ajouter la coloration lexicale (sur Mac uniquement)}

With GALGAS, you can easily embbed your compiler in a GUI application (currently available only for Mac OS X). This application has a built-in text editor, from which you can modify, save and compile source file. With //style declarations//, you can add automatic coloring in the built-in text editor.

A //style declaration// associates a message to a style identifier. For example:

|''**style** keywordsStyle -> %%"%%Keywords:%%"%% ;''|

The associated message is used in application preferences window as a comment of each color selection item.

A //style declaration// does not link a style identifier to any terminal symbol. You need to add this information to //single terminal symbol declaration// and //list of terminal symbols declaration// by naming the style identifier after the syntax error message:

|''\$literal\_integer\$ error **message** %%"%%a decimal number%%"%% **style** integerStyle;''|

|''**list** delimitorList error **message** %%"%%the '%%"%% . * . %%"%%' delimitor%%"%% **style** keywordsStyle: %%"%%.%%"%%, %%"%%;%%"%%, %%"%%(%%"%%, %%"%%)%%"%%;''|

\subsection{Exemple : les styles de l'analyseur lexical GALGAS}

As an example, you can take a look on GALGAS scanner, in ''galgas/galgas\_sources/galgas\_scanner.ggs'' file. The style declarations are the following:

|''**style** keywordsStyle -> %%"%%Keywords:%%"%% ;\\ **style** delimitersStyle -> %%"%%Delimiters:%%"%% ;\\ **style** terminalStyle -> %%"%%Terminal symbols:%%"%% ;\\ **style** integerStyle -> %%"%%Integer constants:%%"%% ;\\ **style** characterStyle -> %%"%%Character constants:%%"%% ;\\ **style** stringStyle -> %%"%%String constants:%%"%% ;\\ **style** typeNameStyle -> %%"%%Type names (@...):%%"%%'';|

You can search for the occurrence of style identifiers, to see how they are used.

In Cocoa GALGAS application, the Color tab of the Preferences window lists all style comments, each of them being associated to a ''NSColorWell'' for color selection:

{{cocoa\_galgas\_color\_styles.png}}

Note that no default color is defined in style declaration. Until you define yourself a color from Preference window, it defaults to black color.

\subsection{Appliquer un style aux commentaires}
|Available in GALGAS 1.5.6 and later.|

In GALGAS 1.5.6 and later, you can define a color for comments. Proceed as follows:
  - declare a new terminal symbol, for example ''\$comment\$'';
  - declare a style for this new terminal symbol;
  - when a comment is scanned, use the ''**drop**'' instruction for naming the new terminal symbol (instead of the usual ''send'' instruction).

The ''**drop**'' instruction is only significant for syntax coloring.

For example, GALGAS comments are defined in ''galgas/galgas\_sources/galgas\_scanner.ggs'' in this way:

''**style** commentStyle -> "Comments:" ;''\\
''...''\\
''\$comment\$ error **message** %%"%%a comment%%"%% **style** commentStyle ;''\\
''**rule** %%'\#'%% :''\\
'' **repeat**''\\
'' **while** %%'\u0001' -> '\u0009' | '\u000B' | '\u000C' | '\u000E' -> '\uFFFD'%% :''\\
'' **end repeat** ;''\\
'' **drop** \$comment\$ ;''\\
''**end rule** ;''\\


%!TEX encoding = UTF-8 Unicode
%!TEX root = ../galgas-book.tex

%--------------------------------------------------------------
\chapter{Syntax and Grammar Components}
%-------------------------------------------------------------

\section {GALGAS and Context-Free Grammars}


\section{Writing a Syntax Component}\index{Component!Syntax}

\section{Syntax Instructions}

\subsection{Terminal Symbol Instruction}

\subsection{Non Terminal Symbol Instruction}


\subsection{Repeat Instruction}


\subsection{Select Instruction}



\subsection{Parse Instruction}

\subsubsection{Parse do ... Instruction}


\subsubsection{Parse loop ... Instruction}


\subsubsection{Parse when ... Instruction}


\section{Writing a Grammar Component}\index{Component!Grammar}



%!TEX encoding = UTF-8 Unicode
%!TEX root = ../galgas-book.tex

%--------------------------------------------------------------
\chapter{Graphic User Interface Component}\index{Component!Graphic User Interface}
%-------------------------------------------------------------


%!TEX encoding = UTF-8 Unicode
%!TEX root = ../galgas-book.tex

%--------------------------------------------------------------
\chapterLabel{Le composant \texttt{option}}{composantOption}
%-------------------------------------------------------------


Le composant \galgas{option} permet de définir des options qui sont appelables à partir de la ligne de commande. Dans le code, la valeur d'une option est obtenue à partir de l'opérande \emph{appel d'une option}, décrit dans la \refSubsectionPage{appelOption}.

Voici l'exemple d'un composant \galgas{option} qui déclare une option (évidement, un composant \galgas{option} peut déclarer un nombre quelconque d'options) :
\begin{galgascode}
option nom_composant {
  @bool nom_option : 'S', "asm" -> "Extract assembly code"
}
\end{galgascode}


\section{Déclaration d'une option}

La déclaration d'une option présente le syntaxe suivante :
\begin{galgascode}
  @T nom_option : caractere, chaine -> description
\end{galgascode}

Les cinq champs qui définissent une option sont :
\begin{itemize}
  \item \galgas{@T} : le type de l'option ; trois types sont autorisés : \galgas{@bool}, \galgas{@uint} et \galgas{@string} ;
  \item \galgas{nom_option} : c'est le nom, interne à GALGAS, qui permettra de désigner l'option dans l'\emph{appel d'une option} (\refSubsectionPage{appelOption}) ; 
  \item \galgas{caractere} : le caractère qui activera l'option dans la ligne de commande ; par exemple, en écrivant \galgas{'A'}, l'option sera activée par \texttt{-A} dans la ligne de commande ; si vous ne voulez pas d'activation par un caractère, écrivez \galgas{'\\0'} ;
  \item \galgas{chaine} : la chaîne de caractères qui activera l'option dans la ligne de commande ; par exemple, en écrivant \galgas{"ABEDEF"}, l'option sera activée par \texttt{-{}-ABCDEF} dans la ligne de commande ; si vous ne voulez pas d'activation par une chaîne, écrivez \galgas{""} ;
  \item \galgas{description} : une chaîne de caractère qui contient une description de l'option, qui sera affichée par l'option \texttt{-{}-help} de votre compilateur.
\end{itemize}








\section{Option booléenne}

Le champ qui définit le type de l'option est \galgas{@bool} ; par exemple :
\begin{galgascode}
  @bool nom_option : 'S', "asm" -> "Extract assembly code"
\end{galgascode}

Dans la ligne de commande, l'option est activée par \texttt{-A} ou \texttt{-{}-asm}.

Par défaut, l'option n'est pas activée, et sa valeur associée est \galgas{false}. Quand l'option est activée dans la ligne de commande, sa valeur associée est \galgas{true}.








\section{Option entière}

Le champ qui définit le type de l'option est \galgas{@uint} ; par exemple :
\begin{galgascode}
  @uint nom_option : 'M', "max-iterations-count" -> "Max of iteration count"
\end{galgascode}

Dans la ligne de commande, l'option est activée par \texttt{-N=xxx} ou \texttt{-{}-max-iterations-count=xxx}, où \texttt{xxx} est un nombre entier positif ou nul (et inférieur ou égal à $2^{32}-1$).

Par défaut, l'option n'est pas activée, et sa valeur associée est $0$. Quand l'option est activée dans la ligne de commande, sa valeur associée est la valeur \texttt{xxx}. Ainsi, l'option \texttt{-N=0}, comme l'option \texttt{-{}-max-iterations-count=0} n'a aucun effet.










\section{Option chaîne de caractères}

Le champ qui définit le type de l'option est \galgas{@string} ; par exemple :
\begin{galgascode}
  @string nom_option : 'F', "file-name" -> "File name"
\end{galgascode}

Dans la ligne de commande, l'option est activée par \texttt{-F=abc} ou \texttt{-{}-file-name=abc}, où \texttt{abc} est une chaîne de caractères sans espaces. Si vous voulez entrer une chaîne de caractères qui comprend des espaces, écrivez : \texttt{"-F=abc"} ou \texttt{"-{}-file-name=abc"}.

Par défaut, l'option n'est pas activée, et sa valeur associée est la chaîne vide. Quand l'option est activée dans la ligne de commande, sa valeur associée est la chaîne \texttt{abc}. Ainsi, l'option \texttt{-F=}, comme l'option \texttt{-{}-file-name=} n'a aucun effet.




%!TEX encoding = UTF-8 Unicode
%!TEX root = ../galgas-book.tex

%--------------------------------------------------------------
\chapter{Program Component}\index{Component!Program}
%-------------------------------------------------------------


%!TEX encoding = UTF-8 Unicode
%!TEX root = ../galgas-book.tex

%--------------------------------------------------------------
\chapter{Project Component}\index{Component!Project}
%-------------------------------------------------------------


\section{Generated Cocoa Application}

When a project component is compiled with a Xcode project target, a \texttt{project\_xcode} directory is created. This directory contains:
\begin{itemize}
\item the Xcode project file;
\item a \texttt{build.command} file ;
\item an \texttt{Info.plist} file ;
\item an \texttt{English.lproj} directory ;
\item an empty \texttt{userResources} directory.
\end{itemize}

The \texttt{Info.plist}, the \texttt{English.lproj} directory and the \texttt{userResources} directory are used by the Cocoa target of the Xcode project. The \texttt{build.command} file is a command file that builds the Xcode project.

All files you put in the \texttt{userResources} directory are added to the Cocoa target of the Xcode project when the GALGAS Project component is compiled. When the Cocoa target of the Xcode project is compiled, theses files are put in the \texttt{Resources} directory within the application bundle.

Adding files to the \texttt{userResources} directory is the way of customizing the Cocoa Application:
\begin{itemize}
\item adding icons to your Application (\refSubsectionPage{addingIconsCocoaApplication});
\item customizing syntax coloring (\refSubsectionPage{customizingSyntaxColoring}). 
\end{itemize}




\subsectionLabel{Adding Icons to your Cocoa Application}{addingIconsCocoaApplication}




\subsectionLabel{Customizing Syntax Coloring}{customizingSyntaxColoring}


%!TEX encoding = UTF-8 Unicode
%!TEX root = ../galgas-book.tex

%--------------------------------------------------------------
\chapter{Projet \texttt{Xcode} et application Cocoa}
%-------------------------------------------------------------

Vous pouvez demander à GALGAS d'engendrer un projet \texttt{Xcode}, qui contiendra :
\begin{itemize}
  \item le compilateur en version \emph{release} sous la forme d'un utilitaire en ligne de commande ; 
  \item le compilateur en version \emph{debug} sous la forme d'un utilitaire en ligne de commande ; 
  \item une application Cocoa permettant d'appeler les deux utilitaires.
\end{itemize}






\section{Paramétrage du projet GALGAS}

Pour engendrer un projet \texttt{Xcode}, il vous suffit d'ajouter une déclaration telle que \ggs+%Mavericks+ dans votre fichier projet (d'extension \tpp{.galgasProject}). Par exemple :

\begin{galgas}
project (0:0:1) -> "logo" {
  %Mavericks
  %applicationBundleBase : "fr.what"
  ...
\end{galgas}

Un projet \texttt{Xcode} définit la version de Mac OS pour laquelle il va être compilé : évidemment, \ggs+%Mavericks+ définit la version \texttt{Mavericks} (10.9). Le \refTableau{options-pour-xcode} liste les différents options possibles. GALGAS fixe la version indiquée dans le projet \texttt{Xcode}, et il faut ensuite que la version de \texttt{Xcode} utilisée soit compatible avec ce réglage. L'option \ggs+%LatestMacOS+ correspond au réglage correspondant du projet \texttt{Xcode} engendré.

Il y a une seconde option à ajouter dans le projet GALGAS : \ggs+%applicationBundleBase+. Celle-ci fixe le \emph{Bundle Identifier} de l'application Cocoa. À la chaîne définie dans l'option (ici \ggs+"fr.what"+) est ajouté le nom du projet (défini dans l'en-tête, ici \ggs+"logo"+), précédé par un point : le \emph{Bundle Identifier} est donc \tpp{fr.what.logo}.


\begin{table}[t]
  \centering
  \begin{tabular}{rl}
    \textbf{Option} & \textbf{Version Mac OS correspondante}\\
    \ggs+%SnowLeopard+ & \texttt{SnowLeopard} (10.6) \\
    \ggs+%Lion+ & \texttt{Lion} (10.7) \\
    \ggs+%MountainLion+ & \texttt{Mountain Lion} (10.8) \\
    \ggs+%Mavericks+ & \texttt{Mavericks} (10.9) \\
    \ggs+%Yosemite+ & \texttt{Yosemite} (10.10) \\
    \ggs+%LatestMacOS+ & Dernière version Mac OS supportée par \texttt{Xcode} \\
  \end{tabular}
  \caption{Options du projet GALGAS indiquant la version Mac OS}
  \labelTableau{options-pour-xcode}
  \ligne
\end{table}







\section{Projet \texttt{Xcode} engendré}


Quand le projet GALGAS est compilé, un répertoire \tpp{xcode-project} directory est créé, et contient :
\begin{itemize}
\item le fichier projet \texttt{Xcode} ;
\item un fichier \tpp{build.command} ;
\item un fichier \tpp{Info.plist} ;
\item un répertoire \tpp{English.lproj} ;
\item un répertoire \tpp{userResources}.
\end{itemize}

Le rôle de chacun est précisé par le \refTableau{fichiers-repertoires-xcode}. Ne pas modifier ces fichiers et répertoires à la main, une compilation GALGAS supprimerait vos changements. La seule exception est le contenu du répertoire \tpp{userResources} qui n'est pas modifié par les compilations GALGAS.

\begin{table}[t]
  \centering
  \begin{tabular}{rp{11cm}}
    \textbf{Fichier ou répertoire} & \textbf{Rôle}\\
    \tpp{build.command} & Effectue la compilation Xcode, appelable via une commande \emph{Shell} \\
    \tpp{Info.plist}    & Informations pour l'application Cocoa \\
    \tpp{English.lproj} & Informations pour l'application Cocoa \\
    \tpp{userResources} & Permet d'associer des icônes aux fichiers sources de votre compilateur, ainsi qu'à l'application Cocoa engendrée (voir \refSectionPage{ajouterIconesAppliCocoa}) \\
  \end{tabular}
  \caption{Fichiers et répertoires relatifs au projet Xcode}
  \labelTableau{fichiers-repertoires-xcode}
  \ligne
\end{table}





\sectionLabel{Définir des icônes pour votre application Cocoa}{ajouterIconesAppliCocoa}

Vous pouvez définir :
\begin{itemize}
  \item une icône pour l'application Cocoa ;
  \item une icône particulière pour chaque type de fichier source.
\end{itemize}

Le nom de chaque fichier d'icône fixe son rôle :
\begin{itemize}
  \item pour l'application Cocoa, le fichier d'icône doit s'appeler \tpp{application\_icns.icns} ;
  \item pour chaque type de fichier source, le nom est basé sur l'extension du fichier : si celui-ci est par exemple \tpp{.logo}, le fichier d'icônes doit s'appeler \tpp{logo\_icns.icns}.
\end{itemize}

Ces fichiers d'icônes doivent être placés dans le répertoire \tpp{userResources}, et il faut ensuite refaire une compilation GALGAS pour que ces fichiers soient ajoutés au projet \texttt{Xcode}.

En résumé :
\begin{enumerate}
  \item concevoir les fichiers d'icônes, en fixant leur nom comme indiqué ci-dessus ;
  \item placer ces icônes dans le répertoire \tpp{userResources} ;
  \item effectuer une compilation GALGAS : celle-ci met à jour le projet \texttt{Xcode}, en ajoutant les fichiers d'icônes au \emph{target} Cocoa ;
  \item recompiler le \emph{target} Cocoa du projet \texttt{Xcode} : les icônes sont prises en compte.
\end{enumerate}











%\sectionLabel{Customizing Syntax Coloring}{customizingSyntaxColoring}
%
%This feature enables to set particular display attributes to a given list of tokens. This list is defined by a plist file located in the \emph{Resources} directory of the application bundle.
%
%{1} Edit the GALGAS lexique component, and add one (or more) \ggs+style+ entries. For example:
%
%\begin{galgas}
%lexique my_lexique :
%  ...
%style mySpecificStyle -> "My Style" ;
%  ...
%end lexique ;
%\end{galgas}
%
%This new style's feature can be edited as other styles, by the Preferences setting of your Cocoa application.
%
%
%{2} Create a plist file with the \emph{Property List Editor} application. This file should be named with the lexique component name, suffixed by \tpp{-syntax-coloring-adds}: so, for the example, the file name is \tpp{my\_lexique-syntax-coloring-adds.plist}. Put this file in the \emph{userResources} directory: so when the GALGAS project document is compiled, this file is added to the Cocoa Target of the Xcode project. 
%
%{3} Edit the \tpp{my\_lexique-syntax-coloring-adds.plist} with the \emph{Property List Editor} application or Xcode. Add one entry for every custom syntax coloring case: the \emph{key} is the terminal spelling, the \emph{value} has the \emph{String} type, and the specific style name. For example, the \refFigure{}{customSyntaxColoringPropertyList} shows the assignment of the terminal which spelling is \tpp{begin} by the \tpp{mySpecificStyle} style.
%
%\begin{figure}[t]
%  \centering
%  \includegraphics[width=15cm]{chapter-cocoa-features/custom-syntax-coloring-property-list-edition.pdf}
%  \caption{Example of a syntax coloring property list}
%  \labelFigure{customSyntaxColoringPropertyList}
%  \ligne
%\end{figure}
%
%If your provides an undefined style name, you will be warned every time you open a document by a beep and a small explanation window.






\sectionLabel{Indexation des fichiers sources}{indexingYourSourceFiles}

Vous pouvez configurer votre projet GALGAS pour que l'application Cocoa engendrée établisse une indexation et des références croisées : un \tpp{cmd-click} affiche un menu contextuel. Cette indexation est basée sur l'analyse syntaxique. C'est ce qui a été fait pour l'application \texttt{CocoaGalgas} (\refFigure{}{indexingUnderCocoaGALGAS}). On voit dans le menu contextuel trois classes d'index : \tpp{Class Definition}, \tpp{Class Reference as Superclass} et \tpp{Abstract Category Method Definition} ; au dessous, les références croisées correspondantes.


%You can configure your project for enabling cross-referencing entities with your Cocoa application. This has been done in GALGAS, providing such feature (\refFigure{}{indexingUnderCocoaGALGAS}). The contextual menu is displayed with a \texttt{cmd-click}.

\begin{figure}[t]
  \centering
  \includegraphics[width=16cm]{chapter-cocoa-features/indexing-sample.png}
  \caption{Indexation et références croisées dans l'application CocoaGalgas}
  \labelFigure{indexingUnderCocoaGALGAS}
  \ligne
\end{figure}

Pour configurer votre projet, vous avez à modifier le composant \emph{lexique}, le composant \emph{syntax}, le composant \emph{grammar}, et la règle d'analyse du fichier source. Les cinq modifications sont présentées successivement ci-après, en prenant comme exemple le langage LOGO (\refSectionPage{presentation-logo}).





\subsection{En tête du composant \texttt{lexique}}

Il faut modifier l'en-tête, en ajoutant la déclaration  \ggs+indexing in+ :


\begin{galgas}
lexique logo_lexique indexing in "INDEXING" {
  ...
\end{galgas}

La chaîne \ggs+"INDEXING"+ définit le nom du répertoire qui contient les fichiers cache de l'indexation. Ce répertoire est relatif au répertoire qui contient le fichier source.

Note : si vous effectuez maintenant la compilation GALGAS, vous obtiendrez une erreur sur la définition de la grammaire, indiquant qu'elle doit aussi indiquer la prise en compte de l'indexation.




\subsection{En tête du composant \texttt{grammar}}

Il suffit de préfixer par \ggs+indexing+ l'en-tête du composant \ggs+grammar+ :

\begin{galgas}
indexing grammar logo_grammar ... {
  ...
\end{galgas}

Note : maintenant, la compilation GALGAS s'effectue sans erreur.




\subsection{Règle d'analyse des fichiers sources}

La règle d'analyse des fichiers source doit mentionner dans l'en-tête la grammaire utilisée pour l'analyse (pour l'exemple du langage LOGO, la troisième ligne \ggs+grammar logo_grammar+ remplit ce rôle).

\begin{galgas}
case . "logo"
message "a source text file with the .logo extension"
grammar logo_grammar
?sourceFilePath:@lstring inSourceFile {
  grammar logo_grammar in inSourceFile
}
\end{galgas}

Quand le mode d'exécution (absence de l'option \tpp{-{}-mode}) est le mode par défaut, les instructions de la règle sont exécutées. Ci-dessus, la seule instruction est \ggs+grammar logo_grammar in inSourceFile+ (ligne 5).

Quand le mode d'exécution (présence de l'option \tpp{-{}-mode}) n'est pas le mode par défaut, les instructions de la règle ne sont pas exécutées, et les opérations sont guidées par la grammaire indiquée ligne 3. Dans le cas de l'indexation, l'exécution construit l'indexation du fichier source.









\subsection{Déclaration des classes d'index}

La déclaration des classes d'index s'effectue dans l'analyseur lexical. Dans la cadre du langage d'exemple LOGO, on veut simplement indéxer les routines, plus précisément l'endroit de leur définition, et les endroits où elles sont appelées. On définit donc deux classes d'index \ggs+routineDefinition+ et \ggs+routineCall+. À chaque déclaration est associée une chaîne de caractères, qui sera le titre affiché dans le menu contextuel. 


\begin{galgas}
lexique logo_lexique indexing in "INDEXING" {
  ...
indexing routineDefinition : "Routine Definition"
  ...
indexing routineCall : "Routine call"
  ...
\end{galgas}


Ces définitions peuvent être placées à tout endroit dans la définition de l'analyseur lexical.








\subsection{Définition des entrées indexées}

L'analyseur syntaxique va être complété de façon à définir les symboles qui seront indéxés. Plus précisement, c'est l'instruction d'analyse de symbole terminal qui est modifiée.

Considérons d'abord la déclaration de routine. La règle de l'analyseur syntaxique qui définit cette analyse est :

\begin{galgas}
rule <routine_definition> {
  $ROUTINE$
  $identifier$ ?let @lstring routineName
  $BEGIN$
  <instruction_list>
  $END$
}
\end{galgas}

Le nom de la routine est défini par l'instruction \ggs+$identifier$ ?let @lstring routineName+ : on la modifie alors de façon à signifier que l'indentificateur doit être indéxé comme une définition de routine :

\begin{galgas}
rule <routine_definition> {
  $ROUTINE$
  $identifier$ ?let @lstring routineName indexing routineDefinition
  $BEGIN$
  <instruction_list>
  $END$
}
\end{galgas}

Maintenant, l'instruction d'appel de routine :

\begin{galgas}
rule <instruction> {
  select
    $CALL$
    $identifier$ ?let @lstring routineName
    $;$
  or
    ...
  end
}
\end{galgas}

On modifie de manière analogue l'instruction \ggs+$identifier$ ?let @lstring routineName+ :

\begin{galgas}
rule <instruction> {
  select
    $CALL$
    $identifier$ ?let @lstring routineName indexing routineCall
    $;$
  or
    ...
  end
}
\end{galgas}



\subsection{Compilation et essai}

Les modifications sont terminées. Vous pouvez recompiler votre projet (compilation GALGAS puis compilation de la cible Cocoa du projet \texttt{Xcode}). La \refFigure{}{exemple-indexation-logo} montre le résultat obtenu en effectuant un \tpp{cmd-click} sur le nom de la routine.

\begin{figure}[t]
  \centering
  \includegraphics[width=8cm]{chapter-cocoa-features/exemple-indexation-logo.png}
  \caption{Exemple d'indexation en LOGO}
  \labelFigure{exemple-indexation-logo}
  \ligne
\end{figure}



%\noindent{4} \textbf{Program component configuration.} Insert the \ggs+grammar+ declaration after the « \texttt{message ...} » declaration in every program rule concerned by indexing:
%
%\begin{galgas}
%case ...
%message ...
%grammar my_grammar
%?@lstring inSourceFile {
%  ...
%\end{galgas}
%
%
%
%
%
%
%\noindent{5} \textbf{Define indexing entries.} The indexing entries are defined within the rules of syntax components. The \emph{terminal check} instruction is the unique way for definition, by naming an index class name:
%
%\begin{galgas}
%syntax ... ("my_lexique.gLexique") :
%  ...
%rule ... :
%  ...
%  $identifier$ ? ... indexing myIndexClass1 ;
%  ...
%end rule ;
%  ...
%\end{galgas}
%
%Any kind of terminal symbol accepts an « \texttt{indexing} » attribute : keywords, delimiters, literal string, integers, identifiers, \dots
%
%Several index class names can be named, using a comma as separator:
%\begin{galgas}
%  ...
%  $identifier$ ? ... indexing myIndexClass1, myIndexClass2 ;
%  ...
%\end{galgas}
%
%
%
%
%
%
%\noindent{6} \textbf{Compile and play.} Now, you can compile and run the Cocoa Application. With a \texttt{cmd}-click on an indexed terminal symbol, the contextual menu is displayed. You can delete the indexing directory at any moment, it will be rebuilt as needed.










%-----------------------------------------------------------------------------------------------------------------------*
%                                                                                                                       *
%   I N D E X                                                                                                           *
%                                                                                                                       *
%-----------------------------------------------------------------------------------------------------------------------*

\cleardoublepage % Pour commencer a une page impaire
\phantomsection  % Pour faire correctement pointer l'hyperlien dans la table des matieres

%--- Ecrire l'index
{\small
\printindex
}

%-----------------------------------------------------------------------------------------------------------------------*
%                                                                                                                       *
%   F I N    D U    D O C U M E N T                                                                                     *
%                                                                                                                       *
%-----------------------------------------------------------------------------------------------------------------------*

\end{document}

%-----------------------------------------------------------------------------------------------------------------------*

