%!TEX encoding = UTF-8 Unicode

\documentclass [a4paper, 10pt] {book}

%-----------------------------------------------------------------------------------------------------------------------*
%                                                                                                                       *
%   « N O I R    E T    B L A N C »    O U    « C O U L E U R »                                                         *
%                                                                                                                       *
%-----------------------------------------------------------------------------------------------------------------------*

%--- Par défaut, l'impression se fait en couleur
\providecommand{\sortieEnCouleur}{true}

%-----------------------------------------------------------------------------------------------------------------------*
%                                                                                                                       *
%   E N C O D A G E    D E S    S O U R C E S     :     L A T I N 1                                                     *
%                                                                                                                       *
%-----------------------------------------------------------------------------------------------------------------------*

%--- Paquetage pour le codage des sources en UTF-8
\usepackage[utf8]{inputenc}

%--- Latex demande ce paquetage pour mieux afficher le caractère "°" et \textquotesingle "'"
\usepackage{textcomp}

%--- Ce paquetage permet d'effectuer certaines césures, et ainsi d'éviter les messages "Overfull \hbox"
\usepackage[T1]{fontenc}

%-----------------------------------------------------------------------------------------------------------------------*
%                                                                                                                       *
%   M I S E    E N    P A G E S                                                                                         *
%                                                                                                                       *
%-----------------------------------------------------------------------------------------------------------------------*

% Voir "Une courte introduction à Latex2e", § 6.4

%--- Marge gauche : 2,8 cm ; le paramètre \hoffset contient cette valeur, moins 1 pouce
%    \hoffset = 2,8 cm - 2,54 cm = 0,26 cm
\setlength{\hoffset}{0.26 cm}

%--- Marges supplémentaires, différenciées pour les pages gauches et droites ; ici, aucune.
\setlength{\oddsidemargin }{0 cm}
\setlength{\evensidemargin}{0 cm}

%--- Largeur du texte
%    \textwidth = 210 mm - 28 mm - 28 mm = 15,4 cm
\setlength{\textwidth}{15.4 cm}

%--- Marge haute : 2,8 cm ; le paramètre \voffset contient cette valeur, moins 1 pouce
%    \voffset = 2,8 cm - 2,54 cm = 0,26 cm
\setlength{\voffset}{0.26 cm}

%--- Distance entre la marge haute et l'en-tête : 0 cm
\setlength{\topmargin}{0 cm}

%--- Hauteur de l'en-tête de chaque page : 1 cm
\setlength{\headheight}{1 cm}

%--- Distance entre l'en-tête de chaque page et le corps : 0,5 cm
\setlength{\headsep}{0.5 cm}

%--- Hauteur du corps
%    \textheight = 29,7 cm - 2,8 cm - 2,8 cm - 1,5 cm = 22,6 cm
\setlength{\textheight}{22.6 cm}

%-----------------------------------------------------------------------------------------------------------------------*
%                                                                                                                       *
%   C H O I X    D E    L A    P O L I C E                                                                              *
%                                                                                                                       *
%-----------------------------------------------------------------------------------------------------------------------*

% http://www.cuk.ch/articles/4237
% Un seul des choix suivants doit être validé ; si aucun, c'est la police par défaut qui est utilisée

%---------------------------------------------------- Pour utiliser la police "Times"
%\renewcommand{\rmdefault}{ptm}
%\usepackage{mathptmx}

%---------------------------------------------------- Pour utiliser la police "Palatino"
%\renewcommand{\rmdefault}{ppl}
%\usepackage{mathpazo}

%---------------------------------------------------- Pour utiliser la police "Bookman"
%\usepackage{bookman}

%---------------------------------------------------- Pour utiliser la police "Fourier"
\usepackage{fouriernc}
\usepackage[scaled=0.875]{helvet}
%\usepackage{courier}

%---------------------------------------------------- Pour utiliser la police "Beton, euler"
%\usepackage{beton, euler}

%---------------------------------------------------- Pour utiliser la police "Utopia - MathDesign"
%\usepackage[utopia]{mathdesign}

%---------------------------------------------------- Pour utiliser la police "Charter - MathDesign"
%\usepackage[charter]{mathdesign}

%-----------------------------------------------------------------------------------------------------------------------*
%                                                                                                                       *
%   G E S T I O N   D E    L A     C O U L E U R                                                                        *
%                                                                                                                       *
%-----------------------------------------------------------------------------------------------------------------------*

%--- Ce paquetage permet de définir des couleurs de la forme yellow!50 (jaune à 50 %)
\usepackage{xcolor}


%-----------------------------------------------------------------------------------------------------------------------*
%                                                                                                                       *
%   E X T E N S I O N S    P O U R    L ' É C R I T U R E    D E S     F O R M U L E S    M A T H É M A T I Q U E S     *
%                                                                                                                       *
%-----------------------------------------------------------------------------------------------------------------------*

%--- Extensions pour l'écriture des formules mathématiques
\usepackage{amsmath}
\usepackage{amssymb}
\usepackage{amsfonts}

%--- Paquetage "IEEEtrantools"
% Pour créer des tableaux d'équations, bien alignées
% Voir courte-intro-latex.pdf, page §3.5.2 page 83
\usepackage[retainorgcmds]{IEEEtrantools}


%-----------------------------------------------------------------------------------------------------------------------*
%                                                                                                                       *
%   P A Q U E T A G E    « I F T H E N »                                                                                *
%                                                                                                                       *
%-----------------------------------------------------------------------------------------------------------------------*

%--- Ce paquetage permet d'effectuer des tests : \ifthenelse{test}{bloc then}{bloc else}
\usepackage{ifthen}

%-----------------------------------------------------------------------------------------------------------------------*
%                                                                                                                       *
%   P A Q U E T A G E    « L I S T I N G S »                                                                            *
%                                                                                                                       *
%-----------------------------------------------------------------------------------------------------------------------*

\usepackage{listings}

\lstdefinelanguage{galgas}{
  keywordsprefix=@,
  morekeywords={abstract, after, before, between, block, cast, class, const, default, do, drop, else, elsif, end, enum,
    error, extends, extern, extract, false, feature, filewrapper, foreach, function, grammar, graph, gui, here, if,
    import, in, index, indexing, insert, label, lexique, list, listmap, local, log, loop, map, mapproxy, match,
    message, method, mod, modifier, nonterminal, not, on, once, operator, option, or, override, parse, pragma,
    prefixedby, project, program, reader, remove, replace, repeat, rewind, root, routine, rule, search, select,
    self, selfcopy, semantics, send, sortedlist, state, struct, style, super, switch, syntax, tag, template,
    then, true, uniquemap, unused, warning, when, while, with
  },
  comment=[l]\#,
  string=[b]", string=[d]",
  classoffset=1,
  morekeywords={;}, keywordstyle=\color{green}\textbf,
  classoffset=2
}

\ifthenelse{\equal{\sortieEnCouleur}{true}}{
  \lstset{
    language=galgas,
    basicstyle=\normalsize,
    frame=l,
    keywordstyle=\color{blue}\textbf,
    backgroundcolor=\color{yellow!50},
    identifierstyle=\color{brown},
    commentstyle=\color{red},
    stringstyle=\color{gray}
  }
}{
  \lstset{
    language=galgas,
    basicstyle=\normalsize,
    frame=l,
    keywordstyle=\textbf,
  }
}

%-----------------------------------------------------------------------------------------------------------------------*
%                                                                                                                       *
%   P A Q U E T A G E    « L O N G T A B L E »                                                                          *
%                                                                                                                       *
%-----------------------------------------------------------------------------------------------------------------------*

%--- Pour afficher correctement des tables sur plusieurs pages
\usepackage{longtable}

%-----------------------------------------------------------------------------------------------------------------------*
%                                                                                                                       *
%   E N - T Ê T E S    E T    P I E D S    D E    P A G E S                                                             *
%                                                                                                                       *
%-----------------------------------------------------------------------------------------------------------------------*

\usepackage{fancyhdr}
\pagestyle{fancy}


%---------------------------------------------------- Include my new commands
%---------------------------------------------------------------------*
%                                                                     *
% This file is included in galgas-book.tex file                       *
%                                                                     *
%---------------------------------------------------------------------*

%--- Mot clef
\newcommand \motCle[1] {\texttt{\textbf{#1}}}

%---------------------------------------------------------------------*

%--- Nom de type
\newcommand \nomType[1] {\texttt{#1}}

%---------------------------------------------------------------------*
% Section pour la d�finition des types

\newcommand \definitionSectionType[1] {\section{The \nomType{#1} Type}\label{#1}}

\newcommand \lienSectionType[1] {\hyperref[#1]{#1 type (page \pageref{#1})}}

%---------------------------------------------------------------------*

%--- D�finition d'un reader sans argument
% Exemple d'appel :
% \readerSansArgument{line} % Nom du reader
% {@location} % Nom du type
% {1.8.2} % Premi�re version GALGAS qui impl�mente ce reader
% {@uint} % Type renvoy�
% {Returns the line of the receiver's value.} % Description
% {this reader raises a run-time error if ...} % Discussion

\newcommand \readerSansArgument[6] {
  \subsection{\texttt{#1} Reader}\label{reader #2 #1}
  #5
  \newline
  \newline
  \begin{tabular}{|l}
  \texttt{\emph{reader} \nomType{#2} #1 -> #4 ;}
  \end{tabular}
  \newline
  \newline
  \textbf{Availability:} available in GALGAS #3 and later.
  \newline
  \newline
  \textbf{Discussion:} #6
  \newline
}

\newcommand \lienReader[2] {\hyperref[reader #1 #2]{\texttt{#2} reader (page \pageref{reader #1 #2})}}

%---------------------------------------------------------------------*

%--- D�finition d'un reader � 1 argument
% Exemple d'appel :
% \readerSansArgument{line} % Nom du reader
% {@location} % Nom du type
% {1.8.2} % Premi�re version GALGAS qui impl�mente ce reader
% {@uint} % Type renvoy�
% {Returns the line of the receiver's value.} % Description
% {this reader raises a run-time error if ...} % Discussion

\newcommand \readerUnArgument[7] {
  \subsection{\texttt{#1} Reader}\label{reader #2 #1}
  #6
  \newline
  \newline
  \begin{tabular}{|l}
  \texttt{\emph{reader} \nomType{#2} #1}\\
  \texttt{\ \ ?#5}\\
  \texttt{\ \ -> #4 ;}
  \end{tabular}
  \newline
  \newline
  \textbf{Availability:} available in GALGAS #3 and later.
  \newline
  \newline
  \textbf{Discussion:} #7
  \newline
}

%---------------------------------------------------------------------*

%--- D�finition d'un reader � 2 arguments
% Exemple d'appel :
% \readerSansArgument{line} % Nom du reader
% {@location} % Nom du type
% {1.8.2} % Premi�re version GALGAS qui impl�mente ce reader
% {@uint} % Type renvoy�
% {Returns the line of the receiver's value.} % Description
% {this reader raises a run-time error if ...} % Discussion

\newcommand \readerDeuxArguments[8] {
  \subsection{\texttt{#1} Reader}\label{reader #2 #1}
  #7
  \newline
  \newline
  \begin{tabular}{|l}
  \texttt{\emph{reader} \nomType{#2} #1}\\
  \texttt{\ \ ?#5}\\
  \texttt{\ \ ?#6}\\
  \texttt{\ \ -> #4 ;}
  \end{tabular}
  \newline
  \newline
  \textbf{Availability:} available in GALGAS #3 and later.
  \newline
  \newline
  \textbf{Discussion:} #8
  \newline
}

%---------------------------------------------------------------------*

%--- D�finition d'un constructeur sans argument
% Exemple d'appel :
% \constructeurSansArgument{nowhere} % Nom du constructeur
% {@location} % Nom du type
% {1.8.2} % Premi�re version GALGAS qui impl�mente ce constructeur
% {@uint} % Type renvoy�
%{Returns an \nomType{@location} that does not points out any location.} % Description
%{The returned object responds \motCle{true} to the isNowhere reader.} % Discussion

\newcommand \constructeurSansArgument[6] {
  \subsection{\texttt{#1} Constructor}\label{constructor #2 #1}
  #5
  \newline
  \newline
  \begin{tabular}{|l}
  \texttt{\emph{constructor} \nomType{#2} #1 -> #4 ;}
  \end{tabular}
  \newline
  \newline
  \textbf{Availability:} available in GALGAS #3 and later.
  \newline
  \newline
  \textbf{Discussion:} #6
  \newline
}

%---------------------------------------------------------------------*

%--- D�finition d'un constructeur avec 1 argument

\newcommand \constructeurUnArgument[7] {
  \subsection{\texttt{#1} Constructor}\label{constructor #2 #1}
  #6
  \newline
  \newline
  \begin{tabular}{|l}
  \texttt{\emph{constructor} \nomType{#2} #1}\\
  \texttt{\ \ ?#5}\\
  \texttt{\ \ -> #4 ;}
  \end{tabular}
  \newline
  \newline
  \textbf{Availability:} available in GALGAS #3 and later.
  \newline
  \newline
  \textbf{Discussion:} #7
  \newline
}

%---------------------------------------------------------------------*

%--- D�finition d'un constructeur avec 2 arguments

\newcommand \constructeurDeuxArguments[8] {
  \subsection{\texttt{#1} Constructor}\label{constructor #2 #1}
  #7
  \newline
  \newline
  \begin{tabular}{|l}
  \texttt{\emph{constructor} \nomType{#2} #1}\\
  \texttt{\ \ ?#5}\\
  \texttt{\ \ ?#6}\\
  \texttt{\ \ -> #4 ;}
  \end{tabular}
  \newline
  \newline
  \textbf{Availability:} available in GALGAS #3 and later.
  \newline
  \newline
  \textbf{Discussion:} #8
  \newline
}

%---------------------------------------------------------------------*

%--- D�finition d'un modifier � 1 argument

\newcommand \modifierUnArgument[6] {
  \subsection{\texttt{#1} Modifier}\label{modifier #2 #1}
  #5
  \newline
  \newline
  \begin{tabular}{|l}
  \texttt{\emph{modifier} \nomType{#2} #1}\\
  \texttt{\ \ ?#4}\\
  \end{tabular}
  \newline
  \newline
  \textbf{Availability:} available in GALGAS #3 and later.
  \newline
  \newline
  \textbf{Discussion:} #6
  \newline
}

%---------------------------------------------------------------------*

%--- Exemple un ligne

\newcommand \exempleUneLigne[1] {
  \textbf{Example:}
  \texttt{#1}\newline
}

%---------------------------------------------------------------------*

%--- Exemple 2 lignes

\newcommand \exempleDeuxLignes[2] {
  \textbf{Example:}\newline
  \texttt{#1}\newline
  \texttt{#2}\newline
}

%---------------------------------------------------------------------*

%--- Exemple 3 lignes

\newcommand \exempleTroisLignes[3] {
  \textbf{Example:}\newline
  \texttt{#1}\newline
  \texttt{#2}\newline
  \texttt{#3}\newline
}

%---------------------------------------------------------------------*



%-----------------------------------------------------------------------------------------------------------------------*
%                                                                                                                       *
%   G E S T I O N    D E    L ' I N D E X                                                                               *
%                                                                                                                       *
%-----------------------------------------------------------------------------------------------------------------------*

% http://www.cuk.ch/articles/4097
% http://www.tuteurs.ens.fr/logiciels/latex/makeindex.html
% http://linux.die.net/man/1/makeindex
%
% Attention ! Les deux commandes suivantes, ainsi que le \printindex placé plus bas ne
% sont pas suffisants pour construire l'index : il faut utiliser l'utilitaire "makeIndex"
% Voir le fichier de commande "build.command"
\usepackage{makeidx}
\makeindex

%-----------------------------------------------------------------------------------------------------------------------*
%                                                                                                                       *
%   T O C B I D I N D                                                                                                   *
%                                                                                                                       *
%-----------------------------------------------------------------------------------------------------------------------*

%    Pour faire figurer la liste des tableaux (et la table des matières)
%    dans la table des matières
\usepackage{tocbibind}

\setcounter{tocdepth}{3}

%-----------------------------------------------------------------------------------------------------------------------*
%                                                                                                                       *
%   H Y P E R R E F                                                                                                     *
%                                                                                                                       *
%-----------------------------------------------------------------------------------------------------------------------*

%--- Pour les hyperliens, et le contrôle de la génération PDF 
\usepackage{hyperref}
\hypersetup{colorlinks=true}
\hypersetup{linkcolor=blue}

\hypersetup{breaklinks=true}
%\hypersetup{verbose=true}

%-----------------------------------------------------------------------------------------------------------------------*
%                                                                                                                       *
%   S H O W K E Y S    ( P O U R    D É B O G U E R )                                                                   *
%                                                                                                                       *
%-----------------------------------------------------------------------------------------------------------------------*

%\usepackage{showkeys}

%-----------------------------------------------------------------------------------------------------------------------*
%                                                                                                                       *
%   T I T L E T O C                                                                                                     *
%                                                                                                                       *
%-----------------------------------------------------------------------------------------------------------------------*

%--- Description dans le paquetage titlesec
% http://forum.mathematex.net/latex-f6/formatage-avance-de-la-table-des-matieres-t11559.html

\usepackage{titletoc}


%--- Par défaut dans la tables des matières, le numéro de sous-section est trop long et mange le début du titre
%\titlecontents{subsection}[2.5cm]{}{\hspace*{-5.0em}\hyperref[subsection \thecontentslabel]{\thecontentslabel}\hspace*{0.5em}}{}{\titlerule*[0.66pc]{.}\contentspage}{}

\titlecontents{subsection}[2.5cm]{}{\hspace*{-5.0em}\thecontentslabel\hspace*{0.5em}}{}{\titlerule*[0.66pc]{.}\contentspage}{}

%\titlecontents{subsection}
%  [2cm]% retrait gauche
%  {}% pour les entrées numérotées et non numérotées
%  {\hspace*{-3.0em}\makebox[2.5em]{\hspace*{0pt plus 1 fill minus 1fill}\thecontentslabel.}\hspace*{0.5em}}% pour les entrées numérotées uniquement
%  {}% pour les entrées non numérotées uniquement
%  {\titlerule*[0.66em]{.}\contentspage}% numéro de page

%-----------------------------------------------------------------------------------------------------------------------*
%                                                                                                                       *
%   D P R O G R E S S                                                                                                   *
%                                                                                                                       *
%-----------------------------------------------------------------------------------------------------------------------*

%--- Affiche les sections dans le log
\usepackage{dprogress}

%-----------------------------------------------------------------------------------------------------------------------*
%                                                                                                                       *
%   D É B U T    D U    D O C U M E N T                                                                                 *
%                                                                                                                       *
%-----------------------------------------------------------------------------------------------------------------------*


\begin{document} 

%-----------------------------------------------------------------------------------------------------------------------*
%                                                                                                                       *
%   P A G E    D E    T I T R E                                                                                         *
%                                                                                                                       *
%-----------------------------------------------------------------------------------------------------------------------*

\title{\Huge{\textbf{GALGAS Book}}\\~\\ \normalsize{For release GALGAS-CURRENT-VERSION}}
\author{Pierre Molinaro}
\date \today 

\maketitle

%-----------------------------------------------------------------------------------------------------------------------*
%                                                                                                                       *
%   T A B L E    D E S    M A T I È R E S                                                                               *
%                                                                                                                       *
%-----------------------------------------------------------------------------------------------------------------------*

\tableofcontents
 
%-----------------------------------------------------------------------------------------------------------------------*
%                                                                                                                       *
%   L I S T E    D E S    T A B L E A U X                                                                               *
%                                                                                                                       *
%-----------------------------------------------------------------------------------------------------------------------*

\listoftables
\addtocontents{lot}{\protect\thispagestyle{empty}\protect\pagestyle{empty}}

%-----------------------------------------------------------------------------------------------------------------------*
%                                                                                                                       *
%   L I S T E    D E S    F I G U R E S                                                                                 *
%                                                                                                                       *
%-----------------------------------------------------------------------------------------------------------------------*

\listoffigures
\addtocontents{lof}{\protect\thispagestyle{empty}\protect\pagestyle{empty}}

%-----------------------------------------------------------------------------------------------------------------------*
%                                                                                                                       *
%   L E S    C H A P I T R E S                                                                                          *
%                                                                                                                       *
%-----------------------------------------------------------------------------------------------------------------------*

\input{chapter-installation/installation.tex}

%!TEX encoding = UTF-8 Unicode
%!TEX root = ../galgas-book.tex

%--------------------------------------------------------------
\chapter{Using GALGAS}
%-------------------------------------------------------------


\section{Command Line Options}


\section{Creating a New Project}


%!TEX encoding = UTF-8 Unicode
%!TEX root = ../galgas-book.tex

%--------------------------------------------------------------
\chapter{Lexical Elements}
%-------------------------------------------------------------

%Avant
%[\input{|"/bin/echo -n ab"}]
%Après
%
%
%(\immediate\write18{"/bin/echo -n ab"})
%
%\begin{filecontents}{myfile.tex}
%     This text gets written to \texttt{myfile.tex}.\\
%     Zis text gets written to \texttt{myfile.tex}.
%\end{filecontents}
%
%\input{myfile.tex}



%!TEX encoding = UTF-8 Unicode
%!TEX root = ../galgas-book.tex

%--------------------------------------------------------------
\chapter{Project Component}\index{Component!Project}
%-------------------------------------------------------------


\section{Generated Cocoa Application}

When a project component is compiled with a Xcode project target, a \texttt{project\_xcode} directory is created. This directory contains:
\begin{itemize}
\item the Xcode project file;
\item a \texttt{build.command} file ;
\item an \texttt{Info.plist} file ;
\item an \texttt{English.lproj} directory ;
\item an empty \texttt{userResources} directory.
\end{itemize}

The \texttt{Info.plist}, the \texttt{English.lproj} directory and the \texttt{userResources} directory are used by the Cocoa target of the Xcode project. The \texttt{build.command} file is a command file that builds the Xcode project.

All files you put in the \texttt{userResources} directory are added to the Cocoa target of the Xcode project when the GALGAS Project component is compiled. When the Cocoa target of the Xcode project is compiled, theses files are put in the \texttt{Resources} directory within the application bundle.

Adding files to the \texttt{userResources} directory is the way of customizing the Cocoa Application:
\begin{itemize}
\item adding icons to your Application (\refSubsectionPage{addingIconsCocoaApplication});
\item customizing syntax coloring (\refSubsectionPage{customizingSyntaxColoring}). 
\end{itemize}




\subsectionLabel{Adding Icons to your Cocoa Application}{addingIconsCocoaApplication}




\subsectionLabel{Customizing Syntax Coloring}{customizingSyntaxColoring}


%!TEX encoding = UTF-8 Unicode
%!TEX root = ../galgas-book.tex

%--------------------------------------------------------------
\chapter{Le composant \texttt{lexique}}
%-------------------------------------------------------------

Le rôle d'un analyseur lexical est de grouper les caractères de la chaîne d'entrée en \emph{symboles terminaux}, ou encore \emph{terminaux}, en écartant les séparateurs comment les espaces ou les commentaires. 

En GALGAS, un analyseur lexical est défini par un composant \ggs+lexique+. Les composants \ggs+syntax+, qui définissent un ensemble de règles de production, font référence à un composant \ggs+lexique+.









\section{Définition d'un composant \texttt{lexique}}


En GALGAS, un composant \ggs+lexique+ a la structure suivante :

\begin{galgas}
lexique nom {
  declarations
}
\end{galgas}

Le \ggs+nom+ est le nom donné au composant ; il est utilisé pour référencer le composant \ggs+lexique+ dans un composant \ggs+syntax+.


Dans un composant \ggs+lexique+, cinq types de déclarations sont définis :
\begin{itemize}
  \item déclaration d'attribut lexical ;
  \item déclaration d'un symbole terminal ;
  \item déclaration d'une liste de symboles terminaux ;
  \item déclaration d'un message d'erreur lexical ;
  \item déclaration d'un style ;
  \item déclaration de règles d'analyse.
\end{itemize}

A //lexical attribute// carries the value associated with a terminal symbol: for example, the integer value of a literal integer constant, the string value of a character string constant, ...

In GALGAS, all terminal symbols must be declared either by a //single terminal symbol declaration//, either by a //list of terminal symbols declaration//. This defines the set of defined terminal symbols of your grammar.

Lexical error messages need also to be explicitly declared by //lexical error message declaration//. 

A //style declaration// declares a style identifier, for defining automatic coloring in a text editor. Currently, coloring is only available for Mac OS X Cocoa applications.

The order of declarations is not significant, but any entity must be declared before being used.

==== Lexical Rules Overview ====
The //lexical rules// define the executable part of a lexical component. Every lexical rule define //matching strings// that are are tested against substring from current location in input string. A matching string has a one character or more.

%\section{Fichiers engendrés}
%
%A lexical component description is translated in C++ code; for every lexical component, GALGAS generates a specific C++ class:
%  * the name of the class is the name of the \ggs+lexique+ component;
%  * this class is declared in a header file that is named the name of the \ggs+lexique+ component with the ''\textquotesingle.h\textquotesingle'' extension;
%  * this class is implemented in a file that is named the name of the \ggs+lexique+ component with the ''\textquotesingle.cpp\textquotesingle'' extension;
%  * this class inherits from ''C\_Lexique'' class (declared in ''libpm/galgas/C\_Lexique.h'' and implemented in ''libpm/galgas/C\_Lexique.cpp'').
%
%The two generated files are generated according the [[generated\_files|GALGAS file generation process]].


\section{Comment opère un analyseur lexical}

You can consider the lexical analyzer as an autonomous thread which analyzes the input string and which sends the sequence of the terminal symbols to the parser. Of course, for efficiency, the lexical analyzer is actually a parser subroutine.

The flowchart of a GALGAS lexical analyzer execution is:

{{ how\_works\_a\_lexical\_analyzer.png }}

When the input string is loaded from source file, a ''NUL'' character is appended as End Of String (eos) mark.

During execution, the lexical analyzer maintains a //current location// that designates the next character of the input string to be analyzed. Initially, current location points out the first character of the input string.

The lexical analyzer loops until the end of input string is reached. At the beginning of every loop, lexical attributes are reset to their default value.

Then, the first lexical rule matching expressions are tested against substring at current location in input string:
  * on match success, the first lexical rule is executed; usually, this execution sends a terminal symbol to the parser; however, in some cases as parsing a delimitor or a comment, no terminal symbol is sent;
  * on match failure, the lexical analyzer tries to find a match with the second lexical rule, and so on.

If no lexical rule matches, the character at current location is tested against eos character. On match success, the lexical analyzer sends once a predefined terminal symbol (denoted by ''\\$\\$'') to the parser, for telling it the end of input string is reached. On match failure, the //unknow character// lexical error is raised. The character at current location is discarded, that is the current location points out the next character of the input string.

\section{Ambiguïtés lexicales}

**GALGAS does not currently check that the set of lexical rules is unambiguous.** So, if the set is unambiguous, the rule order is not significant; if two or more rules introduce an ambiguity, the first defined one is used. 

\section{Un exemple}

This is very simple scanner, from ''galgas/samples/notSLRgrammar.ggs'':

|''**lexique** my\_scanner\_for\_not\_SLR\_grammar:\\ 
\#--- Identifiers\\ 
\\$id\\$ error **message** %%"%%an identifier%%"%% ;\\ 
**rule** \textquotesingle{a}\textquotesingle -> \textquotesingle{z}\textquotesingle | \textquotesingle{A}\textquotesingle -> \textquotesingle{Z}\textquotesingle :\\ 
 send \\$id\\$ ;\\ **end** **rule** ;\\ 
\#--- Delimitors\\ 
**list** delimitorsList error **message** %%"%%the %%'"%% . * . %%"'%% delimitor%%"%%: %%"%%=%%"%% , %%"%%*%%"%% ;\\ 
**rule** **list** delimitorsList ;\\ 
\#--- Separators\\ 
**rule** \textquotesingle\textbackslash{1}\textquotesingle -> %%' '%%:\\ 
**end** **rule** ;\\ 
**end** **lexique** ;''|

This \ggs+lexique+ component defines the following set of terminal symbols: ''\\$id\\$'' (explicitly declared), ''\\$=\\$'' and ''\\$*\\$'' (declared  by ''delimitorsList'' list.

The first rule sends the ''\\$id\\$'' terminal symbol each time a lower case or upper case character is found. The second rule names the ''delimitorsList'' list and sends the ''\\$=\\$'' or ''\\$*\\$'' terminal symbol each time the corresponding character is found. The last rule discards silently the space character and any control character.

Note that this scanner considers identifiers of only one character: ''ab'' is scanned as two consecutive identifiers.

===== Finding Sample Code =====

You can find examples of \ggs+lexique+ components in:
  * ''galgas/sample/alt\_sample.ggs'' file; this is a very basic scanner that handles one-letter identifier and four delimitors;
  * ''galgas/sample/arith\_expression.ggs'' file (for scanning literal integers); 
  * ''galgas/sample/test\_LR1\_grammar.ggs'' file gives an example of a small scanner for "toy" parser;
  * ''galgas/galgas/galgas\_sources/galgas\_scanner.ggs'' file: this is the actual scanner of the GALGAS language, and scans identifiers, keywords, delimiters, literal integers, literal characters, literal character strings, galgas type names (the '@' character followed by a sequence of letters), comments, ...   

\section{Déclarations lexicales}

\subsection{Déclaration d'un symbole terminal}

The //single terminal symbol declaration// declares a name used for naming a terminal symbol. This declaration just performs declaration, not scanning. For sending this terminal symbol to the parser, it must be named in a ''send'' lexical instruction within a lexical rule.

The declaration associates to the terminal symbol a possibly empty list of lexical attributes and a syntax error message (not a //lexical// error message), defined by a character string.

First example:

|''\$literal\_integer\$ error **message** %%"%%a decimal number%%"%%;''|

This declaration names no lexical attribute. Consequently, when the lexical send instruction ''send \$literal\_integer\$;'' will be called from a lexical rule, only the terminal symbol will be sent to the parser, but not the literal integer value. The parser has no way to get the actual value: all integer values share the same terminal symbol. It is sufficient for a pure parser, however a real compiler needs the actual value.

Second example:

|''@uint unsignedValueAttribute;\\ 
\$literal\_integer\$ !unsignedValueAttribute error **message** %%"%%a decimal number%%"%%;''|

In this declaration, the ''unsignedValueAttribute'' attribute is named in the terminal symbol declaration. So, when the lexical send instruction ''send \$literal\_integer\$;'' will be called from a lexical rule, the terminal symbol will be sent to the parser together with the unsigned value of the ''unsignedValueAttribute'' attribute, enabling the semantic instructions to catch it.

\subsection{Déclaration d'une liste de symboles terminaux}

The //list of terminal symbol declaration// associates to a name a list of terminal symbols with a generic syntax error message. It is typically used for declaring the keywords and the delimiters.

An example of key words declaration:

| ''**list** keywordList error **message** %%"%%the '%K' key word%%"%%: %%"%%if%%"%%, %%"%%then%%"%%, %%"%%else%%"%% ;'' |

The declared terminal symbols are: ''\$if\$'', ''\$then\$'', ''\$else\$''. The actual syntax error message is built from generic error message by replacing ''%K'' with terminal symbol string (for outputing a single ''%'', write ''%''''%''). So the syntax error message associated to the ''\$if\$'' terminal symbol is: "''the 'if' key word''".

An other example is a delimitor list declaration:

|''**list** delimitorList error **message** %%"%%the '%K' delimitor%%"%%: %%"%%.%%"%%, %%"%%;%%"%%, %%"%%(%%"%%, %%"%%)%%"%% ;''|

Actual scanning of a delimitor is done by a ''**rule** **list**'' lexical instruction.

\subsection{Déclaration d'un attribut terminal}

Lexical attributes carry values associated with terminal symbol. GALGAS handles string, unsigned, character, float lexical attributes. Every lexical attribute needs to be declared and its declaration names a GALGAS type name.


 The following table summerizes the attributes features and type notation:

%\^ Attribute Type \^ Type Name \^ Default Value \^ Corresponding C++ type \^
| ASCII String | ''@string'' | ''%%""%%'' (the empty string) | ''C\_String'' |
| ASCII Character | ''@char'' | ''%%'\0'%%'' | ''char'' |
| 32-bit Unsigned Integer | ''@uint'' | ''0'' | ''uint32'' |
| 32-bit Signed Integer | ''@sint'' | ''0'' | ''sint32'' |
| Float | ''@double'' | ''0.0'' | ''double'' |

In GALGAS, type names are identifiers prefixed by a ''@'' character.

An ''@string'', ''@char'', ''@uint'', ''@sint'', ''@double'' lexical attribute carry a string, character, unsigned, signed, double value.

In a ''**syntax**'' component, information that defines the location of the scanned terminal symbol in the input string is added to attribute value: so an ''@string'' object in the lexique component corresponds to an ''@lstring'' object in the syntax component. Location information is used by the parser and the semantic instructions for building syntax and semantic error messages that indicates //where// the error is located.

The //default value// is the one used at the beginning of every scanning loop for resetting lexical attribute.

The //corresponding C type// is useful if you want to write your own lexical actions (in C++). Please note that this correspondance is **only** available for lexical actions, and not for semantic action. The ''C\_String'' type is a C++ class that handles mutable character strings, without being worried about memory management. It is declared in the ''libpm/strings/C\_string.h'' file. The ''uint32'' type is the 32-bit unsigned integer type, and the ''sint32'' type is the 32-bit signed integer type. 
 

\subsection{Déclaration d'un message d'erreur lexicale}

The //lexical error message declaration// associates a name to a string. These error messages are used in lexical actions, and define the message that are displayed when a lexical error occurs.

|  ''**message** decimalNumberTooLarge: %%"%%decimal number too large%%"%%;'' |

 

\section{Règles lexicales}

There are two kinds of //lexical rules//:
  - the //list lexical rule//;
  - the //single lexical rule//.

\subsection{Règle s'appuyant sur une liste}

This is the simpliest form: it just names a previously defined list of terminal symbols; for example:

|''**rule** **list** delimitorList;''|

//Matching expressions// are the set of strings defined by the list. This rule tries to find a substring from input string at current location that matches a terminal symbol string defined in the list, sorted by decreasing length (so longest strings are tested first). On match success, //executing the rule// consists of sending the corresponding terminal symbol.

This kind of rule is typically used for scanning for a delimitor.

\subsection{Simple règle}

A //single lexical rule// has the following form:

|''**rule** //matching\_expression//:\\  //lexical\_instructions//\\ **end** **rule**;''|

The //matching expression// defines a set of matching strings, that are tested against the substring from input string at current location. On match, the //lexical instructions// are executed.

A matching expression can be:
  - a one-character string (for example, ''\textquotesingle{a}\textquotesingle'' matches the ''a'' character);
  - an union of one-character strings, defined by a character subrange (for example, ''\textquotesingle{a}\textquotesingle -> \textquotesingle{z}\textquotesingle'' matches a lower case letter);
  - a one or more characters string (for example, ''%%"%%=%%"%%'' matches the corresponding string);
  - an union of above (for example: ''\textquotesingle{A}\textquotesingle -> \textquotesingle{Z}\textquotesingle | \textquotesingle{a}\textquotesingle -> \textquotesingle{z}\textquotesingle'' matches a lower or upper case letter).

On match success, the current location is moved to designate the character after the matching string.

\section{Instructions lexicales}


\subsectionLabel{Instruction lexicale \texttt{select}}{instructionSelectLexical}

The //lexical select instruction// is the following:

|''**select**\\ **when** //matching\_expression\_1\_in\_select//: //lexical\_instructions\_1//\\ **when** //matching\_expression\_2\_in\_select//: //lexical\_instructions\_2//\\ ...\\ default //default\_lexical\_instructions//\\ **end** **select**;''|

A //lexical select instruction// has one or more ''**when**'' branches.

//matching expression\_1\_in\_select//, //matching expression\_2\_in\_select// conform to the defined above //matching\_expression//.

This instruction tries to match the different //matching expressions// until a matching success is found. In such case, the corresponding //lexical instructions// are executed. If all matching fail, the //default lexical instructions// are executed.

\subsectionLabel{Instruction lexicale \texttt{repeat}}{instructionRepeatLexical}

The //lexical repeat instruction// is the following:

|''**repeat**\\  //lexical\_instructions\_0//\\ **while** //matching\_expression\_1\_in\_repeat//: //lexical\_instructions\_1//\\ **while** //matching\_expression\_2\_in\_repeat//: //lexical\_instructions\_2//\\ ...\\ **end** **repeat**;''|

A //lexical while instruction// has one or more ''**while**'' branches.

//matching expression\_1\_in\_repeat//, //matching expression\_2\_in\_repeat// can be:
  - an expression conform to the defined above //matching\_expression//;
  - the ''~ //string//'' construct: the match succeeds when the //string// **is not** the current string;
  - the ''~ //string1//, //string2//, ...'' construct: the match succeeds when neither of //string1//, //string2//, ... are the current string.

This instruction first executes the //lexical instructions 0//. Then, it tries to match the different //matching expressions// until a matching success is found. In such case, the corresponding //lexical instructions// are executed, then the instruction is executed again (from //lexical instructions 0//). If all matching fail, execution of this instruction is complete (excution goes on the next instruction).

\subsection{Appel d'une action lexicale}

The //lexical action call instruction// calls a C++ defined method for performing computation and checking on lexical attributes. Its syntax is the following:

|''lexical\_action\_name (parameter, ...) ;''|

or

|''lexical\_action\_name (parameter, ...) error message\_name, ... ;''|

A lexical action is designated by its name. It accepts one or more parameters, and zero, one or more messages names.

A parameter is:
  - either a lexical attribute,
  - either a lexical function call;
  - either the joker character ''\textquotesingle*\textquotesingle'' that represents the character at current location.

A lexical action can be predefined or defined by the user. Predefined lexical actions are actually methods of ''C\_Lexique'' class (the generated scanner is a class that inherits from this class). User defined lexical actions must be implemented as methods of the generated scanner class.

**Note that no parameter type checking, no error message count checking is performed by GALGAS. ** A parameter type error or a message count error is detected at C++ compilation stage.
 
\subsection{Appel d'une fonction lexicale}

The //lexical function call// calls a C++ defined method for performing computation on lexical attributes. It can only appear as parameter of a lexical action call or a parameter of an other lexical function call. Its syntax is the following:

|''lexical\_function\_name (parameter, ...) ;''|

A lexical function is designated by its name. It accepts one or more parameters.

A lexical function parameter is:
  - either a lexical attribute,
  - either a lexical function call;
  - either the joker character ''\textquotesingle*\textquotesingle'' that represents the character at current location.

A lexical function can be predefined or defined by the user. Predefined lexical actions are actually methods of ''C\_Lexique'' class (the generated scanner is a class that inherits from this class). User defined lexical functions must be implemented as methods of the generated scanner class.

**Note that no parameter type checking is performed by GALGAS. ** A parameter type error is detected at C++ compilation stage.
 
\subsection{Instruction lexicale \texttt{error}}

The //lexical error instruction// raises a lexical error. Its syntax is:

|''error message\_name ;''|

The //message name// is the name of a previously declared lexical error message.

\subsection{Instruction lexicale \texttt{send}}

The //lexical send instruction// sends a terminal symbol to the parser. It has several forms:

=== First Form ===

|''send terminal\_symbol ;''|

This instruction sends inconditionnaly the //terminal symbol// to the parser.

=== Second Form ===

|''send search //attribute\_name// in //lexical\_list// default terminal\_symbol ;''|

This instruction first search for //attribute name// value in the //lexical list//. If found, the corresponding terminal symbol is sent to the parser. If not found, the default //terminal symbol// is sent.

Several consecutive ''search'' are accepted, allowing sequential searching in different lists:

|''send search //attribute\_name\_1// in //lexical\_list\_1// default search //attribute\_name\_2// in //lexical\_list\_2// default terminal\_symbol ;''|

\subsectionLabel{Instruction lexicale \texttt{drop}}{instructionLexicaleDrop}

|Available in GALGAS 1.5.6 and later.|


The //lexical drop instruction// does not send any terminal symbol to the parser. It is only significant for lexical coloring (see [[\#coloring\_comments|coloring comments]]).

This instruction names a terminal symbol:
|''**drop** //terminal\_symbol// ;''|


\subsection{Instruction lexicale \texttt{tag}}

|Available in GALGAS 1.5.6 and later.|

This instruction declares a new //tag identifier//.

|''**tag** //tag\_identifier// ;''|

A ''**tag**'' instruction records a location in the scanned file. The only way to use the declared tag identifier is the [[\#lexical\_rewind\_instruction|lexical rewind instruction]].

\subsection{Instruction lexicale \texttt{rewind}}

|Available in GALGAS 1.5.6 and later.|

|''**rewind** //tag\_identifier// send //terminal\_symbol//;''|

This instruction rewinds the scanned location from the tag identifier value, and sends the terminal symbol to the parser.








\section{Routines lexicales prédéfinies}

Lexical routine calls are instructions. Lexical function calls can appear as actual output parameters of routine calls and function calls. GALGAS predefines several lexical routines and several lexical functions (listed below).

A lexical routine accepts:
  * zero, one or more input/output or input formal arguments;
  * zero, one or more error messages.

Running the \texttt{-{}-print-predefined-lexical-actions} command line option lists all predefined routines and functions prototype.

\subsection{Routine \texttt{codePointToUnicode}}

\begin{galgas}
codePointToUnicode !@string inCodePointString
                   ?!@string ioString
\end{galgas}

\subsection{Routine \texttt{convertDecimalStringIntoSInt}}

\begin{galgas}
convertDecimalStringIntoSInt !@string inString
                             ?!@sint ioSignedNumber
                             error inNumberTooLargeError,
                                   inCharacterIsNotDecimalDigitError
\end{galgas}

\subsection{Routine \texttt{convertDecimalStringIntoSInt64}}

\begin{galgas}
convertDecimalStringIntoSInt64 !@string inString
                               ?!@sint64 ioSignedNumber
                               error inNumberTooLargeError,
                                     inCharacterIsNotDecimalDigitError
\end{galgas}

\subsection{Routine \texttt{convertDecimalStringIntoUInt}}

\begin{galgas}
convertDecimalStringIntoUInt !@string inString
                             ?!@uint ioUnsignedNumber
                             error inNumberTooLargeError,
                                   inCharacterIsNotDecimalDigitError
\end{galgas}

\subsection{Routine \texttt{convertDecimalStringIntoUInt64}}

\begin{galgas}
convertDecimalStringIntoUInt64 !@string inString
                               ?!@uint64 ioUnsignedNumber
                               error inNumberTooLargeError,
                                     inCharacterIsNotDecimalDigitError
\end{galgas}

\subsection{Routine \texttt{convertHTMLSequenceToUnicodeCharacter}}

\begin{galgas}
convertHTMLSequenceToUnicodeCharacter ?!@string inString
                                      ?!@char ioUnicodeCharacter
                                      error inUnassignedHTMLSequenceError
\end{galgas}

\subsection{Routine \texttt{convertHexStringIntoSInt}}

\begin{galgas}
convertHexStringIntoSInt !@string inString
                         ?!@sint ioSignedNumber
                         error inNumberTooLargeError,
                               inCharacterIsNotHexDigitError
\end{galgas}

\subsection{Routine \texttt{convertHexStringIntoSInt64}}

\begin{galgas}
convertHexStringIntoSInt64 !@string inString
                           ?!@sint64 ioSignedNumber
                           error inNumberTooLargeError,
                                 inCharacterIsNotHexDigitError
\end{galgas}

\subsection{Routine \texttt{convertHexStringIntoUInt}}

\begin{galgas}
convertHexStringIntoUInt !@string inString
                         ?!@uint ioUnsignedNumber
                         error inNumberTooLargeError,
                               inCharacterIsNotHexDigitError
\end{galgas}

\subsection{Routine \texttt{convertHexStringIntoUInt64}}

\begin{galgas}
convertHexStringIntoUInt64 !@string inString
                           ?!@uint64 ioUnsignedNumber
                           error inNumberTooLargeError,
                                 inCharacterIsNotHexDigitError
\end{galgas}

\subsection{Routine \texttt{convertStringToDouble}}

\begin{galgas}
convertStringToDouble !@string inString
                      ?!@double ioDouble
                      error inConversionError
\end{galgas}

This action tries to convert the string value of the first argument into a double value. On success, the resulting double is set to the second argument. The conversion error message is displayed on conversion error.

\subsection{Routine \texttt{convertUInt64ToSInt64}}

\begin{galgas}
convertUInt64ToSInt64 !@uint64 inUnsignedNumber
                      ?!@sint64 ioSignedNumber
                      error inNumberTooLargeError
\end{galgas}

If the unsigned value of the ''inUnsignedNumber'' argument is greater than ''2<sup>63</sup>-1'', the error is raised. Otherwise, the value is assigned to the ''ioSignedNumber'' argument.

\subsection{Routine \texttt{convertUIntToSInt}}

\begin{galgas}
convertUIntToSInt !@uint inUnsignedNumber
                  ?!@sint ioSignedNumber
                  error inNumberTooLargeError
\end{galgas}

If the unsigned value of the ''inUnsignedNumber'' argument is greater than ''2<sup>31</sup>-1'', the error is raised. Otherwise, the value is assigned to the ''ioSignedNumber'' argument.

\subsection{Routine \texttt{convertUnsignedNumberToUnicodeChar}}

\begin{galgas}
convertUnsignedNumberToUnicodeChar ?!@uint inUnsignedNumber
                                   ?!@char ioUnicodeCharacter
                                   error inUnassignedUnicodeValueError
\end{galgas}

\subsection{Routine \texttt{enterBinDigitIntoUInt}}

\begin{galgas}
enterBinDigitIntoUInt !@char inCharacter
                      ?!@uint ioUnsignedNumber
                      error inNumberTooLargeError,
                            inCharacterIsNotBinDigitError
\end{galgas}

\subsection{Routine \texttt{enterBinDigitIntoUInt64}}

\begin{galgas}
enterBinDigitIntoUInt64 !@char inCharacter
                        ?!@uint64 ioUnsignedNumber
                        error inNumberTooLargeError,
                              inCharacterIsNotBinDigitError
\end{galgas}

\subsection{Routine \texttt{enterCharacterIntoCharacter}}

\begin{galgas}
enterCharacterIntoCharacter ?!@char ioCharacter
                            !@char inCharacter
\end{galgas}

This routine performs ''ioCharacter = inCharacter'' assignment.

\subsection{Routine \texttt{enterCharacterIntoString}}

\begin{galgas}
enterCharacterIntoString ?!@string ioString
                         !@char inCharacter
\end{galgas}

Appends the character value of the second argument to the string value of the first argument. The resulting string is set to the first argument.

\subsection{Routine \texttt{enterDigitIntoASCIIcharacter}}

\begin{galgas}
enterDigitIntoASCIIcharacter ?!@char ioASCIICharacter
                             !@char inDecimalDigitCharacter
                             error inErrorCodeGreaterThan255,
                                   inErrorNotDecimalDigitCharacter
\end{galgas}

Build an ASCII character from its decimal definition.

First, the character value of the ''inDecimalDigitCharacter'' argument is tested to be a valid decimal digit, that is in one range ''[\textquotesingle0\textquotesingle, \textquotesingle9\textquotesingle]''. On failure, the ''inErrorNotDecimalDigitCharacter'' error message is displayed. On success, the unsigned value of the ''ioASCIICharacter'' argument is multiplied by ten, and is added the decimal value corresponding to second argument. If the result is lower or equal to ''2<sup>8</sup>-1'', it is set to the ''ioASCIICharacter'' argument. Otherwise, the ''inErrorCodeGreaterThan255'' error is raised.

Note: this lexical action treats characters as unsigned values.

\subsection{Routine \texttt{enterDigitIntoUInt}}

\begin{galgas}
enterDigitIntoUInt !@char inDecimalDigitCharacter
                   ?!@uint ioUnsignedNumber
                   error inNumberTooLargeError,
                         inCharacterIsNotDecimalDigitError
\end{galgas}

First, the value of ''inDecimalDigitCharacter'' argument is tested to be in the range ''[\textquotesingle0\textquotesingle, \textquotesingle9\textquotesingle]''. On failure, the ''inCharacterIsNotDecimalDigitError'' error message is displayed. On success, the unsigned value of the first argument is multiplied by ten, and is added the decimal value corresponding to the ''ioUnsignedNumber'' argument. If the result is lower or equal to ''2<sup>32</sup>-1'', it is set to the ''ioUnsignedNumber'' argument. Otherwise, the ''inNumberTooLargeError'' error is raised.

\subsection{Routine \texttt{enterDigitIntoUInt64}}

\begin{galgas}
enterDigitIntoUInt64 !@char inDecimalDigitCharacter
                     ?!@uint64 ioUnsignedNumber
                     error inNumberTooLargeError,
                           inCharacterIsNotDecimalDigitError
\end{galgas}

First, the value of ''inDecimalDigitCharacter'' argument is tested to be in the range ''[\textquotesingle0\textquotesingle, \textquotesingle9\textquotesingle]''. On failure, the ''inCharacterIsNotDecimalDigitError'' error message is displayed. On success, the unsigned value of the first argument is multiplied by ten, and is added the decimal value corresponding to the ''ioUnsignedNumber'' argument. If the result is lower or equal to ''2<sup>64</sup>-1'', it is set to the ''ioUnsignedNumber'' argument. Otherwise, the ''inNumberTooLargeError'' error is raised.

\subsection{Routine \texttt{enterHexDigitIntoASCIIcharacter}}

\begin{galgas}
enterHexDigitIntoASCIIcharacter ?!@char ioASCIICharacter
                                !@char inHexDigitCharacter
                                error inErrorCodeGreaterThan255,
                                      inErrorNotHexDigitCharacter
\end{galgas}

Build an ASCII character from its hexadecimal definition.

First, the character value of the ''inHexDigitCharacter'' argument is tested to be a valid hexadecimal digit, that is in one of the ranges ''[\textquotesingle0\textquotesingle, \textquotesingle9\textquotesingle]'', ''[\textquotesingle{a}\textquotesingle, \textquotesingle{f}\textquotesingle]'', ''[\textquotesingle{A}\textquotesingle, \textquotesingle{F}\textquotesingle]''. On failure, the ''inErrorNotHexDigitCharacter'' error message is displayed. On success, the unsigned value of the first argument is multiplied by sixteen, and is added the hexadecimal value corresponding to ''ioASCIICharacter'' argument. If the result is lower or equal to ''2<sup>8</sup>-1'', it is set to the ''ioASCIICharacter'' argument. Otherwise, the ''inErrorCodeGreaterThan255'' error is raised.

Note: this lexical action treats characters as unsigned values.

\subsection{Routine \texttt{enterHexDigitIntoUInt}}

\begin{galgas}
enterHexDigitIntoUInt !@char inHexDigitCharacter
                      ?!@uint ioUnsignedNumber
                      error inNumberTooLargeError,
                            inCharacterIsNotHexDigitError
\end{galgas}

First, the character value of the ''inHexDigitCharacter'' argument is tested to be a valid hexadecimal digit, that in one of the the ranges ''[\textquotesingle0\textquotesingle, \textquotesingle9\textquotesingle]'', ''[\textquotesingle{a}\textquotesingle, \textquotesingle{f}\textquotesingle]'', ''[\textquotesingle{A}\textquotesingle, \textquotesingle{F}\textquotesingle]''. On failure, the ''inCharacterIsNotHexDigitError'' error message is displayed. On success, the unsigned value of the ''ioUnsignedNumber'' argument is multiplied by sixteen, and is added the hexadecimal value corresponding to second argument. If the result is lower or equal to ''2<sup>32</sup>-1'', it is set to the ''ioUnsignedNumber'' argument. Otherwise, the first error is raised.

\subsection{Routine \texttt{enterHexDigitIntoUInt64}}

\begin{galgas}
enterHexDigitIntoUInt64 !@char inHexDigitCharacter
                        ?!@uint64 ioUnsignedNumber
                        error inNumberTooLargeError,
                              inCharacterIsNotHexDigitError
\end{galgas}

First, the character value of the ''inHexDigitCharacter'' argument is tested to be a valid hexadecimal digit, that in one of the the ranges ''[\textquotesingle0\textquotesingle, \textquotesingle9\textquotesingle]'', ''[\textquotesingle{a}\textquotesingle, \textquotesingle{f}\textquotesingle]'', ''[\textquotesingle{A}\textquotesingle, \textquotesingle{F}\textquotesingle]''. On failure, the ''inCharacterIsNotHexDigitError'' error message is displayed. On success, the unsigned value of the ''ioUnsignedNumber'' argument is multiplied by sixteen, and is added the hexadecimal value corresponding to second argument. If the result is lower or equal to ''2<sup>64</sup>-1'', it is set to the ''ioUnsignedNumber'' argument. Otherwise, the first error is raised.

\subsection{Routine \texttt{enterOctDigitIntoUInt}}

\begin{galgas}
enterOctDigitIntoUInt !@char inString
                      ?!@uint ioUnsignedNumber
                      error inNumberTooLargeError,
                            inCharacterIsNotOctDigitError
\end{galgas}

\subsection{Routine \texttt{enterOctDigitIntoUInt64}}

\begin{galgas}
enterOctDigitIntoUInt64 !@char inString
                        ?!@uint64 ioUnsignedNumber
                        error inNumberTooLargeError,
                              inCharacterIsNotOctDigitError
\end{galgas}

\subsection{Routine \texttt{multiplyUInt}}

\begin{galgas}
multiplyUInt !@uint inUnsignedNumber
             ?!@uint ioUnsignedNumber
             error inResultTooLargeError
\end{galgas}

Multiply the ''ioUnsignedNumber'' value by ''inUnsignedNumber'' value. Detection of overflow is performed.

\subsection{Routine \texttt{multiplyUInt64}}

\begin{galgas}
multiplyUInt64 !@uint inUnsignedNumber
               ?!@uint64 ioUnsignedNumber
               error inResultTooLargeError
\end{galgas}

Multiply the ''ioUnsignedNumber'' value by ''inUnsignedNumber'' value. Detection of overflow is performed.

\subsection{Routine \texttt{negateSInt}}

\begin{galgas}
negateSInt ?!@sint ioNumber
           error inNumberTooLargeError
\end{galgas}

\subsection{Routine \texttt{negateSInt64}}

\begin{galgas}
negateSInt64 ?!@sint64 ioNumber
             error inNumberTooLargeError
\end{galgas}


\subsection{Routine \texttt{resetString}}

\begin{galgas}
resetString ?!@string ioString
\end{galgas}








\section{Fonctions lexicales prédéfinies}


A lexical function accepts:
  * zero, one or more input formal arguments.

Running the \texttt{-{}-print-predefined-lexical-actions} command line option lists all predefined routines and functions prototype.

\subsection{Fonction \texttt{toLower}}

\begin{galgas}
toLower ?@char inCharacter -> @char
\end{galgas}

If the character value of the argument is an upper case letter, this function returns the corresponding lower case letter. Otherwise, it returns the unchanged character value of the argument.

\subsection{Fonction \texttt{toUpper}}

\begin{galgas}
toUpper ?@char inCharacter -> @char
\end{galgas}


If the character value of the argument is an lower case letter, this function returns the corresponding upper case letter. Otherwise, it returns the unchanged character value of the argument.



\section{Définir vos propres actions et fonctions lexicales}

You can define your own lexical actions and functions in C++ and make them available to called by lexical action call instructions.

\subsection{Où ?}

You must define your lexical actions and functions as a method of the C++ class generated by compilation of the \ggs+lexique+ component. You need to modify the generated code, adding method prototype declaration in class declaration.

**So that the method declaration that you added is not deleted at the time of a future compilation, define it in user zone 2 of the generated header file.** For more details, see [[generated\_files |file generation process page]].

For implementing your method, you can insert it in user zone 2 of the generated implementation file (for more details, see [[generated\_files |file generation process page]]). Alternatively, you can implement it in any other file, provided you include the needed header files.

\subsection{Correspondance entre les appels d'actions GALGAS et C++}

This table gives the correspondance between lexical argument types and C++ types. **Note this correspondance is only available for lexical arguments**.

%\^Lexical Formal Argument Type  \^C++ Type  \^
|''? @string''  |''**const** C\_String \&''|
|''?! @string''  |''C\_String \&''|
|''? @char''  |''**const** **char**''|
|''?! @char''  |''**char** \&''|
|''? @uint''  |''**const** uint32''|
|''?! @uint''  |''uint32 \&''|
|''? @sint''  |''**const** sint32''|
|''?! @sint''  |''sint32 \&''|
|''? @double''  |''**const** **double**''|
|''?! @double''  |''**double** \&''|

''?'' means the formal argument has input passing mode: it cannot be modified by the lexical action. ''?!'' means the formal argument has in/out passing mode: its value is got from the caller, can modified by the lexical action and is returned to the caller.

An error message argument corresponds to the C++ type ''**const** **char** *''.

In C++ generated code, the method call instruction generated by lexical action call names the lexical action name, prefixed by ''scanner\_routine\_''.

For example, consider the ''convertStringToDouble'' lexical action described below. This corresponds to the following method prototype:

''**void** scanner\_routine\_convertStringToDouble (**const** C\_String \&, **double** \&, **const char** *) ;''
==== Defining Action and Function Prototype ====

The prototype must conform to the rules presented in the [[\#Correspondance between Lexical Action Calls and C++ Called Methods|above]] section.

%\^Remember that GALGAS does not perform any checking on lexical action calls. Errors are detected at C++ compilation stage.\^

\section{Exemples d'analyseurs lexicaux}

\subsection{Analyser des identificateurs}

|''@string identifierString;\\ 
\$identifier\$ !identifierString error **message** %%"%%an identifier%%"%%;\\ 
**rule** %%'a'->'z' | 'A'->'Z'%%:\\ 
 **repeat**\\ 
  enterCharacterIntoString !?identifierString !* ;\\ 
 **while** %%'a'->'z' | 'A'->'Z' | '\_' | '0'->'9'%%:\\ 
 **end** **repeat** ;\\ 
 send \$identifier\$ ;\\
**end** **rule** ;''|

|''@string identifierString;\\ 
\$identifier\$ !identifierString error **message** %%"%%an identifier%%"%%;\\ 
**rule** %%'a'->'z' | 'A'->'Z'%%:\\ 
 **repeat**\\ 
  enterCharacterIntoString !?identifierString !toLower (!*) ;\\ 
 **while** %%'a'->'z' | 'A'->'Z' | '\_' | '0'->'9'%%:\\ 
 **end** **repeat** ;\\ 
 send \$identifier\$ ;\\
**end** **rule** ;''|

\subsection{Analyser des identificateurs et des mots-clés}

|''@string identifierString;\\ 
\\ 
\$identifier\$ !identifierString error **message** %%"%%an identifier%%"%%;\\ 
\\ 
**list** keywordList error **message** %%"the '%K' key word": "begin", "else", "end"%%;\\
\\ 
**rule** %%'a'->'z' | 'A'->'Z'%%:\\ 
 **repeat**\\ 
  enterCharacterIntoString !?identifierString !* ;\\ 
 **while** %%'a'->'z' | 'A'->'Z' | '\_' | '0'->'9'%%:\\ 
 **end** **repeat** ;\\ 
 send search identifierString in keywordList  default \$identifier\$ ;\\
**end** **rule** ;''|

\subsection{Analyser des délimiteurs}

|''**list** galgasDelimitorsList **error message** %%"the '%K' delimitor"%%:\\ 
 %%"*",  "|", ",",  ".",  "<>", "::", ">",  "<",  ";",  ":",%%\\ 
 %%"-",  "(", ")",  "->", "?", "==", "?", "!",  "=", "...",%%\\ 
 %%"[",  "]", "+=", "?!", "!?", "/",  "!=", "<=", ">=", "\&",%%\\ 
 %%"++", "{", "}"%% ;\\ 
\\ 
**rule list** galgasDelimitorsList ;''|

\subsection{Analyser des séparateurs}

|''**rule** %%'\u0001' -> ' '%% :\\ 
**end rule** ;''|

\subsection{Analyser des commentaires}

|''**rule** '\#' :\\ 
 **repeat**\\ 
 **while** %%'\u0001' -> '\u0009' | '\u000B' -> '\uFFFD'%% :\\ 
 **end repeat** ;\\ 
**end rule** ;''|

\subsection{Analyser des entiers décimaux non signés}

|''\$unsigned\_literal\_integer\$ !ulongValue **error message** %%"a decimal number"%% ;\\ 
\$signed\_literal\_integer\$ !longValue error **message** %%"a signed decimal number"%% ;\\ 
\\ 
**message** decimalNumberTooLarge : %%"decimal number too large"%% ;\\ 
\\ 
**message** internalError : %%"internal error"%% ;\\ 
\\ 
**rule** %%'0'->'9'%% :\\ 
 enterDigitIntoUlong !?ulongValue !* error decimalNumberTooLarge, internalError ;\\ 
 **repeat**\\ 
 **while** %%'0'->'9'%% :\\ 
  enterDigitIntoUlong !?ulongValue !* error decimalNumberTooLarge, internalError ;\\ 
 **while** %%'\_'%% :\\ 
 **end repeat** ;\\ 
 **select**\\ 
 **when** %%'S' | 's'%% :\\ 
  convertUlongToLong !?longValue !ulongValue %%error%% decimalNumberTooLarge ;\\ 
  send \$signed\_literal\_integer\$ ;\\ 
 default\\ 
  send \$unsigned\_literal\_integer\$ ;\\ 
 **end select** ;\\ 
**end rule** ;''|

\subsection{Analyser des entiers hexadécimaux non signés}

\subsection{Analyser des constantes caractère}

|''\$literal\_char\$ ! charValue **error message** %%"a character constant"%% ;\\ 
\\ 
**message** incorrectCharConstant : %%"incorrect literal character"%% ;\\ 
\\ 
**message** ASCIIcodeTooLargeError : %%"ASCII code > 255"%% ;\\ 
\\ 
**rule** %%'\''%% :\\ 
 **select**\\ 
 **when** %%'\\'%% :\\ 
  **select**\\ 
  **when** %%'f'%% :\\ 
   enterCharacterIntoCharacter !?charValue !%%'\f'%% ;\\ 
  **when** %%'n'%% :\\ 
   enterCharacterIntoCharacter !?charValue !%%'\n'%% ;\\ 
  **when** %%'r'%% :\\ 
   enterCharacterIntoCharacter !?charValue !%%'\r'%% ;\\ 
  **when** %%'t'%% :\\ 
   enterCharacterIntoCharacter !?charValue !%%'\t'%% ;\\ 
  **when** %%'v'%% :\\ 
   enterCharacterIntoCharacter !?charValue !%%'\v'%% ;\\ 
  **when** %%'\\'%% :\\ 
   enterCharacterIntoCharacter !?charValue !%%'\\'%% ;\\ 
  **when** %%'0'%% :\\ 
   enterCharacterIntoCharacter !?charValue !%%'\0'%% ;\\ 
  **when** %%'\''%% :\\ 
   enterCharacterIntoCharacter !?charValue !%%'\''%% ;\\ 
  **when** %%'0' -> '9'%% :\\ 
   **repeat**\\ 
    enterHexDigitIntoASCIIcharacter !?charValue !* error ASCIIcodeTooLargeError, internalError ;\\ 
   **while** %%'0' -> '9'%% :\\ 
   **end repeat** ;\\ 
  default\\ 
   error incorrectCharConstant ;\\ 
  **end select** ;\\ 
 **when** %%' ' -> '\uFFFD'%% :\\ 
  enterCharacterIntoCharacter !?charValue !* ;\\ 
 default\\ 
  error incorrectCharConstant ;\\ 
 **end select** ;\\ 
 **select**\\ 
 **when** %%'\''%% :\\ 
  send \$literal\_char\$ ;\\ 
 default\\ 
  error incorrectCharConstant ;\\ 
 **end select** ;\\ 
**end rule** ;''|

\subsection{Analyser des constantes chaîne de caractères}

\subsection{Analyser des constantes flottantes}

|''\$literal\_double\$ !floatValue !tokenString **error message** %%"a float number"%%;\\ 
\\ 
\$.\$ **error message** %%"the '.' delimitor"%%;\\ 
\\ 
**message** floatNumberConversionError : %%"invalid float number"%% ;\\ 
\\ 
**rule** %%'.'%% :\\ 
 **select**\\ 
 **when** %%'0'->'9'%% :\\ 
  enterCharacterIntoString !?tokenString !%%'0'%% ;\\ 
  enterCharacterIntoString !?tokenString !%%'.'%% ;\\ 
  enterCharacterIntoString !?tokenString !* ;\\ 
  **repeat**\\ 
  **while** %%'0'->'9'%% :\\ 
   enterCharacterIntoString !?tokenString !* ;\\ 
  **while** %%'\_'%% :\\ 
  **end repeat** ;\\ 
  convertStringToDouble !tokenString !?floatValue error floatNumberConversionError ;\\ 
  send \$literal\_double\$ ;\\ 
 default\\ 
  send \$.\$ ;\\ 
 **end select** ;\\
**end rule** ;''|

\section{\emph{Back tracking} avec les instructions \texttt{tag} et \texttt{rewind}}

|Available in GALGAS 1.5.6 and later.|

The ''**tag**'' and ''**rewind**'' instructions can be used for performing back tracking.

The first example is the way the non terminal symbols are scanned in GALGAS 1.5.6 (and later).

A non terminal is composed of a single '<' character, followed by a letter, zero, one or more letters, digits or underscore characters, is ended by a single '>' character. For example ''<abcdef>'' is a valid non terminal. However, ''<abcdef >'' is //not// a valid non terminal (because of the space before the final '>' character): it is considered as a '<' delimitor, followed by the ''abcdef'' identifier and by the '>' delimitor.

In the file ''galgas/galgas\_sources/galgas\_scanner.ggs'', the three delimitors befgging with a '<' character and the non terminal symbols are scanned by the following code:

''\$<\$ **error message** "the '<' delimitor" **style** delimitersStyle ;''\\
''%%\$<=\$%% **error message** "the '<=' delimitor" **style** delimitersStyle ;''\\
''%%\$<<\$%% **error message** "the '<<' delimitor" **style** delimitersStyle ;''\\
''\$non\_terminal\_symbol\$ ! tokenString **error message** "a non terminal symbol <...>" **style** nonTerminalStyle ;''\\

''**rule** '<' :''\\
'' **tag** onlyInfDelimiter ;''\\
'' **select**''\\
'' **when** '=' :''\\
'' send %%\$<=\$%% ;''\\
'' **when** '<' :''\\
''  send %%\$<<\$%% ;''\\
'' **when** %%'a' -> 'z' | 'A' ->'Z'%% :''\\
''  **repeat**''\\
''   enterCharacterIntoString !?tokenString !* ;''\\
''  **while** %%'a' -> 'z' | 'A' ->'Z' | '0' -> '9' | '\_'%% :''\\
''  **end repeat** ;''\\
''  **select**''\\
''  **when** '>' :''\\
''   send \$non\_terminal\_symbol\$ ;''\\
''  default''\\
''   **rewind** onlyInfDelimiter send \$<\$ ;''\\
''  **end select** ;''\\
'' default''\\
''  send \$<\$ ;''\\
'' **end select** ;''\\
''**end rule** ;''\\

The ''**tag**'' instruction records a scanning location. When the final '>' character is not found, the scanner is rewinded at the character following the '<' character, and the ''\$<\$'' terminal is sent. On next scanning, an identifier (or a key word) will be found.

The second examples shows how to scan for integer constants, float constants, and array bounds in Pascal :
  * an integer constant is a (non empty) sequence of digits ;
  * a float constant is a (non empty) sequence of digits, following by a dot and a (possibly empty) sequence of digits;
  * an array bound is an integer constant, followed by the '..' delimitor (two dots) and an integer constant.

The problem is that ''1..2'' should not be scanned as a float constant, a single dot delimitor, and an integer constant.

This can be achieved by the following code:

''**rule** %%'0' -> '9'%% :''\\
'' **repeat**''\\
'' **while** %%'0' -> '9'%% :''\\
'' **end repeat** ;''\\
'' **tag** endOfIntegerConstant ;''\\
'' **select**''\\
'' **when** %%'.'%% :''\\
''  **select**''\\
''  **when** %%'.'%% :''\\
''   **rewind** endOfIntegerConstant send \$integer\_constant\$ ;''\\
''  **when** %%'0' -> '9'%% :''\\
''   **repeat**''\\
''   **while** %%'0' -> '9'%% :''\\
''   **end repeat** ;''\\
''   send \$float\_constant\$ ;''\\
''  default''\\
''   send \$float\_constant\$ ;''\\
''  **end select** ;''\\
'' default''\\
''  send \$integer\_constant\$ ;''\\
'' **end select** ;''\\
''**end rule** ;''\\


\section{Ajouter la coloration lexicale (sur Mac uniquement)}

With GALGAS, you can easily embbed your compiler in a GUI application (currently available only for Mac OS X). This application has a built-in text editor, from which you can modify, save and compile source file. With //style declarations//, you can add automatic coloring in the built-in text editor.

A //style declaration// associates a message to a style identifier. For example:

|''**style** keywordsStyle -> %%"%%Keywords:%%"%% ;''|

The associated message is used in application preferences window as a comment of each color selection item.

A //style declaration// does not link a style identifier to any terminal symbol. You need to add this information to //single terminal symbol declaration// and //list of terminal symbols declaration// by naming the style identifier after the syntax error message:

|''\$literal\_integer\$ error **message** %%"%%a decimal number%%"%% **style** integerStyle;''|

|''**list** delimitorList error **message** %%"%%the '%%"%% . * . %%"%%' delimitor%%"%% **style** keywordsStyle: %%"%%.%%"%%, %%"%%;%%"%%, %%"%%(%%"%%, %%"%%)%%"%%;''|

\subsection{Exemple : les styles de l'analyseur lexical GALGAS}

As an example, you can take a look on GALGAS scanner, in ''galgas/galgas\_sources/galgas\_scanner.ggs'' file. The style declarations are the following:

|''**style** keywordsStyle -> %%"%%Keywords:%%"%% ;\\ **style** delimitersStyle -> %%"%%Delimiters:%%"%% ;\\ **style** terminalStyle -> %%"%%Terminal symbols:%%"%% ;\\ **style** integerStyle -> %%"%%Integer constants:%%"%% ;\\ **style** characterStyle -> %%"%%Character constants:%%"%% ;\\ **style** stringStyle -> %%"%%String constants:%%"%% ;\\ **style** typeNameStyle -> %%"%%Type names (@...):%%"%%'';|

You can search for the occurrence of style identifiers, to see how they are used.

In Cocoa GALGAS application, the Color tab of the Preferences window lists all style comments, each of them being associated to a ''NSColorWell'' for color selection:

{{cocoa\_galgas\_color\_styles.png}}

Note that no default color is defined in style declaration. Until you define yourself a color from Preference window, it defaults to black color.

\subsection{Appliquer un style aux commentaires}
|Available in GALGAS 1.5.6 and later.|

In GALGAS 1.5.6 and later, you can define a color for comments. Proceed as follows:
  - declare a new terminal symbol, for example ''\$comment\$'';
  - declare a style for this new terminal symbol;
  - when a comment is scanned, use the ''**drop**'' instruction for naming the new terminal symbol (instead of the usual ''send'' instruction).

The ''**drop**'' instruction is only significant for syntax coloring.

For example, GALGAS comments are defined in ''galgas/galgas\_sources/galgas\_scanner.ggs'' in this way:

''**style** commentStyle -> "Comments:" ;''\\
''...''\\
''\$comment\$ error **message** %%"%%a comment%%"%% **style** commentStyle ;''\\
''**rule** %%'\#'%% :''\\
'' **repeat**''\\
'' **while** %%'\u0001' -> '\u0009' | '\u000B' | '\u000C' | '\u000E' -> '\uFFFD'%% :''\\
'' **end repeat** ;''\\
'' **drop** \$comment\$ ;''\\
''**end rule** ;''\\


%!TEX encoding = UTF-8 Unicode
%!TEX root = ../galgas-book.tex

%--------------------------------------------------------------
\chapterLabel{Le composant \texttt{option}}{composantOption}
%-------------------------------------------------------------


Le composant \galgas{option} permet de définir des options qui sont appelables à partir de la ligne de commande. Dans le code, la valeur d'une option est obtenue à partir de l'opérande \emph{appel d'une option}, décrit dans la \refSubsectionPage{appelOption}.

Voici l'exemple d'un composant \galgas{option} qui déclare une option (évidement, un composant \galgas{option} peut déclarer un nombre quelconque d'options) :
\begin{galgascode}
option nom_composant {
  @bool nom_option : 'S', "asm" -> "Extract assembly code"
}
\end{galgascode}


\section{Déclaration d'une option}

La déclaration d'une option présente le syntaxe suivante :
\begin{galgascode}
  @T nom_option : caractere, chaine -> description
\end{galgascode}

Les cinq champs qui définissent une option sont :
\begin{itemize}
  \item \galgas{@T} : le type de l'option ; trois types sont autorisés : \galgas{@bool}, \galgas{@uint} et \galgas{@string} ;
  \item \galgas{nom_option} : c'est le nom, interne à GALGAS, qui permettra de désigner l'option dans l'\emph{appel d'une option} (\refSubsectionPage{appelOption}) ; 
  \item \galgas{caractere} : le caractère qui activera l'option dans la ligne de commande ; par exemple, en écrivant \galgas{'A'}, l'option sera activée par \texttt{-A} dans la ligne de commande ; si vous ne voulez pas d'activation par un caractère, écrivez \galgas{'\\0'} ;
  \item \galgas{chaine} : la chaîne de caractères qui activera l'option dans la ligne de commande ; par exemple, en écrivant \galgas{"ABEDEF"}, l'option sera activée par \texttt{-{}-ABCDEF} dans la ligne de commande ; si vous ne voulez pas d'activation par une chaîne, écrivez \galgas{""} ;
  \item \galgas{description} : une chaîne de caractère qui contient une description de l'option, qui sera affichée par l'option \texttt{-{}-help} de votre compilateur.
\end{itemize}








\section{Option booléenne}

Le champ qui définit le type de l'option est \galgas{@bool} ; par exemple :
\begin{galgascode}
  @bool nom_option : 'S', "asm" -> "Extract assembly code"
\end{galgascode}

Dans la ligne de commande, l'option est activée par \texttt{-A} ou \texttt{-{}-asm}.

Par défaut, l'option n'est pas activée, et sa valeur associée est \galgas{false}. Quand l'option est activée dans la ligne de commande, sa valeur associée est \galgas{true}.








\section{Option entière}

Le champ qui définit le type de l'option est \galgas{@uint} ; par exemple :
\begin{galgascode}
  @uint nom_option : 'M', "max-iterations-count" -> "Max of iteration count"
\end{galgascode}

Dans la ligne de commande, l'option est activée par \texttt{-N=xxx} ou \texttt{-{}-max-iterations-count=xxx}, où \texttt{xxx} est un nombre entier positif ou nul (et inférieur ou égal à $2^{32}-1$).

Par défaut, l'option n'est pas activée, et sa valeur associée est $0$. Quand l'option est activée dans la ligne de commande, sa valeur associée est la valeur \texttt{xxx}. Ainsi, l'option \texttt{-N=0}, comme l'option \texttt{-{}-max-iterations-count=0} n'a aucun effet.










\section{Option chaîne de caractères}

Le champ qui définit le type de l'option est \galgas{@string} ; par exemple :
\begin{galgascode}
  @string nom_option : 'F', "file-name" -> "File name"
\end{galgascode}

Dans la ligne de commande, l'option est activée par \texttt{-F=abc} ou \texttt{-{}-file-name=abc}, où \texttt{abc} est une chaîne de caractères sans espaces. Si vous voulez entrer une chaîne de caractères qui comprend des espaces, écrivez : \texttt{"-F=abc"} ou \texttt{"-{}-file-name=abc"}.

Par défaut, l'option n'est pas activée, et sa valeur associée est la chaîne vide. Quand l'option est activée dans la ligne de commande, sa valeur associée est la chaîne \texttt{abc}. Ainsi, l'option \texttt{-F=}, comme l'option \texttt{-{}-file-name=} n'a aucun effet.





%--------------------------------------------------------------
\chapter{Predefined Types} \label{predefinedTypes}
%-------------------------------------------------------------

GALGAS predefines several types. This chapter presents all their features, including their constructors, readers, modifiers, methods, ...


\begin{description}
\item The predefined types are:
\begin{itemize}
\item \lienSectionType{@location}, whose value points out a location in a source file;
\item \lienSectionType{@uint}, the 32-bit unsigned integers.
\end{itemize}
\end{description}

%!TEX encoding = UTF-8 Unicode
%!TEX root = ../galgas-book.tex

\chapitreTypePredefiniLabelIndex{location}

An \galgas{@location} object value is a location in a source file. Objects of this type are useful for pointing out an error or a warning location.

\section{The \texttt{here} Keyword}

The \galgas{here} keyword indicates the current parsing location is the current source file. Assigning an \galgas{@location} object from the \galgas{here} keyword is a way for initializing an \galgas{@location} object:\newline

\texttt{@location currentLocation := here ;}

\section{Constructor}

\constructeurSansArgument{nowhere}
{location}
{2.1.2}
{location}
{Returns an \galgas{@location} that does not points out any location.}
{The returned object responds \galgas{true} to the \refReaderPage{location}{isNowhere}.}

\section{Readers}

\readerSansArgument{column}
{location}
{1.8.2}
{uint}
{Returns an \galgas{@uint} value containing the column of the receiver's value.}
{this reader raises a run-time error if the receiver's value responds \galgas{true} to the \refReaderPage{location}{isNowhere}.}


\readerSansArgument{isNowhere}
{location}
{2.1.2}
{bool}
{Returns an \galgas{@bool} value indicating whether the receiver'value points out a source location or does not.}
{this reader returns \galgas{true} if the receiver's value does not point out an actual location in a text source (i.e. it has been constructed using the nowhere constructor), and \galgas{false} if the receiver's value points out an actual location in a text source (i.e. it has been constructed using the \galgas{here} keyword.}


\readerSansArgument{line}
{location}
{1.8.2}
{uint}
{Returns an \galgas{@uint} value containing the line of the receiver's value.}
{this reader raises a run-time error if the receiver's value responds \galgas{true} to the \refReaderPage{location}{isNowhere}.}


\readerSansArgument{locationIndex}
{location}
{1.8.2}
{uint}
{Returns an \galgas{@uint} value containing the the offset from the the beginning of the source of the location defined by receiver's value.}
{this reader raises a run-time error if the receiver's value responds \galgas{true} to the \refReaderPage{location}{isNowhere}.}


\readerSansArgument{locationString}
{location}
{1.8.2}
{string}
{returns an \galgas{@string} object that contains a string representation of the location defined by receiver's value.}
{this reader raises a run-time error if the receiver's value responds \galgas{true} to the \refReaderPage{location}{isNowhere}.}

%!TEX encoding = UTF-8 Unicode
%!TEX root = ../galgas-book.tex

\chapitreTypePredefiniLabelIndex{uint}

\tableDesMatieresLocaleDeProfondeurRelative{1}


An \ggst+@uint+ object value is a 32-bit unsigned integer value. You can initialize an \ggst+@uint+ object from an unsigned integer constant:\\

\begin{galgas3}
@uint myUnsignedInteger = 123_456 ;
\end{galgas3}

Note that a 32-bit unsigned integer constant is characterized by no suffix.

\section{Constructors}

\subsectionConstructor{errorCount}{uint}

\begin{galgas3}
constructor errorCount -> @uint
\end{galgas3}


Returns an \ggst+@uint+ object that contains the number of errors. The returned value is the cumulative count of errors from the beginning of execution.

\textbf{Exemple :}
\begin{galgas3}
@uint x = [@uint errorCount] ;
\end{galgas3}




\subsectionConstructor{max}{uint}

\begin{galgas3}
constructor max -> @uint
\end{galgas3}

Returns an \ggst+@uint+ object that the maximum value of the 32-bit unsigned range ($2^{32}-1$).






\subsectionConstructor{random}{uint}

\begin{galgas3}
constructor random -> @uint
\end{galgas3}

Retourne une valeur aléatoire de type \ggst+@uint+. La procédure de type \refStaticProcPage{uint}{setRandomSeed} permet d'en fixer la valeur initiale.

\begin{galgas3}
  let v = @uint.random
\end{galgas3}


{\bf Note. } Sur Unix, la valeur renvoyée est la valeur renvoyée par l'appel de la fonction \texttt{random} de la librairie \texttt{libc}. Sur Windows, c'est la fonction \texttt{rand} qui est appelée.


\subsectionConstructor{valueWithMask}{uint}

\begin{galgas3}
constructor valueWithMask ?@uint inLowerIndex ?@uint inUpperIndex -> @uint
\end{galgas3}


Returns an \ggst+@uint+ object with bits from \emph{inLowerIndex} to \emph{inUpperIndex} equal to 1.

A run-time error is raised if \emph{inLowerIndex $>$ inUpperIndex} or if \emph{inUpperIndex $>$ 31}.



\textbf{Exemple :}
\begin{galgas3}
@uint x = [@uint valueWithMask !2 !4] ; # x is equal to 28 (0b1_1100)
\end{galgas3}




\subsectionConstructor{warningCount}{uint}

\begin{galgas3}
constructor warningCount -> @uint
\end{galgas3}


Returns an \ggst+@uint+ object that contains the number of warnings. The returned value is the cumulative count of warnings from the beginning of execution.





\section{Procédure de type}


\subsectionStaticProc{setRandomSeed}{uint}


\begin{galgas3box}
proc @uint setRandomSeed ?@uint inSeed
\end{galgas3box}

Affecte la valeur initiale utilisée par le générateur de nombres aléatoires (voir le \refConstructorPage{uint}{random}) Par exemple~:

\begin{galgas3}
  [@uint setRandomSeed !0]
\end{galgas3}






\section{Getters}

\subsectionGetter{alphaString}{uint}

Ce \emph{getter} permet de convertir un \ggst!@uint! en une chaîne de caractères, telle que l'ordre des entiers est conservé sur la chaîne obtenue.

La chaîne obtenue comporte exactement 7 lettres minuscules. C'est en fait une conversion en base 26, la lettre \ggst=a= ayant la valeur $0$, et la lettre \ggst=z= la valeur $25$.


\begin{galgas3}
  message [0 alphaString] + "\n"         # aaaaaaa
  message [12_345 alphaString] + "\n"    # aaaasgv
  message [@uint.max alphaString] + "\n" # nxmrlxv
\end{galgas3}



\subsectionGetter{bigint}{uint}

Ce \emph{getter} permet de convertir un \ggst!@uint! en \ggst!@bigint!. Comme la plage des valeurs des \ggst!bigint! n'est limitée que par la mémoire disponible, il n'échoue jamais.

\begin{galgas3}
  message [[1234 bigint] string] + "\n" # 1234
\end{galgas3}


\subsectionGetter{double}{uint}

\begin{galgas3}
getter double -> @double
\end{galgas3}

Returns the receiver's value converted in a \ggst+@double+ object. As a 32-bit integer value can always be converted in a \ggst+@double+ value, this getter never fails.



\subsectionGetter{hexString}{uint}

\begin{galgas3}
getter hexString -> @string
\end{galgas3}

Returns the an hexadecimal string representation of the receiver value, prefixed by the string \texttt{0x}. For getting an hexadecimal representation string without any prefix, see \refGetterPage{uint}{xString}.



\subsectionGetter{hexStringSeparatedBy}{uint}

\begin{galgas3}
getter hexStringSeparatedBy ?@char inSeparator ?@uint inGroup -> @string
\end{galgas3}

Returns the an hexadecimal string representation of the receiver value, prefixed by the string \texttt{0x}. Groups of \ggst=inGroup= digits are separated by the \ggst=inSeparator= character.

If \ggst=inGroup= is equal to zero, a run-time error is raised.

For example:
\begin{galgas3}
let s = [0x12345678 hexStringSeparatedBy !'_' !2] # 0x12_34_56_78
\end{galgas3}



\subsectionGetter{isInRange}{uint}

\begin{galgas3}
getter isInRange ?@range inRange -> @bool
\end{galgas3}

{Returns an \ggst+@bool+ value indicating whether the receiver'value belongs to \ggst+inRange+ range : for a receiver's value equal to $v$ and a range of length $length$ starting at $start$, it returns \ggst+true+ if $((v \geqslant start)~and~(v<(start+length)))$, and \ggst+false+ otherwise.



\subsectionGetter{isUnicodeValueAssigned}{uint}

\begin{galgas3}
getter isUnicodeValueAssigned -> @bool
\end{galgas3}

Returns an \ggst+@bool+ value indicating whether the receiver'value represents an assigned Unicode character. It returns \ggst+true+ if the receiver value represents an assigned Unicode character, \ggst+false+ and otherwise.

\textbf{Exemple :}
\begin{galgas3}
[0xFFFF isUnicodeValueAssigned] # is false, as \uFFFF is not assigned.
[0x41 isUnicodeValueAssigned] # is true, as \u0041 is assigned (LATIN CAPITAL LETTER A).
\end{galgas3}



\subsectionGetter{lsbIndex}{uint}

\begin{galgas3}
getter lsbIndex -> @uint
\end{galgas3}

Returns an \ggst+@uint+ value of the index of the most significant bit of the receiver value. It raises a run-time error if the receiver value is zero.

\textbf{Exemple :}
\begin{galgas3}
@uint value = 192 ; # 192 is ...011000000 in binary
@uint x = [value lsbIndex] ; # x is equal to 7
\end{galgas3}

The most significant bit of 192 is the 7th bit.




\subsectionGetter{significantBitCount}{uint}

\begin{galgas3}
getter significantBitCount -> @uint
\end{galgas3}

Returns the number of bits needed to express the receiver value. If the receiver value is zero, it returns 0 ; otherwise, it returns the most significant bit index plus one.

\textbf{Exemple :}
\begin{galgas3}
@uint value = 145 ; # 145 is 10010001 in binary
@uint x = [value significantBitCount] ; # x is equal to 8
\end{galgas3}






\subsectionGetter{sint}{uint}

\begin{galgas3}
getter sint -> @sint
\end{galgas3}

Returns the receiver's value in an \refTypePredefini{sint} (32-bit signed integer) object. An error is raised is receiver's value is greater than $2^{31}-1$.

This getter is the only way to convert an \refTypePredefini{uint} object into an \refTypePredefini{sint} object.




\subsectionGetter{sint64}{uint}

\begin{galgas3}
getter sint64 -> @sint64
\end{galgas3}

Returns the receiver's value in an \refTypePredefini{sint64} (64-bit signed integer) object. As a 32-bit unsigned value can always be converted in a 64-bit signed value, this getter never fails.

This getter is the only way to convert an \refTypePredefini{uint} object into an \refTypePredefini{sint64} object.


\subsectionGetter{string}{uint}

\begin{galgas3}
getter string -> @string
\end{galgas3}

Returns a decimal string representation of the receiver's value. For an hexadecimal string representation of the receiver's value, see \refGetterPage{uint}{hexString} and \refGetterPage{uint}{xString}.




\subsectionGetter{uint64}{uint}

\begin{galgas3}
getter uint64 -> @uint64
\end{galgas3}

Returns the receiver's value in an \refTypePredefini{uint64} (64-bit unsigned integer) object. As a 32-bit unsigned value can always be converted in a 64-bit unsigned value, this getter never fails.

This getter is the only way to convert an \refTypePredefini{uint} object into an \refTypePredefini{uint64} object.




\subsectionGetter{xString}{uint}

\begin{galgas3}
getter xString -> @string
\end{galgas3}

Returns an hexadecimal string representation of the receiver's value (without any prefix). For an decimal string representation of the receiver's value, see the \refGetterPage{uint}{hexString}; for a decimal string representation of the receiver's value, see the \refGetterPage{uint}{string}.







\section{Arithmétique}

\subsection{Opérateurs infixés}

Le type \ggst+@uint+ accepte les opérateurs arithmétiques infixés suivants :
\begin{itemize}
  \item \ggst!+!, addition, une erreur d'exécution est déclenchée en cas de débordement ;
  \item \ggst!-!, soustraction, une erreur d'exécution est déclenchée en cas de débordement ;
  \item \ggst!*!, multiplication, une erreur d'exécution est déclenchée en cas de débordement ;
  \item \ggst!/!, division, une erreur d'exécution est déclenchée si le diviseur est nul ;
  \item \ggst!mod!, calcul du reste, une erreur d'exécution est déclenchée si le diviseur est nul ;
  \item \ggst!&+!, addition, le résultat étant silencieusement tronqué sur 32 bits ;
  \item \ggst!&-!, soustraction, le résultat étant silencieusement tronqué sur 32 bits ;
  \item \ggst!&*!, multiplication, le résultat étant silencieusement tronqué sur 32 bits ;
  \item \ggst!&/!, division, qui retourne zéro si le diviseur est nul.
\end{itemize}

Ces opérateurs exigent que les deux opérandes soient des objets du même type \ggst+@uint+.

\subsection{Opérateur préfixé}
Le type \ggst+@uint+ accepte un opérateur arithmétique préfixé :
\begin{itemize}
  \item \ggst!+!, qui retourne simplement la valeur de l'opérande.
\end{itemize}

\subsectionLabel{Instructions}{instructionsUINT}

Le type \ggst+@uint+ accepte les deux instructions arithmétiques suivantes :
\begin{itemize}
  \item \ggst!+=!, addition, une erreur d'exécution est déclenchée en cas de débordement ;
  \item \ggst!-=!, soustraction, une erreur d'exécution est déclenchée en cas de débordement ;
  \item \ggst!*=!, multiplication, une erreur d'exécution est déclenchée en cas de débordement ;
  \item \ggst!/=!, division, une erreur d'exécution est déclenchée en cas division par zéro ;
  \item \ggst!++!, incrémentation, une erreur d'exécution est déclenchée en cas de débordement ;
  \item \ggst!--!, décrémentation, une erreur d'exécution est déclenchée en cas de débordement ;
  \item \ggst!&++!, incrémentation, le résultat étant silencieusement tronqué sur 32 bits ;
  \item \ggst!&--!, décrémentation, le résultat étant silencieusement tronqué sur 32 bits.
\end{itemize}

\ggst!x+=y! est équivalent à \ggst!x=x+y! ; \ggst!x-=y! est équivalent à \ggst!x=x-y!.
La variable cible \ggst!x!, comme l'expression source \ggst!y! doivent être du même type \ggst+@uint+.

Incrémentation et décrémentation sont des instructions, et ne peuvent pas apparaître des expressions.
\begin{galgas3}
@uint n = ... ; n ++ # Incrémentation
\end{galgas3}

\begin{galgas3}
@uint n = ... ; n -- # Décrémentation
\end{galgas3}




\section{Shift Operators}


The \ggst+@uint+ type supports right and left shift operators:\newline

\begin{tabular}{|c|c|}
\hline
$<<$ & Left shift \\
\hline
$>>$ & Right shift \\
\hline
\end{tabular}

Theses operators require both arguments to be \ggst+@uint+ objects.\newline

Note the right shift inserts always a zero bit in the most significant bit location (it is a logical right shift).\newline

The actual amount of the shift is the value of the right-hand operand masked by 31, i.e. the shift distance is always between 0 and 31.




\section{Logical Operators}

The \ggst+@uint+ type supports the three bit-wise logical operators:\newline

\begin{tabular}{|c|c|}
\hline
$\&$ & Bit-wise and \\
\hline
\textbar & Bit-wise or \\
\hline
\^\  & Bit-wise exclusive or \\
\hline
\end{tabular}

Theses operators require both arguments to be \ggst+@uint+ objects.\newline


The \ggst+@uint+ type supports the bit-wise logical unary operator:\newline

\begin{tabular}{|c|c|}
\hline
$\sim$ & Bit-wise complementation \\
\hline
\end{tabular}

This operator returns an \ggst+@uint+ object.







\section{Comparison Operators}

The \ggst+@uint+ type supports the six comparison operators:\newline

\begin{tabular}{|c|c|}
\hline
$=$ & Equality \\
\hline
$!=$ & Non Equality \\
\hline
$<$  & Strict Lower Than \\
\hline
$<=$  & Lower or Equal \\
\hline
$>$  & Strict Greater Than \\
\hline
$>=$  & Greater or Equal \\
\hline
\end{tabular}

\vspace{2mm}
Theses operators require both arguments to be \ggst+@uint+ objects, and return a \ggst+@bool+ object.






%!TEX encoding = UTF-8 Unicode
%!TEX root = ../galgas-book.tex

%--------------------------------------------------------------
\chapter{User Types} \label{userTypes}
%-------------------------------------------------------------


\section{List type}


\section{Sorted list type}


\section{Struct type}



\section{Class type}


\section{Map type}


\section{Map proxy type}




\section{Graph type}


\section{Predefined user types}

\subsectionTypePredefiniLabelIndex{lbool}

\subsectionTypePredefiniLabelIndex{lchar}

\subsectionTypePredefiniLabelIndex{ldouble}

\subsectionTypePredefiniLabelIndex{lsint}

\subsectionTypePredefiniLabelIndex{lsint64}

\subsectionTypePredefiniLabelIndex{lstring}

\subsectionTypePredefiniLabelIndex{luint}

\subsectionTypePredefiniLabelIndex{luint64}



%!TEX encoding = UTF-8 Unicode
%!TEX root = ../galgas-book.tex

%--------------------------------------------------------------
\chapter{Semantic Instructions}
%-------------------------------------------------------------



\sectionLabel{Append Instruction}{appendInstruction}


\sectionLabel{Assignment Instruction}{assignmentInstruction}


\section{Cast Instruction}


\sectionLabel{Concat Instruction}{concatInstruction}


\sectionLabel{Decrement Instruction}{decrementInstruction}




\section{Drop Instruction}

{\lstset{emph={variable}, emphstyle=\emph}
\begin{galgascode}
drop variable, ... ;
\end{galgascode}
}

\section{Error Instruction}


\section{Extern Action Call Instruction}




\section{L'instruction \texttt{for}}





\sectionLabel{L'instruction \texttt{foreach}}{instructionForeach}




\sectionLabel{Increment Instruction}{incrementInstruction}










\section{If Instruction}


\subsection{Syntax}

The \emph{if} instruction has the following syntax:
{\lstset{emph={expression, instructions, expression2, instructions2, else_instructions}, emphstyle=\emph}
\begin{galgascode}
if expression then
  instructions
elsif expression2 then
  instructions2
...
else
  else_instructions
end if ;  
\end{galgascode}
}

More precisely, it contains :
\begin{itemize}
\item zero, one or more \emph{elsif} branches ;
\item zero or one \emph{else} branch.
\end{itemize}


\subsection{Static semantics}


No \emph{else} branch is equivalent to an \emph{else} branch without any instruction.


The \emph{elsif} branches are just syntactic sugar: it is semantically equivalent to use embedded \emph{if} instructions instead. For example:
{\lstset{emph={expression, instructions, expression2, instructions2, else_instructions}, emphstyle=\emph}
\begin{galgascode}
if expression then
  instructions
elsif expression2 then
  instructions2
else
  else_instructions
end if ;  
\end{galgascode}
}
is equivalent to:
{\lstset{emph={expression, instructions, expression2, instructions2, else_instructions}, emphstyle=\emph}
\begin{galgascode}
if expression then
  instructions
else
  if expression2 then
    instructions2
  else
    else_instructions
  end if ;  
end if ;  
\end{galgascode}
}

So, for describing \emph{if} instruction static and dynamic semantics, we only need to describe an \emph{if} instruction without any \emph{elsif} branch and with an \emph{else} branch:
{\lstset{emph={expression, instructions, else_instructions}, emphstyle=\emph}
\begin{galgascode}
if expression then
  instructions
else
  else_instructions
end if ;
\end{galgascode}
}

The static semantics evaluates the \emph{expression} type, and applies the following rules until success:
\begin{enumerate}
\item the \emph{expression} type is \refTypePredefini{bool};
\item the \emph{expression} type is an \emph{structure} type, it has a attribute named \emph{bool}, whose type is \refTypePredefini{bool};
\item the \emph{expression} type has a reader without any argument named \emph{bool} that returns a \refTypePredefini{bool} value.
\end{enumerate}

Most expressions you write fall in the first case.

Applying the second rule enables to use an \refTypePredefini{lbool} expression as an \emph{if} expression. For example:
{\lstset{emph={expression, instructions, else_instructions}, emphstyle=\emph}
\begin{galgascode}
@lbool var := ... ;
if var then
  instructions
else
  else_instructions
end if ;
\end{galgascode}
}

The \emph{var} object belongs to the \refTypePredefini{lbool} type: so first rule fails. But \refTypePredefini{lbool} is a \emph{structure} type, it has a \emph{bool} attribute with the \refTypePredefini{bool} type, so the second rule succeeds. It is semantically equivalent to write:
{\lstset{emph={expression, instructions, else_instructions}, emphstyle=\emph}
\begin{galgascode}
@lbool var := ... ;
if var->bool then
  instructions
else
  else_instructions
end if ;
\end{galgascode}
}

The third rule applies on a \emph{class} type that defines a category reader with argument named \emph{bool} that returns a \refTypePredefini{bool} type. For example, declaring:
\begin{galgascode}
class @myClass { ... }

reader @myClass bool -> @bool outResult : ... end reader ;
\end{galgascode}

enables to write:
{\lstset{emph={expression, instructions, else_instructions}, emphstyle=\emph}
\begin{galgascode}
@myClass myObject := ... ;
if myObject then
  instructions
else
  else_instructions
end if ;
\end{galgascode}
}

It is semantically equivalent to write:
{\lstset{emph={expression, instructions, else_instructions}, emphstyle=\emph}
\begin{galgascode}
@myClass myObject := ... ;
if [myObject bool] then
  instructions
else
  else_instructions
end if ;
\end{galgascode}
}


\subsection{Dynamic semantics}

According to the preceding section, we only need to describe the dynamic semantic of an \emph{if} instruction without any \emph{elsif} branch and with an \emph{else} branch:
{\lstset{emph={expression, instructions, else_instructions}, emphstyle=\emph}
\begin{galgascode}
if expression then
  instructions
else
  else_instructions
end if ;  
\end{galgascode}
}



The \emph{expression} is first computed :
\begin{itemize}
\item if the evaluation fails, neither the \emph{if} instructions, neither the \emph{else} instructions are executed;
\item if the evaluation result is \emph{true}, the \emph{if} instructions are executed ;
\item if the evaluation result is \emph{false}, the \emph{else} instructions are executed.
\end{itemize}


\section{Grammar Instruction}

\section{Local Variable Declaration Instruction}


{\lstset{emph={variable}, emphstyle=\emph}
\begin{galgascode}
@type variable ;
\end{galgascode}
}

{\lstset{emph={variable, expression}, emphstyle=\emph}
\begin{galgascode}
@type variable := expression ;
\end{galgascode}
}

{\lstset{emph={variable, constructor, arguments}, emphstyle=\emph}
\begin{galgascode}
@type variable [constructor arguments] ;
\end{galgascode}
}


\section{Local Constant Declaration Instruction}




\sectionLabel{L'instruction \texttt{log}}{instructionLog}




\section{Loop Instruction}


\subsection{Syntax}

The \emph{loop} instruction has the following syntax:
{\lstset{emph={expression, instructions_1, instructions_2, variant_expression}, emphstyle=\emph}
\begin{galgascode}
loop variant_expression
: instructions_1
while expression do
  instructions_2
end loop ;  
\end{galgascode}
}

The \emph{instructions\_1} and \emph{instructions\_2} are possibly empty instruction lists. If the \emph{instructions\_1} is empty, the preceeding « : » can be omitted :
{\lstset{emph={expression, instructions_1, instructions_2, variant_expression}, emphstyle=\emph}
\begin{galgascode}
loop variant_expression
while expression do
  instructions_2
end loop ;  
\end{galgascode}
}

\subsection{Semantics}

The \emph{variant\_expression} is an \galgas{@uint} expression that ensures the loop is not endless: it is computed at the beginning of the loop execution, and is decremented by one at the end of every iteration. If it reaches zero, a run-time error is raised.

The \emph{expression} is an \galgas{@bool} expression that expresses repetitive execution.

The \emph{loop} instruction execution is illustrated by the flowchart given in \refFigure{}{loopInstructionFlowchart}.

\begin{figure}[ht]
  \centering
  \small
  \begin{tikzpicture}[very thick]
    \node [rounded corners=5pt, shape=rectangle, draw] (start) {\textsc{begin}} ;
    \node [shape=rectangle, draw] (variant) [below=of start] {$variant := variant\_expression~value$} ;
    \node [shape=diamond, draw] (premierTest) [below=of variant] {$variant > 0$} ;
    \node [shape=rectangle, draw] (error1) [right=of premierTest] {$loop~variant~error$} ;
    \node [shape=rectangle, draw] (body0) [below=of premierTest] {$instructions\_1$} ;
    \node [shape=diamond, draw] (exp) [below=of body0] {$expression$} ;
    \node [shape=diamond, draw] (variantTest) [below=of exp] {$variant > 0$} ;
    \node [shape=rectangle, draw] (decTest) [left=of variantTest] {$variant {-}{-}$} ;
    \node [shape=rectangle, draw] (body1) [above=of decTest] {$instructions\_2$} ;
    \node [shape=rectangle, draw] (error) [right=of variantTest] {$loop~variant~error$} ;
    \node [rounded corners=5pt, shape=rectangle, draw] (end) [right=of error] {\textsc{end}} ;
    
    \draw [->] (start) -- (variant) ;
    \draw [->] (variant) -- (premierTest) ;
    \draw [->] (premierTest) to node[right] {$yes$} (body0) ;
    \draw [->] (premierTest) to node[above] {$no$} (error1) ;
    \draw [->] (body0) -- (exp) ;
    \draw [->] (exp) to node[right] {$true$} (variantTest) ;
    \draw [->] (variantTest) to node[above] {$yes$} (decTest) ;
    \draw [->] (variantTest) to node[above] {$no$} (error) ;
    \draw [->] (decTest) -- (body1) ;
    \draw [->, bend left] (exp.east) to node[above] {$false$} (end.north) ;
    \draw [->] (body1.north) .. controls +(north:2cm) and +(left:2cm) .. (body0.west) ;
    \draw [->] (error) -- (end) ;
    \draw [->] (error1.east) .. controls +(right:2cm) .. (end) ;
  \end{tikzpicture}
  \caption{\emph{loop} instruction flowchart}
  \labelFigure{loopInstructionFlowchart}
\end{figure}


















\sectionLabel{Method Call Instruction}{methodCallInstruction}




\sectionLabel{Modifier Call Instruction}{modifierCallInstruction}




\section{Switch Instruction}




\section{Send Instruction}




\section{Warning Instruction}




%!TEX encoding = UTF-8 Unicode
%!TEX root = ../galgas-book.tex

%--------------------------------------------------------------
\chapter{Syntax and Grammar Components}
%-------------------------------------------------------------

\section {GALGAS and Context-Free Grammars}


\section{Writing a Syntax Component}\index{Component!Syntax}

\section{Syntax Instructions}

\subsection{Terminal Symbol Instruction}

\subsection{Non Terminal Symbol Instruction}


\subsection{Repeat Instruction}


\subsection{Select Instruction}



\subsection{Parse Instruction}

\subsubsection{Parse do ... Instruction}


\subsubsection{Parse loop ... Instruction}


\subsubsection{Parse when ... Instruction}


\section{Writing a Grammar Component}\index{Component!Grammar}



%!TEX encoding = UTF-8 Unicode
%!TEX root = ../galgas-book.tex

%--------------------------------------------------------------
\chapter{Graphic User Interface Component}\index{Component!Graphic User Interface}
%-------------------------------------------------------------


%!TEX encoding = UTF-8 Unicode
%!TEX root = ../galgas-book.tex

%--------------------------------------------------------------
\chapter{Program Component}\index{Component!Program}
%-------------------------------------------------------------



%-----------------------------------------------------------------------------------------------------------------------*
%                                                                                                                       *
%   I N D E X                                                                                                           *
%                                                                                                                       *
%-----------------------------------------------------------------------------------------------------------------------*

\cleardoublepage % Pour commencer à une page impaire
\phantomsection  % Pour faire correctement pointer l'hyperlien dans la table des matières

%--- Écrire l'index
{\small
\printindex
}

%-----------------------------------------------------------------------------------------------------------------------*
%                                                                                                                       *
%   F I N    D U    D O C U M E N T                                                                                     *
%                                                                                                                       *
%-----------------------------------------------------------------------------------------------------------------------*

\end{document}

%-----------------------------------------------------------------------------------------------------------------------*

