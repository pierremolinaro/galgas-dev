%---------------------------------------------------------------------*
%                                                                     *
% This file is included in galgas-book.tex file                       *
%                                                                     *
%---------------------------------------------------------------------*

%--- Mot clef
\newcommand \motCle[1] {\texttt{\textbf{#1}}}

%---------------------------------------------------------------------*

%--- Nom de type
\newcommand \nomType[1] {\texttt{#1}}

%---------------------------------------------------------------------*
% Section pour la d�finition des types

\newcommand \definitionSectionType[1] {\section{The \nomType{#1} Type}\label{#1}}

\newcommand \lienSectionType[1] {\hyperref[#1]{#1 type (page \pageref{#1})}}

%---------------------------------------------------------------------*

%--- D�finition d'un reader sans argument
% Exemple d'appel :
% \readerSansArgument{line} % Nom du reader
% {@location} % Nom du type
% {1.8.2} % Premi�re version GALGAS qui impl�mente ce reader
% {@uint} % Type renvoy�
% {Returns the line of the receiver's value.} % Description
% {this reader raises a run-time error if ...} % Discussion

\newcommand \readerSansArgument[6] {
  \subsection{\texttt{#1} Reader}\label{reader #2 #1}
  #5
  \newline
  \newline
  \begin{tabular}{|l}
  \texttt{\emph{reader} \nomType{#2} #1 -> #4 ;}
  \end{tabular}
  \newline
  \newline
  \textbf{Availabilty:} available in GALGAS #3 and later.
  \newline
  \newline
  \textbf{Discussion:} #6
  \newline
}

\newcommand \lienReader[2] {\hyperref[reader #1 #2]{\texttt{#2} reader (page \pageref{reader #1 #2})}}

%---------------------------------------------------------------------*

%--- D�finition d'un constructeur sans argument
% Exemple d'appel :
% \constructeurSansArgument{nowhere} % Nom du constructeur
% {@location} % Nom du type
% {1.8.2} % Premi�re version GALGAS qui impl�mente ce constructeur
% {@uint} % Type renvoy�
%{Returns an \nomType{@location} that does not points out any location.} % Description
%{The returned object responds \motCle{true} to the isNowhere reader.} % Discussion

\newcommand \contructeurSansArgument[6] {
  \subsection{\texttt{#1} Constructor}\label{constructor #2 #1}
  #5
  \newline
  \newline
  \begin{tabular}{|l}
  \texttt{\emph{constructor} \nomType{#2} #1 -> #4 ;}
  \end{tabular}
  \newline
  \newline
  \textbf{Availabilty:} available in GALGAS #3 and later.
  \newline
  \newline
  \textbf{Discussion:} #6
  \newline
}

%---------------------------------------------------------------------*

%--- D�finition d'un constructeur avec 2 arguments
% Exemple d'appel :
% \constructeurSansArgument{nowhere} % Nom du constructeur
% {@location} % Nom du type
% {1.8.2} % Premi�re version GALGAS qui impl�mente ce constructeur
% {@uint} % Type renvoy�
%{Returns an \nomType{@location} that does not points out any location.} % Description
%{The returned object responds \motCle{true} to the isNowhere reader.} % Discussion

\newcommand \contructeurDeuxArguments[8] {
  \subsection{\texttt{#1} Constructor}\label{constructor #2 #1}
  #7
  \newline
  \newline
  \begin{tabular}{|l}
  \texttt{\emph{constructor} \nomType{#2} #1}\\
  \texttt{\ \ ?#5}\\
  \texttt{\ \ ?#6}\\
  \texttt{\ \ -> #4 ;}
  \end{tabular}
  \newline
  \newline
  \textbf{Availabilty:} available in GALGAS #3 and later.
  \newline
  \newline
  \textbf{Discussion:} #8
  \newline
}

%---------------------------------------------------------------------*
