%!TEX encoding = UTF-8 Unicode
%!TEX root = ../galgas-book.tex

%--------------------------------------------------------------
\chapterLabel{Le type \texttt{map}}{typeMap}
%-------------------------------------------------------------

\tableDesMatieresLocaleDeProfondeurRelative{2}


Un objet de type \ggst+map+ est une table de symboles, chaque symbole étant associé à des valeurs. Un objet de type \ggst+map+ a une sémantique de valeur.

\section{Déclaration}

La déclaration d'un type \ggst+map+ nomme~:
\begin{itemize}
  \item les \emph{setters} d'insertion (\refSectionPage{setterInsertionTable})~;
  \item les \emph{méthodes} de recherche (\refSectionPage{methodesRechercheTable})~;
  \item les \emph{setters} de retrait (\refSectionPage{setterRetraitTable})~
  \item les \emph{setters} de remplacement (\refSectionPage{setterRemplacementTable}).
\end{itemize}

Les clés sont déclarées implicitement et sont du type \refTypePredefini{lstring}.

Par exemple~:

\begin{galgas3}
map @MaTable {
  @string mPremier
  @bool mSecond
  insert insertKey error message "the '%K' key is already declared in %L"
  search searchKey error message "the '%K' key is not defined"
  %searchSubscript searchKey %errorMessage "the '%K' key is not defined"
  remove removeKey error message "the '%K' key is not defined"
  replace replaceKey error message "the '%K' key is not defined"
  insert or replace
}
\end{galgas3}


\begin{galgas4}
map @MaTable {
  public let @string mPremier
  public var @bool mSecond
  %insertSetter insertKey %errorMessage "the '%K' key is already declared in %L"
  %searchMethod searchKey %errorMessage "the '%K' key is not defined"
  %searchSubscript searchKey %errorMessage "the '%K' key is not defined"
  %removeSetter removeKey %errorMessage "the '%K' key is not defined"
  %replaceSetter replaceKey %errorMessage "the '%K' key is not defined"
  %insertOrReplaceSetter
}
\end{galgas4}

La déclaration d'un type table \ggsq!@T! déclare implicitement un type de structure nommé \ggsq!@T.@element!, qui contient les propriétés déclarées dans le type de table précédées par la propriété \ggsq!key!, de type \ggsq!@lstring!, qui est non mutable. Pour le type table \ggsq!@MaTable! défini ci-dessus, la structure déclarée est :
 
\begin{galgas4}
struct @MaTable.@element {
  public let @lstring lkey
  public let @string mPremier
  public var @bool mSecond
}
\end{galgas4}





\section{Constructeurs}

Pour initialiser une table vide, il y a trois possibilités, sémantiquement identiques~:
\begin{itemize}
  \item la constante \texttt{\{\}}, \texttt{()} (\refSubsectionPage{constanteMapVide})~;
  \item le constructeur \texttt{emptyMap} (\refSubsectionPage{constructeurEmptyMap}).
\end{itemize}


\subsectionLabel{Constante \texttt{\{\}}, \texttt{()}}{constanteMapVide}

La constante \texttt{\{\}} en Galgas 3 permet d'instancier une table vide. Exemple~:
\begin{galgas3}
var @MaTable uneTable = {}
\end{galgas3}

Ou encore~:

\begin{galgas3}
var uneTable = @MaTable {}
\end{galgas3}

En Galgas 4, cette constante est \texttt{()}. Exemple~:
\begin{galgas4}
var @MaTable uneTable = ()
\end{galgas4}

Ou encore~:

\begin{galgas4}
var uneTable = @MaTable ()
\end{galgas4}




\subsectionLabel{Constructeur \texttt{emptyMap}}{constructeurEmptyMap}

Pour instancier une table vide, une autre possibilité est d'appeler le constructeur \ggst=emptyMap=. Exemple~:
\begin{galgas3}
@MaTable uneTable = .emptyMap
\end{galgas3}

Ou encore~:

\begin{galgas3}
var uneTable = @MaTable.emptyMap
\end{galgas3}




\subsection{Constructeur \texttt{mapWithMapToOverride}}

\begin{galgas3}
constructor mapWithMapToOverride ?@T inMapToOverride -> @T
\end{galgas3}

Ce constructeur permet d'instancier une table vide, qui surcharge la table \ggst+inMapToOverride+ citée en argument. Exemple~:
\begin{galgas3}
@MaTable uneTable = {}
@MaTable autreTableTable = .mapWithMapToOverride {!uneTable}
\end{galgas3}






\sectionLabel{Setters d'insertion}{setterInsertionTable}

Une \ggst+map+ peut déclarer zéro, un ou plusieurs \emph{setters} d'insertion. Un \emph{setter} d'insertion permet d'insérer une nouvelle entrée à une table. Une erreur est déclenchée en cas de tentative d'une clé déjà existante.


Un \emph{setter} d'insertion est déclaré par~:

\begin{galgas3}
insert nom error message "message_erreur"
\end{galgas3}

\begin{galgas4}
%insertSetter nom %errorMessage "message_erreur"
\end{galgas4}

L'identificateur \ggst+nom+ donne un nom au \emph{setter} d'insertion~; ce nom doit être unique parmi les \emph{setters} d'insertion et de retrait. La chaîne de caractères \ggst+"message_erreur"+ définit le message d'erreur qui est affiché en cas de tentative d'une clé déjà existante. Cette chaîne accepte deux séquences d'échappement~:
\begin{itemize}
  \item \texttt{\%K}, qui est remplacée par la chaîne de caractères de la clé existante~;
  \item \texttt{\%L}, qui est remplacée par la chaîne décrivant la position de la clé existante dans les fichiers source.
\end{itemize}


Un \emph{setter} d'insertion est appelé dans une \emph{instruction d'appel de setter}, comprenant tous ses arguments en sortie~:
\begin{itemize}
  \item le premier argument est une expression de type \ggst+@lstring+ qui caractérise la clé à insérer~;
  \item ensuite, pour chaque attribut déclaré, une expression du type de cet attribut.
\end{itemize}

Par exemple~:
\begin{galgas3}
@MaTable uneTable = {}
@lstring clef = ...
@string s = ...
@uint v = ...
[!?uneTable insertKey !clef !s !v]
\end{galgas3}



\sectionLabel{Setter d'insertion ou de remplacement}{setterInsertOrReplace}

Une \ggsq+map+ peut déclarer un \emph{setter} effectuant selon le contexte une insertion ou un remplacement~:
\begin{itemize}
  \item si la clé n'existe pas, une insertion est réalisée~;
  \item si elle existe , le remplacement est effectué.
\end{itemize}

Un \emph{setter} d'insertion ou de remplacement est déclaré par~:

\begin{galgas3}
insert or replace
\end{galgas3}

\begin{galgas4}
%insertOrReplaceSetter
\end{galgas4}

Son nom est toujours \ggsq!insertOrReplace!. Il est appelé dans une \emph{instruction d'appel de setter}, comprenant tous ses arguments en sortie~:
\begin{itemize}
  \item le premier argument est une expression de type \ggsq+@lstring+ qui caractérise la clé à insérer~;
  \item ensuite, pour chaque propriété déclarée, une expression du type de cette propriété.
\end{itemize}

Par exemple~:
\begin{galgas3}
var @MaTable uneTable = {}
var @lstring clef = ...
var @string s = ...
var @uint v = ...
[!?uneTable insertOrReplace !clef !s !v] # Insertion
[!?uneTable insertOrReplace !clef !s !v] # Remplacement
\end{galgas3}








\sectionLabel{Recherche dans une table}{methodesRechercheTable}

Pour rechercher une entrée dans une table, on peut appeler :
\begin{itemize}
  \item l'opérateur d'indiçage, qui renvoie une valeur optionnelle (\refSubsectionPage{rechercheOptionnelleTable}).
  \item une \emph{méthode de recherche} (\refSubsectionPage{methodeRechercheTable})~;
  \item un \emph{opérateur d'indiçage dédié} (\refSubsectionPage{opérateurIndiçageDédiéRecherche}).
\end{itemize}








\subsectionLabel{Opérateur d'indiçage par défaut}{rechercheOptionnelleTable}

%L'instruction \ggst!with! a été supprimée. À la place on utilise l'expression de recherche optionnelle.

Toute table définit implicitement l'opérateur d'indiçage. Pour une instance de table nommée \ggst!uneTable! de type \ggst!@MaTable! , l'appel de l'opérateur d'indiçage s'écrit :

\begin{galgas3}
let @string clé = ...
let @MaTable.@element? entry = uneTable @[clé]
\end{galgas3}

\begin{galgas4}
let @string clé = ...
let @MaTable.@element? entry = uneTable [clé]
\end{galgas4}

On remarquera que la clé est une \ggsq!@string! et non pas une \ggsq!@lstring!, et que le sélecteur est vide. La valeur renvoyée est une valeur optionnelle, qui vaut \ggsq!nil! si la clé n'existe pas. Par exemple~:

%Pour remplacer l'instruction \ggsq!with! qui a été supprimée :
%
%\begin{galgas3}
%let @uint terminalSymbolIndex
%with self.mTerminalName.string in ioActuallyUsedTerminalSymbolMap do
%  terminalSymbolIndex = mTerminalIndex // Accès direct à l'entrée
%else // L'entrée n'existe pas
%  terminalSymbolIndex = ...
%end
%\end{galgas3}
%
%On écrit :

\begin{galgas3}
let @uint terminalSymbolIndex
if let entry = ioActuallyUsedTerminalSymbolMap @[self.mTerminalName.string] then
  terminalSymbolIndex = entry.mTerminalIndex // Accès direct à l'entrée
else // L'entrée n'existe pas
  terminalSymbolIndex = ...
end
\end{galgas3}


\begin{galgas4}
let @uint terminalSymbolIndex
if let entry = ioActuallyUsedTerminalSymbolMap [self.mTerminalName.string] then
  terminalSymbolIndex = entry.mTerminalIndex // Accès direct à l'entrée
else // L'entrée n'existe pas
  terminalSymbolIndex = ...
end
\end{galgas4}








\subsectionLabel{Méthode de recherche}{methodeRechercheTable}

Une \ggst+map+ peut déclarer zéro, une ou plusieurs \emph{méthodes} de recherche, qui permettent de rechercher une entrée d'une table via un appel de méthode et en particularisant le message d'eerur si la clé n'est pas définie par la table. 


Une \emph{méthode} de recherche est déclarée par~:


\begin{galgas3}
search nomMethodeRecherche error message "message_erreur_associe"
\end{galgas3}

\begin{galgas4}
%searchMethod nomMethodeRecherche %errorMessage "message_erreur_associe"
\end{galgas4}

L'identificateur \ggst+nomMethodeRecherche+ donne un nom à la méthode de recherche~; ce nom doit être unique parmi ces \emph{identificateurs}. La chaîne de caractères \ggst+"message_erreur_associe"+ définit le message d'erreur qui est affiché en cas de recherche d'une clé inexistante dans la table. Cette chaîne accepte une séquence d'échappement~:
\begin{itemize}
  \item \texttt{\%K}, qui est remplacée par la chaîne de caractères de la clé inexistante recherchée.
\end{itemize}


Une \emph{méthode} de recherche est appelée dans une \emph{instruction d'appel de méthode}~:
\begin{itemize}
  \item le premier argument (sortie) est une expression de type \ggst+@lstring+ qui caractérise la clé à rechercher~;
  \item ensuite, pour chaque attribut déclaré, un argument en entrée nommant une variable destinée à recevoir la valeur de l'attribut correspondant.
\end{itemize}

Par exemple~:

\begin{galgas3}
var @MaTable uneTable = {}
...
var @lstring clef = ...
[uneTable searchKey !clef ?@string s ?@uint v]
\end{galgas3}


\begin{galgas4}
var @MaTable uneTable = ()
...
var @lstring clef = ...
uneTable.searchKey (!clef, ?@string s, ?@uint v)
\end{galgas4}




\subsectionLabel{Opérateur d'indiçage dédié}{opérateurIndiçageDédiéRecherche}


Une \ggst+map+ peut déclarer zéro, une ou plusieurs \emph{opérateurs d'indiçage dédié} de recherche, qui permettent de rechercher une entrée d'une table via un appel de méthode et en particularisant le message d'erreur si la clé n'est pas définie par la table. 


Un \emph{opérateur d'indiçage dédié} de recherche est déclaré par~:


\begin{galgas3}
%searchSubscript nomSelecteur error message "message_erreur_associe"
\end{galgas3}

\begin{galgas4}
%searchSubscript nomSelecteur %errorMessage "message_erreur_associe"
\end{galgas4}



Un \emph{getter} de recherche est appelé via l'opérateur d'indiçage, dont le sélecteur est l'identificateur de recherche. L'argument est une expression de type \ggst+@lstring+ qui caractérise la clé à rechercher.

Il renvoie une copie de l'entrée, et déclenche le message d'erreur associé à l'identifficateur si la clef n'existe dans la table. 


Par exemple~:
\begin{galgas3}
var @MaTable uneTable = {}
...
var @lstring clef = ...
let @MaTable.@element entry = uneTable @[!searchKey: clef]
\end{galgas3}

\begin{galgas4}
var @MaTable uneTable = ()
...
var @lstring clef = ...
let @MaTable.@element entry = uneTable [!searchKey: clef]
\end{galgas4}

Par rapport à l'opérateur d'indiçage par défaut (\refSubsectionPage{rechercheOptionnelleTable}), l'opérateur d'indiçage dédié :
\begin{itemize}
  \item présente un sélecteur non vide ;
  \item ne renvoie pas un type optionnel~;
  \item par conséquent, déclencle une erreur si la clé n'existe pas.
\end{itemize}




\sectionLabel{Setters de retrait}{setterRetraitTable}

Une \ggst+map+ peut déclarer zéro, un ou plusieurs \emph{setters} de retrait. Un \emph{setter} de retrait permet de retirer une entrée d'une table, et retourne la valeur des attributs de la clé retirée. Une erreur est déclenchée si la clé n'existe pas.


Un \emph{setter} de retrait est déclaré par~:

\begin{galgas3}
remove nom error message "message_erreur"
\end{galgas3}

\begin{galgas4}
%removeSetter nom %errorMessage "message_erreur"
\end{galgas4}

L'identificateur \ggst+nom+ donne un nom au \emph{setter} de retrait~; ce nom doit être unique parmi les \emph{setters} d'insertion et de retrait. La chaîne de caractères \ggst+"message_erreur"+ définit le message d'erreur qui est affiché en cas de recherche d'une clé inexistante. Cette chaîne accepte une séquence d'échappement~:
\begin{itemize}
  \item \texttt{\%K}, qui est remplacée par la chaîne de caractères de la clé inexistante à retirer.
\end{itemize}


Un \emph{setter} de retrait est appelé dans une \emph{instruction d'appel de setter}~:
\begin{itemize}
  \item le premier argument (sortie) est une expression de type \ggst+@lstring+ qui caractérise la clé à retirer~;
  \item ensuite, pour chaque attribut déclaré, un argument en entrée nommant une variable destinée à recevoir la valeur de l'attribut correspondant de la clé retirée.
\end{itemize}

Par exemple~:
\begin{galgas3}
@MaTable uneTable = {}
...
@lstring clef = ...
[!?uneTable removeKey !clef ?@string s ?@uint v]
\end{galgas3}









\sectionLabel{Setters de remplacement}{setterRemplacementTable}

Une \ggst+map+ peut déclarer zéro, un ou plusieurs \emph{setters} de remplacement. Un \emph{setter} de remplacement permet de remplacer une entrée d'une table. Une erreur est déclenchée si la clé n'existe pas.


Un \emph{setter} de remplacement est déclaré par~:

\begin{galgas3}
replace nom error message "message_erreur"
\end{galgas3}

\begin{galgas4}
%replaceSetter nom %errorMessage "message_erreur"
\end{galgas4}

L'identificateur \ggst+nom+ donne un nom au \emph{setter} de remplacement~; ce nom doit être unique parmi les \emph{setters} de la table. La chaîne de caractères \ggst+"message_erreur"+ définit le message d'erreur qui est affiché en cas de recherche d'une clé inexistante. Cette chaîne accepte une séquence d'échappement~:
\begin{itemize}
  \item \texttt{\%K}, qui est remplacée par la chaîne de caractères de la clé inexistante à retirer.
\end{itemize}


Un \emph{setter} de remplacement est appelé dans une \emph{instruction d'appel de setter}. Il comporte un seul argument, l'élement qui doit être remplacé. Le sélecteur est \ggsq+!with:+.


Par exemple~:

\begin{galgas3}
var @MaTable uneTable = {}
...
var @MaTable.@element element = ...
[!?uneTable replaceKey !with: element]
\end{galgas3}


\begin{galgas4}
var @MaTable uneTable = ()
...
var @MaTable.@element element = ...
!?uneTable.replaceKey (!with: element)
\end{galgas4}







\sectionLabel{Getters}{gettersDeTable}

%\subsection{Getter \texttt{allKeyList}}
%
%\begin{galgas3}
%getter @T allKeyList -> @lstringlist
%\end{galgas3}
%
%Le \emph{getter} \ggst+allKeyList+ retourne la liste construite avec toutes les clés du récepteur, dans la table de premier niveau et dans les tables surchargées. L'ordre de la liste est~:
%\begin{itemize}
%  \item d'abord les clés de la table de premier niveau, puis celles des tables surchargées, dans l'ordre de la surcharge~;
%  \itel pour chaque table, les clés apparaissent dans l'ordre alphabétique croissant.
%\end{itemize}

\subsection{Getter \texttt{count}}

\begin{galgas3}
getter count -> @uint
\end{galgas3}


Le \emph{getter} \ggst+count+ retourne un \ggst+@uint+ qui contient le nombre d'entrées de la table de premier niveau du récepteur.



\subsection{Getter \texttt{hasKey}}

\begin{galgas3}
getter hasKey ?@string inKey -> @bool
\end{galgas3}


Le \emph{getter} \ggst+hasKey+ retourne un \ggst+@bool+ qui est \ggst+true+ si la clé \ggst+inKey+ est dans la table de premier niveau du récepteur, \ggst+false+ dans le cas contraire.



\subsection{Getter \texttt{keyList}}

\begin{galgas3}
getter keyList -> @lstringlist
\end{galgas3}


Le \emph{getter} \ggst+keyList+ retourne la liste construite avec toutes les clés de la table de premier niveau du récepteur. L'ordre de la liste est l'ordre alphabétique croissant des clés.



\subsection{Getter \texttt{keySet}}

\begin{galgas3}
getter keySet -> @stringset
\end{galgas3}


Le \emph{getter} \ggst+keySet+ retourne l'ensemble de toutes les clés de la table de premier niveau du récepteur.





\subsectionLabel{Getter \texttt{locationForKey}}{getterLocationForKey}

\begin{galgas3}
getter locationForKey ?@string inKey -> @location
\end{galgas3}


Le \emph{getter} \ggst+locationForKey+ retourne un \ggst+@location+ qui contient l'information de position de la clé \ggst+inKey+ dans la table de premier niveau du récepteur. Une erreur d'exécution est déclenchée si cette clé n'existe pas.








\subsection{Getter \texttt{overriddenMap}}

\begin{galgas3}
getter overriddenMap -> @T
\end{galgas3}


Le \emph{getter} \ggst+overriddenMap+ retourne la table obtenue en amputant de la valeur du récepteur la table de premier niveau. Si le récepteur n'a pas de table surchargée, une erreur d'exécution est déclenchée.









%\sectionLabel{Attributs d'une table}{attributsTable}
%
%Deux attributs peuvent être définis dans la déclaration d'une table : \ggst!%selectors! et \ggst!%noSuggestion!.
%
%L'attribut \ggst!%selectors! impose l'utilisation de sélecteurs dans l'appel des setters d'insertion (\refSectionPage{setterInsertionTable}), des méthodes de recherche(\refSectionPage{methodesRechercheTable}) et des \emph{setters} de retrait (\refSectionPage{setterRetraitTable}). Le sélecteur associé à la clé est \ggst=!lkey:=. Le sélecteur associé à une propriété porte le nom de cette propriété.
%
%
%\begin{galgas3}
%map @MaTable %selectors {
%  @string mPremier
%  @uint mSecond
%  insert insertKey error message "the '%K' key is already declared in %L"
%  search searchKey error message "the '%K' key is not defined"
%  remove removeKey error message "the '%K' key is not defined"
%}
%\end{galgas3}
%
%L'appel d'un \emph{setter} d'insertion :
%\begin{galgas3}
%@MaTable uneTable = {}
%@lstring clef = ...
%@string s = ...
%@uint v = ...
%[!?uneTable insertKey !lkey: clef !mPremier: s !mSecond: v]
%\end{galgas3}
%
%
%
%L'appel d'une méthode de recherche :
%\begin{galgas3}
%@MaTable uneTable = {}
%...
%@lstring clef = ...
%[uneTable searchKey !lkey: clef ?mPremier: @string s ?mSecond: @uint v]
%\end{galgas3}
%
%
%L'appel d'un \emph{setter} de retrait :
%\begin{galgas3}
%@MaTable uneTable = {}
%...
%@lstring clef = ...
%[!?uneTable removeKey !lkey: clef ?mPremier: @string s ?mSecond: @uint v]
%\end{galgas3}
%
%Lors de l'échec d'une recherche de clés, une méthode de recherche ou un setter de retrait affiche les clés voisines contenues dans la table. L'attribut \ggst!%noSuggestion! inhibe l'affichage de ces suggestions.




\section{Énumération}

L'instruction \ggst+for+ permet d'énumérer des objets de type \ggst+map+ ; elle est décrite à la \refSectionPage{instructionFor}.

% Uniquement la table de premier niveau est énumérée. Par défaut, l'énumération s'effectue dans l'ordre croissant des clés. Pour énumérer dans l'ordre décroissant, placer \ggst+>+ après le mot réservé \ggst=for=.
%
%À l'intérieur du corps de la boucle, sont implicitement définies~:
%\begin{itemize}
%  \item la constante \ggst+lkey+, de type \ggst+@lstring+, qui a pour valeur la clé de l'entrée courante~;
%  \item pour chaque attribut, une constante du type de l'attribut, et portant le nom de cet attribut, qui a pour valeur la valeur de cet attribut de l'entrée courante.
%\end{itemize}
%
%Par exemple~:
%\begin{galgas3}
%@MaTable uneTable = {}
%[!?uneTable insertKey !@lstring.new {!"z" !.here} !"world" !5]
%[!?uneTable insertKey !@lstring.new {!"a" !.here} !"hello" !10]
%for () in aMap do
%  message lkey.string . " " . mPremier . " " . mSecond . "\n"
%end
%for > () in aMap do
%  message lkey.string . " " . mPremier . " " . mSecond . "\n"
%end
%\end{galgas3}
%
%L'affichage produit est~:
%
%\begin{galgas3}
%a hello 10
%z world 5
%z world 5
%a hello 10
%\end{galgas3}

%====== Setting an attribute of an entry ======
%
%^Available in GALGAS 1.8.4 and later. ^
%
%Given a key, you can directly set an attribute of a map entry.
%
%In order to enable this feature, you have to associate the setter feature to the given attribute.
%
%map @MaTable {
%  @string mPremier
%  @bool mSecond feature setter
%}
%
%For every attribute declared with this feature, a setter is available; its name is build by the concatenation of three patterns:
%  - the string set;
%  - the attribute name, with the first letter capitalized;
%  - the string ForKey.
%
%So, for the mSecond attribute, the associated setter name is setmSecondForKey.
%
%The setter has two input arguments:
%  - the key, an @string expression;
%  - the value to set to the attribute.
%
%For example:
%
%@MaTable aMap = ...
%@string s = ...
%@bool v = ...
%[!?aMap setmSecondForKey !s !v]
%
%A run-time error is raised if the key value does not exist in the map.
%
%====== Using the with instruction on a map object ======
%
%^Available in GALGAS 1.8.4 and later. ^
%
%The with instruction enables a direct access on all attributes of an entry. Its syntax is:
%
%with //prefix// !?//map_object// //search_method// !//key_expression// do
%  ...
%else
%  ...
%end with
%
%The //prefix// and the else part are optional.
%
%There are two different behaviours, depending from the //search_method//:
%  * the //search_method// is one declared in the search declarations;
%  * the //search_method// is the predefined hasKey identifier.
%
%===== with instruction naming a declared search method =====
%
%The //key_expression// should be an @lstring expression.
%
%Given this map type declaration:
%
%map @MaTable {
%  @string mPremier
%  @bool mSecond
%  search searchKey error message "the %%'%K'%% key is not defined"
%  ...
%}
%
%You can write:
%
%@MaTable aMap = ...
%@lstring aKey = ...
%with !?aMap searchKey !aKey do
%  # ...
%else
%  # ...
%end with
%
%The aMap object is accessed in read/write mode.
%
%If the aKey object value does not correspond to an existing entry, an error message is displayed and the else part is executed. The error message is based upon the search method name, here searchKey. The error location is given by aKey location.
%
%If the aKey object value corresponds to an existing entry, the do part is executed. The entry's attributes can be fully accessed in this part (you can read, write or modify them), using directly their names. The entry's key (an @lstring object) can be accessed in read mode (you cannot modify it), using the key identifier. For example:
%
%
%@MaTable aMap = ...
%@lstring aKey = ...
%with !?aMap searchKey !aKey do
%  mPremier .= %%"xyz"%% # mPremier is accessed in read/write mode
%  mSecond = true # mSecond is accessed in write mode
%  log key # key is can only be accessed in read mode
%end with
%
%The only constraint is that all attributes should be valuated at the end of the do part.
%
%===== with instruction naming the predefined hasKey method =====
%
%The //key_expression// should be an @string expression.
%
%Given this map type declaration:
%
%map @MaTable {
%  @string mPremier
%  @bool mSecond
%  ...
%}
%
%You can write:
%
%@MaTable aMap = ...
%@string aKey = ...
%with !?aMap hasKey !aKey do
%  # ...
%else
%  # ...
%end with
%
%The aMap object is accessed in read/write mode.
%
%If the aKey object value does not correspond to an existing entry, no error message is displayed and the else part is executed.
%
%If the aKey object value corresponds to an existing entry, the do part is executed.  The entry's attributes can be fully accessed in this part (you can read, write or modify them), using directly their names, the entry's key (an @lstring object) can be accessed in read mode (you cannot modify it), using the key identifier, exactly as in the previously described behaviour.
%
%===== Using a prefix =====
%
%This feature enables to prepend with a prefix the names used for accessing the attributes and the key in the do part. It could be useful for avoiding name conflicts.
%
%It is available for both search methods.
%
%Given this map type declaration:
%
%map @MaTable {
%  @string mPremier
%  @bool mSecond
%  ...
%}
%
%You can write:
%
%@MaTable aMap = ...
%@string aKey = ...
%with xyz_ : !?aMap hasKey !aKey do
%  xyz_mPremier .= %%"xyz"%% # mPremier is accessed in read/write mode
%  xyz_mSecond = true # mSecond is accessed in write mode
%  log xyz_key # key is can only be accessed in read mode
%end with
%
%Using the xyz_ prefix, all attributes and the key should be accessed using this prefix.
