  %!TEX encoding = UTF-8 Unicode
%!TEX root = ../galgas-book.tex

%--------------------------------------------------------------
\chapterLabel{Le type \texttt{boolset}}{typeBoolset}
%-------------------------------------------------------------

\tableDesMatieresLocaleDeProfondeurRelative{1}



Le mot-clé \ggsq!boolset! permet de définir des types d'ensembles non ordonnés. Un tel objet a une sémantique de valeur.

La syntaxe de définition d'un type ensemble est de la forme~:

\begin{galgas34}
boolset @MonEnsemble {
  // Liste de déclaration d'éléments, par exemple :
  case element0
  case element1
  case element2
}
\end{galgas34}

L'implémentation actuelle limite à 64 le nombre d'éléments qui peuvent être définis.

\section{Initialisateur}

Un \ggsq!boolset! ne définit qu'un seul initialisateur, qui construit un ensemble vide~:
\begin{galgas34}
init @MonEnsemble () -> @MonEnsemble
\end{galgas34}
Pour construire un ensemble vide~:
\begin{galgas34}
var @MonEnsemble ensembleVide = ()
\end{galgas34}

Pour construire un ensemble plein, utiliser l'opérateur de complémentation \ggsq!~!~:
\begin{galgas34}
var @MonEnsemble ensemblePlein = ~ ()
\end{galgas34}










\section{Fonctions de classes : construire un ensemble d'un seul élément}

Pour chaque élément, une fonction de classe est implicitement définie, qui construit un ensemble constitué d'un seul élément~:

Pour initialiser un \ggsq!boolset!, on utilise un des constructeurs définis~:
\begin{galgas34}
var x = @MonEnsemble.element1
\end{galgas34}

Si on veut un ensemble contenant plusieurs éléments, on utilise l'opérateur \ggsq=|=, qui effectue l'union de ses opérandes~:
\begin{galgas34}
var x = @MonEnsemble.element1 | @MonEnsemble.element2
log x // LOGGING x: <boolset @MonEnsemble: element1 element2>
\end{galgas34}

L'inférence de type permet d'éliminer les annotations de type non nécessaires~:
\begin{galgas34}
var x = @MonEnsemble.element1 | .element2
\end{galgas34}

Ou encore~:
\begin{galgas34}
var @MonEnsemble x = .element1 | .element2
\end{galgas34}





















\section{Getter}


Le \emph{getter} \ggsq!contains! permet de savoir si le récepteur inclut la valeur de l'argument.


\begin{galgas3}
var @MonEnsemble x = ...
var @MonEnsemble y = ...
let @bool b = [x contains !y] // Vrai si x inclut y
\end{galgas3}

\begin{galgas4}
var @MonEnsemble x = ...
var @MonEnsemble y = ...
let @bool b = x.contains (y) // Vrai si x inclut y
\end{galgas4}

Ce \emph{getter} permet de tester si un élément appartient à un ensemble :
\begin{galgas3}
var @MonEnsemble x = ...
let @bool b = [x contains !.element1] // Vrai si x contient element1
\end{galgas3}

\begin{galgas4}
var @MonEnsemble x = ...
let @bool b = x.contains (.element1) // Vrai si x contient element1
\end{galgas4}






\section{Opérateur préfixe}

Un seul opérateur préfixe est défini pour tout \ggsq=boolset= (\refTableau{operateurPrefixeBoolset}).

\begin{table}[ht!]
  \centering
  \begin{tabular}{rl}
    {\bf Expression} & {\bf Signification} \\
    \ggsq+~ a+ & Complémentation.\\
  \end{tabular}
  \caption{Opérateur préfixe des types \texttt{boolset}}
  \labelTableau{operateurPrefixeBoolset}
\end{table}








\section{Opérateurs infixes}

Les opérateurs infixes du \refTableau{operateursInfixesBoolset} sont définis pour tout \ggsq=boolset=.

\begin{table}[ht!]
  \centering
  \begin{tabular}{rl}
    {\bf Expression} & {\bf Signification} \\
    \ggsq+a & b+ & Intersection~: ensemble des éléments appartenant à \ggsq=a= et à \ggsq=b=.\\
    \ggsq+a | b+ & Union~: ensemble des éléments appartenant à \ggsq=a= ou à \ggsq=b=. \\
    \ggsq+a ^ b+ & Exclusion~: ensemble des éléments appartenant soit à \ggsq=a=, soit à \ggsq=b=. \\
    \ggsq+a - b+ & Différence~: ensemble des éléments appartenant à \ggsq=a= et n'appartenant pas à \ggsq=b=.
%    \ggsq+a == b+ & Test d'égalité \\
%    \ggsq+a != b+ & Test d'inégalité
  \end{tabular}
  \caption{Opérateurs infixes des types \texttt{boolset}}
  \labelTableau{operateursInfixesBoolset}
\end{table}









\section{Opérateurs combinés avec l'affectation}

Quatre opérateurs combinés avec l'affectation (\refTableau{operateursCombinesAffectationBoolset}) sont définis pour tout \ggsq=boolset=.

\begin{table}[ht!]
  \centering
  \begin{tabular}{rl}
    {\bf Expression} & {\bf Signification} \\
    \ggsq+a &= b+ & Intersection~: équivalent à \ggsq!a = a & b!.\\
    \ggsq+a |= b+ & Union~: équivalent à \ggsq!a = a | b!. \\
    \ggsq+a ^= b+ & Intersection~: équivalent à \ggsq!a = a ^ b!.\\
    \ggsq+a -= b+ & Union~: équivalent à \ggsq!a = a - b!. \\
  \end{tabular}
  \caption{Opérateurs combinés avec l'affectation des types \texttt{boolset}}
  \labelTableau{operateursCombinesAffectationBoolset}
\end{table}






\section{Comparaison}

Par défaut, une instance de \ggsq!boolset! n'est ni égalable, ni comparable.

On peut uniquement la rendre égalable (on peut alors tester l'égalité \ggsq!==! et l'inégalité \ggsq@!=@ entre deux instances), en ajoutant l'attribut \ggsq!%equatable! dans la déclaration~:

\begin{galgas34}
boolset @MonEnsemble %equatable {
  // Liste de déclaration d'éléments, par exemple :
  case element0
  case element1
  case element2
}
\end{galgas34}


