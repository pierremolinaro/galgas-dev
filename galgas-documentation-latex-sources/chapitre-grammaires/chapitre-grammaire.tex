%!TEX encoding = UTF-8 Unicode
%!TEX root = ../galgas-book.tex


Le langage GALGAS définit trois grammaires :
\begin{itemize}
\item la grammaire du fichier projet GALGAS d'extension \texttt{.galgasProject} (\refChapterPage{grammaireProjet}) ;
\item la grammaire des sources GALGAS d'extension \texttt{.galgas} (\refChapterPage{grammaireSource}) ;
\item la grammaire des fichiers \emph{template} d'extension \texttt{.galgasTemplate} (\refChapterPage{grammaireTemplate}).
\end{itemize}

Ces chapitres listent l'ensemble des règles de production de ces trois grammaires.




{
\tikzset{
  nonterminal/.style={
    % The shape:
    rectangle,
    % The size:
    minimum size=6mm,
    % The border:
    very thick,
    draw=red!50!black!50,         % 50% red and 50% black,
                                  % and that mixed with 50% white
    % The filling:
    top color=white,              % a shading that is white at the top...
    bottom color=red!50!black!20, % and something else at the bottom
    % Font
    font=\itshape\scriptsize
  },
  terminal/.style={
    % The shape:
    rounded rectangle,
    minimum size=6mm,
    % The rest
    very thick,draw=black!50,
    top color=white,bottom color=black!20,
    font=\ttfamily\scriptsize
  },
  firstPoint/.style={circle,>=stealth',thick,draw=black!50},
  point/.style={coordinate,>=stealth',thick,draw=black!50},
  tip/.style={->,shorten >=0.007pt},
  lastPoint/.style={rectangle,>=stealth',thick,draw=black!50},
  every join/.style={rounded corners}
}

\newcommand\nonTerminalSummaryStart{Voici la liste alphabétique des non terminaux: }
\newcommand\ruleSubsection[3]{} % \subsection{Component \texttt{#1}, in file \texttt{#2}, line #3}}
\newcommand\ruleMatrixColumnSeparation{1.5mm}
\newcommand\ruleMatrixRowSeparation{1mm}
\newcommand\nonTerminalSummarySeparator{, }
\newcommand\nonTerminalSummaryEnd{.\\}

\newcommand\nonTerminalSection[2]{\section{Non terminal \texttt{\it#1}}\label{nt:#2}}
\newcommand\nonTerminalSymbol[2]{\hyperref[nt:#2]{#1}}
\newcommand\startSymbol[2]{L'axiome de la grammaire est \hyperref[nt:#2]{#1}.}
\newcommand\nonTerminalSummary[2]{\hyperref[nt:#2]{#1}}

\chapterLabel{Grammaire du projet GALGAS}{grammaireProjet}

\startSymbol{project\_component\_start\_symbol}{1}

\nonTerminalSummaryStart \nonTerminalSummary{project\_component\_start\_symbol}{1}\nonTerminalSummarySeparator \nonTerminalSummary{project\_header}{0}\nonTerminalSummaryEnd \nonTerminalSection{project\_component\_start\_symbol}{1}

\ruleSubsection{galgas3ProjectSyntax}{galgasProject}{64}

\begin{tikzpicture}
  \matrix[column sep=\ruleMatrixColumnSeparation, row sep=\ruleMatrixRowSeparation] {
    & & & & & & & & & & \node (p19-10) [point] {}; & \\
    & & & & & & & \node (p18-7) [terminal] {"string"}; & \\
    & & & & & & & \node (p17-7) [terminal] {\verb=%=app-link}; & \node (p17-8) [terminal] {:}; & \node (p17-9) [terminal] {"string"}; & \\
    & & & & & & & \node (p16-7) [terminal] {\verb=%=app-source}; & \node (p16-8) [terminal] {:}; & \node (p16-9) [terminal] {"string"}; & \\
    & & & & & & & \node (p15-7) [terminal] {\verb=%=macCodeSign}; & \node (p15-8) [terminal] {:}; & \node (p15-9) [terminal] {"string"}; & \\
    & & & & & & & \node (p14-7) [terminal] {\verb=%=MacOSDeployment}; & \node (p14-8) [terminal] {:}; & \node (p14-9) [terminal] {"string"}; & \\
    & & & & & & & \node (p13-7) [terminal] {\verb=%=tool-source}; & \node (p13-8) [terminal] {:}; & \node (p13-9) [terminal] {"string"}; & \\
    & & & & & & & \node (p12-7) [terminal] {\verb=%=libpmAtPath}; & \node (p12-8) [terminal] {:}; & \node (p12-9) [terminal] {"string"}; & \\
    & & & & & & & \node (p11-7) [terminal] {\verb=%=applicationBundleBase}; & \node (p11-8) [terminal] {:}; & \node (p11-9) [terminal] {"string"}; & \\
    & & & & & & & \node (p10-7) [terminal] {\verb=%=codeblocks-linux64}; & \\
    & & & & & & & \node (p9-7) [terminal] {\verb=%=codeblocks-linux32}; & \\
    & & & & & & & \node (p8-7) [terminal] {\verb=%=codeblocks-windows}; & \\
    & & & & & & & \node (p7-7) [terminal] {\verb=%=makefile-win32-on-macosx}; & \\
    & & & & & & & \node (p6-7) [terminal] {\verb=%=makefile-x86linux64-on-macosx}; & \\
    & & & & & & & \node (p5-7) [terminal] {\verb=%=makefile-x86linux32-on-macosx}; & \\
    & & & & & & & \node (p4-7) [terminal] {\verb=%=makefile-unix}; & \\
    & & & & & & & \node (p3-7) [terminal] {\verb=%=makefile-macosx}; & \\
    & & & & & & & \node (p2-7) [terminal] {\verb=%=MacOS}; & \\
    & & & & & & & \node (p1-7) [terminal] {\verb=%=quietOutputByDefault}; & \\
    \node (P0start) [firstPoint] {}; & & \node (p0-2) [nonterminal] {\nonTerminalSymbol{project\_header}{0}}; & \node (p0-3) [terminal] {\{}; & \node (p0-4) [point] {}; & \node (p0-5) [point] {}; & \node (p0-6) [point] {}; & & & & & \node (p0-11) [terminal] {\}}; & \node (p0-12) [lastPoint] {}; & \\
  };
  \draw[->] (P0start) -- (p0-2) ;
  \draw[->] (p0-2) -- (p0-3) ;
  \draw (p0-3) -- (p0-5) ;
  \draw[->] (p0-6) |- (p1-7) ;
  \draw[->] (p0-6) |- (p2-7) ;
  \draw[->] (p0-6) |- (p3-7) ;
  \draw[->] (p0-6) |- (p4-7) ;
  \draw[->] (p0-6) |- (p5-7) ;
  \draw[->] (p0-6) |- (p6-7) ;
  \draw[->] (p0-6) |- (p7-7) ;
  \draw[->] (p0-6) |- (p8-7) ;
  \draw[->] (p0-6) |- (p9-7) ;
  \draw[->] (p0-6) |- (p10-7) ;
  \draw[->] (p0-6) |- (p11-7) ;
  \draw[->] (p11-7) -- (p11-8) ;
  \draw[->] (p11-8) -- (p11-9) ;
  \draw[->] (p0-6) |- (p12-7) ;
  \draw[->] (p12-7) -- (p12-8) ;
  \draw[->] (p12-8) -- (p12-9) ;
  \draw[->] (p0-6) |- (p13-7) ;
  \draw[->] (p13-7) -- (p13-8) ;
  \draw[->] (p13-8) -- (p13-9) ;
  \draw[->] (p0-6) |- (p14-7) ;
  \draw[->] (p14-7) -- (p14-8) ;
  \draw[->] (p14-8) -- (p14-9) ;
  \draw[->] (p0-6) |- (p15-7) ;
  \draw[->] (p15-7) -- (p15-8) ;
  \draw[->] (p15-8) -- (p15-9) ;
  \draw[->] (p0-6) |- (p16-7) ;
  \draw[->] (p16-7) -- (p16-8) ;
  \draw[->] (p16-8) -- (p16-9) ;
  \draw[->] (p0-6) |- (p17-7) ;
  \draw[->] (p17-7) -- (p17-8) ;
  \draw[->] (p17-8) -- (p17-9) ;
  \draw[->] (p0-6) |- (p18-7) ;
  \draw[->] (p19-10) -| (p0-4) ;
  \draw[->] (p1-7) -| (p19-10) ;
  \draw[->] (p2-7) -| (p19-10) ;
  \draw[->] (p3-7) -| (p19-10) ;
  \draw[->] (p4-7) -| (p19-10) ;
  \draw[->] (p5-7) -| (p19-10) ;
  \draw[->] (p6-7) -| (p19-10) ;
  \draw[->] (p7-7) -| (p19-10) ;
  \draw[->] (p8-7) -| (p19-10) ;
  \draw[->] (p9-7) -| (p19-10) ;
  \draw[->] (p10-7) -| (p19-10) ;
  \draw[->] (p11-9) -| (p19-10) ;
  \draw[->] (p12-9) -| (p19-10) ;
  \draw[->] (p13-9) -| (p19-10) ;
  \draw[->] (p14-9) -| (p19-10) ;
  \draw[->] (p15-9) -| (p19-10) ;
  \draw[->] (p16-9) -| (p19-10) ;
  \draw[->] (p17-9) -| (p19-10) ;
  \draw[->] (p18-7) -| (p19-10) ;
  \draw[->] (p0-5) -- (p0-11) ;
  \draw[->] (p0-11) -- (p0-12) ;
\end{tikzpicture}

\nonTerminalSection{project\_header}{0}

\ruleSubsection{galgas3ProjectSyntax}{galgasProject}{49}

\begin{tikzpicture}
  \matrix[column sep=\ruleMatrixColumnSeparation, row sep=\ruleMatrixRowSeparation] {
    \node (P0start) [firstPoint] {}; & & \node (p9-2) [terminal] {project}; & \\
    & & \node (p8-2) [terminal] {(}; & \\
    & & \node (p7-2) [terminal] {literalInt}; & \\
    & & \node (p6-2) [terminal] {:}; & \\
    & & \node (p5-2) [terminal] {literalInt}; & \\
    & & \node (p4-2) [terminal] {:}; & \\
    & & \node (p3-2) [terminal] {literalInt}; & \\
    & & \node (p2-2) [terminal] {)}; & \\
    & & \node (p1-2) [terminal] {->}; & \\
    & & \node (p0-2) [terminal] {"string"}; & \node (p0-3) [lastPoint] {}; & \\
  };
  \draw[->] (P0start) -- (p9-2) ;
  \draw[->] (p9-2) -- (p8-2) ;
  \draw[->] (p8-2) -- (p7-2) ;
  \draw[->] (p7-2) -- (p6-2) ;
  \draw[->] (p6-2) -- (p5-2) ;
  \draw[->] (p5-2) -- (p4-2) ;
  \draw[->] (p4-2) -- (p3-2) ;
  \draw[->] (p3-2) -- (p2-2) ;
  \draw[->] (p2-2) -- (p1-2) ;
  \draw[->] (p1-2) -- (p0-2) ;
  \draw[->] (p0-2) -- (p0-3) ;
\end{tikzpicture}




\renewcommand\nonTerminalSection[2]{\section{Non terminal \texttt{\it#1}}\label{nt1:#2}}
\renewcommand\nonTerminalSymbol[2]{\hyperref[nt1:#2]{#1}}
\renewcommand\startSymbol[2]{L'axiome de la grammaire est \hyperref[nt1:#2]{#1}.}
\renewcommand\nonTerminalSummary[2]{\hyperref[nt1:#2]{#1}}

\chapterLabel{Grammaire des sources GALGAS}{grammaireSource}

\startSymbol{start\_symbol}{32}

\nonTerminalSummaryStart \nonTerminalSummary{actual\_input\_parameter\_list}{19}\nonTerminalSummarySeparator \nonTerminalSummary{actual\_parameter\_list}{12}\nonTerminalSummarySeparator \nonTerminalSummary{branchOfParseWhithInstruction}{87}\nonTerminalSummarySeparator \nonTerminalSummary{cast\_else\_or\_default}{21}\nonTerminalSummarySeparator \nonTerminalSummary{cast\_instruction\_branch}{20}\nonTerminalSummarySeparator \nonTerminalSummary{casted\_expression}{2}\nonTerminalSummarySeparator \nonTerminalSummary{collection\_value\_element}{10}\nonTerminalSummarySeparator \nonTerminalSummary{declaration}{15}\nonTerminalSummarySeparator \nonTerminalSummary{declaration\_with\_private}{16}\nonTerminalSummarySeparator \nonTerminalSummary{expression}{1}\nonTerminalSummarySeparator \nonTerminalSummary{extern\_function\_declaration}{72}\nonTerminalSummarySeparator \nonTerminalSummary{extern\_routine\_declaration}{71}\nonTerminalSummarySeparator \nonTerminalSummary{externtype\_constructor}{39}\nonTerminalSummarySeparator \nonTerminalSummary{externtype\_cpp\_classdeclaration}{38}\nonTerminalSummarySeparator \nonTerminalSummary{externtype\_cpp\_predeclaration}{37}\nonTerminalSummarySeparator \nonTerminalSummary{externtype\_getter}{40}\nonTerminalSummarySeparator \nonTerminalSummary{externtype\_method}{42}\nonTerminalSummarySeparator \nonTerminalSummary{externtype\_setter}{41}\nonTerminalSummarySeparator \nonTerminalSummary{factor}{7}\nonTerminalSummarySeparator \nonTerminalSummary{filewrapper\_binary\_files}{35}\nonTerminalSummarySeparator \nonTerminalSummary{filewrapper\_templates}{36}\nonTerminalSummarySeparator \nonTerminalSummary{filewrapper\_text\_files}{34}\nonTerminalSummarySeparator \nonTerminalSummary{for\_instruction\_element}{24}\nonTerminalSummarySeparator \nonTerminalSummary{for\_instruction\_enumerated\_object}{25}\nonTerminalSummarySeparator \nonTerminalSummary{formal\_input\_parameter\_list}{13}\nonTerminalSummarySeparator \nonTerminalSummary{formal\_parameter\_list}{11}\nonTerminalSummarySeparator \nonTerminalSummary{grammar\_instruction\_core}{26}\nonTerminalSummarySeparator \nonTerminalSummary{grammar\_start\_symbol\_label}{88}\nonTerminalSummarySeparator \nonTerminalSummary{gui\_attributes}{80}\nonTerminalSummarySeparator \nonTerminalSummary{gui\_with\_lexique\_declaration}{78}\nonTerminalSummarySeparator \nonTerminalSummary{gui\_with\_option\_declaration}{79}\nonTerminalSummarySeparator \nonTerminalSummary{if\_instruction\_core}{27}\nonTerminalSummarySeparator \nonTerminalSummary{insert\_or\_replace\_declaration}{45}\nonTerminalSummarySeparator \nonTerminalSummary{issue\_fixit}{23}\nonTerminalSummarySeparator \nonTerminalSummary{label\_formal\_parameter}{89}\nonTerminalSummarySeparator \nonTerminalSummary{lexical\_attribute\_declaration}{68}\nonTerminalSummarySeparator \nonTerminalSummary{lexical\_explicit\_rule}{58}\nonTerminalSummarySeparator \nonTerminalSummary{lexical\_expression}{63}\nonTerminalSummarySeparator \nonTerminalSummary{lexical\_factor}{65}\nonTerminalSummarySeparator \nonTerminalSummary{lexical\_function\_declaration}{73}\nonTerminalSummarySeparator \nonTerminalSummary{lexical\_function\_expression}{74}\nonTerminalSummarySeparator \nonTerminalSummary{lexical\_function\_factor}{76}\nonTerminalSummarySeparator \nonTerminalSummary{lexical\_function\_term}{75}\nonTerminalSummarySeparator \nonTerminalSummary{lexical\_implicit\_rule}{57}\nonTerminalSummarySeparator \nonTerminalSummary{lexical\_indexing\_declaration}{53}\nonTerminalSummarySeparator \nonTerminalSummary{lexical\_instruction}{59}\nonTerminalSummarySeparator \nonTerminalSummary{lexical\_list\_declaration}{66}\nonTerminalSummarySeparator \nonTerminalSummary{lexical\_list\_entry}{67}\nonTerminalSummarySeparator \nonTerminalSummary{lexical\_message\_declaration}{56}\nonTerminalSummarySeparator \nonTerminalSummary{lexical\_output\_effective\_argument}{62}\nonTerminalSummarySeparator \nonTerminalSummary{lexical\_send\_instruction}{60}\nonTerminalSummarySeparator \nonTerminalSummary{lexical\_term}{64}\nonTerminalSummarySeparator \nonTerminalSummary{map\_insert\_setter\_declaration}{46}\nonTerminalSummarySeparator \nonTerminalSummary{match\_entry}{28}\nonTerminalSummarySeparator \nonTerminalSummary{match\_instruction\_branch}{29}\nonTerminalSummarySeparator \nonTerminalSummary{non\_empty\_output\_expression\_list}{22}\nonTerminalSummarySeparator \nonTerminalSummary{nonterminal\_declaration}{81}\nonTerminalSummarySeparator \nonTerminalSummary{option\_declaration}{77}\nonTerminalSummarySeparator \nonTerminalSummary{optional\_type}{9}\nonTerminalSummarySeparator \nonTerminalSummary{output\_expression\_list}{0}\nonTerminalSummarySeparator \nonTerminalSummary{primary}{8}\nonTerminalSummarySeparator \nonTerminalSummary{property\_declaration}{33}\nonTerminalSummarySeparator \nonTerminalSummary{relation\_factor}{4}\nonTerminalSummarySeparator \nonTerminalSummary{relation\_term}{3}\nonTerminalSummarySeparator \nonTerminalSummary{remove\_declaration}{44}\nonTerminalSummarySeparator \nonTerminalSummary{repeat\_while\_branch}{61}\nonTerminalSummarySeparator \nonTerminalSummary{search\_declaration}{43}\nonTerminalSummarySeparator \nonTerminalSummary{semantic\_instruction}{17}\nonTerminalSummarySeparator \nonTerminalSummary{semantic\_instruction\_list}{14}\nonTerminalSummarySeparator \nonTerminalSummary{shared\_map\_attribute}{48}\nonTerminalSummarySeparator \nonTerminalSummary{shared\_map\_override}{47}\nonTerminalSummarySeparator \nonTerminalSummary{shared\_map\_search\_method\_declaration}{49}\nonTerminalSummarySeparator \nonTerminalSummary{shared\_map\_state\_list}{50}\nonTerminalSummarySeparator \nonTerminalSummary{shared\_map\_state\_transition}{51}\nonTerminalSummarySeparator \nonTerminalSummary{simple\_expression}{5}\nonTerminalSummarySeparator \nonTerminalSummary{sortedlist\_sort\_descriptor}{52}\nonTerminalSummarySeparator \nonTerminalSummary{start\_symbol}{32}\nonTerminalSummarySeparator \nonTerminalSummary{style\_declaration}{70}\nonTerminalSummarySeparator \nonTerminalSummary{switch\_case}{30}\nonTerminalSummarySeparator \nonTerminalSummary{syntax\_directed\_translation\_result}{18}\nonTerminalSummarySeparator \nonTerminalSummary{syntax\_instruction}{85}\nonTerminalSummarySeparator \nonTerminalSummary{syntax\_instruction\_list}{84}\nonTerminalSummarySeparator \nonTerminalSummary{syntax\_rule\_declaration}{83}\nonTerminalSummarySeparator \nonTerminalSummary{syntax\_rule\_label}{82}\nonTerminalSummarySeparator \nonTerminalSummary{template\_delimitor}{54}\nonTerminalSummarySeparator \nonTerminalSummary{template\_replacement}{55}\nonTerminalSummarySeparator \nonTerminalSummary{term}{6}\nonTerminalSummarySeparator \nonTerminalSummary{terminal\_declaration}{69}\nonTerminalSummarySeparator \nonTerminalSummary{terminal\_instruction\_indexing}{86}\nonTerminalSummarySeparator \nonTerminalSummary{with\_instruction\_core}{31}\nonTerminalSummaryEnd \nonTerminalSection{actual\_input\_parameter\_list}{19}

\ruleSubsection{galgas3InstructionsSyntax}{galgas3InstructionsSyntax}{345}

\begin{tikzpicture}
  \matrix[column sep=\ruleMatrixColumnSeparation, row sep=\ruleMatrixRowSeparation] {
    & & & & & & & & & & & \node (p7-11) [point] {}; & \\
    & & & & & \node (p6-5) [terminal] {?}; & \node (p6-6) [terminal] {identifier}; & \\
    & & & & & & & & \node (p5-8) [terminal] {@type}; & \\
    & & & & & \node (p4-5) [terminal] {?}; & \node (p4-6) [terminal] {let}; & \node (p4-7) [point] {}; & \node (p4-8) [point] {}; & \node (p4-9) [point] {}; & \node (p4-10) [terminal] {identifier}; & \\
    & & & & & & & & \node (p3-8) [terminal] {@type}; & \\
    & & & & & \node (p2-5) [terminal] {?}; & \node (p2-6) [terminal] {var}; & \node (p2-7) [point] {}; & \node (p2-8) [point] {}; & \node (p2-9) [point] {}; & \node (p2-10) [terminal] {identifier}; & \\
    & & & & & \node (p1-5) [terminal] {?}; & \node (p1-6) [terminal] {*}; & \\
    \node (P0start) [firstPoint] {}; & & \node (p0-2) [point] {}; & \node (p0-3) [point] {}; & \node (p0-4) [point] {}; & & & & & & & & \node (p0-12) [lastPoint] {}; & \\
  };
  \draw (P0start) -- (p0-3) ;
  \draw[->] (p0-4) |- (p1-5) ;
  \draw[->] (p1-5) -- (p1-6) ;
  \draw[->] (p0-4) |- (p2-5) ;
  \draw[->] (p2-5) -- (p2-6) ;
  \draw (p2-6) -- (p2-8) ;
  \draw[->] (p2-7) |- (p3-8) ;
  \draw (p2-8) -- (p2-9) ;
  \draw[->] (p3-8) -| (p2-9) ;
  \draw[->] (p2-9) -- (p2-10) ;
  \draw[->] (p0-4) |- (p4-5) ;
  \draw[->] (p4-5) -- (p4-6) ;
  \draw (p4-6) -- (p4-8) ;
  \draw[->] (p4-7) |- (p5-8) ;
  \draw (p4-8) -- (p4-9) ;
  \draw[->] (p5-8) -| (p4-9) ;
  \draw[->] (p4-9) -- (p4-10) ;
  \draw[->] (p0-4) |- (p6-5) ;
  \draw[->] (p6-5) -- (p6-6) ;
  \draw[->] (p7-11) -| (p0-2) ;
  \draw[->] (p1-6) -| (p7-11) ;
  \draw[->] (p2-10) -| (p7-11) ;
  \draw[->] (p4-10) -| (p7-11) ;
  \draw[->] (p6-6) -| (p7-11) ;
  \draw[->] (p0-3) -- (p0-12) ;
\end{tikzpicture}

\nonTerminalSection{actual\_parameter\_list}{12}

\ruleSubsection{galgas3ParameterArgumentSyntax}{galgas3ParameterArgumentSyntax}{72}

\begin{tikzpicture}
  \matrix[column sep=\ruleMatrixColumnSeparation, row sep=\ruleMatrixRowSeparation] {
    & & & & & & & & & & & & & & \node (p15-14) [point] {}; & \\
    & & & & & & & & \node (p14-8) [terminal] {@type}; & & & \node (p14-11) [terminal] {unused}; & \\
    & & & & & \node (p13-5) [terminal] {?}; & \node (p13-6) [terminal] {let}; & \node (p13-7) [point] {}; & \node (p13-8) [point] {}; & \node (p13-9) [point] {}; & \node (p13-10) [point] {}; & \node (p13-11) [point] {}; & \node (p13-12) [point] {}; & \node (p13-13) [terminal] {identifier}; & \\
    & & & & & \node (p12-5) [terminal] {?}; & \node (p12-6) [terminal] {@type}; & \node (p12-7) [terminal] {identifier}; & \\
    & & & & & & & & \node (p11-8) [terminal] {@type}; & \\
    & & & & & \node (p10-5) [terminal] {?}; & \node (p10-6) [terminal] {var}; & \node (p10-7) [point] {}; & \node (p10-8) [point] {}; & \node (p10-9) [point] {}; & \node (p10-10) [terminal] {identifier}; & \\
    & & & & & \node (p9-5) [terminal] {?}; & \node (p9-6) [terminal] {identifier}; & \\
    & & & & & \node (p8-5) [terminal] {!?}; & \node (p8-6) [terminal] {uint32}; & \node (p8-7) [terminal] {*}; & \\
    & & & & & \node (p7-5) [terminal] {!?}; & \node (p7-6) [terminal] {*}; & \\
    & & & & & & & & & & & & \node (p6-12) [point] {}; & \\
    & & & & & & & & & & \node (p5-10) [terminal] {.}; & \node (p5-11) [terminal] {identifier}; & \\
    & & & & & \node (p4-5) [terminal] {!?}; & \node (p4-6) [terminal] {identifier}; & \node (p4-7) [point] {}; & \node (p4-8) [point] {}; & \node (p4-9) [point] {}; & \\
    & & & & & \node (p3-5) [terminal] {!}; & \node (p3-6) [nonterminal] {\nonTerminalSymbol{expression}{1}}; & \\
    & & & & & & & \node (p2-7) [terminal] {uint32}; & \node (p2-8) [terminal] {*}; & \\
    & & & & & \node (p1-5) [terminal] {?}; & \node (p1-6) [point] {}; & \node (p1-7) [terminal] {*}; & & \node (p1-9) [point] {}; & \\
    \node (P0start) [firstPoint] {}; & & \node (p0-2) [point] {}; & \node (p0-3) [point] {}; & \node (p0-4) [point] {}; & & & & & & & & & & & \node (p0-15) [lastPoint] {}; & \\
  };
  \draw (P0start) -- (p0-3) ;
  \draw[->] (p0-4) |- (p1-5) ;
  \draw[->] (p1-5) -- (p1-7) ;
  \draw[->] (p1-6) |- (p2-7) ;
  \draw[->] (p2-7) -- (p2-8) ;
  \draw (p1-7) -- (p1-9) ;
  \draw[->] (p2-8) -| (p1-9) ;
  \draw[->] (p0-4) |- (p3-5) ;
  \draw[->] (p3-5) -- (p3-6) ;
  \draw[->] (p0-4) |- (p4-5) ;
  \draw[->] (p4-5) -- (p4-6) ;
  \draw (p4-6) -- (p4-8) ;
  \draw[->] (p4-9) |- (p5-10) ;
  \draw[->] (p5-10) -- (p5-11) ;
  \draw[->] (p6-12) -| (p4-7) ;
  \draw[->] (p5-11) -| (p6-12) ;
  \draw[->] (p0-4) |- (p7-5) ;
  \draw[->] (p7-5) -- (p7-6) ;
  \draw[->] (p0-4) |- (p8-5) ;
  \draw[->] (p8-5) -- (p8-6) ;
  \draw[->] (p8-6) -- (p8-7) ;
  \draw[->] (p0-4) |- (p9-5) ;
  \draw[->] (p9-5) -- (p9-6) ;
  \draw[->] (p0-4) |- (p10-5) ;
  \draw[->] (p10-5) -- (p10-6) ;
  \draw (p10-6) -- (p10-8) ;
  \draw[->] (p10-7) |- (p11-8) ;
  \draw (p10-8) -- (p10-9) ;
  \draw[->] (p11-8) -| (p10-9) ;
  \draw[->] (p10-9) -- (p10-10) ;
  \draw[->] (p0-4) |- (p12-5) ;
  \draw[->] (p12-5) -- (p12-6) ;
  \draw[->] (p12-6) -- (p12-7) ;
  \draw[->] (p0-4) |- (p13-5) ;
  \draw[->] (p13-5) -- (p13-6) ;
  \draw (p13-6) -- (p13-8) ;
  \draw[->] (p13-7) |- (p14-8) ;
  \draw (p13-8) -- (p13-9) ;
  \draw[->] (p14-8) -| (p13-9) ;
  \draw (p13-9) -- (p13-11) ;
  \draw[->] (p13-10) |- (p14-11) ;
  \draw (p13-11) -- (p13-12) ;
  \draw[->] (p14-11) -| (p13-12) ;
  \draw[->] (p13-12) -- (p13-13) ;
  \draw[->] (p15-14) -| (p0-2) ;
  \draw[->] (p1-9) -| (p15-14) ;
  \draw[->] (p3-6) -| (p15-14) ;
  \draw[->] (p4-8) -| (p15-14) ;
  \draw[->] (p7-6) -| (p15-14) ;
  \draw[->] (p8-7) -| (p15-14) ;
  \draw[->] (p9-6) -| (p15-14) ;
  \draw[->] (p10-10) -| (p15-14) ;
  \draw[->] (p12-7) -| (p15-14) ;
  \draw[->] (p13-13) -| (p15-14) ;
  \draw[->] (p0-3) -- (p0-15) ;
\end{tikzpicture}

\nonTerminalSection{branchOfParseWhithInstruction}{87}

\ruleSubsection{galgas3SyntaxComponentSyntax}{instruction-parse-when}{22}

\begin{tikzpicture}
  \matrix[column sep=\ruleMatrixColumnSeparation, row sep=\ruleMatrixRowSeparation] {
    \node (P0start) [firstPoint] {}; & & \node (p0-2) [terminal] {else}; & \node (p0-3) [nonterminal] {\nonTerminalSymbol{syntax\_instruction\_list}{84}}; & \node (p0-4) [lastPoint] {}; & \\
  };
  \draw[->] (P0start) -- (p0-2) ;
  \draw[->] (p0-2) -- (p0-3) ;
  \draw[->] (p0-3) -- (p0-4) ;
\end{tikzpicture}

\ruleSubsection{galgas3SyntaxComponentSyntax}{instruction-parse-when}{29}

\begin{tikzpicture}
  \matrix[column sep=\ruleMatrixColumnSeparation, row sep=\ruleMatrixRowSeparation] {
    \node (P0start) [firstPoint] {}; & & \node (p4-2) [terminal] {case}; & \\
    & & \node (p3-2) [nonterminal] {\nonTerminalSymbol{expression}{1}}; & \\
    & & \node (p2-2) [terminal] {:}; & \\
    & & \node (p1-2) [nonterminal] {\nonTerminalSymbol{syntax\_instruction\_list}{84}}; & \\
    & & \node (p0-2) [nonterminal] {\nonTerminalSymbol{branchOfParseWhithInstruction}{87}}; & \node (p0-3) [lastPoint] {}; & \\
  };
  \draw[->] (P0start) -- (p4-2) ;
  \draw[->] (p4-2) -- (p3-2) ;
  \draw[->] (p3-2) -- (p2-2) ;
  \draw[->] (p2-2) -- (p1-2) ;
  \draw[->] (p1-2) -- (p0-2) ;
  \draw[->] (p0-2) -- (p0-3) ;
\end{tikzpicture}

\nonTerminalSection{cast\_else\_or\_default}{21}

\ruleSubsection{galgas3InstructionsSyntax}{instruction-cast}{62}

\begin{tikzpicture}
  \matrix[column sep=\ruleMatrixColumnSeparation, row sep=\ruleMatrixRowSeparation] {
    & & & \node (p2-3) [terminal] {default}; & \node (p2-4) [terminal] {error}; & \node (p2-5) [nonterminal] {\nonTerminalSymbol{expression}{1}}; & \\
    & & & \node (p1-3) [terminal] {else}; & \node (p1-4) [nonterminal] {\nonTerminalSymbol{semantic\_instruction\_list}{14}}; & \\
    \node (P0start) [firstPoint] {}; & & \node (p0-2) [point] {}; & \node (p0-3) [point] {}; & & & \node (p0-6) [point] {}; & \node (p0-7) [lastPoint] {}; & \\
  };
  \draw (P0start) -- (p0-3) ;
  \draw[->] (p0-2) |- (p1-3) ;
  \draw[->] (p1-3) -- (p1-4) ;
  \draw[->] (p0-2) |- (p2-3) ;
  \draw[->] (p2-3) -- (p2-4) ;
  \draw[->] (p2-4) -- (p2-5) ;
  \draw (p0-3) -- (p0-6) ;
  \draw[->] (p1-4) -| (p0-6) ;
  \draw[->] (p2-5) -| (p0-6) ;
  \draw[->] (p0-6) -- (p0-7) ;
\end{tikzpicture}

\nonTerminalSection{cast\_instruction\_branch}{20}

\ruleSubsection{galgas3InstructionsSyntax}{instruction-cast}{30}

\begin{tikzpicture}
  \matrix[column sep=\ruleMatrixColumnSeparation, row sep=\ruleMatrixRowSeparation] {
    & & & & \node (p2-4) [terminal] {>}; & \\
    & & & & \node (p1-4) [terminal] {>=}; & & & & \node (p1-8) [point] {}; & \\
    \node (P0start) [firstPoint] {}; & & \node (p0-2) [terminal] {case}; & \node (p0-3) [point] {}; & \node (p0-4) [terminal] {==}; & \node (p0-5) [point] {}; & \node (p0-6) [terminal] {@type}; & \node (p0-7) [point] {}; & \node (p0-8) [terminal] {identifier}; & \node (p0-9) [point] {}; & \node (p0-10) [terminal] {:}; & \node (p0-11) [nonterminal] {\nonTerminalSymbol{semantic\_instruction\_list}{14}}; & \node (p0-12) [lastPoint] {}; & \\
  };
  \draw[->] (P0start) -- (p0-2) ;
  \draw[->] (p0-2) -- (p0-4) ;
  \draw[->] (p0-3) |- (p1-4) ;
  \draw[->] (p0-3) |- (p2-4) ;
  \draw (p0-4) -- (p0-5) ;
  \draw[->] (p1-4) -| (p0-5) ;
  \draw[->] (p2-4) -| (p0-5) ;
  \draw[->] (p0-5) -- (p0-6) ;
  \draw[->] (p0-6) -- (p0-8) ;
  \draw (p0-7) |- (p1-8) ;
  \draw (p0-8) -- (p0-9) ;
  \draw[->] (p1-8) -| (p0-9) ;
  \draw[->] (p0-9) -- (p0-10) ;
  \draw[->] (p0-10) -- (p0-11) ;
  \draw[->] (p0-11) -- (p0-12) ;
\end{tikzpicture}

\nonTerminalSection{casted\_expression}{2}

\ruleSubsection{galgas3ExpressionSyntax}{galgas3ExpressionSyntax}{84}

\begin{tikzpicture}
  \matrix[column sep=\ruleMatrixColumnSeparation, row sep=\ruleMatrixRowSeparation] {
    & & & & & & & & \node (p6-8) [point] {}; & \\
    & & & & & & \node (p5-6) [terminal] {..<}; & \node (p5-7) [nonterminal] {\nonTerminalSymbol{relation\_term}{3}}; & \\
    & & & & & & \node (p4-6) [terminal] {...}; & \node (p4-7) [nonterminal] {\nonTerminalSymbol{relation\_term}{3}}; & \\
    & & & & & & \node (p3-6) [terminal] {\verb=^=}; & \node (p3-7) [nonterminal] {\nonTerminalSymbol{relation\_term}{3}}; & \\
    & & & & & & \node (p2-6) [terminal] {||}; & \node (p2-7) [nonterminal] {\nonTerminalSymbol{relation\_term}{3}}; & \\
    & & & & & & \node (p1-6) [terminal] {|}; & \node (p1-7) [nonterminal] {\nonTerminalSymbol{relation\_term}{3}}; & \\
    \node (P0start) [firstPoint] {}; & & \node (p0-2) [nonterminal] {\nonTerminalSymbol{relation\_term}{3}}; & \node (p0-3) [point] {}; & \node (p0-4) [point] {}; & \node (p0-5) [point] {}; & & & & \node (p0-9) [lastPoint] {}; & \\
  };
  \draw[->] (P0start) -- (p0-2) ;
  \draw (p0-2) -- (p0-4) ;
  \draw[->] (p0-5) |- (p1-6) ;
  \draw[->] (p1-6) -- (p1-7) ;
  \draw[->] (p0-5) |- (p2-6) ;
  \draw[->] (p2-6) -- (p2-7) ;
  \draw[->] (p0-5) |- (p3-6) ;
  \draw[->] (p3-6) -- (p3-7) ;
  \draw[->] (p0-5) |- (p4-6) ;
  \draw[->] (p4-6) -- (p4-7) ;
  \draw[->] (p0-5) |- (p5-6) ;
  \draw[->] (p5-6) -- (p5-7) ;
  \draw[->] (p6-8) -| (p0-3) ;
  \draw[->] (p1-7) -| (p6-8) ;
  \draw[->] (p2-7) -| (p6-8) ;
  \draw[->] (p3-7) -| (p6-8) ;
  \draw[->] (p4-7) -| (p6-8) ;
  \draw[->] (p5-7) -| (p6-8) ;
  \draw[->] (p0-4) -- (p0-9) ;
\end{tikzpicture}

\nonTerminalSection{collection\_value\_element}{10}

\ruleSubsection{galgas3ExpressionSyntax}{galgas3ExpressionSyntax}{696}

\begin{tikzpicture}
  \matrix[column sep=\ruleMatrixColumnSeparation, row sep=\ruleMatrixRowSeparation] {
    & & & & & & \node (p1-6) [point] {}; & \\
    \node (P0start) [firstPoint] {}; & & \node (p0-2) [point] {}; & \node (p0-3) [terminal] {!}; & \node (p0-4) [nonterminal] {\nonTerminalSymbol{expression}{1}}; & \node (p0-5) [point] {}; & & \node (p0-7) [lastPoint] {}; & \\
  };
  \draw[->] (P0start) -- (p0-3) ;
  \draw[->] (p0-3) -- (p0-4) ;
  \draw[->] (p1-6) -| (p0-2) ;
  \draw[->] (p0-5) -| (p1-6) ;
  \draw[->] (p0-4) -- (p0-7) ;
\end{tikzpicture}

\ruleSubsection{galgas3ExpressionSyntax}{galgas3ExpressionSyntax}{709}

\begin{tikzpicture}
  \matrix[column sep=\ruleMatrixColumnSeparation, row sep=\ruleMatrixRowSeparation] {
    \node (P0start) [firstPoint] {}; & & \node (p0-2) [nonterminal] {\nonTerminalSymbol{expression}{1}}; & \node (p0-3) [lastPoint] {}; & \\
  };
  \draw[->] (P0start) -- (p0-2) ;
  \draw[->] (p0-2) -- (p0-3) ;
\end{tikzpicture}

\nonTerminalSection{declaration}{15}

\ruleSubsection{galgas3InstructionsSyntax}{galgas3InstructionsSyntax}{49}

\begin{tikzpicture}
  \matrix[column sep=\ruleMatrixColumnSeparation, row sep=\ruleMatrixRowSeparation] {
    & & & \node (p1-3) [terminal] {private}; & \\
    \node (P0start) [firstPoint] {}; & & \node (p0-2) [point] {}; & \node (p0-3) [point] {}; & \node (p0-4) [point] {}; & \node (p0-5) [nonterminal] {\nonTerminalSymbol{declaration\_with\_private}{16}}; & \node (p0-6) [lastPoint] {}; & \\
  };
  \draw (P0start) -- (p0-3) ;
  \draw[->] (p0-2) |- (p1-3) ;
  \draw (p0-3) -- (p0-4) ;
  \draw[->] (p1-3) -| (p0-4) ;
  \draw[->] (p0-4) -- (p0-5) ;
  \draw[->] (p0-5) -- (p0-6) ;
\end{tikzpicture}

\ruleSubsection{galgas3InstructionsSyntax}{galgas3InstructionsSyntax}{67}

\begin{tikzpicture}
  \matrix[column sep=\ruleMatrixColumnSeparation, row sep=\ruleMatrixRowSeparation] {
    \node (P0start) [firstPoint] {}; & & \node (p0-2) [terminal] {extern}; & \node (p0-3) [terminal] {proc}; & \node (p0-4) [terminal] {identifier}; & \node (p0-5) [nonterminal] {\nonTerminalSymbol{formal\_parameter\_list}{11}}; & \node (p0-6) [lastPoint] {}; & \\
  };
  \draw[->] (P0start) -- (p0-2) ;
  \draw[->] (p0-2) -- (p0-3) ;
  \draw[->] (p0-3) -- (p0-4) ;
  \draw[->] (p0-4) -- (p0-5) ;
  \draw[->] (p0-5) -- (p0-6) ;
\end{tikzpicture}

\ruleSubsection{galgas3InstructionsSyntax}{galgas3InstructionsSyntax}{199}

\begin{tikzpicture}
  \matrix[column sep=\ruleMatrixColumnSeparation, row sep=\ruleMatrixRowSeparation] {
    \node (P0start) [firstPoint] {}; & & \node (p5-2) [terminal] {extern}; & \\
    & & \node (p4-2) [terminal] {func}; & \\
    & & \node (p3-2) [terminal] {identifier}; & \\
    & & \node (p2-2) [nonterminal] {\nonTerminalSymbol{formal\_input\_parameter\_list}{13}}; & \\
    & & \node (p1-2) [terminal] {->}; & \\
    & & \node (p0-2) [terminal] {@type}; & \node (p0-3) [lastPoint] {}; & \\
  };
  \draw[->] (P0start) -- (p5-2) ;
  \draw[->] (p5-2) -- (p4-2) ;
  \draw[->] (p4-2) -- (p3-2) ;
  \draw[->] (p3-2) -- (p2-2) ;
  \draw[->] (p2-2) -- (p1-2) ;
  \draw[->] (p1-2) -- (p0-2) ;
  \draw[->] (p0-2) -- (p0-3) ;
\end{tikzpicture}

\ruleSubsection{galgas3DeclarationsSyntax}{type-array}{17}

\begin{tikzpicture}
  \matrix[column sep=\ruleMatrixColumnSeparation, row sep=\ruleMatrixRowSeparation] {
    \node (P0start) [firstPoint] {}; & & \node (p6-2) [terminal] {array}; & \\
    & & \node (p5-2) [terminal] {@type}; & \\
    & & \node (p4-2) [terminal] {:}; & \\
    & & \node (p3-2) [terminal] {@type}; & \\
    & & \node (p2-2) [terminal] {[}; & \\
    & & \node (p1-2) [terminal] {uint32}; & \\
    & & \node (p0-2) [terminal] {]}; & \node (p0-3) [lastPoint] {}; & \\
  };
  \draw[->] (P0start) -- (p6-2) ;
  \draw[->] (p6-2) -- (p5-2) ;
  \draw[->] (p5-2) -- (p4-2) ;
  \draw[->] (p4-2) -- (p3-2) ;
  \draw[->] (p3-2) -- (p2-2) ;
  \draw[->] (p2-2) -- (p1-2) ;
  \draw[->] (p1-2) -- (p0-2) ;
  \draw[->] (p0-2) -- (p0-3) ;
\end{tikzpicture}

\ruleSubsection{galgas3DeclarationsSyntax}{type-class}{21}

\begin{tikzpicture}
  \matrix[column sep=\ruleMatrixColumnSeparation, row sep=\ruleMatrixRowSeparation] {
    & & & & & \node (p3-5) [terminal] {shared}; & & & & & & & & & & & & & & & & & & & & & & & \node (p3-28) [point] {}; & \\
    & & & \node (p2-3) [terminal] {abstract}; & \node (p2-4) [point] {}; & \node (p2-5) [point] {}; & \node (p2-6) [point] {}; & & & & & & & \node (p2-13) [point] {}; & & & & & & & & & \node (p2-22) [point] {}; & & & & & \node (p2-27) [terminal] {;}; & \\
    & & & \node (p1-3) [terminal] {shared}; & & & & & & & & & \node (p1-12) [terminal] {,}; & & & \node (p1-15) [terminal] {:}; & \node (p1-16) [terminal] {@type}; & & & & & \node (p1-21) [terminal] {\verb=%=attribute}; & & & & & & \node (p1-27) [nonterminal] {\nonTerminalSymbol{property\_declaration}{33}}; & \\
    \node (P0start) [firstPoint] {}; & & \node (p0-2) [point] {}; & \node (p0-3) [point] {}; & & & & \node (p0-7) [point] {}; & \node (p0-8) [terminal] {class}; & \node (p0-9) [point] {}; & \node (p0-10) [terminal] {@type}; & \node (p0-11) [point] {}; & & & \node (p0-14) [point] {}; & \node (p0-15) [point] {}; & & \node (p0-17) [point] {}; & \node (p0-18) [point] {}; & \node (p0-19) [point] {}; & \node (p0-20) [point] {}; & & & \node (p0-23) [terminal] {\{}; & \node (p0-24) [point] {}; & \node (p0-25) [point] {}; & \node (p0-26) [point] {}; & & & \node (p0-29) [terminal] {\}}; & \node (p0-30) [lastPoint] {}; & \\
  };
  \draw (P0start) -- (p0-3) ;
  \draw[->] (p0-2) |- (p1-3) ;
  \draw[->] (p0-2) |- (p2-3) ;
  \draw (p2-3) -- (p2-5) ;
  \draw[->] (p2-4) |- (p3-5) ;
  \draw (p2-5) -- (p2-6) ;
  \draw[->] (p3-5) -| (p2-6) ;
  \draw (p0-3) -- (p0-7) ;
  \draw[->] (p1-3) -| (p0-7) ;
  \draw[->] (p2-6) -| (p0-7) ;
  \draw[->] (p0-7) -- (p0-8) ;
  \draw[->] (p0-8) -- (p0-10) ;
  \draw[->] (p0-11) |- (p1-12) ;
  \draw[->] (p2-13) -| (p0-9) ;
  \draw[->] (p1-12) -| (p2-13) ;
  \draw (p0-10) -- (p0-15) ;
  \draw[->] (p0-14) |- (p1-15) ;
  \draw[->] (p1-15) -- (p1-16) ;
  \draw (p0-15) -- (p0-17) ;
  \draw[->] (p1-16) -| (p0-17) ;
  \draw (p0-17) -- (p0-19) ;
  \draw[->] (p0-20) |- (p1-21) ;
  \draw[->] (p2-22) -| (p0-18) ;
  \draw[->] (p1-21) -| (p2-22) ;
  \draw[->] (p0-19) -- (p0-23) ;
  \draw (p0-23) -- (p0-25) ;
  \draw[->] (p0-26) |- (p1-27) ;
  \draw[->] (p0-26) |- (p2-27) ;
  \draw[->] (p3-28) -| (p0-24) ;
  \draw[->] (p1-27) -| (p3-28) ;
  \draw[->] (p2-27) -| (p3-28) ;
  \draw[->] (p0-25) -- (p0-29) ;
  \draw[->] (p0-29) -- (p0-30) ;
\end{tikzpicture}

\ruleSubsection{galgas3DeclarationsSyntax}{type-enum}{28}

\begin{tikzpicture}
  \matrix[column sep=\ruleMatrixColumnSeparation, row sep=\ruleMatrixRowSeparation] {
    & & & & & & & & & & & & & & & & & & & & \node (p4-20) [point] {}; & \\
    & & & & & & & & & & & & & & & & \node (p3-16) [point] {}; & \\
    & & & & & & & & & & & & & \node (p2-13) [terminal] {identifier}; & \\
    & & & & & & & & & \node (p1-9) [terminal] {(}; & \node (p1-10) [point] {}; & \node (p1-11) [terminal] {@type}; & \node (p1-12) [point] {}; & \node (p1-13) [point] {}; & \node (p1-14) [point] {}; & \node (p1-15) [point] {}; & & \node (p1-17) [terminal] {)}; & \\
    \node (P0start) [firstPoint] {}; & & \node (p0-2) [terminal] {enum}; & \node (p0-3) [terminal] {@type}; & \node (p0-4) [terminal] {\{}; & \node (p0-5) [point] {}; & \node (p0-6) [terminal] {case}; & \node (p0-7) [terminal] {identifier}; & \node (p0-8) [point] {}; & \node (p0-9) [point] {}; & & & & & & & & & \node (p0-18) [point] {}; & \node (p0-19) [point] {}; & & \node (p0-21) [terminal] {\}}; & \node (p0-22) [lastPoint] {}; & \\
  };
  \draw[->] (P0start) -- (p0-2) ;
  \draw[->] (p0-2) -- (p0-3) ;
  \draw[->] (p0-3) -- (p0-4) ;
  \draw[->] (p0-4) -- (p0-6) ;
  \draw[->] (p0-6) -- (p0-7) ;
  \draw (p0-7) -- (p0-9) ;
  \draw[->] (p0-8) |- (p1-9) ;
  \draw[->] (p1-9) -- (p1-11) ;
  \draw (p1-11) -- (p1-13) ;
  \draw[->] (p1-12) |- (p2-13) ;
  \draw (p1-13) -- (p1-14) ;
  \draw[->] (p2-13) -| (p1-14) ;
  \draw[->] (p3-16) -| (p1-10) ;
  \draw[->] (p1-15) -| (p3-16) ;
  \draw[->] (p1-14) -- (p1-17) ;
  \draw (p0-9) -- (p0-18) ;
  \draw[->] (p1-17) -| (p0-18) ;
  \draw[->] (p4-20) -| (p0-5) ;
  \draw[->] (p0-19) -| (p4-20) ;
  \draw[->] (p0-18) -- (p0-21) ;
  \draw[->] (p0-21) -- (p0-22) ;
\end{tikzpicture}

\ruleSubsection{galgas3DeclarationsSyntax}{type-extern}{63}

\begin{tikzpicture}
  \matrix[column sep=\ruleMatrixColumnSeparation, row sep=\ruleMatrixRowSeparation] {
    & & & & & & & & & & & \node (p5-11) [point] {}; & \\
    & & & & & & & & & & \node (p4-10) [nonterminal] {\nonTerminalSymbol{externtype\_method}{42}}; & \\
    & & & & & & & & & & \node (p3-10) [nonterminal] {\nonTerminalSymbol{externtype\_setter}{41}}; & \\
    & & & & & & & & & & \node (p2-10) [nonterminal] {\nonTerminalSymbol{externtype\_getter}{40}}; & \\
    & & & & & & & & & & \node (p1-10) [nonterminal] {\nonTerminalSymbol{externtype\_constructor}{39}}; & \\
    \node (P0start) [firstPoint] {}; & & \node (p0-2) [terminal] {extern}; & \node (p0-3) [terminal] {@type}; & \node (p0-4) [nonterminal] {\nonTerminalSymbol{externtype\_cpp\_predeclaration}{37}}; & \node (p0-5) [nonterminal] {\nonTerminalSymbol{externtype\_cpp\_classdeclaration}{38}}; & \node (p0-6) [terminal] {\{}; & \node (p0-7) [point] {}; & \node (p0-8) [point] {}; & \node (p0-9) [point] {}; & & & \node (p0-12) [terminal] {\}}; & \node (p0-13) [lastPoint] {}; & \\
  };
  \draw[->] (P0start) -- (p0-2) ;
  \draw[->] (p0-2) -- (p0-3) ;
  \draw[->] (p0-3) -- (p0-4) ;
  \draw[->] (p0-4) -- (p0-5) ;
  \draw[->] (p0-5) -- (p0-6) ;
  \draw (p0-6) -- (p0-8) ;
  \draw[->] (p0-9) |- (p1-10) ;
  \draw[->] (p0-9) |- (p2-10) ;
  \draw[->] (p0-9) |- (p3-10) ;
  \draw[->] (p0-9) |- (p4-10) ;
  \draw[->] (p5-11) -| (p0-7) ;
  \draw[->] (p1-10) -| (p5-11) ;
  \draw[->] (p2-10) -| (p5-11) ;
  \draw[->] (p3-10) -| (p5-11) ;
  \draw[->] (p4-10) -| (p5-11) ;
  \draw[->] (p0-8) -- (p0-12) ;
  \draw[->] (p0-12) -- (p0-13) ;
\end{tikzpicture}

\ruleSubsection{galgas3DeclarationsSyntax}{type-graph}{24}

\begin{tikzpicture}
  \matrix[column sep=\ruleMatrixColumnSeparation, row sep=\ruleMatrixRowSeparation] {
    & & & & & & & & & & & & & & & & \node (p2-16) [point] {}; & \\
    & & & & & & & & & & & \node (p1-11) [terminal] {insert}; & \node (p1-12) [terminal] {identifier}; & \node (p1-13) [terminal] {error}; & \node (p1-14) [terminal] {message}; & \node (p1-15) [terminal] {"string"}; & \\
    \node (P0start) [firstPoint] {}; & & \node (p0-2) [terminal] {graph}; & \node (p0-3) [terminal] {@type}; & \node (p0-4) [terminal] {(}; & \node (p0-5) [terminal] {@type}; & \node (p0-6) [terminal] {)}; & \node (p0-7) [terminal] {\{}; & \node (p0-8) [point] {}; & \node (p0-9) [point] {}; & \node (p0-10) [point] {}; & & & & & & & \node (p0-17) [terminal] {\}}; & \node (p0-18) [lastPoint] {}; & \\
  };
  \draw[->] (P0start) -- (p0-2) ;
  \draw[->] (p0-2) -- (p0-3) ;
  \draw[->] (p0-3) -- (p0-4) ;
  \draw[->] (p0-4) -- (p0-5) ;
  \draw[->] (p0-5) -- (p0-6) ;
  \draw[->] (p0-6) -- (p0-7) ;
  \draw (p0-7) -- (p0-9) ;
  \draw[->] (p0-10) |- (p1-11) ;
  \draw[->] (p1-11) -- (p1-12) ;
  \draw[->] (p1-12) -- (p1-13) ;
  \draw[->] (p1-13) -- (p1-14) ;
  \draw[->] (p1-14) -- (p1-15) ;
  \draw[->] (p2-16) -| (p0-8) ;
  \draw[->] (p1-15) -| (p2-16) ;
  \draw[->] (p0-9) -- (p0-17) ;
  \draw[->] (p0-17) -- (p0-18) ;
\end{tikzpicture}

\ruleSubsection{galgas3DeclarationsSyntax}{type-list}{16}

\begin{tikzpicture}
  \matrix[column sep=\ruleMatrixColumnSeparation, row sep=\ruleMatrixRowSeparation] {
    & & & & & & & & & & & \node (p2-11) [point] {}; & \\
    & & & & & & & & \node (p1-8) [terminal] {;}; & \\
    \node (P0start) [firstPoint] {}; & & \node (p0-2) [terminal] {list}; & \node (p0-3) [terminal] {@type}; & \node (p0-4) [terminal] {\{}; & \node (p0-5) [point] {}; & \node (p0-6) [nonterminal] {\nonTerminalSymbol{property\_declaration}{33}}; & \node (p0-7) [point] {}; & \node (p0-8) [point] {}; & \node (p0-9) [point] {}; & \node (p0-10) [point] {}; & & \node (p0-12) [terminal] {\}}; & \node (p0-13) [lastPoint] {}; & \\
  };
  \draw[->] (P0start) -- (p0-2) ;
  \draw[->] (p0-2) -- (p0-3) ;
  \draw[->] (p0-3) -- (p0-4) ;
  \draw[->] (p0-4) -- (p0-6) ;
  \draw (p0-6) -- (p0-8) ;
  \draw[->] (p0-7) |- (p1-8) ;
  \draw (p0-8) -- (p0-9) ;
  \draw[->] (p1-8) -| (p0-9) ;
  \draw[->] (p2-11) -| (p0-5) ;
  \draw[->] (p0-10) -| (p2-11) ;
  \draw[->] (p0-9) -- (p0-12) ;
  \draw[->] (p0-12) -- (p0-13) ;
\end{tikzpicture}

\ruleSubsection{galgas3DeclarationsSyntax}{type-listmap}{16}

\begin{tikzpicture}
  \matrix[column sep=\ruleMatrixColumnSeparation, row sep=\ruleMatrixRowSeparation] {
    \node (P0start) [firstPoint] {}; & & \node (p4-2) [terminal] {listmap}; & \\
    & & \node (p3-2) [terminal] {@type}; & \\
    & & \node (p2-2) [terminal] {(}; & \\
    & & \node (p1-2) [terminal] {@type}; & \\
    & & \node (p0-2) [terminal] {)}; & \node (p0-3) [lastPoint] {}; & \\
  };
  \draw[->] (P0start) -- (p4-2) ;
  \draw[->] (p4-2) -- (p3-2) ;
  \draw[->] (p3-2) -- (p2-2) ;
  \draw[->] (p2-2) -- (p1-2) ;
  \draw[->] (p1-2) -- (p0-2) ;
  \draw[->] (p0-2) -- (p0-3) ;
\end{tikzpicture}

\ruleSubsection{galgas3DeclarationsSyntax}{type-map}{54}

\begin{tikzpicture}
  \matrix[column sep=\ruleMatrixColumnSeparation, row sep=\ruleMatrixRowSeparation] {
    & & & & & & & & & & & & & & \node (p7-14) [point] {}; & \\
    & & & & & & & & & & & & & \node (p6-13) [terminal] {;}; & \\
    & & & & & & & & & & & & & \node (p5-13) [nonterminal] {\nonTerminalSymbol{insert\_or\_replace\_declaration}{45}}; & \\
    & & & & & & & & & & & & & \node (p4-13) [nonterminal] {\nonTerminalSymbol{remove\_declaration}{44}}; & \\
    & & & & & & & & & & & & & \node (p3-13) [nonterminal] {\nonTerminalSymbol{search\_declaration}{43}}; & \\
    & & & & & & & & \node (p2-8) [point] {}; & & & & & \node (p2-13) [nonterminal] {\nonTerminalSymbol{map\_insert\_setter\_declaration}{46}}; & \\
    & & & & & & & \node (p1-7) [terminal] {\verb=%=attribute}; & & & & & & \node (p1-13) [nonterminal] {\nonTerminalSymbol{property\_declaration}{33}}; & \\
    \node (P0start) [firstPoint] {}; & & \node (p0-2) [terminal] {map}; & \node (p0-3) [terminal] {@type}; & \node (p0-4) [point] {}; & \node (p0-5) [point] {}; & \node (p0-6) [point] {}; & & & \node (p0-9) [terminal] {\{}; & \node (p0-10) [point] {}; & \node (p0-11) [point] {}; & \node (p0-12) [point] {}; & & & \node (p0-15) [terminal] {\}}; & \node (p0-16) [lastPoint] {}; & \\
  };
  \draw[->] (P0start) -- (p0-2) ;
  \draw[->] (p0-2) -- (p0-3) ;
  \draw (p0-3) -- (p0-5) ;
  \draw[->] (p0-6) |- (p1-7) ;
  \draw[->] (p2-8) -| (p0-4) ;
  \draw[->] (p1-7) -| (p2-8) ;
  \draw[->] (p0-5) -- (p0-9) ;
  \draw (p0-9) -- (p0-11) ;
  \draw[->] (p0-12) |- (p1-13) ;
  \draw[->] (p0-12) |- (p2-13) ;
  \draw[->] (p0-12) |- (p3-13) ;
  \draw[->] (p0-12) |- (p4-13) ;
  \draw[->] (p0-12) |- (p5-13) ;
  \draw[->] (p0-12) |- (p6-13) ;
  \draw[->] (p7-14) -| (p0-10) ;
  \draw[->] (p1-13) -| (p7-14) ;
  \draw[->] (p2-13) -| (p7-14) ;
  \draw[->] (p3-13) -| (p7-14) ;
  \draw[->] (p4-13) -| (p7-14) ;
  \draw[->] (p5-13) -| (p7-14) ;
  \draw[->] (p6-13) -| (p7-14) ;
  \draw[->] (p0-11) -- (p0-15) ;
  \draw[->] (p0-15) -- (p0-16) ;
\end{tikzpicture}

\ruleSubsection{galgas3DeclarationsSyntax}{type-shared-map}{91}

\begin{tikzpicture}
  \matrix[column sep=\ruleMatrixColumnSeparation, row sep=\ruleMatrixRowSeparation] {
    & & & & & & & & & & & & & & & \node (p8-15) [point] {}; & \\
    & & & & & & & & & & & & & & \node (p7-14) [terminal] {;}; & \\
    & & & & & & & & & & & & & & \node (p6-14) [nonterminal] {\nonTerminalSymbol{shared\_map\_override}{47}}; & \\
    & & & & & & & & & & & & & & \node (p5-14) [nonterminal] {\nonTerminalSymbol{shared\_map\_search\_method\_declaration}{49}}; & \\
    & & & & & & & & & & & & & & \node (p4-14) [nonterminal] {\nonTerminalSymbol{map\_insert\_setter\_declaration}{46}}; & \\
    & & & & & & & & & & & & & & \node (p3-14) [nonterminal] {\nonTerminalSymbol{property\_declaration}{33}}; & \\
    & & & & & & & & & \node (p2-9) [point] {}; & & & & & \node (p2-14) [nonterminal] {\nonTerminalSymbol{shared\_map\_attribute}{48}}; & \\
    & & & & & & & & \node (p1-8) [terminal] {\verb=%=attribute}; & & & & & & \node (p1-14) [nonterminal] {\nonTerminalSymbol{shared\_map\_state\_list}{50}}; & \\
    \node (P0start) [firstPoint] {}; & & \node (p0-2) [terminal] {shared}; & \node (p0-3) [terminal] {map}; & \node (p0-4) [terminal] {@type}; & \node (p0-5) [point] {}; & \node (p0-6) [point] {}; & \node (p0-7) [point] {}; & & & \node (p0-10) [terminal] {\{}; & \node (p0-11) [point] {}; & \node (p0-12) [point] {}; & \node (p0-13) [point] {}; & & & \node (p0-16) [terminal] {\}}; & \node (p0-17) [lastPoint] {}; & \\
  };
  \draw[->] (P0start) -- (p0-2) ;
  \draw[->] (p0-2) -- (p0-3) ;
  \draw[->] (p0-3) -- (p0-4) ;
  \draw (p0-4) -- (p0-6) ;
  \draw[->] (p0-7) |- (p1-8) ;
  \draw[->] (p2-9) -| (p0-5) ;
  \draw[->] (p1-8) -| (p2-9) ;
  \draw[->] (p0-6) -- (p0-10) ;
  \draw (p0-10) -- (p0-12) ;
  \draw[->] (p0-13) |- (p1-14) ;
  \draw[->] (p0-13) |- (p2-14) ;
  \draw[->] (p0-13) |- (p3-14) ;
  \draw[->] (p0-13) |- (p4-14) ;
  \draw[->] (p0-13) |- (p5-14) ;
  \draw[->] (p0-13) |- (p6-14) ;
  \draw[->] (p0-13) |- (p7-14) ;
  \draw[->] (p8-15) -| (p0-11) ;
  \draw[->] (p1-14) -| (p8-15) ;
  \draw[->] (p2-14) -| (p8-15) ;
  \draw[->] (p3-14) -| (p8-15) ;
  \draw[->] (p4-14) -| (p8-15) ;
  \draw[->] (p5-14) -| (p8-15) ;
  \draw[->] (p6-14) -| (p8-15) ;
  \draw[->] (p7-14) -| (p8-15) ;
  \draw[->] (p0-12) -- (p0-16) ;
  \draw[->] (p0-16) -- (p0-17) ;
\end{tikzpicture}

\ruleSubsection{galgas3DeclarationsSyntax}{type-sorted-list}{26}

\begin{tikzpicture}
  \matrix[column sep=\ruleMatrixColumnSeparation, row sep=\ruleMatrixRowSeparation] {
    & & & & & & & & & \node (p3-9) [point] {}; & \\
    & & & & & & & & \node (p2-8) [terminal] {;}; & & & & & & & & \node (p2-16) [point] {}; & \\
    & & & & & & & & \node (p1-8) [nonterminal] {\nonTerminalSymbol{property\_declaration}{33}}; & & & & & & & \node (p1-15) [terminal] {,}; & \\
    \node (P0start) [firstPoint] {}; & & \node (p0-2) [terminal] {sortedlist}; & \node (p0-3) [terminal] {@type}; & \node (p0-4) [terminal] {\{}; & \node (p0-5) [point] {}; & \node (p0-6) [point] {}; & \node (p0-7) [point] {}; & & & \node (p0-10) [terminal] {\}}; & \node (p0-11) [terminal] {\{}; & \node (p0-12) [point] {}; & \node (p0-13) [nonterminal] {\nonTerminalSymbol{sortedlist\_sort\_descriptor}{52}}; & \node (p0-14) [point] {}; & & & \node (p0-17) [terminal] {\}}; & \node (p0-18) [lastPoint] {}; & \\
  };
  \draw[->] (P0start) -- (p0-2) ;
  \draw[->] (p0-2) -- (p0-3) ;
  \draw[->] (p0-3) -- (p0-4) ;
  \draw (p0-4) -- (p0-6) ;
  \draw[->] (p0-7) |- (p1-8) ;
  \draw[->] (p0-7) |- (p2-8) ;
  \draw[->] (p3-9) -| (p0-5) ;
  \draw[->] (p1-8) -| (p3-9) ;
  \draw[->] (p2-8) -| (p3-9) ;
  \draw[->] (p0-6) -- (p0-10) ;
  \draw[->] (p0-10) -- (p0-11) ;
  \draw[->] (p0-11) -- (p0-13) ;
  \draw[->] (p0-14) |- (p1-15) ;
  \draw[->] (p2-16) -| (p0-12) ;
  \draw[->] (p1-15) -| (p2-16) ;
  \draw[->] (p0-13) -- (p0-17) ;
  \draw[->] (p0-17) -- (p0-18) ;
\end{tikzpicture}

\ruleSubsection{galgas3DeclarationsSyntax}{type-struct}{17}

\begin{tikzpicture}
  \matrix[column sep=\ruleMatrixColumnSeparation, row sep=\ruleMatrixRowSeparation] {
    & & & & & & & & & \node (p3-9) [point] {}; & \\
    & & & & & & & & \node (p2-8) [terminal] {;}; & \\
    & & & & & & & & \node (p1-8) [nonterminal] {\nonTerminalSymbol{property\_declaration}{33}}; & \\
    \node (P0start) [firstPoint] {}; & & \node (p0-2) [terminal] {struct}; & \node (p0-3) [terminal] {@type}; & \node (p0-4) [terminal] {\{}; & \node (p0-5) [point] {}; & \node (p0-6) [point] {}; & \node (p0-7) [point] {}; & & & \node (p0-10) [terminal] {\}}; & \node (p0-11) [lastPoint] {}; & \\
  };
  \draw[->] (P0start) -- (p0-2) ;
  \draw[->] (p0-2) -- (p0-3) ;
  \draw[->] (p0-3) -- (p0-4) ;
  \draw (p0-4) -- (p0-6) ;
  \draw[->] (p0-7) |- (p1-8) ;
  \draw[->] (p0-7) |- (p2-8) ;
  \draw[->] (p3-9) -| (p0-5) ;
  \draw[->] (p1-8) -| (p3-9) ;
  \draw[->] (p2-8) -| (p3-9) ;
  \draw[->] (p0-6) -- (p0-10) ;
  \draw[->] (p0-10) -- (p0-11) ;
\end{tikzpicture}

\ruleSubsection{galgas3DeclarationsSyntax}{extension-abstract-getter}{20}

\begin{tikzpicture}
  \matrix[column sep=\ruleMatrixColumnSeparation, row sep=\ruleMatrixRowSeparation] {
    & & & & & & & & & & \node (p1-10) [terminal] {identifier}; & \\
    \node (P0start) [firstPoint] {}; & & \node (p0-2) [terminal] {abstract}; & \node (p0-3) [terminal] {getter}; & \node (p0-4) [terminal] {@type}; & \node (p0-5) [terminal] {identifier}; & \node (p0-6) [nonterminal] {\nonTerminalSymbol{formal\_input\_parameter\_list}{13}}; & \node (p0-7) [terminal] {->}; & \node (p0-8) [terminal] {@type}; & \node (p0-9) [point] {}; & \node (p0-10) [point] {}; & \node (p0-11) [point] {}; & \node (p0-12) [lastPoint] {}; & \\
  };
  \draw[->] (P0start) -- (p0-2) ;
  \draw[->] (p0-2) -- (p0-3) ;
  \draw[->] (p0-3) -- (p0-4) ;
  \draw[->] (p0-4) -- (p0-5) ;
  \draw[->] (p0-5) -- (p0-6) ;
  \draw[->] (p0-6) -- (p0-7) ;
  \draw[->] (p0-7) -- (p0-8) ;
  \draw (p0-8) -- (p0-10) ;
  \draw[->] (p0-9) |- (p1-10) ;
  \draw (p0-10) -- (p0-11) ;
  \draw[->] (p1-10) -| (p0-11) ;
  \draw[->] (p0-11) -- (p0-12) ;
\end{tikzpicture}

\ruleSubsection{galgas3DeclarationsSyntax}{extension-abstract-method}{19}

\begin{tikzpicture}
  \matrix[column sep=\ruleMatrixColumnSeparation, row sep=\ruleMatrixRowSeparation] {
    \node (P0start) [firstPoint] {}; & & \node (p4-2) [terminal] {abstract}; & \\
    & & \node (p3-2) [terminal] {method}; & \\
    & & \node (p2-2) [terminal] {@type}; & \\
    & & \node (p1-2) [terminal] {identifier}; & \\
    & & \node (p0-2) [nonterminal] {\nonTerminalSymbol{formal\_parameter\_list}{11}}; & \node (p0-3) [lastPoint] {}; & \\
  };
  \draw[->] (P0start) -- (p4-2) ;
  \draw[->] (p4-2) -- (p3-2) ;
  \draw[->] (p3-2) -- (p2-2) ;
  \draw[->] (p2-2) -- (p1-2) ;
  \draw[->] (p1-2) -- (p0-2) ;
  \draw[->] (p0-2) -- (p0-3) ;
\end{tikzpicture}

\ruleSubsection{galgas3DeclarationsSyntax}{extension-abstract-setter}{19}

\begin{tikzpicture}
  \matrix[column sep=\ruleMatrixColumnSeparation, row sep=\ruleMatrixRowSeparation] {
    \node (P0start) [firstPoint] {}; & & \node (p4-2) [terminal] {abstract}; & \\
    & & \node (p3-2) [terminal] {setter}; & \\
    & & \node (p2-2) [terminal] {@type}; & \\
    & & \node (p1-2) [terminal] {identifier}; & \\
    & & \node (p0-2) [nonterminal] {\nonTerminalSymbol{formal\_parameter\_list}{11}}; & \node (p0-3) [lastPoint] {}; & \\
  };
  \draw[->] (P0start) -- (p4-2) ;
  \draw[->] (p4-2) -- (p3-2) ;
  \draw[->] (p3-2) -- (p2-2) ;
  \draw[->] (p2-2) -- (p1-2) ;
  \draw[->] (p1-2) -- (p0-2) ;
  \draw[->] (p0-2) -- (p0-3) ;
\end{tikzpicture}

\ruleSubsection{galgas3DeclarationsSyntax}{extension-getter}{23}

\begin{tikzpicture}
  \matrix[column sep=\ruleMatrixColumnSeparation, row sep=\ruleMatrixRowSeparation] {
    & & & & & & & & & \node (p1-9) [point] {}; & \\
    \node (P0start) [firstPoint] {}; & & \node (p0-2) [terminal] {getter}; & \node (p0-3) [terminal] {@type}; & \node (p0-4) [terminal] {identifier}; & \node (p0-5) [nonterminal] {\nonTerminalSymbol{formal\_input\_parameter\_list}{13}}; & \node (p0-6) [terminal] {->}; & \node (p0-7) [terminal] {@type}; & \node (p0-8) [point] {}; & \node (p0-9) [terminal] {identifier}; & \node (p0-10) [point] {}; & \node (p0-11) [terminal] {\{}; & \node (p0-12) [nonterminal] {\nonTerminalSymbol{semantic\_instruction\_list}{14}}; & \node (p0-13) [terminal] {\}}; & \node (p0-14) [lastPoint] {}; & \\
  };
  \draw[->] (P0start) -- (p0-2) ;
  \draw[->] (p0-2) -- (p0-3) ;
  \draw[->] (p0-3) -- (p0-4) ;
  \draw[->] (p0-4) -- (p0-5) ;
  \draw[->] (p0-5) -- (p0-6) ;
  \draw[->] (p0-6) -- (p0-7) ;
  \draw[->] (p0-7) -- (p0-9) ;
  \draw (p0-8) |- (p1-9) ;
  \draw (p0-9) -- (p0-10) ;
  \draw[->] (p1-9) -| (p0-10) ;
  \draw[->] (p0-10) -- (p0-11) ;
  \draw[->] (p0-11) -- (p0-12) ;
  \draw[->] (p0-12) -- (p0-13) ;
  \draw[->] (p0-13) -- (p0-14) ;
\end{tikzpicture}

\ruleSubsection{galgas3DeclarationsSyntax}{extension-method}{21}

\begin{tikzpicture}
  \matrix[column sep=\ruleMatrixColumnSeparation, row sep=\ruleMatrixRowSeparation] {
    \node (P0start) [firstPoint] {}; & & \node (p6-2) [terminal] {method}; & \\
    & & \node (p5-2) [terminal] {@type}; & \\
    & & \node (p4-2) [terminal] {identifier}; & \\
    & & \node (p3-2) [nonterminal] {\nonTerminalSymbol{formal\_parameter\_list}{11}}; & \\
    & & \node (p2-2) [terminal] {\{}; & \\
    & & \node (p1-2) [nonterminal] {\nonTerminalSymbol{semantic\_instruction\_list}{14}}; & \\
    & & \node (p0-2) [terminal] {\}}; & \node (p0-3) [lastPoint] {}; & \\
  };
  \draw[->] (P0start) -- (p6-2) ;
  \draw[->] (p6-2) -- (p5-2) ;
  \draw[->] (p5-2) -- (p4-2) ;
  \draw[->] (p4-2) -- (p3-2) ;
  \draw[->] (p3-2) -- (p2-2) ;
  \draw[->] (p2-2) -- (p1-2) ;
  \draw[->] (p1-2) -- (p0-2) ;
  \draw[->] (p0-2) -- (p0-3) ;
\end{tikzpicture}

\ruleSubsection{galgas3DeclarationsSyntax}{extension-setter}{21}

\begin{tikzpicture}
  \matrix[column sep=\ruleMatrixColumnSeparation, row sep=\ruleMatrixRowSeparation] {
    \node (P0start) [firstPoint] {}; & & \node (p6-2) [terminal] {setter}; & \\
    & & \node (p5-2) [terminal] {@type}; & \\
    & & \node (p4-2) [terminal] {identifier}; & \\
    & & \node (p3-2) [nonterminal] {\nonTerminalSymbol{formal\_parameter\_list}{11}}; & \\
    & & \node (p2-2) [terminal] {\{}; & \\
    & & \node (p1-2) [nonterminal] {\nonTerminalSymbol{semantic\_instruction\_list}{14}}; & \\
    & & \node (p0-2) [terminal] {\}}; & \node (p0-3) [lastPoint] {}; & \\
  };
  \draw[->] (P0start) -- (p6-2) ;
  \draw[->] (p6-2) -- (p5-2) ;
  \draw[->] (p5-2) -- (p4-2) ;
  \draw[->] (p4-2) -- (p3-2) ;
  \draw[->] (p3-2) -- (p2-2) ;
  \draw[->] (p2-2) -- (p1-2) ;
  \draw[->] (p1-2) -- (p0-2) ;
  \draw[->] (p0-2) -- (p0-3) ;
\end{tikzpicture}

\ruleSubsection{galgas3DeclarationsSyntax}{extension-overriding-abstract-getter}{20}

\begin{tikzpicture}
  \matrix[column sep=\ruleMatrixColumnSeparation, row sep=\ruleMatrixRowSeparation] {
    & & & & & & & & & & & \node (p1-11) [point] {}; & \\
    \node (P0start) [firstPoint] {}; & & \node (p0-2) [terminal] {override}; & \node (p0-3) [terminal] {abstract}; & \node (p0-4) [terminal] {getter}; & \node (p0-5) [terminal] {@type}; & \node (p0-6) [terminal] {identifier}; & \node (p0-7) [nonterminal] {\nonTerminalSymbol{formal\_input\_parameter\_list}{13}}; & \node (p0-8) [terminal] {->}; & \node (p0-9) [terminal] {@type}; & \node (p0-10) [point] {}; & \node (p0-11) [terminal] {identifier}; & \node (p0-12) [point] {}; & \node (p0-13) [lastPoint] {}; & \\
  };
  \draw[->] (P0start) -- (p0-2) ;
  \draw[->] (p0-2) -- (p0-3) ;
  \draw[->] (p0-3) -- (p0-4) ;
  \draw[->] (p0-4) -- (p0-5) ;
  \draw[->] (p0-5) -- (p0-6) ;
  \draw[->] (p0-6) -- (p0-7) ;
  \draw[->] (p0-7) -- (p0-8) ;
  \draw[->] (p0-8) -- (p0-9) ;
  \draw[->] (p0-9) -- (p0-11) ;
  \draw (p0-10) |- (p1-11) ;
  \draw (p0-11) -- (p0-12) ;
  \draw[->] (p1-11) -| (p0-12) ;
  \draw[->] (p0-12) -- (p0-13) ;
\end{tikzpicture}

\ruleSubsection{galgas3DeclarationsSyntax}{extension-overriding-abstract-method}{19}

\begin{tikzpicture}
  \matrix[column sep=\ruleMatrixColumnSeparation, row sep=\ruleMatrixRowSeparation] {
    \node (P0start) [firstPoint] {}; & & \node (p5-2) [terminal] {override}; & \\
    & & \node (p4-2) [terminal] {abstract}; & \\
    & & \node (p3-2) [terminal] {method}; & \\
    & & \node (p2-2) [terminal] {@type}; & \\
    & & \node (p1-2) [terminal] {identifier}; & \\
    & & \node (p0-2) [nonterminal] {\nonTerminalSymbol{formal\_parameter\_list}{11}}; & \node (p0-3) [lastPoint] {}; & \\
  };
  \draw[->] (P0start) -- (p5-2) ;
  \draw[->] (p5-2) -- (p4-2) ;
  \draw[->] (p4-2) -- (p3-2) ;
  \draw[->] (p3-2) -- (p2-2) ;
  \draw[->] (p2-2) -- (p1-2) ;
  \draw[->] (p1-2) -- (p0-2) ;
  \draw[->] (p0-2) -- (p0-3) ;
\end{tikzpicture}

\ruleSubsection{galgas3DeclarationsSyntax}{extension-overriding-abstract-setter}{19}

\begin{tikzpicture}
  \matrix[column sep=\ruleMatrixColumnSeparation, row sep=\ruleMatrixRowSeparation] {
    \node (P0start) [firstPoint] {}; & & \node (p5-2) [terminal] {override}; & \\
    & & \node (p4-2) [terminal] {abstract}; & \\
    & & \node (p3-2) [terminal] {setter}; & \\
    & & \node (p2-2) [terminal] {@type}; & \\
    & & \node (p1-2) [terminal] {identifier}; & \\
    & & \node (p0-2) [nonterminal] {\nonTerminalSymbol{formal\_parameter\_list}{11}}; & \node (p0-3) [lastPoint] {}; & \\
  };
  \draw[->] (P0start) -- (p5-2) ;
  \draw[->] (p5-2) -- (p4-2) ;
  \draw[->] (p4-2) -- (p3-2) ;
  \draw[->] (p3-2) -- (p2-2) ;
  \draw[->] (p2-2) -- (p1-2) ;
  \draw[->] (p1-2) -- (p0-2) ;
  \draw[->] (p0-2) -- (p0-3) ;
\end{tikzpicture}

\ruleSubsection{galgas3DeclarationsSyntax}{extension-overriding-getter}{23}

\begin{tikzpicture}
  \matrix[column sep=\ruleMatrixColumnSeparation, row sep=\ruleMatrixRowSeparation] {
    & & & & & & & & & & \node (p1-10) [point] {}; & \\
    \node (P0start) [firstPoint] {}; & & \node (p0-2) [terminal] {override}; & \node (p0-3) [terminal] {getter}; & \node (p0-4) [terminal] {@type}; & \node (p0-5) [terminal] {identifier}; & \node (p0-6) [nonterminal] {\nonTerminalSymbol{formal\_input\_parameter\_list}{13}}; & \node (p0-7) [terminal] {->}; & \node (p0-8) [terminal] {@type}; & \node (p0-9) [point] {}; & \node (p0-10) [terminal] {identifier}; & \node (p0-11) [point] {}; & \node (p0-12) [terminal] {\{}; & \node (p0-13) [nonterminal] {\nonTerminalSymbol{semantic\_instruction\_list}{14}}; & \node (p0-14) [terminal] {\}}; & \node (p0-15) [lastPoint] {}; & \\
  };
  \draw[->] (P0start) -- (p0-2) ;
  \draw[->] (p0-2) -- (p0-3) ;
  \draw[->] (p0-3) -- (p0-4) ;
  \draw[->] (p0-4) -- (p0-5) ;
  \draw[->] (p0-5) -- (p0-6) ;
  \draw[->] (p0-6) -- (p0-7) ;
  \draw[->] (p0-7) -- (p0-8) ;
  \draw[->] (p0-8) -- (p0-10) ;
  \draw (p0-9) |- (p1-10) ;
  \draw (p0-10) -- (p0-11) ;
  \draw[->] (p1-10) -| (p0-11) ;
  \draw[->] (p0-11) -- (p0-12) ;
  \draw[->] (p0-12) -- (p0-13) ;
  \draw[->] (p0-13) -- (p0-14) ;
  \draw[->] (p0-14) -- (p0-15) ;
\end{tikzpicture}

\ruleSubsection{galgas3DeclarationsSyntax}{extension-overriding-method}{21}

\begin{tikzpicture}
  \matrix[column sep=\ruleMatrixColumnSeparation, row sep=\ruleMatrixRowSeparation] {
    \node (P0start) [firstPoint] {}; & & \node (p7-2) [terminal] {override}; & \\
    & & \node (p6-2) [terminal] {method}; & \\
    & & \node (p5-2) [terminal] {@type}; & \\
    & & \node (p4-2) [terminal] {identifier}; & \\
    & & \node (p3-2) [nonterminal] {\nonTerminalSymbol{formal\_parameter\_list}{11}}; & \\
    & & \node (p2-2) [terminal] {\{}; & \\
    & & \node (p1-2) [nonterminal] {\nonTerminalSymbol{semantic\_instruction\_list}{14}}; & \\
    & & \node (p0-2) [terminal] {\}}; & \node (p0-3) [lastPoint] {}; & \\
  };
  \draw[->] (P0start) -- (p7-2) ;
  \draw[->] (p7-2) -- (p6-2) ;
  \draw[->] (p6-2) -- (p5-2) ;
  \draw[->] (p5-2) -- (p4-2) ;
  \draw[->] (p4-2) -- (p3-2) ;
  \draw[->] (p3-2) -- (p2-2) ;
  \draw[->] (p2-2) -- (p1-2) ;
  \draw[->] (p1-2) -- (p0-2) ;
  \draw[->] (p0-2) -- (p0-3) ;
\end{tikzpicture}

\ruleSubsection{galgas3DeclarationsSyntax}{extension-overriding-setter}{21}

\begin{tikzpicture}
  \matrix[column sep=\ruleMatrixColumnSeparation, row sep=\ruleMatrixRowSeparation] {
    \node (P0start) [firstPoint] {}; & & \node (p7-2) [terminal] {override}; & \\
    & & \node (p6-2) [terminal] {setter}; & \\
    & & \node (p5-2) [terminal] {@type}; & \\
    & & \node (p4-2) [terminal] {identifier}; & \\
    & & \node (p3-2) [nonterminal] {\nonTerminalSymbol{formal\_parameter\_list}{11}}; & \\
    & & \node (p2-2) [terminal] {\{}; & \\
    & & \node (p1-2) [nonterminal] {\nonTerminalSymbol{semantic\_instruction\_list}{14}}; & \\
    & & \node (p0-2) [terminal] {\}}; & \node (p0-3) [lastPoint] {}; & \\
  };
  \draw[->] (P0start) -- (p7-2) ;
  \draw[->] (p7-2) -- (p6-2) ;
  \draw[->] (p6-2) -- (p5-2) ;
  \draw[->] (p5-2) -- (p4-2) ;
  \draw[->] (p4-2) -- (p3-2) ;
  \draw[->] (p3-2) -- (p2-2) ;
  \draw[->] (p2-2) -- (p1-2) ;
  \draw[->] (p1-2) -- (p0-2) ;
  \draw[->] (p0-2) -- (p0-3) ;
\end{tikzpicture}

\ruleSubsection{galgas3LexiqueComponentSyntax}{galgas3LexiqueComponentSyntax}{20}

\begin{tikzpicture}
  \matrix[column sep=\ruleMatrixColumnSeparation, row sep=\ruleMatrixRowSeparation] {
    & & & & & & & & & & & & & & & & & \node (p14-17) [point] {}; & \\
    & & & & & & & & & & & & & & & & \node (p13-16) [nonterminal] {\nonTerminalSymbol{lexical\_function\_declaration}{73}}; & \\
    & & & & & & & & & & & & & & & & \node (p12-16) [nonterminal] {\nonTerminalSymbol{lexical\_indexing\_declaration}{53}}; & \\
    & & & & & & & & & & & & & & & & \node (p11-16) [nonterminal] {\nonTerminalSymbol{lexical\_message\_declaration}{56}}; & \\
    & & & & & & & & & & & & & & & & \node (p10-16) [nonterminal] {\nonTerminalSymbol{lexical\_implicit\_rule}{57}}; & \\
    & & & & & & & & & & & & & & & & \node (p9-16) [nonterminal] {\nonTerminalSymbol{lexical\_explicit\_rule}{58}}; & \\
    & & & & & & & & & & & & & & & & \node (p8-16) [nonterminal] {\nonTerminalSymbol{lexical\_list\_declaration}{66}}; & \\
    & & & & & & & & & & & & & & & & \node (p7-16) [nonterminal] {\nonTerminalSymbol{terminal\_declaration}{69}}; & \\
    & & & & & & & & & & & & & & & & \node (p6-16) [nonterminal] {\nonTerminalSymbol{style\_declaration}{70}}; & \\
    & & & & & & & & & & & & & & & & \node (p5-16) [nonterminal] {\nonTerminalSymbol{lexical\_attribute\_declaration}{68}}; & \\
    & & & & & & & & & & & & & & & & \node (p4-16) [nonterminal] {\nonTerminalSymbol{template\_replacement}{55}}; & \\
    & & & & & & & & & & & & & & & & \node (p3-16) [nonterminal] {\nonTerminalSymbol{template\_delimitor}{54}}; & \\
    & & & & & & & & & & & & & & & & \node (p2-16) [nonterminal] {\nonTerminalSymbol{extern\_function\_declaration}{72}}; & \\
    & & & \node (p1-3) [point] {}; & & & & & \node (p1-8) [terminal] {indexing}; & \node (p1-9) [terminal] {in}; & \node (p1-10) [terminal] {"string"}; & & & & & & \node (p1-16) [nonterminal] {\nonTerminalSymbol{extern\_routine\_declaration}{71}}; & \\
    \node (P0start) [firstPoint] {}; & & \node (p0-2) [point] {}; & \node (p0-3) [terminal] {template}; & \node (p0-4) [point] {}; & \node (p0-5) [terminal] {lexique}; & \node (p0-6) [terminal] {identifier}; & \node (p0-7) [point] {}; & \node (p0-8) [point] {}; & & & \node (p0-11) [point] {}; & \node (p0-12) [terminal] {\{}; & \node (p0-13) [point] {}; & \node (p0-14) [point] {}; & \node (p0-15) [point] {}; & & & \node (p0-18) [terminal] {\}}; & \node (p0-19) [lastPoint] {}; & \\
  };
  \draw[->] (P0start) -- (p0-3) ;
  \draw (p0-2) |- (p1-3) ;
  \draw (p0-3) -- (p0-4) ;
  \draw[->] (p1-3) -| (p0-4) ;
  \draw[->] (p0-4) -- (p0-5) ;
  \draw[->] (p0-5) -- (p0-6) ;
  \draw (p0-6) -- (p0-8) ;
  \draw[->] (p0-7) |- (p1-8) ;
  \draw[->] (p1-8) -- (p1-9) ;
  \draw[->] (p1-9) -- (p1-10) ;
  \draw (p0-8) -- (p0-11) ;
  \draw[->] (p1-10) -| (p0-11) ;
  \draw[->] (p0-11) -- (p0-12) ;
  \draw (p0-12) -- (p0-14) ;
  \draw[->] (p0-15) |- (p1-16) ;
  \draw[->] (p0-15) |- (p2-16) ;
  \draw[->] (p0-15) |- (p3-16) ;
  \draw[->] (p0-15) |- (p4-16) ;
  \draw[->] (p0-15) |- (p5-16) ;
  \draw[->] (p0-15) |- (p6-16) ;
  \draw[->] (p0-15) |- (p7-16) ;
  \draw[->] (p0-15) |- (p8-16) ;
  \draw[->] (p0-15) |- (p9-16) ;
  \draw[->] (p0-15) |- (p10-16) ;
  \draw[->] (p0-15) |- (p11-16) ;
  \draw[->] (p0-15) |- (p12-16) ;
  \draw[->] (p0-15) |- (p13-16) ;
  \draw[->] (p14-17) -| (p0-13) ;
  \draw[->] (p1-16) -| (p14-17) ;
  \draw[->] (p2-16) -| (p14-17) ;
  \draw[->] (p3-16) -| (p14-17) ;
  \draw[->] (p4-16) -| (p14-17) ;
  \draw[->] (p5-16) -| (p14-17) ;
  \draw[->] (p6-16) -| (p14-17) ;
  \draw[->] (p7-16) -| (p14-17) ;
  \draw[->] (p8-16) -| (p14-17) ;
  \draw[->] (p9-16) -| (p14-17) ;
  \draw[->] (p10-16) -| (p14-17) ;
  \draw[->] (p11-16) -| (p14-17) ;
  \draw[->] (p12-16) -| (p14-17) ;
  \draw[->] (p13-16) -| (p14-17) ;
  \draw[->] (p0-14) -- (p0-18) ;
  \draw[->] (p0-18) -- (p0-19) ;
\end{tikzpicture}

\ruleSubsection{galgas3OptionComponentSyntax}{optionCompilation}{37}

\begin{tikzpicture}
  \matrix[column sep=\ruleMatrixColumnSeparation, row sep=\ruleMatrixRowSeparation] {
    & & & & & & & & & \node (p2-9) [point] {}; & \\
    & & & & & & & & \node (p1-8) [nonterminal] {\nonTerminalSymbol{option\_declaration}{77}}; & \\
    \node (P0start) [firstPoint] {}; & & \node (p0-2) [terminal] {option}; & \node (p0-3) [terminal] {identifier}; & \node (p0-4) [terminal] {\{}; & \node (p0-5) [point] {}; & \node (p0-6) [point] {}; & \node (p0-7) [point] {}; & & & \node (p0-10) [terminal] {\}}; & \node (p0-11) [lastPoint] {}; & \\
  };
  \draw[->] (P0start) -- (p0-2) ;
  \draw[->] (p0-2) -- (p0-3) ;
  \draw[->] (p0-3) -- (p0-4) ;
  \draw (p0-4) -- (p0-6) ;
  \draw[->] (p0-7) |- (p1-8) ;
  \draw[->] (p2-9) -| (p0-5) ;
  \draw[->] (p1-8) -| (p2-9) ;
  \draw[->] (p0-6) -- (p0-10) ;
  \draw[->] (p0-10) -- (p0-11) ;
\end{tikzpicture}

\ruleSubsection{galgas3GuiComponentSyntax}{guiCompilation}{92}

\begin{tikzpicture}
  \matrix[column sep=\ruleMatrixColumnSeparation, row sep=\ruleMatrixRowSeparation] {
    & & & & & & & & & \node (p4-9) [point] {}; & \\
    & & & & & & & & \node (p3-8) [nonterminal] {\nonTerminalSymbol{gui\_with\_lexique\_declaration}{78}}; & \\
    & & & & & & & & \node (p2-8) [nonterminal] {\nonTerminalSymbol{gui\_with\_option\_declaration}{79}}; & \\
    & & & & & & & & \node (p1-8) [nonterminal] {\nonTerminalSymbol{gui\_attributes}{80}}; & \\
    \node (P0start) [firstPoint] {}; & & \node (p0-2) [terminal] {gui}; & \node (p0-3) [terminal] {identifier}; & \node (p0-4) [terminal] {\{}; & \node (p0-5) [point] {}; & \node (p0-6) [point] {}; & \node (p0-7) [point] {}; & & & \node (p0-10) [terminal] {\}}; & \node (p0-11) [lastPoint] {}; & \\
  };
  \draw[->] (P0start) -- (p0-2) ;
  \draw[->] (p0-2) -- (p0-3) ;
  \draw[->] (p0-3) -- (p0-4) ;
  \draw (p0-4) -- (p0-6) ;
  \draw[->] (p0-7) |- (p1-8) ;
  \draw[->] (p0-7) |- (p2-8) ;
  \draw[->] (p0-7) |- (p3-8) ;
  \draw[->] (p4-9) -| (p0-5) ;
  \draw[->] (p1-8) -| (p4-9) ;
  \draw[->] (p2-8) -| (p4-9) ;
  \draw[->] (p3-8) -| (p4-9) ;
  \draw[->] (p0-6) -- (p0-10) ;
  \draw[->] (p0-10) -- (p0-11) ;
\end{tikzpicture}

\ruleSubsection{galgas3SyntaxComponentSyntax}{galgas3SyntaxComponentSyntax}{40}

\begin{tikzpicture}
  \matrix[column sep=\ruleMatrixColumnSeparation, row sep=\ruleMatrixRowSeparation] {
    & & & & & & & & & & & & & & & & & \node (p3-17) [point] {}; & \\
    & & & & & & & & & & & & & & & & \node (p2-16) [nonterminal] {\nonTerminalSymbol{syntax\_rule\_declaration}{83}}; & \\
    & & & & & \node (p1-5) [terminal] {(}; & \node (p1-6) [terminal] {identifier}; & \node (p1-7) [terminal] {)}; & & & \node (p1-10) [terminal] {\verb=%=attribute}; & & & & & & \node (p1-16) [nonterminal] {\nonTerminalSymbol{nonterminal\_declaration}{81}}; & \\
    \node (P0start) [firstPoint] {}; & & \node (p0-2) [terminal] {syntax}; & \node (p0-3) [terminal] {identifier}; & \node (p0-4) [point] {}; & \node (p0-5) [point] {}; & & & \node (p0-8) [point] {}; & \node (p0-9) [point] {}; & \node (p0-10) [point] {}; & \node (p0-11) [point] {}; & \node (p0-12) [terminal] {\{}; & \node (p0-13) [point] {}; & \node (p0-14) [point] {}; & \node (p0-15) [point] {}; & & & \node (p0-18) [terminal] {\}}; & \node (p0-19) [lastPoint] {}; & \\
  };
  \draw[->] (P0start) -- (p0-2) ;
  \draw[->] (p0-2) -- (p0-3) ;
  \draw (p0-3) -- (p0-5) ;
  \draw[->] (p0-4) |- (p1-5) ;
  \draw[->] (p1-5) -- (p1-6) ;
  \draw[->] (p1-6) -- (p1-7) ;
  \draw (p0-5) -- (p0-8) ;
  \draw[->] (p1-7) -| (p0-8) ;
  \draw (p0-8) -- (p0-10) ;
  \draw[->] (p0-9) |- (p1-10) ;
  \draw (p0-10) -- (p0-11) ;
  \draw[->] (p1-10) -| (p0-11) ;
  \draw[->] (p0-11) -- (p0-12) ;
  \draw (p0-12) -- (p0-14) ;
  \draw[->] (p0-15) |- (p1-16) ;
  \draw[->] (p0-15) |- (p2-16) ;
  \draw[->] (p3-17) -| (p0-13) ;
  \draw[->] (p1-16) -| (p3-17) ;
  \draw[->] (p2-16) -| (p3-17) ;
  \draw[->] (p0-14) -- (p0-18) ;
  \draw[->] (p0-18) -- (p0-19) ;
\end{tikzpicture}

\ruleSubsection{galgas3SyntaxComponentSyntax}{galgas3SyntaxComponentSyntax}{93}

\begin{tikzpicture}
  \matrix[column sep=\ruleMatrixColumnSeparation, row sep=\ruleMatrixRowSeparation] {
    & & & & & & & & & & \node (p3-10) [point] {}; & \\
    & & & & & & & & & \node (p2-9) [nonterminal] {\nonTerminalSymbol{syntax\_rule\_declaration}{83}}; & \\
    & & & & & & & & & \node (p1-9) [nonterminal] {\nonTerminalSymbol{nonterminal\_declaration}{81}}; & \\
    \node (P0start) [firstPoint] {}; & & \node (p0-2) [terminal] {syntax}; & \node (p0-3) [terminal] {extension}; & \node (p0-4) [terminal] {identifier}; & \node (p0-5) [terminal] {\{}; & \node (p0-6) [point] {}; & \node (p0-7) [point] {}; & \node (p0-8) [point] {}; & & & \node (p0-11) [terminal] {\}}; & \node (p0-12) [lastPoint] {}; & \\
  };
  \draw[->] (P0start) -- (p0-2) ;
  \draw[->] (p0-2) -- (p0-3) ;
  \draw[->] (p0-3) -- (p0-4) ;
  \draw[->] (p0-4) -- (p0-5) ;
  \draw (p0-5) -- (p0-7) ;
  \draw[->] (p0-8) |- (p1-9) ;
  \draw[->] (p0-8) |- (p2-9) ;
  \draw[->] (p3-10) -| (p0-6) ;
  \draw[->] (p1-9) -| (p3-10) ;
  \draw[->] (p2-9) -| (p3-10) ;
  \draw[->] (p0-7) -- (p0-11) ;
  \draw[->] (p0-11) -- (p0-12) ;
\end{tikzpicture}

\ruleSubsection{galgas3GrammarComponentSyntax}{galgas3GrammarComponentSyntax}{24}

\begin{tikzpicture}
  \matrix[column sep=\ruleMatrixColumnSeparation, row sep=\ruleMatrixRowSeparation] {
    & & & & & & & & & & & & & & & & \node (p2-16) [point] {}; & & & & & & & & & \node (p2-25) [point] {}; & & & & & & \node (p2-31) [point] {}; & \\
    & & & \node (p1-3) [terminal] {indexing}; & & & & & & \node (p1-9) [terminal] {\verb=%=attribute}; & & & & & & & & & & & & & \node (p1-22) [terminal] {label}; & \node (p1-23) [terminal] {identifier}; & \node (p1-24) [nonterminal] {\nonTerminalSymbol{grammar\_start\_symbol\_label}{88}}; & & & & & \node (p1-29) [terminal] {unused}; & \node (p1-30) [terminal] {<non\_terminal>}; & \\
    \node (P0start) [firstPoint] {}; & & \node (p0-2) [point] {}; & \node (p0-3) [point] {}; & \node (p0-4) [point] {}; & \node (p0-5) [terminal] {grammar}; & \node (p0-6) [terminal] {identifier}; & \node (p0-7) [terminal] {"string"}; & \node (p0-8) [point] {}; & \node (p0-9) [point] {}; & \node (p0-10) [point] {}; & \node (p0-11) [terminal] {\{}; & \node (p0-12) [point] {}; & \node (p0-13) [terminal] {syntax}; & \node (p0-14) [terminal] {identifier}; & \node (p0-15) [point] {}; & & \node (p0-17) [terminal] {<non\_terminal>}; & \node (p0-18) [nonterminal] {\nonTerminalSymbol{grammar\_start\_symbol\_label}{88}}; & \node (p0-19) [point] {}; & \node (p0-20) [point] {}; & \node (p0-21) [point] {}; & & & & & \node (p0-26) [point] {}; & \node (p0-27) [point] {}; & \node (p0-28) [point] {}; & & & & \node (p0-32) [terminal] {\}}; & \node (p0-33) [lastPoint] {}; & \\
  };
  \draw (P0start) -- (p0-3) ;
  \draw[->] (p0-2) |- (p1-3) ;
  \draw (p0-3) -- (p0-4) ;
  \draw[->] (p1-3) -| (p0-4) ;
  \draw[->] (p0-4) -- (p0-5) ;
  \draw[->] (p0-5) -- (p0-6) ;
  \draw[->] (p0-6) -- (p0-7) ;
  \draw (p0-7) -- (p0-9) ;
  \draw[->] (p0-8) |- (p1-9) ;
  \draw (p0-9) -- (p0-10) ;
  \draw[->] (p1-9) -| (p0-10) ;
  \draw[->] (p0-10) -- (p0-11) ;
  \draw[->] (p0-11) -- (p0-13) ;
  \draw[->] (p0-13) -- (p0-14) ;
  \draw[->] (p2-16) -| (p0-12) ;
  \draw[->] (p0-15) -| (p2-16) ;
  \draw[->] (p0-14) -- (p0-17) ;
  \draw[->] (p0-17) -- (p0-18) ;
  \draw (p0-18) -- (p0-20) ;
  \draw[->] (p0-21) |- (p1-22) ;
  \draw[->] (p1-22) -- (p1-23) ;
  \draw[->] (p1-23) -- (p1-24) ;
  \draw[->] (p2-25) -| (p0-19) ;
  \draw[->] (p1-24) -| (p2-25) ;
  \draw (p0-20) -- (p0-27) ;
  \draw[->] (p0-28) |- (p1-29) ;
  \draw[->] (p1-29) -- (p1-30) ;
  \draw[->] (p2-31) -| (p0-26) ;
  \draw[->] (p1-30) -| (p2-31) ;
  \draw[->] (p0-27) -- (p0-32) ;
  \draw[->] (p0-32) -- (p0-33) ;
\end{tikzpicture}

\ruleSubsection{galgas3ProgramDeclarations}{galgas3ProgramDeclarations}{30}

\begin{tikzpicture}
  \matrix[column sep=\ruleMatrixColumnSeparation, row sep=\ruleMatrixRowSeparation] {
    \node (P0start) [firstPoint] {}; & & \node (p0-2) [terminal] {before}; & \node (p0-3) [terminal] {\{}; & \node (p0-4) [nonterminal] {\nonTerminalSymbol{semantic\_instruction\_list}{14}}; & \node (p0-5) [terminal] {\}}; & \node (p0-6) [lastPoint] {}; & \\
  };
  \draw[->] (P0start) -- (p0-2) ;
  \draw[->] (p0-2) -- (p0-3) ;
  \draw[->] (p0-3) -- (p0-4) ;
  \draw[->] (p0-4) -- (p0-5) ;
  \draw[->] (p0-5) -- (p0-6) ;
\end{tikzpicture}

\ruleSubsection{galgas3ProgramDeclarations}{galgas3ProgramDeclarations}{48}

\begin{tikzpicture}
  \matrix[column sep=\ruleMatrixColumnSeparation, row sep=\ruleMatrixRowSeparation] {
    \node (P0start) [firstPoint] {}; & & \node (p0-2) [terminal] {after}; & \node (p0-3) [terminal] {\{}; & \node (p0-4) [nonterminal] {\nonTerminalSymbol{semantic\_instruction\_list}{14}}; & \node (p0-5) [terminal] {\}}; & \node (p0-6) [lastPoint] {}; & \\
  };
  \draw[->] (P0start) -- (p0-2) ;
  \draw[->] (p0-2) -- (p0-3) ;
  \draw[->] (p0-3) -- (p0-4) ;
  \draw[->] (p0-4) -- (p0-5) ;
  \draw[->] (p0-5) -- (p0-6) ;
\end{tikzpicture}

\ruleSubsection{galgas3ProgramDeclarations}{galgas3ProgramDeclarations}{66}

\begin{tikzpicture}
  \matrix[column sep=\ruleMatrixColumnSeparation, row sep=\ruleMatrixRowSeparation] {
    & & & & & & & & \node (p1-8) [terminal] {grammar}; & \node (p1-9) [terminal] {identifier}; & & & & \node (p1-13) [terminal] {@type}; & & & \node (p1-16) [terminal] {unused}; & \\
    \node (P0start) [firstPoint] {}; & & \node (p0-2) [terminal] {case}; & \node (p0-3) [terminal] {.}; & \node (p0-4) [terminal] {"string"}; & \node (p0-5) [terminal] {message}; & \node (p0-6) [terminal] {"string"}; & \node (p0-7) [point] {}; & \node (p0-8) [point] {}; & & \node (p0-10) [point] {}; & \node (p0-11) [terminal] {?}; & \node (p0-12) [point] {}; & \node (p0-13) [point] {}; & \node (p0-14) [point] {}; & \node (p0-15) [point] {}; & \node (p0-16) [point] {}; & \node (p0-17) [point] {}; & \node (p0-18) [terminal] {identifier}; & \node (p0-19) [terminal] {\{}; & \node (p0-20) [nonterminal] {\nonTerminalSymbol{semantic\_instruction\_list}{14}}; & \node (p0-21) [terminal] {\}}; & \node (p0-22) [lastPoint] {}; & \\
  };
  \draw[->] (P0start) -- (p0-2) ;
  \draw[->] (p0-2) -- (p0-3) ;
  \draw[->] (p0-3) -- (p0-4) ;
  \draw[->] (p0-4) -- (p0-5) ;
  \draw[->] (p0-5) -- (p0-6) ;
  \draw (p0-6) -- (p0-8) ;
  \draw[->] (p0-7) |- (p1-8) ;
  \draw[->] (p1-8) -- (p1-9) ;
  \draw (p0-8) -- (p0-10) ;
  \draw[->] (p1-9) -| (p0-10) ;
  \draw[->] (p0-10) -- (p0-11) ;
  \draw (p0-11) -- (p0-13) ;
  \draw[->] (p0-12) |- (p1-13) ;
  \draw (p0-13) -- (p0-14) ;
  \draw[->] (p1-13) -| (p0-14) ;
  \draw (p0-14) -- (p0-16) ;
  \draw[->] (p0-15) |- (p1-16) ;
  \draw (p0-16) -- (p0-17) ;
  \draw[->] (p1-16) -| (p0-17) ;
  \draw[->] (p0-17) -- (p0-18) ;
  \draw[->] (p0-18) -- (p0-19) ;
  \draw[->] (p0-19) -- (p0-20) ;
  \draw[->] (p0-20) -- (p0-21) ;
  \draw[->] (p0-21) -- (p0-22) ;
\end{tikzpicture}

\nonTerminalSection{declaration\_with\_private}{16}

\ruleSubsection{galgas3InstructionsSyntax}{galgas3InstructionsSyntax}{91}

\begin{tikzpicture}
  \matrix[column sep=\ruleMatrixColumnSeparation, row sep=\ruleMatrixRowSeparation] {
    \node (P0start) [firstPoint] {}; & & \node (p5-2) [terminal] {proc}; & \\
    & & \node (p4-2) [terminal] {identifier}; & \\
    & & \node (p3-2) [nonterminal] {\nonTerminalSymbol{formal\_parameter\_list}{11}}; & \\
    & & \node (p2-2) [terminal] {\{}; & \\
    & & \node (p1-2) [nonterminal] {\nonTerminalSymbol{semantic\_instruction\_list}{14}}; & \\
    & & \node (p0-2) [terminal] {\}}; & \node (p0-3) [lastPoint] {}; & \\
  };
  \draw[->] (P0start) -- (p5-2) ;
  \draw[->] (p5-2) -- (p4-2) ;
  \draw[->] (p4-2) -- (p3-2) ;
  \draw[->] (p3-2) -- (p2-2) ;
  \draw[->] (p2-2) -- (p1-2) ;
  \draw[->] (p1-2) -- (p0-2) ;
  \draw[->] (p0-2) -- (p0-3) ;
\end{tikzpicture}

\ruleSubsection{galgas3InstructionsSyntax}{galgas3InstructionsSyntax}{122}

\begin{tikzpicture}
  \matrix[column sep=\ruleMatrixColumnSeparation, row sep=\ruleMatrixRowSeparation] {
    & & & & & & & \node (p2-7) [point] {}; & \\
    & & & & & & \node (p1-6) [terminal] {\verb=%=attribute}; & & & & & & & \node (p1-13) [point] {}; & \\
    \node (P0start) [firstPoint] {}; & & \node (p0-2) [terminal] {func}; & \node (p0-3) [point] {}; & \node (p0-4) [point] {}; & \node (p0-5) [point] {}; & & & \node (p0-8) [terminal] {identifier}; & \node (p0-9) [nonterminal] {\nonTerminalSymbol{formal\_input\_parameter\_list}{13}}; & \node (p0-10) [terminal] {->}; & \node (p0-11) [terminal] {@type}; & \node (p0-12) [point] {}; & \node (p0-13) [terminal] {identifier}; & \node (p0-14) [point] {}; & \node (p0-15) [terminal] {\{}; & \node (p0-16) [nonterminal] {\nonTerminalSymbol{semantic\_instruction\_list}{14}}; & \node (p0-17) [terminal] {\}}; & \node (p0-18) [lastPoint] {}; & \\
  };
  \draw[->] (P0start) -- (p0-2) ;
  \draw (p0-2) -- (p0-4) ;
  \draw[->] (p0-5) |- (p1-6) ;
  \draw[->] (p2-7) -| (p0-3) ;
  \draw[->] (p1-6) -| (p2-7) ;
  \draw[->] (p0-4) -- (p0-8) ;
  \draw[->] (p0-8) -- (p0-9) ;
  \draw[->] (p0-9) -- (p0-10) ;
  \draw[->] (p0-10) -- (p0-11) ;
  \draw[->] (p0-11) -- (p0-13) ;
  \draw (p0-12) |- (p1-13) ;
  \draw (p0-13) -- (p0-14) ;
  \draw[->] (p1-13) -| (p0-14) ;
  \draw[->] (p0-14) -- (p0-15) ;
  \draw[->] (p0-15) -- (p0-16) ;
  \draw[->] (p0-16) -- (p0-17) ;
  \draw[->] (p0-17) -- (p0-18) ;
\end{tikzpicture}

\ruleSubsection{galgas3DeclarationsSyntax}{galgas3DeclarationsSyntax}{83}

\begin{tikzpicture}
  \matrix[column sep=\ruleMatrixColumnSeparation, row sep=\ruleMatrixRowSeparation] {
    \node (P0start) [firstPoint] {}; & & \node (p6-2) [terminal] {filewrapper}; & \\
    & & \node (p5-2) [terminal] {identifier}; & \\
    & & \node (p4-2) [terminal] {in}; & \\
    & & \node (p3-2) [terminal] {"string"}; & \\
    & & \node (p2-2) [nonterminal] {\nonTerminalSymbol{filewrapper\_text\_files}{34}}; & \\
    & & \node (p1-2) [nonterminal] {\nonTerminalSymbol{filewrapper\_binary\_files}{35}}; & \\
    & & \node (p0-2) [nonterminal] {\nonTerminalSymbol{filewrapper\_templates}{36}}; & \node (p0-3) [lastPoint] {}; & \\
  };
  \draw[->] (P0start) -- (p6-2) ;
  \draw[->] (p6-2) -- (p5-2) ;
  \draw[->] (p5-2) -- (p4-2) ;
  \draw[->] (p4-2) -- (p3-2) ;
  \draw[->] (p3-2) -- (p2-2) ;
  \draw[->] (p2-2) -- (p1-2) ;
  \draw[->] (p1-2) -- (p0-2) ;
  \draw[->] (p0-2) -- (p0-3) ;
\end{tikzpicture}

\nonTerminalSection{expression}{1}

\ruleSubsection{galgas3ExpressionSyntax}{galgas3ExpressionSyntax}{42}

\begin{tikzpicture}
  \matrix[column sep=\ruleMatrixColumnSeparation, row sep=\ruleMatrixRowSeparation] {
    & & & & \node (p4-4) [terminal] {as}; & \node (p4-5) [terminal] {@type}; & \\
    & & & & & & \node (p3-6) [terminal] {>}; & \\
    & & & & & & \node (p2-6) [terminal] {>=}; & \\
    & & & & \node (p1-4) [terminal] {is}; & \node (p1-5) [point] {}; & \node (p1-6) [terminal] {==}; & \node (p1-7) [point] {}; & \node (p1-8) [terminal] {@type}; & \\
    \node (P0start) [firstPoint] {}; & & \node (p0-2) [nonterminal] {\nonTerminalSymbol{casted\_expression}{2}}; & \node (p0-3) [point] {}; & \node (p0-4) [point] {}; & & & & & \node (p0-9) [point] {}; & \node (p0-10) [lastPoint] {}; & \\
  };
  \draw[->] (P0start) -- (p0-2) ;
  \draw (p0-2) -- (p0-4) ;
  \draw[->] (p0-3) |- (p1-4) ;
  \draw[->] (p1-4) -- (p1-6) ;
  \draw[->] (p1-5) |- (p2-6) ;
  \draw[->] (p1-5) |- (p3-6) ;
  \draw (p1-6) -- (p1-7) ;
  \draw[->] (p2-6) -| (p1-7) ;
  \draw[->] (p3-6) -| (p1-7) ;
  \draw[->] (p1-7) -- (p1-8) ;
  \draw[->] (p0-3) |- (p4-4) ;
  \draw[->] (p4-4) -- (p4-5) ;
  \draw (p0-4) -- (p0-9) ;
  \draw[->] (p1-8) -| (p0-9) ;
  \draw[->] (p4-5) -| (p0-9) ;
  \draw[->] (p0-9) -- (p0-10) ;
\end{tikzpicture}

\nonTerminalSection{extern\_function\_declaration}{72}

\ruleSubsection{galgas3LexiqueComponentSyntax}{galgas3LexiqueComponentSyntax}{636}

\begin{tikzpicture}
  \matrix[column sep=\ruleMatrixColumnSeparation, row sep=\ruleMatrixRowSeparation] {
    & & & & & & & & & & & \node (p2-11) [point] {}; & \\
    & & & & & & & & \node (p1-8) [terminal] {?}; & \node (p1-9) [terminal] {@type}; & \node (p1-10) [terminal] {identifier}; & \\
    \node (P0start) [firstPoint] {}; & & \node (p0-2) [terminal] {extern}; & \node (p0-3) [terminal] {func}; & \node (p0-4) [terminal] {identifier}; & \node (p0-5) [point] {}; & \node (p0-6) [point] {}; & \node (p0-7) [point] {}; & & & & & \node (p0-12) [terminal] {->}; & \node (p0-13) [terminal] {@type}; & \node (p0-14) [lastPoint] {}; & \\
  };
  \draw[->] (P0start) -- (p0-2) ;
  \draw[->] (p0-2) -- (p0-3) ;
  \draw[->] (p0-3) -- (p0-4) ;
  \draw (p0-4) -- (p0-6) ;
  \draw[->] (p0-7) |- (p1-8) ;
  \draw[->] (p1-8) -- (p1-9) ;
  \draw[->] (p1-9) -- (p1-10) ;
  \draw[->] (p2-11) -| (p0-5) ;
  \draw[->] (p1-10) -| (p2-11) ;
  \draw[->] (p0-6) -- (p0-12) ;
  \draw[->] (p0-12) -- (p0-13) ;
  \draw[->] (p0-13) -- (p0-14) ;
\end{tikzpicture}

\nonTerminalSection{extern\_routine\_declaration}{71}

\ruleSubsection{galgas3LexiqueComponentSyntax}{galgas3LexiqueComponentSyntax}{595}

\begin{tikzpicture}
  \matrix[column sep=\ruleMatrixColumnSeparation, row sep=\ruleMatrixRowSeparation] {
    & & & & & & & & & & & & & \node (p3-13) [point] {}; & & & & & & & \node (p3-20) [point] {}; & \\
    & & & & & & & & & \node (p2-9) [terminal] {?}; & & & & & & & & & & \node (p2-19) [terminal] {,}; & \\
    & & & & & & & & \node (p1-8) [point] {}; & \node (p1-9) [terminal] {?!}; & \node (p1-10) [point] {}; & \node (p1-11) [terminal] {@type}; & \node (p1-12) [terminal] {identifier}; & & & \node (p1-15) [terminal] {error}; & \node (p1-16) [point] {}; & \node (p1-17) [terminal] {identifier}; & \node (p1-18) [point] {}; & \\
    \node (P0start) [firstPoint] {}; & & \node (p0-2) [terminal] {extern}; & \node (p0-3) [terminal] {proc}; & \node (p0-4) [terminal] {identifier}; & \node (p0-5) [point] {}; & \node (p0-6) [point] {}; & \node (p0-7) [point] {}; & & & & & & & \node (p0-14) [point] {}; & \node (p0-15) [point] {}; & & & & & & \node (p0-21) [point] {}; & \node (p0-22) [lastPoint] {}; & \\
  };
  \draw[->] (P0start) -- (p0-2) ;
  \draw[->] (p0-2) -- (p0-3) ;
  \draw[->] (p0-3) -- (p0-4) ;
  \draw (p0-4) -- (p0-6) ;
  \draw[->] (p0-7) |- (p1-9) ;
  \draw[->] (p1-8) |- (p2-9) ;
  \draw (p1-9) -- (p1-10) ;
  \draw[->] (p2-9) -| (p1-10) ;
  \draw[->] (p1-10) -- (p1-11) ;
  \draw[->] (p1-11) -- (p1-12) ;
  \draw[->] (p3-13) -| (p0-5) ;
  \draw[->] (p1-12) -| (p3-13) ;
  \draw (p0-6) -- (p0-15) ;
  \draw[->] (p0-14) |- (p1-15) ;
  \draw[->] (p1-15) -- (p1-17) ;
  \draw[->] (p1-18) |- (p2-19) ;
  \draw[->] (p3-20) -| (p1-16) ;
  \draw[->] (p2-19) -| (p3-20) ;
  \draw (p0-15) -- (p0-21) ;
  \draw[->] (p1-17) -| (p0-21) ;
  \draw[->] (p0-21) -- (p0-22) ;
\end{tikzpicture}

\nonTerminalSection{externtype\_constructor}{39}

\ruleSubsection{galgas3DeclarationsSyntax}{type-extern}{124}

\begin{tikzpicture}
  \matrix[column sep=\ruleMatrixColumnSeparation, row sep=\ruleMatrixRowSeparation] {
    & & & & & & & & & & \node (p2-10) [point] {}; & \\
    & & & & & & & \node (p1-7) [terminal] {?}; & \node (p1-8) [terminal] {@type}; & \node (p1-9) [terminal] {identifier}; & \\
    \node (P0start) [firstPoint] {}; & & \node (p0-2) [terminal] {constructor}; & \node (p0-3) [terminal] {identifier}; & \node (p0-4) [point] {}; & \node (p0-5) [point] {}; & \node (p0-6) [point] {}; & & & & & \node (p0-11) [terminal] {->}; & \node (p0-12) [terminal] {@type}; & \node (p0-13) [lastPoint] {}; & \\
  };
  \draw[->] (P0start) -- (p0-2) ;
  \draw[->] (p0-2) -- (p0-3) ;
  \draw (p0-3) -- (p0-5) ;
  \draw[->] (p0-6) |- (p1-7) ;
  \draw[->] (p1-7) -- (p1-8) ;
  \draw[->] (p1-8) -- (p1-9) ;
  \draw[->] (p2-10) -| (p0-4) ;
  \draw[->] (p1-9) -| (p2-10) ;
  \draw[->] (p0-5) -- (p0-11) ;
  \draw[->] (p0-11) -- (p0-12) ;
  \draw[->] (p0-12) -- (p0-13) ;
\end{tikzpicture}

\nonTerminalSection{externtype\_cpp\_classdeclaration}{38}

\ruleSubsection{galgas3DeclarationsSyntax}{type-extern}{111}

\begin{tikzpicture}
  \matrix[column sep=\ruleMatrixColumnSeparation, row sep=\ruleMatrixRowSeparation] {
    & & & & & & & \node (p2-7) [point] {}; & \\
    & & & & & & \node (p1-6) [terminal] {"string"}; & \\
    \node (P0start) [firstPoint] {}; & & \node (p0-2) [terminal] {\{}; & \node (p0-3) [point] {}; & \node (p0-4) [point] {}; & \node (p0-5) [point] {}; & & & \node (p0-8) [terminal] {\}}; & \node (p0-9) [lastPoint] {}; & \\
  };
  \draw[->] (P0start) -- (p0-2) ;
  \draw (p0-2) -- (p0-4) ;
  \draw[->] (p0-5) |- (p1-6) ;
  \draw[->] (p2-7) -| (p0-3) ;
  \draw[->] (p1-6) -| (p2-7) ;
  \draw[->] (p0-4) -- (p0-8) ;
  \draw[->] (p0-8) -- (p0-9) ;
\end{tikzpicture}

\nonTerminalSection{externtype\_cpp\_predeclaration}{37}

\ruleSubsection{galgas3DeclarationsSyntax}{type-extern}{98}

\begin{tikzpicture}
  \matrix[column sep=\ruleMatrixColumnSeparation, row sep=\ruleMatrixRowSeparation] {
    & & & & & & & \node (p2-7) [point] {}; & \\
    & & & & & & \node (p1-6) [terminal] {"string"}; & \\
    \node (P0start) [firstPoint] {}; & & \node (p0-2) [terminal] {\{}; & \node (p0-3) [point] {}; & \node (p0-4) [point] {}; & \node (p0-5) [point] {}; & & & \node (p0-8) [terminal] {\}}; & \node (p0-9) [lastPoint] {}; & \\
  };
  \draw[->] (P0start) -- (p0-2) ;
  \draw (p0-2) -- (p0-4) ;
  \draw[->] (p0-5) |- (p1-6) ;
  \draw[->] (p2-7) -| (p0-3) ;
  \draw[->] (p1-6) -| (p2-7) ;
  \draw[->] (p0-4) -- (p0-8) ;
  \draw[->] (p0-8) -- (p0-9) ;
\end{tikzpicture}

\nonTerminalSection{externtype\_getter}{40}

\ruleSubsection{galgas3DeclarationsSyntax}{type-extern}{142}

\begin{tikzpicture}
  \matrix[column sep=\ruleMatrixColumnSeparation, row sep=\ruleMatrixRowSeparation] {
    & & & & & & & & & & \node (p2-10) [point] {}; & \\
    & & & & & & & \node (p1-7) [terminal] {?}; & \node (p1-8) [terminal] {@type}; & \node (p1-9) [terminal] {identifier}; & \\
    \node (P0start) [firstPoint] {}; & & \node (p0-2) [terminal] {getter}; & \node (p0-3) [terminal] {identifier}; & \node (p0-4) [point] {}; & \node (p0-5) [point] {}; & \node (p0-6) [point] {}; & & & & & \node (p0-11) [terminal] {->}; & \node (p0-12) [terminal] {@type}; & \node (p0-13) [lastPoint] {}; & \\
  };
  \draw[->] (P0start) -- (p0-2) ;
  \draw[->] (p0-2) -- (p0-3) ;
  \draw (p0-3) -- (p0-5) ;
  \draw[->] (p0-6) |- (p1-7) ;
  \draw[->] (p1-7) -- (p1-8) ;
  \draw[->] (p1-8) -- (p1-9) ;
  \draw[->] (p2-10) -| (p0-4) ;
  \draw[->] (p1-9) -| (p2-10) ;
  \draw[->] (p0-5) -- (p0-11) ;
  \draw[->] (p0-11) -- (p0-12) ;
  \draw[->] (p0-12) -- (p0-13) ;
\end{tikzpicture}

\nonTerminalSection{externtype\_method}{42}

\ruleSubsection{galgas3DeclarationsSyntax}{type-extern}{169}

\begin{tikzpicture}
  \matrix[column sep=\ruleMatrixColumnSeparation, row sep=\ruleMatrixRowSeparation] {
    \node (P0start) [firstPoint] {}; & & \node (p0-2) [terminal] {method}; & \node (p0-3) [terminal] {identifier}; & \node (p0-4) [nonterminal] {\nonTerminalSymbol{formal\_parameter\_list}{11}}; & \node (p0-5) [lastPoint] {}; & \\
  };
  \draw[->] (P0start) -- (p0-2) ;
  \draw[->] (p0-2) -- (p0-3) ;
  \draw[->] (p0-3) -- (p0-4) ;
  \draw[->] (p0-4) -- (p0-5) ;
\end{tikzpicture}

\nonTerminalSection{externtype\_setter}{41}

\ruleSubsection{galgas3DeclarationsSyntax}{type-extern}{160}

\begin{tikzpicture}
  \matrix[column sep=\ruleMatrixColumnSeparation, row sep=\ruleMatrixRowSeparation] {
    \node (P0start) [firstPoint] {}; & & \node (p0-2) [terminal] {setter}; & \node (p0-3) [terminal] {identifier}; & \node (p0-4) [nonterminal] {\nonTerminalSymbol{formal\_parameter\_list}{11}}; & \node (p0-5) [lastPoint] {}; & \\
  };
  \draw[->] (P0start) -- (p0-2) ;
  \draw[->] (p0-2) -- (p0-3) ;
  \draw[->] (p0-3) -- (p0-4) ;
  \draw[->] (p0-4) -- (p0-5) ;
\end{tikzpicture}

\nonTerminalSection{factor}{7}

\ruleSubsection{galgas3ExpressionSyntax}{galgas3ExpressionSyntax}{288}

\begin{tikzpicture}
  \matrix[column sep=\ruleMatrixColumnSeparation, row sep=\ruleMatrixRowSeparation] {
    \node (P0start) [firstPoint] {}; & & \node (p0-2) [terminal] {+}; & \node (p0-3) [nonterminal] {\nonTerminalSymbol{factor}{7}}; & \node (p0-4) [lastPoint] {}; & \\
  };
  \draw[->] (P0start) -- (p0-2) ;
  \draw[->] (p0-2) -- (p0-3) ;
  \draw[->] (p0-3) -- (p0-4) ;
\end{tikzpicture}

\ruleSubsection{galgas3ExpressionSyntax}{galgas3ExpressionSyntax}{301}

\begin{tikzpicture}
  \matrix[column sep=\ruleMatrixColumnSeparation, row sep=\ruleMatrixRowSeparation] {
    \node (P0start) [firstPoint] {}; & & \node (p0-2) [terminal] {-}; & \node (p0-3) [nonterminal] {\nonTerminalSymbol{factor}{7}}; & \node (p0-4) [lastPoint] {}; & \\
  };
  \draw[->] (P0start) -- (p0-2) ;
  \draw[->] (p0-2) -- (p0-3) ;
  \draw[->] (p0-3) -- (p0-4) ;
\end{tikzpicture}

\ruleSubsection{galgas3ExpressionSyntax}{galgas3ExpressionSyntax}{314}

\begin{tikzpicture}
  \matrix[column sep=\ruleMatrixColumnSeparation, row sep=\ruleMatrixRowSeparation] {
    \node (P0start) [firstPoint] {}; & & \node (p0-2) [terminal] {\&-}; & \node (p0-3) [nonterminal] {\nonTerminalSymbol{factor}{7}}; & \node (p0-4) [lastPoint] {}; & \\
  };
  \draw[->] (P0start) -- (p0-2) ;
  \draw[->] (p0-2) -- (p0-3) ;
  \draw[->] (p0-3) -- (p0-4) ;
\end{tikzpicture}

\ruleSubsection{galgas3ExpressionSyntax}{galgas3ExpressionSyntax}{327}

\begin{tikzpicture}
  \matrix[column sep=\ruleMatrixColumnSeparation, row sep=\ruleMatrixRowSeparation] {
    \node (P0start) [firstPoint] {}; & & \node (p0-2) [terminal] {not}; & \node (p0-3) [nonterminal] {\nonTerminalSymbol{factor}{7}}; & \node (p0-4) [lastPoint] {}; & \\
  };
  \draw[->] (P0start) -- (p0-2) ;
  \draw[->] (p0-2) -- (p0-3) ;
  \draw[->] (p0-3) -- (p0-4) ;
\end{tikzpicture}

\ruleSubsection{galgas3ExpressionSyntax}{galgas3ExpressionSyntax}{340}

\begin{tikzpicture}
  \matrix[column sep=\ruleMatrixColumnSeparation, row sep=\ruleMatrixRowSeparation] {
    \node (P0start) [firstPoint] {}; & & \node (p0-2) [terminal] {$\sim$}; & \node (p0-3) [nonterminal] {\nonTerminalSymbol{factor}{7}}; & \node (p0-4) [lastPoint] {}; & \\
  };
  \draw[->] (P0start) -- (p0-2) ;
  \draw[->] (p0-2) -- (p0-3) ;
  \draw[->] (p0-3) -- (p0-4) ;
\end{tikzpicture}

\ruleSubsection{galgas3ExpressionSyntax}{galgas3ExpressionSyntax}{353}

\begin{tikzpicture}
  \matrix[column sep=\ruleMatrixColumnSeparation, row sep=\ruleMatrixRowSeparation] {
    & & & & & & & & \node (p2-8) [point] {}; & \\
    & & & & & & \node (p1-6) [terminal] {.}; & \node (p1-7) [terminal] {identifier}; & \\
    \node (P0start) [firstPoint] {}; & & \node (p0-2) [nonterminal] {\nonTerminalSymbol{primary}{8}}; & \node (p0-3) [point] {}; & \node (p0-4) [point] {}; & \node (p0-5) [point] {}; & & & & \node (p0-9) [lastPoint] {}; & \\
  };
  \draw[->] (P0start) -- (p0-2) ;
  \draw (p0-2) -- (p0-4) ;
  \draw[->] (p0-5) |- (p1-6) ;
  \draw[->] (p1-6) -- (p1-7) ;
  \draw[->] (p2-8) -| (p0-3) ;
  \draw[->] (p1-7) -| (p2-8) ;
  \draw[->] (p0-4) -- (p0-9) ;
\end{tikzpicture}

\nonTerminalSection{filewrapper\_binary\_files}{35}

\ruleSubsection{galgas3DeclarationsSyntax}{galgas3DeclarationsSyntax}{125}

\begin{tikzpicture}
  \matrix[column sep=\ruleMatrixColumnSeparation, row sep=\ruleMatrixRowSeparation] {
    & & & & & & & & \node (p3-8) [point] {}; & \\
    & & & & & & & \node (p2-7) [terminal] {,}; & \\
    & & & & \node (p1-4) [point] {}; & \node (p1-5) [terminal] {"string"}; & \node (p1-6) [point] {}; & \\
    \node (P0start) [firstPoint] {}; & & \node (p0-2) [terminal] {\{}; & \node (p0-3) [point] {}; & \node (p0-4) [point] {}; & & & & & \node (p0-9) [point] {}; & \node (p0-10) [terminal] {\}}; & \node (p0-11) [lastPoint] {}; & \\
  };
  \draw[->] (P0start) -- (p0-2) ;
  \draw (p0-2) -- (p0-4) ;
  \draw[->] (p0-3) |- (p1-5) ;
  \draw[->] (p1-6) |- (p2-7) ;
  \draw[->] (p3-8) -| (p1-4) ;
  \draw[->] (p2-7) -| (p3-8) ;
  \draw (p0-4) -- (p0-9) ;
  \draw[->] (p1-5) -| (p0-9) ;
  \draw[->] (p0-9) -- (p0-10) ;
  \draw[->] (p0-10) -- (p0-11) ;
\end{tikzpicture}

\nonTerminalSection{filewrapper\_templates}{36}

\ruleSubsection{galgas3DeclarationsSyntax}{galgas3DeclarationsSyntax}{142}

\begin{tikzpicture}
  \matrix[column sep=\ruleMatrixColumnSeparation, row sep=\ruleMatrixRowSeparation] {
    & & & & & & & & & & & & & & & & & & & \node (p5-19) [point] {}; & \\
    & & & & & & & & & & & & & & & & & & \node (p4-18) [point] {}; & \\
    & & & & & & & & & & & & & & & \node (p3-15) [terminal] {unused}; & \\
    & & & & & & & & & & & & \node (p2-12) [terminal] {?}; & \node (p2-13) [terminal] {@type}; & \node (p2-14) [point] {}; & \node (p2-15) [point] {}; & \node (p2-16) [point] {}; & \node (p2-17) [terminal] {identifier}; & \\
    & & & & & & \node (p1-6) [terminal] {template}; & \node (p1-7) [terminal] {identifier}; & \node (p1-8) [terminal] {"string"}; & \node (p1-9) [point] {}; & \node (p1-10) [point] {}; & \node (p1-11) [point] {}; & \\
    \node (P0start) [firstPoint] {}; & & \node (p0-2) [terminal] {\{}; & \node (p0-3) [point] {}; & \node (p0-4) [point] {}; & \node (p0-5) [point] {}; & & & & & & & & & & & & & & & \node (p0-20) [terminal] {\}}; & \node (p0-21) [lastPoint] {}; & \\
  };
  \draw[->] (P0start) -- (p0-2) ;
  \draw (p0-2) -- (p0-4) ;
  \draw[->] (p0-5) |- (p1-6) ;
  \draw[->] (p1-6) -- (p1-7) ;
  \draw[->] (p1-7) -- (p1-8) ;
  \draw (p1-8) -- (p1-10) ;
  \draw[->] (p1-11) |- (p2-12) ;
  \draw[->] (p2-12) -- (p2-13) ;
  \draw (p2-13) -- (p2-15) ;
  \draw[->] (p2-14) |- (p3-15) ;
  \draw (p2-15) -- (p2-16) ;
  \draw[->] (p3-15) -| (p2-16) ;
  \draw[->] (p2-16) -- (p2-17) ;
  \draw[->] (p4-18) -| (p1-9) ;
  \draw[->] (p2-17) -| (p4-18) ;
  \draw[->] (p5-19) -| (p0-3) ;
  \draw[->] (p1-10) -| (p5-19) ;
  \draw[->] (p0-4) -- (p0-20) ;
  \draw[->] (p0-20) -- (p0-21) ;
\end{tikzpicture}

\nonTerminalSection{filewrapper\_text\_files}{34}

\ruleSubsection{galgas3DeclarationsSyntax}{galgas3DeclarationsSyntax}{108}

\begin{tikzpicture}
  \matrix[column sep=\ruleMatrixColumnSeparation, row sep=\ruleMatrixRowSeparation] {
    & & & & & & & & \node (p3-8) [point] {}; & \\
    & & & & & & & \node (p2-7) [terminal] {,}; & \\
    & & & & \node (p1-4) [point] {}; & \node (p1-5) [terminal] {"string"}; & \node (p1-6) [point] {}; & \\
    \node (P0start) [firstPoint] {}; & & \node (p0-2) [terminal] {\{}; & \node (p0-3) [point] {}; & \node (p0-4) [point] {}; & & & & & \node (p0-9) [point] {}; & \node (p0-10) [terminal] {\}}; & \node (p0-11) [lastPoint] {}; & \\
  };
  \draw[->] (P0start) -- (p0-2) ;
  \draw (p0-2) -- (p0-4) ;
  \draw[->] (p0-3) |- (p1-5) ;
  \draw[->] (p1-6) |- (p2-7) ;
  \draw[->] (p3-8) -| (p1-4) ;
  \draw[->] (p2-7) -| (p3-8) ;
  \draw (p0-4) -- (p0-9) ;
  \draw[->] (p1-5) -| (p0-9) ;
  \draw[->] (p0-9) -- (p0-10) ;
  \draw[->] (p0-10) -- (p0-11) ;
\end{tikzpicture}

\nonTerminalSection{for\_instruction\_element}{24}

\ruleSubsection{galgas3InstructionsSyntax}{instruction-for}{74}

\begin{tikzpicture}
  \matrix[column sep=\ruleMatrixColumnSeparation, row sep=\ruleMatrixRowSeparation] {
    \node (P0start) [firstPoint] {}; & & \node (p0-2) [terminal] {*}; & \node (p0-3) [lastPoint] {}; & \\
  };
  \draw[->] (P0start) -- (p0-2) ;
  \draw[->] (p0-2) -- (p0-3) ;
\end{tikzpicture}

\ruleSubsection{galgas3InstructionsSyntax}{instruction-for}{82}

\begin{tikzpicture}
  \matrix[column sep=\ruleMatrixColumnSeparation, row sep=\ruleMatrixRowSeparation] {
    \node (P0start) [firstPoint] {}; & & \node (p0-2) [terminal] {uint32}; & \node (p0-3) [terminal] {*}; & \node (p0-4) [lastPoint] {}; & \\
  };
  \draw[->] (P0start) -- (p0-2) ;
  \draw[->] (p0-2) -- (p0-3) ;
  \draw[->] (p0-3) -- (p0-4) ;
\end{tikzpicture}

\ruleSubsection{galgas3InstructionsSyntax}{instruction-for}{98}

\begin{tikzpicture}
  \matrix[column sep=\ruleMatrixColumnSeparation, row sep=\ruleMatrixRowSeparation] {
    & & & \node (p1-3) [terminal] {@type}; & \\
    \node (P0start) [firstPoint] {}; & & \node (p0-2) [point] {}; & \node (p0-3) [point] {}; & \node (p0-4) [point] {}; & \node (p0-5) [terminal] {identifier}; & \node (p0-6) [lastPoint] {}; & \\
  };
  \draw (P0start) -- (p0-3) ;
  \draw[->] (p0-2) |- (p1-3) ;
  \draw (p0-3) -- (p0-4) ;
  \draw[->] (p1-3) -| (p0-4) ;
  \draw[->] (p0-4) -- (p0-5) ;
  \draw[->] (p0-5) -- (p0-6) ;
\end{tikzpicture}

\nonTerminalSection{for\_instruction\_enumerated\_object}{25}

\ruleSubsection{galgas3InstructionsSyntax}{instruction-for}{112}

\begin{tikzpicture}
  \matrix[column sep=\ruleMatrixColumnSeparation, row sep=\ruleMatrixRowSeparation] {
    & & & & & & & \node (p2-7) [point] {}; & & \node (p2-9) [terminal] {...}; & \\
    & & & & \node (p1-4) [point] {}; & \node (p1-5) [nonterminal] {\nonTerminalSymbol{for\_instruction\_element}{24}}; & \node (p1-6) [point] {}; & & \node (p1-8) [point] {}; & \node (p1-9) [point] {}; & \node (p1-10) [point] {}; & \\
    \node (P0start) [firstPoint] {}; & & \node (p0-2) [terminal] {(}; & \node (p0-3) [point] {}; & \node (p0-4) [terminal] {...}; & & & & & & & \node (p0-11) [point] {}; & \node (p0-12) [terminal] {)}; & \node (p0-13) [terminal] {in}; & \node (p0-14) [nonterminal] {\nonTerminalSymbol{expression}{1}}; & \node (p0-15) [lastPoint] {}; & \\
  };
  \draw[->] (P0start) -- (p0-2) ;
  \draw[->] (p0-2) -- (p0-4) ;
  \draw[->] (p0-3) |- (p1-5) ;
  \draw[->] (p2-7) -| (p1-4) ;
  \draw[->] (p1-6) -| (p2-7) ;
  \draw (p1-5) -- (p1-9) ;
  \draw[->] (p1-8) |- (p2-9) ;
  \draw (p1-9) -- (p1-10) ;
  \draw[->] (p2-9) -| (p1-10) ;
  \draw (p0-4) -- (p0-11) ;
  \draw[->] (p1-10) -| (p0-11) ;
  \draw[->] (p0-11) -- (p0-12) ;
  \draw[->] (p0-12) -- (p0-13) ;
  \draw[->] (p0-13) -- (p0-14) ;
  \draw[->] (p0-14) -- (p0-15) ;
\end{tikzpicture}

\ruleSubsection{galgas3InstructionsSyntax}{instruction-for}{144}

\begin{tikzpicture}
  \matrix[column sep=\ruleMatrixColumnSeparation, row sep=\ruleMatrixRowSeparation] {
    & & & \node (p1-3) [terminal] {@type}; & \\
    \node (P0start) [firstPoint] {}; & & \node (p0-2) [point] {}; & \node (p0-3) [point] {}; & \node (p0-4) [point] {}; & \node (p0-5) [terminal] {identifier}; & \node (p0-6) [terminal] {in}; & \node (p0-7) [nonterminal] {\nonTerminalSymbol{expression}{1}}; & \node (p0-8) [lastPoint] {}; & \\
  };
  \draw (P0start) -- (p0-3) ;
  \draw[->] (p0-2) |- (p1-3) ;
  \draw (p0-3) -- (p0-4) ;
  \draw[->] (p1-3) -| (p0-4) ;
  \draw[->] (p0-4) -- (p0-5) ;
  \draw[->] (p0-5) -- (p0-6) ;
  \draw[->] (p0-6) -- (p0-7) ;
  \draw[->] (p0-7) -- (p0-8) ;
\end{tikzpicture}

\ruleSubsection{galgas3InstructionsSyntax}{instruction-for}{164}

\begin{tikzpicture}
  \matrix[column sep=\ruleMatrixColumnSeparation, row sep=\ruleMatrixRowSeparation] {
    & & & & & \node (p1-5) [terminal] {identifier}; & \\
    \node (P0start) [firstPoint] {}; & & \node (p0-2) [terminal] {(}; & \node (p0-3) [terminal] {)}; & \node (p0-4) [point] {}; & \node (p0-5) [point] {}; & \node (p0-6) [point] {}; & \node (p0-7) [terminal] {in}; & \node (p0-8) [nonterminal] {\nonTerminalSymbol{expression}{1}}; & \node (p0-9) [lastPoint] {}; & \\
  };
  \draw[->] (P0start) -- (p0-2) ;
  \draw[->] (p0-2) -- (p0-3) ;
  \draw (p0-3) -- (p0-5) ;
  \draw[->] (p0-4) |- (p1-5) ;
  \draw (p0-5) -- (p0-6) ;
  \draw[->] (p1-5) -| (p0-6) ;
  \draw[->] (p0-6) -- (p0-7) ;
  \draw[->] (p0-7) -- (p0-8) ;
  \draw[->] (p0-8) -- (p0-9) ;
\end{tikzpicture}

\nonTerminalSection{formal\_input\_parameter\_list}{13}

\ruleSubsection{galgas3ParameterArgumentSyntax}{galgas3ParameterArgumentSyntax}{208}

\begin{tikzpicture}
  \matrix[column sep=\ruleMatrixColumnSeparation, row sep=\ruleMatrixRowSeparation] {
    & & & & & & & & & & & & & & \node (p3-14) [point] {}; & \\
    & & & & & & & \node (p2-7) [terminal] {let}; & & & & \node (p2-11) [terminal] {unused}; & \\
    & & & & & \node (p1-5) [terminal] {?}; & \node (p1-6) [point] {}; & \node (p1-7) [point] {}; & \node (p1-8) [point] {}; & \node (p1-9) [terminal] {@type}; & \node (p1-10) [point] {}; & \node (p1-11) [point] {}; & \node (p1-12) [point] {}; & \node (p1-13) [terminal] {identifier}; & \\
    \node (P0start) [firstPoint] {}; & & \node (p0-2) [point] {}; & \node (p0-3) [point] {}; & \node (p0-4) [point] {}; & & & & & & & & & & & \node (p0-15) [lastPoint] {}; & \\
  };
  \draw (P0start) -- (p0-3) ;
  \draw[->] (p0-4) |- (p1-5) ;
  \draw (p1-5) -- (p1-7) ;
  \draw[->] (p1-6) |- (p2-7) ;
  \draw (p1-7) -- (p1-8) ;
  \draw[->] (p2-7) -| (p1-8) ;
  \draw[->] (p1-8) -- (p1-9) ;
  \draw (p1-9) -- (p1-11) ;
  \draw[->] (p1-10) |- (p2-11) ;
  \draw (p1-11) -- (p1-12) ;
  \draw[->] (p2-11) -| (p1-12) ;
  \draw[->] (p1-12) -- (p1-13) ;
  \draw[->] (p3-14) -| (p0-2) ;
  \draw[->] (p1-13) -| (p3-14) ;
  \draw[->] (p0-3) -- (p0-15) ;
\end{tikzpicture}

\nonTerminalSection{formal\_parameter\_list}{11}

\ruleSubsection{galgas3ParameterArgumentSyntax}{galgas3ParameterArgumentSyntax}{28}

\begin{tikzpicture}
  \matrix[column sep=\ruleMatrixColumnSeparation, row sep=\ruleMatrixRowSeparation] {
    & & & & & & & & & & & & & & \node (p5-14) [point] {}; & \\
    & & & & & & \node (p4-6) [terminal] {!}; & \\
    & & & & & & \node (p3-6) [terminal] {?!}; & \\
    & & & & & & \node (p2-6) [terminal] {?}; & \node (p2-7) [terminal] {let}; & & & & \node (p2-11) [terminal] {unused}; & \\
    & & & & & \node (p1-5) [point] {}; & \node (p1-6) [terminal] {?}; & & \node (p1-8) [point] {}; & \node (p1-9) [terminal] {@type}; & \node (p1-10) [point] {}; & \node (p1-11) [point] {}; & \node (p1-12) [point] {}; & \node (p1-13) [terminal] {identifier}; & \\
    \node (P0start) [firstPoint] {}; & & \node (p0-2) [point] {}; & \node (p0-3) [point] {}; & \node (p0-4) [point] {}; & & & & & & & & & & & \node (p0-15) [lastPoint] {}; & \\
  };
  \draw (P0start) -- (p0-3) ;
  \draw[->] (p0-4) |- (p1-6) ;
  \draw[->] (p1-5) |- (p2-6) ;
  \draw[->] (p2-6) -- (p2-7) ;
  \draw[->] (p1-5) |- (p3-6) ;
  \draw[->] (p1-5) |- (p4-6) ;
  \draw (p1-6) -- (p1-8) ;
  \draw[->] (p2-7) -| (p1-8) ;
  \draw[->] (p3-6) -| (p1-8) ;
  \draw[->] (p4-6) -| (p1-8) ;
  \draw[->] (p1-8) -- (p1-9) ;
  \draw (p1-9) -- (p1-11) ;
  \draw[->] (p1-10) |- (p2-11) ;
  \draw (p1-11) -- (p1-12) ;
  \draw[->] (p2-11) -| (p1-12) ;
  \draw[->] (p1-12) -- (p1-13) ;
  \draw[->] (p5-14) -| (p0-2) ;
  \draw[->] (p1-13) -| (p5-14) ;
  \draw[->] (p0-3) -- (p0-15) ;
\end{tikzpicture}

\nonTerminalSection{grammar\_instruction\_core}{26}

\ruleSubsection{galgas3InstructionsSyntax}{instruction-grammar}{92}

\begin{tikzpicture}
  \matrix[column sep=\ruleMatrixColumnSeparation, row sep=\ruleMatrixRowSeparation] {
    & & & & & & \node (p1-6) [terminal] {:>}; & \node (p1-7) [nonterminal] {\nonTerminalSymbol{syntax\_directed\_translation\_result}{18}}; & \\
    \node (P0start) [firstPoint] {}; & & \node (p0-2) [terminal] {in}; & \node (p0-3) [nonterminal] {\nonTerminalSymbol{expression}{1}}; & \node (p0-4) [nonterminal] {\nonTerminalSymbol{actual\_parameter\_list}{12}}; & \node (p0-5) [point] {}; & \node (p0-6) [point] {}; & & \node (p0-8) [point] {}; & \node (p0-9) [lastPoint] {}; & \\
  };
  \draw[->] (P0start) -- (p0-2) ;
  \draw[->] (p0-2) -- (p0-3) ;
  \draw[->] (p0-3) -- (p0-4) ;
  \draw (p0-4) -- (p0-6) ;
  \draw[->] (p0-5) |- (p1-6) ;
  \draw[->] (p1-6) -- (p1-7) ;
  \draw (p0-6) -- (p0-8) ;
  \draw[->] (p1-7) -| (p0-8) ;
  \draw[->] (p0-8) -- (p0-9) ;
\end{tikzpicture}

\ruleSubsection{galgas3InstructionsSyntax}{instruction-grammar}{123}

\begin{tikzpicture}
  \matrix[column sep=\ruleMatrixColumnSeparation, row sep=\ruleMatrixRowSeparation] {
    & & & & \node (p1-4) [terminal] {(}; & \node (p1-5) [nonterminal] {\nonTerminalSymbol{expression}{1}}; & \node (p1-6) [terminal] {,}; & \node (p1-7) [nonterminal] {\nonTerminalSymbol{expression}{1}}; & \node (p1-8) [terminal] {)}; & & & & \node (p1-12) [terminal] {:>}; & \node (p1-13) [nonterminal] {\nonTerminalSymbol{syntax\_directed\_translation\_result}{18}}; & \\
    \node (P0start) [firstPoint] {}; & & \node (p0-2) [terminal] {on}; & \node (p0-3) [point] {}; & \node (p0-4) [nonterminal] {\nonTerminalSymbol{expression}{1}}; & & & & & \node (p0-9) [point] {}; & \node (p0-10) [nonterminal] {\nonTerminalSymbol{actual\_parameter\_list}{12}}; & \node (p0-11) [point] {}; & \node (p0-12) [point] {}; & & \node (p0-14) [point] {}; & \node (p0-15) [lastPoint] {}; & \\
  };
  \draw[->] (P0start) -- (p0-2) ;
  \draw[->] (p0-2) -- (p0-4) ;
  \draw[->] (p0-3) |- (p1-4) ;
  \draw[->] (p1-4) -- (p1-5) ;
  \draw[->] (p1-5) -- (p1-6) ;
  \draw[->] (p1-6) -- (p1-7) ;
  \draw[->] (p1-7) -- (p1-8) ;
  \draw (p0-4) -- (p0-9) ;
  \draw[->] (p1-8) -| (p0-9) ;
  \draw[->] (p0-9) -- (p0-10) ;
  \draw (p0-10) -- (p0-12) ;
  \draw[->] (p0-11) |- (p1-12) ;
  \draw[->] (p1-12) -- (p1-13) ;
  \draw (p0-12) -- (p0-14) ;
  \draw[->] (p1-13) -| (p0-14) ;
  \draw[->] (p0-14) -- (p0-15) ;
\end{tikzpicture}

\nonTerminalSection{grammar\_start\_symbol\_label}{88}

\ruleSubsection{galgas3GrammarComponentSyntax}{galgas3GrammarComponentSyntax}{90}

\begin{tikzpicture}
  \matrix[column sep=\ruleMatrixColumnSeparation, row sep=\ruleMatrixRowSeparation] {
    & & & & & & \node (p2-6) [point] {}; & \\
    & & & & & \node (p1-5) [nonterminal] {\nonTerminalSymbol{label\_formal\_parameter}{89}}; & \\
    \node (P0start) [firstPoint] {}; & & \node (p0-2) [point] {}; & \node (p0-3) [point] {}; & \node (p0-4) [point] {}; & & & \node (p0-7) [lastPoint] {}; & \\
  };
  \draw (P0start) -- (p0-3) ;
  \draw[->] (p0-4) |- (p1-5) ;
  \draw[->] (p2-6) -| (p0-2) ;
  \draw[->] (p1-5) -| (p2-6) ;
  \draw[->] (p0-3) -- (p0-7) ;
\end{tikzpicture}

\nonTerminalSection{gui\_attributes}{80}

\ruleSubsection{galgas3GuiComponentSyntax}{guiCompilation}{196}

\begin{tikzpicture}
  \matrix[column sep=\ruleMatrixColumnSeparation, row sep=\ruleMatrixRowSeparation] {
    & & & & & & & \node (p1-7) [point] {}; & \\
    \node (P0start) [firstPoint] {}; & & \node (p0-2) [terminal] {identifier}; & \node (p0-3) [terminal] {:}; & \node (p0-4) [point] {}; & \node (p0-5) [terminal] {"string"}; & \node (p0-6) [point] {}; & & \node (p0-8) [lastPoint] {}; & \\
  };
  \draw[->] (P0start) -- (p0-2) ;
  \draw[->] (p0-2) -- (p0-3) ;
  \draw[->] (p0-3) -- (p0-5) ;
  \draw[->] (p1-7) -| (p0-4) ;
  \draw[->] (p0-6) -| (p1-7) ;
  \draw[->] (p0-5) -- (p0-8) ;
\end{tikzpicture}

\nonTerminalSection{gui\_with\_lexique\_declaration}{78}

\ruleSubsection{galgas3GuiComponentSyntax}{guiCompilation}{118}

\begin{tikzpicture}
  \matrix[column sep=\ruleMatrixColumnSeparation, row sep=\ruleMatrixRowSeparation] {
    & & & & & & & & & & & & & & & & & & & & & & & & & \node (p8-25) [point] {}; & \\
    & & & & & & & & & \node (p7-9) [terminal] {identifier}; & \node (p7-10) [terminal] {:}; & \node (p7-11) [terminal] {"string"}; & \\
    & & & & & & & & & & & & & & & \node (p6-15) [point] {}; & \\
    & & & & & & & & & \node (p5-9) [terminal] {identifier}; & \node (p5-10) [terminal] {"string"}; & \node (p5-11) [terminal] {:}; & \node (p5-12) [point] {}; & \node (p5-13) [terminal] {"string"}; & \node (p5-14) [point] {}; & \\
    & & & & & & & & & & & & & & & & & & & & & & & & \node (p4-24) [point] {}; & \\
    & & & & & & & & & & & & & & & & \node (p3-16) [terminal] {(}; & \node (p3-17) [terminal] {uint32}; & \node (p3-18) [terminal] {,}; & \node (p3-19) [terminal] {uint32}; & \node (p3-20) [terminal] {)}; & \\
    & & & & & & & & & & & \node (p2-11) [terminal] {*}; & & & & & \node (p2-16) [terminal] {-}; & & & & & & & \node (p2-23) [terminal] {,}; & \\
    & & & & & & & & & \node (p1-9) [terminal] {label}; & \node (p1-10) [point] {}; & \node (p1-11) [point] {}; & \node (p1-12) [point] {}; & \node (p1-13) [point] {}; & \node (p1-14) [terminal] {\$terminal\$}; & \node (p1-15) [point] {}; & \node (p1-16) [point] {}; & & & & & \node (p1-21) [point] {}; & \node (p1-22) [point] {}; & \\
    \node (P0start) [firstPoint] {}; & & \node (p0-2) [terminal] {with}; & \node (p0-3) [terminal] {lexique}; & \node (p0-4) [terminal] {identifier}; & \node (p0-5) [terminal] {\{}; & \node (p0-6) [point] {}; & \node (p0-7) [point] {}; & \node (p0-8) [point] {}; & & & & & & & & & & & & & & & & & & \node (p0-26) [terminal] {\}}; & \node (p0-27) [lastPoint] {}; & \\
  };
  \draw[->] (P0start) -- (p0-2) ;
  \draw[->] (p0-2) -- (p0-3) ;
  \draw[->] (p0-3) -- (p0-4) ;
  \draw[->] (p0-4) -- (p0-5) ;
  \draw (p0-5) -- (p0-7) ;
  \draw[->] (p0-8) |- (p1-9) ;
  \draw (p1-9) -- (p1-11) ;
  \draw[->] (p1-10) |- (p2-11) ;
  \draw (p1-11) -- (p1-12) ;
  \draw[->] (p2-11) -| (p1-12) ;
  \draw[->] (p1-12) -- (p1-14) ;
  \draw (p1-14) -- (p1-16) ;
  \draw[->] (p1-15) |- (p2-16) ;
  \draw[->] (p1-15) |- (p3-16) ;
  \draw[->] (p3-16) -- (p3-17) ;
  \draw[->] (p3-17) -- (p3-18) ;
  \draw[->] (p3-18) -- (p3-19) ;
  \draw[->] (p3-19) -- (p3-20) ;
  \draw (p1-16) -- (p1-21) ;
  \draw[->] (p2-16) -| (p1-21) ;
  \draw[->] (p3-20) -| (p1-21) ;
  \draw[->] (p1-22) |- (p2-23) ;
  \draw[->] (p4-24) -| (p1-13) ;
  \draw[->] (p2-23) -| (p4-24) ;
  \draw[->] (p0-8) |- (p5-9) ;
  \draw[->] (p5-9) -- (p5-10) ;
  \draw[->] (p5-10) -- (p5-11) ;
  \draw[->] (p5-11) -- (p5-13) ;
  \draw[->] (p6-15) -| (p5-12) ;
  \draw[->] (p5-14) -| (p6-15) ;
  \draw[->] (p0-8) |- (p7-9) ;
  \draw[->] (p7-9) -- (p7-10) ;
  \draw[->] (p7-10) -- (p7-11) ;
  \draw[->] (p8-25) -| (p0-6) ;
  \draw[->] (p1-21) -| (p8-25) ;
  \draw[->] (p5-13) -| (p8-25) ;
  \draw[->] (p7-11) -| (p8-25) ;
  \draw[->] (p0-7) -- (p0-26) ;
  \draw[->] (p0-26) -- (p0-27) ;
\end{tikzpicture}

\nonTerminalSection{gui\_with\_option\_declaration}{79}

\ruleSubsection{galgas3GuiComponentSyntax}{guiCompilation}{187}

\begin{tikzpicture}
  \matrix[column sep=\ruleMatrixColumnSeparation, row sep=\ruleMatrixRowSeparation] {
    \node (P0start) [firstPoint] {}; & & \node (p0-2) [terminal] {with}; & \node (p0-3) [terminal] {option}; & \node (p0-4) [terminal] {identifier}; & \node (p0-5) [lastPoint] {}; & \\
  };
  \draw[->] (P0start) -- (p0-2) ;
  \draw[->] (p0-2) -- (p0-3) ;
  \draw[->] (p0-3) -- (p0-4) ;
  \draw[->] (p0-4) -- (p0-5) ;
\end{tikzpicture}

\nonTerminalSection{if\_instruction\_core}{27}

\ruleSubsection{galgas3InstructionsSyntax}{instruction-if}{21}

\begin{tikzpicture}
  \matrix[column sep=\ruleMatrixColumnSeparation, row sep=\ruleMatrixRowSeparation] {
    & & & & & & \node (p2-6) [terminal] {elsif}; & \node (p2-7) [nonterminal] {\nonTerminalSymbol{if\_instruction\_core}{27}}; & \\
    & & & & & & \node (p1-6) [terminal] {else}; & \node (p1-7) [nonterminal] {\nonTerminalSymbol{semantic\_instruction\_list}{14}}; & \\
    \node (P0start) [firstPoint] {}; & & \node (p0-2) [nonterminal] {\nonTerminalSymbol{expression}{1}}; & \node (p0-3) [terminal] {then}; & \node (p0-4) [nonterminal] {\nonTerminalSymbol{semantic\_instruction\_list}{14}}; & \node (p0-5) [point] {}; & \node (p0-6) [point] {}; & & \node (p0-8) [point] {}; & \node (p0-9) [lastPoint] {}; & \\
  };
  \draw[->] (P0start) -- (p0-2) ;
  \draw[->] (p0-2) -- (p0-3) ;
  \draw[->] (p0-3) -- (p0-4) ;
  \draw (p0-4) -- (p0-6) ;
  \draw[->] (p0-5) |- (p1-6) ;
  \draw[->] (p1-6) -- (p1-7) ;
  \draw[->] (p0-5) |- (p2-6) ;
  \draw[->] (p2-6) -- (p2-7) ;
  \draw (p0-6) -- (p0-8) ;
  \draw[->] (p1-7) -| (p0-8) ;
  \draw[->] (p2-7) -| (p0-8) ;
  \draw[->] (p0-8) -- (p0-9) ;
\end{tikzpicture}

\nonTerminalSection{insert\_or\_replace\_declaration}{45}

\ruleSubsection{galgas3DeclarationsSyntax}{type-map}{134}

\begin{tikzpicture}
  \matrix[column sep=\ruleMatrixColumnSeparation, row sep=\ruleMatrixRowSeparation] {
    \node (P0start) [firstPoint] {}; & & \node (p0-2) [terminal] {insert}; & \node (p0-3) [terminal] {or}; & \node (p0-4) [terminal] {replace}; & \node (p0-5) [lastPoint] {}; & \\
  };
  \draw[->] (P0start) -- (p0-2) ;
  \draw[->] (p0-2) -- (p0-3) ;
  \draw[->] (p0-3) -- (p0-4) ;
  \draw[->] (p0-4) -- (p0-5) ;
\end{tikzpicture}

\nonTerminalSection{issue\_fixit}{23}

\ruleSubsection{galgas3InstructionsSyntax}{instruction-error}{35}

\begin{tikzpicture}
  \matrix[column sep=\ruleMatrixColumnSeparation, row sep=\ruleMatrixRowSeparation] {
    & & & & & & & & & & \node (p6-10) [point] {}; & \\
    & & & & & & & & \node (p5-8) [terminal] {before}; & \node (p5-9) [nonterminal] {\nonTerminalSymbol{expression}{1}}; & \\
    & & & & & & & & \node (p4-8) [terminal] {after}; & \node (p4-9) [nonterminal] {\nonTerminalSymbol{expression}{1}}; & \\
    & & & & & & & & \node (p3-8) [terminal] {replace}; & \node (p3-9) [nonterminal] {\nonTerminalSymbol{expression}{1}}; & \\
    & & & & & & & & \node (p2-8) [terminal] {remove}; & \\
    & & & \node (p1-3) [terminal] {fixit}; & \node (p1-4) [terminal] {\{}; & \node (p1-5) [point] {}; & \node (p1-6) [point] {}; & \node (p1-7) [point] {}; & & & & \node (p1-11) [terminal] {\}}; & \\
    \node (P0start) [firstPoint] {}; & & \node (p0-2) [point] {}; & \node (p0-3) [point] {}; & & & & & & & & & \node (p0-12) [point] {}; & \node (p0-13) [lastPoint] {}; & \\
  };
  \draw (P0start) -- (p0-3) ;
  \draw[->] (p0-2) |- (p1-3) ;
  \draw[->] (p1-3) -- (p1-4) ;
  \draw (p1-4) -- (p1-6) ;
  \draw[->] (p1-7) |- (p2-8) ;
  \draw[->] (p1-7) |- (p3-8) ;
  \draw[->] (p3-8) -- (p3-9) ;
  \draw[->] (p1-7) |- (p4-8) ;
  \draw[->] (p4-8) -- (p4-9) ;
  \draw[->] (p1-7) |- (p5-8) ;
  \draw[->] (p5-8) -- (p5-9) ;
  \draw[->] (p6-10) -| (p1-5) ;
  \draw[->] (p2-8) -| (p6-10) ;
  \draw[->] (p3-9) -| (p6-10) ;
  \draw[->] (p4-9) -| (p6-10) ;
  \draw[->] (p5-9) -| (p6-10) ;
  \draw[->] (p1-6) -- (p1-11) ;
  \draw (p0-3) -- (p0-12) ;
  \draw[->] (p1-11) -| (p0-12) ;
  \draw[->] (p0-12) -- (p0-13) ;
\end{tikzpicture}

\nonTerminalSection{label\_formal\_parameter}{89}

\ruleSubsection{galgas3GrammarComponentSyntax}{galgas3GrammarComponentSyntax}{109}

\begin{tikzpicture}
  \matrix[column sep=\ruleMatrixColumnSeparation, row sep=\ruleMatrixRowSeparation] {
    & & & & & \node (p1-5) [terminal] {unused}; & \\
    \node (P0start) [firstPoint] {}; & & \node (p0-2) [terminal] {?}; & \node (p0-3) [terminal] {@type}; & \node (p0-4) [point] {}; & \node (p0-5) [point] {}; & \node (p0-6) [point] {}; & \node (p0-7) [terminal] {identifier}; & \node (p0-8) [lastPoint] {}; & \\
  };
  \draw[->] (P0start) -- (p0-2) ;
  \draw[->] (p0-2) -- (p0-3) ;
  \draw (p0-3) -- (p0-5) ;
  \draw[->] (p0-4) |- (p1-5) ;
  \draw (p0-5) -- (p0-6) ;
  \draw[->] (p1-5) -| (p0-6) ;
  \draw[->] (p0-6) -- (p0-7) ;
  \draw[->] (p0-7) -- (p0-8) ;
\end{tikzpicture}

\ruleSubsection{galgas3GrammarComponentSyntax}{galgas3GrammarComponentSyntax}{129}

\begin{tikzpicture}
  \matrix[column sep=\ruleMatrixColumnSeparation, row sep=\ruleMatrixRowSeparation] {
    & & & & & & \node (p1-6) [terminal] {unused}; & \\
    \node (P0start) [firstPoint] {}; & & \node (p0-2) [terminal] {?}; & \node (p0-3) [terminal] {let}; & \node (p0-4) [terminal] {@type}; & \node (p0-5) [point] {}; & \node (p0-6) [point] {}; & \node (p0-7) [point] {}; & \node (p0-8) [terminal] {identifier}; & \node (p0-9) [lastPoint] {}; & \\
  };
  \draw[->] (P0start) -- (p0-2) ;
  \draw[->] (p0-2) -- (p0-3) ;
  \draw[->] (p0-3) -- (p0-4) ;
  \draw (p0-4) -- (p0-6) ;
  \draw[->] (p0-5) |- (p1-6) ;
  \draw (p0-6) -- (p0-7) ;
  \draw[->] (p1-6) -| (p0-7) ;
  \draw[->] (p0-7) -- (p0-8) ;
  \draw[->] (p0-8) -- (p0-9) ;
\end{tikzpicture}

\ruleSubsection{galgas3GrammarComponentSyntax}{galgas3GrammarComponentSyntax}{150}

\begin{tikzpicture}
  \matrix[column sep=\ruleMatrixColumnSeparation, row sep=\ruleMatrixRowSeparation] {
    & & & & & \node (p1-5) [terminal] {unused}; & \\
    \node (P0start) [firstPoint] {}; & & \node (p0-2) [terminal] {?!}; & \node (p0-3) [terminal] {@type}; & \node (p0-4) [point] {}; & \node (p0-5) [point] {}; & \node (p0-6) [point] {}; & \node (p0-7) [terminal] {identifier}; & \node (p0-8) [lastPoint] {}; & \\
  };
  \draw[->] (P0start) -- (p0-2) ;
  \draw[->] (p0-2) -- (p0-3) ;
  \draw (p0-3) -- (p0-5) ;
  \draw[->] (p0-4) |- (p1-5) ;
  \draw (p0-5) -- (p0-6) ;
  \draw[->] (p1-5) -| (p0-6) ;
  \draw[->] (p0-6) -- (p0-7) ;
  \draw[->] (p0-7) -- (p0-8) ;
\end{tikzpicture}

\ruleSubsection{galgas3GrammarComponentSyntax}{galgas3GrammarComponentSyntax}{170}

\begin{tikzpicture}
  \matrix[column sep=\ruleMatrixColumnSeparation, row sep=\ruleMatrixRowSeparation] {
    \node (P0start) [firstPoint] {}; & & \node (p0-2) [terminal] {!}; & \node (p0-3) [terminal] {@type}; & \node (p0-4) [terminal] {identifier}; & \node (p0-5) [lastPoint] {}; & \\
  };
  \draw[->] (P0start) -- (p0-2) ;
  \draw[->] (p0-2) -- (p0-3) ;
  \draw[->] (p0-3) -- (p0-4) ;
  \draw[->] (p0-4) -- (p0-5) ;
\end{tikzpicture}

\nonTerminalSection{lexical\_attribute\_declaration}{68}

\ruleSubsection{galgas3LexiqueComponentSyntax}{galgas3LexiqueComponentSyntax}{541}

\begin{tikzpicture}
  \matrix[column sep=\ruleMatrixColumnSeparation, row sep=\ruleMatrixRowSeparation] {
    \node (P0start) [firstPoint] {}; & & \node (p0-2) [terminal] {@type}; & \node (p0-3) [terminal] {identifier}; & \node (p0-4) [lastPoint] {}; & \\
  };
  \draw[->] (P0start) -- (p0-2) ;
  \draw[->] (p0-2) -- (p0-3) ;
  \draw[->] (p0-3) -- (p0-4) ;
\end{tikzpicture}

\nonTerminalSection{lexical\_explicit\_rule}{58}

\ruleSubsection{galgas3LexiqueComponentSyntax}{galgas3LexiqueComponentSyntax}{172}

\begin{tikzpicture}
  \matrix[column sep=\ruleMatrixColumnSeparation, row sep=\ruleMatrixRowSeparation] {
    & & & & & & & & & \node (p2-9) [point] {}; & \\
    & & & & & & & & \node (p1-8) [nonterminal] {\nonTerminalSymbol{lexical\_instruction}{59}}; & \\
    \node (P0start) [firstPoint] {}; & & \node (p0-2) [terminal] {rule}; & \node (p0-3) [nonterminal] {\nonTerminalSymbol{lexical\_expression}{63}}; & \node (p0-4) [terminal] {\{}; & \node (p0-5) [point] {}; & \node (p0-6) [point] {}; & \node (p0-7) [point] {}; & & & \node (p0-10) [terminal] {\}}; & \node (p0-11) [lastPoint] {}; & \\
  };
  \draw[->] (P0start) -- (p0-2) ;
  \draw[->] (p0-2) -- (p0-3) ;
  \draw[->] (p0-3) -- (p0-4) ;
  \draw (p0-4) -- (p0-6) ;
  \draw[->] (p0-7) |- (p1-8) ;
  \draw[->] (p2-9) -| (p0-5) ;
  \draw[->] (p1-8) -| (p2-9) ;
  \draw[->] (p0-6) -- (p0-10) ;
  \draw[->] (p0-10) -- (p0-11) ;
\end{tikzpicture}

\nonTerminalSection{lexical\_expression}{63}

\ruleSubsection{galgas3LexiqueComponentSyntax}{galgas3LexiqueComponentSyntax}{433}

\begin{tikzpicture}
  \matrix[column sep=\ruleMatrixColumnSeparation, row sep=\ruleMatrixRowSeparation] {
    \node (P0start) [firstPoint] {}; & & \node (p0-2) [nonterminal] {\nonTerminalSymbol{lexical\_term}{64}}; & \node (p0-3) [lastPoint] {}; & \\
  };
  \draw[->] (P0start) -- (p0-2) ;
  \draw[->] (p0-2) -- (p0-3) ;
\end{tikzpicture}

\nonTerminalSection{lexical\_factor}{65}

\ruleSubsection{galgas3LexiqueComponentSyntax}{galgas3LexiqueComponentSyntax}{451}

\begin{tikzpicture}
  \matrix[column sep=\ruleMatrixColumnSeparation, row sep=\ruleMatrixRowSeparation] {
    \node (P0start) [firstPoint] {}; & & \node (p0-2) [terminal] {"string"}; & \node (p0-3) [lastPoint] {}; & \\
  };
  \draw[->] (P0start) -- (p0-2) ;
  \draw[->] (p0-2) -- (p0-3) ;
\end{tikzpicture}

\ruleSubsection{galgas3LexiqueComponentSyntax}{galgas3LexiqueComponentSyntax}{458}

\begin{tikzpicture}
  \matrix[column sep=\ruleMatrixColumnSeparation, row sep=\ruleMatrixRowSeparation] {
    \node (P0start) [firstPoint] {}; & & \node (p0-2) [terminal] {$\sim$}; & \node (p0-3) [terminal] {"string"}; & \node (p0-4) [terminal] {error}; & \node (p0-5) [terminal] {identifier}; & \node (p0-6) [lastPoint] {}; & \\
  };
  \draw[->] (P0start) -- (p0-2) ;
  \draw[->] (p0-2) -- (p0-3) ;
  \draw[->] (p0-3) -- (p0-4) ;
  \draw[->] (p0-4) -- (p0-5) ;
  \draw[->] (p0-5) -- (p0-6) ;
\end{tikzpicture}

\ruleSubsection{galgas3LexiqueComponentSyntax}{galgas3LexiqueComponentSyntax}{468}

\begin{tikzpicture}
  \matrix[column sep=\ruleMatrixColumnSeparation, row sep=\ruleMatrixRowSeparation] {
    & & & & \node (p1-4) [terminal] {->}; & \node (p1-5) [terminal] {'char'}; & \\
    \node (P0start) [firstPoint] {}; & & \node (p0-2) [terminal] {'char'}; & \node (p0-3) [point] {}; & \node (p0-4) [point] {}; & & \node (p0-6) [point] {}; & \node (p0-7) [lastPoint] {}; & \\
  };
  \draw[->] (P0start) -- (p0-2) ;
  \draw (p0-2) -- (p0-4) ;
  \draw[->] (p0-3) |- (p1-4) ;
  \draw[->] (p1-4) -- (p1-5) ;
  \draw (p0-4) -- (p0-6) ;
  \draw[->] (p1-5) -| (p0-6) ;
  \draw[->] (p0-6) -- (p0-7) ;
\end{tikzpicture}

\ruleSubsection{galgas3LexiqueComponentSyntax}{galgas3LexiqueComponentSyntax}{481}

\begin{tikzpicture}
  \matrix[column sep=\ruleMatrixColumnSeparation, row sep=\ruleMatrixRowSeparation] {
    \node (P0start) [firstPoint] {}; & & \node (p0-2) [terminal] {identifier}; & \node (p0-3) [lastPoint] {}; & \\
  };
  \draw[->] (P0start) -- (p0-2) ;
  \draw[->] (p0-2) -- (p0-3) ;
\end{tikzpicture}

\nonTerminalSection{lexical\_function\_declaration}{73}

\ruleSubsection{galgas3LexiqueComponentSyntax}{lexicalUnicodeTestFunctions}{64}

\begin{tikzpicture}
  \matrix[column sep=\ruleMatrixColumnSeparation, row sep=\ruleMatrixRowSeparation] {
    \node (P0start) [firstPoint] {}; & & \node (p0-2) [terminal] {func}; & \node (p0-3) [terminal] {identifier}; & \node (p0-4) [terminal] {:}; & \node (p0-5) [nonterminal] {\nonTerminalSymbol{lexical\_function\_expression}{74}}; & \node (p0-6) [lastPoint] {}; & \\
  };
  \draw[->] (P0start) -- (p0-2) ;
  \draw[->] (p0-2) -- (p0-3) ;
  \draw[->] (p0-3) -- (p0-4) ;
  \draw[->] (p0-4) -- (p0-5) ;
  \draw[->] (p0-5) -- (p0-6) ;
\end{tikzpicture}

\nonTerminalSection{lexical\_function\_expression}{74}

\ruleSubsection{galgas3LexiqueComponentSyntax}{lexicalUnicodeTestFunctions}{74}

\begin{tikzpicture}
  \matrix[column sep=\ruleMatrixColumnSeparation, row sep=\ruleMatrixRowSeparation] {
    \node (P0start) [firstPoint] {}; & & \node (p0-2) [nonterminal] {\nonTerminalSymbol{lexical\_function\_term}{75}}; & \node (p0-3) [lastPoint] {}; & \\
  };
  \draw[->] (P0start) -- (p0-2) ;
  \draw[->] (p0-2) -- (p0-3) ;
\end{tikzpicture}

\nonTerminalSection{lexical\_function\_factor}{76}

\ruleSubsection{galgas3LexiqueComponentSyntax}{lexicalUnicodeTestFunctions}{92}

\begin{tikzpicture}
  \matrix[column sep=\ruleMatrixColumnSeparation, row sep=\ruleMatrixRowSeparation] {
    & & & & \node (p1-4) [terminal] {->}; & \node (p1-5) [terminal] {'char'}; & \\
    \node (P0start) [firstPoint] {}; & & \node (p0-2) [terminal] {'char'}; & \node (p0-3) [point] {}; & \node (p0-4) [point] {}; & & \node (p0-6) [point] {}; & \node (p0-7) [lastPoint] {}; & \\
  };
  \draw[->] (P0start) -- (p0-2) ;
  \draw (p0-2) -- (p0-4) ;
  \draw[->] (p0-3) |- (p1-4) ;
  \draw[->] (p1-4) -- (p1-5) ;
  \draw (p0-4) -- (p0-6) ;
  \draw[->] (p1-5) -| (p0-6) ;
  \draw[->] (p0-6) -- (p0-7) ;
\end{tikzpicture}

\ruleSubsection{galgas3LexiqueComponentSyntax}{lexicalUnicodeTestFunctions}{105}

\begin{tikzpicture}
  \matrix[column sep=\ruleMatrixColumnSeparation, row sep=\ruleMatrixRowSeparation] {
    \node (P0start) [firstPoint] {}; & & \node (p0-2) [terminal] {identifier}; & \node (p0-3) [lastPoint] {}; & \\
  };
  \draw[->] (P0start) -- (p0-2) ;
  \draw[->] (p0-2) -- (p0-3) ;
\end{tikzpicture}

\nonTerminalSection{lexical\_function\_term}{75}

\ruleSubsection{galgas3LexiqueComponentSyntax}{lexicalUnicodeTestFunctions}{80}

\begin{tikzpicture}
  \matrix[column sep=\ruleMatrixColumnSeparation, row sep=\ruleMatrixRowSeparation] {
    & & & & & & & & \node (p2-8) [point] {}; & \\
    & & & & & & \node (p1-6) [terminal] {|}; & \node (p1-7) [nonterminal] {\nonTerminalSymbol{lexical\_function\_factor}{76}}; & \\
    \node (P0start) [firstPoint] {}; & & \node (p0-2) [nonterminal] {\nonTerminalSymbol{lexical\_function\_factor}{76}}; & \node (p0-3) [point] {}; & \node (p0-4) [point] {}; & \node (p0-5) [point] {}; & & & & \node (p0-9) [lastPoint] {}; & \\
  };
  \draw[->] (P0start) -- (p0-2) ;
  \draw (p0-2) -- (p0-4) ;
  \draw[->] (p0-5) |- (p1-6) ;
  \draw[->] (p1-6) -- (p1-7) ;
  \draw[->] (p2-8) -| (p0-3) ;
  \draw[->] (p1-7) -| (p2-8) ;
  \draw[->] (p0-4) -- (p0-9) ;
\end{tikzpicture}

\nonTerminalSection{lexical\_implicit\_rule}{57}

\ruleSubsection{galgas3LexiqueComponentSyntax}{galgas3LexiqueComponentSyntax}{163}

\begin{tikzpicture}
  \matrix[column sep=\ruleMatrixColumnSeparation, row sep=\ruleMatrixRowSeparation] {
    \node (P0start) [firstPoint] {}; & & \node (p0-2) [terminal] {rule}; & \node (p0-3) [terminal] {list}; & \node (p0-4) [terminal] {identifier}; & \node (p0-5) [lastPoint] {}; & \\
  };
  \draw[->] (P0start) -- (p0-2) ;
  \draw[->] (p0-2) -- (p0-3) ;
  \draw[->] (p0-3) -- (p0-4) ;
  \draw[->] (p0-4) -- (p0-5) ;
\end{tikzpicture}

\nonTerminalSection{lexical\_indexing\_declaration}{53}

\ruleSubsection{galgas3LexiqueComponentSyntax}{galgas3LexiqueComponentSyntax}{106}

\begin{tikzpicture}
  \matrix[column sep=\ruleMatrixColumnSeparation, row sep=\ruleMatrixRowSeparation] {
    \node (P0start) [firstPoint] {}; & & \node (p0-2) [terminal] {indexing}; & \node (p0-3) [terminal] {identifier}; & \node (p0-4) [terminal] {:}; & \node (p0-5) [terminal] {"string"}; & \node (p0-6) [lastPoint] {}; & \\
  };
  \draw[->] (P0start) -- (p0-2) ;
  \draw[->] (p0-2) -- (p0-3) ;
  \draw[->] (p0-3) -- (p0-4) ;
  \draw[->] (p0-4) -- (p0-5) ;
  \draw[->] (p0-5) -- (p0-6) ;
\end{tikzpicture}

\nonTerminalSection{lexical\_instruction}{59}

\ruleSubsection{galgas3LexiqueComponentSyntax}{galgas3LexiqueComponentSyntax}{187}

\begin{tikzpicture}
  \matrix[column sep=\ruleMatrixColumnSeparation, row sep=\ruleMatrixRowSeparation] {
    \node (P0start) [firstPoint] {}; & & \node (p0-2) [terminal] {send}; & \node (p0-3) [nonterminal] {\nonTerminalSymbol{lexical\_send\_instruction}{60}}; & \node (p0-4) [lastPoint] {}; & \\
  };
  \draw[->] (P0start) -- (p0-2) ;
  \draw[->] (p0-2) -- (p0-3) ;
  \draw[->] (p0-3) -- (p0-4) ;
\end{tikzpicture}

\ruleSubsection{galgas3LexiqueComponentSyntax}{galgas3LexiqueComponentSyntax}{231}

\begin{tikzpicture}
  \matrix[column sep=\ruleMatrixColumnSeparation, row sep=\ruleMatrixRowSeparation] {
    & & & & & & & & & & & \node (p3-11) [point] {}; & \\
    & & & & & & & \node (p2-7) [point] {}; & \\
    & & & & & & \node (p1-6) [nonterminal] {\nonTerminalSymbol{lexical\_instruction}{59}}; & \\
    \node (P0start) [firstPoint] {}; & & \node (p0-2) [terminal] {repeat}; & \node (p0-3) [point] {}; & \node (p0-4) [point] {}; & \node (p0-5) [point] {}; & & & \node (p0-8) [point] {}; & \node (p0-9) [nonterminal] {\nonTerminalSymbol{repeat\_while\_branch}{61}}; & \node (p0-10) [point] {}; & & \node (p0-12) [terminal] {end}; & \node (p0-13) [lastPoint] {}; & \\
  };
  \draw[->] (P0start) -- (p0-2) ;
  \draw (p0-2) -- (p0-4) ;
  \draw[->] (p0-5) |- (p1-6) ;
  \draw[->] (p2-7) -| (p0-3) ;
  \draw[->] (p1-6) -| (p2-7) ;
  \draw[->] (p0-4) -- (p0-9) ;
  \draw[->] (p3-11) -| (p0-8) ;
  \draw[->] (p0-10) -| (p3-11) ;
  \draw[->] (p0-9) -- (p0-12) ;
  \draw[->] (p0-12) -- (p0-13) ;
\end{tikzpicture}

\ruleSubsection{galgas3LexiqueComponentSyntax}{galgas3LexiqueComponentSyntax}{266}

\begin{tikzpicture}
  \matrix[column sep=\ruleMatrixColumnSeparation, row sep=\ruleMatrixRowSeparation] {
    & & & & & & & & & & & & & \node (p3-13) [point] {}; & \\
    & & & & & & & & & & & \node (p2-11) [point] {}; & & & & & & & & \node (p2-19) [point] {}; & \\
    & & & & & & & & & & \node (p1-10) [nonterminal] {\nonTerminalSymbol{lexical\_instruction}{59}}; & & & & & & & & \node (p1-18) [nonterminal] {\nonTerminalSymbol{lexical\_instruction}{59}}; & \\
    \node (P0start) [firstPoint] {}; & & \node (p0-2) [terminal] {select}; & \node (p0-3) [point] {}; & \node (p0-4) [terminal] {case}; & \node (p0-5) [nonterminal] {\nonTerminalSymbol{lexical\_expression}{63}}; & \node (p0-6) [terminal] {:}; & \node (p0-7) [point] {}; & \node (p0-8) [point] {}; & \node (p0-9) [point] {}; & & & \node (p0-12) [point] {}; & & \node (p0-14) [terminal] {default}; & \node (p0-15) [point] {}; & \node (p0-16) [point] {}; & \node (p0-17) [point] {}; & & & \node (p0-20) [terminal] {end}; & \node (p0-21) [lastPoint] {}; & \\
  };
  \draw[->] (P0start) -- (p0-2) ;
  \draw[->] (p0-2) -- (p0-4) ;
  \draw[->] (p0-4) -- (p0-5) ;
  \draw[->] (p0-5) -- (p0-6) ;
  \draw (p0-6) -- (p0-8) ;
  \draw[->] (p0-9) |- (p1-10) ;
  \draw[->] (p2-11) -| (p0-7) ;
  \draw[->] (p1-10) -| (p2-11) ;
  \draw[->] (p3-13) -| (p0-3) ;
  \draw[->] (p0-12) -| (p3-13) ;
  \draw[->] (p0-8) -- (p0-14) ;
  \draw (p0-14) -- (p0-16) ;
  \draw[->] (p0-17) |- (p1-18) ;
  \draw[->] (p2-19) -| (p0-15) ;
  \draw[->] (p1-18) -| (p2-19) ;
  \draw[->] (p0-16) -- (p0-20) ;
  \draw[->] (p0-20) -- (p0-21) ;
\end{tikzpicture}

\ruleSubsection{galgas3LexiqueComponentSyntax}{galgas3LexiqueComponentSyntax}{296}

\begin{tikzpicture}
  \matrix[column sep=\ruleMatrixColumnSeparation, row sep=\ruleMatrixRowSeparation] {
    & & & & & & & & & \node (p3-9) [point] {}; & & & & & & & \node (p3-16) [point] {}; & \\
    & & & & & & & \node (p2-7) [terminal] {!?}; & \node (p2-8) [terminal] {identifier}; & & & & & & & \node (p2-15) [terminal] {,}; & \\
    & & & & & & & \node (p1-7) [terminal] {!}; & \node (p1-8) [nonterminal] {\nonTerminalSymbol{lexical\_output\_effective\_argument}{62}}; & & & \node (p1-11) [terminal] {error}; & \node (p1-12) [point] {}; & \node (p1-13) [terminal] {identifier}; & \node (p1-14) [point] {}; & \\
    \node (P0start) [firstPoint] {}; & & \node (p0-2) [terminal] {identifier}; & \node (p0-3) [terminal] {(}; & \node (p0-4) [point] {}; & \node (p0-5) [point] {}; & \node (p0-6) [point] {}; & & & & \node (p0-10) [point] {}; & \node (p0-11) [point] {}; & & & & & & \node (p0-17) [point] {}; & \node (p0-18) [terminal] {)}; & \node (p0-19) [lastPoint] {}; & \\
  };
  \draw[->] (P0start) -- (p0-2) ;
  \draw[->] (p0-2) -- (p0-3) ;
  \draw (p0-3) -- (p0-5) ;
  \draw[->] (p0-6) |- (p1-7) ;
  \draw[->] (p1-7) -- (p1-8) ;
  \draw[->] (p0-6) |- (p2-7) ;
  \draw[->] (p2-7) -- (p2-8) ;
  \draw[->] (p3-9) -| (p0-4) ;
  \draw[->] (p1-8) -| (p3-9) ;
  \draw[->] (p2-8) -| (p3-9) ;
  \draw (p0-5) -- (p0-11) ;
  \draw[->] (p0-10) |- (p1-11) ;
  \draw[->] (p1-11) -- (p1-13) ;
  \draw[->] (p1-14) |- (p2-15) ;
  \draw[->] (p3-16) -| (p1-12) ;
  \draw[->] (p2-15) -| (p3-16) ;
  \draw (p0-11) -- (p0-17) ;
  \draw[->] (p1-13) -| (p0-17) ;
  \draw[->] (p0-17) -- (p0-18) ;
  \draw[->] (p0-18) -- (p0-19) ;
\end{tikzpicture}

\ruleSubsection{galgas3LexiqueComponentSyntax}{galgas3LexiqueComponentSyntax}{384}

\begin{tikzpicture}
  \matrix[column sep=\ruleMatrixColumnSeparation, row sep=\ruleMatrixRowSeparation] {
    \node (P0start) [firstPoint] {}; & & \node (p0-2) [terminal] {error}; & \node (p0-3) [terminal] {identifier}; & \node (p0-4) [lastPoint] {}; & \\
  };
  \draw[->] (P0start) -- (p0-2) ;
  \draw[->] (p0-2) -- (p0-3) ;
  \draw[->] (p0-3) -- (p0-4) ;
\end{tikzpicture}

\ruleSubsection{galgas3LexiqueComponentSyntax}{galgas3LexiqueComponentSyntax}{392}

\begin{tikzpicture}
  \matrix[column sep=\ruleMatrixColumnSeparation, row sep=\ruleMatrixRowSeparation] {
    \node (P0start) [firstPoint] {}; & & \node (p0-2) [terminal] {warning}; & \node (p0-3) [terminal] {identifier}; & \node (p0-4) [lastPoint] {}; & \\
  };
  \draw[->] (P0start) -- (p0-2) ;
  \draw[->] (p0-2) -- (p0-3) ;
  \draw[->] (p0-3) -- (p0-4) ;
\end{tikzpicture}

\ruleSubsection{galgas3LexiqueComponentSyntax}{galgas3LexiqueComponentSyntax}{400}

\begin{tikzpicture}
  \matrix[column sep=\ruleMatrixColumnSeparation, row sep=\ruleMatrixRowSeparation] {
    \node (P0start) [firstPoint] {}; & & \node (p0-2) [terminal] {drop}; & \node (p0-3) [terminal] {\$terminal\$}; & \node (p0-4) [lastPoint] {}; & \\
  };
  \draw[->] (P0start) -- (p0-2) ;
  \draw[->] (p0-2) -- (p0-3) ;
  \draw[->] (p0-3) -- (p0-4) ;
\end{tikzpicture}

\ruleSubsection{galgas3LexiqueComponentSyntax}{galgas3LexiqueComponentSyntax}{408}

\begin{tikzpicture}
  \matrix[column sep=\ruleMatrixColumnSeparation, row sep=\ruleMatrixRowSeparation] {
    \node (P0start) [firstPoint] {}; & & \node (p0-2) [terminal] {tag}; & \node (p0-3) [terminal] {identifier}; & \node (p0-4) [lastPoint] {}; & \\
  };
  \draw[->] (P0start) -- (p0-2) ;
  \draw[->] (p0-2) -- (p0-3) ;
  \draw[->] (p0-3) -- (p0-4) ;
\end{tikzpicture}

\ruleSubsection{galgas3LexiqueComponentSyntax}{galgas3LexiqueComponentSyntax}{416}

\begin{tikzpicture}
  \matrix[column sep=\ruleMatrixColumnSeparation, row sep=\ruleMatrixRowSeparation] {
    \node (P0start) [firstPoint] {}; & & \node (p0-2) [terminal] {rewind}; & \node (p0-3) [terminal] {identifier}; & \node (p0-4) [terminal] {send}; & \node (p0-5) [terminal] {\$terminal\$}; & \node (p0-6) [lastPoint] {}; & \\
  };
  \draw[->] (P0start) -- (p0-2) ;
  \draw[->] (p0-2) -- (p0-3) ;
  \draw[->] (p0-3) -- (p0-4) ;
  \draw[->] (p0-4) -- (p0-5) ;
  \draw[->] (p0-5) -- (p0-6) ;
\end{tikzpicture}

\ruleSubsection{galgas3LexiqueComponentSyntax}{galgas3LexiqueComponentSyntax}{426}

\begin{tikzpicture}
  \matrix[column sep=\ruleMatrixColumnSeparation, row sep=\ruleMatrixRowSeparation] {
    \node (P0start) [firstPoint] {}; & & \node (p0-2) [terminal] {log}; & \node (p0-3) [lastPoint] {}; & \\
  };
  \draw[->] (P0start) -- (p0-2) ;
  \draw[->] (p0-2) -- (p0-3) ;
\end{tikzpicture}

\nonTerminalSection{lexical\_list\_declaration}{66}

\ruleSubsection{galgas3LexiqueComponentSyntax}{galgas3LexiqueComponentSyntax}{488}

\begin{tikzpicture}
  \matrix[column sep=\ruleMatrixColumnSeparation, row sep=\ruleMatrixRowSeparation] {
    & & & & & & & & & \node (p2-9) [point] {}; & & & & & & & & & & & & & \node (p2-22) [point] {}; & \\
    & & & & & & & \node (p1-7) [terminal] {!}; & \node (p1-8) [terminal] {identifier}; & & & \node (p1-11) [point] {}; & & & & & & & & & & \node (p1-21) [terminal] {,}; & \\
    \node (P0start) [firstPoint] {}; & & \node (p0-2) [terminal] {list}; & \node (p0-3) [terminal] {identifier}; & \node (p0-4) [point] {}; & \node (p0-5) [point] {}; & \node (p0-6) [point] {}; & & & & \node (p0-10) [point] {}; & \node (p0-11) [terminal] {style}; & \node (p0-12) [terminal] {identifier}; & \node (p0-13) [point] {}; & \node (p0-14) [terminal] {error}; & \node (p0-15) [terminal] {message}; & \node (p0-16) [terminal] {"string"}; & \node (p0-17) [terminal] {\{}; & \node (p0-18) [point] {}; & \node (p0-19) [nonterminal] {\nonTerminalSymbol{lexical\_list\_entry}{67}}; & \node (p0-20) [point] {}; & & & \node (p0-23) [terminal] {\}}; & \node (p0-24) [lastPoint] {}; & \\
  };
  \draw[->] (P0start) -- (p0-2) ;
  \draw[->] (p0-2) -- (p0-3) ;
  \draw (p0-3) -- (p0-5) ;
  \draw[->] (p0-6) |- (p1-7) ;
  \draw[->] (p1-7) -- (p1-8) ;
  \draw[->] (p2-9) -| (p0-4) ;
  \draw[->] (p1-8) -| (p2-9) ;
  \draw[->] (p0-5) -- (p0-11) ;
  \draw[->] (p0-11) -- (p0-12) ;
  \draw (p0-10) |- (p1-11) ;
  \draw (p0-12) -- (p0-13) ;
  \draw[->] (p1-11) -| (p0-13) ;
  \draw[->] (p0-13) -- (p0-14) ;
  \draw[->] (p0-14) -- (p0-15) ;
  \draw[->] (p0-15) -- (p0-16) ;
  \draw[->] (p0-16) -- (p0-17) ;
  \draw[->] (p0-17) -- (p0-19) ;
  \draw[->] (p0-20) |- (p1-21) ;
  \draw[->] (p2-22) -| (p0-18) ;
  \draw[->] (p1-21) -| (p2-22) ;
  \draw[->] (p0-19) -- (p0-23) ;
  \draw[->] (p0-23) -- (p0-24) ;
\end{tikzpicture}

\nonTerminalSection{lexical\_list\_entry}{67}

\ruleSubsection{galgas3LexiqueComponentSyntax}{galgas3LexiqueComponentSyntax}{521}

\begin{tikzpicture}
  \matrix[column sep=\ruleMatrixColumnSeparation, row sep=\ruleMatrixRowSeparation] {
    & & & & \node (p1-4) [terminal] {\verb=%=attribute}; & & & \node (p1-7) [point] {}; & \\
    \node (P0start) [firstPoint] {}; & & \node (p0-2) [terminal] {"string"}; & \node (p0-3) [point] {}; & \node (p0-4) [point] {}; & \node (p0-5) [point] {}; & \node (p0-6) [point] {}; & \node (p0-7) [terminal] {->}; & \node (p0-8) [terminal] {\$terminal\$}; & \node (p0-9) [point] {}; & \node (p0-10) [lastPoint] {}; & \\
  };
  \draw[->] (P0start) -- (p0-2) ;
  \draw (p0-2) -- (p0-4) ;
  \draw[->] (p0-3) |- (p1-4) ;
  \draw (p0-4) -- (p0-5) ;
  \draw[->] (p1-4) -| (p0-5) ;
  \draw[->] (p0-5) -- (p0-7) ;
  \draw[->] (p0-7) -- (p0-8) ;
  \draw (p0-6) |- (p1-7) ;
  \draw (p0-8) -- (p0-9) ;
  \draw[->] (p1-7) -| (p0-9) ;
  \draw[->] (p0-9) -- (p0-10) ;
\end{tikzpicture}

\nonTerminalSection{lexical\_message\_declaration}{56}

\ruleSubsection{galgas3LexiqueComponentSyntax}{galgas3LexiqueComponentSyntax}{152}

\begin{tikzpicture}
  \matrix[column sep=\ruleMatrixColumnSeparation, row sep=\ruleMatrixRowSeparation] {
    \node (P0start) [firstPoint] {}; & & \node (p0-2) [terminal] {message}; & \node (p0-3) [terminal] {identifier}; & \node (p0-4) [terminal] {:}; & \node (p0-5) [terminal] {"string"}; & \node (p0-6) [lastPoint] {}; & \\
  };
  \draw[->] (P0start) -- (p0-2) ;
  \draw[->] (p0-2) -- (p0-3) ;
  \draw[->] (p0-3) -- (p0-4) ;
  \draw[->] (p0-4) -- (p0-5) ;
  \draw[->] (p0-5) -- (p0-6) ;
\end{tikzpicture}

\nonTerminalSection{lexical\_output\_effective\_argument}{62}

\ruleSubsection{galgas3LexiqueComponentSyntax}{galgas3LexiqueComponentSyntax}{339}

\begin{tikzpicture}
  \matrix[column sep=\ruleMatrixColumnSeparation, row sep=\ruleMatrixRowSeparation] {
    \node (P0start) [firstPoint] {}; & & \node (p0-2) [terminal] {'char'}; & \node (p0-3) [lastPoint] {}; & \\
  };
  \draw[->] (P0start) -- (p0-2) ;
  \draw[->] (p0-2) -- (p0-3) ;
\end{tikzpicture}

\ruleSubsection{galgas3LexiqueComponentSyntax}{galgas3LexiqueComponentSyntax}{346}

\begin{tikzpicture}
  \matrix[column sep=\ruleMatrixColumnSeparation, row sep=\ruleMatrixRowSeparation] {
    \node (P0start) [firstPoint] {}; & & \node (p0-2) [terminal] {uint32}; & \node (p0-3) [lastPoint] {}; & \\
  };
  \draw[->] (P0start) -- (p0-2) ;
  \draw[->] (p0-2) -- (p0-3) ;
\end{tikzpicture}

\ruleSubsection{galgas3LexiqueComponentSyntax}{galgas3LexiqueComponentSyntax}{353}

\begin{tikzpicture}
  \matrix[column sep=\ruleMatrixColumnSeparation, row sep=\ruleMatrixRowSeparation] {
    \node (P0start) [firstPoint] {}; & & \node (p0-2) [terminal] {*}; & \node (p0-3) [lastPoint] {}; & \\
  };
  \draw[->] (P0start) -- (p0-2) ;
  \draw[->] (p0-2) -- (p0-3) ;
\end{tikzpicture}

\ruleSubsection{galgas3LexiqueComponentSyntax}{galgas3LexiqueComponentSyntax}{361}

\begin{tikzpicture}
  \matrix[column sep=\ruleMatrixColumnSeparation, row sep=\ruleMatrixRowSeparation] {
    & & & & & & & & & & \node (p3-10) [point] {}; & \\
    & & & & & & & & \node (p2-8) [terminal] {!}; & \node (p2-9) [nonterminal] {\nonTerminalSymbol{lexical\_output\_effective\_argument}{62}}; & \\
    & & & & \node (p1-4) [terminal] {(}; & \node (p1-5) [point] {}; & \node (p1-6) [point] {}; & \node (p1-7) [point] {}; & & & & \node (p1-11) [terminal] {)}; & \\
    \node (P0start) [firstPoint] {}; & & \node (p0-2) [terminal] {identifier}; & \node (p0-3) [point] {}; & \node (p0-4) [point] {}; & & & & & & & & \node (p0-12) [point] {}; & \node (p0-13) [lastPoint] {}; & \\
  };
  \draw[->] (P0start) -- (p0-2) ;
  \draw (p0-2) -- (p0-4) ;
  \draw[->] (p0-3) |- (p1-4) ;
  \draw (p1-4) -- (p1-6) ;
  \draw[->] (p1-7) |- (p2-8) ;
  \draw[->] (p2-8) -- (p2-9) ;
  \draw[->] (p3-10) -| (p1-5) ;
  \draw[->] (p2-9) -| (p3-10) ;
  \draw[->] (p1-6) -- (p1-11) ;
  \draw (p0-4) -- (p0-12) ;
  \draw[->] (p1-11) -| (p0-12) ;
  \draw[->] (p0-12) -- (p0-13) ;
\end{tikzpicture}

\nonTerminalSection{lexical\_send\_instruction}{60}

\ruleSubsection{galgas3LexiqueComponentSyntax}{galgas3LexiqueComponentSyntax}{195}

\begin{tikzpicture}
  \matrix[column sep=\ruleMatrixColumnSeparation, row sep=\ruleMatrixRowSeparation] {
    \node (P0start) [firstPoint] {}; & & \node (p0-2) [terminal] {\$terminal\$}; & \node (p0-3) [lastPoint] {}; & \\
  };
  \draw[->] (P0start) -- (p0-2) ;
  \draw[->] (p0-2) -- (p0-3) ;
\end{tikzpicture}

\ruleSubsection{galgas3LexiqueComponentSyntax}{galgas3LexiqueComponentSyntax}{202}

\begin{tikzpicture}
  \matrix[column sep=\ruleMatrixColumnSeparation, row sep=\ruleMatrixRowSeparation] {
    & & & & & & & & & & \node (p2-10) [point] {}; & \\
    & & & & & & & & & \node (p1-9) [terminal] {search}; & & & \node (p1-12) [terminal] {error}; & \node (p1-13) [terminal] {identifier}; & \\
    \node (P0start) [firstPoint] {}; & & \node (p0-2) [terminal] {search}; & \node (p0-3) [point] {}; & \node (p0-4) [terminal] {identifier}; & \node (p0-5) [terminal] {in}; & \node (p0-6) [terminal] {identifier}; & \node (p0-7) [terminal] {default}; & \node (p0-8) [point] {}; & & & \node (p0-11) [point] {}; & \node (p0-12) [terminal] {\$terminal\$}; & & \node (p0-14) [point] {}; & \node (p0-15) [lastPoint] {}; & \\
  };
  \draw[->] (P0start) -- (p0-2) ;
  \draw[->] (p0-2) -- (p0-4) ;
  \draw[->] (p0-4) -- (p0-5) ;
  \draw[->] (p0-5) -- (p0-6) ;
  \draw[->] (p0-6) -- (p0-7) ;
  \draw[->] (p0-8) |- (p1-9) ;
  \draw[->] (p2-10) -| (p0-3) ;
  \draw[->] (p1-9) -| (p2-10) ;
  \draw[->] (p0-7) -- (p0-12) ;
  \draw[->] (p0-11) |- (p1-12) ;
  \draw[->] (p1-12) -- (p1-13) ;
  \draw (p0-12) -- (p0-14) ;
  \draw[->] (p1-13) -| (p0-14) ;
  \draw[->] (p0-14) -- (p0-15) ;
\end{tikzpicture}

\nonTerminalSection{lexical\_term}{64}

\ruleSubsection{galgas3LexiqueComponentSyntax}{galgas3LexiqueComponentSyntax}{439}

\begin{tikzpicture}
  \matrix[column sep=\ruleMatrixColumnSeparation, row sep=\ruleMatrixRowSeparation] {
    & & & & & & & & \node (p2-8) [point] {}; & \\
    & & & & & & \node (p1-6) [terminal] {|}; & \node (p1-7) [nonterminal] {\nonTerminalSymbol{lexical\_factor}{65}}; & \\
    \node (P0start) [firstPoint] {}; & & \node (p0-2) [nonterminal] {\nonTerminalSymbol{lexical\_factor}{65}}; & \node (p0-3) [point] {}; & \node (p0-4) [point] {}; & \node (p0-5) [point] {}; & & & & \node (p0-9) [lastPoint] {}; & \\
  };
  \draw[->] (P0start) -- (p0-2) ;
  \draw (p0-2) -- (p0-4) ;
  \draw[->] (p0-5) |- (p1-6) ;
  \draw[->] (p1-6) -- (p1-7) ;
  \draw[->] (p2-8) -| (p0-3) ;
  \draw[->] (p1-7) -| (p2-8) ;
  \draw[->] (p0-4) -- (p0-9) ;
\end{tikzpicture}

\nonTerminalSection{map\_insert\_setter\_declaration}{46}

\ruleSubsection{galgas3DeclarationsSyntax}{type-map}{143}

\begin{tikzpicture}
  \matrix[column sep=\ruleMatrixColumnSeparation, row sep=\ruleMatrixRowSeparation] {
    & & & & & \node (p1-5) [terminal] {state}; & \node (p1-6) [terminal] {identifier}; & & & & & & \node (p1-12) [terminal] {,}; & \node (p1-13) [terminal] {"string"}; & \\
    \node (P0start) [firstPoint] {}; & & \node (p0-2) [terminal] {insert}; & \node (p0-3) [terminal] {identifier}; & \node (p0-4) [point] {}; & \node (p0-5) [point] {}; & & \node (p0-7) [point] {}; & \node (p0-8) [terminal] {error}; & \node (p0-9) [terminal] {message}; & \node (p0-10) [terminal] {"string"}; & \node (p0-11) [point] {}; & \node (p0-12) [point] {}; & & \node (p0-14) [point] {}; & \node (p0-15) [lastPoint] {}; & \\
  };
  \draw[->] (P0start) -- (p0-2) ;
  \draw[->] (p0-2) -- (p0-3) ;
  \draw (p0-3) -- (p0-5) ;
  \draw[->] (p0-4) |- (p1-5) ;
  \draw[->] (p1-5) -- (p1-6) ;
  \draw (p0-5) -- (p0-7) ;
  \draw[->] (p1-6) -| (p0-7) ;
  \draw[->] (p0-7) -- (p0-8) ;
  \draw[->] (p0-8) -- (p0-9) ;
  \draw[->] (p0-9) -- (p0-10) ;
  \draw (p0-10) -- (p0-12) ;
  \draw[->] (p0-11) |- (p1-12) ;
  \draw[->] (p1-12) -- (p1-13) ;
  \draw (p0-12) -- (p0-14) ;
  \draw[->] (p1-13) -| (p0-14) ;
  \draw[->] (p0-14) -- (p0-15) ;
\end{tikzpicture}

\nonTerminalSection{match\_entry}{28}

\ruleSubsection{galgas3InstructionsSyntax}{instruction-match}{37}

\begin{tikzpicture}
  \matrix[column sep=\ruleMatrixColumnSeparation, row sep=\ruleMatrixRowSeparation] {
    \node (P0start) [firstPoint] {}; & & \node (p0-2) [terminal] {identifier}; & \node (p0-3) [lastPoint] {}; & \\
  };
  \draw[->] (P0start) -- (p0-2) ;
  \draw[->] (p0-2) -- (p0-3) ;
\end{tikzpicture}

\ruleSubsection{galgas3InstructionsSyntax}{instruction-match}{47}

\begin{tikzpicture}
  \matrix[column sep=\ruleMatrixColumnSeparation, row sep=\ruleMatrixRowSeparation] {
    & & & & \node (p1-4) [terminal] {identifier}; & \\
    \node (P0start) [firstPoint] {}; & & \node (p0-2) [terminal] {@type}; & \node (p0-3) [point] {}; & \node (p0-4) [point] {}; & \node (p0-5) [point] {}; & \node (p0-6) [lastPoint] {}; & \\
  };
  \draw[->] (P0start) -- (p0-2) ;
  \draw (p0-2) -- (p0-4) ;
  \draw[->] (p0-3) |- (p1-4) ;
  \draw (p0-4) -- (p0-5) ;
  \draw[->] (p1-4) -| (p0-5) ;
  \draw[->] (p0-5) -- (p0-6) ;
\end{tikzpicture}

\nonTerminalSection{match\_instruction\_branch}{29}

\ruleSubsection{galgas3InstructionsSyntax}{instruction-match}{63}

\begin{tikzpicture}
  \matrix[column sep=\ruleMatrixColumnSeparation, row sep=\ruleMatrixRowSeparation] {
    & & & & & & & \node (p2-7) [point] {}; & \\
    & & & & & & \node (p1-6) [terminal] {,}; & \\
    \node (P0start) [firstPoint] {}; & & \node (p0-2) [terminal] {case}; & \node (p0-3) [point] {}; & \node (p0-4) [nonterminal] {\nonTerminalSymbol{match\_entry}{28}}; & \node (p0-5) [point] {}; & & & \node (p0-8) [terminal] {:}; & \node (p0-9) [nonterminal] {\nonTerminalSymbol{semantic\_instruction\_list}{14}}; & \node (p0-10) [lastPoint] {}; & \\
  };
  \draw[->] (P0start) -- (p0-2) ;
  \draw[->] (p0-2) -- (p0-4) ;
  \draw[->] (p0-5) |- (p1-6) ;
  \draw[->] (p2-7) -| (p0-3) ;
  \draw[->] (p1-6) -| (p2-7) ;
  \draw[->] (p0-4) -- (p0-8) ;
  \draw[->] (p0-8) -- (p0-9) ;
  \draw[->] (p0-9) -- (p0-10) ;
\end{tikzpicture}

\nonTerminalSection{non\_empty\_output\_expression\_list}{22}

\ruleSubsection{galgas3InstructionsSyntax}{instruction-concat}{112}

\begin{tikzpicture}
  \matrix[column sep=\ruleMatrixColumnSeparation, row sep=\ruleMatrixRowSeparation] {
    & & & & & & \node (p1-6) [point] {}; & \\
    \node (P0start) [firstPoint] {}; & & \node (p0-2) [point] {}; & \node (p0-3) [terminal] {!}; & \node (p0-4) [nonterminal] {\nonTerminalSymbol{expression}{1}}; & \node (p0-5) [point] {}; & & \node (p0-7) [lastPoint] {}; & \\
  };
  \draw[->] (P0start) -- (p0-3) ;
  \draw[->] (p0-3) -- (p0-4) ;
  \draw[->] (p1-6) -| (p0-2) ;
  \draw[->] (p0-5) -| (p1-6) ;
  \draw[->] (p0-4) -- (p0-7) ;
\end{tikzpicture}

\nonTerminalSection{nonterminal\_declaration}{81}

\ruleSubsection{galgas3SyntaxComponentSyntax}{galgas3SyntaxComponentSyntax}{124}

\begin{tikzpicture}
  \matrix[column sep=\ruleMatrixColumnSeparation, row sep=\ruleMatrixRowSeparation] {
    & & & & & & & & & & & \node (p2-11) [point] {}; & \\
    & & & & & & & & \node (p1-8) [terminal] {label}; & \node (p1-9) [terminal] {identifier}; & \node (p1-10) [nonterminal] {\nonTerminalSymbol{formal\_parameter\_list}{11}}; & \\
    \node (P0start) [firstPoint] {}; & & \node (p0-2) [terminal] {rule}; & \node (p0-3) [terminal] {<non\_terminal>}; & \node (p0-4) [nonterminal] {\nonTerminalSymbol{formal\_parameter\_list}{11}}; & \node (p0-5) [point] {}; & \node (p0-6) [point] {}; & \node (p0-7) [point] {}; & & & & & \node (p0-12) [lastPoint] {}; & \\
  };
  \draw[->] (P0start) -- (p0-2) ;
  \draw[->] (p0-2) -- (p0-3) ;
  \draw[->] (p0-3) -- (p0-4) ;
  \draw (p0-4) -- (p0-6) ;
  \draw[->] (p0-7) |- (p1-8) ;
  \draw[->] (p1-8) -- (p1-9) ;
  \draw[->] (p1-9) -- (p1-10) ;
  \draw[->] (p2-11) -| (p0-5) ;
  \draw[->] (p1-10) -| (p2-11) ;
  \draw[->] (p0-6) -- (p0-12) ;
\end{tikzpicture}

\nonTerminalSection{option\_declaration}{77}

\ruleSubsection{galgas3OptionComponentSyntax}{optionCompilation}{56}

\begin{tikzpicture}
  \matrix[column sep=\ruleMatrixColumnSeparation, row sep=\ruleMatrixRowSeparation] {
    & & & & & & & & & & & & & \node (p2-13) [terminal] {uint32}; & \\
    & & & & & & & & & & & \node (p1-11) [terminal] {default}; & \node (p1-12) [point] {}; & \node (p1-13) [terminal] {"string"}; & \node (p1-14) [point] {}; & \\
    \node (P0start) [firstPoint] {}; & & \node (p0-2) [terminal] {@type}; & \node (p0-3) [terminal] {identifier}; & \node (p0-4) [terminal] {:}; & \node (p0-5) [terminal] {'char'}; & \node (p0-6) [terminal] {,}; & \node (p0-7) [terminal] {"string"}; & \node (p0-8) [terminal] {->}; & \node (p0-9) [terminal] {"string"}; & \node (p0-10) [point] {}; & \node (p0-11) [point] {}; & & & & \node (p0-15) [point] {}; & \node (p0-16) [lastPoint] {}; & \\
  };
  \draw[->] (P0start) -- (p0-2) ;
  \draw[->] (p0-2) -- (p0-3) ;
  \draw[->] (p0-3) -- (p0-4) ;
  \draw[->] (p0-4) -- (p0-5) ;
  \draw[->] (p0-5) -- (p0-6) ;
  \draw[->] (p0-6) -- (p0-7) ;
  \draw[->] (p0-7) -- (p0-8) ;
  \draw[->] (p0-8) -- (p0-9) ;
  \draw (p0-9) -- (p0-11) ;
  \draw[->] (p0-10) |- (p1-11) ;
  \draw[->] (p1-11) -- (p1-13) ;
  \draw[->] (p1-12) |- (p2-13) ;
  \draw (p1-13) -- (p1-14) ;
  \draw[->] (p2-13) -| (p1-14) ;
  \draw (p0-11) -- (p0-15) ;
  \draw[->] (p1-14) -| (p0-15) ;
  \draw[->] (p0-15) -- (p0-16) ;
\end{tikzpicture}

\nonTerminalSection{optional\_type}{9}

\ruleSubsection{galgas3ExpressionSyntax}{galgas3ExpressionSyntax}{629}

\begin{tikzpicture}
  \matrix[column sep=\ruleMatrixColumnSeparation, row sep=\ruleMatrixRowSeparation] {
    & & & \node (p1-3) [terminal] {@type}; & \\
    \node (P0start) [firstPoint] {}; & & \node (p0-2) [point] {}; & \node (p0-3) [point] {}; & \node (p0-4) [point] {}; & \node (p0-5) [lastPoint] {}; & \\
  };
  \draw (P0start) -- (p0-3) ;
  \draw[->] (p0-2) |- (p1-3) ;
  \draw (p0-3) -- (p0-4) ;
  \draw[->] (p1-3) -| (p0-4) ;
  \draw[->] (p0-4) -- (p0-5) ;
\end{tikzpicture}

\nonTerminalSection{output\_expression\_list}{0}

\ruleSubsection{galgas3ExpressionSyntax}{galgas3ExpressionSyntax}{26}

\begin{tikzpicture}
  \matrix[column sep=\ruleMatrixColumnSeparation, row sep=\ruleMatrixRowSeparation] {
    & & & & & & & \node (p2-7) [point] {}; & \\
    & & & & & \node (p1-5) [terminal] {!}; & \node (p1-6) [nonterminal] {\nonTerminalSymbol{expression}{1}}; & \\
    \node (P0start) [firstPoint] {}; & & \node (p0-2) [point] {}; & \node (p0-3) [point] {}; & \node (p0-4) [point] {}; & & & & \node (p0-8) [lastPoint] {}; & \\
  };
  \draw (P0start) -- (p0-3) ;
  \draw[->] (p0-4) |- (p1-5) ;
  \draw[->] (p1-5) -- (p1-6) ;
  \draw[->] (p2-7) -| (p0-2) ;
  \draw[->] (p1-6) -| (p2-7) ;
  \draw[->] (p0-3) -- (p0-8) ;
\end{tikzpicture}

\nonTerminalSection{primary}{8}

\ruleSubsection{galgas3ExpressionSyntax}{galgas3ExpressionSyntax}{369}

\begin{tikzpicture}
  \matrix[column sep=\ruleMatrixColumnSeparation, row sep=\ruleMatrixRowSeparation] {
    \node (P0start) [firstPoint] {}; & & \node (p0-2) [terminal] {identifier}; & \node (p0-3) [lastPoint] {}; & \\
  };
  \draw[->] (P0start) -- (p0-2) ;
  \draw[->] (p0-2) -- (p0-3) ;
\end{tikzpicture}

\ruleSubsection{galgas3ExpressionSyntax}{galgas3ExpressionSyntax}{380}

\begin{tikzpicture}
  \matrix[column sep=\ruleMatrixColumnSeparation, row sep=\ruleMatrixRowSeparation] {
    \node (P0start) [firstPoint] {}; & & \node (p0-2) [terminal] {self}; & \node (p0-3) [lastPoint] {}; & \\
  };
  \draw[->] (P0start) -- (p0-2) ;
  \draw[->] (p0-2) -- (p0-3) ;
\end{tikzpicture}

\ruleSubsection{galgas3ExpressionSyntax}{galgas3ExpressionSyntax}{391}

\begin{tikzpicture}
  \matrix[column sep=\ruleMatrixColumnSeparation, row sep=\ruleMatrixRowSeparation] {
    \node (P0start) [firstPoint] {}; & & \node (p0-2) [terminal] {(}; & \node (p0-3) [nonterminal] {\nonTerminalSymbol{expression}{1}}; & \node (p0-4) [terminal] {)}; & \node (p0-5) [lastPoint] {}; & \\
  };
  \draw[->] (P0start) -- (p0-2) ;
  \draw[->] (p0-2) -- (p0-3) ;
  \draw[->] (p0-3) -- (p0-4) ;
  \draw[->] (p0-4) -- (p0-5) ;
\end{tikzpicture}

\ruleSubsection{galgas3ExpressionSyntax}{galgas3ExpressionSyntax}{403}

\begin{tikzpicture}
  \matrix[column sep=\ruleMatrixColumnSeparation, row sep=\ruleMatrixRowSeparation] {
    \node (P0start) [firstPoint] {}; & & \node (p0-2) [terminal] {true}; & \node (p0-3) [lastPoint] {}; & \\
  };
  \draw[->] (P0start) -- (p0-2) ;
  \draw[->] (p0-2) -- (p0-3) ;
\end{tikzpicture}

\ruleSubsection{galgas3ExpressionSyntax}{galgas3ExpressionSyntax}{414}

\begin{tikzpicture}
  \matrix[column sep=\ruleMatrixColumnSeparation, row sep=\ruleMatrixRowSeparation] {
    \node (P0start) [firstPoint] {}; & & \node (p0-2) [terminal] {false}; & \node (p0-3) [lastPoint] {}; & \\
  };
  \draw[->] (P0start) -- (p0-2) ;
  \draw[->] (p0-2) -- (p0-3) ;
\end{tikzpicture}

\ruleSubsection{galgas3ExpressionSyntax}{galgas3ExpressionSyntax}{425}

\begin{tikzpicture}
  \matrix[column sep=\ruleMatrixColumnSeparation, row sep=\ruleMatrixRowSeparation] {
    & & & & & \node (p1-5) [point] {}; & \\
    \node (P0start) [firstPoint] {}; & & \node (p0-2) [point] {}; & \node (p0-3) [terminal] {"string"}; & \node (p0-4) [point] {}; & & \node (p0-6) [lastPoint] {}; & \\
  };
  \draw[->] (P0start) -- (p0-3) ;
  \draw[->] (p1-5) -| (p0-2) ;
  \draw[->] (p0-4) -| (p1-5) ;
  \draw[->] (p0-3) -- (p0-6) ;
\end{tikzpicture}

\ruleSubsection{galgas3ExpressionSyntax}{galgas3ExpressionSyntax}{441}

\begin{tikzpicture}
  \matrix[column sep=\ruleMatrixColumnSeparation, row sep=\ruleMatrixRowSeparation] {
    \node (P0start) [firstPoint] {}; & & \node (p0-2) [terminal] {uint32}; & \node (p0-3) [lastPoint] {}; & \\
  };
  \draw[->] (P0start) -- (p0-2) ;
  \draw[->] (p0-2) -- (p0-3) ;
\end{tikzpicture}

\ruleSubsection{galgas3ExpressionSyntax}{galgas3ExpressionSyntax}{452}

\begin{tikzpicture}
  \matrix[column sep=\ruleMatrixColumnSeparation, row sep=\ruleMatrixRowSeparation] {
    \node (P0start) [firstPoint] {}; & & \node (p0-2) [terminal] {sint32\_S}; & \node (p0-3) [lastPoint] {}; & \\
  };
  \draw[->] (P0start) -- (p0-2) ;
  \draw[->] (p0-2) -- (p0-3) ;
\end{tikzpicture}

\ruleSubsection{galgas3ExpressionSyntax}{galgas3ExpressionSyntax}{463}

\begin{tikzpicture}
  \matrix[column sep=\ruleMatrixColumnSeparation, row sep=\ruleMatrixRowSeparation] {
    \node (P0start) [firstPoint] {}; & & \node (p0-2) [terminal] {uint64\_L}; & \node (p0-3) [lastPoint] {}; & \\
  };
  \draw[->] (P0start) -- (p0-2) ;
  \draw[->] (p0-2) -- (p0-3) ;
\end{tikzpicture}

\ruleSubsection{galgas3ExpressionSyntax}{galgas3ExpressionSyntax}{474}

\begin{tikzpicture}
  \matrix[column sep=\ruleMatrixColumnSeparation, row sep=\ruleMatrixRowSeparation] {
    \node (P0start) [firstPoint] {}; & & \node (p0-2) [terminal] {sint64\_LS}; & \node (p0-3) [lastPoint] {}; & \\
  };
  \draw[->] (P0start) -- (p0-2) ;
  \draw[->] (p0-2) -- (p0-3) ;
\end{tikzpicture}

\ruleSubsection{galgas3ExpressionSyntax}{galgas3ExpressionSyntax}{485}

\begin{tikzpicture}
  \matrix[column sep=\ruleMatrixColumnSeparation, row sep=\ruleMatrixRowSeparation] {
    \node (P0start) [firstPoint] {}; & & \node (p0-2) [terminal] {bigint\_G}; & \node (p0-3) [lastPoint] {}; & \\
  };
  \draw[->] (P0start) -- (p0-2) ;
  \draw[->] (p0-2) -- (p0-3) ;
\end{tikzpicture}

\ruleSubsection{galgas3ExpressionSyntax}{galgas3ExpressionSyntax}{496}

\begin{tikzpicture}
  \matrix[column sep=\ruleMatrixColumnSeparation, row sep=\ruleMatrixRowSeparation] {
    \node (P0start) [firstPoint] {}; & & \node (p0-2) [terminal] {'char'}; & \node (p0-3) [lastPoint] {}; & \\
  };
  \draw[->] (P0start) -- (p0-2) ;
  \draw[->] (p0-2) -- (p0-3) ;
\end{tikzpicture}

\ruleSubsection{galgas3ExpressionSyntax}{galgas3ExpressionSyntax}{507}

\begin{tikzpicture}
  \matrix[column sep=\ruleMatrixColumnSeparation, row sep=\ruleMatrixRowSeparation] {
    \node (P0start) [firstPoint] {}; & & \node (p0-2) [terminal] {double.xxx}; & \node (p0-3) [lastPoint] {}; & \\
  };
  \draw[->] (P0start) -- (p0-2) ;
  \draw[->] (p0-2) -- (p0-3) ;
\end{tikzpicture}

\ruleSubsection{galgas3ExpressionSyntax}{galgas3ExpressionSyntax}{518}

\begin{tikzpicture}
  \matrix[column sep=\ruleMatrixColumnSeparation, row sep=\ruleMatrixRowSeparation] {
    \node (P0start) [firstPoint] {}; & & \node (p6-2) [terminal] {if}; & \\
    & & \node (p5-2) [nonterminal] {\nonTerminalSymbol{expression}{1}}; & \\
    & & \node (p4-2) [terminal] {then}; & \\
    & & \node (p3-2) [nonterminal] {\nonTerminalSymbol{expression}{1}}; & \\
    & & \node (p2-2) [terminal] {else}; & \\
    & & \node (p1-2) [nonterminal] {\nonTerminalSymbol{expression}{1}}; & \\
    & & \node (p0-2) [terminal] {end}; & \node (p0-3) [lastPoint] {}; & \\
  };
  \draw[->] (P0start) -- (p6-2) ;
  \draw[->] (p6-2) -- (p5-2) ;
  \draw[->] (p5-2) -- (p4-2) ;
  \draw[->] (p4-2) -- (p3-2) ;
  \draw[->] (p3-2) -- (p2-2) ;
  \draw[->] (p2-2) -- (p1-2) ;
  \draw[->] (p1-2) -- (p0-2) ;
  \draw[->] (p0-2) -- (p0-3) ;
\end{tikzpicture}

\ruleSubsection{galgas3ExpressionSyntax}{galgas3ExpressionSyntax}{545}

\begin{tikzpicture}
  \matrix[column sep=\ruleMatrixColumnSeparation, row sep=\ruleMatrixRowSeparation] {
    \node (P0start) [firstPoint] {}; & & \node (p4-2) [terminal] {[}; & \\
    & & \node (p3-2) [nonterminal] {\nonTerminalSymbol{expression}{1}}; & \\
    & & \node (p2-2) [terminal] {identifier}; & \\
    & & \node (p1-2) [nonterminal] {\nonTerminalSymbol{output\_expression\_list}{0}}; & \\
    & & \node (p0-2) [terminal] {]}; & \node (p0-3) [lastPoint] {}; & \\
  };
  \draw[->] (P0start) -- (p4-2) ;
  \draw[->] (p4-2) -- (p3-2) ;
  \draw[->] (p3-2) -- (p2-2) ;
  \draw[->] (p2-2) -- (p1-2) ;
  \draw[->] (p1-2) -- (p0-2) ;
  \draw[->] (p0-2) -- (p0-3) ;
\end{tikzpicture}

\ruleSubsection{galgas3ExpressionSyntax}{galgas3ExpressionSyntax}{559}

\begin{tikzpicture}
  \matrix[column sep=\ruleMatrixColumnSeparation, row sep=\ruleMatrixRowSeparation] {
    \node (P0start) [firstPoint] {}; & & \node (p6-2) [terminal] {[}; & \\
    & & \node (p5-2) [terminal] {option}; & \\
    & & \node (p4-2) [terminal] {identifier}; & \\
    & & \node (p3-2) [terminal] {.}; & \\
    & & \node (p2-2) [terminal] {identifier}; & \\
    & & \node (p1-2) [terminal] {identifier}; & \\
    & & \node (p0-2) [terminal] {]}; & \node (p0-3) [lastPoint] {}; & \\
  };
  \draw[->] (P0start) -- (p6-2) ;
  \draw[->] (p6-2) -- (p5-2) ;
  \draw[->] (p5-2) -- (p4-2) ;
  \draw[->] (p4-2) -- (p3-2) ;
  \draw[->] (p3-2) -- (p2-2) ;
  \draw[->] (p2-2) -- (p1-2) ;
  \draw[->] (p1-2) -- (p0-2) ;
  \draw[->] (p0-2) -- (p0-3) ;
\end{tikzpicture}

\ruleSubsection{galgas3ExpressionSyntax}{galgas3ExpressionSyntax}{572}

\begin{tikzpicture}
  \matrix[column sep=\ruleMatrixColumnSeparation, row sep=\ruleMatrixRowSeparation] {
    \node (P0start) [firstPoint] {}; & & \node (p5-2) [terminal] {[}; & \\
    & & \node (p4-2) [terminal] {option}; & \\
    & & \node (p3-2) [terminal] {.}; & \\
    & & \node (p2-2) [terminal] {identifier}; & \\
    & & \node (p1-2) [terminal] {identifier}; & \\
    & & \node (p0-2) [terminal] {]}; & \node (p0-3) [lastPoint] {}; & \\
  };
  \draw[->] (P0start) -- (p5-2) ;
  \draw[->] (p5-2) -- (p4-2) ;
  \draw[->] (p4-2) -- (p3-2) ;
  \draw[->] (p3-2) -- (p2-2) ;
  \draw[->] (p2-2) -- (p1-2) ;
  \draw[->] (p1-2) -- (p0-2) ;
  \draw[->] (p0-2) -- (p0-3) ;
\end{tikzpicture}

\ruleSubsection{galgas3ExpressionSyntax}{galgas3ExpressionSyntax}{584}

\begin{tikzpicture}
  \matrix[column sep=\ruleMatrixColumnSeparation, row sep=\ruleMatrixRowSeparation] {
    \node (P0start) [firstPoint] {}; & & \node (p5-2) [terminal] {[}; & \\
    & & \node (p4-2) [terminal] {lexique}; & \\
    & & \node (p3-2) [terminal] {identifier}; & \\
    & & \node (p2-2) [terminal] {:}; & \\
    & & \node (p1-2) [terminal] {identifier}; & \\
    & & \node (p0-2) [terminal] {]}; & \node (p0-3) [lastPoint] {}; & \\
  };
  \draw[->] (P0start) -- (p5-2) ;
  \draw[->] (p5-2) -- (p4-2) ;
  \draw[->] (p4-2) -- (p3-2) ;
  \draw[->] (p3-2) -- (p2-2) ;
  \draw[->] (p2-2) -- (p1-2) ;
  \draw[->] (p1-2) -- (p0-2) ;
  \draw[->] (p0-2) -- (p0-3) ;
\end{tikzpicture}

\ruleSubsection{galgas3ExpressionSyntax}{galgas3ExpressionSyntax}{596}

\begin{tikzpicture}
  \matrix[column sep=\ruleMatrixColumnSeparation, row sep=\ruleMatrixRowSeparation] {
    & & & & & & & & \node (p2-8) [terminal] {identifier}; & \node (p2-9) [nonterminal] {\nonTerminalSymbol{output\_expression\_list}{0}}; & \\
    & & & & & & \node (p1-6) [terminal] {.}; & \node (p1-7) [point] {}; & \node (p1-8) [terminal] {"string"}; & & \node (p1-10) [point] {}; & \\
    \node (P0start) [firstPoint] {}; & & \node (p0-2) [terminal] {[}; & \node (p0-3) [terminal] {filewrapper}; & \node (p0-4) [terminal] {identifier}; & \node (p0-5) [point] {}; & \node (p0-6) [point] {}; & & & & & \node (p0-11) [point] {}; & \node (p0-12) [terminal] {]}; & \node (p0-13) [lastPoint] {}; & \\
  };
  \draw[->] (P0start) -- (p0-2) ;
  \draw[->] (p0-2) -- (p0-3) ;
  \draw[->] (p0-3) -- (p0-4) ;
  \draw (p0-4) -- (p0-6) ;
  \draw[->] (p0-5) |- (p1-6) ;
  \draw[->] (p1-6) -- (p1-8) ;
  \draw[->] (p1-7) |- (p2-8) ;
  \draw[->] (p2-8) -- (p2-9) ;
  \draw (p1-8) -- (p1-10) ;
  \draw[->] (p2-9) -| (p1-10) ;
  \draw (p0-6) -- (p0-11) ;
  \draw[->] (p1-10) -| (p0-11) ;
  \draw[->] (p0-11) -- (p0-12) ;
  \draw[->] (p0-12) -- (p0-13) ;
\end{tikzpicture}

\ruleSubsection{galgas3ExpressionSyntax}{galgas3ExpressionSyntax}{639}

\begin{tikzpicture}
  \matrix[column sep=\ruleMatrixColumnSeparation, row sep=\ruleMatrixRowSeparation] {
    & & & & & & \node (p1-6) [terminal] {\{}; & \node (p1-7) [nonterminal] {\nonTerminalSymbol{output\_expression\_list}{0}}; & \node (p1-8) [terminal] {\}}; & \\
    \node (P0start) [firstPoint] {}; & & \node (p0-2) [nonterminal] {\nonTerminalSymbol{optional\_type}{9}}; & \node (p0-3) [terminal] {.}; & \node (p0-4) [terminal] {identifier}; & \node (p0-5) [point] {}; & \node (p0-6) [point] {}; & & & \node (p0-9) [point] {}; & \node (p0-10) [lastPoint] {}; & \\
  };
  \draw[->] (P0start) -- (p0-2) ;
  \draw[->] (p0-2) -- (p0-3) ;
  \draw[->] (p0-3) -- (p0-4) ;
  \draw (p0-4) -- (p0-6) ;
  \draw[->] (p0-5) |- (p1-6) ;
  \draw[->] (p1-6) -- (p1-7) ;
  \draw[->] (p1-7) -- (p1-8) ;
  \draw (p0-6) -- (p0-9) ;
  \draw[->] (p1-8) -| (p0-9) ;
  \draw[->] (p0-9) -- (p0-10) ;
\end{tikzpicture}

\ruleSubsection{galgas3ExpressionSyntax}{galgas3ExpressionSyntax}{662}

\begin{tikzpicture}
  \matrix[column sep=\ruleMatrixColumnSeparation, row sep=\ruleMatrixRowSeparation] {
    \node (P0start) [firstPoint] {}; & & \node (p0-2) [nonterminal] {\nonTerminalSymbol{optional\_type}{9}}; & \node (p0-3) [terminal] {.}; & \node (p0-4) [terminal] {default}; & \node (p0-5) [lastPoint] {}; & \\
  };
  \draw[->] (P0start) -- (p0-2) ;
  \draw[->] (p0-2) -- (p0-3) ;
  \draw[->] (p0-3) -- (p0-4) ;
  \draw[->] (p0-4) -- (p0-5) ;
\end{tikzpicture}

\ruleSubsection{galgas3ExpressionSyntax}{galgas3ExpressionSyntax}{673}

\begin{tikzpicture}
  \matrix[column sep=\ruleMatrixColumnSeparation, row sep=\ruleMatrixRowSeparation] {
    & & & & & & & & & \node (p3-9) [point] {}; & \\
    & & & & & & & & \node (p2-8) [terminal] {,}; & \\
    & & & & & \node (p1-5) [point] {}; & \node (p1-6) [nonterminal] {\nonTerminalSymbol{collection\_value\_element}{10}}; & \node (p1-7) [point] {}; & \\
    \node (P0start) [firstPoint] {}; & & \node (p0-2) [nonterminal] {\nonTerminalSymbol{optional\_type}{9}}; & \node (p0-3) [terminal] {\{}; & \node (p0-4) [point] {}; & \node (p0-5) [point] {}; & & & & & \node (p0-10) [point] {}; & \node (p0-11) [terminal] {\}}; & \node (p0-12) [lastPoint] {}; & \\
  };
  \draw[->] (P0start) -- (p0-2) ;
  \draw[->] (p0-2) -- (p0-3) ;
  \draw (p0-3) -- (p0-5) ;
  \draw[->] (p0-4) |- (p1-6) ;
  \draw[->] (p1-7) |- (p2-8) ;
  \draw[->] (p3-9) -| (p1-5) ;
  \draw[->] (p2-8) -| (p3-9) ;
  \draw (p0-5) -- (p0-10) ;
  \draw[->] (p1-6) -| (p0-10) ;
  \draw[->] (p0-10) -- (p0-11) ;
  \draw[->] (p0-11) -- (p0-12) ;
\end{tikzpicture}

\ruleSubsection{galgas3ExpressionSyntax}{galgas3ExpressionSyntax}{724}

\begin{tikzpicture}
  \matrix[column sep=\ruleMatrixColumnSeparation, row sep=\ruleMatrixRowSeparation] {
    \node (P0start) [firstPoint] {}; & & \node (p0-2) [terminal] {identifier}; & \node (p0-3) [terminal] {(}; & \node (p0-4) [nonterminal] {\nonTerminalSymbol{output\_expression\_list}{0}}; & \node (p0-5) [terminal] {)}; & \node (p0-6) [lastPoint] {}; & \\
  };
  \draw[->] (P0start) -- (p0-2) ;
  \draw[->] (p0-2) -- (p0-3) ;
  \draw[->] (p0-3) -- (p0-4) ;
  \draw[->] (p0-4) -- (p0-5) ;
  \draw[->] (p0-5) -- (p0-6) ;
\end{tikzpicture}

\ruleSubsection{galgas3ExpressionSyntax}{galgas3ExpressionSyntax}{743}

\begin{tikzpicture}
  \matrix[column sep=\ruleMatrixColumnSeparation, row sep=\ruleMatrixRowSeparation] {
    \node (P0start) [firstPoint] {}; & & \node (p0-2) [terminal] {`}; & \node (p0-3) [terminal] {@type}; & \node (p0-4) [lastPoint] {}; & \\
  };
  \draw[->] (P0start) -- (p0-2) ;
  \draw[->] (p0-2) -- (p0-3) ;
  \draw[->] (p0-3) -- (p0-4) ;
\end{tikzpicture}

\nonTerminalSection{property\_declaration}{33}

\ruleSubsection{galgas3DeclarationsSyntax}{galgas3DeclarationsSyntax}{58}

\begin{tikzpicture}
  \matrix[column sep=\ruleMatrixColumnSeparation, row sep=\ruleMatrixRowSeparation] {
    & & & & & & & & \node (p2-8) [point] {}; & \\
    & & & & & & & \node (p1-7) [terminal] {\verb=%=attribute}; & \\
    \node (P0start) [firstPoint] {}; & & \node (p0-2) [terminal] {@type}; & \node (p0-3) [terminal] {identifier}; & \node (p0-4) [point] {}; & \node (p0-5) [point] {}; & \node (p0-6) [point] {}; & & & \node (p0-9) [lastPoint] {}; & \\
  };
  \draw[->] (P0start) -- (p0-2) ;
  \draw[->] (p0-2) -- (p0-3) ;
  \draw (p0-3) -- (p0-5) ;
  \draw[->] (p0-6) |- (p1-7) ;
  \draw[->] (p2-8) -| (p0-4) ;
  \draw[->] (p1-7) -| (p2-8) ;
  \draw[->] (p0-5) -- (p0-9) ;
\end{tikzpicture}

\nonTerminalSection{relation\_factor}{4}

\ruleSubsection{galgas3ExpressionSyntax}{galgas3ExpressionSyntax}{150}

\begin{tikzpicture}
  \matrix[column sep=\ruleMatrixColumnSeparation, row sep=\ruleMatrixRowSeparation] {
    & & & & & & & & \node (p7-8) [point] {}; & \\
    & & & & & & \node (p6-6) [terminal] {<}; & \node (p6-7) [nonterminal] {\nonTerminalSymbol{simple\_expression}{5}}; & \\
    & & & & & & \node (p5-6) [terminal] {>}; & \node (p5-7) [nonterminal] {\nonTerminalSymbol{simple\_expression}{5}}; & \\
    & & & & & & \node (p4-6) [terminal] {>=}; & \node (p4-7) [nonterminal] {\nonTerminalSymbol{simple\_expression}{5}}; & \\
    & & & & & & \node (p3-6) [terminal] {<=}; & \node (p3-7) [nonterminal] {\nonTerminalSymbol{simple\_expression}{5}}; & \\
    & & & & & & \node (p2-6) [terminal] {!=}; & \node (p2-7) [nonterminal] {\nonTerminalSymbol{simple\_expression}{5}}; & \\
    & & & & & & \node (p1-6) [terminal] {==}; & \node (p1-7) [nonterminal] {\nonTerminalSymbol{simple\_expression}{5}}; & \\
    \node (P0start) [firstPoint] {}; & & \node (p0-2) [nonterminal] {\nonTerminalSymbol{simple\_expression}{5}}; & \node (p0-3) [point] {}; & \node (p0-4) [point] {}; & \node (p0-5) [point] {}; & & & & \node (p0-9) [lastPoint] {}; & \\
  };
  \draw[->] (P0start) -- (p0-2) ;
  \draw (p0-2) -- (p0-4) ;
  \draw[->] (p0-5) |- (p1-6) ;
  \draw[->] (p1-6) -- (p1-7) ;
  \draw[->] (p0-5) |- (p2-6) ;
  \draw[->] (p2-6) -- (p2-7) ;
  \draw[->] (p0-5) |- (p3-6) ;
  \draw[->] (p3-6) -- (p3-7) ;
  \draw[->] (p0-5) |- (p4-6) ;
  \draw[->] (p4-6) -- (p4-7) ;
  \draw[->] (p0-5) |- (p5-6) ;
  \draw[->] (p5-6) -- (p5-7) ;
  \draw[->] (p0-5) |- (p6-6) ;
  \draw[->] (p6-6) -- (p6-7) ;
  \draw[->] (p7-8) -| (p0-3) ;
  \draw[->] (p1-7) -| (p7-8) ;
  \draw[->] (p2-7) -| (p7-8) ;
  \draw[->] (p3-7) -| (p7-8) ;
  \draw[->] (p4-7) -| (p7-8) ;
  \draw[->] (p5-7) -| (p7-8) ;
  \draw[->] (p6-7) -| (p7-8) ;
  \draw[->] (p0-4) -- (p0-9) ;
\end{tikzpicture}

\nonTerminalSection{relation\_term}{3}

\ruleSubsection{galgas3ExpressionSyntax}{galgas3ExpressionSyntax}{126}

\begin{tikzpicture}
  \matrix[column sep=\ruleMatrixColumnSeparation, row sep=\ruleMatrixRowSeparation] {
    & & & & & & & & \node (p3-8) [point] {}; & \\
    & & & & & & \node (p2-6) [terminal] {\&\&}; & \node (p2-7) [nonterminal] {\nonTerminalSymbol{relation\_factor}{4}}; & \\
    & & & & & & \node (p1-6) [terminal] {\&}; & \node (p1-7) [nonterminal] {\nonTerminalSymbol{relation\_factor}{4}}; & \\
    \node (P0start) [firstPoint] {}; & & \node (p0-2) [nonterminal] {\nonTerminalSymbol{relation\_factor}{4}}; & \node (p0-3) [point] {}; & \node (p0-4) [point] {}; & \node (p0-5) [point] {}; & & & & \node (p0-9) [lastPoint] {}; & \\
  };
  \draw[->] (P0start) -- (p0-2) ;
  \draw (p0-2) -- (p0-4) ;
  \draw[->] (p0-5) |- (p1-6) ;
  \draw[->] (p1-6) -- (p1-7) ;
  \draw[->] (p0-5) |- (p2-6) ;
  \draw[->] (p2-6) -- (p2-7) ;
  \draw[->] (p3-8) -| (p0-3) ;
  \draw[->] (p1-7) -| (p3-8) ;
  \draw[->] (p2-7) -| (p3-8) ;
  \draw[->] (p0-4) -- (p0-9) ;
\end{tikzpicture}

\nonTerminalSection{remove\_declaration}{44}

\ruleSubsection{galgas3DeclarationsSyntax}{type-map}{123}

\begin{tikzpicture}
  \matrix[column sep=\ruleMatrixColumnSeparation, row sep=\ruleMatrixRowSeparation] {
    \node (P0start) [firstPoint] {}; & & \node (p4-2) [terminal] {remove}; & \\
    & & \node (p3-2) [terminal] {identifier}; & \\
    & & \node (p2-2) [terminal] {error}; & \\
    & & \node (p1-2) [terminal] {message}; & \\
    & & \node (p0-2) [terminal] {"string"}; & \node (p0-3) [lastPoint] {}; & \\
  };
  \draw[->] (P0start) -- (p4-2) ;
  \draw[->] (p4-2) -- (p3-2) ;
  \draw[->] (p3-2) -- (p2-2) ;
  \draw[->] (p2-2) -- (p1-2) ;
  \draw[->] (p1-2) -- (p0-2) ;
  \draw[->] (p0-2) -- (p0-3) ;
\end{tikzpicture}

\nonTerminalSection{repeat\_while\_branch}{61}

\ruleSubsection{galgas3LexiqueComponentSyntax}{galgas3LexiqueComponentSyntax}{252}

\begin{tikzpicture}
  \matrix[column sep=\ruleMatrixColumnSeparation, row sep=\ruleMatrixRowSeparation] {
    & & & & & & & & & \node (p2-9) [point] {}; & \\
    & & & & & & & & \node (p1-8) [nonterminal] {\nonTerminalSymbol{lexical\_instruction}{59}}; & \\
    \node (P0start) [firstPoint] {}; & & \node (p0-2) [terminal] {while}; & \node (p0-3) [nonterminal] {\nonTerminalSymbol{lexical\_expression}{63}}; & \node (p0-4) [terminal] {:}; & \node (p0-5) [point] {}; & \node (p0-6) [point] {}; & \node (p0-7) [point] {}; & & & \node (p0-10) [lastPoint] {}; & \\
  };
  \draw[->] (P0start) -- (p0-2) ;
  \draw[->] (p0-2) -- (p0-3) ;
  \draw[->] (p0-3) -- (p0-4) ;
  \draw (p0-4) -- (p0-6) ;
  \draw[->] (p0-7) |- (p1-8) ;
  \draw[->] (p2-9) -| (p0-5) ;
  \draw[->] (p1-8) -| (p2-9) ;
  \draw[->] (p0-6) -- (p0-10) ;
\end{tikzpicture}

\nonTerminalSection{search\_declaration}{43}

\ruleSubsection{galgas3DeclarationsSyntax}{type-map}{101}

\begin{tikzpicture}
  \matrix[column sep=\ruleMatrixColumnSeparation, row sep=\ruleMatrixRowSeparation] {
    & & & & & \node (p1-5) [terminal] {\verb=%=attribute}; & \\
    \node (P0start) [firstPoint] {}; & & \node (p0-2) [terminal] {search}; & \node (p0-3) [terminal] {identifier}; & \node (p0-4) [point] {}; & \node (p0-5) [point] {}; & \node (p0-6) [point] {}; & \node (p0-7) [terminal] {error}; & \node (p0-8) [terminal] {message}; & \node (p0-9) [terminal] {"string"}; & \node (p0-10) [lastPoint] {}; & \\
  };
  \draw[->] (P0start) -- (p0-2) ;
  \draw[->] (p0-2) -- (p0-3) ;
  \draw (p0-3) -- (p0-5) ;
  \draw[->] (p0-4) |- (p1-5) ;
  \draw (p0-5) -- (p0-6) ;
  \draw[->] (p1-5) -| (p0-6) ;
  \draw[->] (p0-6) -- (p0-7) ;
  \draw[->] (p0-7) -- (p0-8) ;
  \draw[->] (p0-8) -- (p0-9) ;
  \draw[->] (p0-9) -- (p0-10) ;
\end{tikzpicture}

\nonTerminalSection{semantic\_instruction}{17}

\ruleSubsection{galgas3InstructionsSyntax}{galgas3InstructionsSyntax}{227}

\begin{tikzpicture}
  \matrix[column sep=\ruleMatrixColumnSeparation, row sep=\ruleMatrixRowSeparation] {
    \node (P0start) [firstPoint] {}; & & \node (p4-2) [terminal] {var}; & \\
    & & \node (p3-2) [terminal] {@type}; & \\
    & & \node (p2-2) [terminal] {identifier}; & \\
    & & \node (p1-2) [terminal] {=}; & \\
    & & \node (p0-2) [nonterminal] {\nonTerminalSymbol{expression}{1}}; & \node (p0-3) [lastPoint] {}; & \\
  };
  \draw[->] (P0start) -- (p4-2) ;
  \draw[->] (p4-2) -- (p3-2) ;
  \draw[->] (p3-2) -- (p2-2) ;
  \draw[->] (p2-2) -- (p1-2) ;
  \draw[->] (p1-2) -- (p0-2) ;
  \draw[->] (p0-2) -- (p0-3) ;
\end{tikzpicture}

\ruleSubsection{galgas3InstructionsSyntax}{galgas3InstructionsSyntax}{244}

\begin{tikzpicture}
  \matrix[column sep=\ruleMatrixColumnSeparation, row sep=\ruleMatrixRowSeparation] {
    \node (P0start) [firstPoint] {}; & & \node (p0-2) [terminal] {@type}; & \node (p0-3) [terminal] {identifier}; & \node (p0-4) [terminal] {=}; & \node (p0-5) [nonterminal] {\nonTerminalSymbol{expression}{1}}; & \node (p0-6) [lastPoint] {}; & \\
  };
  \draw[->] (P0start) -- (p0-2) ;
  \draw[->] (p0-2) -- (p0-3) ;
  \draw[->] (p0-3) -- (p0-4) ;
  \draw[->] (p0-4) -- (p0-5) ;
  \draw[->] (p0-5) -- (p0-6) ;
\end{tikzpicture}

\ruleSubsection{galgas3InstructionsSyntax}{instruction-assignment}{19}

\begin{tikzpicture}
  \matrix[column sep=\ruleMatrixColumnSeparation, row sep=\ruleMatrixRowSeparation] {
    & & & & & & & & \node (p2-8) [point] {}; & \\
    & & & & & & \node (p1-6) [terminal] {.}; & \node (p1-7) [terminal] {identifier}; & \\
    \node (P0start) [firstPoint] {}; & & \node (p0-2) [terminal] {identifier}; & \node (p0-3) [point] {}; & \node (p0-4) [point] {}; & \node (p0-5) [point] {}; & & & & \node (p0-9) [terminal] {=}; & \node (p0-10) [nonterminal] {\nonTerminalSymbol{expression}{1}}; & \node (p0-11) [lastPoint] {}; & \\
  };
  \draw[->] (P0start) -- (p0-2) ;
  \draw (p0-2) -- (p0-4) ;
  \draw[->] (p0-5) |- (p1-6) ;
  \draw[->] (p1-6) -- (p1-7) ;
  \draw[->] (p2-8) -| (p0-3) ;
  \draw[->] (p1-7) -| (p2-8) ;
  \draw[->] (p0-4) -- (p0-9) ;
  \draw[->] (p0-9) -- (p0-10) ;
  \draw[->] (p0-10) -- (p0-11) ;
\end{tikzpicture}

\ruleSubsection{galgas3InstructionsSyntax}{instruction-cast}{86}

\begin{tikzpicture}
  \matrix[column sep=\ruleMatrixColumnSeparation, row sep=\ruleMatrixRowSeparation] {
    & & & & & & & \node (p1-7) [point] {}; & & & & \node (p1-11) [terminal] {\verb=%=attribute}; & \\
    \node (P0start) [firstPoint] {}; & & \node (p0-2) [terminal] {cast}; & \node (p0-3) [nonterminal] {\nonTerminalSymbol{expression}{1}}; & \node (p0-4) [point] {}; & \node (p0-5) [nonterminal] {\nonTerminalSymbol{cast\_instruction\_branch}{20}}; & \node (p0-6) [point] {}; & & \node (p0-8) [nonterminal] {\nonTerminalSymbol{cast\_else\_or\_default}{21}}; & \node (p0-9) [terminal] {end}; & \node (p0-10) [point] {}; & \node (p0-11) [point] {}; & \node (p0-12) [point] {}; & \node (p0-13) [lastPoint] {}; & \\
  };
  \draw[->] (P0start) -- (p0-2) ;
  \draw[->] (p0-2) -- (p0-3) ;
  \draw[->] (p0-3) -- (p0-5) ;
  \draw[->] (p1-7) -| (p0-4) ;
  \draw[->] (p0-6) -| (p1-7) ;
  \draw[->] (p0-5) -- (p0-8) ;
  \draw[->] (p0-8) -- (p0-9) ;
  \draw (p0-9) -- (p0-11) ;
  \draw[->] (p0-10) |- (p1-11) ;
  \draw (p0-11) -- (p0-12) ;
  \draw[->] (p1-11) -| (p0-12) ;
  \draw[->] (p0-12) -- (p0-13) ;
\end{tikzpicture}

\ruleSubsection{galgas3InstructionsSyntax}{instruction-concat}{51}

\begin{tikzpicture}
  \matrix[column sep=\ruleMatrixColumnSeparation, row sep=\ruleMatrixRowSeparation] {
    & & & & & & & & & & \node (p4-10) [terminal] {/=}; & \node (p4-11) [nonterminal] {\nonTerminalSymbol{expression}{1}}; & \\
    & & & & & & & & & & \node (p3-10) [terminal] {*=}; & \node (p3-11) [nonterminal] {\nonTerminalSymbol{expression}{1}}; & \\
    & & & & & & & & \node (p2-8) [point] {}; & & & & \node (p2-12) [nonterminal] {\nonTerminalSymbol{expression}{1}}; & \\
    & & & & & & \node (p1-6) [terminal] {.}; & \node (p1-7) [terminal] {identifier}; & & & \node (p1-10) [terminal] {+=}; & \node (p1-11) [point] {}; & \node (p1-12) [nonterminal] {\nonTerminalSymbol{non\_empty\_output\_expression\_list}{22}}; & \node (p1-13) [point] {}; & \\
    \node (P0start) [firstPoint] {}; & & \node (p0-2) [terminal] {identifier}; & \node (p0-3) [point] {}; & \node (p0-4) [point] {}; & \node (p0-5) [point] {}; & & & & \node (p0-9) [point] {}; & \node (p0-10) [terminal] {-=}; & \node (p0-11) [nonterminal] {\nonTerminalSymbol{expression}{1}}; & & & \node (p0-14) [point] {}; & \node (p0-15) [lastPoint] {}; & \\
  };
  \draw[->] (P0start) -- (p0-2) ;
  \draw (p0-2) -- (p0-4) ;
  \draw[->] (p0-5) |- (p1-6) ;
  \draw[->] (p1-6) -- (p1-7) ;
  \draw[->] (p2-8) -| (p0-3) ;
  \draw[->] (p1-7) -| (p2-8) ;
  \draw[->] (p0-4) -- (p0-10) ;
  \draw[->] (p0-10) -- (p0-11) ;
  \draw[->] (p0-9) |- (p1-10) ;
  \draw[->] (p1-10) -- (p1-12) ;
  \draw[->] (p1-11) |- (p2-12) ;
  \draw (p1-12) -- (p1-13) ;
  \draw[->] (p2-12) -| (p1-13) ;
  \draw[->] (p0-9) |- (p3-10) ;
  \draw[->] (p3-10) -- (p3-11) ;
  \draw[->] (p0-9) |- (p4-10) ;
  \draw[->] (p4-10) -- (p4-11) ;
  \draw (p0-11) -- (p0-14) ;
  \draw[->] (p1-13) -| (p0-14) ;
  \draw[->] (p3-11) -| (p0-14) ;
  \draw[->] (p4-11) -| (p0-14) ;
  \draw[->] (p0-14) -- (p0-15) ;
\end{tikzpicture}

\ruleSubsection{galgas3InstructionsSyntax}{instruction-drop}{17}

\begin{tikzpicture}
  \matrix[column sep=\ruleMatrixColumnSeparation, row sep=\ruleMatrixRowSeparation] {
    & & & & & & & \node (p2-7) [point] {}; & \\
    & & & & & & \node (p1-6) [terminal] {,}; & \\
    \node (P0start) [firstPoint] {}; & & \node (p0-2) [terminal] {drop}; & \node (p0-3) [point] {}; & \node (p0-4) [terminal] {identifier}; & \node (p0-5) [point] {}; & & & \node (p0-8) [lastPoint] {}; & \\
  };
  \draw[->] (P0start) -- (p0-2) ;
  \draw[->] (p0-2) -- (p0-4) ;
  \draw[->] (p0-5) |- (p1-6) ;
  \draw[->] (p2-7) -| (p0-3) ;
  \draw[->] (p1-6) -| (p2-7) ;
  \draw[->] (p0-4) -- (p0-8) ;
\end{tikzpicture}

\ruleSubsection{galgas3InstructionsSyntax}{instruction-error}{73}

\begin{tikzpicture}
  \matrix[column sep=\ruleMatrixColumnSeparation, row sep=\ruleMatrixRowSeparation] {
    & & & & & & & & & & & & \node (p3-12) [point] {}; & \\
    & & & & & & & & & & & \node (p2-11) [terminal] {,}; & \\
    & & & & & & & \node (p1-7) [terminal] {:}; & \node (p1-8) [point] {}; & \node (p1-9) [terminal] {identifier}; & \node (p1-10) [point] {}; & \\
    \node (P0start) [firstPoint] {}; & & \node (p0-2) [terminal] {error}; & \node (p0-3) [nonterminal] {\nonTerminalSymbol{expression}{1}}; & \node (p0-4) [terminal] {:}; & \node (p0-5) [nonterminal] {\nonTerminalSymbol{expression}{1}}; & \node (p0-6) [point] {}; & \node (p0-7) [point] {}; & & & & & & \node (p0-13) [point] {}; & \node (p0-14) [nonterminal] {\nonTerminalSymbol{issue\_fixit}{23}}; & \node (p0-15) [lastPoint] {}; & \\
  };
  \draw[->] (P0start) -- (p0-2) ;
  \draw[->] (p0-2) -- (p0-3) ;
  \draw[->] (p0-3) -- (p0-4) ;
  \draw[->] (p0-4) -- (p0-5) ;
  \draw (p0-5) -- (p0-7) ;
  \draw[->] (p0-6) |- (p1-7) ;
  \draw[->] (p1-7) -- (p1-9) ;
  \draw[->] (p1-10) |- (p2-11) ;
  \draw[->] (p3-12) -| (p1-8) ;
  \draw[->] (p2-11) -| (p3-12) ;
  \draw (p0-7) -- (p0-13) ;
  \draw[->] (p1-9) -| (p0-13) ;
  \draw[->] (p0-13) -- (p0-14) ;
  \draw[->] (p0-14) -- (p0-15) ;
\end{tikzpicture}

\ruleSubsection{galgas3InstructionsSyntax}{instruction-for}{184}

\begin{tikzpicture}
  \matrix[column sep=\ruleMatrixColumnSeparation, row sep=\ruleMatrixRowSeparation] {
    & & & & & & & & & & \node (p2-10) [point] {}; & \\
    & & & & & \node (p1-5) [terminal] {>}; & & & & \node (p1-9) [terminal] {,}; & & & \node (p1-12) [terminal] {while}; & \node (p1-13) [nonterminal] {\nonTerminalSymbol{expression}{1}}; & & & \node (p1-16) [terminal] {before}; & \node (p1-17) [nonterminal] {\nonTerminalSymbol{semantic\_instruction\_list}{14}}; & & & & \node (p1-21) [terminal] {(}; & \node (p1-22) [terminal] {identifier}; & \node (p1-23) [terminal] {)}; & & & & \node (p1-27) [terminal] {between}; & \node (p1-28) [nonterminal] {\nonTerminalSymbol{semantic\_instruction\_list}{14}}; & & & \node (p1-31) [terminal] {after}; & \node (p1-32) [nonterminal] {\nonTerminalSymbol{semantic\_instruction\_list}{14}}; & & & & \node (p1-36) [terminal] {\verb=%=attribute}; & \\
    \node (P0start) [firstPoint] {}; & & \node (p0-2) [terminal] {for}; & \node (p0-3) [point] {}; & \node (p0-4) [point] {}; & \node (p0-5) [point] {}; & \node (p0-6) [point] {}; & \node (p0-7) [nonterminal] {\nonTerminalSymbol{for\_instruction\_enumerated\_object}{25}}; & \node (p0-8) [point] {}; & & & \node (p0-11) [point] {}; & \node (p0-12) [point] {}; & & \node (p0-14) [point] {}; & \node (p0-15) [point] {}; & \node (p0-16) [point] {}; & & \node (p0-18) [point] {}; & \node (p0-19) [terminal] {do}; & \node (p0-20) [point] {}; & \node (p0-21) [point] {}; & & & \node (p0-24) [point] {}; & \node (p0-25) [nonterminal] {\nonTerminalSymbol{semantic\_instruction\_list}{14}}; & \node (p0-26) [point] {}; & \node (p0-27) [point] {}; & & \node (p0-29) [point] {}; & \node (p0-30) [point] {}; & \node (p0-31) [point] {}; & & \node (p0-33) [point] {}; & \node (p0-34) [terminal] {end}; & \node (p0-35) [point] {}; & \node (p0-36) [point] {}; & \node (p0-37) [point] {}; & \node (p0-38) [lastPoint] {}; & \\
  };
  \draw[->] (P0start) -- (p0-2) ;
  \draw (p0-2) -- (p0-5) ;
  \draw[->] (p0-4) |- (p1-5) ;
  \draw (p0-5) -- (p0-6) ;
  \draw[->] (p1-5) -| (p0-6) ;
  \draw[->] (p0-6) -- (p0-7) ;
  \draw[->] (p0-8) |- (p1-9) ;
  \draw[->] (p2-10) -| (p0-3) ;
  \draw[->] (p1-9) -| (p2-10) ;
  \draw (p0-7) -- (p0-12) ;
  \draw[->] (p0-11) |- (p1-12) ;
  \draw[->] (p1-12) -- (p1-13) ;
  \draw (p0-12) -- (p0-14) ;
  \draw[->] (p1-13) -| (p0-14) ;
  \draw (p0-14) -- (p0-16) ;
  \draw[->] (p0-15) |- (p1-16) ;
  \draw[->] (p1-16) -- (p1-17) ;
  \draw (p0-16) -- (p0-18) ;
  \draw[->] (p1-17) -| (p0-18) ;
  \draw[->] (p0-18) -- (p0-19) ;
  \draw (p0-19) -- (p0-21) ;
  \draw[->] (p0-20) |- (p1-21) ;
  \draw[->] (p1-21) -- (p1-22) ;
  \draw[->] (p1-22) -- (p1-23) ;
  \draw (p0-21) -- (p0-24) ;
  \draw[->] (p1-23) -| (p0-24) ;
  \draw[->] (p0-24) -- (p0-25) ;
  \draw (p0-25) -- (p0-27) ;
  \draw[->] (p0-26) |- (p1-27) ;
  \draw[->] (p1-27) -- (p1-28) ;
  \draw (p0-27) -- (p0-29) ;
  \draw[->] (p1-28) -| (p0-29) ;
  \draw (p0-29) -- (p0-31) ;
  \draw[->] (p0-30) |- (p1-31) ;
  \draw[->] (p1-31) -- (p1-32) ;
  \draw (p0-31) -- (p0-33) ;
  \draw[->] (p1-32) -| (p0-33) ;
  \draw[->] (p0-33) -- (p0-34) ;
  \draw (p0-34) -- (p0-36) ;
  \draw[->] (p0-35) |- (p1-36) ;
  \draw (p0-36) -- (p0-37) ;
  \draw[->] (p1-36) -| (p0-37) ;
  \draw[->] (p0-37) -- (p0-38) ;
\end{tikzpicture}

\ruleSubsection{galgas3InstructionsSyntax}{instruction-grammar}{76}

\begin{tikzpicture}
  \matrix[column sep=\ruleMatrixColumnSeparation, row sep=\ruleMatrixRowSeparation] {
    & & & & & \node (p1-5) [terminal] {label}; & \node (p1-6) [terminal] {identifier}; & \\
    \node (P0start) [firstPoint] {}; & & \node (p0-2) [terminal] {grammar}; & \node (p0-3) [terminal] {identifier}; & \node (p0-4) [point] {}; & \node (p0-5) [point] {}; & & \node (p0-7) [point] {}; & \node (p0-8) [nonterminal] {\nonTerminalSymbol{grammar\_instruction\_core}{26}}; & \node (p0-9) [lastPoint] {}; & \\
  };
  \draw[->] (P0start) -- (p0-2) ;
  \draw[->] (p0-2) -- (p0-3) ;
  \draw (p0-3) -- (p0-5) ;
  \draw[->] (p0-4) |- (p1-5) ;
  \draw[->] (p1-5) -- (p1-6) ;
  \draw (p0-5) -- (p0-7) ;
  \draw[->] (p1-6) -| (p0-7) ;
  \draw[->] (p0-7) -- (p0-8) ;
  \draw[->] (p0-8) -- (p0-9) ;
\end{tikzpicture}

\ruleSubsection{galgas3InstructionsSyntax}{instruction-inc-dec}{47}

\begin{tikzpicture}
  \matrix[column sep=\ruleMatrixColumnSeparation, row sep=\ruleMatrixRowSeparation] {
    & & & & & & & & & & \node (p3-10) [terminal] {--}; & \\
    & & & & & & & & \node (p2-8) [point] {}; & & \node (p2-10) [terminal] {\&--}; & \\
    & & & & & & \node (p1-6) [terminal] {.}; & \node (p1-7) [terminal] {identifier}; & & & \node (p1-10) [terminal] {\&++}; & \\
    \node (P0start) [firstPoint] {}; & & \node (p0-2) [terminal] {identifier}; & \node (p0-3) [point] {}; & \node (p0-4) [point] {}; & \node (p0-5) [point] {}; & & & & \node (p0-9) [point] {}; & \node (p0-10) [terminal] {++}; & \node (p0-11) [point] {}; & \node (p0-12) [lastPoint] {}; & \\
  };
  \draw[->] (P0start) -- (p0-2) ;
  \draw (p0-2) -- (p0-4) ;
  \draw[->] (p0-5) |- (p1-6) ;
  \draw[->] (p1-6) -- (p1-7) ;
  \draw[->] (p2-8) -| (p0-3) ;
  \draw[->] (p1-7) -| (p2-8) ;
  \draw[->] (p0-4) -- (p0-10) ;
  \draw[->] (p0-9) |- (p1-10) ;
  \draw[->] (p0-9) |- (p2-10) ;
  \draw[->] (p0-9) |- (p3-10) ;
  \draw (p0-10) -- (p0-11) ;
  \draw[->] (p1-10) -| (p0-11) ;
  \draw[->] (p2-10) -| (p0-11) ;
  \draw[->] (p3-10) -| (p0-11) ;
  \draw[->] (p0-11) -- (p0-12) ;
\end{tikzpicture}

\ruleSubsection{galgas3InstructionsSyntax}{instruction-inc-dec}{93}

\begin{tikzpicture}
  \matrix[column sep=\ruleMatrixColumnSeparation, row sep=\ruleMatrixRowSeparation] {
    & & & & \node (p3-4) [terminal] {\&--}; & \\
    & & & & \node (p2-4) [terminal] {\&++}; & \\
    & & & & \node (p1-4) [terminal] {--}; & \\
    \node (P0start) [firstPoint] {}; & & \node (p0-2) [terminal] {self}; & \node (p0-3) [point] {}; & \node (p0-4) [terminal] {++}; & \node (p0-5) [point] {}; & \node (p0-6) [lastPoint] {}; & \\
  };
  \draw[->] (P0start) -- (p0-2) ;
  \draw[->] (p0-2) -- (p0-4) ;
  \draw[->] (p0-3) |- (p1-4) ;
  \draw[->] (p0-3) |- (p2-4) ;
  \draw[->] (p0-3) |- (p3-4) ;
  \draw (p0-4) -- (p0-5) ;
  \draw[->] (p1-4) -| (p0-5) ;
  \draw[->] (p2-4) -| (p0-5) ;
  \draw[->] (p3-4) -| (p0-5) ;
  \draw[->] (p0-5) -- (p0-6) ;
\end{tikzpicture}

\ruleSubsection{galgas3InstructionsSyntax}{instruction-if}{50}

\begin{tikzpicture}
  \matrix[column sep=\ruleMatrixColumnSeparation, row sep=\ruleMatrixRowSeparation] {
    & & & & & & \node (p1-6) [terminal] {\verb=%=attribute}; & \\
    \node (P0start) [firstPoint] {}; & & \node (p0-2) [terminal] {if}; & \node (p0-3) [nonterminal] {\nonTerminalSymbol{if\_instruction\_core}{27}}; & \node (p0-4) [terminal] {end}; & \node (p0-5) [point] {}; & \node (p0-6) [point] {}; & \node (p0-7) [point] {}; & \node (p0-8) [lastPoint] {}; & \\
  };
  \draw[->] (P0start) -- (p0-2) ;
  \draw[->] (p0-2) -- (p0-3) ;
  \draw[->] (p0-3) -- (p0-4) ;
  \draw (p0-4) -- (p0-6) ;
  \draw[->] (p0-5) |- (p1-6) ;
  \draw (p0-6) -- (p0-7) ;
  \draw[->] (p1-6) -| (p0-7) ;
  \draw[->] (p0-7) -- (p0-8) ;
\end{tikzpicture}

\ruleSubsection{galgas3InstructionsSyntax}{instruction-let}{34}

\begin{tikzpicture}
  \matrix[column sep=\ruleMatrixColumnSeparation, row sep=\ruleMatrixRowSeparation] {
    \node (P0start) [firstPoint] {}; & & \node (p0-2) [terminal] {let}; & \node (p0-3) [terminal] {@type}; & \node (p0-4) [terminal] {identifier}; & \node (p0-5) [lastPoint] {}; & \\
  };
  \draw[->] (P0start) -- (p0-2) ;
  \draw[->] (p0-2) -- (p0-3) ;
  \draw[->] (p0-3) -- (p0-4) ;
  \draw[->] (p0-4) -- (p0-5) ;
\end{tikzpicture}

\ruleSubsection{galgas3InstructionsSyntax}{instruction-let}{47}

\begin{tikzpicture}
  \matrix[column sep=\ruleMatrixColumnSeparation, row sep=\ruleMatrixRowSeparation] {
    \node (P0start) [firstPoint] {}; & & \node (p4-2) [terminal] {let}; & \\
    & & \node (p3-2) [terminal] {@type}; & \\
    & & \node (p2-2) [terminal] {identifier}; & \\
    & & \node (p1-2) [terminal] {=}; & \\
    & & \node (p0-2) [nonterminal] {\nonTerminalSymbol{expression}{1}}; & \node (p0-3) [lastPoint] {}; & \\
  };
  \draw[->] (P0start) -- (p4-2) ;
  \draw[->] (p4-2) -- (p3-2) ;
  \draw[->] (p3-2) -- (p2-2) ;
  \draw[->] (p2-2) -- (p1-2) ;
  \draw[->] (p1-2) -- (p0-2) ;
  \draw[->] (p0-2) -- (p0-3) ;
\end{tikzpicture}

\ruleSubsection{galgas3InstructionsSyntax}{instruction-let}{63}

\begin{tikzpicture}
  \matrix[column sep=\ruleMatrixColumnSeparation, row sep=\ruleMatrixRowSeparation] {
    \node (P0start) [firstPoint] {}; & & \node (p0-2) [terminal] {let}; & \node (p0-3) [terminal] {identifier}; & \node (p0-4) [terminal] {=}; & \node (p0-5) [nonterminal] {\nonTerminalSymbol{expression}{1}}; & \node (p0-6) [lastPoint] {}; & \\
  };
  \draw[->] (P0start) -- (p0-2) ;
  \draw[->] (p0-2) -- (p0-3) ;
  \draw[->] (p0-3) -- (p0-4) ;
  \draw[->] (p0-4) -- (p0-5) ;
  \draw[->] (p0-5) -- (p0-6) ;
\end{tikzpicture}

\ruleSubsection{galgas3InstructionsSyntax}{instruction-log}{25}

\begin{tikzpicture}
  \matrix[column sep=\ruleMatrixColumnSeparation, row sep=\ruleMatrixRowSeparation] {
    & & & & & & & & & & & \node (p2-11) [point] {}; & \\
    & & & & & \node (p1-5) [terminal] {"string"}; & \node (p1-6) [terminal] {:}; & \node (p1-7) [nonterminal] {\nonTerminalSymbol{expression}{1}}; & & & \node (p1-10) [terminal] {,}; & \\
    \node (P0start) [firstPoint] {}; & & \node (p0-2) [terminal] {log}; & \node (p0-3) [point] {}; & \node (p0-4) [point] {}; & \node (p0-5) [terminal] {identifier}; & & & \node (p0-8) [point] {}; & \node (p0-9) [point] {}; & & & \node (p0-12) [lastPoint] {}; & \\
  };
  \draw[->] (P0start) -- (p0-2) ;
  \draw[->] (p0-2) -- (p0-5) ;
  \draw[->] (p0-4) |- (p1-5) ;
  \draw[->] (p1-5) -- (p1-6) ;
  \draw[->] (p1-6) -- (p1-7) ;
  \draw (p0-5) -- (p0-8) ;
  \draw[->] (p1-7) -| (p0-8) ;
  \draw[->] (p0-9) |- (p1-10) ;
  \draw[->] (p2-11) -| (p0-3) ;
  \draw[->] (p1-10) -| (p2-11) ;
  \draw[->] (p0-8) -- (p0-12) ;
\end{tikzpicture}

\ruleSubsection{galgas3InstructionsSyntax}{instruction-loop}{25}

\begin{tikzpicture}
  \matrix[column sep=\ruleMatrixColumnSeparation, row sep=\ruleMatrixRowSeparation] {
    & & & & & & & & & & & & & \node (p1-13) [terminal] {\verb=%=attribute}; & \\
    \node (P0start) [firstPoint] {}; & & \node (p0-2) [terminal] {loop}; & \node (p0-3) [terminal] {(}; & \node (p0-4) [nonterminal] {\nonTerminalSymbol{expression}{1}}; & \node (p0-5) [terminal] {)}; & \node (p0-6) [nonterminal] {\nonTerminalSymbol{semantic\_instruction\_list}{14}}; & \node (p0-7) [terminal] {while}; & \node (p0-8) [nonterminal] {\nonTerminalSymbol{expression}{1}}; & \node (p0-9) [terminal] {do}; & \node (p0-10) [nonterminal] {\nonTerminalSymbol{semantic\_instruction\_list}{14}}; & \node (p0-11) [terminal] {end}; & \node (p0-12) [point] {}; & \node (p0-13) [point] {}; & \node (p0-14) [point] {}; & \node (p0-15) [lastPoint] {}; & \\
  };
  \draw[->] (P0start) -- (p0-2) ;
  \draw[->] (p0-2) -- (p0-3) ;
  \draw[->] (p0-3) -- (p0-4) ;
  \draw[->] (p0-4) -- (p0-5) ;
  \draw[->] (p0-5) -- (p0-6) ;
  \draw[->] (p0-6) -- (p0-7) ;
  \draw[->] (p0-7) -- (p0-8) ;
  \draw[->] (p0-8) -- (p0-9) ;
  \draw[->] (p0-9) -- (p0-10) ;
  \draw[->] (p0-10) -- (p0-11) ;
  \draw (p0-11) -- (p0-13) ;
  \draw[->] (p0-12) |- (p1-13) ;
  \draw (p0-13) -- (p0-14) ;
  \draw[->] (p1-13) -| (p0-14) ;
  \draw[->] (p0-14) -- (p0-15) ;
\end{tikzpicture}

\ruleSubsection{galgas3InstructionsSyntax}{instruction-match}{84}

\begin{tikzpicture}
  \matrix[column sep=\ruleMatrixColumnSeparation, row sep=\ruleMatrixRowSeparation] {
    & & & & & & & & & & & & \node (p2-12) [point] {}; & \\
    & & & & & & & & \node (p1-8) [point] {}; & & & & & & \node (p1-14) [terminal] {else}; & \node (p1-15) [nonterminal] {\nonTerminalSymbol{semantic\_instruction\_list}{14}}; & & & & \node (p1-19) [terminal] {\verb=%=attribute}; & \\
    \node (P0start) [firstPoint] {}; & & \node (p0-2) [terminal] {match}; & \node (p0-3) [nonterminal] {\nonTerminalSymbol{expression}{1}}; & \node (p0-4) [point] {}; & \node (p0-5) [terminal] {,}; & \node (p0-6) [nonterminal] {\nonTerminalSymbol{expression}{1}}; & \node (p0-7) [point] {}; & & \node (p0-9) [point] {}; & \node (p0-10) [nonterminal] {\nonTerminalSymbol{match\_instruction\_branch}{29}}; & \node (p0-11) [point] {}; & & \node (p0-13) [point] {}; & \node (p0-14) [point] {}; & & \node (p0-16) [point] {}; & \node (p0-17) [terminal] {end}; & \node (p0-18) [point] {}; & \node (p0-19) [point] {}; & \node (p0-20) [point] {}; & \node (p0-21) [lastPoint] {}; & \\
  };
  \draw[->] (P0start) -- (p0-2) ;
  \draw[->] (p0-2) -- (p0-3) ;
  \draw[->] (p0-3) -- (p0-5) ;
  \draw[->] (p0-5) -- (p0-6) ;
  \draw[->] (p1-8) -| (p0-4) ;
  \draw[->] (p0-7) -| (p1-8) ;
  \draw[->] (p0-6) -- (p0-10) ;
  \draw[->] (p2-12) -| (p0-9) ;
  \draw[->] (p0-11) -| (p2-12) ;
  \draw (p0-10) -- (p0-14) ;
  \draw[->] (p0-13) |- (p1-14) ;
  \draw[->] (p1-14) -- (p1-15) ;
  \draw (p0-14) -- (p0-16) ;
  \draw[->] (p1-15) -| (p0-16) ;
  \draw[->] (p0-16) -- (p0-17) ;
  \draw (p0-17) -- (p0-19) ;
  \draw[->] (p0-18) |- (p1-19) ;
  \draw (p0-19) -- (p0-20) ;
  \draw[->] (p1-19) -| (p0-20) ;
  \draw[->] (p0-20) -- (p0-21) ;
\end{tikzpicture}

\ruleSubsection{galgas3InstructionsSyntax}{instruction-message}{17}

\begin{tikzpicture}
  \matrix[column sep=\ruleMatrixColumnSeparation, row sep=\ruleMatrixRowSeparation] {
    \node (P0start) [firstPoint] {}; & & \node (p0-2) [terminal] {message}; & \node (p0-3) [nonterminal] {\nonTerminalSymbol{expression}{1}}; & \node (p0-4) [lastPoint] {}; & \\
  };
  \draw[->] (P0start) -- (p0-2) ;
  \draw[->] (p0-2) -- (p0-3) ;
  \draw[->] (p0-3) -- (p0-4) ;
\end{tikzpicture}

\ruleSubsection{galgas3InstructionsSyntax}{instruction-method-call}{19}

\begin{tikzpicture}
  \matrix[column sep=\ruleMatrixColumnSeparation, row sep=\ruleMatrixRowSeparation] {
    \node (P0start) [firstPoint] {}; & & \node (p4-2) [terminal] {[}; & \\
    & & \node (p3-2) [nonterminal] {\nonTerminalSymbol{expression}{1}}; & \\
    & & \node (p2-2) [terminal] {identifier}; & \\
    & & \node (p1-2) [nonterminal] {\nonTerminalSymbol{actual\_parameter\_list}{12}}; & \\
    & & \node (p0-2) [terminal] {]}; & \node (p0-3) [lastPoint] {}; & \\
  };
  \draw[->] (P0start) -- (p4-2) ;
  \draw[->] (p4-2) -- (p3-2) ;
  \draw[->] (p3-2) -- (p2-2) ;
  \draw[->] (p2-2) -- (p1-2) ;
  \draw[->] (p1-2) -- (p0-2) ;
  \draw[->] (p0-2) -- (p0-3) ;
\end{tikzpicture}

\ruleSubsection{galgas3InstructionsSyntax}{instruction-proc-call}{18}

\begin{tikzpicture}
  \matrix[column sep=\ruleMatrixColumnSeparation, row sep=\ruleMatrixRowSeparation] {
    \node (P0start) [firstPoint] {}; & & \node (p0-2) [terminal] {identifier}; & \node (p0-3) [terminal] {(}; & \node (p0-4) [nonterminal] {\nonTerminalSymbol{actual\_parameter\_list}{12}}; & \node (p0-5) [terminal] {)}; & \node (p0-6) [lastPoint] {}; & \\
  };
  \draw[->] (P0start) -- (p0-2) ;
  \draw[->] (p0-2) -- (p0-3) ;
  \draw[->] (p0-3) -- (p0-4) ;
  \draw[->] (p0-4) -- (p0-5) ;
  \draw[->] (p0-5) -- (p0-6) ;
\end{tikzpicture}

\ruleSubsection{galgas3InstructionsSyntax}{instruction-self-assignment}{17}

\begin{tikzpicture}
  \matrix[column sep=\ruleMatrixColumnSeparation, row sep=\ruleMatrixRowSeparation] {
    \node (P0start) [firstPoint] {}; & & \node (p0-2) [terminal] {self}; & \node (p0-3) [terminal] {=}; & \node (p0-4) [nonterminal] {\nonTerminalSymbol{expression}{1}}; & \node (p0-5) [lastPoint] {}; & \\
  };
  \draw[->] (P0start) -- (p0-2) ;
  \draw[->] (p0-2) -- (p0-3) ;
  \draw[->] (p0-3) -- (p0-4) ;
  \draw[->] (p0-4) -- (p0-5) ;
\end{tikzpicture}

\ruleSubsection{galgas3InstructionsSyntax}{instruction-self-concat}{41}

\begin{tikzpicture}
  \matrix[column sep=\ruleMatrixColumnSeparation, row sep=\ruleMatrixRowSeparation] {
    \node (P0start) [firstPoint] {}; & & \node (p0-2) [terminal] {self}; & \node (p0-3) [terminal] {+=}; & \node (p0-4) [nonterminal] {\nonTerminalSymbol{non\_empty\_output\_expression\_list}{22}}; & \node (p0-5) [lastPoint] {}; & \\
  };
  \draw[->] (P0start) -- (p0-2) ;
  \draw[->] (p0-2) -- (p0-3) ;
  \draw[->] (p0-3) -- (p0-4) ;
  \draw[->] (p0-4) -- (p0-5) ;
\end{tikzpicture}

\ruleSubsection{galgas3InstructionsSyntax}{instruction-self-concat}{54}

\begin{tikzpicture}
  \matrix[column sep=\ruleMatrixColumnSeparation, row sep=\ruleMatrixRowSeparation] {
    \node (P0start) [firstPoint] {}; & & \node (p0-2) [terminal] {self}; & \node (p0-3) [terminal] {+=}; & \node (p0-4) [nonterminal] {\nonTerminalSymbol{expression}{1}}; & \node (p0-5) [lastPoint] {}; & \\
  };
  \draw[->] (P0start) -- (p0-2) ;
  \draw[->] (p0-2) -- (p0-3) ;
  \draw[->] (p0-3) -- (p0-4) ;
  \draw[->] (p0-4) -- (p0-5) ;
\end{tikzpicture}

\ruleSubsection{galgas3InstructionsSyntax}{instruction-self-concat}{67}

\begin{tikzpicture}
  \matrix[column sep=\ruleMatrixColumnSeparation, row sep=\ruleMatrixRowSeparation] {
    \node (P0start) [firstPoint] {}; & & \node (p0-2) [terminal] {self}; & \node (p0-3) [terminal] {-=}; & \node (p0-4) [nonterminal] {\nonTerminalSymbol{expression}{1}}; & \node (p0-5) [lastPoint] {}; & \\
  };
  \draw[->] (P0start) -- (p0-2) ;
  \draw[->] (p0-2) -- (p0-3) ;
  \draw[->] (p0-3) -- (p0-4) ;
  \draw[->] (p0-4) -- (p0-5) ;
\end{tikzpicture}

\ruleSubsection{galgas3InstructionsSyntax}{instruction-self-concat}{80}

\begin{tikzpicture}
  \matrix[column sep=\ruleMatrixColumnSeparation, row sep=\ruleMatrixRowSeparation] {
    \node (P0start) [firstPoint] {}; & & \node (p0-2) [terminal] {self}; & \node (p0-3) [terminal] {*=}; & \node (p0-4) [nonterminal] {\nonTerminalSymbol{expression}{1}}; & \node (p0-5) [lastPoint] {}; & \\
  };
  \draw[->] (P0start) -- (p0-2) ;
  \draw[->] (p0-2) -- (p0-3) ;
  \draw[->] (p0-3) -- (p0-4) ;
  \draw[->] (p0-4) -- (p0-5) ;
\end{tikzpicture}

\ruleSubsection{galgas3InstructionsSyntax}{instruction-self-concat}{93}

\begin{tikzpicture}
  \matrix[column sep=\ruleMatrixColumnSeparation, row sep=\ruleMatrixRowSeparation] {
    \node (P0start) [firstPoint] {}; & & \node (p0-2) [terminal] {self}; & \node (p0-3) [terminal] {/=}; & \node (p0-4) [nonterminal] {\nonTerminalSymbol{expression}{1}}; & \node (p0-5) [lastPoint] {}; & \\
  };
  \draw[->] (P0start) -- (p0-2) ;
  \draw[->] (p0-2) -- (p0-3) ;
  \draw[->] (p0-3) -- (p0-4) ;
  \draw[->] (p0-4) -- (p0-5) ;
\end{tikzpicture}

\ruleSubsection{galgas3InstructionsSyntax}{instruction-setter-call}{28}

\begin{tikzpicture}
  \matrix[column sep=\ruleMatrixColumnSeparation, row sep=\ruleMatrixRowSeparation] {
    & & & & & & & & & & \node (p2-10) [point] {}; & \\
    & & & & & & & & \node (p1-8) [terminal] {.}; & \node (p1-9) [terminal] {identifier}; & & & \node (p1-12) [terminal] {as}; & \node (p1-13) [terminal] {@type}; & \\
    \node (P0start) [firstPoint] {}; & & \node (p0-2) [terminal] {[}; & \node (p0-3) [terminal] {!?}; & \node (p0-4) [terminal] {identifier}; & \node (p0-5) [point] {}; & \node (p0-6) [point] {}; & \node (p0-7) [point] {}; & & & & \node (p0-11) [point] {}; & \node (p0-12) [point] {}; & & \node (p0-14) [point] {}; & \node (p0-15) [terminal] {identifier}; & \node (p0-16) [nonterminal] {\nonTerminalSymbol{actual\_parameter\_list}{12}}; & \node (p0-17) [terminal] {]}; & \node (p0-18) [lastPoint] {}; & \\
  };
  \draw[->] (P0start) -- (p0-2) ;
  \draw[->] (p0-2) -- (p0-3) ;
  \draw[->] (p0-3) -- (p0-4) ;
  \draw (p0-4) -- (p0-6) ;
  \draw[->] (p0-7) |- (p1-8) ;
  \draw[->] (p1-8) -- (p1-9) ;
  \draw[->] (p2-10) -| (p0-5) ;
  \draw[->] (p1-9) -| (p2-10) ;
  \draw (p0-6) -- (p0-12) ;
  \draw[->] (p0-11) |- (p1-12) ;
  \draw[->] (p1-12) -- (p1-13) ;
  \draw (p0-12) -- (p0-14) ;
  \draw[->] (p1-13) -| (p0-14) ;
  \draw[->] (p0-14) -- (p0-15) ;
  \draw[->] (p0-15) -- (p0-16) ;
  \draw[->] (p0-16) -- (p0-17) ;
  \draw[->] (p0-17) -- (p0-18) ;
\end{tikzpicture}

\ruleSubsection{galgas3InstructionsSyntax}{instruction-setter-call}{66}

\begin{tikzpicture}
  \matrix[column sep=\ruleMatrixColumnSeparation, row sep=\ruleMatrixRowSeparation] {
    \node (P0start) [firstPoint] {}; & & \node (p5-2) [terminal] {[}; & \\
    & & \node (p4-2) [terminal] {!?}; & \\
    & & \node (p3-2) [terminal] {self}; & \\
    & & \node (p2-2) [terminal] {identifier}; & \\
    & & \node (p1-2) [nonterminal] {\nonTerminalSymbol{actual\_parameter\_list}{12}}; & \\
    & & \node (p0-2) [terminal] {]}; & \node (p0-3) [lastPoint] {}; & \\
  };
  \draw[->] (P0start) -- (p5-2) ;
  \draw[->] (p5-2) -- (p4-2) ;
  \draw[->] (p4-2) -- (p3-2) ;
  \draw[->] (p3-2) -- (p2-2) ;
  \draw[->] (p2-2) -- (p1-2) ;
  \draw[->] (p1-2) -- (p0-2) ;
  \draw[->] (p0-2) -- (p0-3) ;
\end{tikzpicture}

\ruleSubsection{galgas3InstructionsSyntax}{instruction-switch}{37}

\begin{tikzpicture}
  \matrix[column sep=\ruleMatrixColumnSeparation, row sep=\ruleMatrixRowSeparation] {
    & & & & & & & & & & \node (p1-10) [point] {}; & & & \node (p1-13) [terminal] {\verb=%=attribute}; & \\
    \node (P0start) [firstPoint] {}; & & \node (p0-2) [terminal] {switch}; & \node (p0-3) [nonterminal] {\nonTerminalSymbol{expression}{1}}; & \node (p0-4) [point] {}; & \node (p0-5) [terminal] {case}; & \node (p0-6) [nonterminal] {\nonTerminalSymbol{switch\_case}{30}}; & \node (p0-7) [terminal] {:}; & \node (p0-8) [nonterminal] {\nonTerminalSymbol{semantic\_instruction\_list}{14}}; & \node (p0-9) [point] {}; & & \node (p0-11) [terminal] {end}; & \node (p0-12) [point] {}; & \node (p0-13) [point] {}; & \node (p0-14) [point] {}; & \node (p0-15) [lastPoint] {}; & \\
  };
  \draw[->] (P0start) -- (p0-2) ;
  \draw[->] (p0-2) -- (p0-3) ;
  \draw[->] (p0-3) -- (p0-5) ;
  \draw[->] (p0-5) -- (p0-6) ;
  \draw[->] (p0-6) -- (p0-7) ;
  \draw[->] (p0-7) -- (p0-8) ;
  \draw[->] (p1-10) -| (p0-4) ;
  \draw[->] (p0-9) -| (p1-10) ;
  \draw[->] (p0-8) -- (p0-11) ;
  \draw (p0-11) -- (p0-13) ;
  \draw[->] (p0-12) |- (p1-13) ;
  \draw (p0-13) -- (p0-14) ;
  \draw[->] (p1-13) -| (p0-14) ;
  \draw[->] (p0-14) -- (p0-15) ;
\end{tikzpicture}

\ruleSubsection{galgas3InstructionsSyntax}{instruction-type-method-call}{19}

\begin{tikzpicture}
  \matrix[column sep=\ruleMatrixColumnSeparation, row sep=\ruleMatrixRowSeparation] {
    \node (P0start) [firstPoint] {}; & & \node (p4-2) [terminal] {[}; & \\
    & & \node (p3-2) [terminal] {@type}; & \\
    & & \node (p2-2) [terminal] {identifier}; & \\
    & & \node (p1-2) [nonterminal] {\nonTerminalSymbol{actual\_parameter\_list}{12}}; & \\
    & & \node (p0-2) [terminal] {]}; & \node (p0-3) [lastPoint] {}; & \\
  };
  \draw[->] (P0start) -- (p4-2) ;
  \draw[->] (p4-2) -- (p3-2) ;
  \draw[->] (p3-2) -- (p2-2) ;
  \draw[->] (p2-2) -- (p1-2) ;
  \draw[->] (p1-2) -- (p0-2) ;
  \draw[->] (p0-2) -- (p0-3) ;
\end{tikzpicture}

\ruleSubsection{galgas3InstructionsSyntax}{instruction-var-declaration-with-assignment}{18}

\begin{tikzpicture}
  \matrix[column sep=\ruleMatrixColumnSeparation, row sep=\ruleMatrixRowSeparation] {
    \node (P0start) [firstPoint] {}; & & \node (p0-2) [terminal] {var}; & \node (p0-3) [terminal] {identifier}; & \node (p0-4) [terminal] {=}; & \node (p0-5) [nonterminal] {\nonTerminalSymbol{expression}{1}}; & \node (p0-6) [lastPoint] {}; & \\
  };
  \draw[->] (P0start) -- (p0-2) ;
  \draw[->] (p0-2) -- (p0-3) ;
  \draw[->] (p0-3) -- (p0-4) ;
  \draw[->] (p0-4) -- (p0-5) ;
  \draw[->] (p0-5) -- (p0-6) ;
\end{tikzpicture}

\ruleSubsection{galgas3InstructionsSyntax}{instruction-warning}{19}

\begin{tikzpicture}
  \matrix[column sep=\ruleMatrixColumnSeparation, row sep=\ruleMatrixRowSeparation] {
    \node (P0start) [firstPoint] {}; & & \node (p4-2) [terminal] {warning}; & \\
    & & \node (p3-2) [nonterminal] {\nonTerminalSymbol{expression}{1}}; & \\
    & & \node (p2-2) [terminal] {:}; & \\
    & & \node (p1-2) [nonterminal] {\nonTerminalSymbol{expression}{1}}; & \\
    & & \node (p0-2) [nonterminal] {\nonTerminalSymbol{issue\_fixit}{23}}; & \node (p0-3) [lastPoint] {}; & \\
  };
  \draw[->] (P0start) -- (p4-2) ;
  \draw[->] (p4-2) -- (p3-2) ;
  \draw[->] (p3-2) -- (p2-2) ;
  \draw[->] (p2-2) -- (p1-2) ;
  \draw[->] (p1-2) -- (p0-2) ;
  \draw[->] (p0-2) -- (p0-3) ;
\end{tikzpicture}

\ruleSubsection{galgas3InstructionsSyntax}{instruction-with}{42}

\begin{tikzpicture}
  \matrix[column sep=\ruleMatrixColumnSeparation, row sep=\ruleMatrixRowSeparation] {
    & & & & & \node (p1-5) [terminal] {:}; & \node (p1-6) [terminal] {identifier}; & & & & & & \node (p1-12) [terminal] {\verb=%=attribute}; & \\
    \node (P0start) [firstPoint] {}; & & \node (p0-2) [terminal] {with}; & \node (p0-3) [nonterminal] {\nonTerminalSymbol{expression}{1}}; & \node (p0-4) [point] {}; & \node (p0-5) [point] {}; & & \node (p0-7) [point] {}; & \node (p0-8) [terminal] {in}; & \node (p0-9) [nonterminal] {\nonTerminalSymbol{with\_instruction\_core}{31}}; & \node (p0-10) [terminal] {end}; & \node (p0-11) [point] {}; & \node (p0-12) [point] {}; & \node (p0-13) [point] {}; & \node (p0-14) [lastPoint] {}; & \\
  };
  \draw[->] (P0start) -- (p0-2) ;
  \draw[->] (p0-2) -- (p0-3) ;
  \draw (p0-3) -- (p0-5) ;
  \draw[->] (p0-4) |- (p1-5) ;
  \draw[->] (p1-5) -- (p1-6) ;
  \draw (p0-5) -- (p0-7) ;
  \draw[->] (p1-6) -| (p0-7) ;
  \draw[->] (p0-7) -- (p0-8) ;
  \draw[->] (p0-8) -- (p0-9) ;
  \draw[->] (p0-9) -- (p0-10) ;
  \draw (p0-10) -- (p0-12) ;
  \draw[->] (p0-11) |- (p1-12) ;
  \draw (p0-12) -- (p0-13) ;
  \draw[->] (p1-12) -| (p0-13) ;
  \draw[->] (p0-13) -- (p0-14) ;
\end{tikzpicture}

\ruleSubsection{galgas3DeclarationsSyntax}{instruction-var-declaration}{19}

\begin{tikzpicture}
  \matrix[column sep=\ruleMatrixColumnSeparation, row sep=\ruleMatrixRowSeparation] {
    \node (P0start) [firstPoint] {}; & & \node (p0-2) [terminal] {var}; & \node (p0-3) [terminal] {@type}; & \node (p0-4) [terminal] {identifier}; & \node (p0-5) [lastPoint] {}; & \\
  };
  \draw[->] (P0start) -- (p0-2) ;
  \draw[->] (p0-2) -- (p0-3) ;
  \draw[->] (p0-3) -- (p0-4) ;
  \draw[->] (p0-4) -- (p0-5) ;
\end{tikzpicture}

\ruleSubsection{galgas3DeclarationsSyntax}{instruction-var-declaration}{28}

\begin{tikzpicture}
  \matrix[column sep=\ruleMatrixColumnSeparation, row sep=\ruleMatrixRowSeparation] {
    \node (P0start) [firstPoint] {}; & & \node (p0-2) [terminal] {@type}; & \node (p0-3) [terminal] {identifier}; & \node (p0-4) [lastPoint] {}; & \\
  };
  \draw[->] (P0start) -- (p0-2) ;
  \draw[->] (p0-2) -- (p0-3) ;
  \draw[->] (p0-3) -- (p0-4) ;
\end{tikzpicture}

\nonTerminalSection{semantic\_instruction\_list}{14}

\ruleSubsection{galgas3InstructionsSyntax}{galgas3InstructionsSyntax}{36}

\begin{tikzpicture}
  \matrix[column sep=\ruleMatrixColumnSeparation, row sep=\ruleMatrixRowSeparation] {
    & & & & & & \node (p3-6) [point] {}; & \\
    & & & & & \node (p2-5) [nonterminal] {\nonTerminalSymbol{semantic\_instruction}{17}}; & \\
    & & & & & \node (p1-5) [terminal] {;}; & \\
    \node (P0start) [firstPoint] {}; & & \node (p0-2) [point] {}; & \node (p0-3) [point] {}; & \node (p0-4) [point] {}; & & & \node (p0-7) [lastPoint] {}; & \\
  };
  \draw (P0start) -- (p0-3) ;
  \draw[->] (p0-4) |- (p1-5) ;
  \draw[->] (p0-4) |- (p2-5) ;
  \draw[->] (p3-6) -| (p0-2) ;
  \draw[->] (p1-5) -| (p3-6) ;
  \draw[->] (p2-5) -| (p3-6) ;
  \draw[->] (p0-3) -- (p0-7) ;
\end{tikzpicture}

\nonTerminalSection{shared\_map\_attribute}{48}

\ruleSubsection{galgas3DeclarationsSyntax}{type-shared-map}{211}

\begin{tikzpicture}
  \matrix[column sep=\ruleMatrixColumnSeparation, row sep=\ruleMatrixRowSeparation] {
    & & & \node (p1-3) [terminal] {warning}; & \\
    \node (P0start) [firstPoint] {}; & & \node (p0-2) [point] {}; & \node (p0-3) [terminal] {error}; & \node (p0-4) [point] {}; & \node (p0-5) [terminal] {\verb=%=attribute}; & \node (p0-6) [terminal] {"string"}; & \node (p0-7) [lastPoint] {}; & \\
  };
  \draw[->] (P0start) -- (p0-3) ;
  \draw[->] (p0-2) |- (p1-3) ;
  \draw (p0-3) -- (p0-4) ;
  \draw[->] (p1-3) -| (p0-4) ;
  \draw[->] (p0-4) -- (p0-5) ;
  \draw[->] (p0-5) -- (p0-6) ;
  \draw[->] (p0-6) -- (p0-7) ;
\end{tikzpicture}

\nonTerminalSection{shared\_map\_override}{47}

\ruleSubsection{galgas3DeclarationsSyntax}{type-shared-map}{151}

\begin{tikzpicture}
  \matrix[column sep=\ruleMatrixColumnSeparation, row sep=\ruleMatrixRowSeparation] {
    & & & & & & & & & & & & & & & & & \node (p4-17) [point] {}; & & & & & & & & & & & & & & & \node (p4-32) [point] {}; & \\
    & & & & & & & & & & & & & & \node (p3-14) [terminal] {error}; & \node (p3-15) [terminal] {"string"}; & & & & & & & & & & & & & & \node (p3-29) [terminal] {error}; & \node (p3-30) [terminal] {"string"}; & \\
    & & & & & & & & & & & & & & \node (p2-14) [terminal] {warning}; & \node (p2-15) [terminal] {"string"}; & & & & & & & & & & & & & & \node (p2-29) [terminal] {warning}; & \node (p2-30) [terminal] {"string"}; & \\
    & & & & & & & & \node (p1-8) [terminal] {identifier}; & \node (p1-9) [terminal] {:}; & \node (p1-10) [terminal] {identifier}; & \node (p1-11) [terminal] {->}; & \node (p1-12) [terminal] {identifier}; & \node (p1-13) [point] {}; & \node (p1-14) [point] {}; & & \node (p1-16) [point] {}; & & & & & & & \node (p1-23) [terminal] {identifier}; & \node (p1-24) [terminal] {:}; & \node (p1-25) [terminal] {identifier}; & \node (p1-26) [terminal] {->}; & \node (p1-27) [terminal] {identifier}; & \node (p1-28) [point] {}; & \node (p1-29) [point] {}; & & \node (p1-31) [point] {}; & \\
    \node (P0start) [firstPoint] {}; & & \node (p0-2) [terminal] {override}; & \node (p0-3) [terminal] {identifier}; & \node (p0-4) [terminal] {\{}; & \node (p0-5) [point] {}; & \node (p0-6) [point] {}; & \node (p0-7) [point] {}; & & & & & & & & & & & \node (p0-18) [terminal] {\}}; & \node (p0-19) [terminal] {\{}; & \node (p0-20) [point] {}; & \node (p0-21) [point] {}; & \node (p0-22) [point] {}; & & & & & & & & & & & \node (p0-33) [terminal] {\}}; & \node (p0-34) [lastPoint] {}; & \\
  };
  \draw[->] (P0start) -- (p0-2) ;
  \draw[->] (p0-2) -- (p0-3) ;
  \draw[->] (p0-3) -- (p0-4) ;
  \draw (p0-4) -- (p0-6) ;
  \draw[->] (p0-7) |- (p1-8) ;
  \draw[->] (p1-8) -- (p1-9) ;
  \draw[->] (p1-9) -- (p1-10) ;
  \draw[->] (p1-10) -- (p1-11) ;
  \draw[->] (p1-11) -- (p1-12) ;
  \draw (p1-12) -- (p1-14) ;
  \draw[->] (p1-13) |- (p2-14) ;
  \draw[->] (p2-14) -- (p2-15) ;
  \draw[->] (p1-13) |- (p3-14) ;
  \draw[->] (p3-14) -- (p3-15) ;
  \draw (p1-14) -- (p1-16) ;
  \draw[->] (p2-15) -| (p1-16) ;
  \draw[->] (p3-15) -| (p1-16) ;
  \draw[->] (p4-17) -| (p0-5) ;
  \draw[->] (p1-16) -| (p4-17) ;
  \draw[->] (p0-6) -- (p0-18) ;
  \draw[->] (p0-18) -- (p0-19) ;
  \draw (p0-19) -- (p0-21) ;
  \draw[->] (p0-22) |- (p1-23) ;
  \draw[->] (p1-23) -- (p1-24) ;
  \draw[->] (p1-24) -- (p1-25) ;
  \draw[->] (p1-25) -- (p1-26) ;
  \draw[->] (p1-26) -- (p1-27) ;
  \draw (p1-27) -- (p1-29) ;
  \draw[->] (p1-28) |- (p2-29) ;
  \draw[->] (p2-29) -- (p2-30) ;
  \draw[->] (p1-28) |- (p3-29) ;
  \draw[->] (p3-29) -- (p3-30) ;
  \draw (p1-29) -- (p1-31) ;
  \draw[->] (p2-30) -| (p1-31) ;
  \draw[->] (p3-30) -| (p1-31) ;
  \draw[->] (p4-32) -| (p0-20) ;
  \draw[->] (p1-31) -| (p4-32) ;
  \draw[->] (p0-21) -- (p0-33) ;
  \draw[->] (p0-33) -- (p0-34) ;
\end{tikzpicture}

\nonTerminalSection{shared\_map\_search\_method\_declaration}{49}

\ruleSubsection{galgas3DeclarationsSyntax}{type-shared-map}{227}

\begin{tikzpicture}
  \matrix[column sep=\ruleMatrixColumnSeparation, row sep=\ruleMatrixRowSeparation] {
    & & & & & \node (p1-5) [terminal] {do}; & \node (p1-6) [terminal] {identifier}; & \\
    \node (P0start) [firstPoint] {}; & & \node (p0-2) [terminal] {search}; & \node (p0-3) [terminal] {identifier}; & \node (p0-4) [point] {}; & \node (p0-5) [point] {}; & & \node (p0-7) [point] {}; & \node (p0-8) [terminal] {error}; & \node (p0-9) [terminal] {message}; & \node (p0-10) [terminal] {"string"}; & \node (p0-11) [lastPoint] {}; & \\
  };
  \draw[->] (P0start) -- (p0-2) ;
  \draw[->] (p0-2) -- (p0-3) ;
  \draw (p0-3) -- (p0-5) ;
  \draw[->] (p0-4) |- (p1-5) ;
  \draw[->] (p1-5) -- (p1-6) ;
  \draw (p0-5) -- (p0-7) ;
  \draw[->] (p1-6) -| (p0-7) ;
  \draw[->] (p0-7) -- (p0-8) ;
  \draw[->] (p0-8) -- (p0-9) ;
  \draw[->] (p0-9) -- (p0-10) ;
  \draw[->] (p0-10) -- (p0-11) ;
\end{tikzpicture}

\nonTerminalSection{shared\_map\_state\_list}{50}

\ruleSubsection{galgas3DeclarationsSyntax}{type-shared-map}{249}

\begin{tikzpicture}
  \matrix[column sep=\ruleMatrixColumnSeparation, row sep=\ruleMatrixRowSeparation] {
    & & & & & \node (p2-5) [terminal] {error}; & \node (p2-6) [terminal] {"string"}; & & & & & & & \node (p2-13) [point] {}; & \\
    & & & & & \node (p1-5) [terminal] {warning}; & \node (p1-6) [terminal] {"string"}; & & & & & & \node (p1-12) [nonterminal] {\nonTerminalSymbol{shared\_map\_state\_transition}{51}}; & \\
    \node (P0start) [firstPoint] {}; & & \node (p0-2) [terminal] {state}; & \node (p0-3) [terminal] {identifier}; & \node (p0-4) [point] {}; & \node (p0-5) [point] {}; & & \node (p0-7) [point] {}; & \node (p0-8) [terminal] {\{}; & \node (p0-9) [point] {}; & \node (p0-10) [point] {}; & \node (p0-11) [point] {}; & & & \node (p0-14) [terminal] {\}}; & \node (p0-15) [lastPoint] {}; & \\
  };
  \draw[->] (P0start) -- (p0-2) ;
  \draw[->] (p0-2) -- (p0-3) ;
  \draw (p0-3) -- (p0-5) ;
  \draw[->] (p0-4) |- (p1-5) ;
  \draw[->] (p1-5) -- (p1-6) ;
  \draw[->] (p0-4) |- (p2-5) ;
  \draw[->] (p2-5) -- (p2-6) ;
  \draw (p0-5) -- (p0-7) ;
  \draw[->] (p1-6) -| (p0-7) ;
  \draw[->] (p2-6) -| (p0-7) ;
  \draw[->] (p0-7) -- (p0-8) ;
  \draw (p0-8) -- (p0-10) ;
  \draw[->] (p0-11) |- (p1-12) ;
  \draw[->] (p2-13) -| (p0-9) ;
  \draw[->] (p1-12) -| (p2-13) ;
  \draw[->] (p0-10) -- (p0-14) ;
  \draw[->] (p0-14) -- (p0-15) ;
\end{tikzpicture}

\nonTerminalSection{shared\_map\_state\_transition}{51}

\ruleSubsection{galgas3DeclarationsSyntax}{type-shared-map}{282}

\begin{tikzpicture}
  \matrix[column sep=\ruleMatrixColumnSeparation, row sep=\ruleMatrixRowSeparation] {
    & & & & & & \node (p2-6) [terminal] {error}; & \node (p2-7) [terminal] {"string"}; & \\
    & & & & & & \node (p1-6) [terminal] {warning}; & \node (p1-7) [terminal] {"string"}; & \\
    \node (P0start) [firstPoint] {}; & & \node (p0-2) [terminal] {identifier}; & \node (p0-3) [terminal] {->}; & \node (p0-4) [terminal] {identifier}; & \node (p0-5) [point] {}; & \node (p0-6) [point] {}; & & \node (p0-8) [point] {}; & \node (p0-9) [lastPoint] {}; & \\
  };
  \draw[->] (P0start) -- (p0-2) ;
  \draw[->] (p0-2) -- (p0-3) ;
  \draw[->] (p0-3) -- (p0-4) ;
  \draw (p0-4) -- (p0-6) ;
  \draw[->] (p0-5) |- (p1-6) ;
  \draw[->] (p1-6) -- (p1-7) ;
  \draw[->] (p0-5) |- (p2-6) ;
  \draw[->] (p2-6) -- (p2-7) ;
  \draw (p0-6) -- (p0-8) ;
  \draw[->] (p1-7) -| (p0-8) ;
  \draw[->] (p2-7) -| (p0-8) ;
  \draw[->] (p0-8) -- (p0-9) ;
\end{tikzpicture}

\nonTerminalSection{simple\_expression}{5}

\ruleSubsection{galgas3ExpressionSyntax}{galgas3ExpressionSyntax}{198}

\begin{tikzpicture}
  \matrix[column sep=\ruleMatrixColumnSeparation, row sep=\ruleMatrixRowSeparation] {
    & & & & & & & & \node (p7-8) [point] {}; & \\
    & & & & & & \node (p6-6) [terminal] {\&-}; & \node (p6-7) [nonterminal] {\nonTerminalSymbol{term}{6}}; & \\
    & & & & & & \node (p5-6) [terminal] {-}; & \node (p5-7) [nonterminal] {\nonTerminalSymbol{term}{6}}; & \\
    & & & & & & \node (p4-6) [terminal] {\&+}; & \node (p4-7) [nonterminal] {\nonTerminalSymbol{term}{6}}; & \\
    & & & & & & \node (p3-6) [terminal] {+}; & \node (p3-7) [nonterminal] {\nonTerminalSymbol{term}{6}}; & \\
    & & & & & & \node (p2-6) [terminal] {>>}; & \node (p2-7) [nonterminal] {\nonTerminalSymbol{term}{6}}; & \\
    & & & & & & \node (p1-6) [terminal] {<<}; & \node (p1-7) [nonterminal] {\nonTerminalSymbol{term}{6}}; & \\
    \node (P0start) [firstPoint] {}; & & \node (p0-2) [nonterminal] {\nonTerminalSymbol{term}{6}}; & \node (p0-3) [point] {}; & \node (p0-4) [point] {}; & \node (p0-5) [point] {}; & & & & \node (p0-9) [lastPoint] {}; & \\
  };
  \draw[->] (P0start) -- (p0-2) ;
  \draw (p0-2) -- (p0-4) ;
  \draw[->] (p0-5) |- (p1-6) ;
  \draw[->] (p1-6) -- (p1-7) ;
  \draw[->] (p0-5) |- (p2-6) ;
  \draw[->] (p2-6) -- (p2-7) ;
  \draw[->] (p0-5) |- (p3-6) ;
  \draw[->] (p3-6) -- (p3-7) ;
  \draw[->] (p0-5) |- (p4-6) ;
  \draw[->] (p4-6) -- (p4-7) ;
  \draw[->] (p0-5) |- (p5-6) ;
  \draw[->] (p5-6) -- (p5-7) ;
  \draw[->] (p0-5) |- (p6-6) ;
  \draw[->] (p6-6) -- (p6-7) ;
  \draw[->] (p7-8) -| (p0-3) ;
  \draw[->] (p1-7) -| (p7-8) ;
  \draw[->] (p2-7) -| (p7-8) ;
  \draw[->] (p3-7) -| (p7-8) ;
  \draw[->] (p4-7) -| (p7-8) ;
  \draw[->] (p5-7) -| (p7-8) ;
  \draw[->] (p6-7) -| (p7-8) ;
  \draw[->] (p0-4) -- (p0-9) ;
\end{tikzpicture}

\nonTerminalSection{sortedlist\_sort\_descriptor}{52}

\ruleSubsection{galgas3DeclarationsSyntax}{type-sorted-list}{58}

\begin{tikzpicture}
  \matrix[column sep=\ruleMatrixColumnSeparation, row sep=\ruleMatrixRowSeparation] {
    & & & & \node (p1-4) [terminal] {>}; & \\
    \node (P0start) [firstPoint] {}; & & \node (p0-2) [terminal] {identifier}; & \node (p0-3) [point] {}; & \node (p0-4) [terminal] {<}; & \node (p0-5) [point] {}; & \node (p0-6) [lastPoint] {}; & \\
  };
  \draw[->] (P0start) -- (p0-2) ;
  \draw[->] (p0-2) -- (p0-4) ;
  \draw[->] (p0-3) |- (p1-4) ;
  \draw (p0-4) -- (p0-5) ;
  \draw[->] (p1-4) -| (p0-5) ;
  \draw[->] (p0-5) -- (p0-6) ;
\end{tikzpicture}

\nonTerminalSection{start\_symbol}{32}

\ruleSubsection{galgas3DeclarationsSyntax}{galgas3DeclarationsSyntax}{30}

\begin{tikzpicture}
  \matrix[column sep=\ruleMatrixColumnSeparation, row sep=\ruleMatrixRowSeparation] {
    & & & & & & \node (p3-6) [point] {}; & \\
    & & & & & \node (p2-5) [nonterminal] {\nonTerminalSymbol{declaration}{15}}; & \\
    & & & & & \node (p1-5) [terminal] {;}; & \\
    \node (P0start) [firstPoint] {}; & & \node (p0-2) [point] {}; & \node (p0-3) [point] {}; & \node (p0-4) [point] {}; & & & \node (p0-7) [lastPoint] {}; & \\
  };
  \draw (P0start) -- (p0-3) ;
  \draw[->] (p0-4) |- (p1-5) ;
  \draw[->] (p0-4) |- (p2-5) ;
  \draw[->] (p3-6) -| (p0-2) ;
  \draw[->] (p1-5) -| (p3-6) ;
  \draw[->] (p2-5) -| (p3-6) ;
  \draw[->] (p0-3) -- (p0-7) ;
\end{tikzpicture}

\nonTerminalSection{style\_declaration}{70}

\ruleSubsection{galgas3LexiqueComponentSyntax}{galgas3LexiqueComponentSyntax}{579}

\begin{tikzpicture}
  \matrix[column sep=\ruleMatrixColumnSeparation, row sep=\ruleMatrixRowSeparation] {
    \node (P0start) [firstPoint] {}; & & \node (p0-2) [terminal] {style}; & \node (p0-3) [terminal] {identifier}; & \node (p0-4) [terminal] {->}; & \node (p0-5) [terminal] {"string"}; & \node (p0-6) [lastPoint] {}; & \\
  };
  \draw[->] (P0start) -- (p0-2) ;
  \draw[->] (p0-2) -- (p0-3) ;
  \draw[->] (p0-3) -- (p0-4) ;
  \draw[->] (p0-4) -- (p0-5) ;
  \draw[->] (p0-5) -- (p0-6) ;
\end{tikzpicture}

\nonTerminalSection{switch\_case}{30}

\ruleSubsection{galgas3InstructionsSyntax}{instruction-switch}{73}

\begin{tikzpicture}
  \matrix[column sep=\ruleMatrixColumnSeparation, row sep=\ruleMatrixRowSeparation] {
    & & & & & & & & & & & & & & & & & & & & \node (p4-20) [point] {}; & \\
    & & & & & & & & & & & & \node (p3-12) [point] {}; & & & \node (p3-15) [terminal] {unused}; & \\
    & & & & & & \node (p2-6) [point] {}; & & & & & \node (p2-11) [point] {}; & \node (p2-12) [terminal] {@type}; & \node (p2-13) [point] {}; & \node (p2-14) [point] {}; & \node (p2-15) [point] {}; & \node (p2-16) [point] {}; & \node (p2-17) [terminal] {identifier}; & \\
    & & & & & \node (p1-5) [terminal] {,}; & & & \node (p1-8) [terminal] {(}; & \node (p1-9) [point] {}; & \node (p1-10) [point] {}; & \node (p1-11) [terminal] {*}; & & & & & & & \node (p1-18) [point] {}; & \node (p1-19) [point] {}; & & \node (p1-21) [terminal] {)}; & \\
    \node (P0start) [firstPoint] {}; & & \node (p0-2) [point] {}; & \node (p0-3) [terminal] {identifier}; & \node (p0-4) [point] {}; & & & \node (p0-7) [point] {}; & \node (p0-8) [point] {}; & & & & & & & & & & & & & & \node (p0-22) [point] {}; & \node (p0-23) [lastPoint] {}; & \\
  };
  \draw[->] (P0start) -- (p0-3) ;
  \draw[->] (p0-4) |- (p1-5) ;
  \draw[->] (p2-6) -| (p0-2) ;
  \draw[->] (p1-5) -| (p2-6) ;
  \draw (p0-3) -- (p0-8) ;
  \draw[->] (p0-7) |- (p1-8) ;
  \draw[->] (p1-8) -- (p1-11) ;
  \draw[->] (p1-10) |- (p2-12) ;
  \draw (p2-11) |- (p3-12) ;
  \draw (p2-12) -- (p2-13) ;
  \draw[->] (p3-12) -| (p2-13) ;
  \draw (p2-13) -- (p2-15) ;
  \draw[->] (p2-14) |- (p3-15) ;
  \draw (p2-15) -- (p2-16) ;
  \draw[->] (p3-15) -| (p2-16) ;
  \draw[->] (p2-16) -- (p2-17) ;
  \draw (p1-11) -- (p1-18) ;
  \draw[->] (p2-17) -| (p1-18) ;
  \draw[->] (p4-20) -| (p1-9) ;
  \draw[->] (p1-19) -| (p4-20) ;
  \draw[->] (p1-18) -- (p1-21) ;
  \draw (p0-8) -- (p0-22) ;
  \draw[->] (p1-21) -| (p0-22) ;
  \draw[->] (p0-22) -- (p0-23) ;
\end{tikzpicture}

\nonTerminalSection{syntax\_directed\_translation\_result}{18}

\ruleSubsection{galgas3InstructionsSyntax}{galgas3InstructionsSyntax}{273}

\begin{tikzpicture}
  \matrix[column sep=\ruleMatrixColumnSeparation, row sep=\ruleMatrixRowSeparation] {
    & & & & \node (p3-4) [terminal] {identifier}; & \\
    & & & & \node (p2-4) [terminal] {let}; & \node (p2-5) [terminal] {@type}; & \node (p2-6) [terminal] {identifier}; & \\
    & & & & \node (p1-4) [terminal] {@type}; & \node (p1-5) [terminal] {identifier}; & \\
    \node (P0start) [firstPoint] {}; & & \node (p0-2) [terminal] {?}; & \node (p0-3) [point] {}; & \node (p0-4) [terminal] {*}; & & & \node (p0-7) [point] {}; & \node (p0-8) [lastPoint] {}; & \\
  };
  \draw[->] (P0start) -- (p0-2) ;
  \draw[->] (p0-2) -- (p0-4) ;
  \draw[->] (p0-3) |- (p1-4) ;
  \draw[->] (p1-4) -- (p1-5) ;
  \draw[->] (p0-3) |- (p2-4) ;
  \draw[->] (p2-4) -- (p2-5) ;
  \draw[->] (p2-5) -- (p2-6) ;
  \draw[->] (p0-3) |- (p3-4) ;
  \draw (p0-4) -- (p0-7) ;
  \draw[->] (p1-5) -| (p0-7) ;
  \draw[->] (p2-6) -| (p0-7) ;
  \draw[->] (p3-4) -| (p0-7) ;
  \draw[->] (p0-7) -- (p0-8) ;
\end{tikzpicture}

\nonTerminalSection{syntax\_instruction}{85}

\ruleSubsection{galgas3SyntaxComponentSyntax}{instruction-non-terminal}{20}

\begin{tikzpicture}
  \matrix[column sep=\ruleMatrixColumnSeparation, row sep=\ruleMatrixRowSeparation] {
    & & & \node (p2-3) [terminal] {parse}; & \node (p2-4) [terminal] {identifier}; & \\
    & & & \node (p1-3) [terminal] {parse}; & & & & & & \node (p1-9) [terminal] {:>}; & \node (p1-10) [nonterminal] {\nonTerminalSymbol{syntax\_directed\_translation\_result}{18}}; & \\
    \node (P0start) [firstPoint] {}; & & \node (p0-2) [point] {}; & \node (p0-3) [point] {}; & & \node (p0-5) [point] {}; & \node (p0-6) [terminal] {<non\_terminal>}; & \node (p0-7) [nonterminal] {\nonTerminalSymbol{actual\_parameter\_list}{12}}; & \node (p0-8) [point] {}; & \node (p0-9) [point] {}; & & \node (p0-11) [point] {}; & \node (p0-12) [lastPoint] {}; & \\
  };
  \draw (P0start) -- (p0-3) ;
  \draw[->] (p0-2) |- (p1-3) ;
  \draw[->] (p0-2) |- (p2-3) ;
  \draw[->] (p2-3) -- (p2-4) ;
  \draw (p0-3) -- (p0-5) ;
  \draw[->] (p1-3) -| (p0-5) ;
  \draw[->] (p2-4) -| (p0-5) ;
  \draw[->] (p0-5) -- (p0-6) ;
  \draw[->] (p0-6) -- (p0-7) ;
  \draw (p0-7) -- (p0-9) ;
  \draw[->] (p0-8) |- (p1-9) ;
  \draw[->] (p1-9) -- (p1-10) ;
  \draw (p0-9) -- (p0-11) ;
  \draw[->] (p1-10) -| (p0-11) ;
  \draw[->] (p0-11) -- (p0-12) ;
\end{tikzpicture}

\ruleSubsection{galgas3SyntaxComponentSyntax}{instruction-repeat}{20}

\begin{tikzpicture}
  \matrix[column sep=\ruleMatrixColumnSeparation, row sep=\ruleMatrixRowSeparation] {
    & & & & & & & & \node (p1-8) [point] {}; & & & \node (p1-11) [terminal] {\verb=%=attribute}; & \\
    \node (P0start) [firstPoint] {}; & & \node (p0-2) [terminal] {repeat}; & \node (p0-3) [nonterminal] {\nonTerminalSymbol{syntax\_instruction\_list}{84}}; & \node (p0-4) [point] {}; & \node (p0-5) [terminal] {while}; & \node (p0-6) [nonterminal] {\nonTerminalSymbol{syntax\_instruction\_list}{84}}; & \node (p0-7) [point] {}; & & \node (p0-9) [terminal] {end}; & \node (p0-10) [point] {}; & \node (p0-11) [point] {}; & \node (p0-12) [point] {}; & \node (p0-13) [lastPoint] {}; & \\
  };
  \draw[->] (P0start) -- (p0-2) ;
  \draw[->] (p0-2) -- (p0-3) ;
  \draw[->] (p0-3) -- (p0-5) ;
  \draw[->] (p0-5) -- (p0-6) ;
  \draw[->] (p1-8) -| (p0-4) ;
  \draw[->] (p0-7) -| (p1-8) ;
  \draw[->] (p0-6) -- (p0-9) ;
  \draw (p0-9) -- (p0-11) ;
  \draw[->] (p0-10) |- (p1-11) ;
  \draw (p0-11) -- (p0-12) ;
  \draw[->] (p1-11) -| (p0-12) ;
  \draw[->] (p0-12) -- (p0-13) ;
\end{tikzpicture}

\ruleSubsection{galgas3SyntaxComponentSyntax}{instruction-select}{18}

\begin{tikzpicture}
  \matrix[column sep=\ruleMatrixColumnSeparation, row sep=\ruleMatrixRowSeparation] {
    & & & & & & & & \node (p1-8) [point] {}; & & & \node (p1-11) [terminal] {\verb=%=attribute}; & \\
    \node (P0start) [firstPoint] {}; & & \node (p0-2) [terminal] {select}; & \node (p0-3) [nonterminal] {\nonTerminalSymbol{syntax\_instruction\_list}{84}}; & \node (p0-4) [point] {}; & \node (p0-5) [terminal] {or}; & \node (p0-6) [nonterminal] {\nonTerminalSymbol{syntax\_instruction\_list}{84}}; & \node (p0-7) [point] {}; & & \node (p0-9) [terminal] {end}; & \node (p0-10) [point] {}; & \node (p0-11) [point] {}; & \node (p0-12) [point] {}; & \node (p0-13) [lastPoint] {}; & \\
  };
  \draw[->] (P0start) -- (p0-2) ;
  \draw[->] (p0-2) -- (p0-3) ;
  \draw[->] (p0-3) -- (p0-5) ;
  \draw[->] (p0-5) -- (p0-6) ;
  \draw[->] (p1-8) -| (p0-4) ;
  \draw[->] (p0-7) -| (p1-8) ;
  \draw[->] (p0-6) -- (p0-9) ;
  \draw (p0-9) -- (p0-11) ;
  \draw[->] (p0-10) |- (p1-11) ;
  \draw (p0-11) -- (p0-12) ;
  \draw[->] (p1-11) -| (p0-12) ;
  \draw[->] (p0-12) -- (p0-13) ;
\end{tikzpicture}

\ruleSubsection{galgas3SyntaxComponentSyntax}{instruction-terminal}{21}

\begin{tikzpicture}
  \matrix[column sep=\ruleMatrixColumnSeparation, row sep=\ruleMatrixRowSeparation] {
    & & & & & & \node (p1-6) [terminal] {:>}; & \node (p1-7) [nonterminal] {\nonTerminalSymbol{syntax\_directed\_translation\_result}{18}}; & \node (p1-8) [nonterminal] {\nonTerminalSymbol{syntax\_directed\_translation\_result}{18}}; & \\
    \node (P0start) [firstPoint] {}; & & \node (p0-2) [terminal] {\$terminal\$}; & \node (p0-3) [nonterminal] {\nonTerminalSymbol{actual\_input\_parameter\_list}{19}}; & \node (p0-4) [nonterminal] {\nonTerminalSymbol{terminal\_instruction\_indexing}{86}}; & \node (p0-5) [point] {}; & \node (p0-6) [point] {}; & & & \node (p0-9) [point] {}; & \node (p0-10) [lastPoint] {}; & \\
  };
  \draw[->] (P0start) -- (p0-2) ;
  \draw[->] (p0-2) -- (p0-3) ;
  \draw[->] (p0-3) -- (p0-4) ;
  \draw (p0-4) -- (p0-6) ;
  \draw[->] (p0-5) |- (p1-6) ;
  \draw[->] (p1-6) -- (p1-7) ;
  \draw[->] (p1-7) -- (p1-8) ;
  \draw (p0-6) -- (p0-9) ;
  \draw[->] (p1-8) -| (p0-9) ;
  \draw[->] (p0-9) -- (p0-10) ;
\end{tikzpicture}

\ruleSubsection{galgas3SyntaxComponentSyntax}{instruction-parse-loop}{22}

\begin{tikzpicture}
  \matrix[column sep=\ruleMatrixColumnSeparation, row sep=\ruleMatrixRowSeparation] {
    & & & & & & & & & & & \node (p1-11) [terminal] {\verb=%=attribute}; & \\
    \node (P0start) [firstPoint] {}; & & \node (p0-2) [terminal] {parse}; & \node (p0-3) [terminal] {loop}; & \node (p0-4) [nonterminal] {\nonTerminalSymbol{expression}{1}}; & \node (p0-5) [terminal] {while}; & \node (p0-6) [nonterminal] {\nonTerminalSymbol{expression}{1}}; & \node (p0-7) [terminal] {do}; & \node (p0-8) [nonterminal] {\nonTerminalSymbol{syntax\_instruction\_list}{84}}; & \node (p0-9) [terminal] {end}; & \node (p0-10) [point] {}; & \node (p0-11) [point] {}; & \node (p0-12) [point] {}; & \node (p0-13) [lastPoint] {}; & \\
  };
  \draw[->] (P0start) -- (p0-2) ;
  \draw[->] (p0-2) -- (p0-3) ;
  \draw[->] (p0-3) -- (p0-4) ;
  \draw[->] (p0-4) -- (p0-5) ;
  \draw[->] (p0-5) -- (p0-6) ;
  \draw[->] (p0-6) -- (p0-7) ;
  \draw[->] (p0-7) -- (p0-8) ;
  \draw[->] (p0-8) -- (p0-9) ;
  \draw (p0-9) -- (p0-11) ;
  \draw[->] (p0-10) |- (p1-11) ;
  \draw (p0-11) -- (p0-12) ;
  \draw[->] (p1-11) -| (p0-12) ;
  \draw[->] (p0-12) -- (p0-13) ;
\end{tikzpicture}

\ruleSubsection{galgas3SyntaxComponentSyntax}{instruction-parse-rewind}{18}

\begin{tikzpicture}
  \matrix[column sep=\ruleMatrixColumnSeparation, row sep=\ruleMatrixRowSeparation] {
    & & & & & & & & & \node (p1-9) [point] {}; & & & \node (p1-12) [terminal] {\verb=%=attribute}; & \\
    \node (P0start) [firstPoint] {}; & & \node (p0-2) [terminal] {parse}; & \node (p0-3) [terminal] {do}; & \node (p0-4) [nonterminal] {\nonTerminalSymbol{syntax\_instruction\_list}{84}}; & \node (p0-5) [point] {}; & \node (p0-6) [terminal] {rewind}; & \node (p0-7) [nonterminal] {\nonTerminalSymbol{syntax\_instruction\_list}{84}}; & \node (p0-8) [point] {}; & & \node (p0-10) [terminal] {end}; & \node (p0-11) [point] {}; & \node (p0-12) [point] {}; & \node (p0-13) [point] {}; & \node (p0-14) [lastPoint] {}; & \\
  };
  \draw[->] (P0start) -- (p0-2) ;
  \draw[->] (p0-2) -- (p0-3) ;
  \draw[->] (p0-3) -- (p0-4) ;
  \draw[->] (p0-4) -- (p0-6) ;
  \draw[->] (p0-6) -- (p0-7) ;
  \draw[->] (p1-9) -| (p0-5) ;
  \draw[->] (p0-8) -| (p1-9) ;
  \draw[->] (p0-7) -- (p0-10) ;
  \draw (p0-10) -- (p0-12) ;
  \draw[->] (p0-11) |- (p1-12) ;
  \draw (p0-12) -- (p0-13) ;
  \draw[->] (p1-12) -| (p0-13) ;
  \draw[->] (p0-13) -- (p0-14) ;
\end{tikzpicture}

\ruleSubsection{galgas3SyntaxComponentSyntax}{instruction-parse-when}{54}

\begin{tikzpicture}
  \matrix[column sep=\ruleMatrixColumnSeparation, row sep=\ruleMatrixRowSeparation] {
    & & & & & & & & & & \node (p1-10) [terminal] {\verb=%=attribute}; & \\
    \node (P0start) [firstPoint] {}; & & \node (p0-2) [terminal] {parse}; & \node (p0-3) [terminal] {with}; & \node (p0-4) [nonterminal] {\nonTerminalSymbol{expression}{1}}; & \node (p0-5) [terminal] {:}; & \node (p0-6) [nonterminal] {\nonTerminalSymbol{syntax\_instruction\_list}{84}}; & \node (p0-7) [nonterminal] {\nonTerminalSymbol{branchOfParseWhithInstruction}{87}}; & \node (p0-8) [terminal] {end}; & \node (p0-9) [point] {}; & \node (p0-10) [point] {}; & \node (p0-11) [point] {}; & \node (p0-12) [lastPoint] {}; & \\
  };
  \draw[->] (P0start) -- (p0-2) ;
  \draw[->] (p0-2) -- (p0-3) ;
  \draw[->] (p0-3) -- (p0-4) ;
  \draw[->] (p0-4) -- (p0-5) ;
  \draw[->] (p0-5) -- (p0-6) ;
  \draw[->] (p0-6) -- (p0-7) ;
  \draw[->] (p0-7) -- (p0-8) ;
  \draw (p0-8) -- (p0-10) ;
  \draw[->] (p0-9) |- (p1-10) ;
  \draw (p0-10) -- (p0-11) ;
  \draw[->] (p1-10) -| (p0-11) ;
  \draw[->] (p0-11) -- (p0-12) ;
\end{tikzpicture}

\ruleSubsection{galgas3SyntaxComponentSyntax}{instruction-syntax-send}{17}

\begin{tikzpicture}
  \matrix[column sep=\ruleMatrixColumnSeparation, row sep=\ruleMatrixRowSeparation] {
    \node (P0start) [firstPoint] {}; & & \node (p0-2) [terminal] {send}; & \node (p0-3) [nonterminal] {\nonTerminalSymbol{expression}{1}}; & \node (p0-4) [lastPoint] {}; & \\
  };
  \draw[->] (P0start) -- (p0-2) ;
  \draw[->] (p0-2) -- (p0-3) ;
  \draw[->] (p0-3) -- (p0-4) ;
\end{tikzpicture}

\nonTerminalSection{syntax\_instruction\_list}{84}

\ruleSubsection{galgas3SyntaxComponentSyntax}{galgas3SyntaxComponentSyntax}{189}

\begin{tikzpicture}
  \matrix[column sep=\ruleMatrixColumnSeparation, row sep=\ruleMatrixRowSeparation] {
    & & & & & & \node (p4-6) [point] {}; & \\
    & & & & & \node (p3-5) [nonterminal] {\nonTerminalSymbol{syntax\_instruction}{85}}; & \\
    & & & & & \node (p2-5) [nonterminal] {\nonTerminalSymbol{semantic\_instruction}{17}}; & \\
    & & & & & \node (p1-5) [terminal] {;}; & \\
    \node (P0start) [firstPoint] {}; & & \node (p0-2) [point] {}; & \node (p0-3) [point] {}; & \node (p0-4) [point] {}; & & & \node (p0-7) [lastPoint] {}; & \\
  };
  \draw (P0start) -- (p0-3) ;
  \draw[->] (p0-4) |- (p1-5) ;
  \draw[->] (p0-4) |- (p2-5) ;
  \draw[->] (p0-4) |- (p3-5) ;
  \draw[->] (p4-6) -| (p0-2) ;
  \draw[->] (p1-5) -| (p4-6) ;
  \draw[->] (p2-5) -| (p4-6) ;
  \draw[->] (p3-5) -| (p4-6) ;
  \draw[->] (p0-3) -- (p0-7) ;
\end{tikzpicture}

\nonTerminalSection{syntax\_rule\_declaration}{83}

\ruleSubsection{galgas3SyntaxComponentSyntax}{galgas3SyntaxComponentSyntax}{167}

\begin{tikzpicture}
  \matrix[column sep=\ruleMatrixColumnSeparation, row sep=\ruleMatrixRowSeparation] {
    & & & & & & & & & \node (p2-9) [point] {}; & \\
    & & & & & & & \node (p1-7) [terminal] {label}; & \node (p1-8) [terminal] {identifier}; & \\
    \node (P0start) [firstPoint] {}; & & \node (p0-2) [terminal] {rule}; & \node (p0-3) [terminal] {<non\_terminal>}; & \node (p0-4) [point] {}; & \node (p0-5) [nonterminal] {\nonTerminalSymbol{syntax\_rule\_label}{82}}; & \node (p0-6) [point] {}; & & & & \node (p0-10) [lastPoint] {}; & \\
  };
  \draw[->] (P0start) -- (p0-2) ;
  \draw[->] (p0-2) -- (p0-3) ;
  \draw[->] (p0-3) -- (p0-5) ;
  \draw[->] (p0-6) |- (p1-7) ;
  \draw[->] (p1-7) -- (p1-8) ;
  \draw[->] (p2-9) -| (p0-4) ;
  \draw[->] (p1-8) -| (p2-9) ;
  \draw[->] (p0-5) -- (p0-10) ;
\end{tikzpicture}

\nonTerminalSection{syntax\_rule\_label}{82}

\ruleSubsection{galgas3SyntaxComponentSyntax}{galgas3SyntaxComponentSyntax}{148}

\begin{tikzpicture}
  \matrix[column sep=\ruleMatrixColumnSeparation, row sep=\ruleMatrixRowSeparation] {
    \node (P0start) [firstPoint] {}; & & \node (p0-2) [nonterminal] {\nonTerminalSymbol{formal\_parameter\_list}{11}}; & \node (p0-3) [terminal] {\{}; & \node (p0-4) [nonterminal] {\nonTerminalSymbol{syntax\_instruction\_list}{84}}; & \node (p0-5) [terminal] {\}}; & \node (p0-6) [lastPoint] {}; & \\
  };
  \draw[->] (P0start) -- (p0-2) ;
  \draw[->] (p0-2) -- (p0-3) ;
  \draw[->] (p0-3) -- (p0-4) ;
  \draw[->] (p0-4) -- (p0-5) ;
  \draw[->] (p0-5) -- (p0-6) ;
\end{tikzpicture}

\nonTerminalSection{template\_delimitor}{54}

\ruleSubsection{galgas3LexiqueComponentSyntax}{galgas3LexiqueComponentSyntax}{116}

\begin{tikzpicture}
  \matrix[column sep=\ruleMatrixColumnSeparation, row sep=\ruleMatrixRowSeparation] {
    & & & & & & & & \node (p2-8) [point] {}; & \\
    & & & & & & & \node (p1-7) [terminal] {\verb=%=attribute}; & \\
    \node (P0start) [firstPoint] {}; & & \node (p0-2) [terminal] {template}; & \node (p0-3) [terminal] {"string"}; & \node (p0-4) [point] {}; & \node (p0-5) [point] {}; & \node (p0-6) [point] {}; & & & \node (p0-9) [terminal] {...}; & \node (p0-10) [terminal] {"string"}; & \node (p0-11) [lastPoint] {}; & \\
  };
  \draw[->] (P0start) -- (p0-2) ;
  \draw[->] (p0-2) -- (p0-3) ;
  \draw (p0-3) -- (p0-5) ;
  \draw[->] (p0-6) |- (p1-7) ;
  \draw[->] (p2-8) -| (p0-4) ;
  \draw[->] (p1-7) -| (p2-8) ;
  \draw[->] (p0-5) -- (p0-9) ;
  \draw[->] (p0-9) -- (p0-10) ;
  \draw[->] (p0-10) -- (p0-11) ;
\end{tikzpicture}

\nonTerminalSection{template\_replacement}{55}

\ruleSubsection{galgas3LexiqueComponentSyntax}{galgas3LexiqueComponentSyntax}{132}

\begin{tikzpicture}
  \matrix[column sep=\ruleMatrixColumnSeparation, row sep=\ruleMatrixRowSeparation] {
    & & & & & \node (p1-5) [terminal] {->}; & \node (p1-6) [terminal] {"string"}; & \\
    \node (P0start) [firstPoint] {}; & & \node (p0-2) [terminal] {replace}; & \node (p0-3) [terminal] {"string"}; & \node (p0-4) [point] {}; & \node (p0-5) [terminal] {...}; & \node (p0-6) [terminal] {"string"}; & \node (p0-7) [terminal] {:}; & \node (p0-8) [terminal] {identifier}; & \node (p0-9) [point] {}; & \node (p0-10) [lastPoint] {}; & \\
  };
  \draw[->] (P0start) -- (p0-2) ;
  \draw[->] (p0-2) -- (p0-3) ;
  \draw[->] (p0-3) -- (p0-5) ;
  \draw[->] (p0-5) -- (p0-6) ;
  \draw[->] (p0-6) -- (p0-7) ;
  \draw[->] (p0-7) -- (p0-8) ;
  \draw[->] (p0-4) |- (p1-5) ;
  \draw[->] (p1-5) -- (p1-6) ;
  \draw (p0-8) -- (p0-9) ;
  \draw[->] (p1-6) -| (p0-9) ;
  \draw[->] (p0-9) -- (p0-10) ;
\end{tikzpicture}

\nonTerminalSection{term}{6}

\ruleSubsection{galgas3ExpressionSyntax}{galgas3ExpressionSyntax}{246}

\begin{tikzpicture}
  \matrix[column sep=\ruleMatrixColumnSeparation, row sep=\ruleMatrixRowSeparation] {
    & & & & & & & & \node (p6-8) [point] {}; & \\
    & & & & & & \node (p5-6) [terminal] {mod}; & \node (p5-7) [nonterminal] {\nonTerminalSymbol{factor}{7}}; & \\
    & & & & & & \node (p4-6) [terminal] {\&/}; & \node (p4-7) [nonterminal] {\nonTerminalSymbol{factor}{7}}; & \\
    & & & & & & \node (p3-6) [terminal] {/}; & \node (p3-7) [nonterminal] {\nonTerminalSymbol{factor}{7}}; & \\
    & & & & & & \node (p2-6) [terminal] {\&*}; & \node (p2-7) [nonterminal] {\nonTerminalSymbol{factor}{7}}; & \\
    & & & & & & \node (p1-6) [terminal] {*}; & \node (p1-7) [nonterminal] {\nonTerminalSymbol{factor}{7}}; & \\
    \node (P0start) [firstPoint] {}; & & \node (p0-2) [nonterminal] {\nonTerminalSymbol{factor}{7}}; & \node (p0-3) [point] {}; & \node (p0-4) [point] {}; & \node (p0-5) [point] {}; & & & & \node (p0-9) [lastPoint] {}; & \\
  };
  \draw[->] (P0start) -- (p0-2) ;
  \draw (p0-2) -- (p0-4) ;
  \draw[->] (p0-5) |- (p1-6) ;
  \draw[->] (p1-6) -- (p1-7) ;
  \draw[->] (p0-5) |- (p2-6) ;
  \draw[->] (p2-6) -- (p2-7) ;
  \draw[->] (p0-5) |- (p3-6) ;
  \draw[->] (p3-6) -- (p3-7) ;
  \draw[->] (p0-5) |- (p4-6) ;
  \draw[->] (p4-6) -- (p4-7) ;
  \draw[->] (p0-5) |- (p5-6) ;
  \draw[->] (p5-6) -- (p5-7) ;
  \draw[->] (p6-8) -| (p0-3) ;
  \draw[->] (p1-7) -| (p6-8) ;
  \draw[->] (p2-7) -| (p6-8) ;
  \draw[->] (p3-7) -| (p6-8) ;
  \draw[->] (p4-7) -| (p6-8) ;
  \draw[->] (p5-7) -| (p6-8) ;
  \draw[->] (p0-4) -- (p0-9) ;
\end{tikzpicture}

\nonTerminalSection{terminal\_declaration}{69}

\ruleSubsection{galgas3LexiqueComponentSyntax}{galgas3LexiqueComponentSyntax}{549}

\begin{tikzpicture}
  \matrix[column sep=\ruleMatrixColumnSeparation, row sep=\ruleMatrixRowSeparation] {
    & & & & & & & & & & & & & & \node (p3-14) [point] {}; & \\
    & & & & & & & & \node (p2-8) [point] {}; & & & & \node (p2-12) [terminal] {\verb=%=attribute}; & \\
    & & & & & & \node (p1-6) [terminal] {!}; & \node (p1-7) [terminal] {identifier}; & & & & & \node (p1-12) [terminal] {style}; & \node (p1-13) [terminal] {identifier}; & \\
    \node (P0start) [firstPoint] {}; & & \node (p0-2) [terminal] {\$terminal\$}; & \node (p0-3) [point] {}; & \node (p0-4) [point] {}; & \node (p0-5) [point] {}; & & & & \node (p0-9) [point] {}; & \node (p0-10) [point] {}; & \node (p0-11) [point] {}; & & & & \node (p0-15) [terminal] {error}; & \node (p0-16) [terminal] {message}; & \node (p0-17) [terminal] {"string"}; & \node (p0-18) [lastPoint] {}; & \\
  };
  \draw[->] (P0start) -- (p0-2) ;
  \draw (p0-2) -- (p0-4) ;
  \draw[->] (p0-5) |- (p1-6) ;
  \draw[->] (p1-6) -- (p1-7) ;
  \draw[->] (p2-8) -| (p0-3) ;
  \draw[->] (p1-7) -| (p2-8) ;
  \draw (p0-4) -- (p0-10) ;
  \draw[->] (p0-11) |- (p1-12) ;
  \draw[->] (p1-12) -- (p1-13) ;
  \draw[->] (p0-11) |- (p2-12) ;
  \draw[->] (p3-14) -| (p0-9) ;
  \draw[->] (p1-13) -| (p3-14) ;
  \draw[->] (p2-12) -| (p3-14) ;
  \draw[->] (p0-10) -- (p0-15) ;
  \draw[->] (p0-15) -- (p0-16) ;
  \draw[->] (p0-16) -- (p0-17) ;
  \draw[->] (p0-17) -- (p0-18) ;
\end{tikzpicture}

\nonTerminalSection{terminal\_instruction\_indexing}{86}

\ruleSubsection{galgas3SyntaxComponentSyntax}{instruction-terminal}{48}

\begin{tikzpicture}
  \matrix[column sep=\ruleMatrixColumnSeparation, row sep=\ruleMatrixRowSeparation] {
    & & & & & & & & & & & \node (p3-11) [point] {}; & \\
    & & & & & & & \node (p2-7) [terminal] {"string"}; & & & \node (p2-10) [terminal] {,}; & \\
    & & & \node (p1-3) [terminal] {indexing}; & \node (p1-4) [point] {}; & \node (p1-5) [terminal] {identifier}; & \node (p1-6) [point] {}; & \node (p1-7) [point] {}; & \node (p1-8) [point] {}; & \node (p1-9) [point] {}; & \\
    \node (P0start) [firstPoint] {}; & & \node (p0-2) [point] {}; & \node (p0-3) [point] {}; & & & & & & & & & \node (p0-12) [point] {}; & \node (p0-13) [lastPoint] {}; & \\
  };
  \draw (P0start) -- (p0-3) ;
  \draw[->] (p0-2) |- (p1-3) ;
  \draw[->] (p1-3) -- (p1-5) ;
  \draw (p1-5) -- (p1-7) ;
  \draw[->] (p1-6) |- (p2-7) ;
  \draw (p1-7) -- (p1-8) ;
  \draw[->] (p2-7) -| (p1-8) ;
  \draw[->] (p1-9) |- (p2-10) ;
  \draw[->] (p3-11) -| (p1-4) ;
  \draw[->] (p2-10) -| (p3-11) ;
  \draw (p0-3) -- (p0-12) ;
  \draw[->] (p1-8) -| (p0-12) ;
  \draw[->] (p0-12) -- (p0-13) ;
\end{tikzpicture}

\nonTerminalSection{with\_instruction\_core}{31}

\ruleSubsection{galgas3InstructionsSyntax}{instruction-with}{68}

\begin{tikzpicture}
  \matrix[column sep=\ruleMatrixColumnSeparation, row sep=\ruleMatrixRowSeparation] {
    & & & & \node (p1-4) [terminal] {error}; & \node (p1-5) [terminal] {message}; & \node (p1-6) [terminal] {identifier}; & & & & & \node (p1-11) [terminal] {else}; & \node (p1-12) [nonterminal] {\nonTerminalSymbol{semantic\_instruction\_list}{14}}; & \\
    \node (P0start) [firstPoint] {}; & & \node (p0-2) [nonterminal] {\nonTerminalSymbol{expression}{1}}; & \node (p0-3) [point] {}; & \node (p0-4) [point] {}; & & & \node (p0-7) [point] {}; & \node (p0-8) [terminal] {do}; & \node (p0-9) [nonterminal] {\nonTerminalSymbol{semantic\_instruction\_list}{14}}; & \node (p0-10) [point] {}; & \node (p0-11) [point] {}; & & \node (p0-13) [point] {}; & \node (p0-14) [lastPoint] {}; & \\
  };
  \draw[->] (P0start) -- (p0-2) ;
  \draw (p0-2) -- (p0-4) ;
  \draw[->] (p0-3) |- (p1-4) ;
  \draw[->] (p1-4) -- (p1-5) ;
  \draw[->] (p1-5) -- (p1-6) ;
  \draw (p0-4) -- (p0-7) ;
  \draw[->] (p1-6) -| (p0-7) ;
  \draw[->] (p0-7) -- (p0-8) ;
  \draw[->] (p0-8) -- (p0-9) ;
  \draw (p0-9) -- (p0-11) ;
  \draw[->] (p0-10) |- (p1-11) ;
  \draw[->] (p1-11) -- (p1-12) ;
  \draw (p0-11) -- (p0-13) ;
  \draw[->] (p1-12) -| (p0-13) ;
  \draw[->] (p0-13) -- (p0-14) ;
\end{tikzpicture}

\ruleSubsection{galgas3InstructionsSyntax}{instruction-with}{115}

\begin{tikzpicture}
  \matrix[column sep=\ruleMatrixColumnSeparation, row sep=\ruleMatrixRowSeparation] {
    & & & & & & & & & \node (p2-9) [point] {}; & \\
    & & & & & & & \node (p1-7) [terminal] {.}; & \node (p1-8) [terminal] {identifier}; & & & \node (p1-11) [terminal] {error}; & \node (p1-12) [terminal] {message}; & \node (p1-13) [terminal] {identifier}; & & & & & \node (p1-18) [terminal] {else}; & \node (p1-19) [nonterminal] {\nonTerminalSymbol{semantic\_instruction\_list}{14}}; & \\
    \node (P0start) [firstPoint] {}; & & \node (p0-2) [terminal] {!?}; & \node (p0-3) [terminal] {identifier}; & \node (p0-4) [point] {}; & \node (p0-5) [point] {}; & \node (p0-6) [point] {}; & & & & \node (p0-10) [point] {}; & \node (p0-11) [point] {}; & & & \node (p0-14) [point] {}; & \node (p0-15) [terminal] {do}; & \node (p0-16) [nonterminal] {\nonTerminalSymbol{semantic\_instruction\_list}{14}}; & \node (p0-17) [point] {}; & \node (p0-18) [point] {}; & & \node (p0-20) [point] {}; & \node (p0-21) [lastPoint] {}; & \\
  };
  \draw[->] (P0start) -- (p0-2) ;
  \draw[->] (p0-2) -- (p0-3) ;
  \draw (p0-3) -- (p0-5) ;
  \draw[->] (p0-6) |- (p1-7) ;
  \draw[->] (p1-7) -- (p1-8) ;
  \draw[->] (p2-9) -| (p0-4) ;
  \draw[->] (p1-8) -| (p2-9) ;
  \draw (p0-5) -- (p0-11) ;
  \draw[->] (p0-10) |- (p1-11) ;
  \draw[->] (p1-11) -- (p1-12) ;
  \draw[->] (p1-12) -- (p1-13) ;
  \draw (p0-11) -- (p0-14) ;
  \draw[->] (p1-13) -| (p0-14) ;
  \draw[->] (p0-14) -- (p0-15) ;
  \draw[->] (p0-15) -- (p0-16) ;
  \draw (p0-16) -- (p0-18) ;
  \draw[->] (p0-17) |- (p1-18) ;
  \draw[->] (p1-18) -- (p1-19) ;
  \draw (p0-18) -- (p0-20) ;
  \draw[->] (p1-19) -| (p0-20) ;
  \draw[->] (p0-20) -- (p0-21) ;
\end{tikzpicture}




\renewcommand\nonTerminalSection[2]{\section{Non terminal \texttt{\it#1}}\label{nt2:#2}}
\renewcommand\nonTerminalSymbol[2]{\hyperref[nt2:#2]{#1}}
\renewcommand\startSymbol[2]{L'axiome de la grammaire est \hyperref[nt2:#2]{#1}.}
\renewcommand\nonTerminalSummary[2]{\hyperref[nt2:#2]{#1}}

\chapterLabel{Grammaire des fichiers de template}{grammaireTemplate}

\startSymbol{template\_parser\_start\_symbol}{9}

\nonTerminalSummaryStart \nonTerminalSummary{expression}{0}\nonTerminalSummarySeparator \nonTerminalSummary{factor}{5}\nonTerminalSummarySeparator \nonTerminalSummary{for\_instruction\_element}{10}\nonTerminalSummarySeparator \nonTerminalSummary{for\_instruction\_enumerated\_object}{11}\nonTerminalSummarySeparator \nonTerminalSummary{output\_expression\_list}{7}\nonTerminalSummarySeparator \nonTerminalSummary{primary}{6}\nonTerminalSummarySeparator \nonTerminalSummary{relation\_factor}{2}\nonTerminalSummarySeparator \nonTerminalSummary{relation\_term}{1}\nonTerminalSummarySeparator \nonTerminalSummary{simple\_expression}{3}\nonTerminalSummarySeparator \nonTerminalSummary{switch\_case}{12}\nonTerminalSummarySeparator \nonTerminalSummary{template\_instruction}{8}\nonTerminalSummarySeparator \nonTerminalSummary{template\_parser\_start\_symbol}{9}\nonTerminalSummarySeparator \nonTerminalSummary{term}{4}\nonTerminalSummaryEnd \nonTerminalSection{expression}{0}

\ruleSubsection{templateSyntax}{templateSyntax}{26}

\begin{tikzpicture}
  \matrix[column sep=\ruleMatrixColumnSeparation, row sep=\ruleMatrixRowSeparation] {
    & & & & & & & & \node (p3-8) [point] {}; & \\
    & & & & & & \node (p2-6) [terminal] {\verb=^=}; & \node (p2-7) [nonterminal] {\nonTerminalSymbol{relation\_term}{1}}; & \\
    & & & & & & \node (p1-6) [terminal] {|}; & \node (p1-7) [nonterminal] {\nonTerminalSymbol{relation\_term}{1}}; & \\
    \node (P0start) [firstPoint] {}; & & \node (p0-2) [nonterminal] {\nonTerminalSymbol{relation\_term}{1}}; & \node (p0-3) [point] {}; & \node (p0-4) [point] {}; & \node (p0-5) [point] {}; & & & & \node (p0-9) [lastPoint] {}; & \\
  };
  \draw[->] (P0start) -- (p0-2) ;
  \draw (p0-2) -- (p0-4) ;
  \draw[->] (p0-5) |- (p1-6) ;
  \draw[->] (p1-6) -- (p1-7) ;
  \draw[->] (p0-5) |- (p2-6) ;
  \draw[->] (p2-6) -- (p2-7) ;
  \draw[->] (p3-8) -| (p0-3) ;
  \draw[->] (p1-7) -| (p3-8) ;
  \draw[->] (p2-7) -| (p3-8) ;
  \draw[->] (p0-4) -- (p0-9) ;
\end{tikzpicture}

\nonTerminalSection{factor}{5}

\ruleSubsection{templateSyntax}{templateSyntax}{202}

\begin{tikzpicture}
  \matrix[column sep=\ruleMatrixColumnSeparation, row sep=\ruleMatrixRowSeparation] {
    & & & & & & & & \node (p2-8) [point] {}; & \\
    & & & & & & \node (p1-6) [terminal] {.}; & \node (p1-7) [terminal] {identifier}; & \\
    \node (P0start) [firstPoint] {}; & & \node (p0-2) [nonterminal] {\nonTerminalSymbol{primary}{6}}; & \node (p0-3) [point] {}; & \node (p0-4) [point] {}; & \node (p0-5) [point] {}; & & & & \node (p0-9) [lastPoint] {}; & \\
  };
  \draw[->] (P0start) -- (p0-2) ;
  \draw (p0-2) -- (p0-4) ;
  \draw[->] (p0-5) |- (p1-6) ;
  \draw[->] (p1-6) -- (p1-7) ;
  \draw[->] (p2-8) -| (p0-3) ;
  \draw[->] (p1-7) -| (p2-8) ;
  \draw[->] (p0-4) -- (p0-9) ;
\end{tikzpicture}

\ruleSubsection{templateSyntax}{templateSyntax}{218}

\begin{tikzpicture}
  \matrix[column sep=\ruleMatrixColumnSeparation, row sep=\ruleMatrixRowSeparation] {
    \node (P0start) [firstPoint] {}; & & \node (p0-2) [terminal] {-}; & \node (p0-3) [nonterminal] {\nonTerminalSymbol{factor}{5}}; & \node (p0-4) [lastPoint] {}; & \\
  };
  \draw[->] (P0start) -- (p0-2) ;
  \draw[->] (p0-2) -- (p0-3) ;
  \draw[->] (p0-3) -- (p0-4) ;
\end{tikzpicture}

\ruleSubsection{templateSyntax}{templateSyntax}{232}

\begin{tikzpicture}
  \matrix[column sep=\ruleMatrixColumnSeparation, row sep=\ruleMatrixRowSeparation] {
    \node (P0start) [firstPoint] {}; & & \node (p0-2) [terminal] {not}; & \node (p0-3) [nonterminal] {\nonTerminalSymbol{factor}{5}}; & \node (p0-4) [lastPoint] {}; & \\
  };
  \draw[->] (P0start) -- (p0-2) ;
  \draw[->] (p0-2) -- (p0-3) ;
  \draw[->] (p0-3) -- (p0-4) ;
\end{tikzpicture}

\ruleSubsection{templateSyntax}{templateSyntax}{246}

\begin{tikzpicture}
  \matrix[column sep=\ruleMatrixColumnSeparation, row sep=\ruleMatrixRowSeparation] {
    \node (P0start) [firstPoint] {}; & & \node (p0-2) [terminal] {$\sim$}; & \node (p0-3) [nonterminal] {\nonTerminalSymbol{factor}{5}}; & \node (p0-4) [lastPoint] {}; & \\
  };
  \draw[->] (P0start) -- (p0-2) ;
  \draw[->] (p0-2) -- (p0-3) ;
  \draw[->] (p0-3) -- (p0-4) ;
\end{tikzpicture}

\nonTerminalSection{for\_instruction\_element}{10}

\ruleSubsection{templateSyntax}{template-for-instruction}{31}

\begin{tikzpicture}
  \matrix[column sep=\ruleMatrixColumnSeparation, row sep=\ruleMatrixRowSeparation] {
    \node (P0start) [firstPoint] {}; & & \node (p0-2) [terminal] {literalInt}; & \node (p0-3) [terminal] {*}; & \node (p0-4) [lastPoint] {}; & \\
  };
  \draw[->] (P0start) -- (p0-2) ;
  \draw[->] (p0-2) -- (p0-3) ;
  \draw[->] (p0-3) -- (p0-4) ;
\end{tikzpicture}

\ruleSubsection{templateSyntax}{template-for-instruction}{46}

\begin{tikzpicture}
  \matrix[column sep=\ruleMatrixColumnSeparation, row sep=\ruleMatrixRowSeparation] {
    \node (P0start) [firstPoint] {}; & & \node (p0-2) [terminal] {*}; & \node (p0-3) [lastPoint] {}; & \\
  };
  \draw[->] (P0start) -- (p0-2) ;
  \draw[->] (p0-2) -- (p0-3) ;
\end{tikzpicture}

\ruleSubsection{templateSyntax}{template-for-instruction}{53}

\begin{tikzpicture}
  \matrix[column sep=\ruleMatrixColumnSeparation, row sep=\ruleMatrixRowSeparation] {
    \node (P0start) [firstPoint] {}; & & \node (p0-2) [terminal] {identifier}; & \node (p0-3) [lastPoint] {}; & \\
  };
  \draw[->] (P0start) -- (p0-2) ;
  \draw[->] (p0-2) -- (p0-3) ;
\end{tikzpicture}

\nonTerminalSection{for\_instruction\_enumerated\_object}{11}

\ruleSubsection{templateSyntax}{template-for-instruction}{60}

\begin{tikzpicture}
  \matrix[column sep=\ruleMatrixColumnSeparation, row sep=\ruleMatrixRowSeparation] {
    & & & & & & & \node (p1-7) [terminal] {:}; & \node (p1-8) [terminal] {identifier}; & \\
    \node (P0start) [firstPoint] {}; & & \node (p0-2) [terminal] {(}; & \node (p0-3) [terminal] {)}; & \node (p0-4) [terminal] {in}; & \node (p0-5) [nonterminal] {\nonTerminalSymbol{expression}{0}}; & \node (p0-6) [point] {}; & \node (p0-7) [point] {}; & & \node (p0-9) [point] {}; & \node (p0-10) [lastPoint] {}; & \\
  };
  \draw[->] (P0start) -- (p0-2) ;
  \draw[->] (p0-2) -- (p0-3) ;
  \draw[->] (p0-3) -- (p0-4) ;
  \draw[->] (p0-4) -- (p0-5) ;
  \draw (p0-5) -- (p0-7) ;
  \draw[->] (p0-6) |- (p1-7) ;
  \draw[->] (p1-7) -- (p1-8) ;
  \draw (p0-7) -- (p0-9) ;
  \draw[->] (p1-8) -| (p0-9) ;
  \draw[->] (p0-9) -- (p0-10) ;
\end{tikzpicture}

\ruleSubsection{templateSyntax}{template-for-instruction}{81}

\begin{tikzpicture}
  \matrix[column sep=\ruleMatrixColumnSeparation, row sep=\ruleMatrixRowSeparation] {
    & & & & & & \node (p1-6) [point] {}; & & & & & \node (p1-11) [terminal] {:}; & \node (p1-12) [terminal] {identifier}; & \\
    \node (P0start) [firstPoint] {}; & & \node (p0-2) [terminal] {(}; & \node (p0-3) [point] {}; & \node (p0-4) [nonterminal] {\nonTerminalSymbol{for\_instruction\_element}{10}}; & \node (p0-5) [point] {}; & & \node (p0-7) [terminal] {)}; & \node (p0-8) [terminal] {in}; & \node (p0-9) [nonterminal] {\nonTerminalSymbol{expression}{0}}; & \node (p0-10) [point] {}; & \node (p0-11) [point] {}; & & \node (p0-13) [point] {}; & \node (p0-14) [lastPoint] {}; & \\
  };
  \draw[->] (P0start) -- (p0-2) ;
  \draw[->] (p0-2) -- (p0-4) ;
  \draw[->] (p1-6) -| (p0-3) ;
  \draw[->] (p0-5) -| (p1-6) ;
  \draw[->] (p0-4) -- (p0-7) ;
  \draw[->] (p0-7) -- (p0-8) ;
  \draw[->] (p0-8) -- (p0-9) ;
  \draw (p0-9) -- (p0-11) ;
  \draw[->] (p0-10) |- (p1-11) ;
  \draw[->] (p1-11) -- (p1-12) ;
  \draw (p0-11) -- (p0-13) ;
  \draw[->] (p1-12) -| (p0-13) ;
  \draw[->] (p0-13) -- (p0-14) ;
\end{tikzpicture}

\nonTerminalSection{output\_expression\_list}{7}

\ruleSubsection{templateSyntax}{templateSyntax}{477}

\begin{tikzpicture}
  \matrix[column sep=\ruleMatrixColumnSeparation, row sep=\ruleMatrixRowSeparation] {
    & & & & & & & \node (p2-7) [point] {}; & \\
    & & & & & \node (p1-5) [terminal] {!}; & \node (p1-6) [nonterminal] {\nonTerminalSymbol{expression}{0}}; & \\
    \node (P0start) [firstPoint] {}; & & \node (p0-2) [point] {}; & \node (p0-3) [point] {}; & \node (p0-4) [point] {}; & & & & \node (p0-8) [lastPoint] {}; & \\
  };
  \draw (P0start) -- (p0-3) ;
  \draw[->] (p0-4) |- (p1-5) ;
  \draw[->] (p1-5) -- (p1-6) ;
  \draw[->] (p2-7) -| (p0-2) ;
  \draw[->] (p1-6) -| (p2-7) ;
  \draw[->] (p0-3) -- (p0-8) ;
\end{tikzpicture}

\nonTerminalSection{primary}{6}

\ruleSubsection{templateSyntax}{templateSyntax}{261}

\begin{tikzpicture}
  \matrix[column sep=\ruleMatrixColumnSeparation, row sep=\ruleMatrixRowSeparation] {
    \node (P0start) [firstPoint] {}; & & \node (p0-2) [terminal] {identifier}; & \node (p0-3) [lastPoint] {}; & \\
  };
  \draw[->] (P0start) -- (p0-2) ;
  \draw[->] (p0-2) -- (p0-3) ;
\end{tikzpicture}

\ruleSubsection{templateSyntax}{templateSyntax}{272}

\begin{tikzpicture}
  \matrix[column sep=\ruleMatrixColumnSeparation, row sep=\ruleMatrixRowSeparation] {
    \node (P0start) [firstPoint] {}; & & \node (p0-2) [terminal] {(}; & \node (p0-3) [nonterminal] {\nonTerminalSymbol{expression}{0}}; & \node (p0-4) [terminal] {)}; & \node (p0-5) [lastPoint] {}; & \\
  };
  \draw[->] (P0start) -- (p0-2) ;
  \draw[->] (p0-2) -- (p0-3) ;
  \draw[->] (p0-3) -- (p0-4) ;
  \draw[->] (p0-4) -- (p0-5) ;
\end{tikzpicture}

\ruleSubsection{templateSyntax}{templateSyntax}{284}

\begin{tikzpicture}
  \matrix[column sep=\ruleMatrixColumnSeparation, row sep=\ruleMatrixRowSeparation] {
    \node (P0start) [firstPoint] {}; & & \node (p0-2) [terminal] {true}; & \node (p0-3) [lastPoint] {}; & \\
  };
  \draw[->] (P0start) -- (p0-2) ;
  \draw[->] (p0-2) -- (p0-3) ;
\end{tikzpicture}

\ruleSubsection{templateSyntax}{templateSyntax}{295}

\begin{tikzpicture}
  \matrix[column sep=\ruleMatrixColumnSeparation, row sep=\ruleMatrixRowSeparation] {
    \node (P0start) [firstPoint] {}; & & \node (p0-2) [terminal] {false}; & \node (p0-3) [lastPoint] {}; & \\
  };
  \draw[->] (P0start) -- (p0-2) ;
  \draw[->] (p0-2) -- (p0-3) ;
\end{tikzpicture}

\ruleSubsection{templateSyntax}{templateSyntax}{306}

\begin{tikzpicture}
  \matrix[column sep=\ruleMatrixColumnSeparation, row sep=\ruleMatrixRowSeparation] {
    \node (P0start) [firstPoint] {}; & & \node (p0-2) [terminal] {literalInt}; & \node (p0-3) [lastPoint] {}; & \\
  };
  \draw[->] (P0start) -- (p0-2) ;
  \draw[->] (p0-2) -- (p0-3) ;
\end{tikzpicture}

\ruleSubsection{templateSyntax}{templateSyntax}{317}

\begin{tikzpicture}
  \matrix[column sep=\ruleMatrixColumnSeparation, row sep=\ruleMatrixRowSeparation] {
    \node (P0start) [firstPoint] {}; & & \node (p0-2) [terminal] {double.xxx}; & \node (p0-3) [lastPoint] {}; & \\
  };
  \draw[->] (P0start) -- (p0-2) ;
  \draw[->] (p0-2) -- (p0-3) ;
\end{tikzpicture}

\ruleSubsection{templateSyntax}{templateSyntax}{329}

\begin{tikzpicture}
  \matrix[column sep=\ruleMatrixColumnSeparation, row sep=\ruleMatrixRowSeparation] {
    \node (P0start) [firstPoint] {}; & & \node (p0-2) [terminal] {'char'}; & \node (p0-3) [lastPoint] {}; & \\
  };
  \draw[->] (P0start) -- (p0-2) ;
  \draw[->] (p0-2) -- (p0-3) ;
\end{tikzpicture}

\ruleSubsection{templateSyntax}{templateSyntax}{340}

\begin{tikzpicture}
  \matrix[column sep=\ruleMatrixColumnSeparation, row sep=\ruleMatrixRowSeparation] {
    & & & & & \node (p1-5) [point] {}; & \\
    \node (P0start) [firstPoint] {}; & & \node (p0-2) [point] {}; & \node (p0-3) [terminal] {"string"}; & \node (p0-4) [point] {}; & & \node (p0-6) [lastPoint] {}; & \\
  };
  \draw[->] (P0start) -- (p0-3) ;
  \draw[->] (p1-5) -| (p0-2) ;
  \draw[->] (p0-4) -| (p1-5) ;
  \draw[->] (p0-3) -- (p0-6) ;
\end{tikzpicture}

\ruleSubsection{templateSyntax}{templateSyntax}{362}

\begin{tikzpicture}
  \matrix[column sep=\ruleMatrixColumnSeparation, row sep=\ruleMatrixRowSeparation] {
    & & & & & & \node (p5-6) [terminal] {identifier}; & \node (p5-7) [terminal] {.}; & \node (p5-8) [terminal] {identifier}; & \node (p5-9) [terminal] {identifier}; & \\
    & & & & \node (p4-4) [terminal] {option}; & \node (p4-5) [point] {}; & \node (p4-6) [terminal] {.}; & \node (p4-7) [terminal] {identifier}; & \node (p4-8) [terminal] {identifier}; & & \node (p4-10) [point] {}; & \node (p4-11) [terminal] {]}; & \\
    & & & & \node (p3-4) [nonterminal] {\nonTerminalSymbol{expression}{0}}; & \node (p3-5) [terminal] {identifier}; & \node (p3-6) [nonterminal] {\nonTerminalSymbol{output\_expression\_list}{7}}; & \node (p3-7) [terminal] {]}; & \\
    & & & & \node (p2-4) [terminal] {filewrapper}; & \node (p2-5) [terminal] {identifier}; & \node (p2-6) [terminal] {.}; & \node (p2-7) [terminal] {identifier}; & \node (p2-8) [nonterminal] {\nonTerminalSymbol{output\_expression\_list}{7}}; & \node (p2-9) [terminal] {]}; & \\
    & & & & \node (p1-4) [terminal] {@type}; & \node (p1-5) [terminal] {identifier}; & \node (p1-6) [nonterminal] {\nonTerminalSymbol{output\_expression\_list}{7}}; & \node (p1-7) [terminal] {]}; & \\
    \node (P0start) [firstPoint] {}; & & \node (p0-2) [terminal] {[}; & \node (p0-3) [point] {}; & \node (p0-4) [terminal] {template}; & \node (p0-5) [nonterminal] {\nonTerminalSymbol{expression}{0}}; & \node (p0-6) [terminal] {identifier}; & \node (p0-7) [nonterminal] {\nonTerminalSymbol{output\_expression\_list}{7}}; & \node (p0-8) [terminal] {]}; & & & & \node (p0-12) [point] {}; & \node (p0-13) [lastPoint] {}; & \\
  };
  \draw[->] (P0start) -- (p0-2) ;
  \draw[->] (p0-2) -- (p0-4) ;
  \draw[->] (p0-4) -- (p0-5) ;
  \draw[->] (p0-5) -- (p0-6) ;
  \draw[->] (p0-6) -- (p0-7) ;
  \draw[->] (p0-7) -- (p0-8) ;
  \draw[->] (p0-3) |- (p1-4) ;
  \draw[->] (p1-4) -- (p1-5) ;
  \draw[->] (p1-5) -- (p1-6) ;
  \draw[->] (p1-6) -- (p1-7) ;
  \draw[->] (p0-3) |- (p2-4) ;
  \draw[->] (p2-4) -- (p2-5) ;
  \draw[->] (p2-5) -- (p2-6) ;
  \draw[->] (p2-6) -- (p2-7) ;
  \draw[->] (p2-7) -- (p2-8) ;
  \draw[->] (p2-8) -- (p2-9) ;
  \draw[->] (p0-3) |- (p3-4) ;
  \draw[->] (p3-4) -- (p3-5) ;
  \draw[->] (p3-5) -- (p3-6) ;
  \draw[->] (p3-6) -- (p3-7) ;
  \draw[->] (p0-3) |- (p4-4) ;
  \draw[->] (p4-4) -- (p4-6) ;
  \draw[->] (p4-6) -- (p4-7) ;
  \draw[->] (p4-7) -- (p4-8) ;
  \draw[->] (p4-5) |- (p5-6) ;
  \draw[->] (p5-6) -- (p5-7) ;
  \draw[->] (p5-7) -- (p5-8) ;
  \draw[->] (p5-8) -- (p5-9) ;
  \draw (p4-8) -- (p4-10) ;
  \draw[->] (p5-9) -| (p4-10) ;
  \draw[->] (p4-10) -- (p4-11) ;
  \draw (p0-8) -- (p0-12) ;
  \draw[->] (p1-7) -| (p0-12) ;
  \draw[->] (p2-9) -| (p0-12) ;
  \draw[->] (p3-7) -| (p0-12) ;
  \draw[->] (p4-11) -| (p0-12) ;
  \draw[->] (p0-12) -- (p0-13) ;
\end{tikzpicture}

\ruleSubsection{templateSyntax}{templateSyntax}{437}

\begin{tikzpicture}
  \matrix[column sep=\ruleMatrixColumnSeparation, row sep=\ruleMatrixRowSeparation] {
    \node (P0start) [firstPoint] {}; & & \node (p0-2) [terminal] {identifier}; & \node (p0-3) [terminal] {(}; & \node (p0-4) [nonterminal] {\nonTerminalSymbol{output\_expression\_list}{7}}; & \node (p0-5) [terminal] {)}; & \node (p0-6) [lastPoint] {}; & \\
  };
  \draw[->] (P0start) -- (p0-2) ;
  \draw[->] (p0-2) -- (p0-3) ;
  \draw[->] (p0-3) -- (p0-4) ;
  \draw[->] (p0-4) -- (p0-5) ;
  \draw[->] (p0-5) -- (p0-6) ;
\end{tikzpicture}

\ruleSubsection{templateSyntax}{templateSyntax}{447}

\begin{tikzpicture}
  \matrix[column sep=\ruleMatrixColumnSeparation, row sep=\ruleMatrixRowSeparation] {
    & & & & & \node (p2-5) [terminal] {>}; & \\
    & & & & & \node (p1-5) [terminal] {>=}; & \\
    \node (P0start) [firstPoint] {}; & & \node (p0-2) [nonterminal] {\nonTerminalSymbol{primary}{6}}; & \node (p0-3) [terminal] {is}; & \node (p0-4) [point] {}; & \node (p0-5) [terminal] {==}; & \node (p0-6) [point] {}; & \node (p0-7) [terminal] {@type}; & \node (p0-8) [lastPoint] {}; & \\
  };
  \draw[->] (P0start) -- (p0-2) ;
  \draw[->] (p0-2) -- (p0-3) ;
  \draw[->] (p0-3) -- (p0-5) ;
  \draw[->] (p0-4) |- (p1-5) ;
  \draw[->] (p0-4) |- (p2-5) ;
  \draw (p0-5) -- (p0-6) ;
  \draw[->] (p1-5) -| (p0-6) ;
  \draw[->] (p2-5) -| (p0-6) ;
  \draw[->] (p0-6) -- (p0-7) ;
  \draw[->] (p0-7) -- (p0-8) ;
\end{tikzpicture}

\nonTerminalSection{relation\_factor}{2}

\ruleSubsection{templateSyntax}{templateSyntax}{73}

\begin{tikzpicture}
  \matrix[column sep=\ruleMatrixColumnSeparation, row sep=\ruleMatrixRowSeparation] {
    & & & & \node (p6-4) [terminal] {<}; & \node (p6-5) [nonterminal] {\nonTerminalSymbol{simple\_expression}{3}}; & \\
    & & & & \node (p5-4) [terminal] {>}; & \node (p5-5) [nonterminal] {\nonTerminalSymbol{simple\_expression}{3}}; & \\
    & & & & \node (p4-4) [terminal] {>=}; & \node (p4-5) [nonterminal] {\nonTerminalSymbol{simple\_expression}{3}}; & \\
    & & & & \node (p3-4) [terminal] {<=}; & \node (p3-5) [nonterminal] {\nonTerminalSymbol{simple\_expression}{3}}; & \\
    & & & & \node (p2-4) [terminal] {!=}; & \node (p2-5) [nonterminal] {\nonTerminalSymbol{simple\_expression}{3}}; & \\
    & & & & \node (p1-4) [terminal] {==}; & \node (p1-5) [nonterminal] {\nonTerminalSymbol{simple\_expression}{3}}; & \\
    \node (P0start) [firstPoint] {}; & & \node (p0-2) [nonterminal] {\nonTerminalSymbol{simple\_expression}{3}}; & \node (p0-3) [point] {}; & \node (p0-4) [point] {}; & & \node (p0-6) [point] {}; & \node (p0-7) [lastPoint] {}; & \\
  };
  \draw[->] (P0start) -- (p0-2) ;
  \draw (p0-2) -- (p0-4) ;
  \draw[->] (p0-3) |- (p1-4) ;
  \draw[->] (p1-4) -- (p1-5) ;
  \draw[->] (p0-3) |- (p2-4) ;
  \draw[->] (p2-4) -- (p2-5) ;
  \draw[->] (p0-3) |- (p3-4) ;
  \draw[->] (p3-4) -- (p3-5) ;
  \draw[->] (p0-3) |- (p4-4) ;
  \draw[->] (p4-4) -- (p4-5) ;
  \draw[->] (p0-3) |- (p5-4) ;
  \draw[->] (p5-4) -- (p5-5) ;
  \draw[->] (p0-3) |- (p6-4) ;
  \draw[->] (p6-4) -- (p6-5) ;
  \draw (p0-4) -- (p0-6) ;
  \draw[->] (p1-5) -| (p0-6) ;
  \draw[->] (p2-5) -| (p0-6) ;
  \draw[->] (p3-5) -| (p0-6) ;
  \draw[->] (p4-5) -| (p0-6) ;
  \draw[->] (p5-5) -| (p0-6) ;
  \draw[->] (p6-5) -| (p0-6) ;
  \draw[->] (p0-6) -- (p0-7) ;
\end{tikzpicture}

\nonTerminalSection{relation\_term}{1}

\ruleSubsection{templateSyntax}{templateSyntax}{53}

\begin{tikzpicture}
  \matrix[column sep=\ruleMatrixColumnSeparation, row sep=\ruleMatrixRowSeparation] {
    & & & & & & & & \node (p2-8) [point] {}; & \\
    & & & & & & \node (p1-6) [terminal] {\&}; & \node (p1-7) [nonterminal] {\nonTerminalSymbol{relation\_factor}{2}}; & \\
    \node (P0start) [firstPoint] {}; & & \node (p0-2) [nonterminal] {\nonTerminalSymbol{relation\_factor}{2}}; & \node (p0-3) [point] {}; & \node (p0-4) [point] {}; & \node (p0-5) [point] {}; & & & & \node (p0-9) [lastPoint] {}; & \\
  };
  \draw[->] (P0start) -- (p0-2) ;
  \draw (p0-2) -- (p0-4) ;
  \draw[->] (p0-5) |- (p1-6) ;
  \draw[->] (p1-6) -- (p1-7) ;
  \draw[->] (p2-8) -| (p0-3) ;
  \draw[->] (p1-7) -| (p2-8) ;
  \draw[->] (p0-4) -- (p0-9) ;
\end{tikzpicture}

\nonTerminalSection{simple\_expression}{3}

\ruleSubsection{templateSyntax}{templateSyntax}{128}

\begin{tikzpicture}
  \matrix[column sep=\ruleMatrixColumnSeparation, row sep=\ruleMatrixRowSeparation] {
    & & & & & & & & \node (p5-8) [point] {}; & \\
    & & & & & & \node (p4-6) [terminal] {-}; & \node (p4-7) [nonterminal] {\nonTerminalSymbol{term}{4}}; & \\
    & & & & & & \node (p3-6) [terminal] {+}; & \node (p3-7) [nonterminal] {\nonTerminalSymbol{term}{4}}; & \\
    & & & & & & \node (p2-6) [terminal] {>>}; & \node (p2-7) [nonterminal] {\nonTerminalSymbol{term}{4}}; & \\
    & & & & & & \node (p1-6) [terminal] {<<}; & \node (p1-7) [nonterminal] {\nonTerminalSymbol{term}{4}}; & \\
    \node (P0start) [firstPoint] {}; & & \node (p0-2) [nonterminal] {\nonTerminalSymbol{term}{4}}; & \node (p0-3) [point] {}; & \node (p0-4) [point] {}; & \node (p0-5) [point] {}; & & & & \node (p0-9) [lastPoint] {}; & \\
  };
  \draw[->] (P0start) -- (p0-2) ;
  \draw (p0-2) -- (p0-4) ;
  \draw[->] (p0-5) |- (p1-6) ;
  \draw[->] (p1-6) -- (p1-7) ;
  \draw[->] (p0-5) |- (p2-6) ;
  \draw[->] (p2-6) -- (p2-7) ;
  \draw[->] (p0-5) |- (p3-6) ;
  \draw[->] (p3-6) -- (p3-7) ;
  \draw[->] (p0-5) |- (p4-6) ;
  \draw[->] (p4-6) -- (p4-7) ;
  \draw[->] (p5-8) -| (p0-3) ;
  \draw[->] (p1-7) -| (p5-8) ;
  \draw[->] (p2-7) -| (p5-8) ;
  \draw[->] (p3-7) -| (p5-8) ;
  \draw[->] (p4-7) -| (p5-8) ;
  \draw[->] (p0-4) -- (p0-9) ;
\end{tikzpicture}

\nonTerminalSection{switch\_case}{12}

\ruleSubsection{templateSyntax}{template-switch-instruction}{61}

\begin{tikzpicture}
  \matrix[column sep=\ruleMatrixColumnSeparation, row sep=\ruleMatrixRowSeparation] {
    & & & & & & & & & & & & & & & & & & & & \node (p5-20) [point] {}; & \\
    & & & & & & & & & & & & \node (p4-12) [point] {}; & & & \node (p4-15) [terminal] {unused}; & \\
    & & & & & & & & & & & \node (p3-11) [point] {}; & \node (p3-12) [terminal] {@type}; & \node (p3-13) [point] {}; & \node (p3-14) [point] {}; & \node (p3-15) [point] {}; & \node (p3-16) [point] {}; & \node (p3-17) [terminal] {identifier}; & \\
    & & & & & & \node (p2-6) [point] {}; & & & & & \node (p2-11) [terminal] {*}; & \\
    & & & & & \node (p1-5) [terminal] {,}; & & & \node (p1-8) [terminal] {(}; & \node (p1-9) [point] {}; & \node (p1-10) [point] {}; & \node (p1-11) [terminal] {literalInt}; & \node (p1-12) [terminal] {*}; & & & & & & \node (p1-18) [point] {}; & \node (p1-19) [point] {}; & & \node (p1-21) [terminal] {)}; & \\
    \node (P0start) [firstPoint] {}; & & \node (p0-2) [point] {}; & \node (p0-3) [terminal] {identifier}; & \node (p0-4) [point] {}; & & & \node (p0-7) [point] {}; & \node (p0-8) [point] {}; & & & & & & & & & & & & & & \node (p0-22) [point] {}; & \node (p0-23) [lastPoint] {}; & \\
  };
  \draw[->] (P0start) -- (p0-3) ;
  \draw[->] (p0-4) |- (p1-5) ;
  \draw[->] (p2-6) -| (p0-2) ;
  \draw[->] (p1-5) -| (p2-6) ;
  \draw (p0-3) -- (p0-8) ;
  \draw[->] (p0-7) |- (p1-8) ;
  \draw[->] (p1-8) -- (p1-11) ;
  \draw[->] (p1-11) -- (p1-12) ;
  \draw[->] (p1-10) |- (p2-11) ;
  \draw[->] (p1-10) |- (p3-12) ;
  \draw (p3-11) |- (p4-12) ;
  \draw (p3-12) -- (p3-13) ;
  \draw[->] (p4-12) -| (p3-13) ;
  \draw (p3-13) -- (p3-15) ;
  \draw[->] (p3-14) |- (p4-15) ;
  \draw (p3-15) -- (p3-16) ;
  \draw[->] (p4-15) -| (p3-16) ;
  \draw[->] (p3-16) -- (p3-17) ;
  \draw (p1-12) -- (p1-18) ;
  \draw[->] (p2-11) -| (p1-18) ;
  \draw[->] (p3-17) -| (p1-18) ;
  \draw[->] (p5-20) -| (p1-9) ;
  \draw[->] (p1-19) -| (p5-20) ;
  \draw[->] (p1-18) -- (p1-21) ;
  \draw (p0-8) -- (p0-22) ;
  \draw[->] (p1-21) -| (p0-22) ;
  \draw[->] (p0-22) -- (p0-23) ;
\end{tikzpicture}

\nonTerminalSection{template\_instruction}{8}

\ruleSubsection{templateSyntax}{templateSyntax}{493}

\begin{tikzpicture}
  \matrix[column sep=\ruleMatrixColumnSeparation, row sep=\ruleMatrixRowSeparation] {
    \node (P0start) [firstPoint] {}; & & \node (p0-2) [terminal] {!}; & \node (p0-3) [nonterminal] {\nonTerminalSymbol{expression}{0}}; & \node (p0-4) [lastPoint] {}; & \\
  };
  \draw[->] (P0start) -- (p0-2) ;
  \draw[->] (p0-2) -- (p0-3) ;
  \draw[->] (p0-3) -- (p0-4) ;
\end{tikzpicture}

\ruleSubsection{templateSyntax}{templateSyntax}{505}

\begin{tikzpicture}
  \matrix[column sep=\ruleMatrixColumnSeparation, row sep=\ruleMatrixRowSeparation] {
    \node (P0start) [firstPoint] {}; & & \node (p0-2) [terminal] {?\verb=^=}; & \node (p0-3) [lastPoint] {}; & \\
  };
  \draw[->] (P0start) -- (p0-2) ;
  \draw[->] (p0-2) -- (p0-3) ;
\end{tikzpicture}

\ruleSubsection{templateSyntax}{templateSyntax}{512}

\begin{tikzpicture}
  \matrix[column sep=\ruleMatrixColumnSeparation, row sep=\ruleMatrixRowSeparation] {
    \node (P0start) [firstPoint] {}; & & \node (p0-2) [terminal] {!\verb=^=}; & \node (p0-3) [lastPoint] {}; & \\
  };
  \draw[->] (P0start) -- (p0-2) ;
  \draw[->] (p0-2) -- (p0-3) ;
\end{tikzpicture}

\ruleSubsection{templateSyntax}{templateSyntax}{519}

\begin{tikzpicture}
  \matrix[column sep=\ruleMatrixColumnSeparation, row sep=\ruleMatrixRowSeparation] {
    & & & & & & & & & \node (p2-9) [point] {}; & \\
    & & & & & & & & \node (p1-8) [nonterminal] {\nonTerminalSymbol{template\_instruction}{8}}; & \\
    \node (P0start) [firstPoint] {}; & & \node (p0-2) [terminal] {block}; & \node (p0-3) [nonterminal] {\nonTerminalSymbol{expression}{0}}; & \node (p0-4) [terminal] {:}; & \node (p0-5) [point] {}; & \node (p0-6) [point] {}; & \node (p0-7) [point] {}; & & & \node (p0-10) [terminal] {end}; & \node (p0-11) [lastPoint] {}; & \\
  };
  \draw[->] (P0start) -- (p0-2) ;
  \draw[->] (p0-2) -- (p0-3) ;
  \draw[->] (p0-3) -- (p0-4) ;
  \draw (p0-4) -- (p0-6) ;
  \draw[->] (p0-7) |- (p1-8) ;
  \draw[->] (p2-9) -| (p0-5) ;
  \draw[->] (p1-8) -| (p2-9) ;
  \draw[->] (p0-6) -- (p0-10) ;
  \draw[->] (p0-10) -- (p0-11) ;
\end{tikzpicture}

\ruleSubsection{templateSyntax}{templateSyntax}{540}

\begin{tikzpicture}
  \matrix[column sep=\ruleMatrixColumnSeparation, row sep=\ruleMatrixRowSeparation] {
    & & & & & & & & & & & & & \node (p3-13) [point] {}; & & & & & & & \node (p3-20) [point] {}; & \\
    & & & & & & & & & & \node (p2-10) [point] {}; & & & & & & & & & \node (p2-19) [nonterminal] {\nonTerminalSymbol{template\_instruction}{8}}; & \\
    & & & & & & & & & \node (p1-9) [nonterminal] {\nonTerminalSymbol{template\_instruction}{8}}; & & & \node (p1-12) [terminal] {elsif}; & & & \node (p1-15) [terminal] {else}; & \node (p1-16) [point] {}; & \node (p1-17) [point] {}; & \node (p1-18) [point] {}; & \\
    \node (P0start) [firstPoint] {}; & & \node (p0-2) [terminal] {if}; & \node (p0-3) [point] {}; & \node (p0-4) [nonterminal] {\nonTerminalSymbol{expression}{0}}; & \node (p0-5) [terminal] {then}; & \node (p0-6) [point] {}; & \node (p0-7) [point] {}; & \node (p0-8) [point] {}; & & & \node (p0-11) [point] {}; & & & \node (p0-14) [point] {}; & \node (p0-15) [point] {}; & & & & & & \node (p0-21) [point] {}; & \node (p0-22) [terminal] {end}; & \node (p0-23) [lastPoint] {}; & \\
  };
  \draw[->] (P0start) -- (p0-2) ;
  \draw[->] (p0-2) -- (p0-4) ;
  \draw[->] (p0-4) -- (p0-5) ;
  \draw (p0-5) -- (p0-7) ;
  \draw[->] (p0-8) |- (p1-9) ;
  \draw[->] (p2-10) -| (p0-6) ;
  \draw[->] (p1-9) -| (p2-10) ;
  \draw[->] (p0-11) |- (p1-12) ;
  \draw[->] (p3-13) -| (p0-3) ;
  \draw[->] (p1-12) -| (p3-13) ;
  \draw (p0-7) -- (p0-15) ;
  \draw[->] (p0-14) |- (p1-15) ;
  \draw (p1-15) -- (p1-17) ;
  \draw[->] (p1-18) |- (p2-19) ;
  \draw[->] (p3-20) -| (p1-16) ;
  \draw[->] (p2-19) -| (p3-20) ;
  \draw (p0-15) -- (p0-21) ;
  \draw[->] (p1-17) -| (p0-21) ;
  \draw[->] (p0-21) -- (p0-22) ;
  \draw[->] (p0-22) -- (p0-23) ;
\end{tikzpicture}

\ruleSubsection{templateSyntax}{template-for-instruction}{106}

\begin{tikzpicture}
  \matrix[column sep=\ruleMatrixColumnSeparation, row sep=\ruleMatrixRowSeparation] {
    & & & & & & & & & & & & & \node (p3-13) [point] {}; & & & & & & & & & & & & & & & & & & & \node (p3-32) [point] {}; & & & & & & & & \node (p3-40) [point] {}; & \\
    & & & & \node (p2-4) [terminal] {>}; & & & & & & & & \node (p2-12) [nonterminal] {\nonTerminalSymbol{template\_instruction}{8}}; & & & & & & & & & & & & & \node (p2-25) [point] {}; & & & & & & \node (p2-31) [nonterminal] {\nonTerminalSymbol{template\_instruction}{8}}; & & & & & & & & \node (p2-39) [nonterminal] {\nonTerminalSymbol{template\_instruction}{8}}; & \\
    & & & & \node (p1-4) [terminal] {<}; & & & & \node (p1-8) [terminal] {before}; & \node (p1-9) [point] {}; & \node (p1-10) [point] {}; & \node (p1-11) [point] {}; & & & & & & \node (p1-17) [terminal] {(}; & \node (p1-18) [terminal] {identifier}; & \node (p1-19) [terminal] {)}; & & & & & \node (p1-24) [nonterminal] {\nonTerminalSymbol{template\_instruction}{8}}; & & & \node (p1-27) [terminal] {between}; & \node (p1-28) [point] {}; & \node (p1-29) [point] {}; & \node (p1-30) [point] {}; & & & & & \node (p1-35) [terminal] {after}; & \node (p1-36) [point] {}; & \node (p1-37) [point] {}; & \node (p1-38) [point] {}; & \\
    \node (P0start) [firstPoint] {}; & & \node (p0-2) [terminal] {for}; & \node (p0-3) [point] {}; & \node (p0-4) [point] {}; & \node (p0-5) [point] {}; & \node (p0-6) [nonterminal] {\nonTerminalSymbol{for\_instruction\_enumerated\_object}{11}}; & \node (p0-7) [point] {}; & \node (p0-8) [point] {}; & & & & & & \node (p0-14) [point] {}; & \node (p0-15) [terminal] {do}; & \node (p0-16) [point] {}; & \node (p0-17) [point] {}; & & & \node (p0-20) [point] {}; & \node (p0-21) [point] {}; & \node (p0-22) [point] {}; & \node (p0-23) [point] {}; & & & \node (p0-26) [point] {}; & \node (p0-27) [point] {}; & & & & & & \node (p0-33) [point] {}; & \node (p0-34) [point] {}; & \node (p0-35) [point] {}; & & & & & & \node (p0-41) [point] {}; & \node (p0-42) [terminal] {end}; & \node (p0-43) [lastPoint] {}; & \\
  };
  \draw[->] (P0start) -- (p0-2) ;
  \draw (p0-2) -- (p0-4) ;
  \draw[->] (p0-3) |- (p1-4) ;
  \draw[->] (p0-3) |- (p2-4) ;
  \draw (p0-4) -- (p0-5) ;
  \draw[->] (p1-4) -| (p0-5) ;
  \draw[->] (p2-4) -| (p0-5) ;
  \draw[->] (p0-5) -- (p0-6) ;
  \draw (p0-6) -- (p0-8) ;
  \draw[->] (p0-7) |- (p1-8) ;
  \draw (p1-8) -- (p1-10) ;
  \draw[->] (p1-11) |- (p2-12) ;
  \draw[->] (p3-13) -| (p1-9) ;
  \draw[->] (p2-12) -| (p3-13) ;
  \draw (p0-8) -- (p0-14) ;
  \draw[->] (p1-10) -| (p0-14) ;
  \draw[->] (p0-14) -- (p0-15) ;
  \draw (p0-15) -- (p0-17) ;
  \draw[->] (p0-16) |- (p1-17) ;
  \draw[->] (p1-17) -- (p1-18) ;
  \draw[->] (p1-18) -- (p1-19) ;
  \draw (p0-17) -- (p0-20) ;
  \draw[->] (p1-19) -| (p0-20) ;
  \draw (p0-20) -- (p0-22) ;
  \draw[->] (p0-23) |- (p1-24) ;
  \draw[->] (p2-25) -| (p0-21) ;
  \draw[->] (p1-24) -| (p2-25) ;
  \draw (p0-22) -- (p0-27) ;
  \draw[->] (p0-26) |- (p1-27) ;
  \draw (p1-27) -- (p1-29) ;
  \draw[->] (p1-30) |- (p2-31) ;
  \draw[->] (p3-32) -| (p1-28) ;
  \draw[->] (p2-31) -| (p3-32) ;
  \draw (p0-27) -- (p0-33) ;
  \draw[->] (p1-29) -| (p0-33) ;
  \draw (p0-33) -- (p0-35) ;
  \draw[->] (p0-34) |- (p1-35) ;
  \draw (p1-35) -- (p1-37) ;
  \draw[->] (p1-38) |- (p2-39) ;
  \draw[->] (p3-40) -| (p1-36) ;
  \draw[->] (p2-39) -| (p3-40) ;
  \draw (p0-35) -- (p0-41) ;
  \draw[->] (p1-37) -| (p0-41) ;
  \draw[->] (p0-41) -- (p0-42) ;
  \draw[->] (p0-42) -- (p0-43) ;
\end{tikzpicture}

\ruleSubsection{templateSyntax}{template-switch-instruction}{28}

\begin{tikzpicture}
  \matrix[column sep=\ruleMatrixColumnSeparation, row sep=\ruleMatrixRowSeparation] {
    & & & & & & & & & & & & & & & \node (p4-15) [point] {}; & \\
    & & & & & & & & & & & & & & \node (p3-14) [point] {}; & \\
    & & & & & & & & & & & & & \node (p2-13) [nonterminal] {\nonTerminalSymbol{template\_instruction}{8}}; & \\
    & & & & & & & \node (p1-7) [terminal] {case}; & \node (p1-8) [nonterminal] {\nonTerminalSymbol{switch\_case}{12}}; & \node (p1-9) [terminal] {:}; & \node (p1-10) [point] {}; & \node (p1-11) [point] {}; & \node (p1-12) [point] {}; & \\
    \node (P0start) [firstPoint] {}; & & \node (p0-2) [terminal] {switch}; & \node (p0-3) [nonterminal] {\nonTerminalSymbol{expression}{0}}; & \node (p0-4) [point] {}; & \node (p0-5) [point] {}; & \node (p0-6) [point] {}; & & & & & & & & & & \node (p0-16) [terminal] {end}; & \node (p0-17) [lastPoint] {}; & \\
  };
  \draw[->] (P0start) -- (p0-2) ;
  \draw[->] (p0-2) -- (p0-3) ;
  \draw (p0-3) -- (p0-5) ;
  \draw[->] (p0-6) |- (p1-7) ;
  \draw[->] (p1-7) -- (p1-8) ;
  \draw[->] (p1-8) -- (p1-9) ;
  \draw (p1-9) -- (p1-11) ;
  \draw[->] (p1-12) |- (p2-13) ;
  \draw[->] (p3-14) -| (p1-10) ;
  \draw[->] (p2-13) -| (p3-14) ;
  \draw[->] (p4-15) -| (p0-4) ;
  \draw[->] (p1-11) -| (p4-15) ;
  \draw[->] (p0-5) -- (p0-16) ;
  \draw[->] (p0-16) -- (p0-17) ;
\end{tikzpicture}

\nonTerminalSection{template\_parser\_start\_symbol}{9}

\ruleSubsection{templateSyntax}{templateSyntax}{576}

\begin{tikzpicture}
  \matrix[column sep=\ruleMatrixColumnSeparation, row sep=\ruleMatrixRowSeparation] {
    & & & & & & \node (p2-6) [point] {}; & \\
    & & & & & \node (p1-5) [nonterminal] {\nonTerminalSymbol{template\_instruction}{8}}; & \\
    \node (P0start) [firstPoint] {}; & & \node (p0-2) [point] {}; & \node (p0-3) [point] {}; & \node (p0-4) [point] {}; & & & \node (p0-7) [lastPoint] {}; & \\
  };
  \draw (P0start) -- (p0-3) ;
  \draw[->] (p0-4) |- (p1-5) ;
  \draw[->] (p2-6) -| (p0-2) ;
  \draw[->] (p1-5) -| (p2-6) ;
  \draw[->] (p0-3) -- (p0-7) ;
\end{tikzpicture}

\nonTerminalSection{term}{4}

\ruleSubsection{templateSyntax}{templateSyntax}{168}

\begin{tikzpicture}
  \matrix[column sep=\ruleMatrixColumnSeparation, row sep=\ruleMatrixRowSeparation] {
    & & & & & & & & \node (p4-8) [point] {}; & \\
    & & & & & & \node (p3-6) [terminal] {mod}; & \node (p3-7) [nonterminal] {\nonTerminalSymbol{factor}{5}}; & \\
    & & & & & & \node (p2-6) [terminal] {/}; & \node (p2-7) [nonterminal] {\nonTerminalSymbol{factor}{5}}; & \\
    & & & & & & \node (p1-6) [terminal] {*}; & \node (p1-7) [nonterminal] {\nonTerminalSymbol{factor}{5}}; & \\
    \node (P0start) [firstPoint] {}; & & \node (p0-2) [nonterminal] {\nonTerminalSymbol{factor}{5}}; & \node (p0-3) [point] {}; & \node (p0-4) [point] {}; & \node (p0-5) [point] {}; & & & & \node (p0-9) [lastPoint] {}; & \\
  };
  \draw[->] (P0start) -- (p0-2) ;
  \draw (p0-2) -- (p0-4) ;
  \draw[->] (p0-5) |- (p1-6) ;
  \draw[->] (p1-6) -- (p1-7) ;
  \draw[->] (p0-5) |- (p2-6) ;
  \draw[->] (p2-6) -- (p2-7) ;
  \draw[->] (p0-5) |- (p3-6) ;
  \draw[->] (p3-6) -- (p3-7) ;
  \draw[->] (p4-8) -| (p0-3) ;
  \draw[->] (p1-7) -| (p4-8) ;
  \draw[->] (p2-7) -| (p4-8) ;
  \draw[->] (p3-7) -| (p4-8) ;
  \draw[->] (p0-4) -- (p0-9) ;
\end{tikzpicture}


}
