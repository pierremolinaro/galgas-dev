%!TEX encoding = UTF-8 Unicode
%!TEX root = ../galgas-book.tex

\chapitreTypePredefiniLabelIndex{application}

Le \ggs+@application+ ne définit que des constructeurs qui permettent d'obtenir des informations sur le programme courant.


\section{Constructeurs}

\subsectionConstructor{galgasVersionString}{application}

\begin{galgasbox}
constructor @application galgasVersionString -> @string
\end{galgasbox}

Ce constructeur renvoie la version du compilateur GALGAS qui a engendré cet exécutable. Pour le compilateur correspondant à cette documentation, la chaîne renvoyée est \ggs+"GALGASBETAVERSION"+ :
\begin{galgas}
let s = @application.galgasVersionString # "GALGASBETAVERSION"
\end{galgas}








\subsectionConstructor{projectVersionString}{application}

\begin{galgasbox}
constructor @application projectVersionString -> @string
\end{galgasbox}

Ce constructeur renvoie la version du projet GALGAS dont la compilation fournit cet exécutable. C'est l'information qui apparaît après le mot réservé \ggs+project+ (voir \refSubsectionPage{versionProjet}), en utilisant le point « . » comme séparateur. Par exemple, si l'en-tête du projet est :

\begin{galgas}
project (1:2:3) -> "logo" {
  ...
}
\end{galgas}

La chaîne renvoyée est \ggs+"1.2.3"+:
\begin{galgas}
let s = @application.projectVersionString # "1.2.3"
\end{galgas}



