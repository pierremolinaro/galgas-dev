%!TEX encoding = UTF-8 Unicode
%!TEX root = ../galgas-book.tex

\chapitreTypePredefiniLabelIndex{application}

Le type \ggs+@application+ ne définit que des constructeurs qui permettent d'obtenir des informations sur le programme courant et son exécution.


\section{Numéros de version}

\subsectionConstructor{galgasVersionString}{application}

\begin{galgasbox}
constructor @application galgasVersionString -> @string
\end{galgasbox}

Ce constructeur renvoie la version du compilateur GALGAS qui a engendré cet exécutable. Pour le compilateur correspondant à cette documentation, la chaîne renvoyée est \ggs+"GALGASBETAVERSION"+ :

\begin{galgas}
let s = @application.galgasVersionString # "GALGASBETAVERSION"
\end{galgas}








\subsectionConstructor{projectVersionString}{application}

\begin{galgasbox}
constructor @application projectVersionString -> @string
\end{galgasbox}

Ce constructeur renvoie la version du projet GALGAS dont la compilation fournit cet exécutable. C'est l'information qui apparaît après le mot réservé \ggs+project+ (voir \refSubsectionPage{versionProjet}), en utilisant le point « . » comme séparateur. Par exemple, si l'en-tête du projet est :

\begin{galgas}
project (1:2:3) -> "logo" {
  ...
}
\end{galgas}

La chaîne renvoyée est \ggs+"1.2.3"+:
\begin{galgas}
let s = @application.projectVersionString # "1.2.3"
\end{galgas}










\section{Arguments de la ligne de commande}


\subsectionConstructor{commandLineArgumentCount}{application}

\begin{galgasbox}
constructor @application commandLineArgumentCount -> @uint
\end{galgasbox}

Ce constructeur renvoie le nombre d'arguments de la ligne de commande.


\subsectionConstructor{commandLineArgumentAtIndex}{application}

\begin{galgasbox}
constructor @application commandLineArgumentAtIndex ?@uint inIndex
  -> @string
\end{galgasbox}

Ce constructeur renvoie l'argument d'indice \ggs+inIndex+ de la ligne de commande. Les arguments sont indexés à partir de zéro, aussi la valeur de \ggs+inIndex+ doit être strictement inférieur à la valeur retournée par \ggs+@application.commandLineArgumentCount+. Une erreur d'exécution est déclenchée dans le cas contraire.

À titre d'exemple, voici comment imprimer tous les arguments de la ligne de commande~:
\begin{galgas}
for idx in 0 ..< @application.commandLineArgumentCount do
  message "Argument " + idx + ": '"
    + @application.commandLineArgumentAtIndex {!idx} + "'\n"
end
\end{galgas}










\section{Options booléennes de la ligne de commande}

\subsectionConstructor{boolOptionNameList}{application}

\begin{galgasbox}
constructor @application boolOptionNameList -> @2stringlist
\end{galgasbox}

Ce constructeur renvoie la liste des options booléennes définie par l'application, que ces options soient nommées dans la ligne de commande ou non. Chaque option est définie par un couple, son nom de domaine et son identificateur. À titre d'exemple, voici comment imprimer la liste des options booléennes :
\begin{galgas}
for (domain identifier) in @application.boolOptionNameList do
  message "Domain: '" + domain + "', identifier: '" + identifier + "'\n"
end
\end{galgas}


\subsectionConstructor{boolOptionCommentString}{application}

\begin{galgasbox}
constructor @application boolOptionCommentString
    ?@string inDomainName
    ?@string inOptionIdentifier -> @string
\end{galgasbox}

Ce constructeur renvoie la chaîne de commentaires associée à l'option booléenne spécifiée par son nom de domaine et son identificateur. Par exemple :
\begin{galgas}
for (domain identifier) in @application.boolOptionNameList do
  message "Domain: '" + domain + "', identifier: '" + identifier + "'\n"
  message "Comment: '"
    + @application.boolOptionCommentString {!domain !identifier} + "'\n"
end
\end{galgas}

Une erreur d'exécution est déclenchée si l'option n'existe pas, et la chaîne renvoyée n'est pas construite.


\subsectionConstructor{boolOptionInvocationCharacter}{application}

\begin{galgasbox}
constructor @application boolOptionInvocationCharacter
    ?@string inDomainName
    ?@string inOptionIdentifier -> @char
\end{galgasbox}

Ce constructeur renvoie le caractère d'activation associé à l'option booléenne spécifiée par son nom de domaine et son identificateur.

Le caractère d'activation est le caractère qui, précédé de « \texttt{-} » permet l'activation de l'option sur la ligne de commande. Si l'option ne définit pas de caractère d'activation, la valeur renvoyée est \texttt{NUL}.

 Par exemple :
\begin{galgas}
for (domain identifier) in @application.boolOptionNameList do
  message "Domain: '" + domain + "', identifier: '" + identifier + "'\n"
  message "Invocation character: '"
    + @application.boolOptionInvocationCharacter {!domain !identifier} + "'\n"
end
\end{galgas}

Une erreur d'exécution est déclenchée si l'option n'existe pas, et le caractère renvoyé n'est pas construit.


\subsectionConstructor{boolOptionInvocationString}{application}

\begin{galgasbox}
constructor @application boolOptionInvocationString
    ?@string inDomainName
    ?@string inOptionIdentifier -> @string
\end{galgasbox}

Ce constructeur renvoie la chaîne d'activation associée à l'option booléenne spécifiée par son nom de domaine et son identificateur.

La chaîne d'activation est la chaîne qui, précédée de « \texttt{-{}-} » permet l'activation de l'option sur la ligne de commande. Si l'option ne définit pas de chaîne d'activation, la valeur renvoyée est la chaîne vide.

 Par exemple :
\begin{galgas}
for (domain identifier) in @application.boolOptionNameList do
  message "Domain: '" + domain + "', identifier: '" + identifier + "'\n"
  message "Invocation string: '"
    + @application.boolOptionInvocationString {!domain !identifier} + "'\n"
end
\end{galgas}

Une erreur d'exécution est déclenchée si l'option n'existe pas, et la chaîne renvoyée n'est pas construite.











\section{Options entières de la ligne de commande}

\subsectionConstructor{uintOptionNameList}{application}

\begin{galgasbox}
constructor @application uintOptionNameList -> @2stringlist
\end{galgasbox}

Ce constructeur renvoie la liste des options entières définie par l'application, que ces options soient nommées dans la ligne de commande ou non. Chaque option est définie par un couple, son nom de domaine et son identificateur. Par d'exemple :
\begin{galgas}
for (domain identifier) in @application.uintOptionNameList do
  message "Domain: '" + domain + "', identifier: '" + identifier + "'\n"
end
\end{galgas}


\subsectionConstructor{uintOptionCommentString}{application}

\begin{galgasbox}
constructor @application uintOptionCommentString
    ?@string inDomainName
    ?@string inOptionIdentifier -> @string
\end{galgasbox}

Ce constructeur renvoie la chaîne de commentaires associée à l'option entière spécifiée par son nom de domaine et son identificateur. Par exemple :
\begin{galgas}
for (domain identifier) in @application.uintOptionNameList do
  message "Domain: '" + domain + "', identifier: '" + identifier + "'\n"
  message "Comment: '"
    + @application.uintOptionCommentString {!domain !identifier} + "'\n"
end
\end{galgas}

Une erreur d'exécution est déclenchée si l'option n'existe pas, et la chaîne renvoyée n'est pas construite.


\subsectionConstructor{uintOptionInvocationCharacter}{application}

\begin{galgasbox}
constructor @application uintOptionInvocationCharacter
    ?@string inDomainName
    ?@string inOptionIdentifier -> @char
\end{galgasbox}

Ce constructeur renvoie le caractère d'activation associé à l'option entière spécifiée par son nom de domaine et son identificateur.

Le caractère d'activation est le caractère qui, précédé de « \texttt{-} » permet l'activation de l'option sur la ligne de commande. Si l'option ne définit pas de caractère d'activation, la valeur renvoyée est \texttt{NUL}.

 Par exemple :
\begin{galgas}
for (domain identifier) in @application.uintOptionNameList do
  message "Domain: '" + domain + "', identifier: '" + identifier + "'\n"
  message "Invocation character: '"
    + @application.uintOptionInvocationCharacter {!domain !identifier} + "'\n"
end
\end{galgas}

Une erreur d'exécution est déclenchée si l'option n'existe pas, et le caractère renvoyé n'est pas construit.


\subsectionConstructor{uintOptionInvocationString}{application}

\begin{galgasbox}
constructor @application uintOptionInvocationString
    ?@string inDomainName
    ?@string inOptionIdentifier -> @string
\end{galgasbox}

Ce constructeur renvoie la chaîne d'activation associée à l'option entière spécifiée par son nom de domaine et son identificateur.

La chaîne d'activation est la chaîne qui, précédée de « \texttt{-{}-} » permet l'activation de l'option sur la ligne de commande. Si l'option ne définit pas de chaîne d'activation, la valeur renvoyée est la chaîne vide.

 Par exemple :
\begin{galgas}
for (domain identifier) in @application.uintOptionNameList do
  message "Domain: '" + domain + "', identifier: '" + identifier + "'\n"
  message "Invocation string: '"
    + @application.uintOptionInvocationString {!domain !identifier} + "'\n"
end
\end{galgas}

Une erreur d'exécution est déclenchée si l'option n'existe pas, et la chaîne renvoyée n'est pas construite.












\section{Options chaînes de caractères de la ligne de commande}

\subsectionConstructor{stringOptionNameList}{application}

\begin{galgasbox}
constructor @application stringOptionNameList -> @2stringlist
\end{galgasbox}

Ce constructeur renvoie la liste des options chaînes de caractères définie par l'application, que ces options soient nommées dans la ligne de commande ou non. Chaque option est définie par un couple, son nom de domaine et son identificateur. Par d'exemple :
\begin{galgas}
for (domain identifier) in @application.stringOptionNameList do
  message "Domain: '" + domain + "', identifier: '" + identifier + "'\n"
end
\end{galgas}


\subsectionConstructor{stringOptionCommentString}{application}

\begin{galgasbox}
constructor @application stringOptionCommentString
    ?@string inDomainName
    ?@string inOptionIdentifier -> @string
\end{galgasbox}

Ce constructeur renvoie la chaîne de commentaires associée à l'option chaîne de caractères spécifiée par son nom de domaine et son identificateur. Par exemple :
\begin{galgas}
for (domain identifier) in @application.stringOptionNameList do
  message "Domain: '" + domain + "', identifier: '" + identifier + "'\n"
  message "Comment: '"
    + @application.stringOptionCommentString {!domain !identifier} + "'\n"
end
\end{galgas}

Une erreur d'exécution est déclenchée si l'option n'existe pas, et la chaîne renvoyée n'est pas construite.


\subsectionConstructor{stringOptionInvocationCharacter}{application}

\begin{galgasbox}
constructor @application stringOptionInvocationCharacter
    ?@string inDomainName
    ?@string inOptionIdentifier -> @char
\end{galgasbox}

Ce constructeur renvoie le caractère d'activation associé à l'option chaîne de caractères spécifiée par son nom de domaine et son identificateur.

Le caractère d'activation est le caractère qui, précédé de « \texttt{-} » permet l'activation de l'option sur la ligne de commande. Si l'option ne définit pas de caractère d'activation, la valeur renvoyée est \texttt{NUL}.

 Par exemple :
\begin{galgas}
for (domain identifier) in @application.stringOptionNameList do
  message "Domain: '" + domain + "', identifier: '" + identifier + "'\n"
  message "Invocation character: '"
    + @application.uintOptionInvocationCharacter {!domain !identifier} + "'\n"
end
\end{galgas}

Une erreur d'exécution est déclenchée si l'option n'existe pas, et le caractère renvoyé n'est pas construit.


\subsectionConstructor{stringOptionInvocationString}{application}

\begin{galgasbox}
constructor @application stringOptionInvocationString
    ?@string inDomainName
    ?@string inOptionIdentifier -> @string
\end{galgasbox}

Ce constructeur renvoie la chaîne d'activation associée à l'option chaîne de caractères spécifiée par son nom de domaine et son identificateur.

La chaîne d'activation est la chaîne qui, précédée de « \texttt{-{}-} » permet l'activation de l'option sur la ligne de commande. Si l'option ne définit pas de chaîne d'activation, la valeur renvoyée est la chaîne vide.

 Par exemple :
\begin{galgas}
for (domain identifier) in @application.stringOptionNameList do
  message "Domain: '" + domain + "', identifier: '" + identifier + "'\n"
  message "Invocation string: '"
    + @application.stringOptionInvocationString {!domain !identifier} + "'\n"
end
\end{galgas}

Une erreur d'exécution est déclenchée si l'option n'existe pas, et la chaîne renvoyée n'est pas construite.







\sectionConstructor{system}{application}

\begin{galgasbox}
constructor @application system -> @string
\end{galgasbox}

Ce constructeur permet de savoir sur quel type de système l'application tourne en renvoyant la chaîne :
\begin{itemize}
  \item \ggs!"unix"! sur Unix, par exemple OSX ou Linux ;
  \item \ggs!"windows"! sur Windows.
\end{itemize}









\sectionConstructor{verboseOutput}{application}

\begin{galgasbox}
constructor @application verboseOutput -> @bool
\end{galgasbox}

Ce constructeur permet de savoir si l'indicateur de sortie verbeuse est activé ou non.

La sortie verbeuse est controllée par les options de la ligne de commande \emph{quiet} et \emph{verbose} (\refSectionPage{optionsQuietVerbose}) ; leur présence dans le compilateur engendré dépend de la présence de la déclaration \ggs!%quietOutputByDefault!\index{\%quietOutputByDefault} parmi les déclarations du fichier projet (\refSectionPage{projetDeclarationQuietOutputByDefault}).

Les deux options de la ligne de commande \emph{quiet} et \emph{verbose} s'excluent et ne peuvent pas être appelées par la construction \ggs+[option nom_composant_option.nom_option nom_info]+ (voir \refSubsectionPage{appelOption}) : c'est ce constructeur, qui s'adapte à la configuration du compilateur, qu'il faut appeler.

Par exemple :
\begin{galgas}
if @application.verboseOutput then
  # impressions de la sortie verbeuse
end
\end{galgas}














\section{Instrospection des composants lexique}

\subsectionConstructor{keywordIdentifierSet}{application}

\begin{galgasbox}
constructor @application keywordIdentifierSet -> @stringset
\end{galgasbox}


Ce constructeur renvoie l'ensemble des identificateurs des listes de mots réservés définies dans les composants lexiques du projet. Un identificateur est composé du nom du lexique, suivi de «~\texttt{:}~», et du nom de la liste des mots réservés.


Si par exemple un projet définit le composant lexique suivant :

\begin{galgas}
lexique monLexique {
  ...
  list mots1 ... { ... }
  ...
  list mots2 ... { ... }
  ...
}
\end{galgas}

Alors :
\begin{galgas}
let theList = @application.keywordIdentifierSet
log theList # "monLexique:mots1", "monLexique:mots2"
\end{galgas}





\subsectionConstructor{keywordListForIdentifier}{application}

\begin{galgasbox}
constructor @application keywordListForIdentifier
  ?@string inIdentifier
  -> @stringlist
\end{galgasbox}


Ce constructeur renvoie le contenu de la liste désignée par \ggs!inIdentifier!. Si \ggs!inIdentifier! n'est pas une des valeurs renvoyées par le \refConstructorPage{application}{keywordIdentifierSet}, la liste retournée est vide.


Si par exemple un projet définit le composant lexique suivant :

\begin{galgas}
lexique monLexique {
  ...
  list mots ... { "a", "b", "c" }
  ...
}
\end{galgas}

Alors :
\begin{galgas}
let theList = @application.keywordListForIdentifier {!"monLexique:mots"}
log theList # "a", "b", "c"
\end{galgas}


