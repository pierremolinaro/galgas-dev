%!TEX encoding = UTF-8 Unicode
%!TEX root = ../galgas-book.tex

\chapitreTypePredefiniLabelIndex{bool}

Le type \ggs+@bool+ est le type booléen. Les deux mots réservés \ggs+true+ et \ggs+false+ sont du type \ggs+@bool+ type, et dénote les valeurs \emph{vari} et \emph{faux}. Le seul constructeur du \ggs+@bool+ type est le constructeur \ggs!default!, qui initialise un booléen à \ggs+false+.


\section{Conversion en chaîne de caractères}

\subsectionGetter{cString}{bool}

\begin{galgasbox}
getter cString -> @string
\end{galgasbox}

Retourne la chaîne \ggs!"true"! si le booléen est vrai, et la chaîne \ggs!"false"! dans le cas contraire.







\subsectionGetter{ocString}{bool}

\begin{galgasbox}
getter ocString -> @string
\end{galgasbox}

Retourne la chaîne \ggs!"YES"! si le booléen est vrai, et la chaîne \ggs!"NO"! dans le cas contraire.




\section{Conversion en entier}


\subsectionGetter{sint}{bool}

\begin{galgasbox}
getter sint -> @sint
\end{galgasbox}

Retourne l'entier \ggs!1S! si le booléen est vrai, et l'entier \ggs!0S! dans le cas contraire.




\subsectionGetter{sint64}{bool}

\begin{galgasbox}
getter sint64 -> @sint64
\end{galgasbox}

Retourne l'entier \ggs!1LS! si le booléen est vrai, et l'entier \ggs!0LS! dans le cas contraire.




\subsectionGetter{uint}{bool}

\begin{galgasbox}
getter uint -> @uint
\end{galgasbox}

Retourne l'entier \ggs!1! si le booléen est vrai, et l'entier \ggs!0! dans le cas contraire.




\subsectionGetter{uint64}{bool}

\begin{galgasbox}
getter uint64 -> @uint64
\end{galgasbox}

Retourne l'entier \ggs!1L! si le booléen est vrai, et l'entier \ggs!0L! dans le cas contraire.




\section{Opérateurs logiques}

\begin{galgasbox}
operator @bool & @bool -> @bool
operator @bool | @bool -> @bool
operator @bool ^ @bool -> @bool
operator not @bool -> @bool
\end{galgasbox}

Le type \ggs+@bool+ accepte les trois opérateurs suivants
\begin{itemize}
\item l'opérateur \ggs!&! infixé qui effectue un \emph{et logique} ;
\item l'opérateur \ggs!|! infixé qui effectue un \emph{ou logique} ;
\item l'opérateur \ggs!^! infixé qui effectue un \emph{ou exclusif logique} ;
\item l'opérateur \ggs!not! infixe qui effectue la\emph{négation logique}.
\end{itemize}








\section{Comparaison}

Le type \ggs!@bool! implémente les six opérateurs de comparaison \ggs!==!, \ggs+!=+, \ggs!<!, \ggs!<=!, \ggs!>! et \ggs!>=!, avec \ggs!false < true!.
