%!TEX encoding = UTF-8 Unicode
%!TEX root = ../galgas-book.tex

%--------------------------------------------------------------
\chapterLabel{Le type structure}{typeStructure}
%-------------------------------------------------------------

\tableDesMatieresLocaleDeProfondeurRelative{1}



Le mot-clé \ggsq!struct! permet de définir des types de structure. Un objet de type structure a une sémantique de valeur. Une déclaration de structure doit déclarer au moins une propriété. Par exemple :

\begin{galgas34}
struct @MaStructure {
  public var @uint propriété
}
\end{galgas34}

Il est possible d'associer une valeur initiale à la déclaration d'une propriété~:
\begin{galgas34}
struct @MaStructure2 {
  public var @uint propriété = 9
}
\end{galgas34}

Une déclaration de structure peut aussi déclarer~: des initialisateurs (\refSectionPage{initStruct}), des \emph{getters} (\refSectionPage{getterStruct}), des \emph{methodes} (\refSectionPage{methodStruct}) et des \emph{setters} (\refSectionPage{setterStruct}).

Ces déclarations peuvent apparaître soit dans la déclaration de structure, soit comme une extension (\refChapterPage{chapitreExtensions}).












\sectionLabel{initialisateurs}{initStruct}

Lorque que l'on instancie un type structure, on appelle un \emph{initialisateur}. Celui-ci a pour rôle d'attribuer une valeur initiale à toutes les propriétés de la structure instancée.


Toute structure implémente un initialisateur. Si une structure ne déclare aucun initialisateur, un initialisateur synthétisé est automatiquement engendré (\refSubsectionPage{initialisateurDefautStruct}). L'écriture d'un initialisateur est présenté à la \refSubsectionPage{initialisateurStruct}.

\subsectionLabel{Initialisateur synthétisé}{initialisateurDefautStruct}

L'appel de l'initialisateur synthétisé comprend une valeur par propriété non initialisée déclarée par le type structure.

Par exemple, pour la déclaration~:
\begin{galgas34}
struct @maStructure {
  public var @uint propriété1
  public var @bool propriété2
}
\end{galgas34}

La syntaxe la plus générale d'appel de l'initialisateur synthétisé est~:
\begin{galgas34}
var aVariable = @maStructure.init (123, true)
\end{galgas34}
\begin{galgas3}
var aVariable = @maStructure.init {!123 !true}
\end{galgas3}

On peut omettre \ggsq!.init!~:
\begin{galgas34}
var aVariable = @maStructure (123, true)
\end{galgas34}
\begin{galgas3}
var aVariable = @maStructure {!123 !true}
\end{galgas3}


Si le contexte le permet, l'annotation de type peut être omis lors de l'appel de l'initialisateur~:
\begin{galgas34}
var @maStructure aVariable = .init (123, true)
\end{galgas34}
\begin{galgas3}
var @maStructure aVariable = .init {!123 !true}
\end{galgas3}


Il est possible d'ajouter l'attribut \ggsq=%initArgLabel= à la déclaration d'une propriété non initialisée, pour ajouter dans l'appel de l'initialisateur synthétisé l'étiquette d'argument pour cette propriété. Par exemple, si on déclare~:
\begin{galgas3}
struct @maStructure {
  public var @uint propriété1 %initArgLabel
  public var @bool propriété2
}
\end{galgas3}

Alors l'appel de l'initialisateur synthétisé devient~:
\begin{galgas4}
var aVariable = @maStructure.init (!propriété1: 123, true)
\end{galgas4}
\begin{galgas3}
var aVariable = @maStructure.init {!propriété1: 123 !true}
\end{galgas3}

On peut omettre \ggsq!.init! ou l'annotation de type si le contexte le permet.

Si la propriété est initialisée, alors l'attribut \ggsq=%initArgLabel= est invalide (déclenche une erreur de syntaxe).


Si toutes les propriétés sont initialisées, alors l'initialisateur synthétisé n'a pas d'argument. Pour la structure~:
\begin{galgas34}
struct @maStructure {
  public var @uint propriété1 = 1
  public var @bool propriété2 = 2
}
\end{galgas34}

On écrit donc~:
\begin{galgas34}
var aVariable = @maStructure.init ()
// Ou :
var aVariable = @maStructure.init
// Ou :
var aVariable = @maStructure ()
// Ou :
var @maStructure aVariable = .init ()
// Ou :
var @maStructure aVariable = .init
\end{galgas34}
\begin{galgas3}
var aVariable = @maStructure.init {}
// Ou :
var aVariable = @maStructure.init
// Ou :
var aVariable = @maStructure {}
// Ou :
var @maStructure aVariable = .init {}
// Ou :
var @maStructure aVariable = .init
\end{galgas3}







\subsectionLabel{Initialisateur}{initialisateurStruct}

Un initialisateur peut être déclaré à l'intérieur de la déclaration de structure, ou comme une extension. Exemple d'initialisateur déclaré dans la déclaration de structure~:

\begin{galgas34}
struct @maStructure {
  public var @uint propriété1
  public var @bool propriété2
  
  init (let @uint inV1, let @uint inV2) {
    self.propriété1 = inV1
    self.propriété2 = inV2
  }
}
\end{galgas34}

Et comme extension :

\begin{galgas34}
extension @maStructure.init (let @uint inV) {
  self.propriété1 = inV
  self.propriété2 = inV
}
\end{galgas34}

On peut déclarer autant d'initialisateurs que l'on souhaite, du moment qu'ils se distinguent par leur nombre d'arguments et leurs étiquettes d'arguments. Par exemple, si on veut déclarer un autre initialisateur avec un seul argument, il faut ajouter, pour le distinguer du précédent, une étiquette à l'argument~:
\begin{galgas34}
extension @maStructure.init (?plusUn: let @uint inV) {
  self.propriété1 = inV + 1
  self.propriété2 = inV + 1
}
\end{galgas34}

Remarquer que \ggsq!self.! est obligatoire pour accéder aux propriétés.

\subsection{Liste d'instructions d'un initialisateur}

La liste d'instructions d'un initialisateur est divisée en deux parties séquentielles séparées par une limite invisible~:
\begin{itemize}
  \item d'abord l'initialisation des propriétés ;
  \item ensuite, on peut accéder aux fonctions et procédures de la structure sans restriction.
\end{itemize}

Par exemple~:
\begin{galgas34}
struct @maStructure {
  public var @uint propriété1
  public var @bool propriété2
  
  init (let @uint inV1, let @uint inV2) {
    self.propriété1 = inV1
    self.propriété2 = inV2
  //--- À partir d'ici, l'objet courant est complètement initialisé
    self.propriété1 += self.somme ()
  }
  
  func somme () -> @uint {
    result = self.propriété1 + self.propriété2
  }
}
\end{galgas34}


%\subsectionLabel{GALGAS 3 : constructeur \texttt{new}}{constructeurNewStruct}
%
%{\bf Cette construction est obsolète et est remplacée par l'appel d'un initialisateur.}
%
%En GALGAS 3, tout type structure définit implicitement le constructeur \ggsq!new!. Son appel comprend une valeur par propriété non initialisée déclarée par le type structure.
%
%Par exemple, pour la déclaration~:
%\begin{galgas3}
%struct @maStructure {
%  public var @uint propriété1
%  @bool propriété2 // Syntaxe obsolète, autorisée en GALGAS 3
%}
%\end{galgas3}
%
%L'appel du constructeur \ggsq!new! est~:
%\begin{galgas3}
%var aVariable = @maStructure.new {!123 !true}
%\end{galgas3}
%
%Si le contexte le permet, l'annotation de type peut être omis lors de l'appel du constructeur~:
%\begin{galgas3}
%var @maStructure aVariable = .new {!123 !true}
%\end{galgas3}
%
%Il est possible d'ajouter l'attribut \ggsq=%initArgLabel= à la déclaration d'une propriété de structure. Le faire impose d'utiliser le sélecteur portant le nom de la propriété dans l'appel du constructeur \ggsq=new=. Par exemple, si on déclare~:
%\begin{galgas3}
%struct @maStructure {
%  public var @uint propriété1 %initArgLabel
%  @bool propriété2 // Syntaxe obsolète, autorisée en GALGAS 3
%}
%\end{galgas3}
%
%Alors l'appel du constructeur \ggsq!new! devient~:
%\begin{galgas3}
%var aVariable = @maStructure.new {!propriété1: 123 !true}
%\end{galgas3}


%\subsection{Constructeur \texttt{default}}
%
%Si chacune des propriétés accepte le constructeur par défaut, alors le type structure accepte le constructeur pas défaut. C'est le cas de la structure \ggsq!@maStructure! définie au dessus~: \ggsq!@uint! accepte le constructeur par défaut (initialisation à \ggsq!0!), ainsi que \ggsq!@bool! (initialisation à \ggsq!false!). Donc~:
%\begin{galgas3}
%var aVariable = @maStructure.default
%\end{galgas3}
%Initialise les propriétés de \ggsq!aVariable! respectivement à \ggsq!0! et \ggsq!false!. On peut aussi écrire~:
%\begin{galgas3}
%@maStructure aVariable = .default
%\end{galgas3}


\section{Accès aux propriétés}

À l'intérieur d'une fonction ou procédure de la structure, l'accès par \ggsq!self.! est obligatoire. La notation pointée \ggsq!variable.propriété! permet d'accéder à une propriété d'une structure, sous réserve de satisfaire le contrôle d'accès (\refTableau{controleAccessPropStruct})

\begin{table}[ht]
  \centering
  \begin{tabular}{rcccc}
  \bf Déclaration de la  & \bf Lecture       & \bf Écriture       & \bf Lecture           & \bf Écriture \\
  \bf propriété \ggsq!v! & \bf \ggsq!self.p! & \bf \ggsq!self.p!  & \bf \ggsq!variable.p! & \bf \ggsq!variable.p! \\
  \ggsq!public var p!    & \checkmark        & \checkmark         & \checkmark            & \checkmark \\
  \ggsq!public let p!    & \checkmark        &                    & \checkmark            & \\
  \ggsq!private var p!   & \checkmark        & \checkmark         &                       & \\
  \ggsq!private let p!   & \checkmark        &                    &                       & \\
  \ggsq!private(set) var p! & \checkmark     & \checkmark         & \checkmark             & 
  \end{tabular}
  \caption{Contrôle d'accès des propriétés d'une structure}
  \labelTableau{controleAccessPropStruct}
\end{table}


%aussi bien en lecture, en écriture et en lecture/écriture.
%
%Exemple d'accès en lecture~:
%\begin{galgas34}
%var v = aVariable.propriété1
%\end{galgas34}
%
%Exemple d'accès en écriture~:
%\begin{galgas34}
%aVariable.propriété1 = 10
%\end{galgas34}
%
%
%Exemple d'accès en lecture/écriture~:
%\begin{galgas34}
%aVariable.propriété1 += 1
%\end{galgas34}





\sectionLabel{Getters}{getterStruct}

Un type structure définit un \emph{getter} sans argument par propriété, qui permet d'accéder en lecture à cette propriété. Son nom est celui de la propriété. Par exemple, à la place de~:
\begin{galgas3}
@uint v = aVariable.mProp1
\end{galgas3}

On peut écrire~:
\begin{galgas3}
@uint v = [aVariable mProp1]
\end{galgas3}



\sectionLabel{Méthodes}{methodStruct}



\sectionLabel{Setters}{setterStruct}


\section{Types structure prédéfinis}

Plusieurs types préféfinis GALGAS sont des structures.

\subsectionTypePredefiniLabelIndex{lbigint}

\begin{galgas34}
struct @lbigint {
  public var @bigint bigint
  public var @location location
}
\end{galgas34}



\subsectionTypePredefiniLabelIndex{lbool}

\begin{galgas34}
struct @lbool {
  public var @bool bool
  public var @location location
}
\end{galgas34}



\subsectionTypePredefiniLabelIndex{lchar}

\begin{galgas34}
struct @lchar {
  public var @char char
  public var @location location
}
\end{galgas34}


\subsectionTypePredefiniLabelIndex{ldouble}

\begin{galgas34}
struct @ldouble {
  public var @double double
  public var @location location
}
\end{galgas34}







\subsectionTypePredefiniLabelIndex{lsint}

\begin{galgas34}
struct @lsint {
  public var @sint sint
  public var @location location
}
\end{galgas34}








\subsectionTypePredefiniLabelIndex{lsint64}

\begin{galgas34}
struct @lsint64 {
  public var @sint64 sint64
  public var @location location
}
\end{galgas34}







\subsectionTypePredefiniLabelIndex{lstring}

\begin{galgas34}
struct @lstring {
  public var @string string
  public var @location location
}
\end{galgas34}







\subsectionTypePredefiniLabelIndex{luint}

\begin{galgas34}
struct @luint {
  public var @uint uint
  public var @location location
}
\end{galgas34}





\subsectionTypePredefiniLabelIndex{luint64}


\begin{galgas34}
struct @luint64 {
  public var @uint64 uint64
  public var @location location
}
\end{galgas34}


\subsectionTypePredefiniLabelIndex{range}

Le type \ggsq!@range! définit les intervalles d'entiers non signés 32 bits (\ggsq!@uint!).

\begin{galgas34}
struct @range {
  public var @uint start
  public var @uint length
}
\end{galgas34}

La plupart des propriétés du type \ggsq!@range! découle de cette définition (\refChapterPage{typeStructure}).

\ggsq+@range (a, b)+, où \ggsq!a! et \ggsq!b! sont des expressions de type \ggsq!@uint!, représente~:
\begin{itemize}
  \item un intervalle vide si \ggsq!b! est égal à zéro ;
  \item l'intervalle $[a, a+b-1]$ si \ggsq!b! est strictement positif.
\end{itemize}



\subsubsection{Opérateurs \texttt{.{}.{}.} et \texttt{.{}.{}<}}

Deux opérateurs permettent de construire plus facilement des objets de type \ggsq!@range!.

L'opérateur \ggsq!...! permet de définir un intervalle fermé à partir de sa borne inférieure et de sa borne supérieure~: si \ggsq!a! et \ggsq!b! sont des expressions de type \ggsq!@uint!, l'expression \ggsq!a ... b! est équivalente à la construction \ggsq*@range (a, b - a + 1)*. Une exception est levée si $b < a$.

L'opérateur \ggsq!..<! permet de définir un intervalle ouvert à gauche à partir de sa borne inférieure et de sa borne supérieure~: si \ggsq!a! et \ggsq!b! sont des expressions de type \ggsq!@uint!, l'expression \ggsq!a ..< b! est équivalente à \ggsq*@range (a, b - a)*. Une exception est levée si $b < a$.

\subsubsection{Type \texttt{@range} et instruction \texttt{for}}

On peut utiliser une expression de type \ggsq!@range! avec l'instruction \ggsq!for!~:

\begin{galgas34}
for i in @range (12, 5) do
  # i prend successivement les valeurs 12, 13, 14, 15, 16
end
\end{galgas34}

Et, avec l'opérateur \ggsq!...!~:
\begin{galgas34}
for i in 12 ... 16 do
  # i prend successivement les valeurs 12, 13, 14, 15, 16
end
\end{galgas34}

Et l'opérateur \ggsq!..<!~:
\begin{galgas34}
for i in 12 ..< 17 do
  # i prend successivement les valeurs 12, 13, 14, 15, 16
end
\end{galgas34}

Si l'on veut parcourir l'énumération à partir de la dernière valeur, on utilise le modificateur \ggsq!>! après le mot-clé \ggsq!for!~:
\begin{galgas34}
for > i in @range (12, 5) do
  # i prend successivement les valeurs 16, 15, 14, 13, 12
end
\end{galgas34}

\begin{galgas34}
for > i in 12 ... 16 do
  # i prend successivement les valeurs 16, 15, 14, 13, 12
end
\end{galgas34}

\begin{galgas34}
for > i in 12 ..< 17 do
  # i prend successivement les valeurs 16, 15, 14, 13, 12
end
\end{galgas34}


