%!TEX encoding = UTF-8 Unicode
%!TEX root = ../galgas-book.tex

%--------------------------------------------------------------
\chapterLabel{Le type structure}{typeStructure}
%-------------------------------------------------------------


Le mot-clé \ggs!struct! permet de définir des types de structure. La syntaxe de définition d'un type structure est de la forme :

\begin{galgas}
struct @nom_de_type
  # Liste de déclaration de champs, par exemple :
  @uint mChamp1
  @bool mChamp2
\end{galgas}


\section{Types structure prédéfinis}

Plusieurs types préféfinis GALGAS sont des structures.
 
\subsectionTypePredefiniLabelIndex{lbool}

\begin{galgas}
struct @lbool {
  @bool bool
  @location location
}
\end{galgas}




\subsectionTypePredefiniLabelIndex{lchar}

\begin{galgas}
struct @lchar {
  @char char
  @location location
}
\end{galgas}







\subsectionTypePredefiniLabelIndex{ldouble}

\begin{galgas}
struct @ldouble {
  @double double
  @location location
}
\end{galgas}







\subsectionTypePredefiniLabelIndex{lsint}

\begin{galgas}
struct @lsint {
  @sint sint
  @location location
}
\end{galgas}








\subsectionTypePredefiniLabelIndex{lsint64}

\begin{galgas}
struct @lsint64 {
  @sint64 sint64
  @location location
}
\end{galgas}







\subsectionTypePredefiniLabelIndex{lstring}

\begin{galgas}
struct @lstring {
  @string string
  @location location
}
\end{galgas}







\subsectionTypePredefiniLabelIndex{luint}

\begin{galgas}
struct @luint {
  @uint uint
  @location location
}
\end{galgas}





\subsectionTypePredefiniLabelIndex{luint64}

\begin{galgas}
struct @luint64 {
  @uint64 uint64
  @location location
}
\end{galgas}


\subsection{le type \texttt{@range}}

\begin{galgas}
struct @range {
  @uint start
  @uint length
}
\end{galgas}

Un objet de type \ggs+@range+ peut aussi être instancié au moyen des opérateurs \ggs!...! et \ggs!..<! (voir page \refSectionPage{operateurIntervalleRange}).


\section{Constructeurs}

\subsection{Constructeur \texttt{new}}

Tout type structure définit implicitement le constructeur \ggs!new!. Son appel comprend une valeur par attribut déclaré par le type structure.

Par exemple, pour la déclaration :
\begin{galgas}
struct @maStructure
  @uint mChamp1
  @bool mChamp2
\end{galgas}

L'appel du constructeur \ggs!new! est :
\begin{galgas}
var aVariable = @maStructure.new {!123 !true}
\end{galgas}

Si le contexte le permet, l'annotation de type peut être omis lors de l'appel du constructeur :
\begin{galgas}
@maStructure aVariable = .new {!123 !true}
\end{galgas}


\subsection{Constructeur \texttt{default}}

Si chacun des champs accepte le constructeur par défaut, alors le type structure accepte le constructeur pas défaut. C'est le cas de la structure \ggs!@maStructure! définie au dessus : \ggs!@uint! accepte le constructeur par défaut (initialisation à \ggs!0!), ainsi que \ggs!@bool! (initialisation à \ggs!false!). Donc :
\begin{galgas}
var aVariable = @maStructure.default
\end{galgas}
Initialise les champs de \ggs!aVariable! respectivement à \ggs!0! et \ggs!false!. On peut aussi écrire :
\begin{galgas}
@maStructure aVariable = .default
\end{galgas}


\section{Accès aux champs}

La notation pointée \ggs!variable.champ! permet d'accéder à un champ d'une structure, aussi bien en lecture, en écriture et en lecture/écriture.

Exemple d'accès en lecture :
\begin{galgas}
@uint v = aVariable.mChamp1
\end{galgas}

Exemple d'accès en écriture :
\begin{galgas}
aVariable.mChamp1 = 10
\end{galgas}


Exemple d'accès en lecture/écriture :
\begin{galgas}
aVariable.mChamp1 ++
\end{galgas}





\section{Getters}

Un type structure définit un \emph{getter} sans argument par champ, qui permet d'accéder en lecture à ce champ. Son nom est celui du champ. Par exemple, à la place de :
\begin{galgas}
@uint v = aVariable.mChamp1
\end{galgas}

On peut écrire :
\begin{galgas}
@uint v = [aVariable mChamp1]
\end{galgas}

