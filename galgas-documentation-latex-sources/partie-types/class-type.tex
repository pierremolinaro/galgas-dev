%!TEX encoding = UTF-8 Unicode
%!TEX root = ../galgas-book.tex

%--------------------------------------------------------------
\chapterLabel{Le type \texttt{class}}{typeClass}

Le type \ggs=class= GALGAS est classique : il inclut l'\emph{héritage simple} et les \emph{classes abstraites}.

Il n'est pas possible de définir des méthodes dans une classe : on peut le faire uniquement via des \emph{extensions}~: \refChapterPage{chapitreExtensions}.

Les classes ont par défaut une \emph{sémantique de valeur}, c'est-à-dire qu'une affectation entre instances provoque une copie. À partir de la version 3.3.0 de GALGAS, le qualificatif \ggs=shared= donne à la classe qui le porte une \emph{sémantique de référence}, c'est-à-dire qu'une affectation entre instances provoque un partage de données. Ceci est expliqué en détail à la \refSectionPage{semantiqueValeurReference}.








\sectionLabel{Déclaration d'une classe}{declarationClasse}


Voici différents exemples de déclaration de classes :

\begin{galgas}
abstract class @A {
  @uint mA
}
class @B : @A {
  @string mB
}
class @C : @B {
  @data mC
}
\end{galgas}

La classe \ggs+@A+ est abstraite (c'est-à-dire qu'elle ne peut pas être instanciée), la classe \ggs+@B+ hérite de \ggs+@A+. Une classe déclare zéro, un ou plusieurs propriétés. L'héritage multiple n'est pas implémenté en GALGAS.

Une classe qui hérite d'une autre peut être abstraite :
\begin{galgas}
abstract class @D : @C {
  ...
 }
\end{galgas}

Une classe non abstraite définit implicitement le constructeur \ggs+new+, et des \emph{getters} pour lire les propriétés, et des \emph{setters} pour les écrire. On ne peut pas définir explicitement d'autres constructeurs, \emph{getters} ou \emph{setters} à l'intérieur de la classe. Cependant,  les extensions (\refChapterPage{chapitreExtensions}) permettent de définir \emph{getters}, \emph{méthodes} et \emph{setters} associés à une classe.











\sectionLabel{Sémantique de valeur et sémantique de référence}{semantiqueValeurReference}

Par défaut, une classe GALGAS a une \emph{sémantique de valeur}, c'est-à-dire qu'une affectation entre instances provoque une copie. Prenons un exemple, en considérant la classe~:

\begin{galgas}
class @classeSemantiqueDeValeur {
  @uint propriété %setter
}
\end{galgas}

Et le fragment de code suivant :

\begin{galgas}
@classeSemantiqueDeValeur a = .new {!10}
@classeSemantiqueDeValeur b = a # Copie
[!?a setPropriété !5]
message "Propriété de a " + [a propriété] + "\n" # Propriété de a 5
message "Propriété de b " + [b propriété] + "\n" # Propriété de b 10
\end{galgas}

Lors de l'affectation \ggs+b=a+, \ggs=b= reçoit une copie de la valeur de \ggs=a=, si bien que l'affectation ultérieure de la \ggs=propriété= de \ggs=a= n'affecte pas \ggs=b=.

 À partir de la version 3.3.0 de GALGAS, le qualificatif \ggs=shared= donne à la classe qui le porte une \emph{sémantique de référence}, c'est-à-dire qu'une affectation entre instances provoque un partage de données~:

\begin{galgas}
shared class @classeSemantiqueDeReference {
  @uint propriété %setter
}
\end{galgas}

Et l'exécution devient :

\begin{galgas}
@classeSemantiqueDeReference a = .new {!10}
@classeSemantiqueDeReference b = a # Partage
[!?a setPropriété !5]
message "Propriété de a " + [a propriété] + "\n" # Propriété de a 5
message "Propriété de b " + [b propriété] + "\n" # Propriété de b 5
\end{galgas}

L'affectation \ggs+b=a+ provoque maintenant un partage de données, \ggs=a= et \ggs=b= désigne le même objet~: l'affectation de sa \ggs=propriété= via \ggs=a= est visible via \ggs=b=.

Les règles d'utilisation du qualificatif \ggs=shared= sont données à la \refSectionPage{sharedClassQualifier}.








\section{Le constructeur \texttt{new}}

Le constructeur \ggs+new+ est implicitement pour toute classe non abstraite (c'est à dire les classes \ggs+@B+ et \ggs+@C+ de la \refSectionPage{declarationClasse}). Ce constructeur présente un argument par propriété déclaré dans la classe instanciée et dans toutes ses super classes. L'ordre des arguments est celui obtenu en parcourant la hiérarchie de classes, en commençant par la classe de base. Par exemple on écrira~:

\begin{galgas}
@B b = @B.new {
  !0 # Propriété mA de @A
  !"Hello" # Propriété mB de @B
}
@C c = @C.new {
  !0 # Propriété mA de @A
  !"Hello" # Propriété mB de @B
  !@data.emptyData # Propriété mC de @C
}
\end{galgas}

Dans les exemples ci-dessus, les annotations de type apparaissent deux fois, à la déclaration de la variable et devant le constructeur \ggs=new=. Si ces deux annotations nomment le même type, l'une d'entre elles peut être omise. Par exemple~:

\begin{galgas}
@B b = .new {
  !0 # Propriété mA de @A
  !"Hello" # Propriété mB de @B
}
\end{galgas}

Ou bien :
\begin{galgas}
var b = @B.new {
  !0 # Propriété mA de @A
  !"Hello" # Propriété mB de @B
}
\end{galgas}

Aucune des deux annotations ne peut être omise si elles nomment des types différents, comme lorsque l'on réalise une affectation polymorphique~:

\begin{galgas}
@A b = @B.new {
  !0 # Propriété mA de @A
  !"Hello" # Propriété mB de @B
}
\end{galgas}







\sectionLabel{Qualificatif \texttt{shared} appliqué à une classe}{sharedClassQualifier}

Le qualificatif \ggs=shared= donne une \emph{sémantique de référence} à la classe à laquelle il s'applique (voir \refSectionPage{semantiqueValeurReference}).

Une classe de base, c'est-à-dire une classe sans super classe, accepte librement le qualificatif \ggs=shared=.

Pour une classe ayant une super classe :
\begin{itemize}
  \item si la super classe est déclarée avec le qualificatif \ggs=shared=, alors la classe doit avoir le qualificatif \ggs=shared=~;
  \item si la super classe n'est pas déclarée avec le qualificatif \ggs=shared=, alors la classe ne doit pas avoir le qualificatif \ggs=shared=.
\end{itemize}

Les notions de classe abstraite et de sémantique de valeur / de référence sont indépendantes : une classe abstraite peut avoir une sémantique de référence. Détail syntaxique, le mot réservé \ggs=abstract= doit précéder le mot réservé \ggs=shared=~:

abstract shared class @A {
  ...
}








\section{Lecture d'une propriété}

Par défaut, la lecture d'une propriété est activée par la définition implicite d'un \emph{getter}, dont le nom est le nom de la propriété. Ainsi, pour une classe \ggs=@C=~:

\begin{galgas}
class @C {
  @uint prop
}
\end{galgas}

Et pour une variable \ggs+c+ de type \ggs+@C+, on peut écrire~:

\begin{galgas}
@uint v = [c prop]
\end{galgas}

À partir de la version 3.3.0, il est possible d'utiliser la notation pointée~:
\begin{galgas}
@uint v = c.prop
\end{galgas}

Il est possible d'inhiber la génération implicite d'un \emph{getter} de lecture d'une propriété en complétant sa déclaration par \ggs+%nogetter+, comme par exemple~:

\begin{galgas}
class @C {
  @uint prop %nogetter
}
\end{galgas}

Les écritures \ggs+c.prop+ et \ggs+[c prop]+ sont alors rejetées par le compilateur.









\section{Écriture d'une propriété}

Par défaut, l'écriture d'une propriété n'est pas activée : une façon de modifier la valeur d'une propriété est d'écrire une extension (\refChapterPage{chapitreExtensions}) ; l'autre possibilité, décrite ci-dessous, est d'activer le \emph{setter} d'écriture}.

Pour activer la génération d'un \emph{setter} permettant d'écrire une propriété, compléter la déclaration de cette propriété par la propriété \ggs+%setter+. Un \emph{setter} est alors engendré, et porte le nom \texttt{set<Propriété>}, c'est-à-dire le nom de la propriété avec sa première lettre en majuscule, précédé par \texttt{set}. Par exemple :

\begin{galgas}
class @C {
  @uint prop %setter
}
\end{galgas}


Pour modifier la propriété \ggs+prop+ d'un objet \ggs=c= instance de \ggs=@C=, on écrira :

\begin{galgas}
[!?c setProp !12]
\end{galgas}

Si on veut à la fois inhiber la génération implicite d'un \emph{getter} de lecture d'une propriété et engendrer le \emph{setter} d'écriture, il suffit de déclarer la propriété par :

\begin{galgas}
class @C {
  @uint prop %setter %nogetter
}
\end{galgas}

Ou encore, comme l'ordre des attributs est sans importance~:

\begin{galgas}
class @C {
  @uint prop %nogetter %setter
}
\end{galgas}












\section{Conversions entre objets de classes différentes}

Pour toute cette section, nous illustrons les constructions décrites en nous basant sur les trois classes suivantes~:
\begin{galgas}
class @A {
  ...
}
class @B : @A {
  ...
}
class @C : @B {
  ...
}
\end{galgas}

Nous considérons trois variables \ggs=@a=, \ggs=b= et \ggs=c= respectivement de types \ggs=@A=, \ggs=@B= et \ggs=@C=.


\subsection{Affectation polymorphique}

GALGAS accepte l'affectation polymorphique qui est par exemple \ggs+a = b+. Elle est autorisée aussi lors de l'affectation d'une expression effective à un paramètre formel dans une instruction d'appel (de routine, de fonction, de méthode, ...)


\subsection{Affectation polymorphique inverse}

L'affectation polymorphique inverse (qui consisterait à écrire \ggs+b = a+) est logiquement refusée par le compilateur.

Il y a trois constructions qui permettent d'effectuer cette opération :
\begin{itemize}
  \item l'expression de conversion polymorphique inverse (\refSubsectionPage{expConversionPolymorphiqueInverse}) ;
  \item l'expression de test du type dynamique (\refSubsectionPage{testTypeDynamiqueExpression}) ;
  \item l'instruction \ggs+cast+ (\refSectionPage{instructionCast}).
\end{itemize}

%Pour effectuer ponctuellement une affectation polymorphique inverse, on écrit (les parenthèses sont obligatoires) :
%
%\begin{galgas}
%@T resultat = (cast expression : @T) ;
%\end{galgas}
%
%Si le type dynamique de l'\ggs+expression+ est \ggs+@T+ ou une de ses classes héritières, l'expression de conversion polymorphique renvoie un objet de type \ggs+@T+ contenant la valeur de \ggs+expression+. Dans le cas contraire, un message d'erreur est affiché, et la variable \ggs+resultat+ est non construite.
%
%L'exécution échoue donc avec émission de message d'erreur si la conversion n'est pas possible. 
%
%
%Grâce à l'\emph{expression de test du type dynamique}, il est possible de tester si une conversion est possible. On peut donc écrire :
%
%\begin{galgas}
%if (expression is @B) then
%  const @B variable = (cast expression : @B) ;
%  ...
%elsif (expression is @C) then
%  const @C variable = (cast expression : @C) ;
%  ...
%else
%  message "conversion impossible" ;
%end if ;
%\end{galgas}
%
%L'instruction \ggs+cast+ permet simplement d'exprimer de manière plus élégante une série de test de conversion. La forme équivalent à l'instruction \ggs+if+ précédente est :
%
%\begin{galgas}
%cast expression
%when >= @B variable :
%  ...
%when >= @C variable :
%  ...
%else
%  message "conversion impossible" ;
%end cast ;
%\end{galgas}


