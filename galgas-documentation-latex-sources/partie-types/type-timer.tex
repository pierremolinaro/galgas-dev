%!TEX encoding = UTF-8 Unicode
%!TEX root = ../galgas-book.tex

\chapitreTypePredefiniLabelIndex{timer}

%Le type \ggs!@range! définit les intervalles d'entiers non signés 32 bits (\ggs!@uint!).
%
%La déclaration du type \ggs+@range+ est équivalent à la déclaration du type structure :
%\begin{galgas}
%struct @range {
%  @uint start
%  @uint length
%}
%\end{galgas}
%
%La plupart des propriétés du type \ggs!@range! découle de cette définition (\refChapterPage{typeStructure}).
%
%\ggs+@range.new {!a !b}+, où \ggs!a! et \ggs!b! sont des expressions de type \ggs!@uint!, représente :
%\begin{itemize}
%  \item un intervalle vide si \ggs!b! est égal à zéro ;
%  \item l'intervalle $[a, a+b-1]$ si \ggs!b! est strictement positif.
%\end{itemize}
%
%
%
%\sectionLabel{Opérateurs \texttt{...} et \texttt{..<}}{operateurIntervalleRange}
%
%Deux opérateurs permettent de construire plus facilement des objets de type \ggs!@range!.
%
%L'opérateur \ggs!...! permet de définir un intervalle fermé à partir de sa borne inférieure et de sa borne supérieure : si \ggs!a! et \ggs!b! sont des expressions de type \ggs!@uint!, l'expression \ggs!a ... b! est équivalente à la construction \ggs*@range.new {!a !b - a + 1}*. Une exception est levée si $b < a$. 
%
%L'opérateur \ggs!..<! permet de définir un intervalle ouvert à gauche à partir de sa borne inférieure et de sa borne supérieure : si \ggs!a! et \ggs!b! sont des expressions de type \ggs!@uint!, l'expression \ggs!a ..< b! est équivalente à \ggs*@range.new {!a !b - a}*. Une exception est levée si $b < a$.
%
%\section{Type \texttt{@range} et instruction \texttt{for}}
%
%On peut utiliser une expression de type \ggs!@range! avec l'instruction \ggs!for! :
%
%\begin{galgas}
%for i in @range.new {!12 !5} do
%  # i prend successivement les valeurs 12, 13, 14, 15, 16
%end
%\end{galgas}
%
%Et, avec l'opérateur \ggs!...! :
%\begin{galgas}
%for i in 12 ... 16 do
%  # i prend successivement les valeurs 12, 13, 14, 15, 16
%end
%\end{galgas}
%
%Et l'opérateur \ggs!..<! :
%\begin{galgas}
%for i in 12 ..< 17 do
%  # i prend successivement les valeurs 12, 13, 14, 15, 16
%end
%\end{galgas}
%
%Si l'on veut parcourir l'énumération à partir de la dernière valeur, on utilise le modificateur \ggs!>! après le mot-clé \ggs!for! :
%\begin{galgas}
%for > i in @range.new {!12 !5} do
%  # i prend successivement les valeurs 16, 15, 14, 13, 12
%end
%\end{galgas}
% 
%\begin{galgas}
%for > i in 12 ... 16 do
%  # i prend successivement les valeurs 16, 15, 14, 13, 12
%end
%\end{galgas}
% 
%\begin{galgas}
%for > i in 12 ..< 17 do
%  # i prend successivement les valeurs 16, 15, 14, 13, 12
%end
%\end{galgas}
 
