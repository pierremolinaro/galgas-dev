%!TEX encoding = UTF-8 Unicode
%!TEX root = ../galgas-book.tex

%--------------------------------------------------------------
\chapterLabel{Le type \texttt{dict}}{typeDict}
%-------------------------------------------------------------

\tableDesMatieresDuChapitre


Un objet de type \ggs+dict+ est un dictionnaire. Contrairement à un objet de type \ggs+map+ , le type de la clé est déclaré explicitement et peut être d'un type quelconque.

{\bf\color{red} Attention ! } Le type de la clé d'un dictionnaire peut être un \ggs+@lstring+, comme la clé implicite d'une \ggs+map+. Toutefois, la sémantique est différente. Dans une \ggs=map=, c'est la composante \ggs=string= de la clé qui est prise en compte comme clé effective. La composante \ggs=location= n'est utilisée que pour le signalement d'erreur. Dans un \ggs=dict=, le type déclaré comme clé est intégralement pris en compte : si la clé d'un \ggs=dict= est un \ggs=@lstring=, la comparaison des clés prend en compte la composante \ggs=string= {\bf et} la composante \ggs=location=.

\section{Déclaration}

\begin{galgas}
dict @MonDictionnaire : @uint {
  @string mPremier
  @bool mSecond
}
\end{galgas}

La déclaration d'un type \ggs+dict+ nomme~:
\begin{itemize}
  \item le nom du type \ggs+dict+ (ici \ggs+@MonDictionnaire+)~;
  \item le type de la clé (ici \ggs+@uint+)~;
  \item les propriétés associées (ici \ggs=mPremier= et \ggs=mSecond=).
\end{itemize}

Les noms \ggs=key=, \ggs=object= et \ggs=desciption= sont interdits pour une propriété.

On peut déclarer un dictionaire sans propriété, et ainsi gérer des \emph{ensembles} :

\begin{galgas}
dict @autreDictionnaire : @uint { }
\end{galgas}





\section{Constructeurs}

Pour initialiser un dictionnaire vide, il y a trois possibilités, sémantiquement identiques~:
\begin{itemize}
  \item la constante \texttt{\{\}} (\refSubsectionPage{constanteDictVide})~; 
  \item le constructeur \texttt{emptyDict} (\refSubsectionPage{constructeurEmptyDict})~; 
  \item le constructeur \texttt{default} (\refSubsectionPage{constructeurDictDefault}). 
\end{itemize}


\subsectionLabel{Constante \texttt{\{\}}}{constanteDictVide}

Cette constante permet d'instancier un dictionnaire vide. Exemple~:
\begin{galgas}
@MonDictionnaire unDictionnaire = {}
\end{galgas}

Ou encore~:

\begin{galgas}
var unDictionnaire = @MonDictionnaire {}
\end{galgas}

\subsectionLabel{Constructeur \texttt{emptyDict}}{constructeurEmptyDict}

Pour instancier un dictionnaire vide, une autre possibilité est d'appeler le constructeur \ggs=emptyDict=. Exemple~:
\begin{galgas}
@MonDictionnaire unDictionnaire = .emptyDict
\end{galgas}

Ou encore~:

\begin{galgas}
var unDictionnaire = @MonDictionnaire.emptyDict
\end{galgas}

 

\subsectionLabel{Constructeur \texttt{default}}{constructeurDictDefault}


Un dictionnaire accepte le constructeur \ggs=default=. Exemple~:
\begin{galgas}
@MonDictionnaire unDictionnaire = .default
\end{galgas}

Ou encore~:

\begin{galgas}
var unDictionnaire = @MonDictionnaire.default
\end{galgas}

 




\section{Insertion}

Un \ggs+dict+ implémente implicitement l'opérateur \ggs:+=: qui permet d'insérer une nouvelle entrée à un dictionnaire. Ses arguments sont, dans l'ordre~: un objet du type de la clé, puis un objet par propriété déclarée. Aucune erreur n'est déclenchée en cas de tentative d'insertion d'une clé déjà existante~; les valeurs des propriétés associées à la clé sont simplement remplacées.

Par exemple~:
\begin{galgas}
@MonDictionnaire unDictionnaire = {}
@uint clef = ...
@string s = ...
@uint v = ...
unDictionnaire += !clef !s !v
\end{galgas}











\section{Recherche}

Une \ggs+dict+ déclare implicitement la \emph{méthode} de recherche \ggs=searchKey= qui permet de rechercher une entrée d'un dictionnaire, et retourne la valeur de ses propriétés associées. Une erreur est déclenchée si la clé n'existe pas.

La \emph{méthode} de recherche \ggs=searchKey= est appelée dans une \emph{instruction d'appel de méthode}~:
\begin{itemize}
  \item le premier argument (sortie) est une expression de type de la clé qui caractérise la clé à rechercher~;
  \item ensuite, pour chaque propriété déclarée, un argument en entrée nommant une variable destinée à recevoir la valeur de la propriété correspondante.
\end{itemize}


Par exemple~:
\begin{galgas}
@MonDictionnaire unDictionnaire = {}
...
@lstring clef = ...
[unDictionnaire searchKey !clef ?@string s ?@uint v]
\end{galgas}








\section{Retrait}

Un \ggs+dict+ déclare implicitement le \emph{setter} de retrait \ggs=removeKey=. Il permet de retirer une entrée d'un dictionnaire, et retourne la valeur des propriétés associées à la clé retirée. Une erreur est déclenchée si la clé n'existe pas.



Le \emph{setter} de retrait \ggs=removeKey= est appelé dans une \emph{instruction d'appel de setter}~:
\begin{itemize}
  \item le premier argument (sortie) est une expression de type de la clé qui caractérise la clé à retirer~;
  \item ensuite, pour chaque propriété déclarée, un argument en entrée nommant une variable destinée à recevoir la valeur de la propriété correspondant de la clé retirée.
\end{itemize}

Par exemple~:
\begin{galgas}
@MonDictionnaire unDictionnaire = {}
...
@lstring clef = ...
[!?unDictionnaire removeKey !clef ?@string s ?@uint v]
\end{galgas}

\section{Getters}

%\subsection{Getter \texttt{allKeyList}}
%
%\begin{galgas}
%getter @T allKeyList -> @lstringlist
%\end{galgas}
%
%Le \emph{getter} \ggs+allKeyList+ retourne la liste construite avec toutes les clés du récepteur, dans la table de premier niveau et dans les tables surchargées. L'ordre de la liste est~:
%\begin{itemize}
%  \item d'abord les clés du dictionnaire de premier niveau, puis celles des tables surchargées, dans l'ordre de la surcharge~;
%  \itel pour chaque table, les clés apparaissent dans l'ordre alphabétique croissant.
%\end{itemize}

\subsection{Getter \texttt{count}}

\begin{galgas}
getter count -> @uint
\end{galgas}


Le \emph{getter} \ggs+count+ retourne un \ggs+@uint+ qui contient le nombre d'entrées du dictionnaire.



\subsection{Getter \texttt{hasKey}}


Le \emph{getter} \ggs+hasKey+ retourne un \ggs+@bool+ qui est \ggs+true+ si la clé \ggs+inKey+ est dans le dictionnaire, \ggs+false+ dans le cas contraire.



%\subsection{Getter \texttt{keyList}}
%
%\begin{galgas}
%getter keyList -> @lstringlist
%\end{galgas}
%
%
%Le \emph{getter} \ggs+keyList+ retourne la liste construite avec toutes les clés du dictionnaire de premier niveau du récepteur. L'ordre de la liste est l'ordre alphabétique croissant des clés.



%\subsection{Getter \texttt{keySet}}
%
%\begin{galgas}
%getter keySet -> @stringset
%\end{galgas}
%
%
%Le \emph{getter} \ggs+keySet+ retourne l'ensemble de toutes les clés du dictionnaire de premier niveau du récepteur.

















\section{Énumération}

L'instruction \ggs+for+ permet d'énumérer des objets de type \ggs+dict+ ; elle est décrite à la \refSectionPage{instructionFor}.

