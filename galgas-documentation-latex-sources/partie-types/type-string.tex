%!TEX encoding = UTF-8 Unicode
%!TEX root = ../galgas-book.tex

\chapitreTypePredefiniLabelIndex{string}

Le type \ggs!@string! définit les chaînes de caractères Unicode.

\section{Chaînes de caractères littérales}

En GALGAS, les chaînes de caractères littérales sont délimitées par des caractères « " », par exemple : \ggs+"a string"+. Une chaîne de caractères littérale est un objet constant de type \ggs!@string!, si bien que l'on peut lui appliquer méthodes et \emph{getters} : \ggs+["ae" uppercaseString]+ retourne la chaîne \ggs+"AE"+.







\section{Constructeurs}

\subsectionConstructor{componentsJoinedByString}{string}

\begin{galgasbox}
constructor componentsJoinedByString
   ?@stringlist inComponents
   ?@string inSeparator -> @string
\end{galgasbox}

Retourne la chaîne de caractéres obtenue en concaténant tous les éléments de \ggs!inComponents! en insérant une copie de \ggs!inSeparator! entre deux éléments consécutifs.

\begin{galgas}
let aList = @stringlist {!"A", !"B", !"C"}
let s = @string.componentsJoinedByString {!aList !"-"} # "A-B-C"
\end{galgas}




\subsectionConstructor{CppChar}{string}

\begin{galgasbox}
constructor CppChar ?@char inChar -> @string
\end{galgasbox}

Retourne la chaîne de caractéres constitué du caractère \ggs!inChar! précédé et suivi par un caractère « " ».

\begin{galgas}
let s = @string.CppChar {!'A'} # "A"
\end{galgas}



\subsectionConstructor{CppLineComment}{string}

\begin{galgasbox}
constructor CppLineComment -> @string
\end{galgasbox}

Retourne une chaîne de caractères constitué de :
\begin{itemize}
  \item deux caractères « \texttt{/} » ;
  \item suivi de 117 caractères « \texttt{-} » ;
  \item suivi d'un caractère « \texttt{*} » ;
  \item et terminée par un retour à la ligne.
\end{itemize}




\subsectionConstructor{CppTitleComment}{string}

\begin{galgasbox}
constructor CppTitleComment ?@string inString -> @string
\end{galgasbox}

Retourne une chaîne de caractères constitué de cinq lignes de commentaires C++ :
\begin{itemize}
\item une ligne obtenue par appel du \refConstructorPage{string}{CppLineComment} ;
\item une ligne obtenue par appel du \refConstructorPage{string}{CppSpaceComment} ;
\item une ligne de commentaire contenant \ggs!inString! centré ;
\item une ligne obtenue par appel du \refConstructorPage{string}{CppSpaceComment} ;
\item une ligne obtenue par appel du \refConstructorPage{string}{CppLineComment}.
\end{itemize}



\subsectionConstructor{CppSpaceComment}{string}

\begin{galgasbox}
constructor CppSpaceComment -> @string
\end{galgasbox}

Retourne une chaîne de caractères constitué de :
\begin{itemize}
  \item deux caractères « \texttt{/} » ;
  \item suivi de 117 caractères \emph{espace} ;
  \item suivi d'un caractère « \texttt{*} » ;
  \item et terminée par un retour à la ligne.
\end{itemize}

Ce constructeur permet d'écrire des commentaires encadrés dans le code C++ engendré.


\subsectionConstructor{default}{string}

\begin{galgasbox}
constructor default -> @string
\end{galgasbox}

Retourne la chaîne vide (voir \refSubsectionPage{constructeurParDefaut}).




\subsectionConstructor{homeDirectory}{string}

\begin{galgasbox}
constructor homeDirectory -> @string
\end{galgasbox}

Retourne une chaîne de caractères contenant le chemin absolu vers le répertoire \emph{home} de l'utilisateur. Fonctionne sous Unix et Windows.






\subsectionConstructor{retrieveAndResetTemplateString}{string}

\begin{galgasbox}
constructor retrieveAndResetTemplateString -> @string
\end{galgasbox}

Ce constructeur est utilisé pour la génération de \emph{templates}.




\subsectionConstructor{stringWithContentsOfFile}{string}

\begin{galgasbox}
constructor stringWithContentsOfFile ?@string inFilePath -> @string
\end{galgasbox}

Ce constructeur lit le fichier texte désigné par le chemin relatif ou absolu \ggs!inFilePath! et retourne sont contenu. Une erreur d'exécution est déclenché si le fichier ne peut pas être lu.






\subsectionConstructor{stringWithCurrentDateTime}{string}

\begin{galgasbox}
constructor stringWithCurrentDateTime -> @string
\end{galgasbox}

Ce constructeur retourne une chaîne de caractères contenant la date et l'heure courante.

Par exemple :
\begin{galgas}
let s = @string.stringWithCurrentDateTime # "Wed Jan  6 20:08:33 2016"
\end{galgas}







\subsectionConstructor{stringWithCurrentDirectory}{string}

\begin{galgasbox}
constructor stringWithCurrentDirectory -> @string
\end{galgasbox}

Ce constructeur retourne une chaîne de caractères contenant le chemin absolu du répertoire courant.




\subsectionConstructor{stringWithEnvironmentVariable}{string}

\begin{galgasbox}
constructor stringWithEnvironmentVariable
  ?@string inVariableName
  -> @string
\end{galgasbox}

Ce constructeur retourne la valeur associée à la variable d'environnement \ggs!inVariableName!. Une erreur d'exécution est déclenchée si la variable d'environnement n'est pas définie. L'existence d'une variable d'environnement peut être testée par le \refGetterPage{string}{doesEnvironmentVariableExist}.





\subsectionConstructor{stringWithEnvironmentVariableOrEmpty}{string}

\begin{galgasbox}
constructor stringWithEnvironmentVariableOrEmpty
  ?@string inVariableName
  -> @string
\end{galgasbox}

Ce constructeur retourne la valeur associée à la variable d'environnement \ggs!inVariableName!. Si la variable d'environnement n'est pas définie, la chaîne vide est retournée et aucune erreur n'est déclenchée.






\subsectionConstructor{stringWithSequenceOfCharacters}{string}

\begin{galgasbox}
constructor stringWithSequenceOfCharacters
  ?@char inChar
  ?@uint inCount
  -> @string
\end{galgasbox}

Ce constructeur retourne la chaîne de caractères constituée d'une séquence de \ggs!inCount! caractères \ggs!inChar!.





\subsectionConstructor{stringWithSourceFilePath}{string}

\begin{galgasbox}
constructor stringWithSourceFilePath -> @string
\end{galgasbox}

Ce constructeur retourne le chemin absolu du fichier source en cours d'analyse.






\subsectionConstructor{stringWithSymbolicLinkContents}{string}

\begin{galgasbox}
constructor stringWithSymbolicLinkContents ?@string inPath -> @string
\end{galgasbox}



















\section{Getters}

\subsectionGetter{absolutePathFromPath}{string}

\begin{galgasbox}
getter absolutePathFromPath ?@string inPath -> @string
\end{galgasbox}

Si la valeur du récepteur est un chemin absolu, cette valeur est retournée et \ggs!inPath! est inutilisé.

Si la valeur du récepteur est un chemin relatif, cette valeur est retournée préfixée par \ggs!inPath!.






\subsectionGetter{capacity}{string}

\begin{galgasbox}
getter capacity -> @uint
\end{galgasbox}

Retourne le nombre de caractères alloués pour stocker la valeur du récepteur. 







\subsectionGetter{characterAtIndex}{string}

\begin{galgasbox}
getter characterAtIndex ?@uint inIndex -> @char
\end{galgasbox}

Retourne le caractère situé à l'indice \ggs!inIndex! de la valeur du récepteur. Le premier caractère a pour indice $0$. Si \ggs!inIndex! est supérieur au égal à la longueur de la valeur du récepteur, une erreur d'exécution est déclenchée.











\subsectionGetter{componentsSeparatedByString}{string}

\begin{galgasbox}
getter componentsSeparatedByString ?@string inSeparator -> @stringlist
\end{galgasbox}
Retourne une liste des sous-chaînes de la valeur du récepteur qui a été divisée par \ggs!inSeparator!.

\begin{galgas}
let b = ["a--b--c--" componentsSeparatedByString!"--"] # "a", "b", "c", ""
\end{galgas}












\subsectionGetter{containsCharacter}{string}

\begin{galgasbox}
getter containsCharacter ?@char inCharacter -> @bool
\end{galgasbox}
Retourne \ggs!true! si le récepteur contient le caractère \ggs!inCharacter!, et \ggs!false! dans le cas contraire.

\begin{galgas}
let b = ["abcdef" containsCharacter!'c'] # true
\end{galgas}






\subsectionGetter{containsCharacterInRange}{string}
\begin{galgasbox}
getter containsCharacterInRange
  ?@char inFirstCharacter
  ?@char inLastCharacter
  -> @bool
\end{galgasbox}

Retourne \ggs!true! si le récepteur contient un ou plusieurs caractère dont le point de code est supérieur ou égal à celui de \ggs!inLastCharacter! et inférieur ou égal à celui de \ggs!inFirstCharacter!, et \ggs!false! dans le cas contraire. En conséquence, si le point de code Unicode de \ggs!inFirstCharacter! doit être strictement supérieur au point de de code de \ggs!inLastCharacter!, la valeur renvoyée est toujours \ggs!false!.

\begin{galgas}
let b = ["abcdef" containsCharacterInRange !'c' !'d'] # true
let c = ["abcdef" containsCharacterInRange !'x' !'z'] # false
\end{galgas}






\subsectionGetter{currentColumn}{string}

\begin{galgasbox}
getter currentColumn -> @uint
\end{galgasbox}

Retourne l'indice de la colonne, c'est-à-dire :
\begin{itemize}
\item si le récepteur ne contient pas de retour à la ligne, le nombre de caractères du récepteur ;
\item si le récepteur contient des retours à la ligne, le nombre de caractères du récepteur qui suivent la dernière occurrence d'un retour à la ligne.
\end{itemize}







\subsectionGetter{decimalUnsignedNumber}{string}

\begin{galgasbox}
getter decimalUnsignedNumber -> @uint
\end{galgasbox}

Retourne la valeur du récepteur convertie en entier non signé. La valeur du récepteur doit donc ne contenir que des chiffres décimaux. Si ce n'est pas le cas, une erreur d'exécution est déclenchée.

La valeur du récepteur peut être testée en appelant le \refGetterPage{string}{isDecimalUnsignedNumber}.








\subsectionGetter{directories}{string}

\begin{galgasbox}
getter directories ?@bool inRecursiveSearch -> @stringlist
\end{galgasbox}

Retourne la liste des sous-répertoires du répertoire désigné par la valeur du récepteur. Si le paramètre \ggs!inRecursiveSearch! est vrai, une recherche récursive dans les sous répertoires est effectuée.









\subsectionGetter{directoriesWithExtensions}{string}

\begin{galgasbox}
getter directoriesWithExtensions
    ?@bool inRecursiveSearch
    ?@stringlist inExtensionList -> @stringlist
\end{galgasbox}

Retourne la liste des sous-répertoires du répertoire désigné par la valeur du récepteur, en ne retenant que les répertoires dont l'extension appartient à liste \ggs!inExtensionList!. Si le paramètre \ggs!inRecursiveSearch! est vrai, une recherche récursive dans les sous répertoires est effectuée.








\subsectionGetter{directoryExists}{string}

\begin{galgasbox}
getter directoryExists -> @bool
\end{galgasbox}

Retourne \ggs!true! si la valeur du récepteur désigne un répertoire existant, et \ggs!false! dans le cas contraire.








\subsectionGetter{doesEnvironmentVariableExist}{string}

\begin{galgasbox}
getter doesEnvironmentVariableExist -> @bool
\end{galgasbox}

Retourne \ggs!true! si la valeur du récepteur nomme une variable d'environnement existante, et \ggs!false! dans le cas contraire.








\subsectionGetter{fileExists}{string}

\begin{galgasbox}
getter fileExists -> @bool
\end{galgasbox}

Retourne \ggs!true! si la valeur du récepteur désigne un fichier existant, et \ggs!false! dans le cas contraire.









\subsectionGetter{firstCharacterOrNul}{string}

\begin{galgasbox}
getter firstCharacterOrNul -> @char
\end{galgasbox}

Si la longueur de la valeur du récepteur est non nulle, retourne son premier caractère, sinon le caractère \texttt{NUL}.







\subsectionGetter{here}{string}

\begin{galgasbox}
getter here -> @lstring
\end{galgasbox}

Retourne un \ggs!@lstring! dont le champ \ggs!string! est la valeur du récepteur, et dont le champ \ggs!location! désigne la position courante de l'analyse. L'expression \ggs![s here]! est équivalente à \ggs*@lstring.new{!s !@location.here}*.








\subsectionGetter{hiddenFiles}{string}

\begin{galgasbox}
getter hiddenFiles ?@bool inRecursiveSearch -> @stringlist
\end{galgasbox}

Retourne la liste des fichiers cachés du répertoire désigné par la valeur du récepteur. Si le paramètre \ggs!inRecursiveSearch! est vrai, une recherche récursive dans les sous répertoires est effectuée.










\subsectionGetter{HTMLRepresentation}{string}

\begin{galgasbox}
getter HTMLRepresentation -> @string
\end{galgasbox}
Retourne la valeur du récepteur encodée pour former une chaîne HTML valide. Les caractères «~\texttt{\&}~», « \texttt{"} », « \texttt{<} » et « \texttt{>} » sont modifiés selon le \refTableauPage{codageHTML}.


\begin{table}[t]
  \centering
  \begin{tabular}{lllllll}
  \textbf{Caractère} & \textbf{Codage en HTML} \\
  \texttt{\&} & \texttt{\&amp;} \\
  \texttt{"} & \texttt{\&quot;} \\
  \texttt{<} & \texttt{\&lt;} \\
  \texttt{>} & \texttt{\&gt;} \\
  \end{tabular}
  \caption{Codage des caractères, \refGetterPage{string}{HTMLRepresentation}}
  \labelTableau{codageHTML}
  \ligne
\end{table}






\subsectionGetter{identifierRepresentation}{string}

\begin{galgasbox}
getter identifierRepresentation -> @string
\end{galgasbox}

Retourne la valeur du récepteur sous la forme d'une chaîne de caractères qui sera toujours un identificateur C ou C++ valide :
\begin{itemize}
\item une lettre ASCII (minuscule ou majuscule) est inchangée ;
\item tout autre caractère est traduit en son point de code en hexadécimal, précédé et suivi par un caractère « \_ ».
\end{itemize}

Par exemple :
\begin{galgas}
let x = ["Z23" identifierRepresentation] # "Z_32__33_"
let y = [":?" identifierRepresentation] # "_3A__3F_"
\end{galgas}

Voir aussi le \refGetterPage{string}{nameRepresentation} qui laisse inchangé un chiffre décimal.







\subsectionGetter{isDecimalUnsignedNumber}{string}

\begin{galgasbox}
getter isDecimalUnsignedNumber -> @bool
\end{galgasbox}

Ce \emph{getter} permet de savoir si le récepteur représente un entier non signé sur 32 bits, c'est-à-dire si le \refGetterPage{string}{decimalUnsignedNumber} peut être appelé sans qu'une erreur d'exécution se déclenche.







\subsectionGetter{isSymbolicLink}{string}

\begin{galgasbox}
getter isSymbolicLink -> @bool
\end{galgasbox}

Retourne \ggs!true! si la valeur du récepteur désigne un lien symbolique existant, et \ggs!false! dans le cas contraire.








\subsectionGetter{lastCharacter}{string}

\begin{galgasbox}
getter lastCharacter -> @char
\end{galgasbox}

Retourne le dernier caractère du récepteur. Si celui-ci est vide, une erreur d'exécution est déclenchée.








\subsectionGetter{lastPathComponent}{string}

\begin{galgasbox}
getter lastPathComponent -> @string
\end{galgasbox}

Retourne le récepteur ne contient pas de caractère « \texttt{/} », il est retourné inchangé. Sinon, ce \emph{getter} retourne la sous-chaîne de caractères qui suit la dernière occurrence du caractère « \texttt{/} ».








\subsectionGetter{leftSubString}{string}

\begin{galgasbox}
getter leftSubString ?@uint inLength -> @string
\end{galgasbox}

Retourne la sous-chaîne constituée des \ggs!inLength! derniers caractères du récepteur. Si celui-ci contient moins de \ggs!inLength! caractères, une erreur d'exécution est déclenchée.








\subsectionGetter{length}{string}

\begin{galgasbox}
getter length -> @uint
\end{galgasbox}

Retourne le nombre de caractères du récepteur.








\subsectionGetter{lowercaseString}{string}

\begin{galgasbox}
getter lowercaseString -> @string
\end{galgasbox}

Retourne la valeur du récepteur, dans laquelle toutes les lettres majuscules sont transformées en minuscules.

Par exemple :
\begin{galgas}
let x = ["AbcD" lowercaseString] # "abcd"
let y = ["ÊÆ" lowercaseString] # "êæ"
\end{galgas}







\subsectionGetter{md5}{string}

\begin{galgasbox}
getter md5 -> @string
\end{galgasbox}

Retourne la somme de contrôle MD5 du récepteur sous la forme d'une chaîne de 32 caractères hexadécimaux.

Par exemple :
\begin{galgas}
let x = ["Hello" md5] # "8B1A9953C4611296A827ABF8C47804D7"
\end{galgas}







\subsectionGetter{nameRepresentation}{string}

\begin{galgasbox}
getter nameRepresentation -> @string
\end{galgasbox}

Retourne la valeur du récepteur sous la forme d'une chaîne de caractères qui sera toujours un identificateur C ou C++ valide :
\begin{itemize}
\item une lettre ASCII (minuscule ou majuscule) est inchangée ;
\item un chiffre décimal est inchangé ;
\item tout autre caractère est traduit en son point de code en hexadécimal, précédé et suivi par un caractère « \_ ».
\end{itemize}

Par exemple :
\begin{galgas}
let x = ["Z23" nameRepresentation] # "Z23"
let y = [":?" nameRepresentation] # "_3A__3F_"
\end{galgas}

Voir aussi le \refGetterPage{string}{identifierRepresentation} qui encode un chiffre décimal.








\subsectionGetter{nativePathWithUnixPath}{string}

\begin{galgasbox}
getter nativePathWithUnixPath -> @string
\end{galgasbox}

Sous Unix, ce \emph{getter} retourne la valeur du récepteur. Sous Windows, il retourne la valeur du récepteur encodé \emph{à la Windows}.

Par exemple, sous Windows :
\begin{galgas}
let x = ["/C/Program Files/f" nativePathWithUnixPath] # "C:\Program Files\f"
\end{galgas}







\subsectionGetter{nowhere}{string}

\begin{galgasbox}
getter nowhere -> @lstring
\end{galgasbox}

Retourne un \ggs!@lstring! dont le champ \ggs!string! est la valeur du réception, et dont le champ \ggs!location! est vide.







\subsectionGetter{pathExtension}{string}

\begin{galgasbox}
getter pathExtension -> @string
\end{galgasbox}

Si le récepteur ne contient pas de caractère « \texttt{.} », la chaîne vide est retournée. Sinon, ce \emph{getter} retourne la sous-chaîne de caractères qui suit la dernière occurrence du caractère « \texttt{.} ».








\subsectionGetter{popen}{string}

\begin{galgasbox}
getter popen -> @string
\end{galgasbox}

Ce \emph{getter} exécute la commande Shell exprimée par la valeur du récepteur. La sortie de cette commande est accumulée et retournée par ce \emph{getter} lorsque la commande est terminée.








\subsectionGetter{range}{string}

\begin{galgasbox}
getter range -> @range
\end{galgasbox}

Retourne un objet de type \refTypePredefini{range} dont le champ \ggs!start! est $0$ et le champ \ggs!length! est égal au nombre de caractères du récepteur.








\subsectionGetter{regularFiles}{string}

\begin{galgasbox}
getter regularFiles ?@bool inRecursiveSearch -> @stringlist
\end{galgasbox}

Retourne la liste des fichiers non cachés du répertoire désigné par la valeur du récepteur. Si le paramètre \ggs!inRecursiveSearch! est vrai, une recherche récursive dans les sous répertoires est effectuée.









\subsectionGetter{regularFilesWithExtensions}{string}

\begin{galgasbox}
getter regularFilesWithExtensions
    ?@bool inRecursiveSearch
    ?@stringlist inExtensionList -> @stringlist
\end{galgasbox}

Retourne la liste des fichiers non cachés du répertoire désigné par la valeur du récepteur, en ne retenant que les fichiers dont l'extension est nommée dans \ggs!inExtensionList!. Si le paramètre \ggs!inRecursiveSearch! est vrai, une recherche récursive dans les sous répertoires est effectuée.









\subsectionGetter{relativePathFromPath}{string}

\begin{galgasbox}
getter relativePathFromPath ?@string inPath -> @string
\end{galgasbox}

Retourne le chemin relatif du récepteur à partir du chemin \ggs!inPath!.









\subsectionGetter{reversedString}{string}

\begin{galgasbox}
getter reversedString -> @string
\end{galgasbox}

Retourne la valeur renversée du récepteur.

\begin{galgas}
let x = ["abcde" reversedString] # "edcba"
\end{galgas}








\subsectionGetter{rightSubString}{string}

\begin{galgasbox}
getter rightSubString ?@uint inLength -> @string
\end{galgasbox}

Retourne la sous-chaîne constituée des \ggs!inLength! premiers caractères du récepteur. Si celui-ci contient moins de \ggs!inLength! caractères, une erreur d'exécution est déclenchée.








\subsectionGetter{stringByCapitalizingFirstCharacter}{string}

\begin{galgasbox}
getter stringByCapitalizingFirstCharacter -> @string
\end{galgasbox}

Retourne la valeur du récepteur dans laquelle le premier caractère, si il est une lettre, a été mis en majuscule.








\subsectionGetter{stringByDeletingLastPathComponent}{string}

\begin{galgasbox}
getter stringByDeletingLastPathComponent -> @string
\end{galgasbox}

Si le récepteur ne contient pas de caractère « \texttt{/} », la chaîne vide est retournée. Sinon, ce \emph{getter} retourne la sous-chaîne de caractères qui précède la dernière occurrence du caractère « \texttt{/} ».









\subsectionGetter{stringByDeletingPathExtension}{string}

\begin{galgasbox}
getter stringByDeletingPathExtension -> @string
\end{galgasbox}

Si le récepteur ne contient pas de caractère « \texttt{.} », la valeur du récepteur est retournée. Sinon, ce \emph{getter} retourne la sous-chaîne de caractères qui précède la dernière occurrence du caractère «~\texttt{.}~».








\subsectionGetter{stringByLeftAndRightPadding}{string}

\begin{galgasbox}
getter stringByLeftAndRightPadding
   ?@uint inPaddedStringLength
   ?@char inPaddingChar -> @string
\end{galgasbox}

Si la longueur de la valeur du récepteur est supérieure ou égal à \ggs!inPaddedStringLength!, la valeur du récepteur est retournée. Sinon, ce \emph{getter} retourne la valeur du récepteur précédée et suivie d'un nombre égal de caractères \ggs!inPaddingChar!, de façon à atteindre la longueur \ggs!inPaddedStringLength!. Si le nombre de caractères à ajouter est impair, le nombre de caractères ajoutés est supérieur d'une unité au nombre de caractères insérés au début.








\subsectionGetter{stringByLeftPadding}{string}

\begin{galgasbox}
getter stringByLeftPadding
   ?@uint inPaddedStringLength
   ?@char inPaddingChar -> @string
\end{galgasbox}

Si la longueur de la valeur du récepteur est supérieure ou égal à \ggs!inPaddedStringLength!, la valeur du récepteur est retournée. Sinon, ce \emph{getter} retourne la valeur du récepteur suivie d'un nombre de caractères \ggs!inPaddingChar!, de façon à atteindre la longueur \ggs!inPaddedStringLength!.







\subsectionGetter{stringByRemovingCharacterAtIndex}{string}

\begin{galgasbox}
getter stringByRemovingCharacterAtIndex ?@uint inIndex -> @string
\end{galgasbox}

Retourne la valeur du récepteur, amputée du caractère situé à l'indice \ggs!inIndex!. Une erreur d'exécution est déclenchée si \ggs!inIndex! est supérieur ou égal au nombre de caractères du récepteur.







\subsectionGetter{stringByReplacingStringByString}{string}

\begin{galgasbox}
getter stringByReplacingStringByString
     ?@string inSearchedString
     ?@string inReplacementString  -> @string
\end{galgasbox}

Retourne la valeur du récepteur, dans laquelle chaque occurrence de \ggs!inSearchedString! est remplacée par \ggs!inReplacementString!.







\subsectionGetter{stringByRightPadding}{string}

\begin{galgasbox}
getter stringByRightPadding
   ?@uint inPaddedStringLength
   ?@char inPaddingChar -> @string
\end{galgasbox}

Si la longueur de la valeur du récepteur est supérieure ou égal à \ggs!inPaddedStringLength!, la valeur du récepteur est retournée. Sinon, ce \emph{getter} retourne la valeur du récepteur précédée d'un nombre de caractères \ggs!inPaddingChar!, de façon à atteindre la longueur \ggs!inPaddedStringLength!.




\subsectionGetter{stringByStandardizingPath}{string}

\begin{galgasbox}
getter stringByStandardizingPath -> @string
\end{galgasbox}

Retourne la valeur du récepteur, dans laquelle les séquences « \texttt{./} » et « \texttt{../} » sont supprimées.







\subsectionGetter{stringByTrimmingWhiteSpaces}{string}

\begin{galgasbox}
getter stringByTrimmingWhiteSpaces -> @string
\end{galgasbox}

Retourne la valeur du récepteur dans laquelles les espaces en tête et en fin ont été supprimés.




\subsectionGetter{subString}{string}

\begin{galgasbox}
getter subString ?@uint inStart ?@uint inLength -> @string
\end{galgasbox}

Retourne la sous-chaîne de \ggs!inLength! caractères extraite de la valeur du récepteur à partir de l'indice \ggs!inStart!.






\subsectionGetter{subStringFromIndex}{string}

\begin{galgasbox}
getter subStringFromIndex ?@uint inIndex -> @string
\end{galgasbox}

Retourne la sous-chaîne de caractères extraite de la valeur du récepteur à partir de l'indice \ggs!inIndex!. Si \ggs!inIndex! est supérieur ou égal au nombre de caractères du récepteur, la chaîne vide est retournée.






\subsectionGetter{system}{string}

\begin{galgasbox}
getter system -> @sint
\end{galgasbox}

Appelle la routine C \texttt{system} avec la valeur du récepteur en argument. La valeur retournée par cet appel est la valeur retournée par ce \emph{getter}.






\subsectionGetter{unixPathWithNativePath}{string}

\begin{galgasbox}
getter unixPathWithNativePath -> @string
\end{galgasbox}

La valeur du récepteur est un chemin valide pour la plateforme courante.

Sous Windows, ce \emph{getter} retourne la valeur du récepteur sous la forme d'un chemin Unix. Sous Unix, la valeur du récepteur est renvoyée.













\subsectionGetter{uppercaseString}{string}

\begin{galgasbox}
getter uppercaseString -> @string
\end{galgasbox}

Retourne la valeur du récepteur, dans laquelle toutes les lettres minuscules sont transformées en majuscules.

Par exemple :
\begin{galgas}
let x = ["AbcD" uppercaseString] # "ABCD"
let y = ["êæ" uppercaseString] # "ÊÆ"
\end{galgas}














\subsectionGetter{utf32Representation}{string}

\begin{galgasbox}
getter utf32Representation -> @string
\end{galgasbox}















\subsectionGetter{utf8Representation}{string}

\begin{galgasbox}
getter utf8Representation -> @string
\end{galgasbox}

Retourne la valeur du récepteur sous la forme d'une chaîne UTF-8, c'est-à-dire qu'un caractère « \texttt{"} » a été inséré au début et à la fin. Tous les 150 caractères, la séquence « \texttt{"\textbackslash n~~"} » est insérée.








\subsectionGetter{utf8RepresentationWithoutDelimiters}{string}

\begin{galgasbox}
getter utf8RepresentationWithoutDelimiters -> @string
\end{galgasbox}

Retourne la valeur du récepteur sous la forme d'une chaîne UTF-8 sans le délimiteur initial et ni le délimiteur terminal, c'est-à-dire qu'aucun caractère « \texttt{"} » n'est inséré au début et à la fin. Tous les 150 caractères, la séquence « \texttt{"\textbackslash n~~"} » est insérée.














\section{Méthodes}





\subsectionMethod{makeDirectory}{string}

\begin{galgasbox}
method makeDirectory
\end{galgasbox}

Crée le répertoire désigné par la valeur du récepteur. Les répertoires intermédiaires sont créés si besoin. Si le répertoire existe déjà, aucune erreur n'est déclenchée.






\subsectionMethod{makeDirectoryAndWriteToExecutableFile}{string}

\begin{galgasbox}
method makeDirectoryAndWriteToExecutableFile ?@string inFilePath
\end{galgasbox}

Le récepteur contient la chaîne de caractères qui est écrit dans le fichier \ggs!inFilePath!. Les répertoires intermédiaires sont créés si besoin. Le fichier \ggs!inFilePath! est rendu exécutable.







\subsectionMethod{makeDirectoryAndWriteToFile}{string}

\begin{galgasbox}
method makeDirectoryAndWriteToFile ?@string inFilePath
\end{galgasbox}

Le récepteur contient la chaîne de caractères qui est écrit dans le fichier \ggs!inFilePath!. Les répertoires intermédiaires sont créés si besoin.








\subsectionMethod{makeSymbolicLinkWithPath}{string}

\begin{galgasbox}
method makeSymbolicLinkWithPath ?@string inFilePath
\end{galgasbox}








\subsectionMethod{writeToExecutableFile}{string}

\begin{galgasbox}
method writeToExecutableFile ?@string inFilePath
\end{galgasbox}

Le récepteur contient la chaîne de caractères qui est écrit dans le fichier \ggs!inFilePath!. Aucun répertoire intermédiaire n'est créé, c'est une erreur d'exécution si ils n'existent pas. Si il a été écrit avec succès, le fichier \ggs!inFilePath! est rendu exécutable.









\subsectionMethod{writeToExecutableFileWhenDifferentContents}{string}

\begin{galgasbox}
method writeToExecutableFileWhenDifferentContents
    ?@string inFilePath
    !@bool outFileModified
\end{galgasbox}

Le récepteur contient la chaîne de caractères qui est écrit dans le fichier \ggs!inFilePath!. Aucun répertoire intermédiaire n'est créé, c'est une erreur d'exécution si ils n'existent pas. Si il a été écrit avec succès, le fichier \ggs!inFilePath! est rendu exécutable. Le booléen \ggs!outFileModified! permet de savoir si le contenu original du fichier a été modifié.









\subsectionMethod{writeToFile}{string}

\begin{galgasbox}
method writeToFile ?@string inFilePath
\end{galgasbox}

Le récepteur contient la chaîne de caractères qui est écrit dans le fichier \ggs!inFilePath!. Aucun répertoire intermédiaire n'est créé, c'est une erreur d'exécution si ils n'existent pas.









\subsectionMethod{writeToFileWhenDifferentContents}{string}

\begin{galgasbox}
method writeToFileWhenDifferentContents
    ?@string inFilePath
    !@bool outFileModified
\end{galgasbox}

Le récepteur contient la chaîne de caractères qui est écrit dans le fichier \ggs!inFilePath!. Aucun répertoire intermédiaire n'est créé, c'est une erreur d'exécution si ils n'existent pas.Le booléen \ggs!outFileModified! permet de savoir si le contenu original du fichier a été modifié.















\section{setters}

\subsectionSetter{appendSpacesUntilColumn}{string}

\begin{galgasbox}
setter appendSpacesUntilColumn ?@uint inColumnIndex
\end{galgasbox}

Ce \emph{setter} ajoute des espaces en fin de ligne jusqu'à atteindre la colonne \ggs!inColumnIndex!.




\subsectionSetter{decIndentation}{string}

\begin{galgasbox}
setter decIndentation ?@uint inAmount
\end{galgasbox}

Lorsqu'un retour à la ligne est inséré, des espaces sont automatiquement insérés à la suite. Initialement, ce nombre est nul. Le \refSetterPage{string}{incIndentation} permet de l'incrémenter, ce \emph{setter} de le décrémenter. Ces deux \emph{setters} permettent d'obtenir facilement des sorties où le texte est indenté.





\subsectionSetter{incIndentation}{string}

\begin{galgasbox}
setter incIndentation ?@uint inAmount
\end{galgasbox}

Lorsqu'un retour à la ligne est inséré, des espaces sont automatiquement insérés à la suite. Initialement, ce nombre est nul. Ce \emph{setter} permet de l'incrémenter, le \refSetterPage{string}{incIndentation} de le décrémenter. Ces deux \emph{setters} permettent d'obtenir facilement des sorties où le texte est indenté.








\subsectionSetter{setCapacity}{string}

\begin{galgasbox}
setter setCapacity ?@uint inCapacity
\end{galgasbox}


Ce \emph{setter} ajuste la zone mémoire allouée au buffer de la chaîne de caractères. Si \ggs!inCapacity! est strictemement inférieur au nombre de caractères, ce \emph{setter} n'a pas d'effet.










\subsectionSetter{setCharacterAtIndex}{string}

\begin{galgasbox}
setter setCharacterAtIndex
   ?@char inChar
   ?@uint inIndex
\end{galgasbox}


Ce \emph{setter} remplace le caractère d'indice \ggs!inIndex! par le caractère \ggs!inChar!. Si \ggs!inIndex! est supérieur ou égal au nombre de caractères, une erreur d'exécution est déclenchée. 




\section{Procédures de type}


\subsectionStaticProc{deleteFile}{string}


\begin{galgasbox}
proc @string deleteFile ?@string inFilePath
\end{galgasbox}

Supprime le fichier \ggs!inFilePath!. Une erreur d'exécution est déclenchée si le fichier n'existe pas.




\subsectionStaticProc{deleteFileIfExists}{string}


\begin{galgasbox}
proc @string deleteFileIfExists ?@string inFilePath
\end{galgasbox}

Supprime le fichier \ggs!inFilePath!. Aucune erreur d'exécution n'est déclenchée si le fichier n'existe pas.






\subsectionStaticProc{generateFile}{string}


\begin{galgasbox}
proc @string generateFile
   ?@string inStartPath
   ?@string inFileName
   ?@string inContents
\end{galgasbox}

Cette procédure commence par rechercher le fichier \ggs!inFileName! dans le répertoire \ggs!inStartPath!, et récursivement dans ses sous-répertoires.

Si le fichier existe, son contenu est remplacé par \ggs!inContents!.

Si il n'existe pas, il est créé dans le répertoire \ggs!inStartPath! avec le nom \ggs!inFileName!, et le contenu \ggs!inContents!.








\subsectionStaticProc{generateFileWithPattern}{string}


\begin{galgasbox}
proc @string generateFileWithPattern
   ?startPath:@string inStartPath
   ?fileName:@string inFileName
   ?lineComment:@string inLineCommentPrefix
   ?header:@string inHeader
   ?defaultUserZone1:@string inDefaultUserZone1
   ?generatedZone2:@string inGeneratedZone2
   ?defaultUserZone2:@string inDefaultUserZone2
   ?generatedZone3:@string inGeneratedZone3
   ?makeExecutable:@bool inMakeExecutable
\end{galgasbox}






\subsectionStaticProc{removeDirectoryRecursively}{string}


\begin{galgasbox}
proc @string removeDirectoryRecursively ?@string inDirectoryPath
\end{galgasbox}

Supprime le répertoire \ggs!inDirectoryPath!, après avoir supprimé tous ses fichiers et récursivement tous ses sous-répertoires.







\subsectionStaticProc{removeEmptyDirectory}{string}


\begin{galgasbox}
proc @string removeEmptyDirectory ?@string inDirectoryPath
\end{galgasbox}

Supprime le répertoire \ggs!inDirectoryPath!. Une erreur d'exécution est déclenchée si le répertoire n'est pas vide.


