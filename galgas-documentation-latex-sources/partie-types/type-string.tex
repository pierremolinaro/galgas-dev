%!TEX encoding = UTF-8 Unicode
%!TEX root = ../galgas-book.tex

\chapitreTypePredefiniLabelIndex{string}

A \ggs+@string+ object value is an Unicode character string value. The \ggs+@string+ type defines several constructors, getters constant methods and setters, described below.

\textbf{Literal String Constants.} Characters strings are written enclosed within quotation marks (") characters, as in many languages. For example: \ggs+"a string"+. Note that a literal string constant is an actual \ggs+@string+ object, so a getter can be used on it. For example: \ggs+["ae" uppercaseString]+ returns the \ggs+"AE"+ string.

\section{Getters}







\subsectionGetter{containsCharacter}{string}

\begin{galgascode}
getter count ?@char inCharacter -> @bool
\end{galgascode}
Returns true if the receiver contains the given character, and false oteherwise.

\begin{galgascode}
@string s = "abcdef";
@string s2 = [s rightSubString!3]; # The value of s2 is "def"
\end{galgascode}






\subsectionGetter{length}{string}

\begin{galgascode}
getter length -> @uint
\end{galgascode}

Retourne le nombre de caractères du récepteur.








\subsectionGetter{subString}{string}

\begin{galgascode}
getter subString ?@uint inStart ?@uint inLength -> @string
\end{galgascode}

Creates and returns the string built with the \emph{inLength} last characters of the receiver. If the receiver contains less than inLength characters, the receiver’s value is returned.




%
%
%\subsection{the \emph{$\&$} Operator}
%
%The \emph{$\&$} operator returns the intersection of its operand values.
%
%\exempleTroisLignes
%{}
%{@stringset s1 = ... ;}
%{@stringset s2 = ... ;}
%{@stringset s = s1 \& s2 ; \# s is the intersection of s1 and s2}
%
%
%
%
%
%
%\subsection{the \emph{$\textbar$} Operator}
%
%The \emph{$\textbar$} operator returns the union of its operand values.
%
%\exempleTroisLignes
%{}
%{@stringset s1 = ... ;}
%{@stringset s2 = ... ;}
%{@stringset s = s1 \textbar s2 ; \# s is the union of s1 and s2}
%
%
%
%
%
%
%\subsection{the \emph{$-$} Operator}
%
%The \emph{$-$} operator returns the difference of its operand values.
%
%\exempleTroisLignes
%{}
%{@stringset s1 = ... ;}
%{@stringset s2 = ... ;}
%{@stringset s = s1 - s2 ; \# s is the difference of s1 and s2}
%
%
%
%
%
%
%
%
%\subsection{Enumerating \ggs+@stringset+ objects}
%
%
%The \motCle{foreach} instruction can be used for enumerating \ggs+@stringset+ values; enumeration is performed in the ascending order, or in the reverse alphabetical order using the '>' qualifier.
%
%\texttt{@stringset s = ... ;}\newline
%\textbf{foreach} \texttt {s} \textbf {do}\newline
%\texttt{\# the \emph{key} constant has the value of current entry of \emph{s} stringset}\newline
%\textbf{end foreach} \texttt{;}
%
%
%
%
%
%
%
%\subsection{Comparison Operators}
%
%The \ggs+@stringset+ type supports the six comparison operators:\newline
%
%\begin{tabular}{|c|c|}
%\hline
%$=$ & Equality \\
%\hline
%$!=$ & Non Equality \\
%\hline
%$<$  & Strict Inclusion \\
%\hline
%$<=$  & Inclusion or Equality \\
%\hline
%$>$  & Strict Greater \\
%\hline
%$>=$  & Greater or Equality \\
%\hline
%\end{tabular}
%
%Theses operators require both arguments to be \ggs+@stringset+ objects, and return a \ggs+@stringset+ object.
%
%
