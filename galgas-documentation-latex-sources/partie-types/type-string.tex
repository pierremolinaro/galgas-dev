%!TEX encoding = UTF-8 Unicode
%!TEX root = ../galgas-book.tex

\chapitreTypePredefiniLabelIndex{string}

A \galgas{@string} object value is an Unicode character string value. The @string type defines several constructors, readers constant methods and modifiers, described below.

\paragraph{Literal String Constants.}

Characters strings are written enclosed within quotation marks (") characters, as in many languages. For example: "a string". Note that a literal string constant is an actual @string object, so a reader can be used on it. For example: \lstinline[language=galgas]{["ae" uppercaseString]} returns the "AE" string.

\section{Readers}

\readerUnArgument{rightSubString}
{string}
{1.7.7}
{string}
{@uint inLength}
{Creates and returns the string built with the \emph{inLength} last characters of the receiver. If the receiver contains less than inLength characters, the receiver’s value is returned.}
{}

\begin{galgascode}
@string s := "abcdef";
@string s2 := [s rightSubString!3]; # The value of s2 is "def"
\end{galgascode}

\readerDeuxArguments{subString}
{string}
{1.7.8}
{string}
{@uint inStart}
{@uint inLength}
{Creates and returns the string built with the \emph{inLength} last characters of the receiver. If the receiver contains less than inLength characters, the receiver’s value is returned.}
{}


%\constructeurSansArgument{emptySet}
%{@stringset}
%{1.3.0}
%{@stringset}
%{Creates and returns an empty \galgas{@stringset} object.}
%{}
%
%\constructeurUnArgument{setWithString}
%{@stringset}
%{1.3.0}
%{@stringset}
%{@string inString}
%{Creates and returns an \galgas{@stringset} object that contains the value of the \emph{inString} argument object.}
%{}
%
%
%\readerSansArgument{count}
%{@stringset}
%{1.3.0}
%{@uint}
%{Returns the number of strings in the set.}
%{}
%
%
%
%\readerUnArgument{hasKey}
%{@stringset}
%{1.3.0}
%{@bool}
%{@string inString}
%{Returns a boolean value that indicates whether the value of \emph{inString} argument is present in the set.}
%{returns \motCle{true} if the value of \emph{inString} argument is present in the set, \motCle{false} otherwise.}
%
%
%
%
%\modifierUnArgument{removeKey}
%{@stringset}
%{1.3.0}
%{@string inString}
%{Removes the value of \emph{inString} argument from the receiver's value.}
%{if the receiver's value does not contain the value of \emph{inString} argument, this modifier leaves the receiver's value unchanged.}
%
%
%
%
%
%
%\subsection{the \emph{+=} Operator}
%
%The \emph{+=} operator adds a string value to the receiver. If the receiver's value already contains the added value, this operator has no effect.
%
%\exempleTroisLignes
%{}
%{@string aString := ... ;}
%{@stringset aStringSet := ... ;}
%{aStringSet += !aString ;}
%
%
%
%
%\subsection{the \emph{$\&$} Operator}
%
%The \emph{$\&$} operator returns the intersection of its operand values.
%
%\exempleTroisLignes
%{}
%{@stringset s1 := ... ;}
%{@stringset s2 := ... ;}
%{@stringset s := s1 \& s2 ; \# s is the intersection of s1 and s2}
%
%
%
%
%
%
%\subsection{the \emph{$\textbar$} Operator}
%
%The \emph{$\textbar$} operator returns the union of its operand values.
%
%\exempleTroisLignes
%{}
%{@stringset s1 := ... ;}
%{@stringset s2 := ... ;}
%{@stringset s := s1 \textbar s2 ; \# s is the union of s1 and s2}
%
%
%
%
%
%
%\subsection{the \emph{$-$} Operator}
%
%The \emph{$-$} operator returns the difference of its operand values.
%
%\exempleTroisLignes
%{}
%{@stringset s1 := ... ;}
%{@stringset s2 := ... ;}
%{@stringset s := s1 - s2 ; \# s is the difference of s1 and s2}
%
%
%
%
%
%
%
%
%\subsection{Enumerating \galgas{@stringset} objects}
%
%
%The \motCle{foreach} instruction can be used for enumerating \galgas{@stringset} values; enumeration is performed in the ascending order, or in the reverse alphabetical order using the '>' qualifier.
%
%\texttt{@stringset s := ... ;}\newline
%\textbf{foreach} \texttt {s} \textbf {do}\newline
%\texttt{\# the \emph{key} constant has the value of current entry of \emph{s} stringset}\newline
%\textbf{end foreach} \texttt{;}
%
%
%
%
%
%
%
%\subsection{Comparison Operators}
%
%The \galgas{@stringset} type supports the six comparison operators:\newline
%
%\begin{tabular}{|c|c|}
%\hline
%$=$ & Equality \\
%\hline
%$!=$ & Non Equality \\
%\hline
%$<$  & Strict Inclusion \\
%\hline
%$<=$  & Inclusion or Equality \\
%\hline
%$>$  & Strict Greater \\
%\hline
%$>=$  & Greater or Equality \\
%\hline
%\end{tabular}
%
%Theses operators require both arguments to be \galgas{@stringset} objects, and return a \galgas{@stringset} object.
%
%
