%!TEX encoding = UTF-8 Unicode
%!TEX root = ../galgas-book.tex

\chapter{Compléter le système de types}

\section{Ajouter une méthode , un \emph{getter}, un \emph{setter} ou un constructeur à un type prédéfini}

Ajouter une méthode, un \emph{getter}, un \emph{setter} ou un constructeur à un type prédéfini s'effectue en quatre temps :
\begin{enumerate}
\item ajouter la méthode, le \emph{getter}, le \emph{setter} ou le constructeur dans GALGAS ;
\item reconstruire le fichier d'en-tête des types prédéfinis ;
\item implémenter la méthode, le \emph{getter}, le \emph{setter} ou le constructeur en C++ ;
\item mettre à jour la documentation \LaTeX.
\end{enumerate}

À titre d'exemple, nous allons montrer comment la méthode \ggs!makeDirectoryAndWriteToExecutableFile! de la classe \ggs!@string! a été ajoutée.

\subsection{Ajouter la méthode dans GALGAS}

Éditez le fichier GALGAS, en fonction du \refTableau{ajoutMethodeTypePrédéfini}. Comme c'est une méthode que nous voulons ajouter, on édite le fichier \texttt{galgas-sources/semanticsInstanceMethods.galgas}.

Dans ce fichier, il y a une méthode pour chaque type prédéfini. Pour la classe \ggs!@string!, on a :
\begin{galgas}
override method @stringPredefinedTypeAST getInstanceMethodMap
  ?!@unifiedTypeMap ioUnifiedTypeMap
  !@instanceMethodMap outInstanceMethodMap
{
  outInstanceMethodMap = {}
  enterInstanceMethodWithInputArgument (
    !?outInstanceMethodMap
    !?ioUnifiedTypeMap
    !inputArgTypeName:"string"
    !inputArgName:"inFilePath"
    !methodName:"writeToFile"
    !true
  )
  ...
\end{galgas}

\begin{table}[t]
  \centering
  \begin{tabular}{lll}
    \textbf{Opération} & \textbf{Fichier} \\
    Ajouter un constructeur &  \tpp{galgas-sources/semanticsConstructors.galgas}\\
    Ajouter un \emph{getter} &  \tpp{galgas-sources/semanticsGetters.galgas}\\
    Ajouter un \emph{setter} &  \tpp{galgas-sources/semanticsSetters.galgas}\\
    Ajouter une méthode &  \tpp{galgas-sources/semanticsInstanceMethods.galgas}\\
    Ajouter une méthode de type &  \tpp{galgas-sources/semanticsTypeMethods.galgas}\\
  \end{tabular}
  \caption{Fichier GALGAS à éditer pour compléter un type prédéfini}
  \labelTableau{ajoutMethodeTypePrédéfini}
  \ligne
\end{table}

Pour ajouter la méthode \ggs!makeDirectoryAndWriteToExecutableFile! de la classe \ggs!@string!, on complète cette méthode par :
\begin{galgas}
override method @stringPredefinedTypeAST getInstanceMethodMap
  ?!@unifiedTypeMap ioUnifiedTypeMap
  !@instanceMethodMap outInstanceMethodMap
{
  outInstanceMethodMap = {}
  ...
  enterInstanceMethodWithInputArgument (
    !?outInstanceMethodMap
    !?ioUnifiedTypeMap
    !inputArgTypeName:"string"
    !inputArgName:"inFilePath"
    !methodName:"makeDirectoryAndWriteToExecutableFile"
    !true
  )
}
\end{galgas}

\subsection{Reconstruire le fichier d'en-tête des types prédéfinis}

Le fichier \tpp{libpm/galgas2/predefined-types.h} contient la déclaration C++ de tous les types prédéfinis. Le fichier \tpp{libpm/galgas2/predefined-types.cpp} contient l'implémentation des constructions génériques des types prédéfinis. {\bf Surtout n'éditez pas ces fichiers à la main !} On va utiliser GALGAS pour les reconstruire. Pour cela, appeler le script Shell \tpp{libpm/galgas2/-build-builtin-type-headers.command}. L'exécution de celui-ci recompile GALGAS, et engendre les nouvelles versions des fichiers \tpp{predefined-types.h} et \tpp{predefined-types.cpp}. Voici ce que l'on obtient :

\begin{mdframed}[hidealllines=true,backgroundcolor=gray!10] \tt\small
\textcolor{blue}{Native Compiling for Mac OS X (debug): all-declarations-26.cpp}\\
...\\
\textcolor{blue}{Native Compiling for Mac OS X (debug): check-gmp.cpp}\\
\textcolor{blue}{Native Linking for Mac OS X (debug): galgas-debug}\\
Done at +12s\\
\textcolor{OliveGreen}{Replaced '/Volumes/dev-svn/galgas/libpm/galgas2/predefined-types.h'.}\\
No warning, no error.\\
\verb![!Displayed from file 'all-declarations-19.cpp' at line 952\verb!]!\\
11800 memory blocks, 4158 arraies have been used.\\
7052 POD arraies have been used, 706 have been reallocated (509 with pointer change).
\end{mdframed}

Ici, seul le fichier \tpp{predefined-types.h} a été modifié, le fichier \tpp{predefined-types.cpp} ne nécessitait pas de modification.



\subsection{Implémenter la méthode en C++}

Maintenant, effectuer la compilation C++ du projet GALGAS, soit avec Xcode, soit avec le \emph{makefile} natif de votre choix. {\bf Il est normal 
que cette compilation échoue, la méthode n'a pas encore été implémentée.}

Éditez le fichier \tpp{libpm/galgas2/GALGAS\_string.cpp} et ajouter la méthode :

\begin{lstlisting}[language=C++]
void GALGAS_string::
method_makeDirectoryAndWriteToExecutableFile (GALGAS_string inFilePath,
                                              C_Compiler * inCompiler
                                              COMMA_LOCATION_ARGS) const {
  if (isValid () && inFilePath.isValid ()) {
  //--- Make directory
    const C_String directory = inFilePath.mString.stringByDeletingLastPathComponent () ;
    bool ok = C_FileManager::makeDirectoryIfDoesNotExist (directory) ;
    if (! ok) {
      C_String message ;
      message << "cannot create '" << directory << "' directory" ;
      inCompiler->onTheFlyRunTimeError (message COMMA_THERE) ;
    }else{
      method_writeToExecutableFile (inFilePath, inCompiler COMMA_THERE) ;
    }
  }
}
\end{lstlisting}

Maintenant la compilation C++ de GALGAS s'effectue correctement. Mais ce n'est pas terminé !

\subsection{Finaliser le nouveau compilateur GALGAS}

L'exécutable GALGAS embarque le source de la libraire \tpp{libpm}. Or, à ce stade, c'est l'ancienne version qui est empbarquée. Lorsque l'on compile le projet \tpp{+galgas.galgasProject}, la librairie \tpp{libpm} est intégrée dans les sources C++ engendrés. 

Il faut donc effectuer itérativement des cycles \emph{compilation GALGAS} -- \emph{compilation C++} tant que la compilation GALGAS apporte des modifications du code C++ engendré.

