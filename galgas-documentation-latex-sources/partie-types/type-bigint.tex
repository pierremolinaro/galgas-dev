%!TEX encoding = UTF-8 Unicode
%!TEX root = ../galgas-book.tex

\chapitreTypePredefiniLabelIndex{bigint}

Le \ggs+@bigint+ définit les entiers signés d'une taille quelconque, seulement limitée par la mémoire disponible. Ce type est simplement une interface des entiers de la librairie GMP\footnote{\url{http://www.gmplib.org}.}.

\section{Constante littérale}

Utiliser le suffixe « \texttt{G} » pour définir une constante littérale de type \ggs!@bigint! :
\begin{galgas}
@bigint a = 1234567890_1234567890_1234567890_G
message [a string] + "\n" # Affiche "123456789012345678901234567890"
\end{galgas}

Vous pouvez utiliser le caractère de soulignement « \texttt{\_} » pour séparer les chiffres.

Avec le préfixe « \texttt{0x} », vous pouvez écrire les nombres en héxadécimal :
\begin{galgas}
@bigint a = 0x123456789ABCDEF0_123456789abcdefG
message [a hexString] + "\n" # Affiche "123456789ABCDEF0_123456789ABCDEF"
\end{galgas}

Les lettres minuscules et majuscules sont utilisables.

\section{Constructeurs}

\subsectionConstructor{zero}{bigint}

Le constructeur \ggs!zero! renvoie un \ggs!@bigint! initialisé à zéro :
\begin{galgas}
@bigint a = .zero
message [a string] + "\n" # Affiche "0"
\end{galgas}


\subsectionConstructor{sint}{bigint}

Le constructeur \ggs!sint! permet de construire un \ggs!@bigint! à partir d'une valeur de type \ggs!@sint! :
\begin{galgas}
@bigint a = .sint {!-678S}
message [a string] + "\n" # Affiche "-678"
\end{galgas}


\subsectionConstructor{sint64}{bigint}

Le constructeur \ggs!sint64! permet de construire un \ggs!@bigint! à partir d'une valeur de type \ggs!@sint64! :
\begin{galgas}
@bigint a = .sint64 {!-678LS}
message [a string] + "\n" # Affiche "-678"
\end{galgas}




\subsectionConstructor{uint}{bigint}

Le constructeur \ggs!uint! permet de construire un \ggs!@bigint! à partir d'une valeur de type \ggs!@uint! :
\begin{galgas}
@bigint a = .uint {!678}
message [a string] + "\n" # Affiche "678"
\end{galgas}




\subsectionConstructor{uint64}{bigint}

Le constructeur \ggs!uint64! permet de construire un \ggs!@bigint! à partir d'une valeur de type \ggs!@uint64! :
\begin{galgas}
@bigint a = .uint64 {!678L}
message [a string] + "\n" # Affiche "678"
\end{galgas}









\section{Conversions en chaîne de caractères}

\subsectionGetter{string}{bigint}

\begin{galgas}
getter string -> @bigint
\end{galgas}

Ce getter renvoie la valeur du receveur sous la forme d'une chaîne de caractères décimaux (de \ggs!0! à \ggs!9!). Si cette valeur est négative, le premier caractère est un signe \ggs!-!. Par exemple :

\begin{galgas}
@bigint a = -1234567890_1234567890_1234567890_G
message [a string] + "\n" # Affiche "-123456789012345678901234567890"
\end{galgas}





\subsectionGetter{hexString}{bigint}

\begin{galgas}
getter hexString -> @bigint
\end{galgas}

Ce getter renvoie la valeur du receveur sous la forme d'une chaîne de caractères héxadécimaux (\ggs!0! à \ggs!9!, \ggs!A! à \ggs!F!). Si cette valeur est négative, le premier caractère est un signe \ggs!-!. Il n'y a pas de préfixe « \texttt{0x} ». Exemple :

\begin{galgas}
@bigint a = -1234567890_1234567890_1234567890_G
message [a hexString] + "\n" # Affiche "-18EE90FF6C373E0EE4E3F0AD2"
\end{galgas}









\section{Arithmétique}

\subsection{Opérateurs \texttt{+} et \texttt{-} unaires}

L'opérateur \texttt{-} unaire effectue la négation de l'expression qui le suit. L'opérateur \texttt{+} unaire n'a aucun effet, il retourne la valeur de l'expression.

\begin{galgas}
@bigint a = +1234567890_1234567890_1234567890_G
message [a string] + "\n" # Affiche "123456789012345678901234567890"
\end{galgas}


