%!TEX encoding = UTF-8 Unicode
%!TEX root = ../galgas-book.tex

\chapitreTypePredefiniLabelIndex{bigint}

Le \ggs+@bigint+ définit les entiers signés d'une taille quelconque, seulement limitée par la mémoire disponible. Ce type est simplement une interface des entiers de la librairie GMP\footnote{\url{http://www.gmplib.org}.}.

\section{Constante littérale}

Utiliser le suffixe « \texttt{G} » pour définir une constante littérale de type \ggs!@bigint! :
\begin{galgas}
@bigint a = 1234567890_1234567890_1234567890_G
message [a string] + "\n" # Affiche "123456789012345678901234567890"
\end{galgas}

Vous pouvez utiliser le caractère de soulignement « \texttt{\_} » pour séparer les chiffres.

Avec le préfixe « \texttt{0x} », vous pouvez écrire les nombres en héxadécimal :
\begin{galgas}
@bigint a = 0x123456789ABCDEF0_123456789abcdefG
message [a hexString] + "\n" # Affiche "123456789ABCDEF0_123456789ABCDEF"
\end{galgas}

Les lettres minuscules et majuscules sont utilisables.

\section{Constructeurs}

\subsectionConstructor{zero}{bigint}

Le constructeur \ggs!zero! renvoie un \ggs!@bigint! initialisé à zéro :
\begin{galgas}
@bigint a = .zero
message [a string] + "\n" # Affiche "0"
\end{galgas}


\subsectionConstructor{sint}{bigint}

Le constructeur \ggs!sint! permet de construire un \ggs!@bigint! à partir d'une valeur de type \ggs!@sint! :
\begin{galgas}
@bigint a = .sint {!-678S}
message [a string] + "\n" # Affiche "-678"
\end{galgas}


\subsectionConstructor{sint64}{bigint}

Le constructeur \ggs!sint64! permet de construire un \ggs!@bigint! à partir d'une valeur de type \ggs!@sint64! :
\begin{galgas}
@bigint a = .sint64 {!-678LS}
message [a string] + "\n" # Affiche "-678"
\end{galgas}




\subsectionConstructor{uint}{bigint}

Le constructeur \ggs!uint! permet de construire un \ggs!@bigint! à partir d'une valeur de type \ggs!@uint! :
\begin{galgas}
@bigint a = .uint {!678}
message [a string] + "\n" # Affiche "678"
\end{galgas}




\subsectionGetter{uint64}{bigint}

Le constructeur \ggs!uint64! permet de construire un \ggs!@bigint! à partir d'une valeur de type \ggs!@uint64! :
\begin{galgas}
@bigint a = .uint64 {!678L}
message [a string] + "\n" # Affiche "678"
\end{galgas}









\section{Conversions en chaîne de caractères}

\subsectionGetter{string}{bigint}

\begin{galgas}
getter string -> @bigint
\end{galgas}

Ce getter renvoie la valeur du receveur sous la forme d'une chaîne de caractères décimaux (de \ggs!0! à \ggs!9!). Si cette valeur est négative, le premier caractère est un signe \ggs!-!. Par exemple :

\begin{galgas}
@bigint a = -1234567890_1234567890_1234567890_G
message [a string] + "\n" # Affiche "-123456789012345678901234567890"
\end{galgas}





\subsectionGetter{hexString}{bigint}

\begin{galgas}
getter hexString -> @bigint
\end{galgas}

Ce getter renvoie la valeur du receveur sous la forme d'une chaîne de caractères héxadécimaux (\ggs!0! à \ggs!9!, \ggs!A! à \ggs!F!). Si cette valeur est négative, le premier caractère est un signe \ggs!-!. Il n'y a pas de préfixe « \texttt{0x} ». Exemple :

\begin{galgas}
@bigint a = -1234567890_1234567890_1234567890_G
message [a hexString] + "\n" # Affiche "-18EE90FF6C373E0EE4E3F0AD2"
\end{galgas}









\section{Arithmétique}


\subsection{Opérateurs \texttt{+} et \texttt{-} préfixés}

L'opérateur « \texttt{-} » préfixé effectue la négation de l'expression qui le suit. L'opérateur « \texttt{+} » préfixé n'a aucun effet, il retourne la valeur de l'expression.

\begin{galgas}
@bigint a = +1234567890_1234567890_1234567890_G
message [a string] + "\n" # Affiche "123456789012345678901234567890"
\end{galgas}









\subsectionGetter{abs}{bigint}

Le \emph{getter} \ggs!abs! retourne la valeur absolue.

\begin{galgas}
@bigint a = [-1234567890_1234567890_1234567890_G abs]
message [a string] + "\n" # Affiche "123456789012345678901234567890"
\end{galgas}






\subsection{Addition et soustraction}

Les opérateurs « \texttt{+} » et « \texttt{-} » infixés effectuent respectivement la somme et la différence de leurs opérandes. Comme la taille des \ggs!@bigint! est non limitée, aucun débordement n'a lieu.



\subsection{Multiplication}

L'opérateur « \texttt{*} » infixé effectue le produit de ses opérandes. Comme la taille des \ggs!@bigint! est non limitée, aucun débordement n'a lieu.




\subsection{Division}



La division d'un entier $n$ par un diviseur $d$ retourne un quotient $q$ et un reste $r$ :
\begin{equation*}
n = q * d + r\text{, avec 0 } \leqslant \mid r\mid < \mid d\mid
\end{equation*}

Trois opérations différentes sont possibles, suivant que l'on veuille obtenir un quotient arrondi :
\begin{itemize}
\item \emph{vers $+\infty$}, et $r$ a un signe opposé à $d$ ;
\item \emph{vers $-\infty$}, et $r$ a le même signe que $d$ ;
\item \emph{vers zéro}, et $r$ a le même signe que $n$.
\end{itemize}

En C, les opérateurs de division (« \texttt{/} »), et de calcul du reste (« \texttt{\%} ») utilisent un quotient arrondi \emph{vers zéro}. L'opérateur de décalage à droite (« \texttt{>{}>} ») de $n$ bits renvoie le quotient arrondi vers \emph{vers $-\infty$} de la division par $2^n$. En GALGAS, les opérateurs correspondants sur les types \ggs!@uint!, \ggs!@sint!, \ggs!@uint64! et \ggs!@sint64! sont conformes à ce comportement.

Le type \ggs!@bigint! obéit aux mêmes règles :
\begin{itemize}
\item les opérateurs \ggs!/! et \ggs!mod! infixés effectuent la division qui calcule le quotient arrondi \emph{vers zéro} ;
  \item l'opérateur \ggs!>>! infixé calcule le quotient arrondi \emph{vers $-\infty$} de la division par $2^n$ ;
\end{itemize}

De plus, trois méthodes sont disponibles, qui retournent quotient et reste de la division :
\begin{itemize}
  \item la méthode \ggs!divideBy! retourne le le quotient arrondi \emph{vers zéro} et le reste correspondant ;
  \item la méthode \ggs!floorDivideBy! retourne le le quotient arrondi \emph{vers $-\infty$} et le reste correspondant ;
  \item la méthode \ggs!ceilDivideBy! retourne le le quotient arrondi \emph{vers $+\infty$} et le reste correspondant.
\end{itemize}


\textbf{Opérateur « \texttt{/} » infixé.} Il effectue la division entière de l'expression de gauche par l'expression de droite et renvoie le quotient. Si l'expression de gauche est nulle, alors un message d'erreur est affiché et le résultat n'est pas construit.

\begin{galgas}
  message [(-7S) / 2S string] + "\n" # Affiche "-3"
  message [(-7G) / 2G string] + "\n" # Affiche "-3"
  message [(-7S) / (-2S) string] + "\n" # Affiche "3"
  message [(-7G) / (-2G) string] + "\n" # Affiche "3"
  message [7S / (-2S) string] + "\n" # Affiche "-3"
  message [7G / (-2G) string] + "\n" # Affiche "-3"
\end{galgas}



\textbf{Opérateur « \texttt{mod} » infixé.} Il renvoie le reste de la division entière de l'expression de gauche par l'expression de droite, telle que décrite au dessus. Si cette dernière est nulle, alors un message d'erreur est affiché et le résultat n'est pas construit.

\begin{galgas}
  message [9876543210G mod 1234567890G string] + "\n" # Affiche "90"
  message [(-9876543210G) mod 1234567890G string] + "\n" # Affiche "-90"
  message [(-9876543210G) mod (-1234567890G) string] + "\n"  # Affiche "-90"
  message [9876543210G mod (-1234567890G) string] + "\n"  # Affiche "90"
  message [2000S mod 183S string] + "\n" # Affiche "170"
  message [(-2000S) mod 183S string] + "\n" # Affiche "-170"
  message [(-2000S) mod (-183S) string] + "\n" # Affiche "-170"
  message [2000S mod (-183S) string] + "\n" # Affiche "170"
\end{galgas}




\textbf{Méthode \texttt{divideBy}.} Elle effectue la division dont le quotient arrondi \emph{vers zéro}, c'est-à-dire elle combine les opérateurs « \ggs!/! » et « \ggs!mod! » en une seule opération pour retourner quotient et reste.

\begin{galgas}
  @bigint quotient
  @bigint remainder
  [9876543210_9876543210G divideBy
    !1234567890G
    ?quotient:quotient
    ?remainder:remainder
  ]
  message [quotient string] + " " + remainder + "\n" # "80000000737 8280"
  [-9876543210_9876543210G divideBy
    !1234567890G
    ?quotient:quotient
    ?remainder:remainder
  ]
  message [quotient string] + " " + remainder + "\n" # "-80000000737 -8280"
  [-9876543210_9876543210G divideBy
    !-1234567890G
    ?quotient:quotient
    ?remainder:remainder
  ]
  message [quotient string] + " " + remainder + "\n" # "80000000737 -8280"
  [9876543210_9876543210G divideBy
    !-1234567890G
    ?quotient:quotient
    ?remainder:remainder
  ]
  message [quotient string] + " " + remainder + "\n" # "-80000000737 8280"
\end{galgas}





\textbf{Méthode \texttt{floorDivideBy}.} Elle effectue toujours la division dont le quotient arrondi \emph{vers $-\infty$}.

\begin{galgas}
  @bigint quotient
  @bigint remainder
  [9876543210_9876543210G floorDivideBy
    !1234567890G
    ?quotient:quotient
    ?remainder:remainder
  ]
  message [quotient string] + " " + remainder + "\n" # "80000000737 8280"
  [-9876543210_9876543210G floorDivideBy
    !1234567890G
    ?quotient:quotient
    ?remainder:remainder
  ]
  message [quotient string] + " " + remainder + "\n" # "-80000000738 1234559610"
  [-9876543210_9876543210G floorDivideBy
    !-1234567890G
    ?quotient:quotient
    ?remainder:remainder
  ]
  message [quotient string] + " " + remainder + "\n" # "80000000737 -8280"
  [9876543210_9876543210G floorDivideBy
    !-1234567890G
    ?quotient:quotient
    ?remainder:remainder
  ]
  message [quotient string] + " " + remainder + "\n" # "-80000000738 -1234559610"
\end{galgas}






\textbf{Méthode \texttt{ceilDivideBy}.} Elle effectue toujours la division dont le quotient arrondi \emph{vers $+\infty$}.

\begin{galgas}
  @bigint quotient
  @bigint remainder
  [9876543210_9876543210G ceilDivideBy
    !1234567890G
    ?quotient:quotient
    ?remainder:remainder
  ]
  message [quotient string] + " " + remainder + "\n" # "80000000738 -1234559610"
  [-9876543210_9876543210G ceilDivideBy
    !1234567890G
    ?quotient:quotient
    ?remainder:remainder
  ]
  message [quotient string] + " " + remainder + "\n" # "-80000000737 -8280"
  [-9876543210_9876543210G ceilDivideBy
    !-1234567890G
    ?quotient:quotient
    ?remainder:remainder
  ]
  message [quotient string] + " " + remainder + "\n" # "80000000738 1234559610"
  [9876543210_9876543210G ceilDivideBy
    !-1234567890G
    ?quotient:quotient
    ?remainder:remainder
  ]
  message [quotient string] + " " + remainder + "\n" # "-80000000737 8280"
\end{galgas}








\section{Décalages}

\subsection{Opérateur \texttt{<{}<}}

L'opérateur « \ggs!<<! » infixé effectue un décalage à gauche. L'expression de droite est toujours un \ggs!@uint!. Un décalage à gauche de $n$ bits est sémantiquement équivalent à une multiplication par $2^n$, que le nombre auquel s'applique le décalage soit signé ou non. C'est la sémantique des décalages à gauche des types \ggs!@sint! et \ggs!@sint64! :

\begin{galgas}
  message [0x1234567890G << 12 hexString] + "\n" # Affiche "1234567890000"
  message [(-0x1234567890G) << 12 hexString] + "\n" # Affiche "-1234567890000"
  message [2000S << 2 string] + "\n" # Affiche "8000"
  message [(-2000S) << 2 string] + "\n" # Affiche "-8000"
\end{galgas}

\subsection{Opérateur \texttt{>{}>}}

L'opérateur « \ggs!>>! » infixé effectue un décalage à droite. L'expression de droite est toujours un \ggs!@uint! :
\begin{galgas}
  message [0x1234567890G >> 12 hexString] + "\n" # Affiche "1234567"
  message [(-0x1234567890G) >> 12 hexString] + "\n" # Affiche "-1234567"
  message [2000S >> 2 string] + "\n" # Affiche "500"
  message [(-2000S) >> 2 string] + "\n" # Affiche "-500"
\end{galgas}

Un décalage à droite de $n$ bits d'un nombre posifif ou négatif est sémantiquement équivalent au quotient \emph{par défaut} d'une division par $2^n$, c'est-à-dire que le reste est toujours positif ou nul.

Quelques exemples de décalage à droite de nombres positifs :

\begin{galgas}
  message [9G >> 1 string] + "\n" # Affiche "4"
  message [9S >> 1 string] + "\n" # Affiche "4"
  message [7G >> 1 string] + "\n" # Affiche "3"
  message [7S >> 1 string] + "\n" # Affiche "3"
  message [3G >> 1 string] + "\n" # Affiche "1"
  message [3S >> 1 string] + "\n" # Affiche "1"
  message [1G >> 1 string] + "\n" # Affiche "0"
  message [1S >> 1 string] + "\n" # Affiche "0"
\end{galgas}


Et pour des nombres négatifs :

\begin{galgas}
  message [-9G >> 1 string] + "\n" # Affiche "-5"
  message [-9 S >> 1 string] + "\n" # Affiche "-5"
  message [-7G >> 1 string] + "\n" # Affiche "-4"
  message [-7S >> 1 string] + "\n" # Affiche "-4"
  message [-3G >> 1 string] + "\n" # Affiche "-2"
  message [-3S >> 1 string] + "\n" # Affiche "-2"
  message [-1G >> 1 string] + "\n" # Affiche "-1"
  message [-1S >> 1 string] + "\n" # Affiche "-1"
\end{galgas}

Dans tous les cas, la sémantique du décalage à droite du type \ggs!@bigint! est la même que celles des types \ggs!@sint! et \ggs!@sint64!.










\section{Opérations logiques}

Le type \ggs!@bigint! implémente les opérations logiques \ggs!&!, \ggs!|!, \ggs!^| et \ggs!~!.

\subsection{Opérateur \texttt{\&} infixé}


\subsection{Opérateur \texttt{|} infixé}


\subsection{Opérateur $\wedge$ infixé}


\subsection{Opérateur $\sim$ préfixé}


