%!TEX encoding = UTF-8 Unicode
%!TEX root = ../galgas-book.tex

\chapter{Présentation du système de types}






\section{Opérations définies pour tous les types}

Tout type implémente implicitement :
\begin{itemize}
  \item l'opérateur \galgas{==} ;
  \item l'opérateur \galgas{\!=} ;
  \item le \emph{reader} \galgas{description} ;
  \item le \emph{reader} \galgas{dynamicType} ;
  \item le \emph{reader} \galgas{object}.
\end{itemize}

La plupart des types implémentent le constructeur par défaut \galgas{default} (voir \refSubsectionPage{constructeurParDefaut}). 


\subsection{L'opérateur \texttt{==}}

\begin{galgascode}
operator @T == -> @bool ;
\end{galgascode}

Cet opérateur permet de tester l'identité entre de deux objets de même type. 

\subsection{L'opérateur \texttt{!=}}

\begin{galgascode}
operator @T != -> @bool ;
\end{galgascode}

Cet opérateur permet de tester la non identité entre de deux objets de même type. Il renvoie le complément logique du résultat de l'application de l'opérateur \galgas{==}.





\subsection{Le reader \texttt{description}}

\begin{galgascode}
reader @T description -> @string ;
\end{galgascode}

Le \emph{reader} \galgas{description} retourne une description textuelle du receveur, la même que celle affichée par l'instruction \galgas{log} (\refSectionPage{instructionLog}).



\subsection{Le reader \texttt{dynamicType}}

\begin{galgascode}
reader @T dynamicType -> @type ;
\end{galgascode}

Le \emph{reader} \galgas{dynamicType} retourne un objet de type \galgas{@type}, dont la valeur représente le type dynamique du receveur (voir aussi la définition du \refTypePredefini{type}).

Pour tous les types sauf les classes, leurs instances sont du même type que le type statique :

\begin{galgascode}
@uint n := 2 ;
@type t := [n dynamicType] ;
log t ; # Affiche @uint
\end{galgascode}

Pour les instances de classes, le jeu des affectations polymorphiques peut entraîner que le type dynamique soit une classe héritière du type statique.

Par exemple, en déclarant :
\begin{galgascode}
class @A { }
class @B extends @A { }
\end{galgascode}

Et avec la séquence d'instructions suivante :
\begin{galgascode}
@B b [new] ;
@type t := [b dynamicType] ;
log t ; # Affiche @B, type statique de b : @B
@A a := b ; # Affectation polymorphique
t := [a dynamicType] ;
log t ; # Affiche @B, type statique de a : @A
\end{galgascode}





\subsection{Le reader \texttt{object}}

\begin{galgascode}
reader @T object -> @object ;
\end{galgascode}


Le \emph{reader} \galgas{object} retourne un objet de type \galgas{@object}. Une variable de \refTypePredefini{object} peut encapsuler tout type de valeur.

%====== Readers ======
%===== description =====
%
%''**reader** description %%->%% @string ;''\\
%
%This reader returns a string representation of the receiver's value.
%
%===== dynamicType =====
%
%|Available on GALGAS 1.9.5 and later|
%
%''**reader** dynamicType %%->%% @type ;''\\
%
%This reader returns the dynamic type of the receiver's value.
%===== object =====
%
%|Available on GALGAS 1.9.5 and later|
%
%''**reader** object %%->%% @object ;''\\
%
%This reader returns an ''@object'' instance that embeds the receiver's value.











