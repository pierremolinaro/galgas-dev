%!TEX encoding = UTF-8 Unicode
%!TEX root = ../galgas-book.tex

\chapter{Présentation du système de types}

GALGAS définit :
\begin{itemize}
\item des types de base, définis en dur dans le langage (\refSectionPage{typeBase}) ;
\item des constructions permettant de construire de nouveaux types (\refSectionPage{typeConstruits}).
\end{itemize}




\sectionLabel{Types de base}{typeBase}

Les types de base sont :
\begin{itemize}
\item \refTypePredefini{application}, accès aux informations relatives à l'application ;
\item \refTypePredefini{bigint}, entiers de taille illimitée ;
\item \refTypePredefini{binaryset}, fonctions booléennes, implementées par des \emph{Binary Decision Diagrams} ;
\item \refTypePredefini{bool}, les booléens ;
\item \refTypePredefini{char}, les caractères Unicode ;
\item \refTypePredefini{data}, les séquences d'octets ;
\item \refTypePredefini{double}, les nombres flottants correspondant au type \texttt{double} du C ;
\item \refTypePredefini{filewrapper}, dont les objets permettent d'explorer les \emph{filewrappers} ;
\item \refTypePredefini{location}, objets dont la valeur désigne un texte source et un indice dans ce texte source ;
\item \refTypePredefini{object}, dont une instance peut encapsuler toute valeur ;
%\item \refTypePredefini{range}, intervalle d'entiers 32 bits non signés ;
\item \refTypePredefini{sint}, entiers 32 bits signés ;
\item \refTypePredefini{sint64}, entiers 64 bits signés ;
\item \refTypePredefini{string}, chaînes de caractères Unicode ;
\item \refTypePredefini{stringset}, ensembles de chaînes de caractères Unicode ;
\item \refTypePredefini{timer} ;
\item \refTypePredefini{type}, dont une instance représente un type ;
\item \refTypePredefini{uint}, entiers 32 bits non signés ;
\item \refTypePredefini{uint64}, entiers 64 bits non signés.
\end{itemize}





\sectionLabel{Constructions de nouveaux types}{typeConstruits}

Les nouveaux types qui peuvent être construits sont :
\begin{itemize}
  \item des \emph{types de listes}, \refChapterPage{typeList} ;
  \item des \emph{types de listes ordonnées}, \refChapterPage{typeSortedList} ;
  \item des \emph{types de tableaux}, \refChapterPage{typeArray} ;
  \item des \emph{types de classes}, \refChapterPage{typeClass} ;
  \item des \emph{types énumérés}, \refChapterPage{typeEnum} ;
  \item des \emph{types de graphes}, \refChapterPage{typeGraph} ;
  \item des \emph{types de dictionnaires}, \refChapterPage{typeMap} ;
  \item des \emph{types de structures}, \refChapterPage{typeStructure} ;
  \item des \emph{types externes}, \refChapterPage{typeExtern}.
\end{itemize}









\section{Types prédéfinis}

Les types prédéfinis sont :
\begin{itemize}
  \item les types de base (\refSectionPage{typeBase}) ;
  \item les types de structure suivants :
  \begin{itemize}
    \item \refTypePredefini{lbool} ;
    \item \refTypePredefini{lbigint} ;
    \item \refTypePredefini{lchar} ;
    \item \refTypePredefini{ldouble} ;
    \item \refTypePredefini{lsint} ;
    \item \refTypePredefini{lsint64} ;
    \item \refTypePredefini{lstring} ;
    \item \refTypePredefini{luint} ;
    \item \refTypePredefini{luint64} ;
    \item \refTypePredefini{range} ;
  \end{itemize}
  \item les types de listes suivants :
  \begin{itemize}
    \item \refTypePredefini{2stringlist} ;
    \item \refTypePredefini{2lstringlist} ;
    \item \refTypePredefini{bigintlist} ;
    \item \refTypePredefini{functionlist} ;
    \item \refTypePredefini{lbigintlist} ;
    \item \refTypePredefini{lstringlist} ;
    \item \refTypePredefini{luintlist} ;
    \item \refTypePredefini{objectlist} ;
    \item \refTypePredefini{stringlist} ;
    \item \refTypePredefini{typelist} ;
    \item \refTypePredefini{uintlist} ;
    \item \refTypePredefini{uint64list}.
  \end{itemize}
\end{itemize}








\section{Opérations définies pour tous les types}

Tout type implémente implicitement :
\begin{itemize}
  \item l'opérateur \ggs+==+ ;
  \item l'opérateur \ggs+!=+ ;
  \item le \emph{getter} \ggs+description+ ;
  \item le \emph{getter} \ggs+dynamicType+ ;
  \item le \emph{getter} \ggs+object+.
\end{itemize}

La plupart des types implémentent le constructeur par défaut \ggs+default+ (voir \refSubsectionPage{constructeurParDefaut}). 


\subsection{L'opérateur \texttt{==}}

\begin{galgas}
func == ?@T ?@T -> @bool
\end{galgas}

Cet opérateur permet de tester l'identité entre de deux objets de même type. 

\subsection{L'opérateur \texttt{!=}}

\begin{galgas}
func != ?@T ?@T -> @bool
\end{galgas}

Cet opérateur permet de tester la non identité entre de deux objets de même type. Il renvoie le complément logique du résultat de l'application de l'opérateur \ggs+==+.





\subsection{Le getter \texttt{description}}

\begin{galgas}
getter @T description -> @string
\end{galgas}

Le \emph{getter} \ggs+description+ retourne une description textuelle du receveur, la même que celle affichée par l'instruction \ggs+log+ (\refSectionPage{instructionLog}).



\subsection{Le getter \texttt{dynamicType}}

\begin{galgas}
getter @T dynamicType -> @type
\end{galgas}

Le \emph{getter} \ggs+dynamicType+ retourne un objet de type \ggs+@type+, dont la valeur représente le type dynamique du receveur (voir aussi la définition du type \refTypePredefini{type}).

Pour tous les types sauf les classes, leurs instances sont du même type que le type statique :

\begin{galgas}
@uint n = 2
@type t = [n dynamicType]
log t # Affiche @uint
\end{galgas}

Pour les instances de classes, le jeu des affectations polymorphiques peut entraîner que le type dynamique soit une classe héritière du type statique.

Par exemple, en déclarant :
\begin{galgas}
class @A { }
class @B : @A { }
\end{galgas}

Et avec la séquence d'instructions suivante :
\begin{galgas}
@B b = .new
@type t = [b dynamicType]
log t # Affiche @B, type statique de b : @B
@A a = b # Affectation polymorphique
t = [a dynamicType]
log t # Affiche @B, type statique de a : @A
\end{galgas}





\subsection{Le getter \texttt{object}}

\begin{galgas}
getter @T object -> @object
\end{galgas}


Le \emph{getter} \ggs+object+ retourne un objet de type \ggs+@object+. Une variable de type \refTypePredefini{object} peut encapsuler tout type de valeur.

