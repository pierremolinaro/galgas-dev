%!TEX encoding = UTF-8 Unicode
%!TEX root = ../galgas-book.tex

\chapitreTypePredefiniLabelIndex{uint}

An \ggs+@uint+ object value is a 32-bit unsigned integer value. You can initialize an \ggs+@uint+ object from an unsigned integer constant:\\

\begin{galgas}
@uint myUnsignedInteger = 123_456 ;
\end{galgas}

Note that a 32-bit unsigned integer constant is characterized by no suffix.

\section{Constructors}

\subsectionConstructor{errorCount}{uint}

\begin{galgas}
constructor errorCount -> @uint
\end{galgas}


Returns an \ggs+@uint+ object that contains the number of errors. The returned value is the cumulative count of errors from the beginning of execution.

\textbf{Exemple :}
\begin{galgas}
@uint x = [@uint errorCount] ;
\end{galgas}




\subsectionConstructor{max}{uint}

\begin{galgas}
constructor max -> @uint
\end{galgas}

Returns an \ggs+@uint+ object that the maximum value of the 32-bit unsigned range ($2^{32}-1$).






\subsectionConstructor{random}{uint}

\begin{galgas}
constructor random -> @uint
\end{galgas}

Retourne une valeur aléatoire de type \ggs+@uint+. La procédure de type \refStaticProcPage{uint}{setRandomSeed} permet d'en fixer la valeur initiale.

\begin{galgas}
  let v = @uint.random
\end{galgas}


{\bf Note. } Sur Unix, la valeur renvoyée est la valeur renvoyée par l'appel de la fonction \texttt{random} de la librairie \texttt{libc}. Sur Windows, c'est la fonction \texttt{rand} qui est appelée.


\subsectionConstructor{valueWithMask}{uint}

\begin{galgas}
constructor valueWithMask ?@uint inLowerIndex ?@uint inUpperIndex -> @uint
\end{galgas}


Returns an \ggs+@uint+ object with bits from \emph{inLowerIndex} to \emph{inUpperIndex} equal to 1.

A run-time error is raised if \emph{inLowerIndex $>$ inUpperIndex} or if \emph{inUpperIndex $>$ 31}.



\textbf{Exemple :}
\begin{galgas}
@uint x = [@uint valueWithMask !2 !4] ; # x is equal to 28 (0b1_1100)
\end{galgas}




\subsectionConstructor{warningCount}{uint}

\begin{galgas}
constructor warningCount -> @uint
\end{galgas}


Returns an \ggs+@uint+ object that contains the number of warnings. The returned value is the cumulative count of warnings from the beginning of execution.





\section{Procédure de type}


\subsectionStaticProc{setRandomSeed}{uint}


\begin{galgasbox}
proc @uint setRandomSeed ?@uint inSeed
\end{galgasbox}

Affecte la valeur initiale utilisée par le générateur de nombres aléatoires (voir le \refConstructorPage{uint}{random}) Par exemple~:

\begin{galgas}
  [@uint setRandomSeed !0]
\end{galgas}






\section{Getters}

\subsectionGetter{alphaString}{uint}

Ce \emph{getter} permet de convertir un \ggs!@uint! en une chaîne de caractères, telle que l'ordre des entiers est conservé sur la chaîne obtenue.

La chaîne obtenue comporte exactement 7 lettres minuscules. C'est en fait une conversion en base 26, la lettre \ggs=a= ayant la valeur $0$, et la lettre \ggs=z= la valeur $25$.


\begin{galgas}
  message [0 alphaString] + "\n"         # aaaaaaa
  message [12_345 alphaString] + "\n"    # aaaasgv
  message [@uint.max alphaString] + "\n" # nxmrlxv
\end{galgas}



\subsectionGetter{bigint}{uint}

Ce \emph{getter} permet de convertir un \ggs!@uint! en \ggs!@bigint!. Comme la plage des valeurs des \ggs!bigint! n'est limitée que par la mémoire disponible, il n'échoue jamais.

\begin{galgas}
  message [[1234 bigint] string] + "\n" # 1234
\end{galgas}


\subsectionGetter{double}{uint}

\begin{galgas}
getter double -> @double
\end{galgas}

Returns the receiver's value converted in a \ggs+@double+ object. As a 32-bit integer value can always be converted in a \ggs+@double+ value, this getter never fails.



\subsectionGetter{hexString}{uint}

\begin{galgas}
getter hexString -> @string
\end{galgas}

Returns the an hexadecimal string representation of the receiver value, prefixed by the string \texttt{0x}. For getting an hexadecimal representation string without any prefix, see \refGetterPage{uint}{xString}.



\subsectionGetter{hexStringSeparatedBy}{uint}

\begin{galgas}
getter hexStringSeparatedBy ?@char inSeparator ?@uint inGroup -> @string
\end{galgas}

Returns the an hexadecimal string representation of the receiver value, prefixed by the string \texttt{0x}. Groups of \ggs=inGroup= digits are separated by the \ggs=inSeparator= character.

If \ggs=inGroup= is equal to zero, a run-time error is raised.

For example:
\begin{galgas}
let s = [0x12345678 hexStringSeparatedBy !'_' !2] # 0x12_34_56_78
\end{galgas}



\subsectionGetter{isInRange}{uint}

\begin{galgas}
getter isInRange ?@range inRange -> @bool
\end{galgas}

{Returns an \ggs+@bool+ value indicating whether the receiver'value belongs to \ggs+inRange+ range : for a receiver's value equal to $v$ and a range of length $length$ starting at $start$, it returns \ggs+true+ if $((v \geqslant start)~and~(v<(start+length)))$, and \ggs+false+ otherwise.



\subsectionGetter{isUnicodeValueAssigned}{uint}

\begin{galgas}
getter isUnicodeValueAssigned -> @bool
\end{galgas}

Returns an \ggs+@bool+ value indicating whether the receiver'value represents an assigned Unicode character. It returns \ggs+true+ if the receiver value represents an assigned Unicode character, \ggs+false+ and otherwise.

\textbf{Exemple :}
\begin{galgas}
[0xFFFF isUnicodeValueAssigned] # is false, as \uFFFF is not assigned.
[0x41 isUnicodeValueAssigned] # is true, as \u0041 is assigned (LATIN CAPITAL LETTER A).
\end{galgas}



\subsectionGetter{lsbIndex}{uint}

\begin{galgas}
getter lsbIndex -> @uint
\end{galgas}

Returns an \ggs+@uint+ value of the index of the most significant bit of the receiver value. It raises a run-time error if the receiver value is zero.

\textbf{Exemple :}
\begin{galgas}
@uint value = 192 ; # 192 is ...011000000 in binary
@uint x = [value lsbIndex] ; # x is equal to 7
\end{galgas}

The most significant bit of 192 is the 7th bit.




\subsectionGetter{significantBitCount}{uint}

\begin{galgas}
getter significantBitCount -> @uint
\end{galgas}

Returns the number of bits needed to express the receiver value. If the receiver value is zero, it returns 0 ; otherwise, it returns the most significant bit index plus one.

\textbf{Exemple :}
\begin{galgas}
@uint value = 145 ; # 145 is 10010001 in binary
@uint x = [value significantBitCount] ; # x is equal to 8
\end{galgas}






\subsectionGetter{sint}{uint}

\begin{galgas}
getter sint -> @sint
\end{galgas}

Returns the receiver's value in an \refTypePredefini{sint} (32-bit signed integer) object. An error is raised is receiver's value is greater than $2^{31}-1$.

This getter is the only way to convert an \refTypePredefini{uint} object into an \refTypePredefini{sint} object.




\subsectionGetter{sint64}{uint}

\begin{galgas}
getter sint64 -> @sint64
\end{galgas}

Returns the receiver's value in an \refTypePredefini{sint64} (64-bit signed integer) object. As a 32-bit unsigned value can always be converted in a 64-bit signed value, this getter never fails.

This getter is the only way to convert an \refTypePredefini{uint} object into an \refTypePredefini{sint64} object.


\subsectionGetter{string}{uint}

\begin{galgas}
getter string -> @string
\end{galgas}

Returns a decimal string representation of the receiver's value. For an hexadecimal string representation of the receiver's value, see \refGetterPage{uint}{hexString} and \refGetterPage{uint}{xString}.




\subsectionGetter{uint64}{uint}

\begin{galgas}
getter uint64 -> @uint64
\end{galgas}

Returns the receiver's value in an \refTypePredefini{uint64} (64-bit unsigned integer) object. As a 32-bit unsigned value can always be converted in a 64-bit unsigned value, this getter never fails.

This getter is the only way to convert an \refTypePredefini{uint} object into an \refTypePredefini{uint64} object.




\subsectionGetter{xString}{uint}

\begin{galgas}
getter xString -> @string
\end{galgas}

Returns an hexadecimal string representation of the receiver's value (without any prefix). For an decimal string representation of the receiver's value, see the \refGetterPage{uint}{hexString}; for a decimal string representation of the receiver's value, see the \refGetterPage{uint}{string}.







\section{Arithmétique}

\subsection{Opérateurs infixés}

Le type \ggs+@uint+ accepte les opérateurs arithmétiques infixés suivants :
\begin{itemize}
  \item \ggs!+!, addition, une erreur d'exécution est déclenchée en cas de débordement ;
  \item \ggs!-!, soustraction, une erreur d'exécution est déclenchée en cas de débordement ;
  \item \ggs!*!, multiplication, une erreur d'exécution est déclenchée en cas de débordement ;
  \item \ggs!/!, division, une erreur d'exécution est déclenchée si le diviseur est nul ;
  \item \ggs!mod!, calcul du reste, une erreur d'exécution est déclenchée si le diviseur est nul ;
  \item \ggs!&+!, addition, le résultat étant silencieusement tronqué sur 32 bits ;
  \item \ggs!&-!, soustraction, le résultat étant silencieusement tronqué sur 32 bits ;
  \item \ggs!&*!, multiplication, le résultat étant silencieusement tronqué sur 32 bits ;
  \item \ggs!&/!, division, qui retourne zéro si le diviseur est nul.
\end{itemize}

Ces opérateurs exigent que les deux opérandes soient des objets du même type \ggs+@uint+. 

\subsection{Opérateur préfixé}
Le type \ggs+@uint+ accepte un opérateur arithmétique préfixé :
\begin{itemize}
  \item \ggs!+!, qui retourne simplement la valeur de l'opérande.
\end{itemize}

\subsectionLabel{Instructions}{instructionsUINT}

Le type \ggs+@uint+ accepte les deux instructions arithmétiques suivantes :
\begin{itemize}
  \item \ggs!+=!, addition, une erreur d'exécution est déclenchée en cas de débordement ;
  \item \ggs!-=!, soustraction, une erreur d'exécution est déclenchée en cas de débordement ;
  \item \ggs!*=!, multiplication, une erreur d'exécution est déclenchée en cas de débordement ;
  \item \ggs!/=!, division, une erreur d'exécution est déclenchée en cas division par zéro ;
  \item \ggs!++!, incrémentation, une erreur d'exécution est déclenchée en cas de débordement ;
  \item \ggs!--!, décrémentation, une erreur d'exécution est déclenchée en cas de débordement ;
  \item \ggs!&++!, incrémentation, le résultat étant silencieusement tronqué sur 32 bits ;
  \item \ggs!&--!, décrémentation, le résultat étant silencieusement tronqué sur 32 bits.
\end{itemize}

\ggs!x+=y! est équivalent à \ggs!x=x+y! ; \ggs!x-=y! est équivalent à \ggs!x=x-y!.
La variable cible \ggs!x!, comme l'expression source \ggs!y! doivent être du même type \ggs+@uint+. 

Incrémentation et décrémentation sont des instructions, et ne peuvent pas apparaître des expressions.
\begin{galgas}
@uint n = ... ; n ++ # Incrémentation
\end{galgas}

\begin{galgas}
@uint n = ... ; n -- # Décrémentation
\end{galgas}




\section{Shift Operators}


The \ggs+@uint+ type supports right and left shift operators:\newline

\begin{tabular}{|c|c|}
\hline
$<<$ & Left shift \\
\hline
$>>$ & Right shift \\
\hline
\end{tabular}

Theses operators require both arguments to be \ggs+@uint+ objects.\newline

Note the right shift inserts always a zero bit in the most significant bit location (it is a logical right shift).\newline

The actual amount of the shift is the value of the right-hand operand masked by 31, i.e. the shift distance is always between 0 and 31.




\section{Logical Operators}

The \ggs+@uint+ type supports the three bit-wise logical operators:\newline

\begin{tabular}{|c|c|}
\hline
$\&$ & Bit-wise and \\
\hline
\textbar & Bit-wise or \\
\hline
\^\  & Bit-wise exclusive or \\
\hline
\end{tabular}

Theses operators require both arguments to be \ggs+@uint+ objects.\newline


The \ggs+@uint+ type supports the bit-wise logical unary operator:\newline

\begin{tabular}{|c|c|}
\hline
$\sim$ & Bit-wise complementation \\
\hline
\end{tabular}

This operator returns an \ggs+@uint+ object.







\section{Comparison Operators}

The \ggs+@uint+ type supports the six comparison operators:\newline

\begin{tabular}{|c|c|}
\hline
$=$ & Equality \\
\hline
$!=$ & Non Equality \\
\hline
$<$  & Strict Lower Than \\
\hline
$<=$  & Lower or Equal \\
\hline
$>$  & Strict Greater Than \\
\hline
$>=$  & Greater or Equal \\
\hline
\end{tabular}

\vspace{2mm}
Theses operators require both arguments to be \ggs+@uint+ objects, and return a \ggs+@bool+ object.


