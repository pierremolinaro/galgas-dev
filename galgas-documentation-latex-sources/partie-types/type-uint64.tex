%!TEX encoding = UTF-8 Unicode
%!TEX root = ../galgas-book.tex

\chapitreTypePredefiniLabelIndex{uint64}

An \galgas{@uint64} object value is a 64-bit unsigned integer value. You can initialize an \galgas{@uint64} object from a 64-bit unsigned integer constant:\\

\texttt{@uint64 myUnsignedInteger := 123\_456L ;}\newline

Note the 'L' suffix is required for a 64-bit unsigned integer constant.

\section{Constructeurs}

\constructeurSansArgument{max}
{uint64}
{1.3.0}
{uint64}
{Returns an \galgas{@uint64} object that the maximum value of the 64-bit unsigned range.}
{The returned value is $2^{64}-1$.}


\constructeurUnArgument{uint64BaseValueWithCompressedBitString}
{uint64}
{1.6.4}
{uint64}
{@string inBitString}
{Returns an \galgas{@uint64} object computed from a string containing '0', '1' or 'X' characters, replacing all occurrences of 'X' by '0'.}
{the inBitString argument should contain only '0', '1' or 'X' characters. A run time exception is raised if an other character appears.

This constructor considers the \emph{inBitString} argument value as a binary encoding of an integer value. First, it internally replaces all 'X's by '0's, and then converts the resulting string into an integer value that is the one returned by this constructor.

Note that the first character of the \emph{inBitString} argument value corresponds to the most significant bit of the converted value.}


\textbf{Exemple :}
\begin{galgascode}
@uint64 v [uint64BaseValueWithCompressedBitString !"01XX10"] ;
log v ; # Displays <@uint64:18> ;
\end{galgascode}





\constructeurUnArgument{uint64MaskWithCompressedBitString}
{uint64}
{1.6.4}
{uint64}
{@string inBitString}
{Returns an \galgas{@uint64} object computed from a string containing '0', '1' or 'X' characters, replacing all occurrences of '0' by '1' and all occurrences of 'X' by '0'.}
{the \emph{inBitString} argument should contain only '0', '1' and 'X' characters. A run time exception is raised if an other character appears.

This constructor considers the \emph{inBitString} argument value as a binary encoding of an integer value. First, it internally replaces all '0's by '1's and all 'X's by '0's, and then converts the resulting string into an integer value that is the one returned by this constructor.

Note that the first '0' or '1' character of the \emph{inBitString} argument value corresponds to the most significant Bit of the converted value.}

\textbf{Exemple :}
\begin{galgascode}
@uint64 v [uint64MaskWithCompressedBitString !"01XX10"] ;
log v ; \# Displays <@uint64:51> ;
\end{galgascode}



\constructeurUnArgument{uint64WithBitString}
{uint64}
{1.6.4}
{uint64}
{@string inBitString}
{Returns an \galgas{@uint64} object computed from a string containing '0' or '1' characters.}
{the \emph{inBitString} argument should contain only '0' and '1' characters. A run time exception is raised if an other character appears.

This constructor considers the \emph{inBitString} argument value as a binary encoding of an integer value. It returns an \galgas{@uint64} object containing the converted value.

Note that the first '1' character of the \emph{inBitString} argument value corresponds to the most significant bit of the converted value.}

\textbf{Exemple :}
\begin{galgascode}
@uint64 v [uint64WithBitString !"0101"]] ;
log v ; # Displays <@uint64:5> ;
\end{galgascode}


\section{Readers}

\readerSansArgument{double}
{uint64}
{1.9.8}
{double}
{Returns the receiver's value converted in a \galgas{@double} object.}
{as a 64-bit integer value can always be converted in a \galgas{@double} value, this reader never fails.}



\readerSansArgument{hexString}
{uint64}
{1.5.2}
{string}
{Returns the an hexadecimal string representation of the receiver value, prefixed by the string "0x".}
{for getting an hexadecimal representation string without "0x" prefix, see \refReaderPage{uint64}{xString}.}





\readerSansArgument{sint}
{uint64}
{1.6.12}
{sint}
{Returns the receiver's value in an \refTypePredefini{sint} (32-bit signed integer) object.}
{an error is raised is receiver's value is greater than $2^{31}-1$.}

This reader is the only way to convert an \refTypePredefini{uint64} object into an \refTypePredefini{sint} object.




\readerSansArgument{sint64}
{uint64}
{1.6.12}
{sint64}
{Returns the receiver's value in an \refTypePredefini{sint64} (64-bit signed integer) object.}
{an error is raised is receiver's value is greater than $2^{63}-1$.}

This reader is the only way to convert an \refTypePredefini{uint64} object into an \refTypePredefini{sint64} object.


\readerSansArgument{string}
{uint64}
{1.6.12}
{string}
{Returns a decimal string representation of the receiver's value.}
{for an hexadecimal string representation of the receiver's value, see \refReaderPage{uint64}{hexString} and \refReaderPage{uint64}{xString}.}



\readerSansArgument{uint}
{uint64}
{1.6.12}
{uint}
{Returns the receiver's value in an \refTypePredefini{uint} (32-bit unsigned integer) object.}
{an error is raised is receiver's value is greater than $2^{32}-1$.}

This reader is the only way to convert an \refTypePredefini{uint64} object into an \refTypePredefini{uint} object.


\readerDeuxArguments{uintSlice}
{uint64}
{1.6.0}
{uint}
{@uint inStartBit}
{@uint inBitCount}
{Returns an \refTypePredefini{uint} value, extracted from a bit slice of the receiver's value.}
{the receiver's value is right shifted by \emph{inStartBit}, and the resulted value is and'ed with a mask equal to $2^{inBitCount}-1$.}


\textbf{Exemple :}
\begin{galgascode}
@uint64 v := 0x1234_5678_9ABC_DEF0L ;
@uint result := [v uintSlice !4 !5] ; # The result value is 0x8_9ABC
\end{galgascode}




%\defReaderSansArgument{xString}{uint64}
%{1.9.10}
%{string}
%{Returns an hexadecimal string representation of the receiver's value (without any prefix).}
%{for an decimal string representation of the receiver's value, see the \refReaderPage{uint64}{hexString}; for a decimal string representation of the receiver's value, see the \refReaderPage{uint64}{string}.}
%{}{}

\readerSansArgument{xString}
{uint64}
{1.9.10}
{string}
{Returns an hexadecimal string representation of the receiver's value (without any prefix).}
{for an decimal string representation of the receiver's value, see the \refReaderPage{uint64}{hexString}; for a decimal string representation of the receiver's value, see the \refReaderPage{uint64}{string}.}






\section{Incrementation and decrementation}

The \refTypePredefini{uint64} supports incrementation and decrementation instructions.

\texttt{@uint64 n := ... ; n ++ ; \# Incrementation}

\texttt{@uint64 p := ... ; p -- ; \# Decrementation}\newline

The incrementation instruction raises an error if receiver's value is equal to $2^{64}-1$.\newline

The incrementation instruction raises an error if receiver's value is equal to 0.\newline

Note that incrementation and decrementation are not available within an expression.




\section{Arithmetic Operators}

The \galgas{@uint64} type supports the five arithmetic diadic operators:\newline

\begin{tabular}{|c|c|}
\hline
$+$ & Addition \\
\hline
$-$ & Substraction \\
\hline
$*$ & Multiplication \\
\hline
$/$ & Division \\
\hline
\galgas{mod} & Modulo \\
\hline
\end{tabular}

Theses operators require both arguments to be \galgas{@uint64} objects.\newline

A run-time error is raised if the operation leads to a 64-bit unsigned overflow.

The \galgas{@uint64} type supports the following arithmetic unary operator:\newline

\begin{tabular}{|c|c|}
\hline
$+$ & No operation \\
\hline
\end{tabular}

This operator returns the receiver's value (an  \galgas{@uint64} object).




\section{Shift Operators}


The \galgas{@uint} type supports right and left shift operators:\newline

\begin{tabular}{|c|c|}
\hline
$<<$ & Left shift \\
\hline
$>>$ & Right shift \\
\hline
\end{tabular}

Theses operators require the left argument to be \galgas{@uint64} object, and  the right argument to be \galgas{@uint} object.\newline

Note the right shift inserts always a zero bit in the most significant bit location (it is a logical right shift).\newline

The actual amount of the shift is the value of the right-hand operand masked by 63, i.e. the shift distance is always between 0 and 63.




\section{Logical Operators}

The \galgas{@uint64} type supports the three bit-wise logical diadic operators:

\begin{tabular}{|c|c|}
\hline
$\&$ & Bit-wise and \\
\hline
\textbar & Bit-wise or \\
\hline
\^\  & Bit-wise exclusive or \\
\hline
\end{tabular}

Theses operators require both arguments to be \galgas{@uint64} objects.\newline


The \galgas{@uint64} type supports the bit-wise logical unary operator:

\begin{tabular}{|c|c|}
  \hline
  $\sim$ & Bit-wise complementation \\
  \hline
\end{tabular}

This operator returns an \galgas{@uint64} object.




\section{Comparison Operators}

The \galgas{@uint64} type supports the six comparison operators:

\begin{tabular}{|c|c|}
\hline
$=$ & Equality \\
\hline
$!=$ & Non Equality \\
\hline
$<$  & Strict Lower Than \\
\hline
$<=$  & Lower or Equal \\
\hline
$>$  & Strict Greater Than \\
\hline
$>=$  & Greater or Equal \\
\hline
\end{tabular}

Theses operators require both arguments to be \galgas{@uint64} objects, and return a \galgas{@bool} object.
