%!TEX encoding = UTF-8 Unicode
%!TEX root = ../galgas-book.tex

%--------------------------------------------------------------
\chapterLabel{Le type \texttt{extern}}{typeExtern}

%-------------------------------------------------------------

Un type \ggs+extern+ est déclaré et spécifié en GALGAS, et implémenté par une classe C++. Ceci permet de définir des types qui seraient difficilement exprimables en GALGAS.

On va voir sur un exemple comment déclarer et implémenter :
\begin{itemize}
  \item un type externe minimum ;
  \item un constructeur ;
  \item un \emph{setter} ;
  \item une \emph{méthode} ;
  \item un \emph{getter} ;
  \item une \emph{méthode} de classe.
\end{itemize}

L'exemple consiste à implémenter le type \ggs+@complex+ qui représente les nombres complexes.

\section{Type externe minimum}

L'implémentation minimum ne sera pas opérationnelle, car elle ne comprendra pas de constructeur : on ne pourra donc pas instancier d'objet du type \ggs+@complex+. L'ajout de constructeur sera présenté à la section suivante. De même, cette implémentation minimum ne définira ni \emph{setter}, ni \emph{méthode}, ni \emph{getter}.


\subsection{Déclaration en GALGAS}

La description minimum est la suivante :
\begin{galgas}
extern @complex {
  "// No Predeclaration\n"
}{
  "  private : bool mIsValid ;\n"
  "  private : double mReal ;\n"
  "  private : double mImaginary ;\n"
}{
}
\end{galgas}

Cette description est divisée en trois parties, délimitées par les accolades \ggs+{+ et \ggs+}+.

\textbf{Première partie.} Elle cite une séquence de chaînes de caractères, qui seront écrites telles quelles dans le fichier d'en-tête C++ engendré, juste avant la déclaration de la classe C++ ; on peut y placer là des pré-déclarations de classe, des inclusions de fichier, … Pour le type \ggs+@complex+, aucune pré-déclaration n'est nécessaire, aussi on place un simple commentaire C++, de façon à le localiser dans le fichier d'en-tête C++ engendré.

\subsection{Implémentation en C++}






\section{Constructeur}


\section{Setter}


\section{Méthode}


\section{Getter}


\section{Méthode de classe}

