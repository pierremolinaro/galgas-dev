  %!TEX encoding = UTF-8 Unicode
%!TEX root = ../galgas-book.tex

%--------------------------------------------------------------
\chapterLabel{Le type \texttt{boolset}}{typeBoolset}
%-------------------------------------------------------------

\tableDesMatieresLocaleDeProfondeurRelative{1}



Le mot-clé \ggsq!boolset! permet de définir des types d'ensembles d'indicateurs booléens. Un tel objet a une sémantique de valeur.

La syntaxe de définition d'un type ensemble d'indicateurs booléens est de la forme~:

\begin{galgas34}
boolset @MonEnsemble {
  # Liste de déclaration d'indicateurs, par exemple :
  case indicateur0
  case indicateur1
  case indicateur2
}
\end{galgas34}

Le nom des indicateurs doivent être différents des noms suivants~: \texttt{description}, \texttt{dynamicType}, \texttt{object}. L'implémentation actuelle limite à 64 le nombre d'indicateurs qui peuvent être définis.

\section{Initialisateur}

Un \ggsq!boolset! ne définit qu'un seul initialisateur~:
\begin{galgas34}
init @MonEnsemble () -> @MonEnsemble
\end{galgas34}
Il construit un ensemble vide.

Pour construire un ensemble plein, utiliser l'opérateur de complémentation \ggsq!~!~:
\begin{galgas34}
var @MonEnsemble ensemblePlein = ~ ()
\end{galgas34}










\section{Fonctions de classes : construire un ensemble d'un seul élément}

Pour chaque indicateur, une fonction de classe est implicitement définie, qui construit un ensemble constitué d'un seul élément~:

Pour initialiser un \ggsq!boolset!, on utilise un des constructeurs définis~:
\begin{galgas34}
var x = @MonEnsemble.indicateur1
\end{galgas34}

Si on veut un ensemble contenant plusieurs éléments, on utilise l'opérateur \ggsq=|=, qui effectue l'union de ses opérandes~:
\begin{galgas34}
var x = @MonEnsemble.indicateur1 | @MonEnsemble.indicateur2
log x # LOGGING x: <boolset @MonEnsemble: indicateur1 indicateur2>
\end{galgas34}

L'inférence de type permet d'éliminer les annotations de type non nécessaires~:
\begin{galgas34}
var x = @MonEnsemble.indicateur1 | .indicateur2
\end{galgas34}

Ou encore~:
\begin{galgas34}
var @MonEnsemble x = .indicateur1 | .indicateur2
\end{galgas34}





















\section{Getters}

Un \emph{getter} sans argument est défini pour chaque élément~: il permet de savoir si un ensemble contient un élément.


\begin{galgas3}
var x = @MonEnsemble.indicateur1
var b = [x indicateur1]
log b # LOGGING b: <@bool:true>
b = [x indicateur2]
log b # LOGGING b: <@bool:false>
\end{galgas3}

\begin{galgas4}
var x = @MonEnsemble.indicateur1
var b = x.indicateur1
log b # LOGGING b: <@bool:true>
b = x.indicateur2
log b # LOGGING b: <@bool:false>
\end{galgas4}






\section{Opérateur préfixe}

Un seul opérateur préfixe est défini pour tout \ggsq=boolset= (\refTableau{operateurPrefixeBoolset}).

\begin{table}[ht!]
  \centering
  \begin{tabular}{rl}
    {\bf Expression} & {\bf Signification} \\
    \ggsq+~ a+ & Complémentation.\\
  \end{tabular}
  \caption{Opérateur préfixe des types \texttt{boolset}}
  \labelTableau{operateurPrefixeBoolset}
\end{table}








\section{Opérateurs infixes}

Les opérateurs infixes du \refTableau{operateursInfixesBoolset} sont définis pour tout \ggsq=boolset=.

\begin{table}[ht!]
  \centering
  \begin{tabular}{rl}
    {\bf Expression} & {\bf Signification} \\
    \ggsq+a & b+ & Intersection~: ensemble des indicateurs appartenant à \ggsq=a= et à \ggsq=b=.\\
    \ggsq+a | b+ & Union~: ensemble des indicateurs appartenant à \ggsq=a= ou à \ggsq=b=. \\
    \ggsq+a ^ b+ & Exclusion~: ensemble des indicateurs appartenant soit à \ggsq=a=, soit à \ggsq=b=. \\
    \ggsq+a - b+ & Différence~: ensemble des indicateurs appartenant à \ggsq=a= et n'appartenant pas à \ggsq=b=.
%    \ggsq+a == b+ & Test d'égalité \\
%    \ggsq+a != b+ & Test d'inégalité
  \end{tabular}
  \caption{Opérateurs infixes des types \texttt{boolset}}
  \labelTableau{operateursInfixesBoolset}
\end{table}

