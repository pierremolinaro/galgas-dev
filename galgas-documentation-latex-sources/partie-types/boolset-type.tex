  %!TEX encoding = UTF-8 Unicode
%!TEX root = ../galgas-book.tex

%--------------------------------------------------------------
\chapterLabel{Le type \texttt{boolset}}{typeBoolset}
%-------------------------------------------------------------

\tableDesMatieresDuChapitre



Le mot-clé \ggs!boolset! permet de définir des types d'ensembles d'indicateurs booléens. Un tel objet a une sémantique de valeur.

La syntaxe de définition d'un type ensemble d'indicateurs booléens est de la forme~:

\begin{galgas}
boolset @MonEnsemble {
  # Liste de déclaration d'indicateurs, par exemple :
  indicateur0,
  indicateur1,
  indicateur2
}
\end{galgas}

Il n'est pas possible de définir du code dans cette déclaration~: la seule possibilité est de le définir dans des extensions (\refChapterPage{chapitreExtensions}).

Le nom des indicateurs doivent être différents des noms suivants~: \texttt{all}, \texttt{description}, \texttt{dynamicType}, \texttt{none}, \texttt{object}. L'implémentation actuelle limite à 64 le nombre d'indicateurs qui peuvent être définis.

Pour initialiser un \ggs!boolset!, on utilise un des constructeurs définis~:
\begin{galgas}
var x = @MonEnsemble.indicateur1
log x # LOGGING x: <boolset @MonEnsemble: indicateur1>
\end{galgas}

Si on veut un ensemble ayant plusieurs indicateurs à vrai, on utilise l'opérateur \ggs=|=, qui effectue l'union de ses opérandes~:
\begin{galgas}
var x = @MonEnsemble.indicateur1 | @MonEnsemble.indicateur2
log x # LOGGING x: <boolset @MonEnsemble: indicateur1 indicateur2>
\end{galgas}

L'inférence de type permet d'éliminer les annotations de type non nécessaires~:
\begin{galgas}
var x = @MonEnsemble.indicateur1 | .indicateur2
\end{galgas}

Ou encore~:
\begin{galgas}
@MonEnsemble x = .indicateur1 | .indicateur2
\end{galgas}

Pour tester la valeur d'un indicateur, on utilise le \emph{getter} du même nom~:
\begin{galgas}
@bool b = [x indicateur2]
\end{galgas}










\section{Constructeurs}

Un constructeur est défini pour chaque indicateur (\refSubsectionPage{constructeurBoolsetIndicateur}).

Trois constructeurs particuliers sont implicitement définis pour tout ensemble de booléens~:
\begin{itemize}
  \item le constructeur \texttt{none} (\refSubsectionPage{constructeurBoolsetNone})~;
  \item le constructeur \texttt{all} (\refSubsectionPage{constructeurBoolsetAll})~;
  \item le constructeur \texttt{default} (\refSubsectionPage{constructeurBoolsetDefault}).
\end{itemize}

\subsectionLabel{Constructeur ayant le nom d'un indicateur}{constructeurBoolsetIndicateur}

Ce constructeur définit un ensemble dont le seul booléen portant le nom de l'indicateur est vrai, les autres sont faux.

\begin{galgas}
var x = @MonEnsemble.indicateur1
log x # LOGGING x: <boolset @MonEnsemble: indicateur1>
\end{galgas}

Si le contexte le permet, l'annotation de type peut être omis lors de l'appel du constructeur~:
\begin{galgas}
@MonEnsemble x = .none
\end{galgas}


\subsectionLabel{Constructeur \texttt{none}}{constructeurBoolsetNone}

Ce constructeur définit un ensemble dont tous les booléens sont faux.

\begin{galgas}
var x = @MonEnsemble.none
log x # LOGGING x: <boolset @MonEnsemble:>
\end{galgas}

Si le contexte le permet, l'annotation de type peut être omis lors de l'appel du constructeur~:
\begin{galgas}
@MonEnsemble x = .none
\end{galgas}


\subsectionLabel{Constructeur \texttt{all}}{constructeurBoolsetAll}

Ce constructeur définit un ensemble dont tous les booléens sont vrais.

\begin{galgas}
var x = @MonEnsemble.all
log x # LOGGING x: <boolset @MonEnsemble: indicateur0 indicateur1 indicateur2>
\end{galgas}

Si le contexte le permet, l'annotation de type peut être omis lors de l'appel du constructeur~:
\begin{galgas}
@MonEnsemble x = .all
\end{galgas}


\subsectionLabel{Constructeur \texttt{default}}{constructeurBoolsetDefault}

Le constructeur \ggs=default= est défini implicitement, et a la même signification que le constructeur \texttt{none}.

\begin{galgas}
var x = @MonEnsemble.default
log x # LOGGING x: <boolset @MonEnsemble:>
\end{galgas}














\section{Getters}

Un \emph{getter} est défini pour chaque indicateur (\refSubsectionPage{getterBoolsetIndicateur})~: il permet de tester un indicateur.

Deux getters particuliers sont implicitement définis pour tout ensemble de booléens~:
\begin{itemize}
  \item le getter \texttt{none} (\refSubsectionPage{getterBoolsetNone})~;
  \item le getter \texttt{all} (\refSubsectionPage{getterBoolsetAll}).
\end{itemize}

\subsectionLabel{Getter ayant le nom d'un indicateur}{getterBoolsetIndicateur}

Ce getter permet d'obtenir la valeur de l'indicateur nommé.

\begin{galgas}
var x = @MonEnsemble.indicateur1
var b = [x indicateur1]
log b # LOGGING b: <@bool:true>
b = [x indicateur2]
log b # LOGGING b: <@bool:false>
\end{galgas}



\subsectionLabel{Getter \texttt{none}}{getterBoolsetNone}

Ce getter renvoie \ggs=true= si tous les indicateurs sont faux.

\begin{galgas}
var x = @MonEnsemble.none
var b = [x none]
log b # LOGGING b: <@bool:true>
\end{galgas}





\subsectionLabel{Getter \texttt{all}}{getterBoolsetAll}

Ce getter renvoie \ggs=true= si tous les indicateurs sont vrais.

\begin{galgas}
var x = @MonEnsemble.all
var b = [x all]
log b # LOGGING b: <@bool:true>
\end{galgas}





\section{Opérateurs infixes}

Les opérateurs infixes du \refTableau{operateursInfixesBoolset} sont définis pour tout \ggs=boolset=.

\begin{table}[t]
  \centering
  \begin{tabular}{ll}
    {\bf Expression} & {\bf Signification} \\
    \ggs+a & b+ & Intersection~: ensemble des indicateurs appartenant à \ggs=a= et à \ggs=b=.\\
    \ggs+a | b+ & Union~: ensemble des indicateurs appartenant à \ggs=a= ou à \ggs=b=. \\
    \ggs+a ^ b+ & Exclusion~: ensemble des indicateurs appartenant soit à \ggs=a=, soit à \ggs=b=. \\
    \ggs+a - b+ & Différence~: ensemble des indicateurs appartenant à \ggs=a= et n'appartenant pas à \ggs=b=.\\
    \ggs+a == b+ & Test d'égalité \\
    \ggs+a != b+ & Test d'inégalité \\
  \end{tabular}
  \caption{Opérateurs infixes des types \texttt{boolset}}
  \labelTableau{operateursInfixesBoolset}
\end{table}






\section{Opérateur préfixe}

Un seul opérateur préfixe est défini pour tout \ggs=boolset= (\refTableau{operateurPrefixeBoolset}).

\begin{table}[t]
  \centering
  \begin{tabular}{ll}
    {\bf Expression} & {\bf Signification} \\
    \ggs+~ a+ & Complémentation~: est équivalent à  \ggs=.all - a=.\\
  \end{tabular}
  \caption{Opérateur préfixe des types \texttt{boolset}}
  \labelTableau{operateurPrefixeBoolset}
\end{table}

