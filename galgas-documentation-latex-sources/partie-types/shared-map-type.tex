%!TEX encoding = UTF-8 Unicode
%!TEX root = ../galgas-book.tex

%--------------------------------------------------------------
\chapterLabel{Le type \texttt{shared map}}{typeSharedMap}
%-------------------------------------------------------------

\tableDesMatieresDuChapitre


Comme un objet de type  \ggs+map+, un objet de type \ggs+shared map+ est une table de symboles (chaque symbole étant associé à des valeurs), mais présente plusieurs différences~:
\begin{itemize}
  \item une entrée peut être \emph{résolue} ou \emph{non résolue}~;
  \item il est possible d'obtenir un \emph{pointeur faible} sur chaque entrée, qu'elle soit résolue ou non ;
  \item un objet de type \ggs+shared map+ a une sémantique de référence.
\end{itemize}



\section{Déclaration}

La déclaration d'un type \ggs+shared map+ nomme~:
\begin{itemize}
  \item des attributs associés au type table (\refSectionPage{attributsTablePartagee})~;
  \item les \emph{setters} d'insertion (\refSectionPage{setterInsertionTablePartagee})~;
  \item les \emph{méthodes} de recherche (\refSectionPage{methodesRechercheTablePartagee})~;
%  \item les \emph{setters} de retrait (\refSectionPage{setterRetraitTablePartagee}).
\end{itemize}

Les clés sont déclarées implicitement et sont du type \refTypePredefini{lstring}. La déclaration d'une \ggs+shared map+ déclare aussi implicitement le type associé \ggs+@t-entry+ (où \ggs+@t+ est le nom du type table), qui représente une entrée de la table (voir \refSectionPage{typeAssocieEntry}).

Par exemple~:

\begin{galgas}
shared map @MaTable {
  @string mPremier
  @bool mSecond
  insert insertKey error message "the '%K' key is already declared in %L"
  search searchKey error message "the '%K' key is not defined"
}
\end{galgas}

Cette déclaration déclare aussi implicitement le type \ggs+@MaTable-entry+.





\section{Constructeurs}

Pour initialiser une table vide, il y a deux possibilités, sémantiquement identiques~:
\begin{itemize}
%  \item la constante \texttt{\{\}} (\refSubsectionPage{constanteSharedMapVide})~; 
  \item le constructeur \texttt{emptyMap} (\refSubsectionPage{constructeurEmptySharedMap})~; 
  \item le constructeur \texttt{default} (\refSubsectionPage{constructeurSharedMapDefault}). 
\end{itemize}


%\subsectionLabel{Constante \texttt{\{\}}}{constanteSharedMapVide}
%
%Cette constante permet d'instancier une table vide. Exemple~:
%\begin{galgas}
%@MaTable uneTable = .emptyMap
%\end{galgas}
%
%Ou encore~:
%
%\begin{galgas}
%var uneTable = @MaTable .emptyMap
%\end{galgas}

\subsectionLabel{Constructeur \texttt{emptyMap}}{constructeurEmptySharedMap}

Pour instancier une table vide, une autre possibilité est d'appeler le constructeur \ggs=emptyMap=. Exemple~:
\begin{galgas}
@MaTable uneTable = .emptyMap
\end{galgas}

Ou encore~:

\begin{galgas}
var uneTable = @MaTable.emptyMap
\end{galgas}

 

\subsectionLabel{Constructeur \texttt{default}}{constructeurSharedMapDefault}


Une table accepte le constructeur \ggs=default=. Exemple~:
\begin{galgas}
@MaTable uneTable = .default
\end{galgas}

Ou encore~:

\begin{galgas}
var uneTable = @MaTable.default
\end{galgas}

 

%\subsection{Constructeur \texttt{mapWithMapToOverride}}
%
%\begin{galgas}
%constructor mapWithMapToOverride ?@T inMapToOverride -> @T
%\end{galgas}
%
%Ce constructeur permet d'instancier une table vide, qui surcharge la table \ggs+inMapToOverride+ citée en argument. Exemple~:
%\begin{galgas}
%@MaTable uneTable = .emptyMap
%@MaTable autreTableTable = .mapWithMapToOverride {!uneTable}
%\end{galgas}






\sectionLabel{Setters d'insertion}{setterInsertionTablePartagee}


Une \ggs+shared map+ peut déclarer zéro, un ou plusieurs \emph{setters} d'insertion. Un \emph{setter} d'insertion permet d'insérer une nouvelle entrée à une table. Une erreur est déclenchée en cas de tentative d'une clé déjà existante.


Un \emph{setter} d'insertion est déclaré par~:

\begin{galgas}
insert nom error message "message_erreur"
\end{galgas}

L'identificateur \ggs+nom+ donne un nom au \emph{setter} d'insertion~; ce nom doit être unique parmi les \emph{setters} d'insertion et de retrait. La chaîne de caractères \ggs+"message_erreur"+ définit le message d'erreur qui est affiché en cas de tentative d'une clé déjà existante. Cette chaîne accepte deux séquences d'échappement~:
\begin{itemize}
  \item \texttt{\%K}, qui est remplacée par la chaîne de caractères de la clé existante~;
  \item \texttt{\%L}, qui est remplacée par la chaîne décrivant la position de la clé existante dans les fichiers source.
\end{itemize}


Un \emph{setter} d'insertion est appelé dans une \emph{instruction d'appel de setter}, comprenant tous ses arguments en sortie~:
\begin{itemize}
  \item le premier argument est une expression de type \ggs+@lstring+ qui caractérise la clé à insérer~;
  \item ensuite, pour chaque attribut déclaré, une expression du type de cet attribut.
\end{itemize}

Par exemple~:
\begin{galgas}
@MaTable uneTable = .emptyMap
@lstring clef = ...
@string s = ...
@uint v = ...
[!?uneTable insertKey !clef !s !v]
\end{galgas}











\sectionLabel{Méthodes de recherche}{methodesRechercheTablePartagee}

Une \ggs+shared map+ peut déclarer zéro, une ou plusieurs \emph{méthodes} de recherche. Une \emph{méthode} de recherche permet de rechercher une entrée d'une table, et retourne la valeur de ses attributs associés. Une erreur est déclenchée si la clé n'existe pas.


Une \emph{méthode} de recherche est déclarée par~:


\begin{galgas}
search nom error message "message_erreur"
\end{galgas}

L'identificateur \ggs+nom+ donne un nom à la \emph{méthode} de recherche~; ce nom doit être unique parmi ces \emph{méthodes}. La chaîne de caractères \ggs+"message_erreur"+ définit le message d'erreur qui est affiché en cas de recherche d'une clé inexistante. Cette chaîne accepte une séquence d'échappement~:
\begin{itemize}
  \item \texttt{\%K}, qui est remplacée par la chaîne de caractères de la clé inexistante recherchée.
\end{itemize}


Une \emph{méthode} de recherche est appelée dans une \emph{instruction d'appel de méthode}~:
\begin{itemize}
  \item le premier argument (sortie) est une expression de type \ggs+@lstring+ qui caractérise la clé à rechercher~;
  \item ensuite, pour chaque attribut déclaré, un argument en entrée nommant une variable destinée à recevoir la valeur de l'attribut correspondant.
\end{itemize}

Par exemple~:
\begin{galgas}
@MaTable uneTable = .emptyMap
...
@lstring clef = ...
[uneTable searchKey !clef ?@string s ?@uint v]
\end{galgas}


%\subsection{Attribut \texttt{\%location}}
%
%À partir de la version 3.2.13, la déclaration d'une méthode de recherche peut comporter l'attribut \ggs=%location=, placée juste après le nom de la méthode~:
%
%\begin{galgas}
%search nom %location error message "message_erreur"
%\end{galgas}
%
%La conséquence est que l'appel a la méthode de recherche doit comporter un argument effectif supplémentaire, de type \ggs=@location=, qui recevra l'information de localisation de la clé recherchée~:
%
%\begin{galgas}
%@MaTable uneTable = .emptyMap
%...
%@lstring clef = ...
%[uneTable searchKey !clef ?@string s ?@uint v ?@location key]
%\end{galgas}
%
%La valeur retournée est la même que celle renvoyée par le getter \ggs=locationForKey= (\refSubsectionPage{getterLocationForKey}). Si la clé recherchée n'existe pas, une valeur poison est retournée.







%\sectionLabel{Setters de retrait}{setterRetraitTablePartagee}
%
%Une \ggs+map+ peut déclarer zéro, un ou plusieurs \emph{setters} de retrait. Un \emph{setter} de recherche permet de retirer une entrée d'une table, et retourne la valeur des attributs de la clé retirée. Une erreur est déclenchée si la clé n'existe pas.
%
%
%Un \emph{setter} de retrait est déclaré par~:
%
%\begin{galgas}
%remove nom error message "message_erreur"
%\end{galgas}
%
%L'identificateur \ggs+nom+ donne un nom au \emph{setter} de retrait~; ce nom doit être unique parmi les \emph{setters} d'insertion et de retrait. La chaîne de caractères \ggs+"message_erreur"+ définit le message d'erreur qui est affiché en cas de recherche d'une clé inexistante. Cette chaîne accepte une séquence d'échappement~:
%\begin{itemize}
%  \item \texttt{\%K}, qui est remplacée par la chaîne de caractères de la clé inexistante à retirer.
%\end{itemize}
%
%
%Un \emph{setter} de retrait est appelé dans une \emph{instruction d'appel de setter}~:
%\begin{itemize}
%  \item le premier argument (sortie) est une expression de type \ggs+@lstring+ qui caractérise la clé à retirer~;
%  \item ensuite, pour chaque attribut déclaré, un argument en entrée nommant une variable destinée à recevoir la valeur de l'attribut correspondant de la clé retirée.
%\end{itemize}
%
%Par exemple~:
%\begin{galgas}
%@MaTable uneTable = .emptyMap
%...
%@lstring clef = ...
%[!?uneTable removeKey !clef ?@string s ?@uint v]
%\end{galgas}
%
\section{Getters}

%\subsection{Getter \texttt{allKeyList}}
%
%\begin{galgas}
%getter @T allKeyList -> @lstringlist
%\end{galgas}
%
%Le \emph{getter} \ggs+allKeyList+ retourne la liste construite avec toutes les clés du récepteur, dans la table de premier niveau et dans les tables surchargées. L'ordre de la liste est~:
%\begin{itemize}
%  \item d'abord les clés de la table de premier niveau, puis celles des tables surchargées, dans l'ordre de la surcharge~;
%  \itel pour chaque table, les clés apparaissent dans l'ordre alphabétique croissant.
%\end{itemize}

\subsection{Getter \texttt{count}}

\begin{galgas}
getter count -> @uint
\end{galgas}


Le \emph{getter} \ggs+count+ retourne un \ggs+@uint+ qui contient le nombre d'entrées de la table de premier niveau du récepteur.



\subsection{Getter \texttt{hasKey}}

\begin{galgas}
getter hasKey ?@string inKey -> @bool
\end{galgas}


Le \emph{getter} \ggs+hasKey+ retourne un \ggs+@bool+ qui est \ggs+true+ si la clé \ggs+inKey+ est dans la table de premier niveau du récepteur, \ggs+false+ dans le cas contraire.



\subsection{Getter \texttt{keyList}}

\begin{galgas}
getter keyList -> @lstringlist
\end{galgas}


Le \emph{getter} \ggs+keyList+ retourne la liste construite avec toutes les clés de la table de premier niveau du récepteur. L'ordre de la liste est l'ordre alphabétique croissant des clés.



\subsection{Getter \texttt{keySet}}

\begin{galgas}
getter keySet -> @stringset
\end{galgas}


Le \emph{getter} \ggs+keySet+ retourne l'ensemble de toutes les clés de la table de premier niveau du récepteur.





\subsection{Getter \texttt{locationForKey}}

\begin{galgas}
getter locationForKey ?@string inKey -> @location
\end{galgas}


Le \emph{getter} \ggs+locationForKey+ retourne un \ggs+@location+ qui contient l'information de position de la clé \ggs+inKey+ dans la table de premier niveau du récepteur. Une erreur d'exécution est déclenchée si cette clé n'existe pas.








%\subsection{Getter \texttt{overriddenMap}}
%
%\begin{galgas}
%getter overriddenMap -> @T
%\end{galgas}
%
%
%Le \emph{getter} \ggs+overriddenMap+ retourne la table obtenue en amputant de la valeur du récepteur la table de premier niveau. Si le récepteur n'a pas de table surchargée, une erreur d'exécution est déclenchée.









\sectionLabel{Attributs d'une table}{attributsTablePartagee}

Un attribut peut être définis dans la déclaration d'une table : \ggs!%selectors!.

L'attribut \ggs!%selectors! impose l'utilisation de sélecteurs dans l'appel des setters d'insertion (\refSectionPage{setterInsertionTablePartagee}), des méthodes de recherche (\refSectionPage{methodesRechercheTablePartagee}). Le sélecteur associé à la clé est \ggs=!lkey:=. Le sélecteur associé à une propriété porte le nom de cette propriété.


\begin{galgas}
shared map @MaTable %selectors {
  @string mPremier
  @uint mSecond
  insert insertKey error message "the '%K' key is already declared in %L"
  search searchKey error message "the '%K' key is not defined"
}
\end{galgas}

L'appel d'un \emph{setter} d'insertion :
\begin{galgas}
@MaTable uneTable = .emptyMap
@lstring clef = ...
@string s = ...
@uint v = ...
[!?uneTable insertKey !lkey: clef !mPremier: s !mSecond: v]
\end{galgas}



L'appel d'une méthode de recherche :
\begin{galgas}
@MaTable uneTable = .emptyMap
...
@lstring clef = ...
[uneTable searchKey !lkey: clef ?mPremier: @string s ?mSecond: @uint v]
\end{galgas}

%
%L'appel d'un \emph{setter} de retrait :
%\begin{galgas}
%@MaTable uneTable = .emptyMap
%...
%@lstring clef = ...
%[!?uneTable removeKey !lkey: clef ?mPremier: @string s ?mSecond: @uint v]
%\end{galgas}
%
%Lors de l'échec d'une recherche de clés, une méthode de recherche ou un setter de retrait affiche les clés voisines contenues dans la table. L'attribut \ggs!%noSuggestion! inhibe l'affichage de ces suggestions.




%\section{Énumération}
%
%L'instruction \ggs+for+ permet d'énumérer des objets de type \ggs+map+ ; elle est décrite à la \refSectionPage{instructionFor}.





\sectionLabel{Le type associé \texttt{@t-entry}}{typeAssocieEntry}

La déclaration de \ggs+shared map @t+ déclare aussi implicitement le type associé \ggs+@t-entry+, qui représente une entrée de la table.

L'implémentation de ce type est implicite.

Un objet de type \ggs+@t-entry+ est un \emph{pointeur faible} vers une entrée d'une table.




\subsection{Constructeurs}

Deux constructeurs sont implémentés, \ggs+null+ et  \ggs+default+, et ont le même effet : construire une entrée non résolue.

\begin{galgas}
  @t-entry uneEntrée = .null
  @t-entry uneEntrée = .default
\end{galgas}


