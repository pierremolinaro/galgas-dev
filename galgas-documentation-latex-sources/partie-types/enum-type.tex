%!TEX encoding = UTF-8 Unicode
%!TEX root = ../galgas-book.tex

%--------------------------------------------------------------
\chapter{Le type \texttt{enum}}
%-------------------------------------------------------------

Galgas permet à l'utilisateur de définir des types énumérés.

\section{Déclaration}

La déclaration d'un type \galgas{enum} nomme l'ensemble des constantes associées à ce type.

Par exemple :

\begin{galgascode}
enum @feuTricolore {
  vert, orange, rouge   
}
\end{galgascode}

Plusieurs types énumérés peuvent définir des constantes de même nom.

\section{Instanciation}

Chaque constante définit un constructeur de même nom. On peut ainsi écrire :

\begin{galgascode}
@feuTricolore feu := [@feuTricolore vert] ;
\end{galgascode}

Ou encore :

\begin{galgascode}
@feuTricolore feu [vert] ;
\end{galgascode}

\section{Comparaison}

Un type enuméré accpete les six opérateurs de comparaison (\galgas{==}, \galgas{\!=}, \galgas{<}, \galgas{<=}, \galgas{>} et \galgas{>}). L'ordre est celui de la déclaration, c'est-à-dire que :
\begin{galgascode}
  [@feuTricolore vert] < [@feuTricolore orange] < [@feuTricolore rouge]
\end{galgascode}


\section{L'instruction \texttt{switch}}

L'instruction \galgas{switch} (\refSectionPage{instructionSwitch}) est dédiée aux types énumérés.
