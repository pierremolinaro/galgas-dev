%!TEX encoding = UTF-8 Unicode
%!TEX root = ../galgas-book.tex

%--------------------------------------------------------------
\chapterLabel{Le type \texttt{enum}}{typeEnum}
%-------------------------------------------------------------

\tableDesMatieresLocaleDeProfondeurRelative{2}



\section{Déclaration}

La déclaration d'un type \ggst+enum+ nomme l'ensemble des constantes du type énuméré. Plusieurs types énumérés peuvent définir des constantes de même nom. Dans ce premier exemple, les constantes n'ont pas de valeur associée.

Par exemple :

\begin{galgas34}
enum @feuTricolore {
  case vert
  case orange
  case rouge
}
\end{galgas34}

Il est possible d'associer plusieurs valeurs à chaque constante d'un type énuméré. Par exemple~:

\begin{galgas3}
enum @monOption {
  case noOption
  case option1 (@string name)
  case option2 (@uint value @string name)
}
\end{galgas3}

\begin{galgas4}
enum @monOption {
  case noOption
  case option1 (@string name)
  case option2 (@uint value, @string name)
}
\end{galgas4}

Chaque valeur associée est spécifiée par son type et son nom. Noter la virgule de séparation en GALGAS4.






\section{Instanciation}

Chaque constante définit un initialisateur de même nom. Si la constante n'a pas de valeur associée, on écrit :

\begin{galgas34}
var @feuTricolore feu = @feuTricolore.vert ()
\end{galgas34}

Les parenthèses peuvent être omises :
\begin{galgas34}
var @feuTricolore feu = @feuTricolore.vert
\end{galgas34}

L'annotation de type peut être omise :

\begin{galgas34}
var @feuTricolore feu = .vert
\end{galgas34}

\begin{galgas34}
var feu = @feuTricolore.vert
\end{galgas34}


Si la constante présente des valeurs associées, l'initialisateur présente autant d'arguments, chaque ayant comme étiquette le nom donné dans la déclaration~:

\begin{galgas3}
var @monOption feu = @monOption.option2 {!name: "toto" !value: 0}
\end{galgas3}

\begin{galgas34}
var @monOption feu = @monOption.option2 (!name: "toto", !value: 0)
\end{galgas34}

Comme dans l'exemple précédent, l'annotation de type peut être omise.








\section{L'instruction \texttt{switch}}

L'instruction \ggst+switch+ (\refSectionPage{instructionSwitch}) est dédiée à tous les types énumérés. On écrit par exemple :

\begin{galgas34}
var @feuTricolore feu = ...
switch feu
case vert : ...
case orange : ...
case rouge : ...
end
\end{galgas34}

Chaque constante doit apparaître une et une seule fois~: l'instruction \ggst+switch+ est exhaustive.

Si une constante présente des valeurs associées, alors elles doivent toutes être mentionnées~:

\begin{galgas3}
var @monOption uneOption = ...
switch uneOption
case noOption : ...
case option1 (@string unNom) : ...
case option2 (@uint unEntier @string unNom) : ...
end
\end{galgas3}

L'annotation de type d'une valeur associée peut être omise~:

\begin{galgas3}
var @monOption uneOption = ...
switch uneOption
case noOption : ...
case option1 (unNom) : ...
case option2 (unEntier unNom) : ...
end
\end{galgas3}

Si une valeur associée n'est pas utile, on peut remplacer son nom par un joker~:
\begin{galgas3}
var @monOption uneOption = ...
switch uneOption
case noOption : ...
case option1 (unNom) : ...
case option2 (unEntier *) : ...
end
\end{galgas3}

Plusieurs \emph{jokers} consécutifs peuvent être regroupés~:

\begin{galgas3}
var @monOption uneOption = ...
switch uneOption
case noOption : ...
case option1 (unNom) : ...
case option2 (2*) : ...
end
\end{galgas3}

En GALGAS4, les valeurs associées consécutives doivent être séparées par des virgules~:

\begin{galgas4}
var @monOption uneOption = ...
switch uneOption
case noOption : ...
case option1 (unNom) : ...
case option2 (unEntier, unNom) : ...
end
\end{galgas4}

\begin{galgas4}
var @monOption uneOption = ...
switch uneOption
case noOption : ...
case option1 (unNom) : ...
case option2 (unEntier, *) : ...
end
\end{galgas4}






\sectionLabel{Tester une valeur}{testerValeurEnum}

Un type émuméré définit implicitement un \emph{getter} sans argument pour chaque constante déclarée~; ce getter a pour nom le nom de la constante.

\subsection{Constante sans valeur associée}

Si la constante n'a pas de valeur associée, ce getter renvoie une valeur booléenne qui permet de savoir si le récepteur a pour valeur cette constante. Ce \emph{getter} a pour signature~:
\begin{galgas4}
func @monOption.noOption () -> @bool
\end{galgas4}
\begin{galgas3}
func @monOption noOption -> @bool
\end{galgas3}

Ainsi~:

\begin{galgas4}
var v = @monOption.noOption
let @bool b = v.noOption // b est true
v = @monOption.option1 (!name: "xyz")
let @bool c = v.noOption // c est false
\end{galgas4}
\begin{galgas3}
var v = @monOption.noOption
let @bool b = [v noOption] // b est true
v = @monOption.option1 (!name: "xyz")
let @bool c = [v noOption] // c est false
\end{galgas3}

\subsection{Constante avec valeurs associées}

Si la constante a des valeurs associées, ce getter renvoie une valeur optionnelle qui permet de savoir si le récepteur a pour valeur cette constante, et d'obtenir les valeurs associées.

Pour chaque constante ayant des valeurs associées, un type structure est implicitement défini, dont les propriétés sont les valeurs associées. Ainsi, sont définis~:
\begin{galgas4}
struct @monOption.@option1 {
  public let @string name
}
struct @monOption.@option2 {
  public let @uint value
  public let @string name
}
\end{galgas4}

Les \emph{getter} associés ont pour signature~:
\begin{galgas4}
func @monOption.option1 () -> @monOption.@option1?
func @monOption.option2 () -> @monOption.@option2?
\end{galgas4}
\begin{galgas3}
func @monOption option1 -> @monOption.@option1?
func @monOption option2 -> @monOption.@option2?
\end{galgas3}

\subsubsection{Pas d'intérêt pour les valeurs associées}

Si l'on ne s'intéresse pas aux valeurs associées, on teste si la valeur retournée est \ggsq!nil!. Tout optionnel présente deux propriétés, \ggsq!isNil! (qui renvoie \ggsq!true! si l'optionnel est \ggsq!nil!), et \ggsq!isSome! (qui renvoie \ggsq!true! si l'optionnel contient une valeur).

\begin{galgas4}
var v = @monOption.noOption
let @bool a = v.option1.isSome // a est false
v = @monOption.option1 (!name: "xyz")
let @bool b = v.option1.isSome // b est true
\end{galgas4}
\begin{galgas3}
var v = @monOption.noOption
let @bool a = [v option1].isSome // a est false
v = @monOption.option1 (!name: "xyz")
let @bool b = [v option1].isSome // b est true
\end{galgas3}

Si l'écriture paraît trop lourde, on peut définir un \emph{getter} qui renvoie une valeur booléenne~:
\begin{galgas4}
func @monOption.isOption1 () -> @bool {
  result = self.option1.isSome
}
\end{galgas4}
\begin{galgas3}
getter @monOption isOption1 -> @bool {
  result = [self option1].isSome
}
\end{galgas3}

Les tests deviennent alors~:
\begin{galgas4}
var v = @monOption.noOption
let @bool a = v.isOption1 // a est false
v = @monOption.option1 (!name: "xyz")
let @bool b = v.isOption1 // b est true
\end{galgas4}
\begin{galgas3}
var v = @monOption.noOption
let @bool a = [v isOption1] // a est false
v = @monOption.option1 (!name: "xyz")
let @bool b = [v isOption1] // b est true
\end{galgas3}




\subsubsection{Récupération des valeurs associées}

Si on s'intéresse aux valeurs associées, il faut extraire la valeur embarquée dans la valeur optionnelle renvoyée, ce qui peut être fait dans une instruction \ggsq!if!~:

\begin{galgas4}
var v = ...
if let x = v.option1 then
  // Ici, v a pour valeur option1
  // x a pour type @monOption.@option1
  // On accède à la valeur associée par x.name
else
  // Ici, v a pour une valeur autre que option1
  // et x n'est pas accessible
end
\end{galgas4}

\begin{galgas3}
var v = ...
if let x = [v option1] then
  // Ici, v a pour valeur option1
  // x a pour type @monOption.@option1
  // On accède à la valeur associée par x.name
else
  // Ici, v a pour une valeur autre que option1
  // et x n'est pas accessible
end
\end{galgas3}


















\section{L'attribut \texttt{\%testGetters}}

En ajoutant l'attribut \ggsq!%testGetters! à la déclaration d'une énumération, un getter est automatiquement engendré pour tester la valeur indépendamment de la présence de valeurs associées.

Par exemple~:

\begin{galgas3}
enum @monOption %testGetters {
  case noOption
  case option1 (@string name)
  case option2 (@uint value @string name)
}
\end{galgas3}

\begin{galgas4}
enum @monOption %testGetters {
  case noOption
  case option1 (@string name)
  case option2 (@uint value, @string name)
}
\end{galgas4}

L'attribut \ggsq!%testGetters! engendre les trois getters suivants~:
\begin{galgas3}
getter @monOption isNoOption -> @bool
getter @monOption isOption1 -> @bool
getter @monOption isOption2 -> @bool
\end{galgas3}
\begin{galgas3}
func @monOption.isNoOption () -> @bool
func @monOption.isOption1 () -> @bool
func @monOption.isOption2 () -> @bool
\end{galgas3}

Le nom du getter est \texttt{is} suivi du nom de la constante dont le premier caractère est mis en majuscule. Il faut veiller que ce nom n'entre pas en collision avec le nom d'une autre constante, comme dans l'exemple suivant~:

\begin{galgas4}
enum @enumerationAvecErreur %testGetters {
  case isTruc
  case truc // Erreur, le getter engendré entre en collision avec la 1re constante
}
\end{galgas4}


On peut donc tester la valeur par~:
\begin{galgas4}
  v.isOption1 // même valeur que v.option1.isSome
\end{galgas4}
\begin{galgas3}
  [v isOption1] // même valeur que [v isOption1].isSome
\end{galgas3}

Si le type énuméré ne définit aucune valeur associée, cette option n'a pas d'intérêt, les getters définis par défaut ont exactement le même rôle.












\section{Comparaison}

Par défaut, un type enuméré n'accepte aucun des six opérateurs de comparaison (\ggsq+==+, \ggsq+!=+, \ggsq+<+, \ggsq+<=+, \ggsq+>+, \ggsq+>+).

\subsection{L'attribut \texttt{\%equatable}}

Si le type énuméré est déclaré avec l'attribut \ggsq!%equatable!, alors les opérateurs \ggst+==+, \ggst+!=+ sont définis. Ceci exige que tous les types des valeurs associées soient eux-mêmes \ggsq!%equatable! ou \ggsq!%comparable!.

Comme aucune relation d'ordre n'est définie, l'ordre de déclaration des constantes est sans importance.

\begin{galgas3}
enum @monOption %equatable {
  case noOption
  case option1 (@string name)
  case option2 (@uint value @string name)
}
\end{galgas3}

\begin{galgas4}
enum @monOption %equatable {
  case noOption
  case option1 (@string name)
  case option2 (@uint value, @string name)
}
\end{galgas4}


\subsection{L'attribut \texttt{\%comparable}}

Si le type énuméré est déclaré avec l'attribut \ggsq!%comparable!, alors les six opérateurs \ggst+==+, \ggst+!=+, \ggsq+<+, \ggsq+<=+, \ggsq+>+ et \ggsq+>+ sont définis. Ceci exige que tous les types des valeurs associées soient eux-mêmes \ggsq!%comparable!.

\begin{galgas3}
enum @monOption %comparable {
  case noOption
  case option1 (@string name)
  case option2 (@uint value @string name)
}
\end{galgas3}

\begin{galgas4}
enum @monOption %comparable {
  case noOption
  case option1 (@string name)
  case option2 (@uint value, @string name)
}
\end{galgas4}

L'ordre est lexicographique : c'est celui de la déclaration, c'est-à-dire que \ggsq!@monOption.noOption < @monOption.option1! et \ggsq!@monOption.option1 < @monOption.option2!.

Pour les constantes ayant des valeurs associées, l'ordre est obtenu en comparant ces valeurs les unes après les autres~: par exemple pour \ggsq!option2!, on compare d'abord \ggsq!value!, puis \ggsq!name!. 









\section{Exemple d'utilisation des valeurs associées}

Associer des valeurs à chaque constante permet d'alléger dans certains cas le code à écrire. Supposons par exemple que l'on ait dans un langage une construction optionnelle :

\begin{galgas34}
rule <regleProduction> {
  select
  or
    $option$
    $identifier$ (?let nomOption)
  end
}
\end{galgas34}

Comment construire l'arbre syntaxique abstrait ? Il y a en fait trois possibilités.

\textbf{Première solution.} La première consiste à considérer la chaîne vide comme significative de l'absence d'option :
\begin{galgas34}
rule <regleProduction> {
  let @lstring nomOption
  select
    nomOption = "".here
  or
    $option$
    $identifier$ (?nomOption)
  end
}
\end{galgas34}

Évidemment, cette solution est acceptable uniquement si l'information associée est simple, et si une valeur particulière peut être considérée comme l'absence d'option.

\textbf{Deuxième solution.} La deuxième solution fait appel à trois classes :
\begin{galgas4}
abstract class @abstractOption {}

class @noOption : @abstractOption {}

class @option : @abstractOption { public let @lstring name }
\end{galgas4}

La construction est réalisée par :
\begin{galgas4}
rule <regleProduction> {
  let @abstractOption optionAST
  select
    optionAST = @noOption ()
  or
    $option$
    $identifier$ (?let nom)
    optionAST = @option (nom)
  end
}
\end{galgas4}

Cette solution, plus générale, est plus lourde à mettre en œuvre : trois classes.

\textbf{Troisième solution.} La troisième et dernière solution consiste à écrire un type énuméré possédant des valeurs associées :

\begin{galgas4}
enum @option {
  case noOption
  case optionPresente (@lstring name)
}
\end{galgas4}

À la constante \ggst+optionPresente+ est associée une valeur de type \ggst+@lstring+, identifiée par le nom \ggst+name+. La construction est maintenant réalisée par :
\begin{galgas4}
rule <regleProduction> {
  let @option optionAST
  select
    optionAST = .noOption
  or
    $option$
    $identifier$ (?let nomOption)
    optionAST = .optionPresente (!name: nomOption)
  end
}
\end{galgas4}












