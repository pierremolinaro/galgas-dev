%!TEX encoding = UTF-8 Unicode
%!TEX root = ../galgas-book.tex

\chapitreTypePredefiniLabelIndex{data}

\tableDesMatieresLocaleDeProfondeurRelative{1}


Le type \ggst=@data= est un buffer d'octets. Il peut être utilisé pour lire et écrire des fichiers binaires.




\section{Constructeurs}


\subsectionConstructor{dataWithContentsOfFile}{data}

\begin{galgas3}
constructor dataWithContentsOfFile ?@string inFilePath -> @data
\end{galgas3}

Ce constructeur instancie un objet \ggst=@data= avec le contenu du fichier désigné par \ggst=inFilePath=. Si le fichier n'existe pas, une erreur d'exécution est déclenchée et le constructeur renvoie une valeur poison.




\subsectionConstructor{emptyData}{data}

\begin{galgas3}
constructor emptyData -> @data
\end{galgas3}

Ce constructeur instancie un objet \ggst=@data= vide.








\section{Getters}



\subsectionGetter{count}{data}

\begin{galgas3}
getter count -> @uint
\end{galgas3}

Ce \emph{getter} renvoie le nombre d'octets du récepteur.


\subsectionGetter{cStringRepresentation}{data}

\begin{galgas3}
getter cStringRepresentation -> @string
\end{galgas3}

Ce \emph{getter} renvoie la valeur du récepteur sous la forme d'une liste d'octets séparés par des virgules. Chaque octet est écrit en décimal. Toutes les 16 valeurs, un retour-chariot est inséré.



\subsectionGetter{length}{data}

\begin{galgas3}
getter length -> @uint # Obsolète, utiliser count
\end{galgas3}

Ce \emph{getter} renvoie le nombre d'octets du récepteur.





\section{Méthodes}


\subsectionMethod{writeToExecutableFile}{data}

\begin{galgas3}
method writeToExecutableFile ?@string inFilePath
\end{galgas3}

Cette méthode écrit le contenu du récepteur dans le fichier désigné par \ggst=inFilePath=, et rend ce fichier exécutable.




\subsectionMethod{writeToFile}{data}

\begin{galgas3}
method writeToFile ?@string inFilePath
\end{galgas3}

Cette méthode écrit le contenu du récepteur dans le fichier désigné par \ggst=inFilePath=.




\subsectionMethod{writeToFileWhenDifferentContents}{data}

\begin{galgas3}
method writeToFileWhenDifferentContents
  ?@string inFilePath
  !@bool outFileModified
\end{galgas3}

Cette méthode écrit le contenu du récepteur dans le fichier désigné par \ggst=inFilePath=, uniquement si la valeur du récepteur est différente du contenu du fihier. La variable \ggst=outFileModified= est retournée à l'appelant, et permet de savoir si le fichier a été modifié ou non.







\section{Setters}


\subsectionSetter{appendByte}{data}

\begin{galgas3}
setter appendByte ?@uint inValue
\end{galgas3}

Ce \emph{setter} ajoute la valeur de \ggst=inValue= à la fin du récepteur. Comme un objet de \ggst=@data= est un tableau d'octets, \ggst=inValue= doit être compris entre $0$ et $255$. Si il est supérieur à $255$, une erreur d'exécution est déclenchée.



\subsectionSetter{appendData}{data}

\begin{galgas3}
setter appendData ?@data inData
\end{galgas3}

Ce \emph{setter} ajoute la valeur de \ggst=inData= à la fin du récepteur.





\subsectionSetter{appendShortBE}{data}

\begin{galgas3}
setter appendShortBE ?@uint inValue
\end{galgas3}

Pour ce \emph{setter}, \ggst=inValue= doit être compris entre $0$ et $2^{16}-1$, c'est-à-dire réprésentable par un entier non signé sur deux octets. Si ce n'est pas le cas, une erreur d'exécution est déclenchée. Si c'est le cas, deux octets sont ajoutés à la fin du récepteur, d'abord l'octet de poids fort, puis l'octet de poids faible.







\subsectionSetter{appendShortLE}{data}

\begin{galgas3}
setter appendShortLE ?@uint inValue
\end{galgas3}

Pour ce \emph{setter}, \ggst=inValue= doit être compris entre $0$ et $2^{16}-1$, c'est-à-dire réprésentable par un entier non signé sur deux octets. Si ce n'est pas le cas, une erreur d'exécution est déclenchée. Si c'est le cas, deux octets sont ajoutés à la fin du récepteur, d'abord l'octet de poids faible, puis l'octet de poids fort.








\subsectionSetter{appendUIntBE}{data}

\begin{galgas3}
setter appendUIntBE ?@uint inValue
\end{galgas3}

Ce \emph{setter} ajoute la valeur de \ggst=inValue= à la fin du récepteur, sous la forme de quatre octets, en commençant par l'octet de poids fort.









\subsectionSetter{appendUIntLE}{data}

\begin{galgas3}
setter appendUIntLE ?@uint inValue
\end{galgas3}

Ce \emph{setter} ajoute la valeur de \ggst=inValue= à la fin du récepteur, sous la forme de quatre octets, en commençant par l'octet de poids faible.











\subsectionSetter{appendUTF8String}{data}

\begin{galgas3}
setter appendUTF8String ?@string inValue
\end{galgas3}

Ce \emph{setter} ajoute la valeur de \ggst=inValue= à la fin du récepteur, sous la forme d'une chaîne de caractères UTF-8, y compris le zéro final.


\section{Énumération des valeurs}

Un objet de type \ggst=@data= est énumérable par une instruction \ggst=for= (\refSectionPage{instructionFor}).
