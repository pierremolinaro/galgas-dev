%!TEX encoding = UTF-8 Unicode
%!TEX root = ../galgas-book.tex

\chapitreTypePredefiniLabelIndex{location}

Un objet de type \galgas{@location} a pour valeur une position dans un texte source. Les objets de ce types sont utilisés dans les messages d'erreurs et les messages d'alerte pour indiquer à l'utilisateur la position de l'erreur ou de l'alerte.





\section{Le mot réservé \texttt{here}}

Le mot réservé \galgas{here} contient la position courante de l'analyse du texte source. Il doit être considéré comme un opérande particulier d'une expression, et, à ce titre, peut apparaître dans toute expression. On peut ainsi écrire :

\galgas{const @location currentLocation := here ;}

Plus précisément, la position capturée est le début du dernier \emph{token} analysé. Ainsi, si l'on écrit :

\begin{galgascode}
  $token$ ;
  ...
  @location currentLocation := here ;
\end{galgascode}

La position capturée est la position du premier caractère du token correspondant à \galgas{$token$}. Si \galgas{here} est appelé avant que le premier token soit analysé, la position capturée est le premier caractère du texte source.

\section{Constructor}

\constructeurSansArgument{nowhere}
{location}
{2.1.2}
{location}
{Returns an \galgas{@location} that does not point out any location.}
{The returned object responds \galgas{true} to the \refReaderPage{location}{isNowhere}.}

\section{Readers}

\readerSansArgument{column}
{location}
{1.8.2}
{uint}
{Returns an \galgas{@uint} value containing the column of the receiver's value.}
{this reader raises a run-time error if the receiver's value responds \galgas{true} to the \refReaderPage{location}{isNowhere}.}


\readerSansArgument{isNowhere}
{location}
{2.1.2}
{bool}
{Returns an \galgas{@bool} value indicating whether the receiver'value points out a source location or does not.}
{this reader returns \galgas{true} if the receiver's value does not point out an actual location in a text source (i.e. it has been constructed using the nowhere constructor), and \galgas{false} if the receiver's value points out an actual location in a text source (i.e. it has been constructed using the \galgas{here} keyword.}


\readerSansArgument{line}
{location}
{1.8.2}
{uint}
{Returns an \galgas{@uint} value containing the line of the receiver's value.}
{this reader raises a run-time error if the receiver's value responds \galgas{true} to the \refReaderPage{location}{isNowhere}.}


\readerSansArgument{locationIndex}
{location}
{1.8.2}
{uint}
{Returns an \galgas{@uint} value containing the the offset from the the beginning of the source of the location defined by receiver's value.}
{this reader raises a run-time error if the receiver's value responds \galgas{true} to the \refReaderPage{location}{isNowhere}.}


\readerSansArgument{locationString}
{location}
{1.8.2}
{string}
{returns an \galgas{@string} object that contains a string representation of the location defined by receiver's value.}
{this reader raises a run-time error if the receiver's value responds \galgas{true} to the \refReaderPage{location}{isNowhere}.}
