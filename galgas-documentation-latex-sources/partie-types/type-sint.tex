%!TEX encoding = UTF-8 Unicode
%!TEX root = ../galgas-book.tex

\chapitreTypePredefiniLabelIndex{sint}

An \ggs+@sint+ object value is a 32-bit signed integer value. You can initialize an \ggs+@sint+ object from an 32-bit signed integer constant:\\

\begin{galgas}
@sint mySignedInteger = 123_456S
\end{galgas}


Note that a 32-bit signed integer constant is characterized by the 'S' suffix.


\section{Constructors}

\subsectionConstructor{min}{sint}

\begin{galgas}
constructor min -> @sint
\end{galgas}

Returns an \ggs+@sint+ object that the minimum value of the 32-bit signed range ($-2^{31}$).





\subsectionConstructor{max}{sint}

\begin{galgas}
constructor max -> @sint
\end{galgas}


Returns an \ggs+@sint+ object that the maximum value of the 32-bit signed range ($2^{31}-1$).



\section{Getters}

\subsectionGetter{bigint}{sint}

\begin{galgasbox}
getter bigint -> @bigint
\end{galgasbox}

Ce \emph{getter} permet de convertir un \ggs!@sint! en \ggs!@bigint!. Comme la plage des valeurs des \ggs!bigint! n'est limitée que par la mémoire disponible, il n'échoue jamais.

\begin{galgas}
  message [[-1234S bigint] string] + "\n" # -1234
\end{galgas}


\subsectionGetter{double}{sint}

\begin{galgas}
getter double -> @double
\end{galgas}

Returns the receiver's value converted in a \ggs+@double+ object. As a 32-bit integer value can always be converted in a \ggs+@double+ value, this getter never fails.


\subsectionGetter{hexStringSeparatedBy}{sint}

\begin{galgas}
getter hexStringSeparatedBy ?@char inSeparator ?@uint inGroup -> @string
\end{galgas}

Returns the an hexadecimal string representation of the receiver value, prefixed by the string \texttt{0x}. Groups of \ggs=inGroup= digits are separated by the \ggs=inSeparator= character.

If \ggs=inGroup= is equal to zero, a run-time error is raised.

For example:
\begin{galgas}
let s = [0x12345678S hexStringSeparatedBy !'_' !2] # 0x12_34_56_78
\end{galgas}





\subsectionGetter{sint64}{sint}

\begin{galgas}
getter sint64 -> @sint64
\end{galgas}

Returns the receiver's value in an \refTypePredefini{sint64} (64-bit signed integer) object. As a 32-bit signed value can always be converted in a 64-bit signed value, this getter never fails.

This getter is the only way to convert an \refTypePredefini{sint} object into an \refTypePredefini{sint64} object.





\subsectionGetter{string}{sint}

\begin{galgas}
getter string -> @string
\end{galgas}

Returns a decimal string representation of the receiver's value. For an hexadecimal string representation of the receiver's value, see \refGetterPage{uint}{hexString} and \refGetterPage{uint}{xString}.







\subsectionGetter{uint}{sint}

\begin{galgas}
getter uint -> @uint
\end{galgas}

Returns the receiver's value in an \refTypePredefini{uint} (32-bit unsigned integer) object. An error is raised is receiver's value is negative.

This getter is the only way to convert an \refTypePredefini{sint} object into an \refTypePredefini{uint} object.




\subsectionGetter{uint64}{sint}

\begin{galgas}
getter uint64 -> @uint64
\end{galgas}

Returns the receiver's value in an \refTypePredefini{uint64} (64-bit unsigned integer) object. An error is raised is receiver's value is negative.

This getter is the only way to convert an \refTypePredefini{sint} object into an \refTypePredefini{uint64} object.





\section{Arithmétique}

\subsection{Opérateurs infixés}

Le type \ggs+@sint+ accepte les opérateurs arithmétiques infixés suivants :
\begin{itemize}
  \item \ggs!+!, addition, une erreur d'exécution est déclenchée en cas de débordement ;
  \item \ggs!-!, soustraction, une erreur d'exécution est déclenchée en cas de débordement ;
  \item \ggs!*!, multiplication, une erreur d'exécution est déclenchée en cas de débordement ;
  \item \ggs!/!, division, une erreur d'exécution est déclenchée si le diviseur est nul ;
  \item \ggs!mod!, calcul du reste, une erreur d'exécution est déclenchée si le diviseur est nul ;
  \item \ggs!&+!, addition, le résultat étant silencieusement tronqué sur 32 bits ;
  \item \ggs!&-!, soustraction, le résultat étant silencieusement tronqué sur 32 bits ;
  \item \ggs!&*!, multiplication, le résultat étant silencieusement tronqué sur 32 bits ;
  \item \ggs!&/!, division, qui retourne zéro si le diviseur est nul.
\end{itemize}

Ces opérateurs exigent que les deux opérandes soient des objets du même type \ggs+@sint+. 

\subsection{Opérateurs préfixés}
Le type \ggs+@sint+ accepte les opérateurs arithmétiques préfixés suivants :
\begin{itemize}
  \item \ggs!+!, qui retourne simplement la valeur de l'opérande ;
  \item \ggs!-!, négation arithmétique, une erreur d'exécution est déclenchée si l'opérande est égal à $-2^{31}$ ;
  \item \ggs!&-!, négation arithmétique, sans détection de débordement : la négation de $-2^{31}$ est $-2^{31}$.
\end{itemize}

La valeur renvoyée est du même type  \ggs+@sint+.


\subsectionLabel{Instructions}{instructionsSINT}

Le type \ggs+@sint+ accepte les instructions arithmétiques suivantes :
\begin{itemize}
  \item \ggs!+=!, addition, une erreur d'exécution est déclenchée en cas de débordement ;
  \item \ggs!-=!, soustraction, une erreur d'exécution est déclenchée en cas de débordement ;
  \item \ggs!*=!, multiplication, une erreur d'exécution est déclenchée en cas de débordement ;
  \item \ggs!/=!, division, une erreur d'exécution est déclenchée en cas division par zéro ;
  \item \ggs!++!, incrémentation, une erreur d'exécution est déclenchée en cas de débordement ;
  \item \ggs!--!, décrémentation, une erreur d'exécution est déclenchée en cas de débordement ;
  \item \ggs!&++!, incrémentation, le résultat étant silencieusement tronqué sur 32 bits ;
  \item \ggs!&--!, décrémentation, le résultat étant silencieusement tronqué sur 32 bits.
\end{itemize}

\ggs!x+=y! est équivalent à \ggs!x=x+y! ; \ggs!x-=y! est équivalent à \ggs!x=x-y!.
La variable cible \ggs!x!, comme l'expression source \ggs!y! doivent être du même type \ggs+@sint+. 

Incrémentation et décrémentation sont des instructions, et ne peuvent pas apparaître des expressions.
\begin{galgas}
@sint n = ... ; n ++ # Incrémentation
\end{galgas}

\begin{galgas}
@sint n = ... ; n -- # Décrémentation
\end{galgas}







\section{Shift Operators}


The \ggs+@sint+ type supports right and left shift operators:\newline

\begin{tabular}{|c|c|}
\hline
$<<$ & Left shift \\
\hline
$>>$ & Right shift \\
\hline
\end{tabular}

Theses operators require the right argument to be \ggs+@sint+ object, and the left argument to be \ggs+@uint+ object.\newline

Note the right shift inserts a zero bit in the most significant bit location if the receiver's value is negative, and a one bit otherwise (it is a arithmetic right shift).\newline

The actual amount of the shift is the value of the right-hand operand masked by 31, i.e. the shift distance is always between 0 and 31.




\section{Logical Operators}

The \ggs+@sint+ type supports the three bit-wise logical operators:\newline

\begin{tabular}{|c|c|}
\hline
$\&$ & Bit-wise and \\
\hline
\textbar & Bit-wise or \\
\hline
\^\  & Bit-wise exclusive or \\
\hline
\end{tabular}

Theses operators require both arguments to be \ggs+@sint+ objects.\newline


The \ggs+@sint+ type supports the bit-wise logical unary operator:\newline

\begin{tabular}{|c|c|}
\hline
$\sim$ & Bit-wise complementation \\
\hline
\end{tabular}

This operator returns an \ggs+@sint+ object.







\section{Comparison Operators}

The \ggs+@sint+ type supports the six comparison operators:\newline

\begin{tabular}{|c|c|}
\hline
$=$ & Equality \\
\hline
$!=$ & Non Equality \\
\hline
$<$  & Strict Lower Than \\
\hline
$<=$  & Lower or Equal \\
\hline
$>$  & Strict Greater Than \\
\hline
$>=$  & Greater or Equal \\
\hline
\end{tabular}

Theses operators require both arguments to be \ggs+@sint+ objects, and return a \ggs+@bool+ object.


