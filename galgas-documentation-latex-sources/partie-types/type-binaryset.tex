%!TEX encoding = UTF-8 Unicode
%!TEX root = ../galgas-book.tex

\chapitreTypePredefiniLabelIndex{binaryset}

Le type \ggs+@binaryset+ encode des ensembles, des relations binaires, des expressions booléennes. Il est implémenté par des BDD (Binary Decision Diagrams).


\section{Constructeurs}

\subsectionConstructor{binarySetWithBit}{binaryset}

\begin{galgas}
constructor binarySetWithBit ?@uint inBitIndex -> @binaryset
\end{galgas}


Retourne un \ggs+@binaryset+ dont le bit \ggs+inBitIndex+ est égal à 1.


\textbf{Exemple :}
\begin{galgas}
@binaryset s = .binarySetWithBit {!2}
log s # Affiche <@binaryset: 1XX>
\end{galgas}


\subsectionConstructor{binarySetWithEqualComparison}{binaryset}

\begin{galgas}
constructor binarySetWithEqualComparison
  ?@uint inLeftFirstIndex
  ?@uint inBitCount
  ?@uint inRightFirstIndex
  -> @binaryset
\end{galgas}




Retourne un \ggs+@binaryset+ qui encode la relation d'égalité entre deux variables.

Ce constructeur retourne un binary set qui encode la relation \emph{a~==~b}, où \emph{a} est encodé à partir du bit d'indice \emph{inLeftFirstIndex} jusqu'au bit d'indice \emph{inLeftFirstIndex  + inBitCount - 1}, et \emph{b} est encodé à partir du bit d'indice bit \emph{inRightFirstIndex} jusqu'au bit d'indice \emph{inRightFirstIndex + inBitCount - 1}.

\textbf{Exemple :}
\begin{galgas}
@binaryset s = .binarySetWithEqualComparison {!0 !2 !3}
log s # Affiche <@binaryset: 00x00, 01X01, 10X10, 11X11>
\end{galgas}




\subsectionConstructor{binarySetWithEqualToConstant}{binaryset}

\begin{galgas}
constructor binarySetWithEqualToConstant
  ?@uint inLeftFirstIndex
  ?@uint inBitCount
  ?@uint64 inConstant
  -> @binaryset
\end{galgas}


Retourne un \ggs+@binaryset+ object that encodes a equality relation between a variable and a constant.

Ce constructeur retourne un objet qui encode la relation the \emph{a~==~cst}, où \emph {a} est encodé à partir du bit d'indice \emph{inBitIndex} jusqu'au bit d'indice \emph{inBitIndex  + inBitCount - 1}, et \emph{cst} est défini par l'argument \emph{inConstant}.

\textbf{Exemple :}
\begin{galgas}
@binaryset s = .binarySetWithEqualToConstant {!0 !6 !23L}
log s # Affiche <@binaryset: 10111>
\end{galgas}




\subsectionConstructor{binarySetWithGreaterOrEqualComparison}{binaryset}

\begin{galgas}
constructor binarySetWithGreaterOrEqualComparison
  ?@uint inLeftFirstIndex
  ?@uint inBitCount
  ?@uint inRightFirstIndex
  -> @binaryset
\end{galgas}


Retourne un \ggs+@binaryset+ object qui encode la relation \emph{supérieur ou égal} entre deux variables.

Ce constructeur retourne un binary set qui encode la relation \emph{a~>=~b}, où \emph{a} est encodé à partir du bit d'indice \emph{inLeftFirstIndex} jusqu'au bit d'indice \emph{inLeftFirstIndex  + inBitCount - 1}, et \emph{b} est encodé à partir du bit d'indice bit \emph{inRightFirstIndex} jusqu'au bit d'indice \emph{inRightFirstIndex + inBitCount - 1}.

\textbf{Exemple :}
\begin{galgas}
@binaryset s = .binarySetWithGreaterOrEqualComparison {!0 !2 !3}
log s # Affiche <@binaryset: 00XXX, 01X01, 01X1X, 10X1X, 11X11>
\end{galgas}



\subsectionConstructor{binarySetWithGreaterOrEqualToConstant}{binaryset}

\begin{galgas}
constructor binarySetWithGreaterOrEqualToConstant
  ?@uint inLeftFirstIndex
  ?@uint inBitCount
  ?@uint64 inConstant
  -> @binaryset
\end{galgas}



Retourne un \ggs+@binaryset+ object that encodes a greater or equal relation between a variable and a constant.

The constructor returns a binary set that encodes the \emph{a~>=~cst} relation, where \emph {a} est encodé à partir du bit d'indice \emph{inBitIndex} jusqu'au bit d'indice \emph{inBitIndex  + inBitCount - 1}, and \emph{cst} is defined by the \emph{inConstant} argument.




\subsectionConstructor{binarySetWithLowerOrEqualComparison}{binaryset}

\begin{galgas}
constructor binarySetWithLowerOrEqualComparison
  ?@uint inLeftFirstIndex
  ?@uint inBitCount
  ?@uint inRightFirstIndex
  -> @binaryset
\end{galgas}


Retourne un \ggs+@binaryset+ object that encodes a lower or equal relation between two variables.

The constructor returns a binary set that encodes the \emph{a~<=~b} relation, where \emph{a} est encodé à partir du bit d'indice \emph{inLeftFirstIndex} jusqu'au bit d'indice \emph{inLeftFirstIndex  + inBitCount - 1}, and \emph{b} est encodé à partir du bit d'indice \emph{inRightFirstIndex} to \emph{inRightFirstIndex + inBitCount - 1}.

\textbf{Exemple :}
\begin{galgas}
@binaryset s = .binarySetWithLowerOrEqualComparison !0 !2 !3]
log s # Affiche <@binaryset: 00X00, 01X0X, 10X0X, 10X10, 11XXX>
\end{galgas}




\subsectionConstructor{binarySetWithLowerOrEqualToConstant}{binaryset}

\begin{galgas}
constructor binarySetWithLowerOrEqualToConstant
  ?@uint inLeftFirstIndex
  ?@uint inBitCount
  ?@uint64 inConstant
  -> @binaryset
\end{galgas}


Retourne un \ggs+@binaryset+ object that encodes a lower or equal relation between a variable and a constant.

The constructor returns a binary set that encodes the \emph{a~<=~cst} relation, where \emph {a} est encodé à partir du bit d'indice \emph{inBitIndex} jusqu'au bit d'indice \emph{inBitIndex  + inBitCount - 1}, and \emph{cst} is defined by the \emph{inConstant} argument.




\subsectionConstructor{binarySetWithNotEqualComparison}{binaryset}

\begin{galgas}
constructor binarySetWithNotEqualComparison
  ?@uint inLeftFirstIndex
  ?@uint inBitCount
  ?@uint inRightFirstIndex
  -> @binaryset
\end{galgas}



Retourne un \ggs+@binaryset+ object that encodes an inequality relation between two variables.

The constructor returns a binary set that encodes the \emph{a~!=~b} relation, where \emph{a} est encodé à partir du bit d'indice \emph{inLeftFirstIndex} jusqu'au bit d'indice \emph{inLeftFirstIndex  + inBitCount - 1}, and \emph{b} est encodé à partir du bit d'indice \emph{inRightFirstIndex} to \emph{inRightFirstIndex + inBitCount - 1}.

\textbf{Exemple :}
\begin{galgas}
@binaryset s = .binarySetWithNotEqualComparison !0 !2 !3]
log s # Affiche <@binaryset: 00X01, 00X1X, 01X00, 01X1X, 10X0X, 10X11, 11X0X, 11X10>
\end{galgas}




\subsectionConstructor{binarySetWithNotEqualToConstant}{binaryset}

\begin{galgas}
constructor binarySetWithNotEqualToConstant
  ?@uint inLeftFirstIndex
  ?@uint inBitCount
  ?@uint64 inConstant
  -> @binaryset
\end{galgas}


Retourne un \ggs+@binaryset+ object that encodes an inequality relation between a variable and a constant.

The constructor returns a binary set that encodes the \emph{a~!=~cst} relation, where \emph {a} est encodé à partir du bit d'indice \emph{inBitIndex} jusqu'au bit d'indice \emph{inBitIndex  + inBitCount - 1}, and \emph{cst} is defined by the \emph{inConstant} argument.







\subsectionConstructor{binarySetWithPredicateString}{binaryset}

\begin{galgas}
constructor binarySetWithPredicateString ?@string inPredicateString -> @binaryset
\end{galgas}

Returns the \ggs+@binaryset+ object described by the \emph{inPredicateString} argument.

The \emph{inBitString} argument string encodes a predicate string, such as those returned by \refGetterPage{binaryset}{predicateStringValue}.

\begin{description}
\item The \emph{inBitString} argument string characters should have one of the five following values:
\begin{itemize}
\item \texttt{\textquotesingle 0\textquotesingle}: a bit set to zero;
\item \texttt{\textquotesingle 1\textquotesingle}: a bit set to one;
\item \texttt{\textquotesingle X\textquotesingle}: a don't care bit;
\item \texttt{\textquotesingle~\textquotesingle}: a separator (non significant character);
\item \texttt{\textquotesingle\textbar\textquotesingle}: the boolean \emph{or} operation (in infix notation).
\end{itemize}
\end{description}


\textbf{Exemple :}
An empty predicate string (or a string containing only spaces) provides an empty binary set:
\begin{galgas}
@binaryset s = .binarySetWithPredicateString !" "]
@bool b = = .s isEmptySet]; # b is true
\end{galgas}


A predicate string containing only 'X' characters (at least one) provides an full binary set:
\begin{galgas}
@binaryset s = .binarySetWithPredicateString !" X X"] # Spaces are non significant
@bool b = [s isFullSet]; # b is true
\end{galgas}


A predicate string can encode a binary value (MSB first):
\begin{galgas}
@binaryset s [binarySetWithPredicateString !"1100"] # 12 in decimal
log s # Affiche <@binaryset: 1100>
\end{galgas}


You can use the boolean '|' operator for providing an or'ed values:
\begin{galgas}
@binaryset s [binarySetWithPredicateString !" 1100 | 1101"]
log s # Affiche <@binaryset: 110X>
\end{galgas}



You can use you can use don't care bits and '|' operator together:
\begin{galgas}
@binaryset s [binarySetWithPredicateString !"11X00X0 | 111XXX"]
log s # Affiche <@binaryset: 1100X0, 111XXX>
\end{galgas}




\subsectionConstructor{binarySetWithStrictGreaterComparison}{binaryset}

\begin{galgas}
constructor binarySetWithStrictGreaterComparison
  ?@uint inLeftFirstIndex
  ?@uint inBitCount
  ?@uint inRightFirstIndex
  -> @binaryset
\end{galgas}


Retourne un \ggs+@binaryset+ object that encodes a strict greater than relation between two variables.

The constructor returns a binary set that encodes the \emph{a~>~b} relation, where \emph{a} est encodé à partir du bit d'indice \emph{inLeftFirstIndex} jusqu'au bit d'indice \emph{inLeftFirstIndex  + inBitCount - 1}, and \emph{b} est encodé à partir du bit d'indice \emph{inRightFirstIndex} to \emph{inRightFirstIndex + inBitCount - 1}.

\textbf{Exemple :}
\begin{galgas}
@binaryset s [binarySetWithStrictGreaterComparison !0 !2 !3]
log s # Affiche <@binaryset: 00X01, 00X1X, 01X1X, 10X11>
\end{galgas}




\subsectionConstructor{binarySetWithStrictGreaterThanConstant}{binaryset}

\begin{galgas}
constructor binarySetWithStrictGreaterThanConstant
  ?@uint inLeftFirstIndex
  ?@uint inBitCount
  ?@uint64 inConstant
  -> @binaryset
\end{galgas}


Retourne un \ggs+@binaryset+ object that encodes a strict greater than relation between a variable and a constant.

The constructor returns a binary set that encodes the \emph{a~>~cst} relation, where \emph {a} est encodé à partir du bit d'indice \emph{inBitIndex} jusqu'au bit d'indice \emph{inBitIndex  + inBitCount - 1}, and \emph{cst} is defined by the \emph{inConstant} argument.




\subsectionConstructor{binarySetWithStrictLowerComparison}{binaryset}

\begin{galgas}
constructor binarySetWithStrictLowerComparison
  ?@uint inLeftFirstIndex
  ?@uint inBitCount
  ?@uint inRightFirstIndex
  -> @binaryset
\end{galgas}


Retourne un \ggs+@binaryset+ object that encodes a strict lower than relation between two variables.

The constructor returns a binary set that encodes the \emph{a~<~b} relation, where \emph{a} est encodé à partir du bit d'indice \emph{inLeftFirstIndex} jusqu'au bit d'indice \emph{inLeftFirstIndex  + inBitCount - 1}, and \emph{b} est encodé à partir du bit d'indice \emph{inRightFirstIndex} to \emph{inRightFirstIndex + inBitCount - 1}.

\textbf{Exemple :}
\begin{galgas}
@binaryset s [binarySetWithStrictLowerComparison !0 !2 !3]
log s # Affiche <@binaryset: 01X00, 10X0X, 11X0X, 11X10>
\end{galgas}




\subsectionConstructor{binarySetWithStrictLowerThanConstant}{binaryset}

\begin{galgas}
constructor binarySetWithStrictLowerThanConstant
  ?@uint inLeftFirstIndex
  ?@uint inBitCount
  ?@uint64 inConstant
  -> @binaryset
\end{galgas}


Retourne un \ggs+@binaryset+ object that encodes a strict lower than relation between a variable and a constant.

The constructor returns a binary set that encodes the \emph{a~<~cst} relation, where \emph {a} est encodé à partir du bit d'indice \emph{inBitIndex} jusqu'au bit d'indice \emph{inBitIndex  + inBitCount - 1}, and \emph{cst} is defined by the \emph{inConstant} argument.




\subsectionConstructor{emptyBinarySet}{binaryset}

\begin{galgas}
constructor emptyBinarySet -> @binaryset
\end{galgas}


Retourne un empty \ggs+@binaryset+ object.





\subsectionConstructor{fullBinarySet}{binaryset}

\begin{galgas}
constructor fullBinarySet -> @binaryset
\end{galgas}

Returns a full \ggs+@binaryset+ object.


\section{Getters}



\subsectionGetter{accessibleStates}{binaryset}

\begin{galgas}
getter accessibleStates -> @binaryset
\end{galgas}

Returns the set of accessible states from an initial state set. It computes the set of accessible states from the \emph{inInitialStateSet} state set using the accessibility relation encoded by the receiver.

\textbf{Exemple :}
\begin{galgas}
@binaryset gr [binarySetWithPredicateString !"0001 0000"] # Edge 0 -> 1
gr = gr | [@binaryset binarySetWithPredicateString !"0010 0001"] # Edge 1 -> 2
gr = gr | [@binaryset binarySetWithPredicateString !"0011 0010"] # Edge 2 -> 3
gr = gr | [@binaryset binarySetWithPredicateString !"0100 0011"] # Edge 3 -> 4
gr = gr | [@binaryset binarySetWithPredicateString !"0101 0100"] # Edge 4 -> 5
@binaryset initialState [binarySetWithPredicateString !"0000"] # 0 is the initial state
@binaryset accessibleStates = [gr accessibleStates !initialState !4]
message " Accessible:"
@uint64list valueList = [accessibleStates uint64ValueList !4]
foreach valueList do
  message " " . [mValue string]
end foreach
message "\n"
\end{galgas}


This program Affiche: \texttt{Accessible: 0 1 2 3 4 5}.



\subsectionGetter{binarySetByTranslatingFromIndex}{binaryset}

\begin{galgas}
getter binarySetByTranslatingFromIndex ?@uint inFirstIndex ?@uint inTranslation -> @string
\end{galgas}


Returns a \ggs+@binaryset+ value computed by translating the receiver's value by \emph{inTranslation} bits from index \emph{inFirstIndex}.



\subsectionGetter{compressedValueCount}{binaryset}

\begin{galgas}
getter compressedValueCount -> @uint64
\end{galgas}

Returns in an \ggs+@uint64+ value the number of different compressed string values encoded by receiver's value.



\subsectionGetter{compressedStringValueList}{binaryset}

\begin{galgas}
getter compressedStringValueList ?@uint inBitCount -> @stringlist
\end{galgas}

Returns the list of compressed string values corresponding to receiver's value, considering it uses \emph{inBitCount} bits.










\subsectionGetter{containsValue}{binaryset}

\begin{galgas}
getter containsValue ?@uint inFirstBit ?@uint inBitCount -> @bool
\end{galgas}


Retourne un \ggs+@bool+ value indicating whether the receiver'value contains a given value: \ggs+true+ if the receiver's contains a value, and \ggs+false+ otherwise; this value is computed from the \emph{inBitCount} first bits of \emph{inValue} value, shifted left by \emph{inFirstBit}.


\textbf{Exemple :}
\begin{galgas}
var s = @binaryset.binarySetWithPredicateString {!"0 00XX X111| 1 1111 1111"}
log s # Affiche <@binaryset: 000XXX111, 111111111>
@bool b = [s containsValue !127L !0 !7]
log b # Affiche <@bool:true>
b = [s containsValue !31L !1 !7]
log b # Affiche <@bool:true>
b = [s containsValue !63L !1 !8]
log b # Affiche <@bool:false>
b = [s containsValue !7L !0 !9]
log b # Affiche <@bool:true>
b = [s containsValue !7L !0 !10]
log b # Affiche <@bool:true>
b = [s containsValue !32767L !1 !12]
log b # Affiche <@bool:true>
\end{galgas}








\subsectionGetter{equalTo}{binaryset}

\begin{galgas}
getter equalTo ?@binaryset inOperand -> @binaryset
\end{galgas}

Returns the complement of the exclusive or between the receiver's value and the operand's value.

Note that \ggs+[a equalTo !b]+ is equivalent to \texttt{$\sim$ (a $\wedge$ b)}.

This operation returns un \ggs+@binaryset+ value; do not confuse with \ggs+==+ operator that Retourne un \ggs+@bool+ value.







\subsectionGetter{existOnBitIndex}{binaryset}

\begin{galgas}
getter existOnBitIndex ?@uint inBitIndex -> @binaryset
\end{galgas}

Returns the binary computed by applying the \emph{exist} operator on the \emph{inBitIndex} bit of the receiver's value.






\subsectionGetter{existsOnBitRange}{binaryset}

\begin{galgas}
getter existsOnBitRange ?@uint inFirstBitIndex ?@uint inBitCount -> @bool
\end{galgas}


Returns the binary computed by applying the \emph{exist} operator on the receiver's value, from \emph{inFirstBitIndex} bit index until the \emph{inFirstBitIndex + inBitCount - 1} bit index.


\textbf{Exemple :}
\begin{galgas}
@binaryset s [binarySetWithPredicateString !"01110010"]
log s # Affiche <@binaryset: 01110010>
@binaryset ss = [s existsOnBitRange !2 !3]
log s # Affiche <@binaryset: 011XXX10>
\end{galgas}







\subsectionGetter{existOnBitIndexAndBeyond}{binaryset}

\begin{galgas}
getter existOnBitIndexAndBeyond ?@uint inBitIndex -> @binaryset
\end{galgas}

Returns the binary set computed by applying the \emph{exist} operator on all bits from \emph{inFirstBitIndex} bit index of the receiver's value.







\subsectionGetter{forAllOnBitIndex}{binaryset}

\begin{galgas}
getter forAllOnBitIndex ?@uint inBitIndex -> @binaryset
\end{galgas}

Returns the binary set computed by applying the \emph{for all} operator on the \emph{inFirstBitIndex} bit index of the receiver's value.







\subsectionGetter{forAllOnBitIndexAndBeyond}{binaryset}

\begin{galgas}
getter forAllOnBitIndexAndBeyond ?@uint inBitIndex -> @binaryset
\end{galgas}


Returns the binary computed by applying the \emph{for all} operator on all bits from \emph{inFirstBitIndex} bit index of the receiver's value.








\subsectionGetter{greaterOrEqualTo}{binaryset}

\begin{galgas}
getter greaterOrEqualTo ?@binaryset inOperand -> @binaryset
\end{galgas}


Returns the complement of the exclusive or between the receiver's value and the operand's value.

Note that \ggs+[a greaterOrEqualTo !b]+ is equivalent to \texttt{(a \textbar ~$\sim$b)}.








\subsectionGetter{isEmpty}{binaryset}

\begin{galgas}
getter isEmpty -> @bool
\end{galgas}

Returns a \ggs+@bool+ value that indicates whether the receiver's value is the empty set :  \ggs+true+ if receiver's value is the empty set, and \ggs+false+ otherwise.







\subsectionGetter{isFull}{binaryset}

\begin{galgas}
getter isFull -> @bool
\end{galgas}

Returns a \ggs+@bool+ value that indicates whether the receiver's value is the full set : \ggs+true+ if receiver's value is the full set, and \ggs+false+ otherwise.







\subsectionGetter{ITE}{binaryset}

\begin{galgas}
getter ITE ?@binaryset inThenOperand ?@binaryset inElseOperand -> @binaryset
\end{galgas}


Returns the binary set computed by applying the \emph{ite} operator on the receiver's value, the \emph{inThenOperand} argument, and the  \emph{inElseOperand} argument.

{\texttt{ite (x, y, z)} is \texttt{(x \& y) \textbar ($\sim$x \& z)}.}







\subsectionGetter{lowerOrEqualTo}{binaryset}

\begin{galgas}
getter lowerOrEqualTo ?@binaryset inOperand -> @binaryset
\end{galgas}


Returns the binary set computed by applying the \emph{lower or equal} operator on the receiver's value and the \emph{inOperand} argument.
{\texttt{[a lowerOrEqualTo !b]} is \texttt{(($\sim$x) \textbar y)}.}







\subsectionGetter{notEqualTo}{binaryset}

\begin{galgas}
getter notEqualTo ?@binaryset inOperand -> @binaryset
\end{galgas}


Returns the binary set computed by applying the \emph{not equal} operator on the receiver's value and the \emph{inOperand} argument.
{\texttt{[a notEqualTo !b]} is \texttt{(x $\wedge$ y)}.}







\subsectionGetter{predicateStringValue}{binaryset}

\begin{galgas}
getter predicateStringValue -> @string
\end{galgas}

Returns a string representation of the receiver's value. The returned string is compatible with the \refConstructorPage{binaryset}{binarySetWithPredicateString}.







\subsectionGetter{strictGreaterThan}{binaryset}

\begin{galgas}
getter strictGreaterThan ?@binaryset inOperand -> @binaryset
\end{galgas}

Returns the binary set computed by applying the \emph{strict greater} operator on the receiver's value and the \emph{inOperand} argument.
{\texttt{[a strictGreaterThan !b]} is \texttt{(x \& $\sim$y)}.}







\subsectionGetter{strictLowerThan}{binaryset}

\begin{galgas}
getter strictLowerThan ?@binaryset inOperand -> @binaryset
\end{galgas}

Returns the binary set computed by applying the \emph{strict lower} operator on the receiver's value and the \emph{inOperand} argument.
{\texttt{[a strictLowerThan !b]} is \texttt{($\sim$x \& y)}.}







\subsectionGetter{stringValueList}{binaryset}

\begin{galgas}
getter stringValueList ?@uint inBitCount -> @stringlist
\end{galgas}

Returns the list of string values corresponding to receiver's value, considering it uses \emph{inBitCount} bits.







\subsectionGetter{stringValueListWithNameList}{binaryset}

\begin{galgas}
getter stringValueListWithNameList
  ?@uint inBitCount
  ?@stringlist inNameList
  -> @stringlist
\end{galgas}


Returns the list of named values corresponding to receiver's value, considering it uses \emph{inBitCount} bits. First, the receiver is enumerated, considering it uses \emph{inBitCount} bits. Each enumerated value is used as an index of \emph{inNameList}, and the string value at this index is appended at the end of the returned value.







\subsectionGetter{swap021}{binaryset}

\begin{galgas}
getter swap021
  ?@uint inBitCount1
  ?@uint inBitCount2
  ?@uint inBitCount3
  -> @binaryset
\end{galgas}



Returns the transposed \emph{(x, z, y)} relation.

This getter considers that the receiver encodes an \emph{(x, y, z)} relation, where \emph{x} is defined by bits index \emph{0} to \emph{inBitCount1  - 1}, \emph{y} is defined by bits index \emph{inBitCount1} to \emph{inBitCount1 + inBitCount2 - 1} and  \emph{z} is defined by bits index \emph{inBitCount1 + inBitCount2} to \emph{inBitCount1 + inBitCount2 + inBitCount3 - 1}.







\subsectionGetter{swap01}{binaryset}

\begin{galgas}
getter swap01 ?@uint inBitCount1 ?@uint inBitCount2 -> @binaryset
\end{galgas}


Returns the transposed \emph{(y, x)} relation.

This getter considers that the receiver encodes an \emph{(x, y)} relation, where \emph{x} is defined by bits index \emph{0} to \emph{inBitCount1  - 1}, \emph{y} is defined by bits index \emph{inBitCount1} to \emph{inBitCount1 + inBitCount2 - 1}.





\subsectionGetter{swap102}{binaryset}

\begin{galgas}
getter swap102
  ?@uint inBitCount1
  ?@uint inBitCount2
  ?@uint inBitCount3
  -> @binaryset
\end{galgas}

Returns the transposed \emph{(y, x, z)} relation.

This getter considers that the receiver encodes an \emph{(x, y, z)} relation, where \emph{x} is defined by bits index \emph{0} to \emph{inBitCount1  - 1}, \emph{y} is defined by bits index \emph{inBitCount1} to \emph{inBitCount1 + inBitCount2 - 1} and  \emph{z} is defined by bits index \emph{inBitCount1 + inBitCount2} to \emph{inBitCount1 + inBitCount2 + inBitCount3 - 1}.






\subsectionGetter{swap120}{binaryset}

\begin{galgas}
getter swap120
  ?@uint inBitCount1
  ?@uint inBitCount2
  ?@uint inBitCount3
  -> @binaryset
\end{galgas}

Returns the transposed \emph{(y, z, x)} relation.

This getter considers that the receiver encodes an \emph{(x, y, z)} relation, where \emph{x} is defined by bits index \emph{0} to \emph{inBitCount1  - 1}, \emph{y} is defined by bits index \emph{inBitCount1} to \emph{inBitCount1 + inBitCount2 - 1} and  \emph{z} is defined by bits index \emph{inBitCount1 + inBitCount2} to \emph{inBitCount1 + inBitCount2 + inBitCount3 - 1}.






\subsectionGetter{swap201}{binaryset}

\begin{galgas}
getter swap201
  ?@uint inBitCount1
  ?@uint inBitCount2
  ?@uint inBitCount3
  -> @binaryset
\end{galgas}

Returns the transposed \emph{(z, x, y)} relation.

This getter considers that the receiver encodes an \emph{(x, y, z)} relation, where \emph{x} is defined by bits index \emph{0} to \emph{inBitCount1  - 1}, \emph{y} is defined by bits index \emph{inBitCount1} to \emph{inBitCount1 + inBitCount2 - 1} and  \emph{z} is defined by bits index \emph{inBitCount1 + inBitCount2} to \emph{inBitCount1 + inBitCount2 + inBitCount3 - 1}.






\subsectionGetter{swap210}{binaryset}

\begin{galgas}
getter swap210
  ?@uint inBitCount1
  ?@uint inBitCount2
  ?@uint inBitCount3
  -> @binaryset
\end{galgas}

Returns the transposed \emph{(z, y, x)} relation.

This getter considers that the receiver encodes an \emph{(x, y, z)} relation, where \emph{x} is defined by bits index \emph{0} to \emph{inBitCount1  - 1}, \emph{y} is defined by bits index \emph{inBitCount1} to \emph{inBitCount1 + inBitCount2 - 1} and  \emph{z} is defined by bits index \emph{inBitCount1 + inBitCount2} to \emph{inBitCount1 + inBitCount2 + inBitCount3 - 1}.








\subsectionGetter{transitiveClosure}{binaryset}

\begin{galgas}
getter transitiveClosure ?@uint inBitCount -> @binaryset
\end{galgas}


Returns the transitive closure of the relation encoded by the receiver.

This getter considers that the receiver encodes an \emph{(x, y)} relation, where \emph{x} is defined by bits index \emph{0} to \emph{inBitCount  - 1}, \emph{y} is defined by bits index \emph{inBitCount} to \emph{2 * inBitCount - 1}.








\subsectionGetter{uint64ValueList}{binaryset}

\begin{galgas}
getter uint64ValueList ?@uint inBitCount -> @uint64list
\end{galgas}


Returns the list of \ggs+@uint64+ values corresponding to receiver's value, considering it uses \emph{inBitCount} bits.








\subsectionGetter{valueCount}{binaryset}

\begin{galgas}
getter valueCount ?@uint inBitCount -> @uint64
\end{galgas}


Returns in an \ggs+@uint64+ object the number of different values encoded by receiver, considering it uses \emph{inBitCount} bits. No overflow test in performed.







%-------------------------------

\section{Logical Operators}

The \ggs+@binaryset+ type supports the three logical operators:\newline

\begin{tabular}{|c|c|}
\hline
\texttt{$\&$} & Logical And, intersection \\
\hline
\texttt{\textbar} & Logical Or, union \\
\hline
\texttt{$\wedge$}  & Exclusive or \\
\hline
\end{tabular}

Theses operators require both arguments to be \ggs+@binaryset+ objects and return an \ggs+@binaryset+ object.\newline


The \ggs+@binaryset+ type supports the logical unary operator:\newline

\begin{tabular}{|c|c|}
\hline
$\sim$ & Negation, Complementation \\
\hline
\end{tabular}

This operator Retourne un \ggs+@binaryset+ object.







\section{Comparison Operators}

The \ggs+@binaryset+ type supports the two comparison operators:\newline

\begin{tabular}{|c|c|}
\hline
$=$ & Equality \\
\hline
$!=$ & Non Equality \\
\hline
\end{tabular}

Theses operators require both arguments to be \ggs+@binaryset+ objects, and return a \ggs+@bool+ object. These operations are very fast and are performed in a constant time (integer equality comparison).

Do not confuse with \refGetterPage{binaryset}{equalTo} and \refGetterPage{binaryset}{notEqualTo} that return a \ggs+@binaryset+ object.







\section{Shift Operators}

The \ggs+@binaryset+ type supports the two shift operators:\newline

\begin{tabular}{|c|c|}
\hline
$<<$ & Left Shift \\
\hline
$>>$ & Right Shift \\
\hline
\end{tabular}

\textbf{Exemple :}
\begin{galgas}
@binaryset b [binarySetWithPredicateString !"1010"]
log b # Affiche: <@binaryset: 1010>
@binaryset bb = b << 3
log bb # Affiche: <@binaryset: 1010XXX>
\end{galgas}

