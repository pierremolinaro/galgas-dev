%!TEX encoding = UTF-8 Unicode
%!TEX root = ../galgas-book.tex

\chapitreTypePredefiniLabelIndex{stringset}

An \galgas{@stringset} object value is a set of \galgas{@string} values.\\

\section{Constructors}

\constructeurSansArgument{emptySet}
{stringset}
{1.3.0}
{stringset}
{Creates and returns an empty \galgas{@stringset} object.}
{}

\constructeurUnArgument{setWithString}
{stringset}
{1.3.0}
{stringset}
{@string inString}
{Creates and returns an \galgas{@stringset} object that contains the value of the \emph{inString} argument object.}
{}

\section{Readers}

\readerSansArgument{count}
{stringset}
{1.3.0}
{uint}
{Returns the number of strings in the set.}
{}



\readerUnArgument{hasKey}
{stringset}
{1.3.0}
{bool}
{@string inString}
{Returns a boolean value that indicates whether the value of \emph{inString} argument is present in the set.}
{returns \galgas{true} if the value of \emph{inString} argument is present in the set, \galgas{false} otherwise.}


\readerSansArgument{anyString}
{stringset}
{1.3.0}
{string}
{Retourne une des chaînes de caractères contenue dans le receveur.}
{si le receiveur est vide, une erreur d'exécution est déclenchée.}




\section{Modifier}

\modifierUnArgument{removeKey}
{stringset}
{1.3.0}
{@string inString}
{Removes the value of \emph{inString} argument from the receiver's value.}
{if the receiver's value does not contain the value of \emph{inString} argument, this modifier leaves the receiver's value unchanged.}






\section{the \emph{+=} Operator}

The \emph{+=} operator adds a string value to the receiver. If the receiver's value already contains the added value, this operator has no effect.

\textbf{exemple :}
\begin{galgascode}
@string aString := ... ;
@stringset aStringSet := ... ;
aStringSet += !aString ;
\end{galgascode}




\section{the \emph{$\&$} Operator}

The \emph{$\&$} operator returns the intersection of its operand values.

\textbf{exemple :}
\begin{galgascode}
@stringset s1 := ... ;
@stringset s2 := ... ;
@stringset s := s1 & s2 ; # s is the intersection of s1 and s2
\end{galgascode}






\section{the \emph{$\textbar$} Operator}

The \emph{$\textbar$} operator returns the union of its operand values.

\textbf{exemple :}
\begin{galgascode}
@stringset s1 := ... ;
@stringset s2 := ... ;
@stringset s := s1 | s2 ; # s is the union of s1 and s2
\end{galgascode}






\section{the \emph{$-$} Operator}

The \emph{$-$} operator returns the difference of its operand values.

\textbf{exemple :}
\begin{galgascode}
@stringset s1 := ... ;
@stringset s2 := ... ;
@stringset s := s1 - s2 ; \# s is the difference of s1 and s2
\end{galgascode}








\section{Enumerating \texttt{@stringset} objects}


The \galgas{foreach} instruction can be used for enumerating \galgas{@stringset} values; enumeration is performed in the ascending order, or in the reverse alphabetical order using the '>' qualifier.

\texttt{@stringset s := ... ;}\newline
\textbf{foreach} \texttt {s} \textbf {do}\newline
\texttt{\# the \emph{key} constant has the value of current entry of \emph{s} stringset}\newline
\textbf{end foreach} \texttt{;}







\section{Comparison Operators}

The \galgas{@stringset} type supports the six comparison operators:\newline

\begin{tabular}{|c|c|}
\hline
$=$ & Equality \\
\hline
$!=$ & Non Equality \\
\hline
$<$  & Strict Inclusion \\
\hline
$<=$  & Inclusion or Equality \\
\hline
$>$  & Strict Greater \\
\hline
$>=$  & Greater or Equality \\
\hline
\end{tabular}

Theses operators require both arguments to be \galgas{@stringset} objects, and return a \galgas{@stringset} object.


