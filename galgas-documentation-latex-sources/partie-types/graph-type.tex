%!TEX encoding = UTF-8 Unicode
%!TEX root = ../galgas-book.tex

%--------------------------------------------------------------
\chapter{Le type \texttt{graph}}
%-------------------------------------------------------------

Le type \galgas{graph} permet de faire des opérations sur les graphes orientés.

Chaque nœud est identifié par un nom qui est une chaîne de caractères (de type \galgas{@string}), et est associé à une information utilisateur de type quelconque.

Un arc est identifié par un couple de nœuds.


Un type \galgas{graph} se déclare comme suit :
\lstset{emph={@nom_du_type_graph, @nom_liste_information}, emphstyle=\emph}
\begin{galgascode}
graph @nom_du_type_graph (@nom_liste_information) {
}
\end{galgascode}

Le nom \galgas{@nom_du_type_graph} est le nom donné au type. Le nom \galgas{@nom_liste_information} nomme un type qui spécifie l'information utilisateur associée à chaque nœud.

Attention, le type \galgas{@nom_liste_information} est un type \emph{liste}, et l'information utilisateur a pour type l'élement associé, c'est à dire \galgas{@nom_liste_information.element}. 

Par exemple, si l'on veut manipuler des graphes dont l'information associée est un entier \galgas{@uint}, on déclarera :
\begin{galgascode}
graph @monGraphe (@uintlist) {
}
\end{galgascode}

Si l'information associée est par exemple composée d'un entier et d'une chaîne de caractères, il faut déclarer un type liste :
\begin{galgascode}
list @maListe {
  @uint monInfo1 ;
  @string monInfo2 ;
}
graph @monGraphe (@maListe) {
}
\end{galgascode}






\section{Entrer les nœuds}



\section{Entrer les arcs}




